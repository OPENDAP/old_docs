%
% This file contains information on URLs. It describes how the different
% parts of DODS use URLs as well as the structure DODS assumes those URLs to
% have. 
%
% $Id$
%

\documentstyle[code,12pt,html,margins]{article}

%
% These are html links which are used often enough in writing about DODS to
% merit an input file.
% jhrg. 4/17/94
%
% File rationalized and updated while writing the DODS User
% Guide. Also includes other useful abbreviations.
% tomfool 3/15/96
%
% $Id$
%
% HTML references to DODS documents
%
% For most of the DODS documentation there are three html references defined:
% 1) The upper case \newcommand produces the title of the paper, an
%    html link in the online documentation and a footnote in the paper
%    version.  
% 2) A capitalized version produces a capitalized description of the
%    paper and a link in the online version (but no footnote in the
%    paper version). 
% 3) A lower case version which produces a lower case description of
%    the paper and a html link, but no footnote.

% External references to documents:
%
% In order for external references to work (via latex2html) the labels used
% throughout the documents must be unique. The text labels are all prefixed
% with a document identifier (e.g., ddd) with a colon separater. The
% \externalrefs below (way below...) includes the perl files html.sty uses to
% make the cross refs.

\newcommand{\DODS}{\htmladdnormallink{Distributed Oceanographic Data System}
  {http://www.unidata.ucar.edu/packages/dods/}}

\newcommand{\Dods}{\htmladdnormallink{DODS}{http://www.unidata.ucar.edu/packages/dods/}}

\newcommand{\wrkshp}{\htmladdnormallink{Report on the First Workshop for the
    Distributed Oceanographic Data System, Proposed System Architectures}
  {http://www.unidata.ucar.edu/packages/dods/archive/reports/workshop1/section3.13.html}}

% not DODS URLs, but useful all the same...

\newcommand{\www}{\htmladdnormallink{World Wide Web} 
  {http://www.w3.org/hypertext/WWW/TheProject.html}}
\newcommand{\url}{\htmladdnormallink{Uniform Resource Locator}
  {http://www.w3.org/hypertext/WWW/Addressing/URL/url-spec.html}}

\newcommand{\uri}{\htmladdnormallinkfoot{Uniform Resource Identifiers}
  {http://www.w3.org/hypertext/WWW/Addressing/URL/uri-spec.html}}

\newcommand{\urn}{\htmladdnormallinkfoot{Uniform Resource Name}
  {http://www.acl.lanl.gov/URI/archive/uri-archive.messages/1143.html}}

\newcommand{\HTTPD}{\htmladdnormallinkfoot{HTTPD}
  {http://hoohoo.ncsa.uiuc.edu/docs/Overview.html}}

\newcommand{\HTTP}{\htmladdnormallinkfoot{HyperText Transfer Protocol}
  {http://www.w3.org/hypertext/WWW/Protocols/HTTP/HTTP2.html}}

%\newcommand{\HTML}{\htmladdnormallinkfoot{HyperText Markup Language}
%  {http://www.w3.org/hypertext/WWW/MarkUp/MarkUp.html}}

% CERN used to be `Conseil Europeen pour la Recherche Nucleaire'
\newcommand{\CERN}
  {\htmladdnormallink{European Laboratory for Particle Physics}
  {http://www.cern.ch/}}

\newcommand{\NCSA}
  {\htmladdnormallink{National Center for Supercomputing Applications}
  {http://www.ncsa.uiuc.edu/}}

\newcommand{\WWWC}{\htmladdnormallink{World Widw Web Consortium}
  {http://www.w3.org/}}

\newcommand{\CGI}{\htmladdnormallinkfoot{Common Gateway Interface}
  {http://hoohoo.ncsa.uiuc.edu/cgi/overview.html}}

\newcommand{\MIME}{\htmladdnormallink{Multipurpose Internet Mail Extensions}
  {http://www.cis.ohio-state.edu/htbin/rfc/rfc1590.html}}

\newcommand{\netcdf}{\htmladdnormallink{NetCDF}
  {http://www.unidata.ucar.edu/packages/netcdf/guide.txn_toc.html}}

\newcommand{\JGOFS}{\htmladdnormallink{Joint Geophysical Ocean Flux Study}
  {http://www1.whoi.edu/jgofs.html}}

\newcommand{\jgofs}{\htmladdnormallink{JGOFS}
  {http://www1.whoi.edu/jgofs.html}}

\newcommand{\hdf}{\htmladdnormallink{HDF}
  {http://www.ncsa.uiuc.edu/SDG/Software/HDF/HDFIntro.html}}

\newcommand{\Cpp}{{\rm {\small C}\raise.5ex\hbox{\footnotesize ++}}}

% Commands

\newcommand{\pslink}[1]{\small
\begin{quote}
  A \htmladdnormallink{postscript}{#1} version of this document is
  available.  You may also use \htmladdnormallink{anonymous
  ftp}{ftp://ftp.unidata.ucar.edu/pub/dods/ps-docs/} to access postscript files
  of all of the DODS documentation.
\end{quote}
\normalsize
}

\newcommand{\declaration}[1]{\small
{\tt {#1}}
\normalsize
}

% All the links from here down are for the white papers. I'm not sure that any
% of these are still valid given the reorganization of the web site. 5/12/98
% jhrg.

\newcommand{\DDA}{\htmladdnormallinkfoot{DODS---Data Delivery Architecture}
  {http://www.unidata.ucar.edu/packages/dods/archive/design/data-delivery-arch/data-delivery-arch.html}}
\newcommand{\Dda}{\htmladdnormallink{Data Delivery Architecture}
  {http://www.unidata.ucar.edu/packages/dods/archive/design/data-delivery-arch/data-delivery-arch.html}}
\newcommand{\dda}{\htmladdnormallink{data delivery architecture}
  {http://www.unidata.ucar.edu/packages/dods/archive/design/data-delivery-arch/data-delivery-arch.html}}

\newcommand{\DDD}{\htmladdnormallinkfoot{DODS---Data Delivery Design}
  {http://www.unidata.ucar.edu/packages/dods/archive/design/data-delivery-design/data-delivery-design.html}}
\newcommand{\Ddd}{\htmladdnormallink{Data Delivery Design}
  {http://www.unidata.ucar.edu/packages/dods/archive/design/data-delivery-design/data-delivery-design.html}}
\newcommand{\ddd}{\htmladdnormallink{data delivery design}
  {http://www.unidata.ucar.edu/packages/dods/archive/design/data-delivery-design/data-delivery-design.html}}

\newcommand{\URL}{\htmladdnormallinkfoot{DODS---Uniform Resource Locator}
  {http://www.unidata.ucar.edu/packages/dods/archive/design/urls/urls.html}}
\newcommand{\Url}{\htmladdnormallink{Uniform Resource Locator}
  {http://www.unidata.ucar.edu/packages/dods/archive/design/urls/urls.html}}
\newcommand{\dodsurl}{\htmladdnormallink{uniform resource locator}
  {http://www.unidata.ucar.edu/packages/dods/archive/design/urls/urls.html}}

\newcommand{\DAP}{\htmladdnormallinkfoot{DODS---Data Access Protocol}
  {http://www.unidata.ucar.edu/packages/dods/archive/design/api/api.html}}
\newcommand{\Dap}{\htmladdnormallink{Data Access Protocol}
  {http://www.unidata.ucar.edu/packages/dods/archive/design/api/api.html}}
\newcommand{\dap}{\htmladdnormallink{data access protocol}
  {http://www.unidata.ucar.edu/packages/dods/archive/design/api/api.html}}

\newcommand{\SOFT}
        {\htmladdnormallinkfoot{DODS---Software Development Environment}
  {http://www.unidata.ucar.edu/packages/dods/archive/managment/software/software.html}}
\newcommand{\Soft}{\htmladdnormallink{Software Development Environment}
  {http://www.unidata.ucar.edu/packages/dods/archive/managment/software/software.html}}
\newcommand{\soft}{\htmladdnormallinkfoot{software development environment}
  {http://www.unidata.ucar.edu/packages/dods/archive/managment/software/software.html}}

% I used to call the DAP the API, then for a few days I called it the
% DTP. These def's were used then. Rather than find everywhere they are used
% (not that hard, but...) I just hacked the macros. NB: the source file for
% the DAP document is still called `api.tex'

\newcommand{\API}{\htmladdnormallinkfoot{DODS---Data Access Protocol}
  {http://www.unidata.ucar.edu/packages/dods/archive/design/api/api.html}}
\newcommand{\api}{\htmladdnormallink{data access protocol}
  {http://www.unidata.ucar.edu/packages/dods/archive/design/api/api.html}}

\newcommand{\DTP}{\htmladdnormallinkfoot{DODS---Data Access Protocol}
  {http://www.unidata.ucar.edu/packages/dods/archive/design/api/api.html}}
\newcommand{\dtp}{\htmladdnormallink{data access protocol}
  {http://www.unidata.ucar.edu/packages/dods/archive/design/api/api.html}}

\newcommand{\TOOLKIT}{\htmladdnormallinkfoot{DODS---Client and Server Toolkit}
  {http://www.unidata.ucar.edu/packages/dods/archive/implementation/toolkits/toolkits.html}}
\newcommand{\Toolkit}{\htmladdnormallink{Client and Server Toolkit}
  {http://www.unidata.ucar.edu/packages/dods/archive/implementation/toolkits/toolkits.html}}
\newcommand{\toolkit}{\htmladdnormallink{client and server toolkit}
  {http://www.unidata.ucar.edu/packages/dods/archive/implementation/toolkits/toolkits.html}}

\newcommand{\TKR}{\htmladdnormallinkfoot{DODS---DAP Toolkit Reference}
  {http://www.unidata.ucar.edu/packages/dods/archive/implementation/toolkits/toolkits.html}}
\newcommand{\Tkr}{\htmladdnormallink{DAP Toolkit Reference}
  {http://www.unidata.ucar.edu/packages/dods/archive/implementation/toolkits/toolkits.html}}
\newcommand{\tkr}{\htmladdnormallink{DAP toolkit reference}
  {http://www.unidata.ucar.edu/packages/dods/archive/implementation/toolkits/toolkits.html}}

% I know that these are wrong. The papers have been moved to the gso web
% site, but I'm not sure exactly where. 5/12/98 jhrg

\newcommand{\DM}{\htmladdnormallinkfoot{DODS---Data Model}
  {http://lake.mit.edu/dods-dir/dodsdm2.html}}
\newcommand{\Dm}{\htmladdnormallink{Data Model}
  {http://lake.mit.edu/dods-dir/dodsdm2.html}}
\newcommand{\dm}{\htmladdnormallink{data model}
  {http://lake.mit.edu/dods-dir/dodsdm2.html}}

\newcommand{\DD}{\htmladdnormallinkfoot{DODS---Data Delivery}
  {http://lake.mit.edu/dods-dir/dods-dd.html}}

% external refs for DODS documents

\externallabels{http://www.unidata.ucar.edu/packages/dods/design/api}
  {/home/www/packages/dods/archive/design/api/labels.pl}
\externallabels{http://www.unidata.ucar.edu/packages/dods/design/data-delivery-arch}
  {/home/www/packages/dods/archive/design/data-delivery-arch/labels.pl}
\externallabels{http://www.unidata.ucar.edu/packages/dods/design/data-delivery-design}
  {/home/www/packages/dods/archive/design/data-delivery-design/labels.pl}
\externallabels{http://www.unidata.ucar.edu/packages/dods/implementation/toolkits}
  {/home/www/packages/dods/archive/implementation/toolkits/labels.pl}

%\renewcommand{\externalref}[2]{#2}

%%% Local Variables: 
%%% mode: latex
%%% TeX-master: t
%%% End: 


\begin{document}

\title{DODS---Uniform Resource Locators}
\author{}
\date{23 August 1996} % was {\today}

\maketitle

\begin{abstract}

  This document describes the syntax used by the Distributed Oceanographic
  Data System (DODS) to reference objects on the Internet. The syntax is
  based on the Uniform Resource Locator syntax used by the World Wide Web
  system designed at \CERN\@. In addition to the syntax description, an
  overview of the use of resource locators by the main components of the
  system is given.

\end{abstract}


% Biolerplate text warning that anything can (and almost certainly will)
% change. 
%
% jhrg 3/25/94
%
% $Id$

\small

\centerline{{\bf Note}} 

\begin{quotation}

\begin{htmlonly}
  This is a working document made available to solicit comments. To receive
  e-mail notice of new or significantly changed documents, send e-mail
  to \htmladdnormallink{majordomo@unidata.ucar.edu}
  {mailto:majordomo@unidata.ucar.edu} with 'subscribe dods' in the
  body of the message.
\end{htmlonly}

\begin{latexonly}
  This is a working document made available to solicit comments. To receive
  e-mail notice of new or significantly changed documents, send e-mail
  to majordomo@unidata.ucar.edu with 'subscribe dods' in the body of
  the message. 

  Hypertext references in the hardcopy version of this document are
  represented using footnotes that contain a Universal Resource Locator (URL)
  to the referenced document.  To read the online documentation, open the URL
  `http://www.unidata.ucar.edu/' with a World Wide Web browser.
\end{latexonly}

\end{quotation}

\normalsize

%
% Biolerplate text expaining that this document is intended for programmers.
%
% jhrg 4/17/94
%
% $Id$

\small

\begin{quotation}
  This document is written for people interested in implementing software
  that will be part of, or work in parallel with, the Distributed
  Oceanographic Data System. 
\end{quotation}

\normalsize

\begin{htmlonly}
\pslink{ftp://ftp.unidata.ucar.edu/pub/dods/ps-docs/urls.ps}
\end{htmlonly}

\newpage

\tableofcontents

\newpage

\section{Introduction}
\label{url:introduction}

The \DODS\ (DODS) needs a syntax for names which describe the network
location for different resources. There are many other systems which must
solve a similar problem. Three different classes of solution to this problem
have been implemented in existing systems: (1) The Network File System
incorporates remote resources directly into the client computer's file system
where they can be referenced as if they were local files, (2) {\tt ftp}
prompts the user for an Internet address and then directs all user actions to
that address until it is told to disconnect and (3) \www\ (WWW) clients use a
string, called a \url\ (URL), which combines the protocol, host and pathname.
We choose the third approach after dismissing the first two as, respectively,
too complex and too rudimentary.

A URL is a generalization of file naming concepts which is already in common
use on the Internet. A URL extends the filename/pathname concept used to
refer to files on a computer by adding an Internet address and protocol
component. A URL specifies the computer and access method (protocol) in
addition to the filename and thus can be used to refer to files anywhere on a
network. While they are used in various forms by many different systems,
URLs, as such, are most closely associated with the WWW project.

DODS chose the URL notation in part because it represents an evolving
standard for resource specification on the Internet. Conforming, at least in
spirit with this standard, puts DODS in a place to benefit from ongoing work
to develop more general network resource names with minimal additional work.
Because the existing definitions of resource locators (by both DODS and the
WWW) do not adequately address some important issues, compliance with an
existing {\em de facto\/} standard increases the chance that DODS will be
able to take advantage of emerging solutions to those problems.

In addition, the syntax of DODS URLs must match that used by the WWW because
the \ddd\ uses the WWW \HTTPD\ server as its base-communication module. While
previous designs for the data delivery system outlined the construction of
special DODS-only data servers using RPC technology, the most recent design
uses NCSA's \HTTPD\ for the communication component of all DODS data
servers. Thus, each URL that refers to a data set must match the syntactical
requirements of that software (see the documentation on \HTTP). 

The Common Gateway Interface (CGI) mechanism of HTTPD is flexible enough to
allow sophisticated specialization of the semantics of WWW URLs without
requiring a change in URL syntax. DODS can further restrict the WWW syntax
for its URLs through conventions supported by the DODS server tool kit and by
forcing all access to take place via the HTTPD's CGI mechanism.

\section{Structure and Interpretation of URLs}
\label{url:structure}

A formal grammar for WWW \uri\ (URIs) exists and provides a framework for
extending the meanings of URLs beyond those used by the WWW project. In this
section the syntax for DODS URLs is described (See Table~\ref{url:tab:url}) and
the differences between the WWW and DODS URLs are also discussed. This syntax
is less formal than that presented in the CERN documentation, however, DODS
servers must be able to parse this syntax. In the case where a DODS server
forms an interface between other DODS components and a foreign system, the
DODS server must first parse this syntax and then translate the parsed
expression into the foreign system's syntax.

\subsection{Syntax of the DODS URL}

\begin{table}
\caption{DODS URL Syntax}
\label{url:tab:url}
\begin{center}
\begin{tabular}{ll} \hline
{\em DODS URL\/} & {\em Access\/} {\tt ://} {\em Host\/} {\tt /} {\em CGI\/}
{\tt /} {\em Args\/} \\
{\em Access\/} & {\tt http} \\
{\em Host\/} & {\em Domain name\/} \\
{\em CGI\/} & {\em CGI-directory\/} {\tt /} {\em String\/} \\
{\em Args\/} & {\em Pathname\/} \\
             & {\em Pathname\/} {\em Constraint} \\
             & {\tt (} {\em DODS URL\/} {\tt )} \\
{\em Constraint\/} & {\tt ?} {\em Expression\/} \\
\end{tabular}
\end{center}
\end{table}

\begin{description}

\item [Access] The access component of a DODS URL is always the literal
  string {\tt http}. The DODS data delivery architecture described in \DDA
  uses the World Wide Web \HTTPD\ server developed by NCSA to transport
  data. Note that just about any WWW server will work for DODS as long as it
  supports the CGI mechanism.

\item [Host] The host component is an Internet-style host name that
  can be resolved to an Internet address.
  
\item [CGI] The CGI directory is the directory where CGI programs are kept.
  This varies from machine to machine but many UNIX computer use {\tt
    cgi-bin}. In addition, it is possible to configure {\tt httpd} to search
  several directories for CGI programs; the DODS dispatch CGI may reside in
  any of these directories. The string is the name of the DODS dispatch
  CGI\@. In the software distributed by URI/MIT this is an abbreviation for
  the API (e.g., the dispatch CGI for NetCDF is {\tt nc}). However, server
  implementors are free to choose any name they want. In addition, data
  providers may rename this script if they want to.
  
\item [Args] The URL components named {\em Args\/} contains a pathname used
  by the CGI to locate the data set. This may or may not be an actual path on
  the remote machine. In some cases the DODS server may be an interface to a
  DBMS\@. In this case the pathname is used (probably in conjunction with
  other information) by the CGI to formulate a query to the DBMS\@. The
  pathname may have a constraint expression appended to it. This is used
  either by the filter program or the CGI to limit the data access beyond the
  pathname. It provides additional flexibility in the way data is accessed
  through the server. The arguments may also be a second URL\@. In this case
  the {\em Access\/} and {\em Host\/} components of the outermost URL (called
  an {\em envelope\/}) specify the target process (server) which should
  receive the embedded DODS URL\@. The source process (client) does not
  interpret this embedded URL at all---only the target process attempts to
  evaluate it.  Embedded URLs may be nested to any depth.

\item [Constraint] The constraint is a boolean expression used to constrain
  access to a variable. Different classes of variables can have different
  types of constraints. See~\API \externalref{tab:expr} for a more complete
  discussion of constraints.

\end{description}

\begin{table}
\caption{Sample URLs which name resources; sometimes there is more than one
  acceptable way to name a single resource.}
\label{url:tab:examples}
\footnotesize
\begin{center}
\begin{tabular}{| l | p{5cm} |} \hline
\multicolumn{1}{| c}{\sc URL} 
& \multicolumn{1}{ c |}{\sc References} \\ \hline \hline

{\tt http://gso.uri.edu/cgi-bin/nc/data/fnoc1.nc} 
& the entire data set \\ \hline

{\tt http://gso.uri.edu/cgi-bin/nc/data/fnoc1.nc?u} 
& {\tt u} from the data set \\ \hline

{\tt http://gso.uri.edu/cgi-bin/nc/data/fnoc1.nc/u} 
& {\tt u} using the virtual file system syntax \\ \hline

{\tt http://gso.uri.edu/cgi-bin/nc/data/fnoc1.nc?u<10} 
& all variables given that {\tt u} is $<$ 10 
(Makes sense for sequence data only) \\ \hline

{\tt http://lake.mit.edu/cgi-bin/jg/bloom/level1}
& all variables in {\tt  level1} \\ \hline

{\tt http://lake.mit.edu/cgi-bin/jg/bloom/level1/temp} 
& only {\tt temp} in {\tt level1} \\ \hline
\end{tabular}
\end{center}
\normalsize
\end{table}

\subsection{Characteristics of DODS and WWW URLs}
\label{url:differences}

DODS URLs are a proper subset of WWW URLs. URLs defined by the WWW project
cover a wide variety of client-server systems while parts of DODS URLs are
interpreted only to be DODS CGIs. WWW URLs include special cases for servers
which support protocols such as {\tt ftp}, {\tt wais} and {\tt gopher}. In
these cases, the {\em Access\/} component of a WWW URL is used to indicate
which protocol to use. When {\em clients\/} receive URLs, they use this
information to select which protocol should be used to access the data
referenced by the URL (i.e., the {\em Access\/} information is used to select
the type of server with which the client will communicate). DODS URLs are
only meaningful to CGI modules accessible over the network via HTTP servers.
Thus, there will never be any variation in the {\em Access\/} component of a
DODS URL\@. It will always be {\tt http}.

In addition to a protocol restriction, DODS URLs always reference a CGI
module on the server. This is necessary because all DODS data access takes
place via one or more CGI modules. The CGI specification must follow the {\em
Host\/} component of the URL\@.

While WWW URLs are the input to WWW browsers written with knowledge of URLs,
and in many cases, the ability to parse parts of URLs, DODS URLs are passed to
programs which nominally have no knowledge of URL notation. In \DDA\ a data
system architecture is presented in which user programs are re-linked with
new implementations of data-access APIs. These {\em client libraries\/}
understand and can manipulate URLs (and in particular, understand the
particular restrictions DODS places on URLs) but the user programs do
not. Some of the user programs were written before any of the URL software
even existed. 

A DODS URL references a data set, or portion of a data set. Accessing a DODS
URL returns an experimental-type binary \MIME\ document. Such a document
cannot be displayed by browsers nor does it contain URLs which link the
document to other documents on the Internet. URLs that are part of the WWW
system of HTTP servers reference documents which can be displayed by a large
number of difference browsers. DODS URLs are not browsable at all, that is,
they cannot meaningfully serve as input to one of the standard WWW
browsers\footnote{However, it is possible to add one of several suffixes to a
DODS URL and get text information which a browser can display. See
Section~\ref{urls:user-vs-generated} for more information.}. DODS uses URLs
simply to reference data sets on the Internet and not for the more general
purpose of linking a document within the larger context of all documents
publicly accessible on the Internet.

There are two ways DODS CGI modules can interpret the part of the URL which
the HTTP server passes to it; as a constraint expression or as an access to a
virtual file system. This second method of interpretation is a specialization
of the general behavior of HTTPD where the text following the CGI name is
passed to the CGI module unprocessed. Accessing a data set as if it were a
virtual file system means specifying variables within the data set as if they
were files or files within directories subordinate to the CGI\@. For example in
Table~\ref{url:tab:examples} the file name separator ({\tt /}) is used to
indicate that the within the named data set (the netCDF data set called {\em
  CDT\/}) only the variable {\tt u1} is to be accessed. If the data set
contained several levels of variables, as in the examples for the three level
JGOFS data set, then a specific level or variable within a level may be
specified.  However, the virtual file system syntax is not capable of
specifying arbitrary set of variables from a data set unless those variables
happen to be the sole members of a structure of sequence (See \DTP\ for more
information on data types DODS supports). The VFS syntax is a special case of
constraint expressions. It is included in DODS because file systems are
familiar tools to many users, much more so than boolean expressions, and
because they are flexible enough for many user's demands.

The second syntax DODS supports for the portion of the URL past the CGI name
is the constraint expression syntax. Constraint expressions are a way of
passing, along with the URL, a set of restrictions to be applied to the data
set when the client software (i.e., the surrogate library) accesses the data
set. These constraints will be used during access to the data set to limit
the variables and variable values extracted from it. The constraint
expression, or portions of the constraint expression, will be evaluated {\em
  by the server\/} at the time of the data access. The syntax of the
expression itself is dependent on the types of variables within the data set
(See \DTP).

\section{Uses of URLs by DODS}

This section briefly describes how URLs are used by the two major components
of DODS: the Data Locator and the Data Delivery Mechanism.

\subsection{URLs and the Data Locator}
\label{url:locator}

The Data Locator is used to find specific data sets within DODS\@. Users can
query the Data Locator and expect that it will return a URL which can be used
by other components in DODS to access that data set. These URLs may be used
as supplied to user programs or they may be edited first. Because the URLs
are ASCII text strings, no special software is required to edit or store
them. In addition, DODS URLs may be obtained using other methods such as
Mosaic, e-mail, or personal communications.

\subsection{Data Delivery and URLs}
\label{delivery}
\label{urls:user-vs-generated}

The Data Delivery Mechanism consists of libraries with which users re-link
their programs as well as stand alone utilities built at least partially from
those libraries. URLs are used verbatim by the data delivery components of
DODS\@. The URL is passed to the server or translator using the API {\em
  open\/} call (e.g., {\tt ncopen()}). The client library stub is responsible
for removing the envelope of the URL and sending either the path or embedded
URL to the correct host/server or host/translator.

When a user wants to access data via a DODS server they must give to a client
(i.e., a program linked with one of the DODS reimplemented API libraries) A
URL which references that data. Users may choose to supply a constraint along
with that URL effectively limiting the parts of the data set that the client
can see. For example, suppose a data set exists which contains several
arrays:

\begin{figure}
\begin{code}{cb}
    Dataset {
        Int32 u[time_a = 16][lat = 17][lon = 21];
        Int32 v[time_a = 16][lat = 17][lon = 21];
        Float64 lat[lat = 17];
        Float64 lon[lon = 21];
        Float64 time[time = 16];
    } fnoc1;
\end{code}
\caption{The DDS of a dataset.}
\label{url:fig:dds}
\end{figure}

\begin{figure}
\begin{code}{cb}
    Dataset {
        Int32 u[time_a = 1][lat = 17][lon = 21];
    } fnoc1;
\end{code}
\caption{The DDS of a dataset constrained by $u[2:2][1:17][1:21]$.}
\label{url:fig:dds2}
\end{figure}

The user can refer to the entire dataset using a URL without a constraint
expression;
\htmladdnormallink{http://dods.gso.uri.edu/cgi-bin/nc/data/fnoc1.nc}
{http://dods.gso.uri.edu/cgi-bin/nc/data/fnoc1.nc.dds}. If the user were to
specify this, then the client-library will receive from the server a DDS like
the one in Figure~\ref{url:fig:dds}. However, if the user knows that they
only want to work with a small part of the data set they can supply a
constraint expression along with the URL\@. For example, suppose that the
user only wants the latitude and longitude values for $time_a = 2$, and
furthermore that they are only interested in the array $u$. They could supply
the user program with the following URL:
\htmladdnormallink{http://dods.gso.uri.edu/cgi-bin/nc/data/fnoc1.nc?u[2:2][1:17][1:21]}
{http://dods.gso.uri.edu/cgi-bin/nc/data/fnoc1.nc?u[2:2][1:17][1:21].dds} The
user program will receive DDS in Figure~\ref{url:fig:dds2}. For program which
are designed to read the entire dataset without user interaction this is a
powerful additional feature.

However, constraint expressions have a second use in DODS\@. They are used by
the reimplemented APIs to extract specific parts of a data set when that is
requested by the user program. Many APIs provide features which make it
possible to write software which opens a data set, presents the user program
with a collection of variables and then provides a way for the program to
read one or more of those variables. In an API reimplemented for DODS, those
calls must all be satisfied by information the API receives from a DODS
server.

In order for the API to get information about the data set\footnote{Metadata
in one sense, although we shy away from that term since we have found that
`metadata' to one person is `data' to another; the categorization often
limits the usefulness of the underlying information.} the API must synthesize
various URLs using the one given by the user as a base. For example, to get
the DDS of the data set referenced by {\tt
http://dods.gso.uri.edu/cgi-bin/nc/data/fnoc1.nc} the reimplemented API must
append the suffix {\tt .dds} (e.g., {\tt
http://dods.gso.uri.edu/cgi-bin/nc/data/fnoc1.nc.dds}. Similarly, the DAS of
that data set is obtained by appending {\tt .das}. 

Getting data is a bit more work than getting the DAS or the DDS\@. The request
for a particular variable must be translated into a DODS constraint
expression. This constraint expression is then appended to the URL given by
the user and then the suffix {\tt .dods} is appended to that. For example,
suppose the user program makes an API function call requesting the value of
the array $u$ in the preceding figures. What the API is supposed to return is
the values of the entire array $u$, but no other values. The reimplemented
API would build the URL {\tt
http://dods.gso.uri.edu/cgi-bin/nc/data/fnoc1.nc?u.dods} where the $?u.dods$
specify that the variable $u$ is to be the only variable in the projection
and the {\tt .dods} selects the data filter from the DODS server. For more
information on the constraint expression syntax, see \DAP; for more
information on the DODS data servers, see \DDD\@.

\section{Deficiencies of URLs}
\label{url:deficiencies}

There are several features which URLs as defined by either DODS or the WWW
lack. These include:

\begin{enumerate}

\item Poor support for version information

\item No support for location-independent naming

\item No formal support for quality rating of data sets

\item No support for binding extent information within the name-space (e.g.,
  for how long will this data set exist)

\item No support for security information

\end{enumerate}

While these are important issues, many are being investigated by the Internet
Engineering Task Force (IETF) and the evolving \urn\ (URN) standard. By
following the de facto URL standard now, DODS is in a good position to benefit
from the future standardization of URNs. Further, some of these deficiencies
are not as severe as they might seem at first. For example, it is not clear
that security information (providing access to a limited-access resource)
belongs in a resource name. It might be more secure to design the resource so
that it prompts specially for any security information required. Similarly,
version and quality information may be incorporated informally to the
existing URL syntax.  Finally, a crude type of location independence may be
achieved using domain name service aliases for important resources. These
approaches are already in widespread use with the WWW\@.

Rather than form our own solutions for the deficiencies of URLs, DODS will
wait until either the IETF or WWW community modifies URLs and/or adopts URNs
or until continued development of DODS is significantly hampered by the
problems. 

\section{Conclusion}

In order to reference data sets on the Internet, DODS must use a naming
convention which includes more information than the conventional
pathname. Uniform Resource Locators (URL), as defined by the World Wide Web
(WWW) project also serve that purpose. Because URLs are a de facto standard,
adopting their syntax simplifies construction of DODS software, and eases
integration of DODS with the emerging Internet information base.

DODS restricts the set of URLs it can use to a proper subset of the URLs
acceptable to the WWW\@. This is done principally by stipulating that all data
access takes place via one or more Common Gateway Interfaces (CGI) which
extract data from the data set and return an experimental-type binary MIME
document. 

URLs, both as defined by the WWW project and in DODS's restricted sense, have
a number of short comings. The URL notation does not carry information about
the quality or longevity of the referenced resource. Furthermore, URLs are
dependent on current naming conventions used on the Internet and thus lack
`location independence'. However, the focus of DODS is in providing network
accessibility to scientific data and {\em not\/} solving the problems
associated with URLs as currently used by much of the existing Internet. By
using URLs, DODS stands to benefit from the ongoing efforts of groups that
are addressing the known deficiencies of URLs.

\end{document}
