
% $Id$
%
% Written by Glenn Flierl in early 1994. jhrg

\magnification=\magstep1
\raggedbottom
\def\title#1{\bigskip\goodbreak\centerline{\bf #1}\medskip}
\def\section#1{\medskip\goodbreak\noindent{\bf #1}\medskip}
\def\subsection#1{\medskip\goodbreak\noindent{\sl #1}\medskip}
\def\references{\narrower\parindent=-20pt}
\def\jour#1#2{{\sl #1}, {\bf #2}}
\def\ni{\noindent}
\def\nobox{\special{ps::/v{pop pop pop pop}def}}

\title{DODS --- Data Delivery}

After some discussion, Jim and I recommend a data delivery system
which combines aspects of the RPC[remote procedure call]--based model
and the virtual file system [database] model. We'll describe each of
these, pointing out the advantages we wish to retain.

\section{Virtual file system}

In this model, the DODS data system would appear to the users as a file
system, with familiar ``path'' specification for the information they
wish to retrieve. The concept of a path would be extended in three
ways: 
\item{(1)} permit specifying the final form the data is to appear in
(netCDF, HDF, ascii, image, ...)  
\item{(2)} permit specifying machine names [note: the virtual
directory allows one to hide the remote machine name within an alias
or abbreviation].
\item{(3)} permit specifying parameters for database operations,
especially ``selection'' and ``projection'' to be applied to restrict
the information being returned.

\ni We feel these additions can be made while retaining the advantages
of a virtual file system:
\item{(4)} Directory structure/ files provide a familiar and natural
structure for organizing and retrieving information.
\item{(5)} Most programs, written by scientists, or freeware, or
commercial, deal with files. Configuring the virtual data system so
that it is able to present the information as files permits these
programs to gather information directly and naturally from the data
system.\footnote{${}^1$}{We may not be be able to have commercial
programs interfacing as directly until we build a true virtual file
system --- which certainly won't happen in the first round.}

\ni What is wrong with a pure virtual file/ database system?
Essentially, the realization we came to was that translators could
become virtually impossible to build. We would be asking the
equivalent of byte--by--byte transformation of the information with
almost no knowledge of the structure of the information. At times,
this could require reading a large mass of information across the
network when all the user might want is a few values.

\section{RPC--based system}

In the RPC model, a particular data structure is defined by an
application program interface [API]. A user program is linked with
this API library and all data is gathered by calls to the various
subroutines. When this library is implemented with RPC's, the
required information -- what is required and how to transfer it is
passed back and forth over the network. Thus a call like

\smallskip
{\tt okflag=readvar(float\_buffer, varnum, begin\_element, end\_element)}
\smallskip

\ni would result in the integers {\tt varnum, begin\_element,
end\_element} being transferred to the remote machine, which would
read the selected stuff from the file and pass it back as floating
point numbers to be placed in the array {\tt float\_buffer}. In
addition, it would pass back the result {\tt okflag} to indicate
whether there was a problem (e.g., end of file) during the read.

Thus a translator program would essentially have subroutine entry
points corresponding to the API of the target system ({\sl e.g.}
netCDF) and, from these entry points, make calls with the source
system's API's ({\sl e.g.} JGOFS). Since there is a lot of
information embedded in the set of API calls --- and many of them will
be rather similar from system to system --- the translation job is
much more straightforward and indeed possible. (Of course, complete
translation will presumably not be possible in all cases, but
certainly much of the information can be passed through successfully.)

Figure 1 shows the types of configurations envisioned: In the upper
panel we show a user program calling an API corresponding to protocol
1, communicating with a server speaking protocol 1 and serving data
stored in the corresponding form --- a native mode server. In the
middle panel, we show a case in which the user program cannot
communicate with the native server --- the user program speaks
protocol 2 --- but the data provider has a alternate server which
speaks protocol 2. The server does whatever translation may be needed.
The final panel shows the generic case in which the user program and
the set of data servers made available by the provider have no common
languages: a translator is interposed in the stream of calls and
responses. It accepts the RPC calls from the program and makes
suitable RPC calls to the server. Note that there is a lot of
similarity between a translator and a non--native mode server.

\section{RPC/VFS system}

How do we combine the advantages of the RPC--based system with the
properties 4--5 of the VFS? Well, the standard interface to the I/O
system is a particularly simple API, consisting of {\tt open}, {\tt
read}, {\tt close}, and perhaps {\tt seek}. Replacing these with a
DODS RPC version means linking programs to a different {\tt stdio}
library. However, the information about data structure is not built in
to these low--level calls (that's the problem with translators in a
pure VFS system). So we need to pass two more pieces to the {\tt open}
call: 
\item{(6)} One to indicate that the DODS system should take interest in the
call.
\item{(7)} A second to indicate the desired output format: this piece
of information also seems important, since some programs, such as {\tt
more} would prefer to see the data in some kind of ascii listing,
while {\tt MATLAB} would prefer a particular floating point
representation. Or {\tt ghostscript} might like a PostScript
representation.

\ni Putting these requirements together with 1--3 suggests a
``filename'' syntax of the form 

\smallskip
{\tt dods://dcz.uri.gso/woce/cruise1/ctd(station=3\&press<100).asc}
\smallskip

\ni where the {\tt dods:} indicates the DODS protocol is in effect and
the server should negotiate to find the possible techniques for
exchanging information [requirement \$6 above], the {\tt //} prefaces
a machine name [\#2], the {\tt /woce/cruise1/ctd} is information that
the DODS server on {\tt dcz} will use to find the list of appropriate
servers [\#4]. The parameters in parentheses {\tt
(station=3\&press<100)} pass data system selection parameters [\#2].
Finally the {\tt .asc} specifies the output format [\#1, \#7] ---
which translator should be invoked.\footnote{${}^2$}{We still have not
fully decided on the ``virtual'' aspect, but expect that one will be
able to form links so that a complicated ``filename'' such as the one
above could be referred to by a shorthand of some kind.} 

When using one of the API's, specification of the form of the output
would be unnecessary; indeed, it would probably not be required to
specify the {\tt dods:} prefix either.

\section{Next steps:}

The plan is to implement the above procedures for data delivery. This
involves building examples for various pieces:
\item{(A)} {\sl User program API's:} NetCDF, JGOFS, stdio, ...
\item{(B)} {\sl Translators:} NetCDF $\rightarrow$ JGOFS, JGOFS
$\rightarrow$ NetCDF, NetCDF $\rightarrow$ ascii, ...
\item{(C)} {\sl Servers:} NetCDF with NetCDF data, JGOFS with NetCDF
data, JGOFS with various formats for which methods already exist, ...
\item{(C)} {\sl User programs:} more, cat, cp; ncdump, ferret; list, pl
and other JGOFS routines, ...

\end
