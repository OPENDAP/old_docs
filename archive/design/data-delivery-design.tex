
% This file contains a fairly detailed description of the design of the
% HTTP-based DODS data servers.
%
% $Id$
%

\documentstyle[12pt,html,psfig,margins]{article}

\psfigurepath{ddd-figs}

%
% These are html links which are used often enough in writing about DODS to
% merit an input file.
% jhrg. 4/17/94
%
% File rationalized and updated while writing the DODS User
% Guide. Also includes other useful abbreviations.
% tomfool 3/15/96
%
% $Id$
%
% HTML references to DODS documents
%
% For most of the DODS documentation there are three html references defined:
% 1) The upper case \newcommand produces the title of the paper, an
%    html link in the online documentation and a footnote in the paper
%    version.  
% 2) A capitalized version produces a capitalized description of the
%    paper and a link in the online version (but no footnote in the
%    paper version). 
% 3) A lower case version which produces a lower case description of
%    the paper and a html link, but no footnote.

% External references to documents:
%
% In order for external references to work (via latex2html) the labels used
% throughout the documents must be unique. The text labels are all prefixed
% with a document identifier (e.g., ddd) with a colon separater. The
% \externalrefs below (way below...) includes the perl files html.sty uses to
% make the cross refs.

\newcommand{\DODS}{\htmladdnormallink{Distributed Oceanographic Data System}
  {http://www.unidata.ucar.edu/packages/dods/}}

\newcommand{\Dods}{\htmladdnormallink{DODS}{http://www.unidata.ucar.edu/packages/dods/}}

\newcommand{\wrkshp}{\htmladdnormallink{Report on the First Workshop for the
    Distributed Oceanographic Data System, Proposed System Architectures}
  {http://www.unidata.ucar.edu/packages/dods/archive/reports/workshop1/section3.13.html}}

% not DODS URLs, but useful all the same...

\newcommand{\www}{\htmladdnormallink{World Wide Web} 
  {http://www.w3.org/hypertext/WWW/TheProject.html}}
\newcommand{\url}{\htmladdnormallink{Uniform Resource Locator}
  {http://www.w3.org/hypertext/WWW/Addressing/URL/url-spec.html}}

\newcommand{\uri}{\htmladdnormallinkfoot{Uniform Resource Identifiers}
  {http://www.w3.org/hypertext/WWW/Addressing/URL/uri-spec.html}}

\newcommand{\urn}{\htmladdnormallinkfoot{Uniform Resource Name}
  {http://www.acl.lanl.gov/URI/archive/uri-archive.messages/1143.html}}

\newcommand{\HTTPD}{\htmladdnormallinkfoot{HTTPD}
  {http://hoohoo.ncsa.uiuc.edu/docs/Overview.html}}

\newcommand{\HTTP}{\htmladdnormallinkfoot{HyperText Transfer Protocol}
  {http://www.w3.org/hypertext/WWW/Protocols/HTTP/HTTP2.html}}

%\newcommand{\HTML}{\htmladdnormallinkfoot{HyperText Markup Language}
%  {http://www.w3.org/hypertext/WWW/MarkUp/MarkUp.html}}

% CERN used to be `Conseil Europeen pour la Recherche Nucleaire'
\newcommand{\CERN}
  {\htmladdnormallink{European Laboratory for Particle Physics}
  {http://www.cern.ch/}}

\newcommand{\NCSA}
  {\htmladdnormallink{National Center for Supercomputing Applications}
  {http://www.ncsa.uiuc.edu/}}

\newcommand{\WWWC}{\htmladdnormallink{World Widw Web Consortium}
  {http://www.w3.org/}}

\newcommand{\CGI}{\htmladdnormallinkfoot{Common Gateway Interface}
  {http://hoohoo.ncsa.uiuc.edu/cgi/overview.html}}

\newcommand{\MIME}{\htmladdnormallink{Multipurpose Internet Mail Extensions}
  {http://www.cis.ohio-state.edu/htbin/rfc/rfc1590.html}}

\newcommand{\netcdf}{\htmladdnormallink{NetCDF}
  {http://www.unidata.ucar.edu/packages/netcdf/guide.txn_toc.html}}

\newcommand{\JGOFS}{\htmladdnormallink{Joint Geophysical Ocean Flux Study}
  {http://www1.whoi.edu/jgofs.html}}

\newcommand{\jgofs}{\htmladdnormallink{JGOFS}
  {http://www1.whoi.edu/jgofs.html}}

\newcommand{\hdf}{\htmladdnormallink{HDF}
  {http://www.ncsa.uiuc.edu/SDG/Software/HDF/HDFIntro.html}}

\newcommand{\Cpp}{{\rm {\small C}\raise.5ex\hbox{\footnotesize ++}}}

% Commands

\newcommand{\pslink}[1]{\small
\begin{quote}
  A \htmladdnormallink{postscript}{#1} version of this document is
  available.  You may also use \htmladdnormallink{anonymous
  ftp}{ftp://ftp.unidata.ucar.edu/pub/dods/ps-docs/} to access postscript files
  of all of the DODS documentation.
\end{quote}
\normalsize
}

\newcommand{\declaration}[1]{\small
{\tt {#1}}
\normalsize
}

% All the links from here down are for the white papers. I'm not sure that any
% of these are still valid given the reorganization of the web site. 5/12/98
% jhrg.

\newcommand{\DDA}{\htmladdnormallinkfoot{DODS---Data Delivery Architecture}
  {http://www.unidata.ucar.edu/packages/dods/archive/design/data-delivery-arch/data-delivery-arch.html}}
\newcommand{\Dda}{\htmladdnormallink{Data Delivery Architecture}
  {http://www.unidata.ucar.edu/packages/dods/archive/design/data-delivery-arch/data-delivery-arch.html}}
\newcommand{\dda}{\htmladdnormallink{data delivery architecture}
  {http://www.unidata.ucar.edu/packages/dods/archive/design/data-delivery-arch/data-delivery-arch.html}}

\newcommand{\DDD}{\htmladdnormallinkfoot{DODS---Data Delivery Design}
  {http://www.unidata.ucar.edu/packages/dods/archive/design/data-delivery-design/data-delivery-design.html}}
\newcommand{\Ddd}{\htmladdnormallink{Data Delivery Design}
  {http://www.unidata.ucar.edu/packages/dods/archive/design/data-delivery-design/data-delivery-design.html}}
\newcommand{\ddd}{\htmladdnormallink{data delivery design}
  {http://www.unidata.ucar.edu/packages/dods/archive/design/data-delivery-design/data-delivery-design.html}}

\newcommand{\URL}{\htmladdnormallinkfoot{DODS---Uniform Resource Locator}
  {http://www.unidata.ucar.edu/packages/dods/archive/design/urls/urls.html}}
\newcommand{\Url}{\htmladdnormallink{Uniform Resource Locator}
  {http://www.unidata.ucar.edu/packages/dods/archive/design/urls/urls.html}}
\newcommand{\dodsurl}{\htmladdnormallink{uniform resource locator}
  {http://www.unidata.ucar.edu/packages/dods/archive/design/urls/urls.html}}

\newcommand{\DAP}{\htmladdnormallinkfoot{DODS---Data Access Protocol}
  {http://www.unidata.ucar.edu/packages/dods/archive/design/api/api.html}}
\newcommand{\Dap}{\htmladdnormallink{Data Access Protocol}
  {http://www.unidata.ucar.edu/packages/dods/archive/design/api/api.html}}
\newcommand{\dap}{\htmladdnormallink{data access protocol}
  {http://www.unidata.ucar.edu/packages/dods/archive/design/api/api.html}}

\newcommand{\SOFT}
        {\htmladdnormallinkfoot{DODS---Software Development Environment}
  {http://www.unidata.ucar.edu/packages/dods/archive/managment/software/software.html}}
\newcommand{\Soft}{\htmladdnormallink{Software Development Environment}
  {http://www.unidata.ucar.edu/packages/dods/archive/managment/software/software.html}}
\newcommand{\soft}{\htmladdnormallinkfoot{software development environment}
  {http://www.unidata.ucar.edu/packages/dods/archive/managment/software/software.html}}

% I used to call the DAP the API, then for a few days I called it the
% DTP. These def's were used then. Rather than find everywhere they are used
% (not that hard, but...) I just hacked the macros. NB: the source file for
% the DAP document is still called `api.tex'

\newcommand{\API}{\htmladdnormallinkfoot{DODS---Data Access Protocol}
  {http://www.unidata.ucar.edu/packages/dods/archive/design/api/api.html}}
\newcommand{\api}{\htmladdnormallink{data access protocol}
  {http://www.unidata.ucar.edu/packages/dods/archive/design/api/api.html}}

\newcommand{\DTP}{\htmladdnormallinkfoot{DODS---Data Access Protocol}
  {http://www.unidata.ucar.edu/packages/dods/archive/design/api/api.html}}
\newcommand{\dtp}{\htmladdnormallink{data access protocol}
  {http://www.unidata.ucar.edu/packages/dods/archive/design/api/api.html}}

\newcommand{\TOOLKIT}{\htmladdnormallinkfoot{DODS---Client and Server Toolkit}
  {http://www.unidata.ucar.edu/packages/dods/archive/implementation/toolkits/toolkits.html}}
\newcommand{\Toolkit}{\htmladdnormallink{Client and Server Toolkit}
  {http://www.unidata.ucar.edu/packages/dods/archive/implementation/toolkits/toolkits.html}}
\newcommand{\toolkit}{\htmladdnormallink{client and server toolkit}
  {http://www.unidata.ucar.edu/packages/dods/archive/implementation/toolkits/toolkits.html}}

\newcommand{\TKR}{\htmladdnormallinkfoot{DODS---DAP Toolkit Reference}
  {http://www.unidata.ucar.edu/packages/dods/archive/implementation/toolkits/toolkits.html}}
\newcommand{\Tkr}{\htmladdnormallink{DAP Toolkit Reference}
  {http://www.unidata.ucar.edu/packages/dods/archive/implementation/toolkits/toolkits.html}}
\newcommand{\tkr}{\htmladdnormallink{DAP toolkit reference}
  {http://www.unidata.ucar.edu/packages/dods/archive/implementation/toolkits/toolkits.html}}

% I know that these are wrong. The papers have been moved to the gso web
% site, but I'm not sure exactly where. 5/12/98 jhrg

\newcommand{\DM}{\htmladdnormallinkfoot{DODS---Data Model}
  {http://lake.mit.edu/dods-dir/dodsdm2.html}}
\newcommand{\Dm}{\htmladdnormallink{Data Model}
  {http://lake.mit.edu/dods-dir/dodsdm2.html}}
\newcommand{\dm}{\htmladdnormallink{data model}
  {http://lake.mit.edu/dods-dir/dodsdm2.html}}

\newcommand{\DD}{\htmladdnormallinkfoot{DODS---Data Delivery}
  {http://lake.mit.edu/dods-dir/dods-dd.html}}

% external refs for DODS documents

\externallabels{http://www.unidata.ucar.edu/packages/dods/design/api}
  {/home/www/packages/dods/archive/design/api/labels.pl}
\externallabels{http://www.unidata.ucar.edu/packages/dods/design/data-delivery-arch}
  {/home/www/packages/dods/archive/design/data-delivery-arch/labels.pl}
\externallabels{http://www.unidata.ucar.edu/packages/dods/design/data-delivery-design}
  {/home/www/packages/dods/archive/design/data-delivery-design/labels.pl}
\externallabels{http://www.unidata.ucar.edu/packages/dods/implementation/toolkits}
  {/home/www/packages/dods/archive/implementation/toolkits/labels.pl}

%\renewcommand{\externalref}[2]{#2}

%%% Local Variables: 
%%% mode: latex
%%% TeX-master: t
%%% End: 


\begin{document}

\title{DODS---Data Delivery Design}
\author{}
\date{23 August 1996} % was \today

\maketitle

\begin{abstract}
  This paper describes the design of the HTTP-based data servers for the
  \DODS, and the clients that access them. The architecture for this system
  is described in ``\DDA'' and the communications protocol used by the
  components is described in ``\DAP''. Those papers are required reading for
  this one. The data servers are built by assembling three programs which
  read parameters passed from \NCSA's \HTTPD\ to the programs via the \CGI
  interface. In the design, HTTPD is used as a network I/O management tool
  which simplifies setup and maintenance of the data server. The programs
  which comprise a data server return \MIME\ documents as query results. This
  paper also describes how to build DODS client libraries which use the
  information provided by data servers to mimic the responses produced by
  various data-access Application Programmer Interfaces (API) when they are
  used to read data files or data sets. While the construction of the data
  servers is typically straight forward, building the surrogate library for a
  particular API can be very complicated. This complexity is the result of
  the need to translate from the DODS \Dap\ (DAP) representation of
  information to the representation used by the API\@. To accomplish this
  translation, the programmer must have a solid command of both the DAP and
  the target API\@. Although hard to implement, this system has the advantage
  of providing for interoperability between several different APIs supported
  by DODS\@. As new APIs are added the set of interoperable APIs will further
  expand.
\end{abstract}


% Biolerplate text warning that anything can (and almost certainly will)
% change. 
%
% jhrg 3/25/94
%
% $Id$

\small

\centerline{{\bf Note}} 

\begin{quotation}

\begin{htmlonly}
  This is a working document made available to solicit comments. To receive
  e-mail notice of new or significantly changed documents, send e-mail
  to \htmladdnormallink{majordomo@unidata.ucar.edu}
  {mailto:majordomo@unidata.ucar.edu} with 'subscribe dods' in the
  body of the message.
\end{htmlonly}

\begin{latexonly}
  This is a working document made available to solicit comments. To receive
  e-mail notice of new or significantly changed documents, send e-mail
  to majordomo@unidata.ucar.edu with 'subscribe dods' in the body of
  the message. 

  Hypertext references in the hardcopy version of this document are
  represented using footnotes that contain a Universal Resource Locator (URL)
  to the referenced document.  To read the online documentation, open the URL
  `http://www.unidata.ucar.edu/' with a World Wide Web browser.
\end{latexonly}

\end{quotation}

\normalsize

%
% Biolerplate text expaining that this document is intended for programmers.
%
% jhrg 4/17/94
%
% $Id$

\small

\begin{quotation}
  This document is written for people interested in implementing software
  that will be part of, or work in parallel with, the Distributed
  Oceanographic Data System. 
\end{quotation}

\normalsize

\begin{htmlonly}
\pslink{ftp://ftp.unidata.ucar.edu/pub/dods/ps-docs/data-delivery-design.ps}
\end{htmlonly}

\tableofcontents

\newpage

\section{Introduction}

This document describes the design and construction of the data delivery
components of the \DODS\ (DODS). The architecture of DODS is described in
detail in ``\DDA''.  DODS is a client-server system which differs from the
conventional notion of such a system, such as {\tt ftp}, because it has many
different clients, not just one.  Instead of the implementor(s) of DODS
building a single client program to read and display data, they build
replacements for existing data access application programmer interfaces
(API). These replacement, or surrogate, API implementations are then linked
to any data access, display or analysis software which uses that API to
access data. The surrogate API implementations use the DODS tool kit, in
conjunction with the \HTTPD, to access data stored on remote computers which
have installed DODS data servers.  Because the reimplemented APIs can access
data over the network using DODS data servers, application programs linked
with them can also access such data, even though the API calls they make are
unchanged.

In addition to the remote access of data sets, DODS provides a limited degree
of API transparency. The DODS data servers transmit information using the
\Dap\ (DAP).  The DODS surrogate libraries issue requests and read data using
the DAP, while the DODS data servers answer requests and provide information
using the DAP as well. Both the surrogate libraries and the data servers can
be said to {\em translate\/} the native access mode of a API or storage
format into or out of the DAP\@. Because a single transmission/access protocol
is used to move all information within DODS, any surrogate library can, in
principle, access data from any of the data servers, regardless of the
storage format of the data accessed by that server. Thus, for example, data
stored in \netcdf\ files can be read using software designed to work with the
\jgofs\ system.

Using the \dap\ has a price. Servers must translate calls in the DAP to calls
in the data set API or, if the data set has no associated API, to various
reads which get information from the data set. Either way the data server
must translate the DAP accesses into the appropriate accesses for the data
set. Section~\ref{ddd:data-servers} describes the design and construction of
these data servers. In addition, the surrogate libraries must translate calls
from the API they replace to calls in the DAP\@. Section~\ref{ddd:client}
describes how these surrogate libraries are built.

Several designs were considered for the data delivery mechanism of DODS\@.
They were socket-based peer-to-peer communications, RPC-based peer-to-peer
communications, virtual file systems and HTTP/CGI-based client/server
systems. The first three of these different designs are compared in: \wrkshp
and ``\DD'', which presents our rationale behind prototyping the RPC-based
design for DODS\@. However, as a result of those prototypes and the development
of HTTP as a {\em de facto\/} data communications standard, we changed the
data delivery design to an HTTP/CGI-based system.

By using HTTP as a transport protocol, we are able to tap into a large base
of existing software which will likely evolve along with the Internet as a
whole. Because the development of large-scale distributed systems is
relatively new\footnote{Even though the UNIX/Internet combination has existed
for more than 10 years, it is only recently that systems like the WWW have
seen widespread use. Previously data was shared over the Internet using ftp
or similar systems.} there are many problems which must still be addressed
for these systems to be robust. These problems include naming resources
independently of their physical location, and choosing between two objects
which appear to be the same but which differ in terms of quality. These are
general problems which are hard to solve because they will be solved
effectively only when the Internet community reaches a consensus on which of
the available solutions are best. HTTP, because it is so widely accepted,
provides a reasonable base for such solutions. This view is supported by the
recent \htmladdnormallink{Internet Engineering Task Force
(IETF)}{http://www.ietf.cnri.reston.va.us/home.html} work on extending the
\HTTP\ and \HTML\ standards.

\subsection{Process Configuration}
\label{ddd:process-config}

The collection of processes involved in a single data transfer from a remote
archive to a user program is shown in Figure~\ref{fig:client-executables}.
In the figure, the components of each process are also shown. Initially the
user program, which has been re-linked with the surrogate API
implementation\footnote{In order to use a given data display or analysis
program with DODS, users or system managers must re-link those programs with
the DODS reimplementations of the APIs used by those programs. DODS currently
supports two APIs: netCDF and JGOFS\@. Support is planed for HDF and the
formats GRIB and BUFR\@.}, sends a \Url\ (URL) to the \HTTPD\@. HTTPD decodes the
URL and executes the named CGI, passing to it parameters via environment
variables. The CGI is run in a separate process, so once started, it
operates independently from HTTPD\@. The CGI in turn examines its arguments and
determines which of the DODS filter programs to execute. The filters
write to standard output which, in this case, is a  socket connecting the
filter to the remote client (because the filter is a child process of the CGI
which in turn is a child of {\tt httpd}; {\tt httpd}'s standard input and
output are connected to the remote client).

Each machine that makes data available to DODS clients must, first and
foremost, have all the data to be accessible {\em on line}. By on line we
mean that the data must be accessible without human intervention and without
significant delay. It is unreasonable to expect that data to be incorporated
into a user's display package will take a day to be transferred to their
computer! Ideally, DODS users will not be able to tell that data they access
is remote. This does not mean that data must be on a hard disk, however. Data
could be stored on an auto changer, in a special purpose database machine or
other mass storage device.

In addition to an on line data set, data providers will need to run HTTPD and
will need to install a CGI and a set of filter programs. HTTPD is used as a
network interfaceq, security and access logging tool for DODS\@. It is a
standard piece of software distributed by \NCSA\ and is already installed on
many computers.

In addition to HTTPD, a DODS data server consists of a CGI and a least three
filter programs (See Figure~\ref{fig:server-executables}). The three filters
handle, respectively, the client's requests to get the Data set Attribute
Structure~\externalref{api:das} (DAS), the Data set Description
Structure~\externalref{api:dds} (DDS) and to read the values of variables in
the data set. These three executables along with HTTPD and the dispatch CGI
make up a complete DODS data server.

\begin{figure}
  \centerline{\psfig{figure=process-config4.ps}}
  \caption{Run-Time Processes. The client connects to a HTTP server which then
    forks and executes the CGI/Translation module software. The latter is
    used to access the data.}
  \label{fig:client-executables}
\end{figure}

\begin{figure}
  \centerline{\psfig{figure=server-executables.ps}}
  \caption{Executables needed to support access to a data set for DODS
    clients. This collection of three CGIs along with the appropriately
    coded translation modules and linked to the original implementation of
    the API used to access the data set constitutes a {\em Data Server\/} for
    DODS\@.}
  \label{fig:server-executables}
\end{figure}

\subsection{Client/Server and Program/Library Interfaces}

The DODS client library and data server programs communicate information
using URLs and MIME documents.
Figures~\ref{fig:data-transport}~and~\ref{fig:url-params} show these two
communication paths.

All information sent from the filter programs to the client library is
enclosed in a MIME document. Two of the three programs return information
about the variables contained in the data set as {\tt text/plain} MIME
documents. These documents can then be parsed by software in the client
library. In addition, these text documents can be read by any software that
can process {\tt ASCII} text. Thus, the responses made by the server are
specifically suited to use by the DODS client libraries, they can also be
used by many more general programs. For example, it is possible the use a
general purpose World Wide Web browser to `read' these documents.

The third filter program returns binary data encoded using Sun Microsystems
External Data representation (XDR) scheme prefixed with text which descibes
the data type of the binary information\footnote{The text is a mini-DDS which
can be parsed and instantiated using the DODS tool kit software.}. The data
is enclosed in a binary MIME document whose subtype is experimental. This
document can be read by software that is part of the client library using
additional information about the data set that is stored as state information
by the client library. Because correct interpretation of this document
requires knowledge of the format of the binary information, it is not
possible for general purpose WWW browsers to interpret this file (although
most browsers can read and save to disk any arbitrary data).

In order to provide link-time compatibility with the original API libraries,
the DODS client libraries must present {\em exactly\/} the same external
interface as the original libraries. However, these new libraries perform
very different operations on the data (although, for a API used to access a
self-describing data format the operations are analogous). One difference
between the two is that most data access APIs use file names to refer to data
sets. In the simplest case these file names are given on the command line by
the user and passed, without modification, to the API\@. The API uses the file
name to open a file and returns an identifier of some type to the user
program. Subsequent access to the data are made through this identifier.

In this simple scenario, it is possible to substitute a URL in place of a
file name (in part because both are stored in string-type variables). This
same user program can be invoked, on the command line, using a URL in place
of the file name. The program will, in almost all cases, pass the URL to the
API to open the data set. However, since the user program has been re-linked
with the DODS re-implementation of the API, the functions in the API will
correctly interpret the URL as a remote reference (see
Section~\ref{ddd:client-lib-open}). Clearly, one requirement that a user
program must meet in order to be re-linked with DODS is that it must not
itself try to open or otherwise manipulate the `file name' which will be
passed to the API\footnote{It is possible that a user program might perform
  an operation such as looking in one data base for a file name and then
  opening the named file---this doesn't count as a manipulation of the file
  name.}.

As shown in Figure~\ref{fig:url-params}, the client library does manipulate
the URL used to access the data set. A DODS data server is actually comprised
of three programs which each respond to a different type of data as described
by the \Dap. These programs are accessed as WWW documents via the HTTPD using
the CGI mechanism. By {\em convention\/} the data set name consists of the
HTTP access string ({\tt http://}) followed by the machine name and a string
which the data providers choose depending on the name and location of the
dispatch CGI and the data. In order to read the DAS, DDS or data, this URL is
modified, again by convention, by adding a suffix. The three suffixes
currently (as of 23 August 1996) % was \today
recognized are {\tt .das} for the dataset attribute
structure, {\tt .dds} for the dataset descriptor structure, and {\tt .dods}
for binary data.

\begin{figure}
\centerline{\psfig{figure=data-transport.ps}}
\caption{MIME documents are used by the server programs to return information
  to the client processes.}
\label{fig:data-transport}
\end{figure}

\begin{figure}
\centerline{\psfig{figure=url-params.ps}}
\caption{URLs are used in place of filenames to reference remote data sets
  accessible via a DODS data server.}
\label{fig:url-params}
\end{figure}

\section{Data Servers}
\label{ddd:data-servers}

A DODS data server is a collection of three executable programs and a CGI
which provide access to data using HTTP\@. Each of the program/CGI units is
capable of satisfying one of the three requests which are defined in the
\Dap. The DAP defines two ways to access metadata which describes the
contents of the data set (one for use by the DODS surrogate libraries and one
for use by third-party software and users) and one way to access data. Data
access is accomplished by reading the variables which comprise the data set.
This access can be modified using a {\em constraint expression\/} so that
only portions of the variable are actually read from the data set.

An essential characteristic of a DODS data server is that it be capable of
evaluating the constraint expression associated with a data access request.
In many cases reading a single variable from a data set results in data which
is of little or no interest to the user. Often users are interested in those
values of a variable which meet some additional criteria (e.g., they fall
within a certain time range). For a complete description of the data types
supported by DODS and the constraint expression operators, see ``\DAP''.

\subsection{Basic Requirements for the Data Servers}

The data servers must satisfy two requirements: they must provide access to
data via the DAP and they must use on-line data without requiring its
modification. 

Providing access to data using the DAP is necessary because that is how the
DODS architecture provides interoperability between different APIs. Because
the data servers translate accesses to a data set from the DAP into either
an API (e.g., \netcdf\ if the data set is stored using that API) or a special
format (e.g., GRIB), any (client) process that uses the DAP can access the
data. The underlying access mechanism is hidden from the client. 

In the current design of DODS, meeting this requirement means that for
each API or format in which data is stored, a new DODS data server must be
built\footnote{In practice data servers for the supported APIs and several
  data formats will be available with the DODS software distribution.}.

The other requirement which each server must satisfy is that data, however it
is stored, should not require modification to be served by a DODS data
server. This is important because many data sets are large and thus very
expensive to modify. It is a poor practice to force data providers to modify
their data to suit the needs of a system. Rather, DODS data servers must be
able to translate access via the DAP into the local storage mechanism
without changing that local storage mechanism.

This requirement limits DODS to those APIs and formats which are, to some
extent, self-describing. Because the DAP bases access on reading a named
variable, it must be possible, for each data set, to define the set of
variables and to `read' those variables from the data set. However some data
sets do not contain enough information to make remote access a
reality. Instead, additional information, not in the data set itself, is
needed. This information can be stored in ancillary data files which
accompany the data set. Note that these files are separate from the data
set, they are not added to the data set and do not require any modification
of the data set. 

\subsection{Construction of the Data Server}

The data server structure is shown in Figure~\ref{fig:server-executables}.
The `data server' is actually a collection of three separate programs which
correspond to the three data objects in the DAP\@. There is no coordination of
effort between the three programs; each operates independently\footnote{Some
  optimizations may introduce limited dependence via a data cache.}.  

Each of the three programs which comprise the data server use the \CGI
mechanism to receive parameters from \HTTPD\ (which is used to invoke the
programs and manage the network I/O connections---see
Figure~\ref{fig:client-executables}). Once parameters have been parsed, the
programs read information from the data set which is used to satisfy the
query. Accesses to the data set are made using an established API\@. Thus each
of the programs consists of two modules:q CGI interface and a data-access
API\@. The paper ``\TOOLKIT'' contains information on software which is
designed to aid in the construction of these parts of the data server
software.

\subsubsection{Building the DAS and DDS filter programs}

The three programs which comprise a DODS data server can be spearated into
two groups, one which provides access to `metadata' and one which reads
data variables from the data set. The DAS and DDS are used by the DAP to
transfer information about the data set to the client. As described in
``\DDA'', the client must decode these structures and use some of the
information in orchestrating variable reads. While the DAS and DDS contain
information that is semantically distinct, the process of building these two
structures is almost identical.

Figure~\ref{fig:server-das} shows a structure chart for the modules that
build the DAS\footnote{Structure charts are used to describe the design of
  the data server programs rather than object diagrams because these
  components depend, to a large extent, on the functional APIs used to access
  the data sets.}. In order to build the DAS object, the DAS program must
parse the parameters passed to the CGI\@. Once the parameters have been parsed,
the program must build a DAS object. The construction of the DAS object can
be simplified by using the DODS \Toolkit\ software. This software handles
the reading data set attribute information stored in the ancillary files (see
the DAP section on ancillary data) and integrating it with information that
is already part of the DAS object. The toolkit also contains
contains software which builds an ASCII text representation of the DAS for
transmission to the client.

In order to completely build the DAS it is, in general, necessary to read the
data set using whatever mechanism is natural for those operations. In many
cases, data sets which are part of DODS will be accessible using an
established API\@. Thus, part of the DAS data server program must use the data
set's API to read information about the variable's attributes (as defined by
the \dap) and record them in the DAS object. The \toolkit\ software contains
DAS object member functions which simplify this task somewhat, but people who
intend to build the DAS program must learn both what type of information is
stored in the DAS object as well as how to read that information using the
data set's API\@.

The construction of the DDS is essentially similar to that of the DAS\@. The
real difference between these two programs is in the content of the object
which they build and send to the client. Thus, the only functional difference
is in the specific API function calls used to build the object.

\begin{figure}
\centerline{\psfig{figure=server-das.ps}}
\caption{Structure Chart for the Data set Attribute Structure Server Program.}
\label{fig:server-das}
\end{figure}

\subsubsection{Building the Data Read Program}

While the two programs used to send the DAS and DDS to the client both may
read from the data set, the {\em send data\/} program necessarily {\em
  must\/} read from the data set. This program reads the values of a single
variable in the data set (as defined by the \Dap) and packages it in a \MIME
document. This MIME document is then sent to the client process. The client
process must decode the binary MIME document and internalize the data it
contains (see Section~\ref{ddd:client}).

A structure chart for the send data program is shown in
Figure~\ref{fig:server-read}. This structure chart shares some features with
the DAS chart in Figure~\ref{fig:server-das}; it also has modules for CGI
parameter parsing and construction of a MIME document. However, there are
some important differences between the two figures. The MIME document sent to
the client by the send data program contains both text and binary parts,
while the DAS and DDS programs send text-onlyq MIME documents. In addition,
the send data program uses the DDS object (which contains instances of
objects for each variable in the data set) when it performs the read
operation.

The implicit dependency between the DDS filter and data filter programs is
important. In order for the send data program to function, it must invoke the
read member function for the requested variable. However, to get the variable
object, the send data program must have access to the DDS object.  Thus,
before send data can do anything beyond parsing its arguments, it must build
the DDS object. This may be accomplished in several ways: the DDS object may
be cached (either as binary data or using the text representation built for
transmission to the client), it may be built explicitly by the send data
program via the DDS program (the send data program acts as a client of the
DDS send server program), or it may be built by including the DDS send
program's logic in the send data program.

Once the DDS object has been instantiated by the send data program, it must
invoke a member function to read the desired variable from the data set. The
read member function for each class of variable described in the \dap\ must
be specialized for any API used to read the data set. The specialization of
the data type hierarchy for a particular API is described in the
documentation for the \toolkit.

Once the data for the variable has been read into the correct object within
the DDS, the data filter must build a multipart MIME document so that the
data can be returned to client program. This MIME document contains a text
part and a binary part. The text part consists of a  DDS which describes just
the variable in this document (as opposed to every variable in the entire
dataset) This is straightforward to make given the completet DDS and the
\toolkit\ software. Following the text section, the data filter program must
append the value of the variable (encoded using XDR). 

\begin{figure}
\centerline{\psfig{figure=server-send.ps}}
\caption{Structure Chart for the Data set Variable Read Server Program.}
\label{fig:server-read}
\end{figure}

\subsection{Data Server Optimizations}
\label{server-opts}

There are two types of optimizations which can clearly boost performance of a
DODS data server\footnote{Which is actually a collection of three largely
  independent programs.}. Both of these optimizations use caching to store
results from one of the programs for later use or reuse by that or another
program. 

\subsubsection{Reusing DAS and DDS Objects}

Because the construction of the DAS and DDS objects requires that an entire
data set be scanned, it can become very inefficient to continually rebuild
these objects. Because the DAS and DDS server programs use a text
representation for transmission of this object from the server to the client,
and because text is easily stored by computer systems, it is possible to
store both the DAS and DDS objects once they have been created. Subsequent
accesses to these objects can be accomplished by reading and transmitting the
textual representation without, in the case of these two programs, actually
building the binary data object. 

When taking advantage of this optimization, it is important that the server
check the date stamp of the DAS/DDS text objects and compare it to the latest
modification date of the data set. For any data set to which new data is
periodically added, the DAS/DDS text object must clearly be updated so that
the cached text object matches exactly the object that would be created if
the object were built by querying the data set.

The update of the DAS/DDS text object can itself be optimized significantly.
It is not actually necessary to completely re-read the entire data set.
Because the software used to build both the DAS and the DDS binary objects
work incrementally, it is possible to read the DAS/DDS text object into
memory (i.e., build the binary object from its text representation) and then
read only the new parts of the data set. The binary object will be added to
as needed. NB: The DAS/DDS software may not properly update {\em changed\/}
data (data that was present in the previous incarnation of the data set, but
is now different in the current version) nor is it straightforward to remove
data which is no longer present in the data set.

\subsubsection{Using the DDS object by the Send Data Program}

Once the data server programs have been optimized to store the DAS and DDS
text objects for future use, it is a simple matter to modify the send data
program so that the cached DDS text object is used by that program. 

\section{User Programs and DODS Client Libraries}
\label{ddd:client}

Most client/server systems are built around a fixed set of client and server
programs. Often, the designers of such a system specify a communication
protocol to be used by the two processes and then design a server and a
client program. Because the communication protocol used by the system is
documented, it is fairly easy for a third party to build new clients (or
servers, although that is less common) for the system. An example of such a
system is the \www\ designed initially at the \CERN\@. Subsequent to the Web's
appearance, many new client programs have been introduced by third parties.
In fact, the DODS client libraries are a type of WWW client program.

However, the DODS client libraries are different from most WWW browsers
because they are themselves used to build programs. The DODS netCDF library,
for example, can be used to add network connectivity to any program which
uses netCDF\@. Even though the DODS client libraries contain all the software
needed to be full-fledged WWW clients, they are {\em not\/} intended to be
stand alone programs themselves; instead they are tools which can be used to
add network I/O capability to programs which already use one of the DODS
supported APIs. 

Because the DODS client libraries are designed to be complete {\em
  replacements\/} for existing libraries, they must be written so as to
faithfully emulate the original implementation's behavior. If the client
library meets this criterion, then it can be used with existing programs
without requiring those programs be modified in any way. This is the
fundamental requirement which any DODS client library must meet.

\subsection{Client Library Functions}

The central premise of the DODS \dda\ is that it is possible to categorize
many APIs used to store/access scientific data as following closely with the
UNIX file system paradigm of open, read and close\footnote{Because the DAP is
  a read-only protocol, we are not concerned with any operations that write
  data.}. Furthermore, it is assumed that any state information used by the
API can be stored locally in the client process and does not need to be
retained by the data server.

In order to build the client library for a particular API, it is necessary to
re-implement that API so that it uses data that can be acquired from the DODS
\dap\ rather than the files or other data store it was originally design to
use. The client library must also maintain the illusion of a connection
(state) for each open data object (typically a file) even though no such
connection actually exists (because the DAP is a stateless protocol). 

To accomplish these tasks with a minimum of difficulty, it is useful to divide
the API to be re-implemented in five types of functions:
\begin{itemize}
\item Open or connect
\item Variable information read
\item Data read
\item Write 
\item Close or disconnect
\end{itemize}

\subsubsection{Rewriting the Open and Close Functions}
\label{ddd:client-lib-open}

The functions which perform the open/close operations will require complete
re-implementation so that information about the data set can be retrieved
from the data server. These re-implemented functions must store the necessary
state information so that subsequent accesses for variable information or
data reads can be satisfied. This state information will, in almost every
case, consist of the data set attribute structure (DAS) and dataset
descriptor structure (DDS).

Figures~\ref{fig:client-open}~and~\ref{fig:client-close} are structure charts
for the DODS client library open and close functions. In order to write the
DODS client library version of {\em open\/} for a given API, the function
must first determine if the data object (typically a file) is local to the
user program making the open call or is a remote data object to be accessed
through DODS\@. It is possible to access DODS objects which are local to a user
program, but there is little reason to do so if the data object can be
accessed through the API used by the user program. In any case, the
distinction of local or remote is made on the basis whether a URL is used to
reference the data object. If that is the case, the object is assumed to be
remote and is accessed using a DODS data server, otherwise it is assumed to
be local and is accessed using the functions of the original API
implementation.

If the data object is remote, then the open function must build a structure
which can hold the DAS and DDS objects which describe the named data
set. Once this object is built, the open function must map this structure to
a file identifier or pointer which can be passed back to the user program as
the return value of the open function. Subsequent accesses to the data set
will include this identifier (or pointer), and each function which is a
member of the API can be modified to use it to gain access to the state
information stored by the open function.

The close function must use the state information accessible with the file
identifier or pointer returned by the open function to determine if the data
set is local (and hence manipulated using the original API implementation) or
remote. In either case, the appropriate actions necessary to free allocated
storage, \ldots, must be taken. In the case of a local data set, the original
implementation's {\em close\/} function must be called. In the case of a remote
data set, the locally stored state information must be freed.

The network I/O tool kit which is part of the DODS software distribution
contains utility functions and classes which simplify most of these
operations. 

\begin{figure}
\centerline{\psfig{figure=client-open.ps}}
\caption{Structure Chart for the Open/Connect Functions in a DODS Client
  Library}
\label{fig:client-open}
\end{figure}

\begin{figure}
\centerline{\psfig{figure=client-close.ps}}
\caption{Structure Chart for the Close/Disconnect Functions in a DODS Client
  Library}
\label{fig:client-close}
\end{figure}
 
\subsubsection{Getting Information about Variables}

Most APIs for self-describing data sets include functions which return
information about the variables that comprise a data set. These functions
return information about the type and shape of variables in a form that can
be used by a program as well as ancillary information about the variables
that is more often than not intended for use by humans. Each of these
functions must be rewritten so that to the extent possible, information
present in the DAS and DDS is used to satisfy them.

Figure~\ref{fig:client-metaread} shows a simple structure chart for this type
of operation. While many `self-describing' APIs may have 20--30 or more of
these functions, the basic structure of the re-implemented code is the same
for each one. If the data set is local, use the original implementation,
otherwise use the locally storedq state information (DAS and DDS) to answer
the function.

However, rewriting these functions can be the most labor intensive part of
re-implementing a given API\@. This is typically the largest group of functions
in the API and the information stored in the DAS and DDS must often
be `massaged' before it fulfills the specifications of the API\@. Thus the
rewritten functions must not only get the necessary information from the DAS
and DDS objects, but they must also transform the types of the objects used
to return that information to the user program into the data types the
program expects.

The DODS \Toolkit\ contains \Cpp\ classes which both define the DAS and DDS
objects and provide member functions which simplify this process.

\begin{figure}
\centerline{\psfig{figure=client-metaread.ps}}
\caption{Simple Structure Chart for the Metadata Access Functions in a DODS
  Client Library}
\label{fig:client-metaread}
\end{figure}

\subsubsection{Reading the Values of Variables from a Data Set}

Reading variables from a data set is the last type of API function which
requires serious attention. In order to accomplish this task, the API
functions which read data must be identified and modified to use data
gathered in the objects which are part of the DDS object. However, before any
of the variable-type objects in the DDS can contain any information a request
to read one or more variables must be issued to the remote data server.

In order to read the contents of a variable from the data set into the
appropriate object within the DDS, the API read function(s) must be rewritten
so that any information needed to build the correct URL is retrieved from the
DDS\@. Once the URL has been built it is used to access a MIME document which
contains, in binary form, the data in the variable. This information is then
added to the correct object in the DDS\@. The read function then reads this
information and returns it to the calling program.

All binary (i.e., variable value) information transferred by DODS is
transfered as multipart \MIME\ documents. These documents contain a text
description of the data type and the binary representation of the variable's
value (the later is encoded using Sun's External Data Representation (XDR)
standard). Once the binary data is received by the client library, it is
transformed, using functions from the XDR library, from XDR's canonical
network representation to the local machine's representation. This binary
information is then stored in the object with the DDS\@. Thus, while binary
information is transferred from server to client using XDR, the binary
information stored in the DDS is in the client CPU's representation.

The DODS \toolkit\ provide \Cpp\ software which can be used to simplify
reading variables and decoding the XDR-encoded binary data. However, each
client library must provide code which extends the client and server tool kit
class library to handle any special formatting/typecasting requirements of a
particular API\@.

\begin{figure}
\centerline{\psfig{figure=client-read.ps,width=6in}}
\caption{Structure Chart for the Data Read Functions in a DODS Client Library}
\label{fig:client-read}
\end{figure}

\subsubsection{Functions that Write to Data Sets}

DODS is a read-only data system. While it is not technically inconceivable, a
system which allows modification of remote data sets is operationally much
more complex. Thus, functions that write data are rewritten so that they call
the original implementation in the case of a local access or return an error
code in the case of a remote access. The error code should indicate a
recoverable error so that programs which perform both reads and writes can
recover if their logic permits. Figure~\ref{fig:client-write} shows the
structure of the typical write function in-the DODS client library.

\begin{figure}
\centerline{\psfig{figure=client-write.ps}}
\caption{Structure Chart for Functions that Write to Data Sets}
\label{fig:client-write}
\end{figure}

\subsection{Adding Local Access to a DODS Client Library}

In order to ensure that programs, once they have been re-linked with DODS
client libraries, can still access local data stores (e.g., files) is is
necessary to add software which can read those local data to the functions
in the re-implemented library. Typically in each function in the new library
the state information accessed by the identifier passed to the function is
used to determine if the call is to access local or remote data. In the
former case, the function must do exactly what the original implementation
of the API would have done to satisfy the function call. 

It is wasteful to completely recode the entire API just to achieve local
access, however, it is also not possible to simply link the user program with
both the DODS client library and the original library. It is not possible to
link with both the DODS client library and the original library because both
must {\em define the same external symbols}. Linking with both libraries
will produce link-time conflicts on most computers or result in an
incorrectly linked binary image.

In order to use the original implementation of the library, it is necessary to
rename all of its external symbols that will appear in user programs. For
example, if an API defines four functions (open, close, read and write) and
one global variable (errno), then each of those must be renamed to some new
symbol (e.g., orig\_open, orig\_close, \ldots). These source modules can then
be added to the set of object modules used to build the DODS client
library. Of course the DODS client library must also include the original
external symbol names; one approach is to recode each of the APIs external
symbols as a function which either calls the DODS-replacement or the original
function (now renamed so that the symbols do not conflict) depending on
whether the access is local or remote.

\section{Conclusion}

This paper describes the design of the data delivery components of the \DODS\@.
This design uses HTTPD and the CGI interface to provide access to remotely
stored data. User programs, which were never intentionally written with
access to remote data in mind, can be re-linked with DODS client libraries
and access data using HTTP\@. The DODS client libraries are direct
call-for-call replacements for existing APIs, thus any user program written
to make use of an API for which a DODS client library exists can be re-linked
and can access data via the DODS HTTP-base data servers. Because the data
servers and client libraries use a single protocol to provide access to the
data regardless of the data's storage format, user programs written to read
data using one API can actually access data through DODS data servers in
several APIs.

The current design of the DODS data delivery mechanism as described here and
in ``\DDA'' and ``\DAP'' is the result of our previous experience building a
similar system using Sun Microsystem's Remote Procedure Call (RPC)
technology. That system demonstrated the feasibility of re-linking user
programs with re-implemented API libraries and using those re-linked programs
to access remotely stored data. That design, however, did not use a single
transmission protocol for all servers and libraries. Thus it was not able to
provide interoperability between APIs. This design reflects our desire to
achieve that goal. 

The decision to abandon the RPC-based transfer technology in favor of
HTTP-based communications rests on the growing prominence of that technology,
the convenience that its supporting software provides and the likelihood
that a variety of interfaces can be constructed to access the HTTP-based data
servers. Because HTTP is receiving widespread use, there are many active
efforts to refine its weak points. By adopting HTTP, DODS stands to benefit
from these refinements. In addition, using text MIME documents to transmit
two of the three responses from the data servers makes interfacing our data
servers with many different programs, in addition to programs re-linked with
DODS client libraries, possible. For example, it is possible to use popular
WWW browsers to read information about the contents of DODS data servers.

\end{document}







