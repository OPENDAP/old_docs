%
% Documentation for the DAP. Intended to be like an RFC document.
%

\documentclass[justify]{dods-paper}
\usepackage{longtable}
\usepackage{acronym}
\usepackage{xspace}
\usepackage{gloss}
\usepackage{changebar}
\rcsInfo $Id$

% latex and HTML macros. Some latex commands become nops for HTML. 4/10/2001
% jhrg 
\T\newcommand{\Cpp}{\rm {\small C}\raise.5ex\hbox{\footnotesize++}\xspace}
\T\newcommand{\C}{\rm {\small C}\xspace}
\W\newcommand{\Cpp}{C++}
\W\newcommand{\C}{C\xspace}
\W\newcommand{\cdots}{...}
\W\newcommand{\ddots}{}
\W\newcommand{\vdots}{.}
\W\newcommand{\pm}{+/-}
\W\newcommand{\times}{*}
\W\newcommand{\uppercase}[1]{\textsc{#1}}
\T\newcommand{\qt}{\lit{\char127}}
\W\newcommand{\qt}{"}

\texorhtml{\def\rearrangedate#1/#2/#3#4{\ifcase#2\or January\or February\or
  March\or April\or May\or June\or July\or August\or September\or
  October\or November\or December\fi\ \ifx0#3\relax\else#3\fi#4, #1}
\def\rcsdocumentdate{\expandafter\rearrangedate\rcsInfoDate}}%
{\HlxEval{
(put 'rearrangedate       'hyperlatex 'hyperlatex-ts-rearrange-date)

(defun hyperlatex-ts-rearrange-date ()
  (let ((date-string (hyperlatex-evaluate-string 
                       (hyperlatex-parse-required-argument))))
    (let ((year-string (substring date-string 0 4))
          (month-string (substring date-string 5 7))
          (day-string (substring date-string 8 10))
          (month-list '("January" "February" "March" "April"
                        "May" "June" "July" "August" 
                        "September" "October" "November" "December")))
       (insert
         (concat (elt month-list (1- (string-to-number month-string)))
                 " " (int-to-string (string-to-number day-string))
                 ", " year-string)))))
}
\newcommand{\rcsdocumentdate}{\rearrangedate{\rcsInfoDate}}}

\newcommand{\dapversion}{Version 4.0\xspace}
\newcommand{\opendap}{OPeNDAP\xspace}
\newcommand{\DAP}{DAP\xspace}
\newcommand{\DODS}{DODS\xspace}
\newcommand{\NVODS}{NVODS\xspace}
\newcommand{\CE}{constraint expression\xspace}
\newcommand{\CEs}{constraint expressions\xspace}
\newcommand{\ErrorX}{ErrorX\xspace}
\newcommand{\CapX}{Server Capabilities Document\xspace}
\newcommand{\Blob}{Blob\xspace}
\newcommand{\DDX}{DDX\xspace}
\newcommand{\DAX}{DAX\xspace}
\newcommand{\DDS}{DDS\xspace}
\newcommand{\DAS}{DAS\xspace}
\newcommand{\URI}{URI\xspace}
\newcommand{\DataDDS}{DataDDS\xspace}

\newcommand{\type}[1]{\emph{#1}}
\newcommand{\Alias}{\type{Alias}\xspace}
\newcommand{\Array}{\type{Array}\xspace}
\newcommand{\Attribute}{\type{Attribute}\xspace}
\newcommand{\ProcAttribute}{\type{Processing Attribute}\xspace}
\newcommand{\Map}{\type{Map}\xspace}
\newcommand{\Target}{\type{Target}\xspace}
\newcommand{\FQN}{fully qualified name\xspace}
\newcommand{\Grid}{\type{Grid}\xspace}
\newcommand{\Structure}{\type{Structure}\xspace}
\newcommand{\Dataset}{\type{Dataset}\xspace}
\newcommand{\Sequence}{\type{Sequence}\xspace}
\newcommand{\Container}{\type{Container}\xspace}
\newcommand{\Bi}{\type{Binary Image}\xspace}
\newcommand{\String}{\type{String}\xspace}
\newcommand{\URL}{\type{URL}\xspace}
\newcommand{\Boolean}{\type{Boolean}\xspace}
\newcommand{\Byte}{\type{Byte}\xspace}
\newcommand{\Enum}{\type{Enumeration}\xspace}
\newcommand{\Time}{\type{Time}\xspace}
\newcommand{\Function}{\type{Function}\xspace}
\newcommand{\Description}{\type{Description}\xspace}
\newcommand{\Parameter}{\type{Parameter}\xspace}
\newcommand{\Constraint}{\type{Constraint}\xspace}
\newcommand{\NoAttributes}{\type{NoAttributes}\xspace}
\newcommand{\Project}{\type{Project}\xspace}
\newcommand{\Select}{\type{Select}\xspace}
\newcommand{\Hyperslab}{\type{Hyperslab}\xspace}

\newcommand{\Aliases}{\type{Aliases}\xspace}
\newcommand{\Arrays}{\type{Arrays}\xspace}
\newcommand{\Attributes}{\type{Attributes}\xspace}
\newcommand{\ProcAttributes}{\type{Processing Attributes}\xspace}
\newcommand{\Maps}{\type{Maps}\xspace}
\newcommand{\Targets}{\type{Targets}\xspace}
\newcommand{\FQNs}{fully qualified names\xspace}
\newcommand{\Grids}{\type{Grids}\xspace}
\newcommand{\Structures}{\type{Structures}\xspace}
\newcommand{\Sequences}{\type{Sequences}\xspace}
\newcommand{\Containers}{\type{Containers}\xspace}
\newcommand{\Bis}{\type{Binary Images}\xspace}
\newcommand{\Strings}{\type{Strings}\xspace}
\newcommand{\URLs}{\type{URLs}\xspace}
\newcommand{\Booleans}{\type{Booleans}\xspace}
\newcommand{\Bytes}{\type{Bytes}\xspace}
\newcommand{\Enums}{\type{Enumerations}\xspace}
\newcommand{\Times}{\type{Times}\xspace}
\newcommand{\CSVs}{comma-separated values\xspace}
\newcommand{\Functions}{\type{Functions}\xspace}
\newcommand{\Descriptions}{\type{Descriptions}\xspace}
\newcommand{\Parameters}{\type{Parameters}\xspace}
\newcommand{\Constraints}{\type{Constraints}\xspace}
\newcommand{\Projects}{\type{Projects}\xspace}
\newcommand{\Selects}{\type{Selects}\xspace}
\newcommand{\Hyperslabs}{\type{Hyperslabs}\xspace}

\newcommand{\DIR}{\textbf{Directory}\xspace}
\newcommand{\TEXT}{\textbf{Text}\xspace}
\newcommand{\HTML}{\textbf{HTML}\xspace}
\newcommand{\HELP}{\textbf{Help}\xspace}
\newcommand{\VER}{\textbf{Version}\xspace}
\newcommand{\INFO}{\textbf{Info}\xspace}

%%%%%%%%%%%%%% Web Services Paper
\newcommand{\GetDDX}{\textbf{GetDDX}\xspace}
\newcommand{\GetData}{\textbf{GetData}\xspace}
\newcommand{\GetBlobData}{\textbf{GetBlobData}\xspace}
\newcommand{\GetBlob}{\textbf{GetBlob}\xspace}
\newcommand{\GetDir}{\textbf{GetDir}\xspace}
\newcommand{\GetInfo}{\textbf{GetInfo}\xspace}

\newcommand{\FSs}{\type{Foundation Services}\xspace}
\newcommand{\FS}{\type{Foundation Service}\xspace}
\newcommand{\ODSN}{\type{OPeNDAP}\xspace}

\newcommand{\blobdataelement}{\texttt{BlobData} element\xspace}
\newcommand{\blobelement}{\texttt{Blob} element\xspace}



%\setcounter{secnumdepth}{4}
%\setcounter{tocdepth}{4}

\newcommand{\Tableref}[1]{Table~\ref{#1}}%
\newcommand{\Figureref}[1]{Figure~\ref{#1}}%
\W\begin{iftex}
\newcommand{\Sectionref}[2]{Section~\ref{#1}%
  \ifx#2)%
    \ on page~\pageref{#1}%
  \else% 
    \ (page~\pageref{#1})\xspace%
  \fi\ifx#2\space\ \else #2\fi}%
\W\end{iftex}
\W\newcommand{\Sectionref}[1]{Section~\ref{#1}}
\W\newcommand{\raggedright}{}


%% Conveniences for documenting XML
\newcommand{\tag}[1]{\emph{#1}}
\newcommand{\element}[1]{\link{\tag{#1}}{sec-xml-#1}}
\newcommand{\attribute}[1]{\emph{#1}}
\newcommand{\currentelement}{}
\newcommand{\ELEMENT}[1]{\renewcommand{\currentelement}{#1 element}%
  \subsubsection{#1}\label{sec-xml-#1}\indc{\currentelement}%
  \indc{catalog tag!#1}\indc{aggregation tag!#1}%
  \indc{XML!#1 element}}
\newcommand{\ATTRIBUTE}[1]{\item{\lit{#1}}\indc{\currentelement!#1}
    \indc{#1 attribute!of \currentelement}%
    \indc{XML!#1 attribute}}

% Conveniences for examples
\newcounter{exampleno}
\setcounter{exampleno}{0}
\newcounter{examplerefno}
\setcounter{examplerefno}{0}
\newcommand{\examplelabel}[1]{\refstepcounter{exampleno}\label{#1}%
  \medskip Example \theexampleno :\smallskip}
\newcommand{\exampleref}[1]{\texorhtml{Example~\ref{#1}%
    \refstepcounter{examplerefno}\label{exref\theexamplerefno}%
    % This is a test whether r@exref... is defined.  If not, skip
    % anything with the \pageref macro.
    \catcode`\@=11%
    \expandafter\ifx\csname r@exref\theexamplerefno\endcsname\relax\else%
    \expandafter\ifx\csname r@#1\endcsname\relax\else%
    \bgroup\count100=\pageref{exref\theexamplerefno}%
    \count101=\pageref{#1}\ifnum\count100=\count101\else~%
    on page~\pageref{#1}\fi\egroup\fi\fi\xspace}%
  {\link{Example \ref{#1}}{#1}}}%
\T\setlength{\vcodeindent}{10pt}
\texorhtml{
%% LaTeX version
\newenvironment{textoutput}[1]{\ifx #1\relax%
  \medskip Output:\vspace{-\medskipamount}\else%
  \medskip #1\vspace{-\medskipamount}\fi%
  \begin{list}{}{\setlength{\leftmargin}{\vcodeindent}}\begin{ttfamily}\item}%
 {\end{ttfamily}\end{list}} %
}{%% Hyperlatex version
\newenvironment{textoutput}[1]{\xml{blockquote}Output:\\ \\ \xml{tt}}%
  {\xml{/tt}\xml{/blockquote}}%
\newenvironment{minipage}[4]{}{}
\newenvironment{ttfamily}{\xml{blockquote}\xml{tt}}%
  {\xml{/tt}\xml{/blockquote}}}

\newcommand{\DAPOverviewTitle}{DAP Specification Overview}
\newcommand{\DAPOverview}{\xlink{\textbf{\textit{\DAPOverviewTitle}}}%
  {dap.html}\xspace}

% Old macros; new values
\newcommand{\DAPObjectsTitle}{The Data Access Protocol---DAP 2.0}
\newcommand{\DAPObjects}{\xlink{\textbf{\textit{\DAPObjectsTitle}}}%
  {http://www.opendap.org/pdf/ESE-RFC-004v0.06.pdf}\xspace}
  
\newcommand{\DAPHTTPTitle}{Using DAP 2.0 with HTTP}
\newcommand{\DAPHTTP}{\xlink{\textbf{\textit{\DAPHTTPTitle}}}%
  {daph.html}\xspace}

% New macros.
\newcommand{\DAPDataModelTitle}{DAP Data Model Specification}
\newcommand{\DAPDataModel}{\xlink{\textbf{\textit{\DAPDataModelTitle}}}%
  {dapo.html}\xspace}
  
\newcommand{\DAPWebTitle}{DAP Web Services Specification}
\newcommand{\DAPWeb}{\xlink{\textbf{\textit{\DAPWebTitle}}}%
  {daph.html}\xspace}
  

\newcommand{\DAPASCIITitle}{DAP Formatted Data Specification}
\newcommand{\DAPASCII}{\xlink{\textbf{\textit{\DAPASCIITitle}}}%
  {dapa.html}\xspace}
\newcommand{\DAPHTMLTitle}{DAP HTTP Query Specification}
\newcommand{\DAPHTML}{\xlink{\textbf{\textit{\DAPHTMLTitle}}}%
  {dapm.html}\xspace}

% It probably doesn't matter what we call the macro, but SOAP does not have
% to run over HTTP and I think that's going to be important for other groups.
% 10/27/03 jhrg
\newcommand{\DAPServicesTitle}{DAP HTTP Services Specification}
\newcommand{\DAPServices}{\xlink{\textbf{\textit{\DAPServicesTitle}}}%
  {daps.html}\xspace}

\newcommand{\thirtytwobitlimit}[1]{$4,294,967,296$ #1 ($2^{32}$)}
%%% Local Variables: 
%%% mode: latex
%%% TeX-master: t
%%% End: 


% Note: to get the glossary to work, run bibtex on the *.gls.aux file,
% then latex the file, then bibtex *.gls, then latex again. Also, make
% sure to set your BST and BIBINPUTS environment variables so that the
% BST and BIB files will be found.
% \makegloss

\title{\DAPObjectsTitle\\ DRAFT}
\htmltitle{\DAPObjectsTitle\ -- DRAFT}
\author{James Gallagher\thanks{The University of Rhode Island,
    jgallagher@gso.uri.edu}, Nathan Potter\thanks{Oregon State University, 
    ndp@coas.oregonstate.edu}, Tom Sgouros}
\date{Printed: \today \\ Revision: \rcsInfoRevision}
\htmladdress{James Gallagher <jgallagher@gso.uri.edu>, \rcsInfoDate, 
  Revision: \rcsInfoRevision}
\htmldirectory{html}
\htmlname{dapo}

\begin{document}

\maketitle
\T\tableofcontents

%%%%%%%%%%%%%%%%%%%%%%%% Introduction %%%%%%%%%%%%%%%%%%%%%%%%

\section{Introduction}

\note{This document contains the design for DAP4. It was written
  before the DAP2 RFC document and contains some information that
  conflicts with the information in that document. The most important
  information here is the DDX and related XML reponses which will be
  included in DAP4 and serve as replacements for the `curly-brace'
  reponses defined by DAP2.}

This document describes the \new{Data Model} of the
\opendap\footnote{Note that \opendap refers to a project, managed by
  OPeNDAP, Inc., a Rhode Island not-for-profit corporation, while \DAP
  refers to the Data Access Protocol which is a central component of
  the \opendap project.} Data Access Protocol (\DAP) version 4. The
data model is the framework---the set of data types and
representations---with which the \DAP represents the contents of a
data source. The data model also encompasses the objects in which
these data types are encoded: the replies with which an \opendap
server responds to requests for data.

This document contains several sections.  

\begin{description}
\item[\Sectionref{sec-variables}] defines what is meant by a data
  source and describes the kinds of variables a data source may
  contain.  These variables include the basic \new{atomic} types,
  and the more complex \new{constructor} types.
\item[\Sectionref{sec-Attributes}] All \DAP variables can have
  \Attributes to describe them further.  Representation of this
  information is described here.
\item[\Sectionref{sec-ce}] Variables in a data source may be
  sampled by means of a constraint expression, defined in this
  section. 
\item[\Sectionref{sec-interaction}] describes the interaction between
  a client requesting data and the server providing it.
\item[\Sectionref{sec-dataobject}] The variables in a data source are
  given their persistent representation in the data objects defined
  here.  This is the representation used to communicate between a
  server and client.
\end{description}

The \DAP is independent of the lower-level communication protocols used to
implement it, such as HTTP, FTP, GridFTP, {\it et cetera}. \DAP
implementations currently exist for several communication protocols. HTTP is
the most commonly-used implementation, and a separate document, \DAPHTTP, is
available from the \opendap project to specify its use. The \opendap project
does not issue specifications for implementations using other protocols. This
is left to the groups making the implementation.

A note about terminology. In this document, the words "client'' and "server''
are meant only to imply a program making a request via the \DAP and another
program making a reply via the \DAP, respectively. The two programs are
probably running on machines remote from one another, but this is not
essential. Like WWW clients and servers, there can be (and are) many
different varieties of \DAP clients and servers. The only thing that ties
them together is that the client making the request and the server fulfilling
it follow the strictures of this specification.

The \DAP specification describes the dialog between requesting clients and
responding servers, but it does not specify the implementation of that
dialog. So long as your program can hold up its end of the conversation,
there is no limit on how it is done. Specifically, this means that though the
\DAP specifies the persistent representation of its abstract data types (this
is the form taken by a message), it does not specify the data structures that
may implement these data types in a computer program. This also means that
the programs can use any means to transport the requests and responses,
although the \DAP protocol has been designed with protocols such as SOAP over
HTTP, GridFTP, {\it et c.}, in mind.

%  %  %  %  %  %  %  %  %  %  %  %  %  %  %  %  %  %  %  %  %  %  %  %  %
\subsection {Lexicographical Conventions}

The keywords ``MUST'', ``MUST NOT'', ``REQUIRED'', ``SHALL'', ``SHALL NOT'',
``SHOULD'', ``SHOULD NOT'', ``RECOMMENDED'', ``MAY'' and ``OPTIONAL'' in this
document are to be interpreted as described in RFC 2119.\cite{rfc2119}

The \DAP data model contains a model element called an \Attribute. XML
element tags also contain attributes. In order to avoid confusion in this
document when we are referring to OPeNDAP \Attributes we will capitalize the
first letter (``A''). When we are referring to XML attributes we will not
capitalize the first letter (``a''). In the event that this distinction is not
adequate we will endeavor to make the distinction clear, either from context
or from additional illuminating language.

\note{Suggestion: Drop the capitalized `Attribute' and say `XML attribute'
  where appropriate. Relying on capitalization confuses people.}

%%%%%%%%%%%%%%%%%%%%%%%%% Variables %%%%%%%%%%%%%%%%%%%%%%%%%
\section{Variables}
\label{sec-variables}

The \DAP characterizes a data source as a collection of variables. Each
variable consists of a name, a type, a value, and a collection of
\Attributes. \Attributes, in turn, are themselves composed of a name, a type,
and a value (\Sectionref{sec-Attributes}). The distinction between
information in a variable and in an \Attribute is somewhat
arbitrary.\footnote{This is especially true in the case of global
\Attributes; see \Sectionref{sec-Attributes} for information about global
attributes.} However, the intention is that \Attributes hold information that
aids in the interpretation of data held in a variable.\footnote{\Attributes
appear in many data storage systems such as netCDF\cite{netcdf},
HDF4\cite{NCSA:HDF4} and HDF5\cite{NCSA:HDF5}. They also appear under the
moniker `property' in Common Lisp\cite{steele:clisp}.} Variables, on the
other hand, hold the primary content of a data source.

Each variable in a data source MUST have a name, a type and some value or
values. Using just this information and armed with an understanding of the
definition of the \DAP data types, a program can read any or all of the
information from a data source. The names and types of a data source's
variables constitute its \new{syntactic metadata}.\cite{sgouros:DAP3}

The \DAP variables come in several different types. There are several
\new{atomic} types, the basic indivisible types representing integers,
floating point numbers and the like, and four \new{constructor} types (also
called \new{container} types)  which can be used to define new types
by combining instances of both the atomic and constructor types.

This section describes the abstractions that constitute the variable
type menagerie: the range of values and the kind of data each type can
represent.  For each abstract variable type, there is a more
concrete persistent representation, which is the information actually
communicated between a \DAP server and \DAP clients.  The persistent
representation consists of two parts: the declaration of the type and
the encoding of its value(s).  For a description of the persistent 
representation see \Sectionref{sec-dataobject}. To see how the types are to
be declared, see \Sectionref{sec-DDX}.  For the encoding of
these variable types (how they are to be packaged for
transmission) see \Sectionref{sec-blob}.

Each variable MAY have one or more \Attributes associated with it.  For
information about \Attributes, see \Sectionref{sec-Attributes}.

%  %  %  %  %  %  %  %  %  %  %  %  %  %  %  %  %  %  %  %  %  %  %  %  %
\subsection{Atomic variables}
\label{sec-atomic-variables}

As their name suggests, \new{atomic} data types are indivisible. There are no
\CE operators\footnote{\Sectionref{sec-ce}} that can be used to request part
of an instance of one of these types (\Sectionref{sec-ce}). Atomic variables
are used to store integers, enumerations, booleans and real numbers as well
as strings, URLs and times. There are four families of atomic types, with
each family containing one or more variation:

\begin{itemize}
\item Integer, Boolean and Enumeration types
\item Floating-point types
\item String types
\item Binary images
\end{itemize}

\subsubsection{Integer types}
\label{sec-atomic-int}

The integer types are summarized in Table~\ref{tab-ints}. Each of the types
is loosely based on the corresponding data type in ANSI \C \cite{ansi:c}.
However, the \DAP, unlike ANSI \C, does specify the bit-size of each of the
integer types. This is done so that when values are transfered between
machines they will be held in the same type of variable, at least within the
limits of the software that implements the \DAP.


\note{{\it move this to a more appropriate section, maybe its
own\ldots Maybe a section that combines the length spec and other
stuff that is used in several places. jhrg 11/2/03}

\begin{quote}
 When implementing the \DAP, it is important, of course, to match information
in a data source or read from a \DAP response to the {\it local} data type
which best fits those data. In some cases an exact match may not be
possible. For example Java lacks unsigned integer types\cite{Arnold:Java}.
Implementations faced with such limitations MUST ensure that clients will be
able to retrieve the full range of values from the data source. As a
practical consideration, this may be implemented by hiding the variable in
question or returning an error.

If a variable is automatically hidden ({\it i.e.}, the server analyzes the data
source and determines that a particular variable cannot be represented
correctly and automatically removes it from those variables made accessible
using the \DAP, this MUST be noted by adding a global \Attribute to the data
source indicating this has taken place. The note MUST include the name of the
variable(s) and the reason(s) for their exclusion. If a variable is removed
by a human, this \Attribute is OPTIONAL.
\end{quote}
} 

In their persistent representation in the \DAP, integer values MUST stored as
 \new{twos-compliment} \new{big-endian} numbers.\footnote{Big-endian is the
 default byte order for data transmissions, but see
 \Sectionref{sec-request-response} regarding negotiation of byte order.} See
 \Sectionref{sec-blob-atomic}.

\begin{table}[htbp]
\caption{The \ac{DAP} Integer Data types.}
\label{tab-ints}
\begin{center}
\begin{tabular}{lll}
\tblhd{name} & \tblhd{description} & \tblhd{range} \\ \hline 
\type{Byte} & 8-bit unsigned char & 0 to $2^{8}-1$ \\ \hline
\type{Int16} & 16-bit signed short integer & -$2^{15}$ to $2^{15}-1$ \\ \hline
\type{Uint16} & 16-bit unsigned short integer & 0 to $2^{16}-1$ \\ \hline
\type{Int32} & 32-bit signed integer & -$2^{31}$ to $2^{31}-1$ \\ \hline
\type{Uint32} & 32-bit unsigned integer & 0 to $2^{32}-1$ \\ \hline
\type{Int64} & 64-bit signed integer & -$2^{63}$ to $2^{63}-1$ \\ \hline
\type{Uint64} & 64-bit unsigned integer & 0 to $2^{64}-1$ \\ \hline
\end{tabular} 
\end{center}
\end{table}

\subsubsection{Booleans}
\label{sec-atomic-boolean}

Data which can take on only one of two values, true or false, may be
represented using the \Boolean data type. This type is used by 
data storage software such as HDF5\cite{NCSA:HDF5} and data communication
specifications such as ASN.1\cite{Dubuisson:ASN1}. 

See \Sectionref{sec-blob-atomic} for the description of how \Booleans
are encoded for transmission.

\begin{table}[hbtp]
\caption{The \DAP \Boolean type.}
\label{tab-boolean}
\begin{center}
\begin{tabular}{lp{3.0in}}
\tblhd{name} & \tblhd{description} \\ \hline
\Boolean & One of two possible values: either true or false. \\ \hline
\end{tabular} 
\end{center}
\end{table}

\subsubsection{Enumerations}

An \Enum is used to represent a set of discrete named values. The values
MUST be integers between $-2^{31}$ and $2^{31}-1$. No value may be used more
than once. The intent is that the size of the set will be small; an \Enum
should not be used to represent a set of thousands or millions of values,
although there's nothing in principle preventing such a use. To represent the
values, a signed 32-bit integer is used. An \Enum MUST include a symbolic
name for each integer value.

See \Sectionref{sec-blob-atomic} for the description of how \Enums
are encoded for transmission.

\begin{table}[hbtp]
\caption{The \DAP \Enum type.}
\label{tab-enum}
\begin{center}
\begin{tabular}{lp{3.0in}}
\tblhd{name} & \tblhd{description} \\ \hline
\Enum & a set of unique discrete integral values between $-2^{31}$ and
$2^{31}-1$, each enumerated and bound to symbol. \\ \hline
\end{tabular} 
\end{center}
\end{table}

\subsubsection{Floating point types}
\label{sec-atomic-float}

The floating point data types are summarized in Table~\ref{tab-fp}.
The two floating point data types use IEEE 754 \cite{IEEE:754} to
represent values.  The two types correspond to ANSI \C's
\texttt{float} and \texttt{double} data types.

In their persistent representation, floating point values MUST be stored by
default using big-endian notation.\footnote{Big-endian is the default byte
order for data transmissions, but see \Sectionref{sec-request-response}
regarding negotiation of byte order.} See \Sectionref{sec-blob-atomic}.

\begin{table}[hbtp]
\caption{The \ac{DAP} Floating Point Data types.}
\label{tab-fp}
\begin{center}
\begin{tabular}{lp{1.5in}p{2.1in}}
\tblhd{name} & \tblhd{description} & \tblhd{range} \\ \hline
\type{Float32} & IEEE 32-bit floating point \cite{IEEE:754}
     & {\raggedright $\pm 1.175494351 \times 10^{-38}$  to \\
       $\pm 3.402823466 \times 10^{38}$} \\ \hline
\type{Float64} & IEEE 64-bit floating point 
     & {\raggedright $\pm 2.2250738585072014 \times 10^{-308}$ to \\
       $\pm 1.7976931348623157 \times 10^{308}$} \\ \hline
\end{tabular} 
\end{center}
\end{table}

\subsubsection{String types}
\label{sec-atomic-string}

The string data types are summarized in Table~\ref{tab-string}.  There
are three.  The first is a simple string type corresponding to the ANSI
\C notion of a string: a series of Unicode (ISO 10646) characters.
The \DAP uses the UTF-8 encoding of Unicode characters.  

There is no limit to the size of a \String; the length is specified using a
\emph{Length Specifiaction} (See \Sectionref{sec-length-spec}). Unicode
characters can each be several bytes long, but note that UTF-8 encoding is
identical to US-ASCII encoding for character values up to 127 (hexadecimal
\lit{7f}). This means that strings that contain only characters from the
7-bit ASCII set are one byte per character, and use the standard ASCII
encoding. Characters from 128 to 255 (hex \lit{80} to \lit{ff}) are encoded
into two bytes in UTF-8 \cite{RFC2277}.

The \DAP also provides a \URL data type which is the same as \String except
that it MUST be limited to standard (7-bit) US-ASCII characters, , due to the
limitations of the syntax of Internet URLs\cite{rfc2396}, and has the
specific meaning of a pointer to some WWW resource. If the W3C definition of
a URL should change, the \DAP definition will change to conform to the W3C
definition. 

In \DAP applications this is usually used to refer to another data source, in
a manor reminiscent of a C pointer.]

The last string type the \DAP provides is the \Time data type which is
the same as \String except that it MUST have the specific syntax of an
ISO8601\cite{ISO:8601} date/time string. 

The ISO Date/Time standard provides a way to encode dates, both local and UTC
times and time ranges\cite{UWV:ISO8601}. The \Time data type is included in
the \DAP so that date and time information may be represented in a standard
fashion.

% Nathan, I think that this is general enough that is should appear here and
% not be moved down to a services paper. SOAP, HTTP/GET, GridFTP, ..., they
% all should provide Time variables for stuff encoded in `unique' ways.
% 10/27/03 jhrg

In general, most data sources will not use ISO8601 date-time
strings; servers SHOULD provide both the native representation of date-time
information and the ISO8601 representation. This will allow savvy clients to
exploit the native representation while more generic clients can access the
data source without the need to accommodate its quirks.

\Strings are individually sized.  This means that in constructor data
types containing multiple instances of some \String, such as \Sequences
and \Arrays, successive instances of that \String MAY be of different
sizes.

See \Sectionref{sec-blob-atomic} for other details of the persistent
representation of \Strings.

\begin{table}[hbtp]
\caption{The \DAP String Data types.}
\label{tab-string}
\begin{center}
\begin{tabular}{lp{3.0in}}
\tblhd{name} & \tblhd{description} \\ \hline
\String & a series of Unicode (UTF-8) characters. \\ \hline
\URL & a series of US-ASCII characters (the Internet doesn't support Unicode in
URLs), meant to represent an on-line resource somewhere, usually
another data source. \\ \hline
\Time & a series of Unicode (UTF-8) characters which contain a valid ISO 8601
date/time string.\cite{ISO:8601}
\end{tabular} 
\end{center}
\end{table}

\subsubsection{Binary images}
\label{sec-atomic-binary}

\Bis are\footnote{Paul suggests 'MUST be' in place of 'are.' I'm not sure
  jhrg 12/8/03.} uninterpreted, opaque, lumps of digital data. There is no
  limit to the size of a \Bi; the length MUST be specified using a \emph{Length
  Specifiaction} (See \Sectionref{sec-length-spec}). 

\Bis are meant as a way for a server to pass elaborate data types to a client
without having to encode them in the \DAP data model. For example, a digital
sound clip, say an MP3 file, could be represented as a one-dimensional \DAP
\Array of integer values. But if the server stores these as MP3 files, and
the client can play them as such, then it may not be efficient to convert
from MP3 to the \DAP \Array and then back again.\footnote{Note that, as with
the case of the MP3, the word "images'' does not necessarily refer to image
data, though obviously you can use a \Bi to transmit GIF data, for example.
The word only implies that the binary data within a \Bi is uninterpreted, and
MUST be preserved intact through any representation transformation.}

\Bis are individually sized.  This means that in constructor data
types containing multiple instances of some image, such as \Sequences
and \Arrays, successive instances of that \Bi MAY be of different
sizes.

See \Sectionref{sec-blob-atomic} for the description of how binary
images are encoded for transmission.

%  %  %  %  %  %  %  %  %  %  %  %  %  %  %  %  %  %  %  %  %  %  %  %  %
\subsection{Constructor variables} 
\label{sec-constructor-vars}

The \new{constructor} types are assembled from collections of other
variables. A constructor type MAY contain both atomic and constructor types.
There are no restrictions on the number of levels of nesting.

There are four constructor data types:

\begin{itemize}
\item \Array
\item \Structure
\item \Grid
\item \Sequence
\end{itemize}

\subsubsection{\Array}
\label{sec-constructor-array}

An \Array is a one-dimensional indexed data structure similar to that defined
by ANSI \C.  An \Array's member variable MAY be of any \DAP data
type. % with the exception of the \Enum type. 

Multidimensional \Arrays are defined as \Arrays of \Arrays. Multi-dimensional
\Arrays MUST be stored in \new{row-major} order (as is the case with ANSI \C).
The size of each \Array's dimensions MUST be given. There is no limit to the
size or number of an \Array's dimensions; the length MUST be specified using a
\emph{Length Specifiaction} (See \Sectionref{sec-length-spec}).

Each dimension of an \Array MAY also be named.

\Arrays of \Strings and binary images MAY contain elements of varying
lengths. However, multi-dimensional \Arrays MUST have rows of the same length
and columns of the same length. 

If you need a data structure which has varying row lengths or an
indeterminate number of rows, consider a \Sequence of \Sequences or a
\Sequence of \Arrays. Note that in the latter case, a \Sequence of \Arrays,
each instance of the array MUST be the same size.\footnote{The types of the
variables comprised bya Sequence MUST be the the same for each instance ({\it
i.e.} 'row') of that Sequence. For arrays, the 'shape' is part of the
type.Where the protocol demands that Sequence be used, an interface for
client application programs is free to make those look like arrays with
varying lengths.}

See \Sectionref{sec-constructor-sequence} for more about the possibilities
and the limitations.

\subsubsection{\Structure}
\label{sec-constructor-structure}

A \new{Structure} groups variables so that the collection can be
manipulated as a single item.  The \Structure's member variables MAY be
of any type, including other constructor types.  The order of items in
the \Structure is significant only in relation to the persistent
representation of that \Structure.

There is a special case of the \Structure data type, called
\Dataset.  This is the container that encompasses all the
variables provided in some data source.

\subsubsection{\Grid}
\label{sec-constructor-grid}

\note{Suggestion: Change this so that there is no distinction between target
  and map(s). make all arrays in a Grid have an equal footing. The Grid
  indicates that the elements are related but does not specify which is an
  independent or dependent variable.}

A \Grid is a special case of a \Structure, used to supply information
to aid in the interpretation of \Arrays.  A \Grid sets up an
association between a target \Array and a collection of map
\Arrays.  Each dimension of the target array MUST correspond to one or
more dimensions of the map arrays.  For example, a two-dimensional
target array could map to a collection of identically-sized
two-dimensional map arrays, or to an assortment of one-dimensional map
vectors.  A three dimensional array might map to a collection of one-,
two-, and three-dimensional map arrays.

A common use for this kind of data might be raw satellite data, where
measurements are frequently not on a regular latitude-longitude grid.
In the example shown, data values ($z_{m n}$) can be associated with
arbitrary latitude ($y_{m n}$) and longitude ($x_{m n}$) values, while
still retaining their gridded nature.  One can easily add a time
dimension to the collection, as well.  The result is that any element
in the target array has a corresponding latitude, longitude, and time
value.

%%% Should deal with Paul's degenerate case, where one map is used
%%% repeatedly.  ts.

\label{fig-grid-ex}
\begin{displaymath}
  target = \left[
    \begin{array}{cccc}
      z_{1 1} & z_{2 1} & \cdots & z_{m 1} \\
      z_{1 2} & z_{2 2} & \cdots & z_{m 2} \\
      z_{1 3} & z_{2 3} & \cdots & z_{m 3} \\
      \vdots  & \vdots  & \ddots & \vdots  \\
      z_{1 n} & z_{2 n} & \cdots & z_{m n} 
    \end{array} \right]
\end{displaymath}

\begin{displaymath}
  map_{longitude} = \left[
    \begin{array}{cccc}
      x_{1 1} & x_{2 1} & \cdots & x_{m 1} \\
      x_{1 2} & x_{2 2} & \cdots & x_{m 2} \\
      x_{1 3} & x_{2 3} & \cdots & x_{m 3} \\
      \vdots  & \vdots  & \ddots & \vdots  \\
      x_{1 n} & x_{2 n} & \cdots & x_{m n} 
    \end{array} \right]
  map_{latitude} = \left[
    \begin{array}{cccc}
      y_{1 1} & y_{2 1} & \cdots & y_{m 1} \\
      y_{1 2} & y_{2 2} & \cdots & y_{m 2} \\
      y_{1 3} & y_{2 3} & \cdots & y_{m 3} \\
      \vdots  & \vdots  & \ddots & \vdots  \\
      y_{1 n} & y_{2 n} & \cdots & y_{m n} 
    \end{array} \right]
  map_{time} = \left[
    \begin{array}{c}
      t_1 \\
      t_2 \\
      t_3 \\
      \vdots \\
      t_n
    \end{array}\right]
\end{displaymath}

This \Grid indicates that each $z$ value $z_{i j}$ corresponds to the values
of the $longitude$ and $latitude$ maps at $(i,j)$. It also indicates that
each row of $z_i$ (and $longitude_i$ and $latitude_i$) correspond to an
element of the column vector $time$. Such a data structure might be used to
hold satellite data before it has been processed into a picture. In that case
the $z$ target array might be a reflectance value from the satellite's
sensor, the $longitude$ and $latitude$ maps would provide the latitude and
longitude for each pixel in $z$ and the $time$ map would hold the time at
which each scan line was collected.

A \Grid MUST always contain a target array, MUST contain at least one map
array. The only other requirement is that each dimension of the target array
MUST correspond to one or more dimensions in the map arrays. The arrays in
the \Grid---target or map--- MAY be an \Alias to arrays somewhere else in the
dataset (\Sectionref{sec-Alias}). This can save transmission bandwidth by
avoiding the repetition of data when maps are common to more than one \Grid.

A special case of \Grid is an association of an $N$ dimensional \Array with
$N$ vectors (one-dimensional \new{map vectors}), each of which has the same
number of elements as the corresponding dimension of the \Array. Each vector
is used to map indexes of one of the \Array's dimensions to a set of values
which are normally non-integer ({\it e.g.}, floating point values).

Schematically, the special case of the \Grid is like the following:
\nopagebreak
\begin{displaymath}
  \begin{array}{cl}  & \left[
      \begin{array}{ccccc}
        x_1~~  & x_2~~    & x_3~~    & \cdots & x_m~~     \\
      \end{array} \right] \\[2mm] \left[
      \begin{array}{c}
        y_1 \\ y_2 \\ y_3 \\ \vdots \\ y_n
      \end{array} \right] & \left[
      \begin{array}{ccccc}
        z_{1 1} & z_{2 1} & z_{3 1} & \cdots & z_{m 1} \\
        z_{1 2} & z_{2 2} & z_{3 2} & \cdots & z_{m 2} \\
        z_{1 3} & z_{2 3} & z_{3 3} & \cdots & z_{m 3} \\
        \vdots  & \vdots  & \vdots  & \ddots & \vdots  \\
        z_{1 n} & z_{2 n} & z_{3 n} & \cdots & z_{m n} 
      \end{array} \right]
  \end{array}
\end{displaymath}

Each column of the $z$ \Array corresponds to an entry in the $x$ map
vector, and each row of $z$ corresponds to some $y$ value.  So, for
example, the data value at $z_{42,33}$ corresponds to the values
$x_{42}$ and $y_{33}$. 

The \Grid type was created to deal with geo-located data, with
irregular spacing of the rows and columns, which is useful when
converting to and from different map projections.  But the \Grid
structure is more generally useful.  For example, one of the map
vectors could be an \Array of $(x,y)$ pairs (stored in a \Structure),
and the other a series of time values, and the \Grid would become a
record of several synoptic time series.  The maps MUST be \Arrays, but
the \Arrays MAY be collections of any \DAP data type except \Sequence, 
\Grid, \Bi, or \Enum.\footnote{This restriction has been put in place to 
keep writing general clients tractable. If the set of data types in a 
\Grid's map \Arrays is allowed to be a \Sequence, for example, any general 
client would have to be capable of processing that data type in a response. 
Such a client would be very hard to build.}

\subsubsection{\Sequence}
\label{sec-constructor-sequence}

A \Sequence can best be described as an ordered collection of zero or more
\Structures. Each instance in the series consists of the same set of
variables, but contains different values.

\note{Not enough of a description. We need to state that the set of types in
  the declaration is used to define a set of instances, etc. And we need to
  clean up the example maybe replacing the $s_{1 1}$-type notation with
  something a little more verbose so the idea that each column is s different
  variable and they can be of different type is made explicit.}

The semantics of the \Sequence data type are very close to those of a table
in a relational database. You can think of the instances in a \Sequence as
rows in a traditional relational table. \opendap servers that serve data from
a DBMS like Oracle or mySQL use \Sequences to reflect the structure of their
data.
 
A \Sequence $S$ can be represented as:

\begin{displaymath}
  \begin{array}{cccc}
    s_{1 1} & s_{2 1} & \cdots & s_{n 1} \\
    s_{1 2} & s_{2 2} & \cdots & s_{n 2} \\
    \vdots & \vdots  & \ddots & \vdots \\
    s_{1 i} & s_{2 i} & \cdots & s_{n i} \\
    \vdots & \vdots  & \vdots & \vdots
  \end{array}
\end{displaymath}

Where each $s_1\cdots s_n$ entry represents a set of \DAP variables,
and the collection of such entries constitutes the \Sequence.  Every
entry of \Sequence $S$ MUST have the same number, order, and type of
variables.  If $s_{2 1}$ is a \type{Float64}, then all the $s_{2 i}$ will
also be \type{Float64} variables.  Similarly, in a \Sequence which contains 
an \Array or \Structure, each instance of the \Array or \Structure will be
the same size.  However, a \Sequence MAY contain a \Sequence and each
instance of the interior \Sequence MAY have a different number of
entries.  Also, unlike an \Array, a \Sequence has no explicit size.

\note{Though the semantics of \Sequences places limitations on the
  kinds of requests a client may make of a server, once the \Sequence has
  been retrieved, a client program may reference it in any way desired.
  The \DAP defines the persistent representation of data types, and the
  interaction between client and server (which includes what kinds of
  requests can be made for what kind of variables), but the \DAP does
  not specify the internal implementation of the data types for any
  client or server.}

%  %  %  %  %  %  %  %  %  %  %  %  %  %  %  %  %  %  %  %  %  %  %  %  %
\subsection{Names}
\label{sec-names}

A \DAP variable's name MUST contain ONLY UTF-8 characters. However, some
interfaces may require that any characters not part of US-ASCII be escaped
so that the names are represented in US-ASCII.\footnote{The HTTP GET
  interface from \opendap will require this if it's implemented for \DAP 4.}

\subsubsection{Constructor variable names}

The members of a constructor variable can be individually addressed in the
following fashion:

\begin{description}
\item[\Array] Individual items MUST be addressed with a subscripted
  expression. For an \Array named \lit{Temp}, the fourteenth member of the
  \Array is referenced as \lit{Temp[13]} (all indexes start at zero). A
  two-dimensional \Array is addressed with two subscripts, contained in
  separate brackets: \lit{SurfaceTemp[13][3]}. See \Sectionref{sec-ce}.
\item[\Structure] Members of the \Structure are addressed by appending
  the member name to the \Structure name, separated by a forward slash
  (\lit{/}).  If the \Structure \lit{Position} has a member named
  \lit{Height}, then it is addressed as \lit{Position/Height}.
  The members of a \Structure MUST have different names from one
  another. 
\item[\Grid] The arrays in a \Grid MAY be referenced in the same
  fashion as the members of a \Structure.  For a two-dimensional \Grid
  named \lit{Cloud}, with one-dimensional map vectors \lit{Latitude}
  and \lit{Longitude}, a member of a map vector is be addressed like
  this: \lit{Cloud/Latitude[36]}.  This refers to a single latitude
  value.  You can also request part of the target array:
  \lit{Cloud/Cloud[36][42]}, which will return a single data
  measurement.  The \Grid itself MAY be addressed like an \Array:
  \lit{Cloud[36][42]}, which will return the same value as
  \lit{Cloud/Cloud[36][42]}, but as a \Grid.  See
  \Sectionref{sec-ce-transform} for an explanation of how data types
  are transformed by constraints.
\item[\Sequence] A \Sequence member is addressed in the same fashion as
  a \Structure.  That is, a time called \lit{Releasedate} of a \Sequence
  named \lit{Balloons} is addressed as
  \lit{Balloons/Releasedate}.  But note that unlike a \Structure,
  this name references as many different values as there are entries
  in the \lit{Balloons} \Sequence.  A single entry or range of entries in a
  \Sequence MAY be addressed with a hyperslab operator like the items
  in an \Array.  The variables in a \Sequence MUST have different
  names from one another.
\end{description}

\subsubsection{Fully Qualified Names}
\label{sec-FQN}

Variables and \Attributes exist in the same name-space. This impacts the way
that datasets can be organized. A variable and an \Attribute may not have the
same name at the same lexical level of a dataset. This is primarily of
concern for constructor variables; if a constructor variable has a member
variable named \emph{time} then it MAY NOT also have an \Attribute named
\emph{time}. This rule allows the \FQN of each \Attribute and variable in a
\Dataset to be unique. Note that variables each hold an anonymous \Attribute
structure which is accessed using the variable's name.

\note{Glenn has suggested making each variable a structure where attributes
  are simply variables that carry an `attribute' tag.}

\paragraph{Variable Names}

The \FQN of a variable is composed of the ordered collection of variable
names, starting at the \Dataset level but not including the \Dataset name,
that can be followed to the terminal variable name. The names MUST be
separated by the slash (``/'') character, and the \FQN MUST begin with the
slash (``/'') character. Said another way, the \FQN of any variable in a
\Dataset is the concatenation of the variable's name, preceded by the forward
slash separated list of the names of the Constructor variables that contain
it. The first name MUST be a variable name at the \Dataset level preceded by
a forward slash. Thus, if a \Dataset named \lit{test} contains a structure
named \lit{sst} which contains a variable named \lit{foo}, the \FQN would be
\lit{/sst/foo}.

\paragraph{Attribute Names}

The \FQN of an \Attribute is composed of the ordered collection of variable
and \Attribute names, starting at the \Dataset level but not including the
\Dataset name, that can be followed to the terminal source \Attribute. The
names MUST be separated by the slash (``/'') character, and the \FQN MUST
begin with the slash (``/'') character. If a \FQN for an \Attribute terminates
with a regular variable name, then it will be interpreted to refer to the
collection of \Attributes associated with said variable. Thus, if a \Dataset
named \lit{test} contains a structure named \lit{sst} which contains a
variable named \lit{foo}, the \FQN of the \Attributes of foo would be
\lit{/sst/foo}. If \lit{foo} possessed an \Attribute named \lit{fruit} then
the \FQN for \lit{fruit} would be \lit{/sst/foo/fruit}.

%  %  %  %  %  %  %  %  %  %  %  %  %  %  %  %  %  %  %  %  %  %  %  %  %
\subsection{Variable Aliases}
\label{sec-Alias}

\note{Suggestion: Since what we're trying to do is set up a scheme whereby a
  variable with one name can be refereed to using another name, lets do just
  that. We can get around the problem of things crossing into and out of a
  Sequence by limiting the scheme to one that operates within a given lexical
  scope. At the same time we can add a mechanism to Grid that provides a way
  for it to use a Map defined somewhere else.}

A variable in a \DAP data source might contain no data of its own, but
simply be a pointer to some other variable in the set.  Such a
variable is called an \Alias.  An \Alias is useful for achieving
compatibility with other data sources, for conforming to metadata
requirements, and for conserving bandwidth in large data transmissions.

A variable \Alias MAY refer only to another variable. It MAY NOT refer to an
\Attribute. It MAY NOT refer to another variable \Alias.
 
An \Alias MUST be defined using the \FQN (see \Sectionref{sec-FQN}) of the
variable to which it refers. This means that if a \Dataset called \lit{test}
contains a \Structure called \lit{S} that contains both a variable called
\lit{V} and a variable \Alias directed at that variable called \lit{M}, then
\lit{M} MUST reference \lit{V} as \lit{/test/S/V}.

There are significant restrictions on the use of \Aliases in
conjunction with the \Sequence data-type. An \Alias member of a
\Sequence MAY refer to a member variable of the same \Sequence. An
\Alias member of a \Sequence MAY NOT refer to a variable outside of
the \Sequence. This may seem a little arbitrary at first, but consider
that every member variable of a \Sequence is multi-valued. It makes
little sense to allow an \Alias to refer to such a variable, unless
the \Alias exists in the same dimensional space.  This can only be
guaranteed if the \Alias is a member of the same \Sequence as the
variable to which it refers.

See the discussion of the \element{Alias} element in
\Sectionref{sec-xml-Alias} for specific information about the syntax
of an \Alias.

%%%%%%%%%%%%%%%%%%%%%%%%% Attributes %%%%%%%%%%%%%%%%%%%%%%%%%
\section{Attributes}
\label{sec-Attributes}

Each variable in a data source MAY have zero or more \Attributes
associated with it.  The entire dataset (see
\Sectionref{sec-constructor-structure}) MAY itself have \Attributes,
too.  These are called \new{global \Attributes}.  All \Attributes are
held within \new{Attribute structures}, even when there is only one
\Attribute associated with a variable.  Every variable acts as an
\Attribute structure.  This includes the \Dataset type, which contains
the global \Attributes. 

While the \DAP does not require any particular \Attributes, some may be
required by various \new{metadata conventions}. The \new{semantic metadata}
for a data source comprises the \Attributes associated with that data source
and its variables.

The data model for \Attributes is somewhat simpler than that for variables. 
An \Attribute MAY be a set of 1 or more values of the same 
\emph{atomic type}, OR it MAY be a \Structure that contains other 
\Attributes, but not individual values. In other words an \Attribute 
MAY be a list of values of the same atomic type, or it MAY be a container 
containing additional \Attributes. An \Attribute MAY NOT contain both values 
and other \Attributes.

An \Attribute's value MAY be any of the following atomic types:

\begin{itemize}
\item \lit{Boolean}
\item \lit{Byte}
\item \lit{Int16}
\item \lit{UInt16}
\item \lit{Int32}
\item \lit{UInt32}
\item \lit{Int64}
\item \lit{UInt64}
\item \lit{Enumeration}
\item \lit{Float32}
\item \lit{Float64}
\item \lit{String}
\item \lit{Time}
\item \lit{URL}
\end{itemize}
 
An \Attribute that contains other \Attributes MUST be of type \lit{structure}
or \lit{modifier}.

There are two special types of \Attributes: \lit{modifier} and \lit{alias}.
\Attributes of type \lit{alias} are discussed in
\Sectionref{sec-attribute-aliases}. \Attributes of type \lit{modifier} are
discussed in \Sectionref{sec-attr-processing}.

There are examples of \Attribute definitions in the description of the
\Attribute element in \Sectionref{sec-xml-Attribute}.

%  %  %  %  %  %  %  %  %  %  %  %  %  %  %  %  %  %  %  %  %  %  %  %  %
\subsection{Attribute Aliases}
\label{sec-attribute-aliases}
 
 A special type of \Attribute is \new{alias}. In a manner comparable to a
variable \Alias (see \Sectionref{sec-Alias}), a variable's \Attributes MAY
also be aliased, with an \Attribute of one name referring to a different
\Attribute (possibly with a different name). An \Attribute of one variable
MAY be an alias that refers to an \Attribute of another variable. \Attribute
aliases MAY only refer to \Attributes. They MAY NOT refer to variables or
other \Attribute aliases.

An \Attribute alias MUST be defined using the \FQN of
the \Attribute to which it refers. See \Sectionref{sec-FQN}

See the discussion of the \Attribute \lit{alias} type in
\Sectionref{sec-xml-Attribute-Alias} for specific information about the
syntax of an \Attribute \lit{alias}.

%  %  %  %  %  %  %  %  %  %  %  %  %  %  %  %  %  %  %  %  %  %  %  %  %
\subsection{Processing Attributes}
\label{sec-attr-processing}

%%% What if modifications are by hand?  not automated?  Stuff like
%%% that happens, you know.  ts.

The processing \Attributes are a special set of \Attributes used to record
modifications to a dataset's \Attributes or its data. After a dataset is
released, other people may copy the data and make it available in modified
form, or with secondary data products added, or with \Attribute information
that may make the data conform to some data standard. However useful these
modifications might be, it is essential that users of some dataset be able to
determine which parts of that set are original, and which have been added
subsequent to the original publication. The processing \Attributes exist to
provide a way to create an ``audit trail'' that will permit users to
determine how a dataset has been modified.

The key piece of information is a global \Attribute of type \lit{modifier},
which is essentially the signature of the organization that has performed the
modification. A \lit{modifier} \Attribute contains:

\begin{itemize}
\item A \DAP URL for the original source of data.
\item The name or description of the organization that is providing
  the modified data.
\item The URL of the service which has introduced the changes.
\end{itemize}

%%% I lightened up this section to reduce duplicated information.  ts.
The details of the \lit{modifier} attribute's syntax is described in
\Sectionref{sec-xml-proc-attr}. 

\Attributes of type \lit{modifier} MAY only appear at the top level of a data
set. They MAY not be used as members of other \Attributes or of other
variable \Attribute structures. \exampleref{ex-modifier-attr} shows an example 
of a modifier \Attribute.



%%%%%%%%%%%%%%%%%%% Constraint Expressions %%%%%%%%%%%%%%%%%%%
\section{Constraint Expressions}
\label{sec-ce}

A \new{\CE} provides a way for \DAP client programs to request certain
variables, or parts of certain variables, from a dataset. Many datasets are
large and many variables in datasets are also large. Often clients are
interested in only a small number of values from the entire dataset.
Constraint expressions provide a way for clients to tell a server which
variables, and in many cases, which parts of those variables, they would
like. In addition, the \CE can be used to request that the server to omit the
\Attribute information from a \DDX.\footnote{\Sectionref{sec-DDX}.}

This section presents the subsampling abilities that MUST be provided by a
\DAP server. It does so without binding these capabilities to any
particular syntax; see \Sectionref{sec-xml-ce} for the XML representation of
a \CE. Some transport protocols may choose to implement additional
syntaxes but MUST implement the syntax described in Section~\ref{sec-xml-ce}.

Note that an empty \CE implies that the entire data source is to be accessed.

%  %  %  %  %  %  %  %  %  %  %  %  %  %  %  %  %  %  %  %  %  %  %  %  %
\subsection{Limiting data by type and by value}
\label{sec-ce-clauses}

A \CE provides two different methods to access the information held by a data
source. The \CE can be used to limit data using the names of variables or by
scanning variables and returning only those values that satisfy certain
relational expressions. The former are referred to as \new{projections} while
the latter are called \new{selections}.

A \CE MAY combine both projection and selection constraints. For example, a
projection might specify that temperatures held in a \Sequence are to be
returned, and a selection would specify that only \Sequence entries with
dates later than 1999 are to be examined. The result returned from a request
like this would be a \Sequence of temperature measurements taken after 1999.

\Sectionref{sec-ce-projections} describes the projection operations which any
\DAP implementation MUST support and, likewise,
\Sectionref{sec-ce-selections} describes the required selection operations.

To provide implementors with a means to extend the \CE mechanism, it is
possible to add functions to a server and to call those as part of the \CE.
Functions are described in \Sectionref{sec-ce-functions}.

\subsubsection{Projections}
\label{sec-ce-projections}

% Describe the types of projection operations
% Choosing parts based on field names
% Choosing parts based on hyperslabs

The \new{projection clause} of a \CE provides a way to choose parts of a data
set based on the shape of the \Dataset and the variables that it comprises.
There are two types of projection operations. First, it is possible to choose
individual fields of the constructor data types. This is called \new{field
projection} and applies to the \Structure, \Grid and \Sequence data types in
the following ways:
\begin{description}
  \item[\Structure] A field projection which chooses one or more fields from
  a \Structure variable causes a \DAP server to return only those named
  fields from the \Structure. Note that the \Dataset itself is a \Structure.

  \item[\Grid] A field projection which chooses one or more fields from
  a \Grid variable causes a \DAP server to return only those named
  fields from the \Grid. It is likely that the variable returned will no
  longer meet the criteria for a correctly formed \Grid data type, so the
  variable may be returned as a \Structure instead (see 
  \Sectionref{sec-ce-transform}).

  \item[\Sequence]  A field projection which chooses one or more fields from
  a \Sequence variable causes a \DAP server to return only those named
  fields from the \Sequence. For the \Sequence type, this means returning the
  $N$ instances but limiting the fields those given in the field
  projection. For example, suppose the \Sequence $S$ has $p$ fields:
  
 \begin{displaymath}
 \begin{array}{ccc}
   s_{1 1} & s_{2 1} & s_{3 1} \\
   s_{1 2} & s_{2 2} & s_{3 2} \\
    \vdots & \vdots  & \vdots  \\
   s_{1 i} & s_{2 i} & s_{3 i} \\
    \vdots & \vdots  & \vdots
 \end{array}
 \end{displaymath} 
 
If a field projection is used to choose only the second field, the result of
accessing $S$ would be:
 \begin{displaymath}
 \begin{array}{c}
   s_{2 1}\\
   s_{2 2}\\
   s_{2 3}\\
   s_{2 4}\\
   \vdots \\
 \end{array}
 \end{displaymath}
\end{description}

The second type of projection is a \new{hyperslab}. A hyperslab is used to
limit returned data to those elements that fall within a range of index
values, and MAY also specify that the range be subsampled using a
\new{stride}. By including a hyperslab projection for one or more dimensions
of a variable it is implied that any unnamed dimensions are to be returned in
their entirety.\footnote{For some interfaces, it may be necessary to place more
restrictions on hyperslab projections.} A hyperslab is applied to the \Array,
\Grid and \Sequence types in the following way:
\begin{description}
  \item[\Array] \Array dimensions are numbered $0, \ldots, N-1$ for an \Array
  of rank $N$. Within each dimension of size $M$, elements are numbered $0,
  \ldots, M-1$. A hyperslab projection for dimension $n, 0 \leq n < N$ MUST
  include the starting index $i_{n_{s}}$ and ending index $i_{n_{e}}$ such
  that $i_{n_{s}} \leq i_{n_{e}} \forall \{ 0 \leq i_{n} < M \}$. If a stride
  is included in the hyperslab and is greater than $i_{n_{e}} - i_{n_{s}}$
  then the hyperslab is equivalent to one where $i_{n_{s}} = i_{n_{e}}$ and
  the original value of $i_{n_{e}}$ is discarded.

\item[\Grid] \Grid dimensions are numbered as are \Array dimensions; \Grid
  dimensions MAY have hyperslab projections applied to them in a manner
  similar to \Arrays except that a hyperslab applied to a \Grid is
  applied to not only the target array, but also all the corresponding map
  arrays. For example, given the \Grid:
\begin{displaymath}
  target = 
   \left[
    \begin{array}{cccc}
      1 & 2 & 3 & 4 \\
      5 & 6 & 7 & 8 \\
      9 & 10 & 11 & 12 \\
      13 & 14 & 15 & 16
    \end{array} \right]
   map_{1} = 
   \left[
    \begin{array}{cccc}
      -53 & -52 & -51 & -50 \\
      -52 & -51 & -50 & -49 \\
      -51 & -50 & -49 & -48 \\
      -50 & -49 & -48 & -47 
    \end{array} \right]
   map_{2} = 
   \left[
    \begin{array}{cccc}
      26 & 25 & 24 & 23 \\
      25 & 24 & 23 & 22 \\
      24 & 23 & 22 & 21 \\
      23 & 22 & 21 & 20
    \end{array} \right]
\end{displaymath}
A hyperslab projection which chose row indexes 1 and 2 and column indexes 1
and 2 would cause a server to return:
\begin{displaymath}
  target = 
   \left[
    \begin{array}{cc}
      6 & 7\\
      10 & 11\\
    \end{array} \right]
   map_{1} = 
   \left[
    \begin{array}{cc}
      -51 & -50\\
      -50 & -49\\
    \end{array} \right]
   map_{2} = 
   \left[
    \begin{array}{cc}
      24 & 23\\
      23 & 22\\
    \end{array} \right]
\end{displaymath}
for the \Grid.

Note that a field and hyperslab projection can be combined for a \Grid to
choose only part of one of the fields, say just part of the the target
\Array. In this case, the hyperslab applied to one field of the \Grid is
equivalent to a hyperslab applied to an \Array. Effectively, the field
projection yields an \Array and the hyperslab is then applied to that \Array.

\item[\Sequence] A hyperslab can be applied to a \Sequence. A \Sequence with
  $M$ instances can have a hyperslab projection applied to it as if it is an
  \Array of rank 1. Since the \Sequence type does not contain an explicit
  dimension size, the size $M$ is not known until the entire \Sequence is
  accessed.\footnote{For many \Sequence variables, it may never be the case
  that the entire \Sequence is accessed since it may contain millons of
  instances.} A hyperslab projection can be used to ask for the first $m$
  elements, the next $m$ elements, etc., which may be very useful for clients
  which need to know the sizes of varaibles before accessing them. A hyperslab
  projection for a \Sequence $(i_{s}, i_{e})$ will return $m$ instances of
  the \Sequence such that $m = \lfloor i_{e}, M-1 \rfloor - i_{s}$ depending
  on whether $i_{e}$ is an index greater than the number of instances in the
  \Sequence.
\end{description}

It is possible to ask for values from several variables in a single \CE by
including several projections in the \CE. Also note that an empty \CE, by
convention, projects all of every variable in a data source.

\subsubsection{Selections}
\label{sec-ce-selections}

% Describe the types of selection operations
% Choosing parts based on relational expresssions
% binding selections to a selectable data type

A \new{selection} provides a way to limit data accessed based on the value(s)
of those data. In many ways selections are similar to WHERE claues in
SQL\cite{date:DBMS}. A selection is composed of one or more relational
sub-expressions. Each sub-expression MUST be bound to a variable listed in a
projection clause. When several sub-expressions constitute a selection, the
boolean value of the selection is the logical \lit{AND} of each of the
boolean values of each sub-expression. Note that there is no way to perform a
logical \lit{OR} operation on the sub-expressions but there is a way, within
a sub-expression, to test several values and return \lit{true} if any satisfy
the releation.

Each of the relational sub-expressions ({\it i.e.}, relations) is composed of
two operands and a relational operator. Each operand MUST be an atomic data
type; it MAY be a \FQN from the data source or a constant. In some cases
there are further limitations on the allowed types based on the relational
operator. \Tableref{tab-ce-relations} lists the operators, their meaning and
the data types on which they may be applied.

\begin{table}[htbp]
  \begin{center}
    \caption{\DAP Selection Relational Operators}
    \label{tab-ce-relations}
%    \begin{tabular}{p{.75in}p{1.5in}p{2.0in}} \\
    \begin{tabular}{cp{1.5in}p{2.0in}} \\
      \tblhd{Operator} & \tblhd{Meaning} & \tblhd{Types} \\ \hline

      $\verb+<+$ & Less than 
      & Byte, Int16, Int32, Int64, UInt16, UInt32, UInt64, Enumeration, Float32,
      Float64, Time \\

      $\verb+<=+$ & Less than or equal to 
      & Byte, Int16, Int32, Int64, UInt16, UInt32, UInt64, Enumeration, Float32,
      Float64, Time \\

      $\verb+>+$ & Greater than 
      & Byte, Int16, Int32, Int64, UInt16, UInt32, UInt64, Enumeration, Float32,
      Float64, Time \\

      $\verb+>=+$ & Greater than or equal to 
      & Byte, Int16, Int32, Int64, UInt16, UInt32, UInt64, Enumeration, Float32,
      Float64, Time \\

      $\verb+=+$ & Equal 
      & Byte, Int16, Int32, Int64, UInt16, UInt32, UInt64, Boolean, Enumeration,
      Float32, Float64, String, Url, Time \\

      $\verb+!=+$ & Not equal 
      & Byte, Int16, Int32, Int64, UInt16, UInt32, UInt64, Boolean, Enumeration,
      Float32, Float64, String, Url, Time \\

      $\verb+=~+$ 
      & Regular expression match & String, Url, Time \\ 

    \end{tabular}
  \end{center}
\end{table}

The operands in a relation MAY be either single or multi-valued. If an
operand has more than one value, each value is used in succession when
evaluating the relation. For example, suppose there is a relation: 

\begin{center}
$site = \{``Diamond\_St'', ``Blacktail\_Loop''\}$
\end{center}

Then that relation is true for any instance where $site$ is either
``Diamond\_St'' \lit{OR} ``Blacktail\_Loop''.

Selections MAY be applied to \Sequence and \Grid data types in
the following ways:

\begin{description}
  \item[\Sequence] Logically, the relations in a selction bound to a
    \Sequence are evaluated once for every instance ({\it i.e.}, row) of the
    \Sequence; the result of applying the selection to the \Sequence is a
    \Sequence where all of the instances satisfy all of the relations.

    A \Sequence $S$ with three fields and four instances such as:
    \begin{displaymath}
      \begin{array}{ccc}
        index & temperature & site \\
        10 & 17.2 & Diamond\_St \\
        11 & 15.1 & Blacktail\_Loop \\
        12 & 15.3 & Platium\_St \\
        13 & 15.1 & Kodiak\_Trail \\
      \end{array}
    \end{displaymath}
    A selection such as $index \verb+>=+ 11$ would choose the last three instances:
    \begin{displaymath}
      \begin{array}{ccc}
        index & temperature & site \\
        11 & 15.1 & Blacktail\_Loop \\
        12 & 15.3 & Platium\_St \\
        13 & 15.1 & Kodiak\_Trail \\
      \end{array}
    \end{displaymath}
    The selection $site \verb+=~ ".*\_St"+$ would choose two instances:
    \begin{displaymath}
      \begin{array}{ccc}
        index & temperature & site \\
        10 & 17.2 & Diamond\_St \\
        12 & 15.3 & Platium\_St \\
      \end{array}
    \end{displaymath}
    And a selection with the two sub-expressions $index\verb+<=+11$, 
    $site\verb+=~".*\_St"+$ would return only one instance:
    \begin{displaymath}
      \begin{array}{ccc}
        index & temperature & site \\
        10 & 17.2 & Diamond\_St \\
      \end{array}
    \end{displaymath}

\item[\Grid] When selections are applied to \Grids with multi-dimensional map
arrays, the returned data MUST be the smallest rectangular (contiguous)
subset of the \Grid that contains all the data that satisfies the constraint.
For example, suppose there is a \Grid which has a target array of rank two
and two map arrays, each of which also have rank two, such as:
\begin{displaymath}
  target = 
   \left[
    \begin{array}{cccc}
      1 & 2 & 3 & 4 \\
      5 & 6 & 7 & 8 \\
      9 & 10 & 11 & 12 \\
      13 & 14 & 15 & 16 \\
    \end{array} \right]
   lat = 
   \left[
    \begin{array}{cccc}
      -53 & -52 & -51 & -50 \\
      -52 & -51 & -50 & -49 \\
      -51 & -50 & -49 & -48 \\
      -50 & -49 & -48 & -47 \\
    \end{array} \right]
   lon = 
   \left[
    \begin{array}{cccc}
      26 & 25 & 24 & 23 \\
      25 & 24 & 23 & 22 \\
      24 & 23 & 22 & 21 \\
      23 & 22 & 21 & 20 \\
    \end{array} \right]
\end{displaymath}
Suppose the \Grid is constrained by a selection clause that limits returned
values to those where \lit{lat} is greater than 24.5 and \lit{lon} is less
than -50.5. The \Grid returned by such a constraint would be:
\begin{displaymath}
   \left[
    \begin{array}{cc}
      1 & 2  \\
      5 & 6  \\
    \end{array} \right]
   \left[
    \begin{array}{cc}
      -53 & -52  \\
      -52 & -51  \\
    \end{array} \right]
   \left[
    \begin{array}{cc}
      26 & 25  \\
      25 & 24  \\
    \end{array} \right]
\end{displaymath}

Notice that one point is returned that does not satisfy the \element{Select}
provision. \Grids MUST be rectangular.
\end{description}

\subsubsection{Server Functions}
\label{sec-ce-functions}

A \CE MAY also use functions executed by the server. These can appear in a
selection or in a projection, although there are restrictions about the data
types functions can return. 

% Functions in projections
% Functions in selections
% Functions and errors

A function which appears in the projection clause MAY return any of the \DAP
data types. In this case the return value of the function is treated as if it
is a variable present in the top level of the \Dataset.

A function which appears in the selection clause MAY return any atomic type
if it is used in one of the relational sub-expressions. If a function in the
selection clause is used as the entire sub-expression, it MUST return a
\Boolean value.

% I think this should stay here; we need to provide for an unambiguous way to
% get error information back to clients. I think that has to come from this
% level and not vary per transport. The transports are free to implement
% error returns however they can. 10/27/03 jhrg
When functions encounter an error, a \DAP server MUST signal that condition
by returning an error response. A server MAY NOT return a partial response;
any error encountered while evaluating the \CE MUST result in a response that
contains an unambiguous error message.

%  %  %  %  %  %  %  %  %  %  %  %  %  %  %  %  %  %  %  %  %  %  %  %  %
\subsection{Data Type Transformation Through Constraints}
\label{sec-ce-transform}

When a constraint expression has a projection clause that identifies a piece
of a constructor variable, such as one field of a \Structure or just the
array part of a \Grid, the \new{lexical scoping} of the variable is not
abandoned. This is important for avoiding name collisions. For example, if
you request only one item from a \Structure, you get a \Structure returned
that has only one member variable.

Here is the behavior for each data type:

\begin{description}
\item[\Array] An \Array MUST be returned as an \Array of the same rank as
  the source \Array. A hyperslab request that effectively eliminates a
  dimension by reducing its size to $1$ does \emph{not} reduce the rank of
  the returned \Array. For example, suppose a 10 by 10 element \Array was
  subsampled to a 1 by 2 \Array. The returned variable would still be
  described as a two dimensional \Array.
  
\item[\Structure] A \Structure MUST bereturned as a \Structure. If the
  projection clause of a \CE selects only one member of the \Structure, then
  a one-member \Structure MUST be returned. If more than one member of the
  \Structure are named in the projection clause, they MUST be returned in the
  same \Structure.
  
\item[\Grid] A \Grid modified with a hyperslab operator MUST return
  another \Grid, following the same rules as an \Array.  But if the
  projection clause specifies the elements of the \Grid independently
  of one another---the target array, or one of the maps---then a
  \Structure is returned containing only the specified variables.  A
  two-dimensional \Grid named \lit{Cloud} will return a \Grid in
  response to a request like this: \lit{Cloud[1:10][20:30]}.  But a
  request for the target array
  alone---\lit{Cloud/Cloud[1:10][20:30]}--- will return a \Structure
  called \lit{Cloud} containing an \Array called \lit{Cloud}.  The map
  arrays will not be returned.

  A \Grid modified with a selection MUST remain a \Grid. The return value of
  such a constraint MUST be the smallest rectangular \Grid that contains all
  the data points that satisfy the given constraint. Further, the rank of the
  \Grid MUST remain the same. A four-dimensional \Grid, when sampled with a
  selection clause, MUST return a four-dimensional \Grid, even if some
  of the dimensions are of length one.
  
\item[\Sequence] A \Sequence MUST be returned as a \Sequence, even
  if a selection clause selects only a single entry or no entry at
  all.  If a projection clause identifies more than one member of the
  \Sequence, they MUST be returned in the same \Sequence.

\end{description}



\note{ \\[2mm]
What about allowing selection based on \Attribute content? 
Say, all variables with origin=``helena''? All variables with a 
\Attribute named ``units''? All variables with an \Attribute 
named ``units'' whose value is ``cm''?\\[2mm]
This might be difficult, but well worth the effort.
}


%%%%%%%%%%%%%%%%%% Client/Server Interaction %%%%%%%%%%%%%%%%%
\section{Client/Server Interaction}
\label{sec-interaction}

The \DAP is based on the request/response paradigm for client-server
interaction.\footnote{Should we add a reference to Fielding, ``REST''
  here?}  This section provides an overview of the requests and
responses ({\it i.e.}, the messages) which \DAP-compliant servers MUST
support. These messages are used to request information about the
capabilities of a server, about data made accessible by that server,
as well as requesting data values themselves.  The latter---the
messages that access a particular data source---use the previously
described data model. The other messages use simpler documents with
contents that do not need a formal abstract definition.

The table below provides a description of the \DAP messages. The precise
details of the requests and responses are described in
\Sectionref{sec-dataobject} and \Sectionref{sec-xml-ce}. The mechanism used to
communicate those requests and responses to/from a client and server depend
on the transport protocol in use. (Consult \DAPHTTP for an HTTP
implementation.) But whatever the protocol, a server MUST be able to provide
the responses outlined in \Tableref{tab-req-and-resp}.

\begin{table}[htbp]
  \begin{center}
    \caption{\DAP Requests and Responses}
    \label{tab-req-and-resp}
    \begin{tabular}{p{2.5in}p{2.5in}} \\
      \tblhd{Requests} & \tblhd{Response} \\ \hline
      Data, MAY include a \CE (\Sectionref{sec-ce})
      & \DDX (\Sectionref{sec-dataobject})\\ \hline

      Data (binary) 
      & \Blob (\Sectionref{sec-blob})\\ \hline

      Characteristics of server 
      & Capabilities document (\Sectionref{sec-ce-functions}) \\ \hline

      & \ErrorX object (\Sectionref{sec-errorx}) \\ \hline
    \end{tabular}
  \end{center}
\end{table}

For a client to get data from a server takes at minimum two exchanges, first
to request the \DDX, and second to get the \Blob. The \DAP is at root a
stateless protocol. The server is not required to remember anything from one
request to another. A client has the responsibility to make the two requests
correspond. 

If a \CE is included in a request for a \DDX, the returned \DDX MUST contain
a \Blob reference that refers to the constrained data. 

In addition to these data objects, a \DAP server MAY provide additional
``services'' which clients may find useful.  The HTTP implementation
of the \DAP, for example, provides HTML-formatted representations of
a dataset's structure and a way to get data represented in CSV-style ASCII
tables. These additional services are not described in this document; they
are considered specific to different transport protocols and are
described by the specifications for those particular protocols (such as the
\DAPHTTP document).

%  %  %  %  %  %  %  %  %  %  %  %  %  %  %  %  %  %  %  %  %  %  %  %  %
\subsection{Request and Response Information}
\label{sec-request-response}

The following information MAY be included in ANY request-response interaction
between a \DAP client-server pair. Because different transport protocols
often provide ways to encode this type of information ({\it e.g.}, HTTP
provides a way to encode the date of the response), a concrete syntax for
representing this information is not presented here; that syntax MUST be
included in the transport-specific \DAP specification.

\begin{description}
\item[Compression] \DAP clients that can process compressed responses MUST be
  provided a way to indicate this to a server. A server MAY compress a
  response ONLY if a client has indicated that it can process the compressed
  response. A server is NEVER under an obligation to compress a response.

  Support for particular compression algorithms is specific to the transport
  protocol. 

\item[User agent] \DAP clients MAY provide information about the client
  software to the server. \DAP servers MAY log this information. Note that
  when requested, \DAP servers MUST provide their version and software
  information to clients and do so using a special response.

\item[Date] Servers MUST provide a date stamp which conforms to RFC 1033 in
  their responses, and they SHOULD also provide the last
  modification date, also conformant to RFC 1033, of the data requested.

\item[Byte Order] \DAP clients MUST be able to indicate their native
  byte order to a server. A server MAY choose to use little-endian byte order
  with a client that indicates that is its native byte order. By default the
  \DAP uses big-endian byte order for all data exchanges. All \DAP clients
  MUST be able to understand responses in big-endian byte order.

\item[Floating-point Format] Like the byte order, a client and server that
  agree on a floating-point format different than the IEEE 754 standard used
  by the \DAP should be able to communicate that fact to each other, and skip
  converting data only to convert it back. A client MUST be able to indicate
  its preferred floating-point format; a server MUST be able to respond with
  floating-point value encoded using IEEE 754 and DOES NOT have to honor a
  client's request for data to be delivered in a different representation.

\end{description}

%%%%%%%%%%%%%%%%%%%%%%%%%% Responses %%%%%%%%%%%%%%%%%%%%%%%%%
\section{Responses}
\label{sec-dataobject}

In order to pass data from server to requester, the data needs to be
transformed into a representation both can understand. In the same way that
the idea behind a book needs to be written down and printed in order to
transfer the idea from the writer to the reader, a dataset---an abstract set
of numbers and the relationships between them---needs to be transformed into
a more tangible form in order to be communicated.

For the \DAP, this form is called the \new{persistent
representation}.\footnote{Sometimes the persistent representation is called
the \new{network representation}.} This is to contrast it with the
representations used within the memory of some program that can process this
data, which of course only persist as long as the program is running. The
persistent representation may also be contrasted with the file format used to
store some data on a disk somewhere. File formats, though persistent, tend to
be specific to particular machine architectures. The \DAP needs a data
representation that can be understood by all the clients and server programs
likely to be used on it.

Under the \DAP, there are four categories of information that pass
from the server to the client: information about data, the data
itself, error messages, and information about the server.  The first
three of these correspond to the three important \DAP data objects:
the \DDX, \Blob, and \ErrorX objects.  Information about a \DAP server
is provided by version messages and the \CapX.  These are described in
detail in this section.

Some of the details about how \DAP data objects are transmitted from
server to client are specific to the communication protocol used.  For
these details relevant to the HTTP version of the \DAP, see \DAPHTTP.
For a description of the data objects described by the \DDX object,
see \Sectionref{sec-variables} and \Sectionref{sec-Attributes}.

%  %  %  %  %  %  %  %  %  %  %  %  %  %  %  %  %  %  %  %  %  %  %  %  %
\subsection{DDX}
\label{sec-DDX}

The \DDX is an XML representation of the structure of all or part of a data
set, as well as a description of the variables within that \Dataset. A data
set's structure is its constituent variables plus its global attributes. Each
variable, has a name, type, value, zero or more \Attributes and an optional
origin. The \Dataset itself is modeled as a \Structure variable and so has
its own set of \Attributes. These attributes are the \Dataset's global
attributes. The values of variables are encoded in the \Blob object, and are
only indirectly part of the \DDX object. \Attribute values, however, are
recorded directly in the \DDX. The \CE mechanism can be used to request a
\DDX that does not contain any \Attribute information (see
\Sectionref{sec-xml-ce})

The \DDX is intended to be a way for client programs to both learn about the
contents of a data source and then to access some or all of the information
held by that data source. It is possible to ask a \DAP server to send a \DDX
which describes only a portion of the data source's complete content
({\it e.g.}, to send only one variable within the data source) and to limit
that to only the first 100 by 100 elements (see \Sectionref{sec-ce} for
information about \CEs). In normal operation a client program will ask a
server for the \DDX for an entire \Dataset, determine the variables in which
it is interested and then request a second \DDX that contains only those
variables using a \CE. It will ignore the \Blob reference in the first \DDX
and use the \Blob in the second to access the values it wants.

When clients access the information held in a data source's variable, they do
so using the \Blob. It is the \Blob which is used to transfer that
information from the data source, via a server, to the client. A \DDX
provides descriptive information about the data source, the names, types and
\Attributes held by the data source. The \DDX also provides a reference to
the \Blob that holds that information. When a client requests the \DDX for an
entire data source, it is sent a \DDX which contains a reference to the \Blob
which, in turn, will return the values of the all the variables. The client
is under no obligation to access a \Blob and the server need not `create' it.
The server's contract is simply that, if asked, it will return the \Blob.

Each \DDX holds not only the name and type of each variable, but also
its \Attributes. \Attributes do not have their values accessed using a
separate document/object as variables do; their values are included in
the \DDX. If a client is to make several requests from a single data
source, it is important to avoid the needless repetition of the
\Attribute information.  It is possible, therefore, using the \CE, to
ask a data source to return a \DDX which contains no \Attribute
information.

\examplelabel{ex-ddx}

Here is an example \DDX which contains a single two dimensional array. The
\Dataset also contains a single global \Attribute:

\begin{vcode}{t}
<Dataset name="fnoc1.nc''
 xmlns:xsi="http://www.w3.org/2001/XMLSchema-instance"
 xmlns="http://www.opendap.org/ns/OPeNDAP"
 xsi:schemaLocation="http://www.opendap.org/ns/OPeNDAP  
 http://dods.coas.oregonstate.edu:8080/opendap/opendap.xsd" >

    <Attribute name="Description" type="String">
        <value>Fleet Numerical Wind Data</value>
    </Attribute>

    <Array name="u">
        <Attribute name="long_name" type="String">
            <value>U_Wind_Vector</value>
        </Attribute>
        
        <Float32/>
        
        <dimension size="16" name="latitude"/>
        <dimension size="17" name="longitude"/>
        <dimension size="21" name="time"/>
    </Array>

    <Blob URL="http://dcz.opendap.org/dap/data/nc/fnoc1.nc?u"/>
</Dataset>
\end{vcode}

\note { Each of the XML elements used to declare a variable has a
\texttt{name} attribute. The XML Schema (the rigorous definition) for the
syntax of the XML document declares that the \texttt{name} attribute is
optional. In practice this is not the case, with one exception. Consider
that, without a name, there is no way for a client to ask for a variable. The
only exception to this is the template variable for an \Array. Each \Array
has a single child element which declares the type of the \Array.
Naming the child element is redundant (see \Sectionref{sec-xml-Array}), and
if named the name will be ignored. }


%  %  %  %  %  %  %  %  %  %  %  %  %  %  %  %  %  %  %  %  %  %  %  %  %
\subsection{XML Schema}
\label{sec-XML-Schema}

The syntax and rules for the \DDX document are encapsulated (to the extent 
possible) in an XML schema. The XML schema language is not adequate to 
completely define and enforce the rules for the \DDX. The description of 
the \DDX elements in \Sectionref{sec-DDX-XML-Elements} constitutes the 
complete list of rules and syntax for the \DDX. 

\subsubsection{XML Schema Validation}
\label{sec-xml-valiadtion}

The XML schema is used to validate \DDX documents as part of parsing them 
into memory resident software entities. The act of validating an instance 
of the \DDX against the schema guarantees that the \DDX will fulfill all 
of the syntax rules encapsulated in the schema, thus allowing subsequent 
software in the processing chain to take as ``true'' a large number of 
facts about the content and structure of the instance of the \DDX document.

OPeNDAP servers MUST validate their \DDX documents before sending them in 
response to a request. OPeNDAP clients MAY validate the returned \DDX 
document, but this is seen as non-essential as the servers should be 
providing correct \DDX instances.




%  %  %  %  %  %  %  %  %  %  %  %  %  %  %  %  %  %  %  %  %  %  %  %  %
\subsection{\DDX XML Elements}
\label{sec-DDX-XML-Elements}

This section contains the detailed syntax descriptions of all of the 
component elements of a \DDX.

% - # - # - # - # - # - # - # - # - # - # - # - # - # - # - # - # - # - #
\ELEMENT{Alias}

This element creates a second name for some dataset variable. References to
the \Alias MUST produce the same results as a reference to the \Alias target
identified in the {\it source} attribute. See \Sectionref{sec-Alias}.

%\Aliases MUST play specific roles when used as the member variables of a
%\Grid. See \Sectionref{sec-constructor-grid} and \Sectionref{sec-xml-grid} 
%for more information.

% If an Alias can contain Attributes we must resolve how to combine those
% Attributes with the Attributes of the thing being aliased. At this point
% we are doing something similar to XSLT which, in the grand scheme is the
% province of the AIS. I think we should hold to the restriction that Aliases
% do not have attributes themselves. 10/27/03 jhrg

The following \Alias declaration creates a second name for a variable named
\lit{pepper}, part of a structure named \lit{spice}. With this declaration in
place, you can refer to the same value with three different names:
\lit{/spice/pepper}, \lit{/spice/poivre} and \lit{/goeswithsalt}.

\examplelabel{ex-alias}
\begin{vcode}{it}
<Dataset name="test">
    <Structure name="spice">
        <Float64 name="pepper"/>
        <Alias name="poivre" source="/spice/pepper"/>
    </Structure>
    <Alias name="goeswithsalt" source="/spice/pepper"/>
    <Blob URL="..."/>
</Dataset>
\end{vcode}

\paragraph{Element Attributes}
\begin{description}
  \ATTRIBUTE{name} [Required] The name of the \Alias.

  \ATTRIBUTE{source} [Required] The name of the variable to which this 
  \Alias refers (the {\it target} variable).  The name given here MUST 
  be a \FQN (See \Sectionref{sec-FQN}) and MAY NOT refer to a member 
  variable of a \Sequence. Similarly , if the \Alias is member of
  a \Sequence, then it MAY NOT refer to a variable outside of it's parent
  \Sequence. See \Sectionref{sec-Alias} for more details.

  \ATTRIBUTE{role} [Conditional] The role that the \Alias will play as
  a member of a \Grid. The only acceptable values are \lit{array} and
  \lit{map}. This attribute MUST be used if the \Alias is a member 
  variable of a \Grid, as either the \Array or the \Map. This attribute 
  MUST NOT be used if the \Alias is not a member element of a \Grid.
  See \exampleref{ex-grid-2} and \exampleref{ex-grid-3} for usage examples.

  \ATTRIBUTE{origin} [Conditional] The \lit{modifier} that added this \Alias 
  to the document. Used only if this element was added in conjunction
  with a \ProcAttribute. See \Sectionref{sec-attr-processing}

\end{description}


\paragraph{Child Elements}  None.

% - # - # - # - # - # - # - # - # - # - # - # - # - # - # - # - # - # - #
\ELEMENT{Array}

Declares an \Array variable.  See \Sectionref{sec-constructor-array}
for a description of \Arrays.\\[2mm]

\paragraph{Element Attributes}
\begin{description}
\ATTRIBUTE{name} [Required] The name of the \Array.

\ATTRIBUTE{origin} [Conditional] The \lit{modifier} that added this \Array 
to the document. Used only if this element was added in conjunction 
with a \ProcAttribute. See \Sectionref{sec-attr-processing}

\end{description}

\paragraph{Child Elements}
\begin{description}
\item \element{dimension} [Required] Each \emph{dimension} element MUST
correspond to a dimension of the \Array. The order of the \emph{dimension}
elements MUST indicate the order of the dimensions of the \Array. As is the
case with ANSI C\cite{ansi:c} and C++\cite{ansi:c++}, the rightmost dimension
varies fastest. At least one \emph{dimension} element is REQUIRED; there is
no upper bound on the number of dimensions.

\item \element{Attribute} [Optional] The \Attributes for this \Array.

\item \element{Template variable declaration} [Required] The \emph{template
  variable} defines the data type of each element of the \Array. An \Array
  MUST have exactly one \emph{template variable} which MAY be any type with
  the exception that it MUST NOT be an \Alias or an \Array
  itself.\footnote{Unlike C, et c., in the \DAP multi-dimensional arrays are
  not built up using arrays or arrays.} The {\it name} attribute of the
  \emph{template variable} is OPTIONAL, and MUST be ignored if used. The name
  of the \Array is defined by the {\it name} attribute of the \Array element.
  The template variable is not directly accessible through the data model.
\end{description}

\textbf {Child element syntax:}

\begin{itemize}
\item Zero or more \Attributes followed by
\item One template variable element followed by
\item One or more \element{dimension} elements.
\end{itemize}

Here is an array of 72 64-bit floating point values:
\examplelabel{ex-array}
\begin{vcode}{t}
<Array name="values">
  <dimension size="72"/>
  <Float64/>
</Array>
\end{vcode}
=======

Here is an array of structures:

\examplelabel{ex-array2}
\begin{vcode}{t}
<Array name="time_series">
  <Structure>
    <Float64 name="X_velocity"/>
    <Float64 name="Y_velocity"/>
  </Structure>
  <dimension size="72"/>
</Array>
\end{vcode}

% - # - # - # - # - # - # - # - # - # - # - # - # - # - # - # - # - # - #


\ELEMENT{Attribute}

Use this element to attach \Attribute values to a variable.  Every variable 
MUST have a name, type, and value.  Beyond that, it MAY have an arbitrary 
number of \Attributes.  See \Sectionref{sec-Attributes} for more detail 
about what \Attributes are and how they are used.

If a variable element has \Attribute elements, then the \Attribute elements 
MUST immediately follow the variable element opening tag.

An \Attribute MAY contain multiple values of the same type (in essence a
one-dimensional array), or it MAY contain other \Attributes. It MAY NOT
contain both values and \Attributes.

\exampleref{ex-attribute} contains a syntax example of \Attributes.

\element{\textbf{Attribute Aliases}}
\label{sec-xml-Attribute-Alias}

\Attribute elements MAY have a type of \lit{alias}. An \lit{alias} typed 
\Attribute MAY refer to any other \Attribute in the dataset. If an \Attribute 
is of type \lit{alias} its XML declaration MUST have a \lit{source} attribute 
whose value is the \FQN in the dataset of the \Attribute to which the 
\lit{alias} refers. See \Sectionref{sec-FQN}. There are specific rules for 
encoding the value of the \lit{source} attribute. Rules for encoding the 
value of the source attribute MUST be applied prior to encoding the content 
for it's XML representation. See \Sectionref{sec-encoding-rules} for the 
details of this encoding.

\exampleref{ex-alias-attribute} contains a syntax example for \lit{alias}
typed \Attributes.


\element{\textbf{Processing Attributes}}
\label{sec-xml-proc-attr}

\Attribute elements MAY have a type of \lit{modifier}. \Attributes of
type \lit{modifier} MUST be global (exist at the top level of the \Dataset) 
and MAY not be used as members of other \Attributes or of other
variable \Attribute structures. See \Sectionref{sec-attr-processing}

\Attributes of type \lit{modifier} MUST contain the following elements:

\begin{description}

  \item [origin\_server] An \Attribute of type \URL whose name MUST be
   \lit{origin\_server} and whose value is the \URL for the original 
   data source.
  
  \item [organization] An \Attribute of type \String whose name MUST be 
  \lit{organization} and whose value MUST be the name or the description 
  of the organization that is providing the modified data.
  
  \item [modifying\_service] An \Attribute of type \URL whose name MUST be 
  \lit{modifying\_service} and whose value MUST be the URL of the service 
  that introduced the changes.
\end{description}
A syntax example of a \lit{modifier} typed \Attribute can be found in 
\exampleref{ex-modifier-attr}.\\[3mm]


 
\paragraph{Element Attributes}
\begin{description}

  \ATTRIBUTE{name} [Required] A string containing the name of the \Attribute.
  
  \ATTRIBUTE{type} [Required] The type of this \Attribute's value.  
  This MUST be one of the following: \\

%%% Enumeration?  Is this duplicating a list elsewhere?  If so, let's
%%%  refer to that list instead of duplicating it. ts.
   
\begin{itemize}
  \item \lit{Boolean}
  \item \lit{Byte}
  \item \lit{Int16}
  \item \lit{UInt16}
  \item \lit{Int32}
  \item \lit{UInt32}
  \item \lit{Int64}
  \item \lit{UInt64}
  \item \lit{Float32}
  \item \lit{Float64}
  \item \lit{String}
  \item \lit{URL}
  \item \lit{modifier}
  \item \lit{alias}
  \item \lit{structure}
\end{itemize}
  
  An \Attribute of type structure has a syntax comparable to that of a 
  \Structure variable. \Bis and \Enums are not permitted \Attribute types.  
  
  \ATTRIBUTE{source} [Conditional] A string containing the name of the source 
  \Attribute for an \Attribute of type \lit{alias}. Used only of the \Attribute 
  is of type \lit{alias}.
  
  \ATTRIBUTE{origin} [Conditional] The \lit{modifier} that added this \Attribute 
  to the document. Used only if this element was added in conjunction with 
  a \ProcAttribute. See \Sectionref{sec-attr-processing}
  
\end{description}

\paragraph{Child Elements} 

\begin{description}
\item \element{value} [Conditional] One or more child \element{value} elements, 
allowed ONLY if there are no child \Attribute elements.
\item \element{Attribute} [Conditional] One or more child \Attribute elements, 
allowed ONLY if there are no child \element{value} elements.
\end{description}

\textbf {Child element syntax:}

\begin{itemize}
\item One or more \Attribute elements OR,
\item One or more \element{value} elements 
\end{itemize}

One of the following:

\begin{itemize}
  \item One or more \Attribute elements\
  \item One or more \element{value} elements 
\end{itemize}
  
\examplelabel{ex-attribute}

Here are examples of the \Attribute element syntax.

\begin{vcode}{t}
<Dataset name="test" >
    <Structure name="measurement">
        <Attribute name="date" type="String">
            <value>18 Mar 03</value>
        </Attribute>
        <Attribute name="other" type="Structure">
            <Attribute name="satellite_name" type="String">
                <value>GOES</value>
            </Attribute>
            <Attribute name="experiment number" type="int32">
                <value>986743</value>
            </Attribute>
            <Attribute name="team" type="String">
                <value>Baker</value>
                <value>Charlie</value>
                <value>Dogg</value>
            </Attribute>
        </Attribute>
        <Array name="time_series"
            <dimension size="32"/>
            <Float64 name="value"/>
        </Array>
    </Structure>
    <Blob URL="..."/>
</Dataset>
\end{vcode}
\\[4mm]

\examplelabel{ex-alias-attribute}

This example shows the use of \lit{alias} typed \Attributes.

\begin{vcode}{t}
<Dataset name="test" >
    <Structure name="measurement">
        <Attribute name="team" type="structure">
            <Attribute name="lead engineer" type="String">
                <value>Chet Baker</value>
            </Attribute>
            <Attribute name="software engineer" type="String">
                <value>Charlie Parker</value>
            </Attribute>
            <Attribute name="electrical engineer" type="String">
                <value>Ozzy Osbourne</value>
            </Attribute>
        </Attribute>
        <Float64 name="value"/>
        <Array name="time_series">
            <Attribute name="author" 
                       type="alias" 
                       source="/measurement/team/software engineer" />
            <dimension size="32"/>
            <FLoat32/>
        </Array>
    </Structure>
    <Blob URL="..."/>
</Dataset>
\end{vcode}
\\[4mm]

\examplelabel{ex-modifier-attr}

In this example of the use of a \lit{modifier} \Attribute, the 
AIS server at ais.gso.uri.edu added a variable called
\lit{sst}, and an \Attribute called ``units'' to the variable called
\lit{Depth}.

\begin{vcode}{it}

<Dataset name="test">
    <Attribute name="helena" type="modifier">
        <Attribute name="origin_server" value="http://dods.gso.uri.edu/cgi/nph-nc"/>
        <Attribute name="modifying_service" value="http://ais.gso.uri.edu"/>
        <Attribute name="organization" value="URI/GSO"/>
    </Attribute>  
      .
      .
      .
    <Float64 name="Depth">
        <Attribute name="units" type="String" origin="helena">
            <value>meters</value>
        </Attribute>
    </Float64>
      .
      .
      .
    <Array name="sst" origin="helena">
      .
      .
      .
    </Array>
    <Blob URL="..."/>
</Datset>
\end{vcode}

% - # - # - # - # - # - # - # - # - # - # - # - # - # - # - # - # - # - #
\ELEMENT{Binary}

Declares a \Bi. This is MUST be\footnote{Does this make sense?} an atomic
type of arbitrary size with an undeclared internal structure. See
\Sectionref{sec-atomic-binary}.

The size of a binary image MUST be encoded in the \Blob object, so declaring
it in the \element{Binary} element is optional. For a constructor which holds
\Bis, declaring the size here is a convenience for clients and is OPTIONAL.
Doing so can increase the efficiency of clients which have to deal with the
data after downloading it. If the size of a \Bi is declared in a Constructor
type such as \Sequence, all instances of the \Bi MUST be the declared size.

\paragraph{Element Attributes} 
\begin{description}
  \ATTRIBUTE{name}  [Conditional] A string containing the name of the variable. 
  May (and should) be omitted if the element is being used as the 
  \new{template variable declaration} for an \Array.
  \ATTRIBUTE{size}  [Optional] The size of the image, in bytes.
  \ATTRIBUTE{origin} [Conditional] The \lit{modifier} that added this \Bi to 
  the document. Used only if this element was added in conjunction 
  with a \ProcAttribute. See \Sectionref{sec-attr-processing}.
  
\end{description}

\paragraph{Child Elements} 
\begin{description}
\item \element{Attribute} [Optional] The \Attributes for this \Bi.
\end{description}


\paragraph{Child Element Syntax}

\begin{itemize}
\item Zero or more \Attribute elements
\end{itemize}


\examplelabel{ex-bimage}
\begin{vcode}{t}
<Binary name="sound_sample" size="17256"/>
\end{vcode}

% - # - # - # - # - # - # - # - # - # - # - # - # - # - # - # - # - # - #
\ELEMENT{Blob}

\Blob MUST be the URL the reference to the serialized binary data content
described by this \DDX. See \Sectionref{sec-blob}.

\paragraph{Element Attributes} 
\begin{description}
  \ATTRIBUTE{URL} [Required] A string containing the web address of the 
  \Blob object associated with this \DDX.
\end{description}

\paragraph{Child Elements} None

See \exampleref{ex-ddx}.

% - # - # - # - # - # - # - # - # - # - # - # - # - # - # - # - # - # - #
\ELEMENT{Boolean}

A variable that MUST be one of two values: \texttt{true} or \texttt{false}.
See \Sectionref{sec-atomic-boolean}.

\paragraph{Element Attributes} 
\begin{description}
  \ATTRIBUTE{name}  [Conditional] A string containing the name of the variable. 
  May (and should) be omitted if the element is being used as the \new{template 
  variable declaration} for an \Array.
  \ATTRIBUTE{origin} [Conditional] The \lit{modifier} that added this \Boolean to 
  the document. Used only if this element was added in conjunction 
  with a \ProcAttribute. See \Sectionref{sec-attr-processing}.
\end{description}


\paragraph{Child Elements} 
\begin{description}
\item \element{Attribute} [Optional] The \Attributes for this \emph{Boolean}.
\end{description}

\paragraph{Child Element Syntax}

\begin{itemize}
\item Zero or more \Attribute elements
\end{itemize}

\examplelabel{ex-boolean}
\begin{vcode}{t}
<Boolean name="QC">
  <Attribute name="long_name" type="String">
    <value>Quality Control Flag</value>
  </Attribute>
</Boolean>
\end{vcode}

% - # - # - # - # - # - # - # - # - # - # - # - # - # - # - # - # - # - #
\ELEMENT{Byte}

Declaration of an eight-bit unsigned integer.  See \Sectionref{sec-atomic-int}.

\paragraph{Element Attributes} 
\begin{description}
  \ATTRIBUTE{name}  [Conditional] A string containing the name of the variable. 
  May (and should) be omitted if the element is being used as the \new{template 
  variable declaration} for an \Array.
  \ATTRIBUTE{origin} [Conditional] The \lit{modifier} that added this \Byte to 
  the document. Used only if this element was added in conjunction 
  with a \ProcAttribute. See \Sectionref{sec-attr-processing}.
\end{description}

\paragraph{Child Elements} 
\begin{description}
\item \element{Attribute} [Optional] The \Attributes for this \emph{Byte}.
\end{description}

\paragraph{Child Element Syntax}

\begin{itemize}
\item Zero or more \Attribute elements
\end{itemize}

\examplelabel{ex-byte}
\begin{vcode}{t}
<Byte name="Temperature">
  <Attribtue name="units" type="String">
    <value>Counts</value>
  </Attribute>
</Byte>
\end{vcode}

% - # - # - # - # - # - # - # - # - # - # - # - # - # - # - # - # - # - #
\ELEMENT{Dataset}

A \Dataset element contains all the variables and global attributes for a
data source. The \DDX MUST contain a \Dataset element as its root. A
\element{Dataset} element MUST be semantically equivalent to the \Structure
variable; the rules for encoding the variables in a \Structure apply to the
variables at the top level of the \element{Dataset} element. See
\Sectionref{sec-constructor-structure} for information about the semantics of
a \Structure. See \Sectionref{sec-Attributes} about global and other
\Attributes.

\paragraph{Element Attributes} 
\begin{description}

  \ATTRIBUTE{name}  [Required] A string containing the name of the variable.
  
  \ATTRIBUTE{xmlns}  [Required] A URI containing the default namespace 
  declaration for the XML content of the \Dataset document. At the time 
  of this writing this value should be 
  {\it http://www.opendap.org/ns/OPeNDAP}
  
  \ATTRIBUTE{xmlns:xsi} [Required] Maps the namespace identifier \lit{xsi} 
  to the URI provided in the value. In this case it should always be set to 
  {\it http://www.w3.org/2001/XMLSchema-instance}
  
  \ATTRIBUTE{xsi:schemaLocation} [Required] Should be set to a pair of
  values containing the default namespace URI followed by a URL that
  when dereferenced will provide the schema for the namespace. The
  schema location URL will typically be set to a location on the
  server that is providing the \Dataset document.
  
%%% What character separates the URI and the URL in the above?  ts.

  \ATTRIBUTE{origin} [Conditional] The ``modifier'' that created this \Dataset.
  This would be used to indicate that the \Dataset was created by 
  an Aggregation Server of some sort.
  Used only if this element was added in conjunction 
  with a \ProcAttribute. See \Sectionref{sec-attr-processing}.
  
\end{description}

\paragraph{Child Elements} 
\begin{description}
  \item \element{Attribute} [Optional] The \Attributes for this \Dataset.
  \item \element{Variable Element Declarations} [Required] The collection of
  variables present in the dataset
\end{description}

\paragraph{Child Element Syntax}

\begin{itemize}
\item Zero or more \Attribute elements; followed by
\item One or more variable elements; followed by
\item One \Blob element
\end{itemize}

See \exampleref{ex-ddx}.

% - # - # - # - # - # - # - # - # - # - # - # - # - # - # - # - # - # - #
\ELEMENT{dimension}

This element appears within \element{Array} and \element{Map}
declarations, and declares the length (and possibly the name) of
a dimension.  For multidimensional \Arrays or \Maps, the first
\element{dimension} element MUST correspond to the left-most \Array or \Map
index, {\it et cetera}.

\paragraph{Element Attributes} 
\begin{description}
  \ATTRIBUTE{name} [Optional] A string containing the name of the dimension.
  \ATTRIBUTE{size} [Required] The number of elements in the dimension under 
  consideration.
\end{description}

\paragraph{Child Elements}  None.

See \exampleref{ex-array} and \exampleref{ex-array2}.

% - # - # - # - # - # - # - # - # - # - # - # - # - # - # - # - # - # - #
\ELEMENT{Enumeration}

An \element{Enumeration} is used to bind symbols to a set of discrete
integral values. Each element of an \Enum is called an
\element{enumerator}.

\paragraph{Element Attributes} 
\begin{description}
  \ATTRIBUTE{name} [Required] A string containing the name of the variable.
  \ATTRIBUTE{origin} [Conditional] The \lit{modifier} that added this \Bi to the
  document. Used only if this element was added in conjunction with a 
  \ProcAttribute. 
  
\end{description}

\paragraph{Child Elements} 
\begin{description}
  \item \element{enumerator} [Required] Used to hold a discrete value of the
  \Enum. Each \Enum MUST have at least one \element{enumerator} element.
  \item \element{Attribute} [Optional] The \Attributes for this 
  \emph{Enumeration}.
\end{description}

\textbf {Child element syntax:}

\begin{itemize}
\item Zero or more \Attribute elements; followed by
\item One or more \element{enum} elements
\end{itemize}

\begin{itemize}
\item Zero or more \Attribute elements; followed by
\item One or more \element{enum} elements
\end{itemize}

\examplelabel{ex-enum}
\begin{vcode}{t}
<Enumeration name="error_codes">
<enumerator name="no_such_file" value="0"/>
<enumerator name="insufficient_permissions" value="1"/>
</Enumeration>
\end{vcode}

% - # - # - # - # - # - # - # - # - # - # - # - # - # - # - # - # - # - #
\ELEMENT{enum}

This element holds a single enumerator for an \Enum element.

\paragraph{Element Attributes} 
\begin{description}
  \ATTRIBUTE{name} [Required] The name of this \element{enum}.
  \ATTRIBUTE{value} [Required] The integral value of this
  \element{enum}. Limited to \thirtytwobitlimit{values}.
\end{description}

\paragraph{Child Elements}  None.

See \exampleref{ex-enum}.

% - # - # - # - # - # - # - # - # - # - # - # - # - # - # - # - # - # - #
\ELEMENT{Float32}

Declares an IEEE 754 conformant data variable to hold a 32-bit
floating-point value.  See \Sectionref{sec-atomic-float}.

\paragraph{Element Attributes} 
\begin{description}
  \ATTRIBUTE{name}  [Conditional] A string containing the name of the variable. 
  May (and should) be omitted if the element is being used as the \new{template 
  variable declaration} for an \Array.
  \ATTRIBUTE{origin} [Conditional] The \lit{modifier} that added this
  \type{Float32} to the document. Used only if this element was added in
  conjunction with a \ProcAttribute. See
  \Sectionref{sec-attr-processing}.
\end{description}

\paragraph{Child Elements} 
\begin{description}
  \item \element{Attribute} [Optional] The \Attributes for this 
  \element{Float32}.
\end{description}

\paragraph{Child Element Syntax}

\begin{itemize}
\item Zero or more \Attribute elements
\end{itemize}

\examplelabel{ex-float32}
\begin{vcode}{t}
<Float32 name="Temperature"/>
\end{vcode}

% - # - # - # - # - # - # - # - # - # - # - # - # - # - # - # - # - # - #
\ELEMENT{Float64}

Declares an IEEE 754 conformant data variable to hold a 64-bit
floating-point value.  See \Sectionref{sec-atomic-float}.

\paragraph{Element Attributes} 
\begin{description}
  \ATTRIBUTE{name}  [Conditional] A string containing the name of the variable. 
  May (and should) be omitted if the element is being used as the \new{template 
  variable declaration} for an \Array.
  \ATTRIBUTE{origin} [Conditional] The \lit{modifier} that added this
  \type{Float64} to  the document. Used only if this element was added
  in conjunction  with a \ProcAttribute. See
  \Sectionref{sec-attr-processing}. 
\end{description}


\paragraph{Child Elements} 
\begin{description}
  \item \element{Attribute} [Optional] The \Attributes for this 
  \element{Float64}. 
\end{description}

\paragraph{Child Element Syntax}

\begin{itemize}
\item Zero or more \Attribute elements
\end{itemize}

\examplelabel{ex-float64}
\begin{vcode}{t}
<Float64 name="Temperature">
  <Attribtue name="units" type="String">
    <value>Degrees_Kelvin</value>
  </Attribute>
</Float64>
\end{vcode}

% - # - # - # - # - # - # - # - # - # - # - # - # - # - # - # - # - # - #
\ELEMENT{Grid}
\label{sec-xml-grid}

Declares a \Grid variable (see \Sectionref{sec-constructor-grid}). Unlike an
\Array, a \Grid can be indexed using types other than integers. The mapping
between different values and discrete elements of the \Grid is given by the
\Maps. Within a \Grid the correspondence between any dimension of the \Grid's
target \Array and a \Map is made by insuring that corresponding target \Array
dimensions and \Map dimensions have the same name and size. A \Map is a type
of an \Array and so has its own dimension element, which MUST be named, and
this name is used to create the binding of the \Map to a dimension with the
same name in the \Grid's target \Array. While the naming of the dimensions
establishes the relationship between \Maps and \Array components in a \Grid,
the dimensions MUST be the same size in order for the mapping to be complete.
If two dimensions in a \Grid of the same name do not have the same size an
error will be generated.

In order to accommodate the re-use of the components of a \Dataset and to
streamline the transmission of data through redundancy reduction, both the
target \Array and the \Map elements of a \Grid MAY be replaced with an \Alias
variable. In both cases the \Alias MUST refer to an \Array type (either an
\Array or a \Map) somewhere else in the \Dataset. The rules about naming and
size for the constituent dimensions of the \Grid components will be evaluated
against the source of the \Alias reference as if they where actually
components of the \Grid. (see the comments in \exampleref{ex-grid-1},

\exampleref{ex-grid-2}, and \exampleref{ex-grid-3}).

\paragraph{Element Attributes} 
\begin{description}
  \ATTRIBUTE{name}  [Conditional] A string containing the name of the variable. 
  May (and should) be omitted if the element is being used as the \new{template 
  variable declaration} for an \Array.
  \ATTRIBUTE{origin} [Conditional] The \lit{modifier} that added this \Grid to 
  the document. Used only if this element was added in conjunction 
  with a \ProcAttribute. See \Sectionref{sec-attr-processing}.
\end{description}

\paragraph{Child Elements} 
\begin{description}

  \item \element{Attribute} [Optional] The \Attributes for this \Grid.

%  \item \element{Target} [Required] Contains the target \Array or \Alias (to
%  the target \Array) containing the data held by this variable.
%  This can often be thought of as a dependent variable. There MUST be 
%  exactly one \Target element in every \Grid.

  \item \element{Map} [Required] The maps for this \Grid. These can 
  be thought of as the independent variables for the \Grid's \Array. 
  There MUST be at least one map and no more than $N$ where $N$ is 
  the rank of the \Grid's \Array.

\end{description}

\textbf {Child element syntax:}

\begin{itemize}
\item Zero or more \Attribute elements; followed by
%\item One \Target element; followed by
\item Zero or one \Array element; followed by
\item Zero or more \Map elements; followed by
\item Zero or more \Alias elements. 
\end{itemize}

Each dimension of the target \Array MUST have one or more corresponding 
\Map dimensions of the same name and size. If a \Map has multiple dimensions, 
then each one MUST be a valid \Map dimension for one of the dimensions in 
the target \Array. An \Alias that refers to an \Array of the correct size 
and shape MAY replace any or all of \Map elements and/or the \Array element.

Here's a \Grid containing a two-dimensional target array and two
one-dimensional map arrays. 

\examplelabel{ex-grid-1}~

This example shows a typical \Grid. All of the \Grid components, 
the target \Array and the \Maps, are declared as members of the \Grid element.

\begin{vcode}{t}

<Grid name="v">
    <Array name="temp">
        <Byte/>
        <dimension name="lat" size="5"/> <!-- bound to the Map dimension named "lat" -->
        <dimension name="lon" size="5"/>
    </Array>
    <Map name="y">                   <!-- this name does not matter -->
        <Float64/>
        <dimension size="5" name="lat"/>  <!-- this name completes the association -->
    <Map name="x">
        <Float64/>
        <dimension size="5" name="lon"/>
    </Map>
</Grid>
\end{vcode}
\\[4mm]

\examplelabel{ex-grid-2}

This example shows a \Grid where the target \Array element is actually an \Alias 
to an \Array outside of the \Grid element.

\begin{vcode}{t}

<Array name="temp">
    <Byte/>
    <dimension name="lat" size="5"/> <!-- bound to the Map dimension named "lat" -->
    <dimension name="lon" size="5"/>
</Array>
<Grid name="v">
    <Alias name="sst" source="/temp" role="Array" />
    <Map name="y">                   <!-- this name does not matter -->
        <Float64/>
        <dimension size="5" name="lat"/>  <!-- this name completes the association -->
    </Map>
    <Map name="x">                        
        <Float64/>
        <dimension size="5" name="lon"/>
    </Map>
</Grid>
\end{vcode}
\\[4mm]

\examplelabel{ex-grid-3}

This example shows a \Grid where all of the member elements (the target \Array and
the \Maps) are actually an \Aliases to a \Arrays outside of the \Grid element.

\begin{vcode}{t}

<Array name="temp">
    <Byte/>
    <dimension name="lat" size="5"/> <!-- bound to the Map dimension named "lat" -->
    <dimension name="lon" size="5"/>
</Array>
< Array name="y">                   <!-- this name does not matter -->
    <Float64/>
    <dimension size="5" name="lat"/>  <!-- this name completes the association -->
</Array >
< Array name="x">                        
    <Float64/>
    <dimension size="5" name="lon"/>
</Array >
<Grid name="v">
    <Alias name="sst" source="/temp" role="Array" />
    <Alias name="NS" source="/y" role="Map" />
    <Alias name="EW" source="/x" role="Map" />
</Grid>

\end{vcode}

% - # - # - # - # - # - # - # - # - # - # - # - # - # - # - # - # - # - #
\ELEMENT{Int16}

A 16-bit signed (twos-complement) integer.  See \Sectionref{sec-atomic-int}.

\paragraph{Element Attributes} 
\begin{description}
  \ATTRIBUTE{name}  [Conditional] A string containing the name of the variable. 
  May (and should) be omitted if the element is being used as the \new{template 
  variable declaration} for an \Array.
  \ATTRIBUTE{origin} [Conditional] The \lit{modifier} that added this 
  \type{Int16} to 
  the document. Used only if this element was added in conjunction 
  with a \ProcAttribute. See \Sectionref{sec-attr-processing}.
\end{description}

\paragraph{Child Elements} 
\begin{description}
  \item \element{Attribute} [Optional] The \Attributes for this \emph{Int16}.
\end{description}

\paragraph{Child Element Syntax}

\begin{itemize}
\item Zero or more \Attribute elements
\end{itemize}

\examplelabel{ex-int16}
\begin{vcode}{t}
<Int16 name="Temperature"/>
\end{vcode}

% - # - # - # - # - # - # - # - # - # - # - # - # - # - # - # - # - # - #
\ELEMENT{Int32}

A 32-bit signed (twos-complement) integer.  See \Sectionref{sec-atomic-int}.

\paragraph{Element Attributes} 
\begin{description}
  \ATTRIBUTE{name}  [Conditional] A string containing the name of the variable. 
  May (and should) be omitted if the element is being used as the \new{template 
  variable declaration} for an \Array.
  \ATTRIBUTE{origin} [Conditional] The \lit{modifier} that added this
  \type{Int32} to 
  the document. Used only if this element was added in conjunction 
  with a \ProcAttribute. See \Sectionref{sec-attr-processing}.
\end{description}

\paragraph{Child Elements} 
\begin{description}
  \item \element{Attribute} [Optional] The \Attributes for this \emph{Int32}.
\end{description}

\paragraph{Child Element Syntax}

\begin{itemize}
\item Zero or more \Attribute elements
\end{itemize}

\examplelabel{ex-int32}
\begin{vcode}{t}
<Int32 name="Temperature"/>
\end{vcode}

% - # - # - # - # - # - # - # - # - # - # - # - # - # - # - # - # - # - #
\ELEMENT{Int64}

A 64-bit signed (twos-complement) integer.  See \Sectionref{sec-atomic-int}.

\paragraph{Element Attributes} 
\begin{description}
  \ATTRIBUTE{name}  [Conditional] A string containing the name of the variable. 
  May (and should) be omitted if the element is being used as the \new{template 
  variable declaration} for an \Array.
  \ATTRIBUTE{origin} [Conditional] The \lit{modifier} that added this 
  \type{Int64} to 
  the document. Used only if this element was added in conjunction 
  with a \ProcAttribute. See \Sectionref{sec-attr-processing}.
\end{description}

\paragraph{Child Elements} 
\begin{description}
  \item \element{Attribute} [Optional] The \Attributes for this \emph{Int64}.
\end{description}

\paragraph{Child Element Syntax}

\begin{itemize}
\item Zero or more \Attribute elements
\end{itemize}

\examplelabel{ex-int64}
\begin{vcode}{t}
<Int64 name="Temperature"/>
\end{vcode}

% - # - # - # - # - # - # - # - # - # - # - # - # - # - # - # - # - # - #
\ELEMENT{Map}

This declaration creates a \Map used in a \element{Grid}. See
\Sectionref{sec-constructor-grid} for a description of \Grids. Also see

\exampleref{ex-grid-1}, \exampleref{ex-grid-2}, and \exampleref{ex-grid-3}. A
\Map is a special case of the \element{\Array} element; it has the same
\Attributes and child elements but MUST appear inside a \Grid.

\paragraph{Element Attributes} 
\begin{description}
  \ATTRIBUTE{name}  [Required] A string containing the name of the \Map variable.
  The relationship between the dimension(s) of the \Map and the target \Array
  of the \Grid is established through the names of the \Map dimensions. The name 
  of the \Map is not used to establish the relationship, but should be used
  in some informative manner for the user. See \exampleref{ex-grid-1}.
\end{description}

\paragraph{Child Elements} 
\begin{description}
  \item \element{dimension} [Required] Each \emph{dimension} element 
  corresponds to a dimension of the \Map. The order of the \emph{dimension}
  elements indicates the order of the dimensions of the \Map. As is the case 
  with ANSI-C\cite{ansi:c} and C++\cite{ansi:c++}, the rightmost/last dimension
  varies fastest. At least one \emph{dimension} element is REQUIRED; there 
  is no upper bound on the number of \element{dimension} elements
  
  \item \element{Attribute} [Optional] The \Attributes for this \Map.
  
  \item \element{Template variable declaration} [Required] A \Map MUST have one
  \emph{template variable} which MAY be any type except that it MUST NOT be an 
  \Alias or an  \Array itself. The  {\it name} attribute of the \emph{template 
  variable} is optional, and MUST be ignored if used. 
  The name of the \Map is defined by the {\it name} attribute of the \Map element. 
  The template variable is not directly accessible through the data model.  
\end{description}

\paragraph{Child Element Syntax}

\begin{itemize}
\item Zero or more \Attributes followed by
\item One template \element{variable} element followed by 
\item One or more \element{dimension} elements.
\end{itemize}

% - # - # - # - # - # - # - # - # - # - # - # - # - # - # - # - # - # - #
\ELEMENT{Sequence}

A \Sequence is an ordered set of entries. Each instance of an entry is
comparable to a \Structure variable. Each entry in a \Sequence contains the
same set of variables. A \Sequence can also be thought of as a relational
database table, with each entry corresponding to a single row.

See \Sectionref{sec-constructor-sequence} for a description of the
\Sequence type.


\paragraph{Element Attributes} 
\begin{description}
  \ATTRIBUTE{name}  [Conditional] A string containing the name of the variable. 
  May (and should) be omitted if the element is being used as the \new{template 
  variable declaration} for an \Array.
%%% Can you have an Array of Sequence?  ts.
  \ATTRIBUTE{origin} [Conditional] The \lit{modifier} that added this \Sequence to 
  the document. Used only if this element was added in conjunction 
  with a \ProcAttribute. See \Sectionref{sec-attr-processing}.
\end{description}

\paragraph{Child Elements} 
\begin{description}
  \item \element{Attribute} [Optional] The \Attributes for this \Sequence.
  \item \element{Variable declaration} [Required] A \Sequence MUST have 
  one or more \emph{Variable declarations}. They MAY be any type of variable.
\end{description}

\paragraph{Child Element Syntax}

\begin{itemize}
\item Zero or more \Attributes; followed by
\item One or more \element{variable} elements
\end{itemize}

\examplelabel{ex-sequence}
\begin{vcode}{t}
<Sequence name="gallimaufry">
  <Float64 name="measurement"/>
  <Array name="measurement_collection">
    <Int16/>  
    <dimension size="32"/>
    <dimension size="45"/>
  </Array>
</Sequence>
\end{vcode}



% - # - # - # - # - # - # - # - # - # - # - # - # - # - # - # - # - # - #
\ELEMENT{String}

A series of Unicode (UTF-8) characters.  See
\Sectionref{sec-atomic-string}.


\paragraph{Element Attributes} 
\begin{description}
  \ATTRIBUTE{name}  [Conditional] A string containing the name of the variable. 
  May (and should) be omitted if the element is being used as the \new{template 
  variable declaration} for an \Array.
  \ATTRIBUTE{origin} [Conditional] The \lit{modifier} that added this \String to 
  the document. Used only if this element was added in conjunction 
  with a \ProcAttribute. See \Sectionref{sec-attr-processing}.
\end{description}

\paragraph{Child Elements} 
\begin{description}
  \item \element{Attribute} [Optional] The \Attributes for this \String.
\end{description}

\paragraph{Child Element Syntax}

\begin{itemize}
\item Zero or more \Attributes 
\end{itemize}

\examplelabel{ex-string}
\begin{vcode}{t}
<String name="Name"/>
\end{vcode}

% - # - # - # - # - # - # - # - # - # - # - # - # - # - # - # - # - # - #
\ELEMENT{Structure}

An ordered set of variables.  See
\Sectionref{sec-constructor-structure} for a description.

\paragraph{Element Attributes} 
\begin{description}
  \ATTRIBUTE{name}  [Conditional] A string containing the name of the variable. 
  May (and should) be omitted if the element is being used as the \new{template 
  variable declaration} for an \Array.
  \ATTRIBUTE{origin} [Conditional] The \lit{modifier} that added this 
  \Structure to 
  the document. Used only if this element was added in conjunction 
  with a \ProcAttribute. See \Sectionref{sec-attr-processing}.
\end{description}


\paragraph{Child Elements} 
\begin{description}
  \item \element{Attribute} [Optional] The \Attributes for this \Structure.
  \item \element{Variable declaration} [Required] A \Structure MUST have 
  one or more \emph{Variable declarations}. They MAY be any type of variable.
\end{description}

\textbf {Child element syntax:}

\begin{itemize}
\item Zero or more \Attributes; followed by
\item One or more \element{variable declarations}
\end{itemize}

\examplelabel{ex-structure}
\begin{vcode}{t}
<Structure name="person">
  <String name="name">
  <Float64 name="height">
  <Int32 name="age">
</Structure>
\end{vcode}


% - # - # - # - # - # - # - # - # - # - # - # - # - # - # - # - # - # - #
%\ELEMENT{Target}
%
%Use this element tag to identify the target \Array of a \Grid.  See
%\exampleref{ex-grid-1}. 
%
%\paragraph{Element Attributes} None.
%
%\paragraph{Child Elements}
%\begin{description}
%  \item \element{Array} [Conditional] The target \Array for the parent \Grid.
%  \item \element{Alias} [Conditional] An \Alias referring to the target \Array
%  for the parent\Grid.
%\end{description}
%
%\paragraph{Child Element Syntax}
%
%$\bullet$ One \Array; \\
%OR\\
%$\bullet$ One \Alias that refers to an \Array.
%


% - # - # - # - # - # - # - # - # - # - # - # - # - # - # - # - # - # - #
\ELEMENT{Time}

A variable which contains an ISO 8601 time string\cite{UWV:ISO8601}. A \Time
can contain only characters that legal for an ISO 8601 time string.
Currently, this is limited to single-byte US-ASCII (7-bit) characters. See
\Sectionref{sec-atomic-string}.

\paragraph{Element Attributes} 
\begin{description}
  \ATTRIBUTE{name}  [Conditional] A string containing the name of the variable. 
  May (and should) be omitted if the element is being used as the \new{template 
  variable declaration} for an \Array.
  \ATTRIBUTE{origin} [Conditional] The \lit{modifier} that added this \Time to 
  the document. Used only if this element was added in conjunction 
  with a \ProcAttribute. See \Sectionref{sec-attr-processing}.
\end{description}

\paragraph{Child Elements} 
\begin{description}
  \item \element{Attribute} [Optional] The \Attributes for this 
  \element{Time}.
\end{description}

\paragraph{Child Element Syntax}

\begin{itemize}
\item Zero or more \Attributes.
\end{itemize}

\examplelabel{ex-time}
\begin{vcode}{t}
<Time name="Sample_time"/>
\end{vcode}

% - # - # - # - # - # - # - # - # - # - # - # - # - # - # - # - # - # - #
\ELEMENT{UInt16}

An unsigned 16-bit integer.  See \Sectionref{sec-atomic-int}.

\paragraph{Element Attributes} 
\begin{description}
  \ATTRIBUTE{name}  [Conditional] A string containing the name of the variable. 
  May (and should) be omitted if the element is being used as the \new{template 
  variable declaration} for an \Array.
  \ATTRIBUTE{origin} [Conditional] The \lit{modifier} that added this 
  \type{UInt16} to 
  the document. Used only if this element was added in conjunction 
  with a \ProcAttribute. See \Sectionref{sec-attr-processing}.
\end{description}

\paragraph{Child Elements} 
\begin{description}
  \item \element{Attribute} [Optional] The \Attributes for this 
  \element{UInt16}.
\end{description}

\paragraph{Child Element Syntax}

\begin{itemize}
\item Zero or more \Attributes
\end{itemize}

\examplelabel{ex-uint16}
\begin{vcode}{t}
<UInt16 name="Temperature"/>
\end{vcode}

% - # - # - # - # - # - # - # - # - # - # - # - # - # - # - # - # - # - #
\ELEMENT{UInt32}

An unsigned 32-bit integer.  See \Sectionref{sec-atomic-int}.

\paragraph{Element Attributes} 
\begin{description}
  \ATTRIBUTE{name}  [Conditional] A string containing the name of the variable. 
  May (and should) be omitted if the element is being used as the \new{template 
  variable declaration} for an \Array.
  \ATTRIBUTE{origin} [Conditional] The \lit{modifier} that added this
  \type{UInt32} to 
  the document. Used only if this element was added in conjunction 
  with a \ProcAttribute. See \Sectionref{sec-attr-processing}.
\end{description}

\paragraph{Child Elements} 
\begin{description}
  \item \element{Attribute} [Optional] The \Attributes for this 
  \element{UInt32}.
\end{description}

\paragraph{Child Element Syntax}

\begin{itemize}
\item Zero or more \Attributes
\end{itemize}

\examplelabel{ex-uint32}
\begin{vcode}{t}
<UInt32 name="Temperature"/>
\end{vcode}

% - # - # - # - # - # - # - # - # - # - # - # - # - # - # - # - # - # - #
\ELEMENT{UInt64}

An unsigned 64-bit integer.  See \Sectionref{sec-atomic-int}.

\paragraph{Element Attributes} 
\begin{description}
  \ATTRIBUTE{name}  [Conditional] A string containing the name of the variable. 
  May (and should) be omitted if the element is being used as the \new{template 
  variable declaration} for an \Array.
  \ATTRIBUTE{origin} [Conditional] The \lit{modifier} that added this 
  \type{UInt64} to 
  the document. Used only if this element was added in conjunction 
  with a \ProcAttribute. See \Sectionref{sec-attr-processing}.
\end{description}

\paragraph{Child Elements} 
\begin{description}
  \item \element{Attribute} [Optional] The \Attributes for this 
  \element{UInt64}.
\end{description}

\paragraph{Child Element Syntax}

\begin{itemize}
\item Zero or more \Attributes
\end{itemize}

\examplelabel{ex-uint64}
\begin{vcode}{t}
<UInt64 name="Temperature"/>
\end{vcode}

% - # - # - # - # - # - # - # - # - # - # - # - # - # - # - # - # - # - #
\ELEMENT{URL}

A  variable which contains a URL.  A \URL MUST only contain characters
that are legal parts of an internet URL.  Currently, this is limited
to single-byte US-ASCII (7-bit) characters.  See
\Sectionref{sec-atomic-string}.

\paragraph{Element Attributes} 
\begin{description}
  \ATTRIBUTE{name}  [Conditional] A string containing the name of the variable. 
  May (and should) be omitted if the element is being used as the \new{template 
  variable declaration} for an \Array.
  \ATTRIBUTE{origin} [Conditional] The \lit{modifier} that added this \URL to 
  the document. Used only if this element was added in conjunction 
  with a \ProcAttribute. See \Sectionref{sec-attr-processing}.
\end{description}

\paragraph{Child Elements} 
\begin{description}
  \item \element{Attribute} [Optional] The \Attributes for this \element{Url}.
\end{description}

\paragraph{Child Element Syntax}

\begin{itemize}
\item Zero or more \Attributes
\end{itemize}

\examplelabel{ex-url}
\begin{vcode}{t}
<URL name="SST_data_server"/>
\end{vcode}

% - # - # - # - # - # - # - # - # - # - # - # - # - # - # - # - # - # - #
\ELEMENT{Time}

A variable which contains an ISO 8601 time string\cite{UWV:ISO8601}. A \Time
MUST contain ONLY characters that legal for an ISO 8601 time string.
Currently, this is limited to single-byte US-ASCII (7-bit) characters. See
\Sectionref{sec-atomic-string}.

\textbf {Element attributes:} 
\begin{description}
  \ATTRIBUTE{name}  [Conditional] A string containing the name of the variable. 
  May (and should) be omitted if the element is being used as the \new{template 
  variable declaration} for an \Array.
  \ATTRIBUTE{origin} [Conditional] The \lit{modifier} that added this \Bi to 
  the document. Used only if this element was added in conjunction 
  with a \ProcAttribute. See \Sectionref{sec-attr-processing}.
\end{description}

\textbf {Child elements:} 
\begin{description}
  \item \element{Attribute} [Optional] The \Attributes for this 
  \element{Time}.
\end{description}

\textbf {Child element syntax:}

\begin{itemize}
\item Zero or more \Attributes
\end{itemize}

\examplelabel{ex-time}
\begin{vcode}{t}
<Time name="Sample_time"/>
\end{vcode}

% - # - # - # - # - # - # - # - # - # - # - # - # - # - # - # - # - # - #
\ELEMENT{value}

Use this tag to identify the value  of an  \Attribute element.  See
\exampleref{ex-attribute}. 

Variable values are recorded in the \Blob object, while \Attribute values are
recorded directly in the \DDX. This means that \Attribute values MUST be able
to be represented in a (UTF-8) text file.\footnote{Are we capable of
supporting UTF-8 attribute values? jhrg 12/17/03}

\paragraph{Element Attributes} None.

\paragraph{Child Elements} None.

\examplelabel{ex-value}
\begin{vcode}{t}
<value>3.1415</value>
<value>2.718</value>
\end{vcode}



%  %  %  %  %  %  %  %  %  %  %  %  %  %  %  %  %  %  %  %  %  %  %  %  %
\subsection{Encoding Rules}
\label{sec-encoding-rules}

[ {\it This section is a first stab, we expect that as we implement code
to support this specification we will have to change (and add to) the
Encoding Rules. }]\\[5mm]

There are several encoding schemes that need to be applied to 
various XML components of the DAP. At the end of the list is 
the encoding for XML. Other encodings MUST be applied to 
different parts of the \DDX.

\subsubsection{\Attribute and variable \Alias source 
attribute encoding}
\label{sec-alias-source-encoding}

The \lit{source} attribute a variable \Alias and the \lit{source} attribute
of an \Attribute of type \lit{alias} are both \FQNs. Since \FQNs use the
slash (``/") character as the delimiter between variable (and \Attribute)
names it is necessary to specially identify this character when it appears as
part of a legitimate variable (or \Attribute) name in the \DDX document
(which must eventually be parsed by software). This identification is called
``escaping the character", and is achieved by using a different character
that will be interpreted to have the meaning of ``escape". The backslash
("$\backslash$") is the escape character. The the slash (``/") and backslash
("$\backslash$") MUST be escaped if they appear as part of a node name in the
absolute path. This encoding MUST be applied prior to any additional encoding
needed to make the representation XML compatible, and must persist after XML
decoding.

\subsubsection{\Project element variable encoding}

This is the same as \Sectionref{sec-alias-source-encoding}.

\subsubsection{\Select element target encoding}

This is the same as \Sectionref{sec-alias-source-encoding}.

\subsubsection{Base 64 \Attribute value encoding}

In order to decrease the total number of bytes in a \DDX response 
it is possible that we may choose to transmit \Attribute values in 
their serialized binary form (see \Sectionref{sec-blob}). To do so, 
and still include them in the DDX, we will have to encode them into 
a form that allows them to be represented as UTF-8. Most likely 
this will achieved by using a base-64 encoding. 

\subsubsection{XML document encoding}

After all of the previous encodings have been applied to the 
appropriate parts for the document the document must be brought 
into a correct encoding for XML. This means that any content (as
opposed to XML syntax characters) must be so encoded. This includes
the values of XML element attributes and the content of XML elements.
This encoding is minimally:

\begin{itemize}
\item  The \lit < (less than) character is replaced with \lit{\&lt;}
\item  The \lit > (greater than) character is replaced with \lit{\&gt;}
\item  The \& (ampersand) character is replaced with \lit{\&amp;}
\item  The ' (apostrophe) character is replaced with \lit{\&apos;}
\item  The " (double quote) character is replaced with \lit{\&quot;}
\end{itemize}

And may in fact be more extensive. See \cite{w3c:xml} for more
detailed information.


%  %  %  %  %  %  %  %  %  %  %  %  %  %  %  %  %  %  %  %  %  %  %  %  %
\subsection{Blob}
\label{sec-blob}

The \Blob contains the serialized data represented by the \DDX. XML documents
like the \DDX cannot efficiently transmit binary data \cite{w3c:outofband}, so
a \DAP \DDX simply references an object that contains the data described in
the \DDX. Every \DDX MUST contain a \Blob reference although a data source is
under no obligation to create the \Blob until a client requests it (by
dereferencing the \Blob's URL).\footnote{In many cases no sensible client
would request the \Blob associated with a \DDX which describes an entire data
source and it's likely that a sensible server would return an \ErrorX if a
client did try to request such a \Blob.} The \Blob is a way of accessing the
data `out-of-band' with respect to the \DDX. In addition to eschewing the
problems of processing large volumes of binary data with an XML parser,
out-of-band access enables servers to support streaming in conjunction with
protocols such as SOAP\cite{SOAP:streaming}.

Data values associated with variables appear in the \Blob in the order in
which the variables are declared in the associated \DDX. The \DDX always
contains a \element{Dataset} element as its root. A \element{Dataset} element
is semantically equivalent to the \Structure variable; the rules for encoding
the variables in a \Structure apply to the variables at the top level of the
\element{Dataset} element.

The \Blob is \emph{not} self-documenting. A client program will be
unable to make sense of it without the declarations in the accompanying \DDX,
since the variable types, sizes, and ordering determined from the 
structure/organization of the associated \DDX.

\note{The \DDX may be used to define a C++, Java, {\it etc.}, object which
may then be used by a \DAP client to allocate memory for the variables it
declares. A \DDX which has been created, but before the \Blob object has been
received is said to be an ``empty'' \DDX. After the \Blob arrives and its
data has been decoded and parceled out to the memory in the \DDX, the \DDX is
said to be ``full.''}

\subsubsection{Length Specification: Representing lengths of encoded data 
elements}
\label{sec-length-spec}

When values are encoded for transmission, they are often preceded by length
information. For example, an \Array is prefixed by the number of elements in
the \Array. Instead of representing this information in a fixed-size integer
data type, the \DAP encodes length information in a way that allows any
length to be represented. These are called \DAP \emph{Length Specifications}
and are defined as follows:

\begin{quote}
Each byte of the length specification MUST be divided into two parts: the
high-order bit MUST indicate whether another byte follows, and the low-order
seven bits MUST provide the length. Higher-order bytes MUST preceed
lower-order bytes (i.e., the representation MUST be bigendian).\footnote{What
if the client and server negotiate byte order and decide to use little-endian
notation? jhrg 12/17/03}. A two-byte specification of a 500-byte image
would be 10000011 01110100 (0x83 0x74), for example, while a 20-byte image
would only need one byte of length information: 00010100 (0x14). Length
specifications can use any number of bytes.
\end{quote}

\emph{Length Specifications} MUST be used to specify the size of variable
length entities throughout the \Blob encoding.

\subsubsection{\Blob framework and reliable error delivery}
\label{sec-blob-framework}

The \Blob response is a multi-part MIME document\cite{rfc2046}. The \Blob
MUST contain the \texttt{Content-Type} MIME header and MUST give the content
type as \texttt{multipart/mixed}. Each part of the multi-part document MUST
have a \texttt{Content-Length} header which indicates the total number of
bytes in this part and MUST have a \texttt{Content-Type} MIME header and the
type of that header MUST be \texttt{application/octet-stream} or it MUST
\texttt{text/plain}. The \texttt{application/octet-stream} indicates that
binary data are contained in this part of the document and MUST be
interpreted as described in \Sectionref{sec-blob-atomic} and
\Sectionref{sec-blob-constructor}. If the type is \texttt{text/plain}, then
the content MUST be an \ErrorX document.

The \Blob response is encoded using a multi-part MIME document to ensure
reliable delivery of error messages if servers stream responses to clients. A
server can iteratively build chunks of the total response and include that in
the \Blob as the next part. If an error is discovered, that can be sent
instead. A server is free to choose the size for each part and is free to
build \Blob responses with only one part.

\subsubsection{Atomic Types}
\label{sec-blob-atomic}

The \DAP atomic data types are encoded as follows. All data MUST be
encoded using big-endian byte order UNLESS that is modified using the
client-server negotiation options described in
\Sectionref{sec-request-response}.

\begin{description}
\item[Integer types] Signed integers: Twos-complement; unsigned integers:
  straight binary. The \Boolean and \Enum types MUST be sent as unsigned
  32-bit integers.\footnote{The \Boolean type could
  conceivably be sent in a single bit, but decoding that may be inefficient
  for some architectures.  A transport protocol that supports compression
  may achieve the same reduction in size.}

\item[Floating-Point types] IEEE 754

\item[String types] \Strings: Unicode, UTF-8, prefixed by a \emph{Length
  Specification} indicating the length \emph{in bytes} of the string. The
  \URL and \Time types MUST be limited to US-ASCII characters, prefixed by a
  \emph{Length Specification} indicating the length \emph{in bytes}.
  
\item[Binary images] A \Bi MUST be sent as a sequence of bytes, prefixed by a
  \emph{Length Specification} indicating the length \emph{in bytes} of the
  \Bi.
\end{description}

\subsubsection{Constructor Types}
\label{sec-blob-constructor}

The constructor types are encoded as follows.  See above for
instructions about how to encode the atomic types.  Note that  \Alias
variables are omitted from the \Blob.  They are reconstructed using
the \DDX. 

\begin{description}
\item[\Array]
  
  \Array members MUST be encoded in row-major order (rightmost subscript
  varies fastest). The \Array MUST be preceded \emph{Length Specification}
  indicating the the number of \emph{elements} (not the number of bytes) in
  the \Array. The size of each element MUST be derived from the declaration
  in the \DDX. \Arrays of \String, \URL or \Time values MUST include the
  \emph{Length Specification}.
  
\item[\Structure] 
  
  Members MUST be placed in order of declaration, with no boundary values.
  \Alias variables MUST be omitted. (There is no place holder for them in the
  \Blob. They are reconstructed using the \DDX.)

\item[\Grid] 
  
  \Grids, which are essentially special cases of the \Structure type,
  MUST be recorded in the same fashion as \Structures.

\item[\Sequence] 
  
  Each entry in a \Sequence is recorded like a \Structure, EXCEPT that each
  entry MUST be preceded by a single ``Start-of-Instance'' flag byte (value
  \lit{0x5a}), and the entire \Sequence MUST be end with an
  ``End-of-Sequence'' flag byte (same value: \lit{0x5}).

  For example, a schematic view of the data and flags for the three element
  sequence: \\ [2mm]
\begin{vcode}{SOB}
<Sequence>
  <Int32 name="Var1">
  <Int32 name="Var2">
  <Int32 name="Var3">
</Sequence>
\end{vcode}

looks like:\\ [2mm]
\begin{vcode}{SOB}
<SOI><Var1><Var2><Var3>
<SOI><Var1><Var2><Var3>
<SOI><Var1><Var2><Var3>
<EOS>
\end{vcode}

For a nested Sequence (a Sequence which contains a Sequence as a child
element) such as:\\ [2mm]
\begin{vcode}{SOB}
<Sequence>
  <Int32 name="Var1">
  <Int32 name="Var2">
  <Sequence>
    <Int32 name="Var3">
    <Int32 name="Var4">
  </Sequence>
</Sequence>
\end{vcode}

the schematic representation looks like:\\ [2mm]
\begin{vcode}{SOB}
<SOI><Var1><Var2>

<SOI><Var3><Var4>
<SOI><Var3><Var4>
<EOS>

<SOI><Var1><Var2>

<SOI><Var3><Var4>
<SOI><Var3><Var4>
<EOS>

<SOI><Var1><Var2>

<SOI><Var3><Var4>
<SOI><Var3><Var4>
<EOS>
<EOS>
\end{vcode}

Note that the outer sequence has three instances and each of those includes
the inner Sequence. 

\end{description}

%  %  %  %  %  %  %  %  %  %  %  %  %  %  %  %  %  %  %  %  %  %  %  %  %
\subsection{ErrorX}
\label{sec-errorx}

The \ErrorX object is an XML document containing information about any
errors that may have been encountered by the server while processing a
request. For any request, a server MAY return an ErrorX response in place of
the normal response ({\it e.g.}, instead of the \DDX).

The \ErrorX object MUST contain:

\begin{description}
\item[Offending request information] This is the complete URL, including
  payload (\CE), or the POST data in effect at the time. The intent is that
  there should be enough information to reproduce the error.
\item[Text message] A description of the problem.
\end{description}

The \ErrorX object is an XML document used to signal a \DAP client that
the server has encountered an error of some kind.  

\subsection{\ErrorX XML Elements}

An \ErrorX object can contain the following XML elements.

% - # - # - # - # - # - # - # - # - # - # - # - # - # - # - # - # - # - #
\ELEMENT{Error}

Describes the type of error encountered.  The element MUST contain a
short text description with the \attribute{description} attribute, OR
a longer description enclosed.

\paragraph{Element Attributes} 
\begin{description}
  \ATTRIBUTE{code} A number, from a set of well know error
  numbers, associated with this error. See
  Appendix~\Sectionref{sec-error-codes} for a list error numbers and their
  meanings. [required]
\end{description}

\paragraph{Child Elements}
\begin{description}
  \item \element{request} [Required] Contains the Base URL given that
  triggered the error.
  
  \item \element{description} [Optional] Contains a short description of the
  error condition.
  
  \item \element{constraint} [Required if a constraint expression is presnet]
  Contains the constraint condition of the request that triggered the error.
  This is a string containing an index value corresponding to the index value
  of the \element{Constraint} element.\footnote{What is the `index value'
  here? jhrg 12/17/03}
\end{description}

\paragraph{Child Element Syntax}

\begin{itemize}
\item One \element{request} element; followed by
\item Zero or One \element{description} element; followed by
\item Zero or One \element{constraint} element
\end{itemize}


\examplelabel{ex-errorx}
\begin{vcode}{t}
<Error code="404"> 
  <description>Not found</description>
  <request>http://dods.org/data.nc</request>
  <constraint>\emph{huh? What is the index element stuff?}</constraint>
</Error>
\end{vcode}

\subsection{\CapX}
\label{sec-capx}

\note{We're going to be careful about the name `Capabilities' since OpenGIS
  may have trademarked that. 10/21/03 jhrg}

A \DAP server MUST be equipped to respond to a client request for an XML
document describing the characteristics and capabilities of that server.

The \CapX MUST contain information about the \DAP version. It MAY contain
software implementation version information. The \CapX MUST contain a
description of ANY \CE function which is intended to be publically accessible
(servers are free to include \CE functions for internal/experimental use and
not document them in the Server Capabilities Document).

\subsection{\CapX\ XML Elements}

The XML syntax of the returned capabilities document is as follows:

% - # - # - # - # - # - # - # - # - # - # - # - # - # - # - # - # - # - #
\ELEMENT{Description}

This element MUST include documentation too long to include as an
attribute to the \element{Function} element.

\paragraph{Element Attributes} None.

\paragraph{Child Elements} None.

% - # - # - # - # - # - # - # - # - # - # - # - # - # - # - # - # - # - #
\ELEMENT{Function}

The \element{Function} declaration MUST identify a function a client program
can use in a \CE. The server MUST identify any part of the \CE that looks
like \lit{function(arg1,arg2,arg3)} as a function. If a function has no
parameters, an empty set of parentheses MUST be included.

If the \element{Function} element has parameters, their declarations
MUST be contained in its declaration.  There SHOULD also be a text
description of the Function.  Short descriptions MAY be included as a
Function attribute, while longer ones MAY be included in a
\element{Description} element in the \element{Function} body.

\paragraph{Element Attributes} 
\begin{description}
  \ATTRIBUTE{name} [Required] A string identifying the name of the function.
  You should try to pick a name unlikely to cause confusion. The best idea is
  to pick a brief acronym with which to identify your server or project, and
  prefix all function names with those letters. That is, don't call your new
  function \lit{exp()}. Instead call it something like \lit{GDSexp()}.

  \ATTRIBUTE{type} [Required] The type of data returned by the function. This
  MUST be one of the \DAP data types (\Sectionref{sec-variables}) or one of
  the special words `AtomicType' or `AnyType'. Functions
  to be used in the selection clause of a \CE should return \type{Boolean}.
\end{description}

\paragraph{Child Elements}
\begin{description}
  \item \element{Description} [Optional] Used to provide a detailed 
  description of the function.
  \item \element{Parameter} [Optional] Used to describe an input parameter for
  the function.
\end{description}

\paragraph{Child Element Syntax}

\begin{itemize}
\item Zero or one \Description; followed by
\item Zero or more \Parameters
\end{itemize}

% - # - # - # - # - # - # - # - # - # - # - # - # - # - # - # - # - # - #
\ELEMENT{Parameter}

Use the \element{Parameter} element to list the formal parameters (in order)
needed by this function.

\paragraph{Element Attributes} 
\begin{description}
\ATTRIBUTE{name} [Optional] A name by which to refer to a formal parameter.
This is a documentation convenience, and there is no default value.

\ATTRIBUTE{type} [Required] The type of the parameter. This type MUST be one
of the \DAP data type names (\Sectionref{sec-variables}). Note that the value
of the \lit{type} attribute is just the name of the type ({\it e.g.},
\lit{Grid}, \lit{Array}, {\it et c.}) and does not include information about
the size of an Array or the names of the fileds in a Structure. Because it
simplies the interface to a function, it is possible to use the special words
\lit{SimpleType} and \lit{AnyType} in place of one of the \DAP type names.
The word \lit{SimpleType} means that the actual parameter MAY be of any of
the atomic types. The word \lit{AnyType} means that the actual parameter MAY
be of any of the \DAP data types.

\ATTRIBUTE{repeats} [Optional] If present and assign the value \lit{true},
this attribute indicates that the parameter MAY appear once or any number of
times greater than once. If not present the parameter MUST appear once.

\ATTRIBUTE{description} [Optional] A brief text description of this parameter.
This is a documentation convenience, and there is no default value.
\end{description}

\paragraph{Child Elements} None.


\examplelabel{ex-function1}
\begin{vcode}{it}
<Function name="average" type="Float64">
  <description>This function averages a group of Float64
  values.</descritpion>
  <parameter type="Float64" repeats="true"/>
</Function> 
\end{vcode}

\examplelabel{ex-function2}
\begin{vcode}{it}
<Function name="mean" type="Float64"> <description>This function averages a
  group of values; it is more liberal about its parameter types than average.
  It will return an ErrorX if called with non-numeric actual
  parameters.</descritpion> <parameter type="SimpleType" repeats="true"/>
  </Function>
\end{vcode}

% - # - # - # - # - # - # - # - # - # - # - # - # - # - # - # - # - # - #
\ELEMENT{Version}

The \element{Version} element indicates the version of some entity
associated with the server in question. The \CapX response MUST contain the
version of the \DAP; other version elements MAY be used to include the
versions of other things (such as implementation software). Note that the
versions of specific data sources SHOULD be included in the \Attributes for
those data sources, not in the version information returned here.

\paragraph{Element Attributes} 
\begin{description}
  \ATTRIBUTE{value} [Required] A string containing the version number
  or name. 
  
  \ATTRIBUTE{entity} [Required] A string containing the name of the
  specification/standard, software or data source to which this
  \element{Version} element refers. The string \lit{DAP} is reserved to refer
  to the \DAP and indicates the version of the protocol, not the
  implementation. Any other value is at the discretion of the software
  implementor (for software version) or server administrator. 
\end{description}

\paragraph{Child Elements} None.

\examplelabel{ex-version}
\begin{vcode}{it}
<Version entity="DAP" value="4.0"/>
<Version entity="OPeNDAP_netCDF_server" value="6.4"/>
\end{vcode}

%%%%%%%%%%%%%%%%%%%%%%%%% Constraint %%%%%%%%%%%%%%%%%%%%%%%%%
\section{Constraint}
\label{sec-xml-ce}

The \DAP uses a single \new{request document} to supply information to a
server about a request for data. Recall in \Sectionref{sec-ce} that the \DAP
uses \CEs to limit data accessed to specific variables, or parts of
variables, in a \Dataset. 

Each constraint is broken into two clauses, the \new{projection clause} and
the \new{selection clause}. Each of the clauses is further broken down into
sub-clauses. A projection clause is simply a collection of one or more
\Project elements, and a selection clause consists of one or more \Select
elements. If no \Project elements are present (the projection clause is
omitted), the server MUST return all the variables in the data source. If the
selection clause is omitted, all instances (values) of the variables
specified in the projection are returned.

It is reasonable for the \Constraint element to be empty, as this will cause
the server to return the complete description (\DDX) for the \Dataset. From
this the user would typically form the \new{projection clauses} and
\new{selection clauses} to constrain the information. A second request would
then follow, this time with a more complex \Constraint.

A client can ask the 
server to omit the \Attribute information from a \DDX by adding a 
\NoAttributes element to the \Constraint. See \Sectionref{sec-ce-examples} 
for examples.

\subsection{Constraint XML Elements}
\label{sec-xml-ce-elements}

Following is a description of each element used in a \CE.

% - # - # - # - # - # - # - # - # - # - # - # - # - # - # - # - # - # - #
\ELEMENT{Constraint}

This element contains the two component clauses of the \CE: the
projection and the selection.  The projection clause specifies which
variables are to be returned, and the selection clause helps select
among the variable values.  See page~\pageref{sec-ce-clauses}.\\[2mm]

\paragraph{Element Attributes} 
\begin{description}
\ATTRIBUTE{name}  [Required] An identifying string for the constraint expression.
\end{description}

\paragraph{Child Elements} 
\begin{description}

  \item \element{NoAttributes} [Optional] Informs server to strip all
  \Attribute information from the returned \DDX.
  
  \item \element{Project} [Optional] Identifies a variable to be returned the
  client. See \Sectionref{sec-xml-project}
  
  \item \element{Select} [Optional] Specifies what conditions need to be met
  for an instance of the set of projected variables to be returned to the
  client. See \Sectionref{sec-xml-select}
  
\end{description}

\textbf {Child element syntax:} 

\begin{itemize}
\item Zero or one \NoAttributes element; followed by
\item Zero or more \Project elements; followed by
\item Zero or more \Select elements
\end{itemize}

See \Sectionref{sec-ce-examples} for examples of complete \CEs.

% - # - # - # - # - # - # - # - # - # - # - # - # - # - # - # - # - # - #
\ELEMENT{Hyperslab}
\label{sec-xml-hyperslab}

Use this element to measure off a rectangular subsection (sometimes called a
\new{hyperslab}) of a \Grid, \Array, or \Sequence variable. See
\Sectionref{sec-ce}. There MUST be only one \element{Hyperslab} element for
each dimension of the \Grid. If a \element{Hyperslab} element is missing,
that dimension will be returned whole. See \exampleref{ce-2}.

A single \element{Hyperslab} element MAY also be used to subsample a
\Sequence.

\paragraph{Element Attributes} 
\begin{description}
\ATTRIBUTE{dimension} [Optional] The name of the dimension to sample with the
parameters given in this \element{Hyperslab} element. This name MUST only
apply to \Grid variables or \Arrays with named dimensions. If the name is
omitted, the order of the \element{Hyperslab} elements MUST be the same as
the order of the variable dimension declarations.

\ATTRIBUTE{start} [Optional] The first index to return. If omitted, zero MUST
be assumed.

\ATTRIBUTE{stop} [Optional] The last index to return. If omitted, the
dimension maximum MUST be assumed.

\ATTRIBUTE{stride} [Optional] Use the \attribute{stride} to skip \Grid rows
(or columns or hyperslabs). If omitted, a value of one MUST be assumed. If
present, \attribute{stride} causes sampling to begin at \attribute{start} (or
zero if there is no \attribute{start} present) and then skip to $start +
stride$, $start + 2*stride$, et cetera.
\end{description}

\paragraph{Child Elements} None.

\examplelabel{ce-hyperslab}
\begin{vcode}{it}
<Constraint>
    <Project variable="/temp">
        <Hyperslab dimension="time" start="1" stop="100" stride="5"/>
        <Hyperslab dimension="depth" start="20" stop="40"/>
    </Project>
</Constraint>
\end{vcode}


% - # - # - # - # - # - # - # - # - # - # - # - # - # - # - # - # - # - #
\ELEMENT{NoAttributes}
\label{sec-xml-no-attr}

This element is used to eliminate the \Attribute content from the returned 
DDX. This is intended to be used by clients making multiple requests of the
same \Dataset and thus do not need to have the \Attribute elements sent 
multiple times.

\paragraph{Element Attributes} None.
\paragraph{Child Elements} None.

\examplelabel{ce-no-attrs}

This \CE will return a \DDX of the entire \Dataset minus any \Attribute information.

\begin{vcode}{it}
<Constraint>
    < NoAttributes />
</Constraint>
\end{vcode}


% - # - # - # - # - # - # - # - # - # - # - # - # - # - # - # - # - # - #
\ELEMENT{Project}
\label{sec-xml-project}

This element identifies a variable to be returned to the client. This element
MUST be EITHER a variable in the \Dataset, OR a function, which, when
evaluated, returns a value which is then returned by the server as if it were
data. Projection functions MUST be identified in the server's \CapX.

Contains: Zero or more \element{Hyperslab} elements.

A \element{Project} element MUST have AT LEAST one of either a
\attribute{variable} OR a \attribute{function} BUT NOT BOTH.

\paragraph{Element Attributes}
\begin{description}
  \ATTRIBUTE{function} [Optional] A function invocation. The function's
  return value (which can take any \DAP\ data type) is returned to the
  client. \textbf{Note:} When variables are passed to a function, the
  variables MUST be passed using their \FQN. See \Sectionref{sec-FQN} for
  more on \FQNs.
  
  \ATTRIBUTE{variable} [Required] The \FQN of the variable to be returned. See
  \Sectionref{sec-FQN} for more on \FQNs.
  
\end{description}

\paragraph{Child Elements}
\begin{description}
  \item \element{Hyperslab}  [Optional] Describes a sub-sample of a \Grid, \Array, 
  or \Sequence variable. 
  See \Sectionref{sec-xml-hyperslab}
  
\end{description}

\paragraph{Child Element Syntax}

\begin{itemize}
\item Zero or more \Hyperslab elements
\end{itemize}

% - # - # - # - # - # - # - # - # - # - # - # - # - # - # - # - # - # - #
\ELEMENT{Select}
\label{sec-xml-select}
Use this element to define the condition under which a \Sequence instance or
a \Grid element is to be returned. A \element{Select} element specifies a
relational operation and two operands to compare with it. See
\exampleref{ce-1} and \exampleref{ce-3} to see how to select from a \Sequence
and a \Grid, respectively. Note that the \element{Select} MUST explicitly
state to which projected variable it applies (using the \lit{target}
attribute).

%% To accomplish a comparison between two variables, or between two
%% functions, a server should provide another function to do the
%% comparison internally.

%% A \element{Select} element that references a \Dataset variable may
%% contain a set of \element{Hyperslab} elements with the limitation that
%% the \element{Hyperslab} elements MUST select a single scalar value
%% from the variable.  That is, a selection clause like the following is
%% permissible only if \lit{sst} is a \Sequence or a one-dimensional
%% \Array or \Grid.

%% \begin{vcode}{it}
%% <Select condition="sst>25.0">
%%    <Hyperslab start="3" stop="3">
%% </Select>
%% \end{vcode}

Contains: Zero or more
\element{Hyperslab} elements.

\paragraph{Element Attributes}
\begin{description}
  \ATTRIBUTE{condition} [Required] A relational expression or a function call.
  In the case of a function call, the function MUST return a \Boolean type.
  In the case of a relational expression, the syntax MUST be: $operand_{1}
  operator operand_{2}$ where $operator$ is one of: 
  $\verb+=, !=, <, <=, >, >=, =~+$
  and are defined in \Tableref{tab-ce-relations}. The operands $operand_{1}$
  and $operand_{2}$ maybe variables of any of the atomic types, including
  fields, constants, or function calls which return atomic types. \textbf{Note:}
  When variables are passed to a function it MUST be done so using their \FQN.
  See \Sectionref{sec-FQN} for more on \FQNs.
  
  \ATTRIBUTE{target} [Required] The \FQN of the variable with which this 
  selection criterion is to be evaluated. See \Sectionref{sec-FQN} 
  for more on \FQNs.
  
\end{description}

\paragraph{Child Elements}
\begin{description}
  \item \element{Hyperslab}  [Optional] Describes a sub-sample of a \Grid, 
  \Array, or \Sequence variable. 
  See \Sectionref{sec-xml-hyperslab}
  
\end{description}
  
\paragraph{Child Element Syntax}

\begin{itemize}
\item Zero or more \Hyperslab elements
\end{itemize}

\subsection{Constraint examples}
\label{sec-ce-examples}

\examplelabel{ce-1}

This \CE is a simple request for temperature and
salinity from a \Sequence \Dataset.  This will return a \Sequence
containing temperature and salinity pairs where all the salinity
values are above 34.0.

\begin{vcode}{it}
<Constraint>
    <Project variable="/sample/temp"/>
    <Project variable="/sample/salt"/>
    <Select condition="/sample/salt>34.0" target="sample"/>
</Constraint>
\end{vcode}

\examplelabel{ce-2}

For sub-sampling gridded data, use the \element{Project} element to
elaborate a projection clause.  This \CE 
subsamples a \Grid, and returns a smaller \Grid, where the \lit{lat}
dimension has rows 1,3,5,7 and 9 of the original \Grid, and the
\lit{lon} dimension has all the columns from 20 to 40 from the
original.

\begin{vcode}{it}
<Constraint>
    <Project variable="/sst">
        <Hyperslab dimension="lat" start="1" stop="10" stride="2"/>
        <Hyperslab dimension="lon" start="20" stop="40"/>
    </Project>
</Constraint>
\end{vcode}


\examplelabel{ce-3}

It is also possible to select from a \Grid, based on the values of the
map arrays.  This \CE shows the selection of a
\Grid called \lit{sst}.  Assuming \lit{sst} is a two-dimensional array
with two one-dimensional maps, this constraint will return a \Grid
where all the \lit{lat} values are above 24.0 and all the \lit{lon}
values are below -50.0. In addition this \CE requests that no \Attribute
information be sent in the returned \DDX.

\begin{vcode}{it}
<Constraint>
    <NoAttributes />
    <Project variable="/sst"/>
    <Select condition="/sst/lat>24.5" target="sst"/>
    <Select condition="/sst/lon<-50.5" target="sst"/>
</Constraint>
\end{vcode}

\examplelabel{ce-4}

This \CE exercises the \Project \Function \lit{make-sst}.

\begin{vcode}{it}
<Constraint>
  <Project function="make-sst(/raw-count, 223)"/>
</Constraint>
\end{vcode}

%    <Argument variable="raw-count"/>
%    <Argument constant="223"/>
%  </Project>
%%   <Select condition="make-sst(raw-count, 223)>25.0">
%% %  <Select function="make-sst" condition="make-sst(raw-count, 223)>25.0">
%% %    <Argument variable="raw-count"/>
%% %    <Argument constant="223"/>
%%   </Select>

%\printgloss{dods-glossary}

\bibliographystyle{plain}
\T\addcontentsline{toc}{section}{References}
\T\raggedright
\bibliography{../../../boiler/dods}


%<<<<<<<<<<<<<<<<<<<<<<<< APPENDIX >>>>>>>>>>>>>>>>>>>>>>>>>>
\appendix

%%%%%%%%%%%%%%%%%%%%%%%% XML Schema %%%%%%%%%%%%%%%%%%%%%%%%%
\section{XML Schema}
\label{sec-xml-schema}


The XML schema has not been fully formalized and will appear 
here (in this section) at later point in this document's development.


%This is the PROTOTYPE XML schema for the \DDX object.

%\begin{verbatim}
%<?xml version="1.0" encoding="UTF-8"?>
%<xs:schema targetNamespace="http://www.dods.org/ns/DODS" 
%           xmlns:xs="http://www.w3.org/2001/XMLSchema" 
%           xmlns:dods="http://www.dods.org/ns/DODS" 
%           xmlns="http://www.dods.org/ns/DODS" 
%           elementFormDefault="qualified" 
%           attributeFormDefault="unqualified">

% <xs:element name="Dataset" type="DODS_Dataset"/>
% <xs:element name="Map" type="Array"/>
% <xs:element name="Attribute" type="DASAttribute"/>
% <xs:element name="Alias" type="DASAlias"/>
% <xs:element name="Byte" type="BaseType"/>
% <xs:element name="Int16" type="BaseType"/>
% <xs:element name="UInt16" type="BaseType"/>
% <xs:element name="Int32" type="BaseType"/>
% <xs:element name="UInt32" type="BaseType"/>
% <xs:element name="Float32" type="BaseType"/>
% <xs:element name="Float64" type="BaseType"/>
% <xs:element name="String" type="BaseType"/>
% <xs:element name="Url" type="BaseType"/>
% <xs:element name="Array" type="Array"/>
% <xs:element name="Grid" type="Grid"/>
% <xs:element name="Structure" type="Structure"/>
% <xs:element name="Sequence" type="Sequence"/>

% <xs:element name="dodsBLOB" type="BLOB"/>

% <xs:group name="allDodsTypes">
%  <xs:annotation>
%   <xs:documentation>Reusable Content Model for 
%                     Complex DODS types</xs:documentation>
%  </xs:annotation>
%  <xs:choice>
%   <xs:element ref="Byte" minOccurs="0" maxOccurs="unbounded"/>
%   <xs:element ref="Int16" minOccurs="0" maxOccurs="unbounded"/>
%   <xs:element ref="UInt16" minOccurs="0" maxOccurs="unbounded"/>
%   <xs:element ref="Int32" minOccurs="0" maxOccurs="unbounded"/>
%   <xs:element ref="UInt32" minOccurs="0" maxOccurs="unbounded"/>
%   <xs:element ref="Float32" minOccurs="0" maxOccurs="unbounded"/>
%   <xs:element ref="Float64" minOccurs="0" maxOccurs="unbounded"/>
%   <xs:element ref="String" minOccurs="0" maxOccurs="unbounded"/>
%   <xs:element ref="Url" minOccurs="0" maxOccurs="unbounded"/>
%   <xs:element ref="Array" minOccurs="0" maxOccurs="unbounded"/>
%   <xs:element ref="Grid" minOccurs="0" maxOccurs="unbounded"/>
%   <xs:element ref="Structure" minOccurs="0" maxOccurs="unbounded"/>
%   <xs:element ref="Sequence" minOccurs="0" maxOccurs="unbounded"/>
%  </xs:choice>
% </xs:group>

% <xs:complexType name="DODS_Dataset">
%  <xs:annotation>
%   <xs:documentation>This is the XML representation of a DODS DDS object.</xs:documentation>
%  </xs:annotation>
%  <xs:sequence>
%   <xs:element ref="Attribute" minOccurs="0" maxOccurs="unbounded"/>
%   <xs:element ref="Alias" minOccurs="0" maxOccurs="unbounded"/>
%   <xs:group ref="allDodsTypes" minOccurs="0" maxOccurs="unbounded"/>
%   <xs:element ref="dodsBLOB" minOccurs="0"/>
%  </xs:sequence>
%  <xs:attribute name="name" type="xs:string" use="required"/>
% </xs:complexType>

% <xs:complexType name="DASAttribute">
%  <xs:annotation>
%   <xs:documentation>DAS Attribute Type</xs:documentation>
%  </xs:annotation>
%  <xs:choice>
%   <xs:sequence>
%    <xs:element name="value" type="xs:string" maxOccurs="unbounded"/>
%   </xs:sequence>
%   <xs:sequence>
%    <xs:element ref="Attribute" minOccurs="0" maxOccurs="unbounded"/>
%    <xs:element ref="Alias" minOccurs="0" maxOccurs="unbounded"/>
%   </xs:sequence>
%  </xs:choice>
%  <xs:attribute name="name" type="xs:string" use="required"/>
%  <xs:attribute name="type" type="dods:DataType" use="required"/>
% </xs:complexType>

% <xs:complexType name="DASAlias">
%  <xs:attribute name="name" type="xs:string" use="required"/>
%  <xs:attribute name="Attribute" type="xs:string" use="required"/>
% </xs:complexType>

% <!-- possible data types for Variable, Attribute objects -->
% <xs:simpleType name="DataType">
%  <xs:restriction base="xs:string">
%   <xs:enumeration value="Byte"/>
%   <xs:enumeration value="Int16"/>
%   <xs:enumeration value="UInt16"/>
%   <xs:enumeration value="Int32"/>
%   <xs:enumeration value="UInt32"/>
%   <xs:enumeration value="Float32"/>
%   <xs:enumeration value="Float64"/>
%   <xs:enumeration value="String"/>
%   <xs:enumeration value="URL"/>
%   <xs:enumeration value="Container"/>
%  </xs:restriction>
% </xs:simpleType>

% <xs:complexType name="BaseType">
%  <xs:annotation>
%   <xs:documentation>DODS Base Type</xs:documentation>
%  </xs:annotation>
%  <xs:sequence>
%   <xs:element ref="Attribute" minOccurs="0" maxOccurs="unbounded"/>
%   <xs:element ref="Alias" minOccurs="0" maxOccurs="unbounded"/>
%  </xs:sequence>
%  <xs:attribute name="name" type="xs:string"/>
% </xs:complexType>

% <xs:complexType name="Array">
%  <xs:complexContent>
%   <xs:extension base="BaseType">
%    <xs:sequence>
%     <xs:choice minOccurs="1" maxOccurs="1">
%      <xs:element ref="Byte"/>
%      <xs:element ref="Int16"/>
%      <xs:element ref="UInt16"/>
%      <xs:element ref="Int32"/>
%      <xs:element ref="UInt32"/>
%      <xs:element ref="Float32"/>
%      <xs:element ref="Float64"/>
%      <xs:element ref="String"/>
%      <xs:element ref="Url"/>
%      <xs:element ref="Grid"/>
%      <xs:element ref="Structure"/>
%      <xs:element ref="Sequence"/>
%     </xs:choice>
%     <xs:element name="dimension" type="dods:ArrayDimension" minOccurs="1" maxOccurs="unbounded"/>
%    </xs:sequence>
%   </xs:extension>
%  </xs:complexContent>
% </xs:complexType>

% <xs:complexType name="ArrayDimension">
%  <xs:attribute name="name" type="xs:string"/>
%  <xs:attribute name="size" type="xs:integer" use="required"/>
% </xs:complexType>

% <xs:complexType name="Grid">
%  <xs:complexContent>
%   <xs:extension base="dods:BaseType">
%    <xs:sequence>
%     <xs:element ref="Array" minOccurs="1" maxOccurs="1"/>
%     <xs:element ref="Map" minOccurs="1" maxOccurs="unbounded"/>
%    </xs:sequence>
%   </xs:extension>
%  </xs:complexContent>
% </xs:complexType>

% <xs:complexType name="Structure">
%  <xs:complexContent>
%   <xs:extension base="BaseType">
%    <xs:group ref="allDodsTypes" minOccurs="1" maxOccurs="unbounded"/>
%   </xs:extension>
%  </xs:complexContent>
% </xs:complexType>

% <xs:complexType name="Sequence">
%  <xs:complexContent>
%   <xs:extension base="BaseType">
%    <xs:group ref="allDodsTypes" minOccurs="1" maxOccurs="unbounded"/>
%   </xs:extension>
%  </xs:complexContent>
% </xs:complexType>

% <xs:complexType name="BLOB">
%  <xs:attribute name="URL" type="xs:string" use="required"/>
% </xs:complexType>

%</xs:schema>
%\end{verbatim}


%%%%%%%%%%%%%%%%%%%%%%%% Error Codes %%%%%%%%%%%%%%%%%%%%%%%%%
\section{Error Codes}
\label{sec-error-codes}

The error messages and codes issued by a \DAP server are shown in
\Tableref{tab-error-codes}.  They are largely taken from the HTTP
error codes outlined in the HTTP standard.  % Different implementations
%of the \DAP may deliver the messages in different fashions, but the
The code and the title MUST be delivered as shown here. The \emph
{description} MAY be altered if you want to translate it into another
language.

\begin{longtable}{p{0.5in}p{1in}p{3in}}
\caption{Error Codes\label{tab-error-codes}} 
\\ 
\textbf{Code} & \textbf{Title} & \textbf{Description} \\ \hline
\endfirsthead
\caption[]{\DAP Error Codes} 
\\ 
\textbf{Code} & \textbf{Title} & \textbf{Description} \\ \hline
\endhead
\hline
\endfoot
400 & Bad Request &
The URL could not be resolved.  The host name is probably incorrect. 
\\ \hline
401 & Unauthorized &
The resource requested is not available without authentication, and
yours has failed.
\\ \hline
402 & Payment Required &
\\ \hline
403 &
Forbidden & 
\\ \hline
404 &
Not Found &
The data file specified in the request is not on the
specified server.
\\ \hline
% 407 &
% Proxy Authentication Required &
% \\ \hline
408 &
Request Timeout &
\\ \hline
409 &
Conflict &
 \\ \hline
410 &
Gone &
 \\ \hline
411 &
Length Required &
 \\ \hline
412 &
Precondition Failed &
 \\ \hline
413 &
Request Entity Too Large &
 \\ \hline
414 &
Request-URI Too Long &
 \\ \hline
%% 415 &
%% Unsupported Media Type &
%%  \\ \hline
%% 416 &
%% Requested Range Not Satisfiable &
%%  \\ \hline
%% 417 &
%% Expectation Failed &
%%  \\ \hline
500  &
Internal Server Error &
 \\ \hline
501  &
Not Implemented &
 \\ \hline
502  &
Bad Gateway &
 \\ \hline
503  &
Service Unavailable &
 \\ \hline
504  &
Gateway Timeout &
 \\ \hline
\end{longtable}



%%%%%%%%%%%%%%%%%%% Notational Conventions %%%%%%%%%%%%%%%%%%%
%\section{Notational Conventions and Generic Grammar}
%\label{app-grammar}
%
% This was taken verbatim from rfc2616. The original section title was
% `Notational Conventions and Generic Grammar' 3/21/2001 jhrg

\subsection{Augmented BNF}
All of the mechanisms specified in this document are described in both prose
and an augmented Backus-Naur Form (BNF) similar to that used by RFC
822~\cite{rfc822}. Implementors will need to be familiar with the notation in
order to understand this specification. The augmented BNF includes the
following constructs:

\begin{description}
  
\item [\texttt{name = definition}] The name of a rule is simply the name
  itself (without any enclosing \texttt{"$<$"} and \texttt{"$>$"}) and is
  separated from its definition by the equal \texttt{"="} character. White
  space is only significant in that indentation of continuation lines is used
  to indicate a rule definition that spans more than one line. Certain basic
  rules are in uppercase, such as SP, LWS, HT, CRLF, DIGIT, ALPHA, etc. Angle
  brackets are used within definitions whenever their presence will
  facilitate discerning the use of rule names.
  
\item [\texttt{"literal"}] Quotation marks surround literal text. Unless
  stated otherwise, the text is case-insensitive.
      
\item [\texttt{rule1 | rule2}] Elements separated by a bar (\texttt{"|"}) are
  alternatives, e.g., \texttt{"yes | no"} will accept \texttt{yes} or
  \texttt{no}.
  
\item [\texttt{(rule1 rule2)}] Elements enclosed in parentheses are treated
  as a single element.  Thus, \texttt{"(elem (foo | bar) elem)"} allows the
  token sequences \texttt{"elem foo elem"} and \texttt{"elem bar elem"}.
  
\item [\texttt{*rule}] The character \texttt{"*"} preceding an element
  indicates repetition. The full form is \texttt{"$<$n$>$*$<$m$>$element"}
  indicating at least \texttt{$<$n$>$} and at most \texttt{$<$m$>$}
  occurrences of element. Default values are 0 and infinity so that
  \texttt{"*(element)"} allows any number, including zero;
  \texttt{"1*element"} requires at least one; and \texttt{"1*2element"}
  allows one or two.
  
\item [\texttt{[rule]}] Square brackets enclose optional elements;
  \texttt{"[foo bar]"} is equivalent to \texttt{"*1(foo bar)"}.
  
\item [\texttt{N rule}] Specific repetition: \texttt{"$<$n$>$(element)"} is
  equivalent to \texttt{"$<$n>*$<$n$>$(element)"}; that is, exactly
  \texttt{$<$n$>$} occurrences of (element).  Thus 2DIGIT is a 2-digit
  number, and 3ALPHA is a string of three alphabetic characters.
  
\item [\texttt{\#rule}] A construct \texttt{"\#"} is defined, similar to
  \texttt{"*"}, for defining lists of elements. The full form is
  \texttt{"$<$n$>$\#$<$m$>$element"} indicating at least \texttt{$<$n$>$} and
  at most \texttt{$<$m$>$} elements, each separated by one or more commas
  (\texttt{","}) and OPTIONAL linear white space (LWS). This makes the usual
  form of lists very easy; a rule such as \texttt{( *LWS element *( *LWS ","
    *LWS element ))} can be shown as \texttt{1\#element} Wherever this
  construct is used, null elements are allowed, but do not contribute to the
  count of elements present. That is, \texttt{"(element), , (element) "} is
  permitted, but counts as only two elements. Therefore, where at least one
  element is required, at least one non-null element MUST be present. Default
  values are 0 and infinity so that \texttt{"\#element"} allows any number,
  including zero; \texttt{"1\#element"} requires at least one; and
  \texttt{"1\#2element"} allows one or two.
  
\item [\texttt{;} comment] A semi-colon, set off some distance to the right
  of rule text, starts a comment that continues to the end of line. This is a
  simple way of including useful notes in parallel with the specifications.
  
\item [implied \texttt{*LWS}] The grammar described by this specification is
  word-based. Except where noted otherwise, linear white space (LWS) can be
  included between any two adjacent words (token or quoted-string), and
  between adjacent words and separators, without changing the interpretation
  of a field. At least one delimiter (LWS and/or separators) MUST exist
  between any two tokens (for the definition of "token" below), since they
  would otherwise be interpreted as a single token.
\end{description}

\subsection{Basic Rules}

   The following rules are used throughout this specification to
   describe basic parsing constructs. The US-ASCII coded character set
   is defined by ANSI X3.4-1986~\cite{ANSI:US-ASCII}.

\begin{vcode}{it}
       OCTET          = <any 8-bit sequence of data>
       CHAR           = <any US-ASCII character (octets 0 - 127)>
       UPALPHA        = <any US-ASCII uppercase letter "A".."Z">
       LOALPHA        = <any US-ASCII lowercase letter "a".."z">
       ALPHA          = UPALPHA | LOALPHA
       DIGIT          = <any US-ASCII digit "0".."9">
       CTL            = <any US-ASCII control character
                        (octets 0 - 31) and DEL (127)>
       CR             = <US-ASCII CR, carriage return (13)>
       LF             = <US-ASCII LF, linefeed (10)>
       SP             = <US-ASCII SP, space (32)>
       HT             = <US-ASCII HT, horizontal-tab (9)>
       <">            = <US-ASCII double-quote mark (34)>
\end{vcode}

HTTP/1.1 defines the sequence CR LF as the end-of-line marker for all
protocol elements except the entity-body (see Appendix 19.3 of RFC
2616[9] for tolerant applications). The end-of-line marker within an
entity-body is defined by its associated media type, as described in
Section 3.7 of RFC 2616[9].

% My original text:
% except the entity-body (see appendix
%    19.3\footnote{In RFC 2616\cite{rfc2616}.} for
%    tolerant applications). The end-of-line marker within an entity-body
%    is defined by its associated media type, as described in section
%    3.7.\footnote{In RFC 2616\cite{rfc2616}.}
% The replacement text above was suggested by Allan Doyle, 20 April
% 2005.

\begin{vcode}{it}
       CRLF           = CR LF
\end{vcode}

   HTTP/1.1 header field values can be folded onto multiple lines if the
   continuation line begins with a space or horizontal tab. All linear
   white space, including folding, has the same semantics as SP. A
   recipient MAY replace any linear white space with a single SP before
   interpreting the field value or forwarding the message downstream.

\begin{vcode}{it}
       LWS            = [CRLF] 1*( SP | HT )
\end{vcode}

   The TEXT rule is only used for descriptive field contents and values
   that are not intended to be interpreted by the message parser. Words
   of *TEXT MAY contain characters from character sets other than ISO-
   8859-1 [22] only when encoded according to the rules of RFC 2047
   [14].

\begin{vcode}{it}
       TEXT           = <any OCTET except CTLs,
                        but including LWS>
\end{vcode}

   A CRLF is allowed in the definition of TEXT only as part of a header
   field continuation. It is expected that the folding LWS will be
   replaced with a single SP before interpretation of the TEXT value.

   Hexadecimal numeric characters are used in several protocol elements.

\begin{vcode}{it}
       HEX            = "A" | "B" | "C" | "D" | "E" | "F"
                      | "a" | "b" | "c" | "d" | "e" | "f" | DIGIT
\end{vcode}

   Many HTTP/1.1 header field values consist of words separated by LWS
   or special characters. These special characters MUST be in a quoted
   string to be used within a parameter value (as defined in section
   3.6).

\begin{vcode}{it}
       token          = 1*<any CHAR except CTLs or separators>
       separators     = "(" | ")" | "<" | ">" | "@"
                      | "," | ";" | ":" | "\" | <">
                      | "/" | "[" | "]" | "?" | "="
                      | "{" | "}" | SP | HT
\end{vcode}

   Comments can be included in some HTTP header fields by surrounding
   the comment text with parentheses. Comments are only allowed in
   fields containing "comment" as part of their field value definition.
   In all other fields, parentheses are considered part of the field
   value.

\begin{vcode}{it}
       comment        = "(" *( ctext | quoted-pair | comment ) ")"
       ctext          = <any TEXT excluding "(" and ")">
\end{vcode}

   A string of text is parsed as a single word if it is quoted using
   double-quote marks.

\begin{vcode}{it}
       quoted-string  = ( <"> *(qdtext | quoted-pair ) <"> )
       qdtext         = <any TEXT except <">>
\end{vcode}

   The backslash character ("\verb+\+") MAY be used as a single-character
   quoting mechanism only within quoted-string and comment constructs.

\begin{vcode}{it}
       quoted-pair    = "\" CHAR
\end{vcode}

\begin{quote}
This appendix was copied from RFC 2616~\cite{rfc2616}. The copyright
from that document reads:

\begin{quote}
  Copyright (C) The Internet Society (1999).  All Rights Reserved.

   This document and translations of it may be copied and furnished to
   others, and derivative works that comment on or otherwise explain it
   or assist in its implementation may be prepared, copied, published
   and distributed, in whole or in part, without restriction of any
   kind, provided that the above copyright notice and this paragraph are
   included on all such copies and derivative works.  However, this
   document itself may not be modified in any way, such as by removing
   the copyright notice or references to the Internet Society or other
   Internet organizations, except as needed for the purpose of
   developing Internet standards in which case the procedures for
   copyrights defined in the Internet Standards process must be
   followed, or as required to translate it into languages other than
   English.
\end{quote}

\end{quote}





%%%%%%%%%%%%%%%%% Acronyms and Abbreviations %%%%%%%%%%%%%%%%%
%\section{Acronyms and Abbreviations}
%\begin{acronym}
%%
% Make one entry per line, even if they are long lines that wrap and look
% ugly. This makes it simple to sort the list using emacs' sort-lines
% command. 3/27/2000 jhrg
%
% $Id$
\acro{AS}{Aggregation Server}
\acro{BNF} {Backus-Naur Form}
\acro{CE}{Constraint Expression}
\acro{CGI}{Common Gateway Interface}
\acro{COARDS}{Cooperative Ocean/Atmosphere Research Data Service}
\acro{CSV}{Comma Separated Values}
\acro{DAP}{Data Access Protocol}
\acro{DAS}{Dataset Attribute Structure}
\acro{DDS}{Dataset Descriptor Structure}
\acro{DODS}{Distributed Oceanographic Data System}, See the DODS home page: \texttt{http://\-unidata.ucar.edu\-/packages\-/dods/}
\acro{DataDDS}{Data Dataset Descriptor Structure}
\acro{FGDC}{Federal Geographic Data Community}
\acro{HTML}{Hypertext Markup Language}
\acro{HTTP}{HyperText Transfer Protocol}
\acro{MIME}{Multimedia Internet Mail Extensions}
\acro{SRS}{Software Requirements Specification}, See IEEE 830--1998
\acro{URI}{Uniform Resource Identifiers}
\acro{URL}{Uniform Resource Locator}
\acro{W3C}{The World Wide Web Consortium}, See http://www.w3c.org/
\acro{WWW}{The World Wide Web}
\acro{XDR}{External Data Representation}
\acro{XML}{Extensible Markup Language}
%%% Local Variables: 
%%% mode: latex
%%% TeX-master: t
%%% End: 

%\end{acronym}

\end{document}

% \begin{comment}
% \subsection{Text Data}


% \begin{description}
% \item[\Array]
  
%   \Array members are stored in row-major order (rightmost subscript
%   varies fastest).  The items in a row are separated by commas.  Rows
%   are separated from one another by end-of-line characters
%   (\lit{0x0a}) with no comma.  For \Arrays with three dimensions, the
%   two-dimensional \Arrays that make up the third dimension are
%   separated by two EOL characters.  \Arrays with four dimensions
%   separate their three-dimensional components with three EOL
%   characters, and so on.

% \item[\Structure] 
  
%   Members are placed in order of declaration, separated by commas, and
%   enclosed with curly brackets ($\{\}$).

% \item[\Grid] 
  
%   \Grids, which are essentially special cases of the \Structure type,
%   are recorded in the same fashion as \Structures.

% \item[\Sequence] 
  
%   \Sequence data are returned as a series of comma-separated rows, with
%   each \Sequence terminated by an EOL character, and the members
%   separated by commas.  \Sequences must be returned flattened into a
%   single \Sequence.  This means that \Sequences contained within
%   \Sequences lose their definition as \Sequences.  For a \Sequence $S$
%   with member variables $s_0\cdots s_n$, that contains another
%   \Sequence $s_i$, with members $s_{i a}$ and $s_{i b}$the returned
%   format looks like this:

%   \begin{displaymath}
%     \begin{array}{ccccccc}
%       s_{1 1}, & s_{2 1}, & \cdots & s_{i 1 a 1}, s_{i 1 b 1}, & \cdots & s_{n 1} \\
%       s_{1 1}, & s_{2 1}, & \cdots & s_{i 1 a 2}, s_{i 1 b 2}, & \cdots & s_{n 1} \\
%       s_{1 1}, & s_{2 1}, & \cdots & s_{i 1 a 3}, s_{i 1 b 3}, & \cdots & s_{n 1} \\
%       s_{1 1}, & s_{2 1}, & \cdots & s_{i 1 a 4}, s_{i 1 b 4}, & \cdots & s_{n 1} \\
%       \vdots & \vdots & \vdots & \vdots & \vdots & \vdots & \vdots \\
%       s_{1 2}, & s_{2 2}, & \cdots & s_{i 2 a 1}, s_{i 2 b 1}, & \cdots & s_{n 2} \\
%       s_{1 2}, & s_{2 2}, & \cdots & s_{i 2 a 2}, s_{i 2 b 2}, & \cdots & s_{n 2} \\
%       s_{1 2}, & s_{2 2}, & \cdots & s_{i 2 a 3}, s_{i 2 b 3}, & \cdots & s_{n 2} \\
%       s_{1 2}, & s_{2 2}, & \cdots & s_{i 2 a 4}, s_{i 2 b 4}, & \cdots & s_{n 2} \\
%       s_{1 2}, & s_{2 2}, & \cdots & s_{i 2 a 5}, s_{i 2 b 5}, & \cdots & s_{n 2} \\
%       \vdots & \vdots & \vdots & \vdots & \vdots & \vdots & \vdots \\
%     \end{array}
%   \end{displaymath}

%   \Sequences that contain more complex data types may return their data
%   as a series of \Structures, with each \Sequence entry enclosed by
%   curly braces ($\{\}$), and \Sequence members separated by commas.
%   An entry in the above \Sequence would be rendered as follows:

%   \begin{displaymath}
%     \begin{array}{l}
%       \{ s_{1 1}, s_{2 1}, \cdots 
%       \{ s_{i 1 a 1}, s_{i 1 b 1} \} 
%       \{ s_{i 1 a 2}, s_{i 1 b 2} \} 
%       \{ s_{i 1 a 3}, s_{i 1 b 3} \} 
%       \{ s_{i 1 a 4}, s_{i 1 b 4} \}, 
%       \cdots s_{n 1} \}   \\
%       \{ s_{1 2}, s_{2 2}, \cdots 
%       \{ s_{i 2 a 1}, s_{i 2 b 1} \} 
%       \{ s_{i 2 a 2}, s_{i 2 b 2} \} 
%       \{ s_{i 2 a 3}, s_{i 2 b 3}, \} 
%       \cdots s_{n 2} \}   \\
%       \vdots \\
%     \end{array}
%   \end{displaymath}

%   Using this format, EOL characters do not retain their meaning,
%   except to indicate the end of \Array dimensions that may be contained
%   in the \Sequence.

% \end{description}
% \end{comment}
