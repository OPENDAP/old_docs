\subsection{Simple Variables}
\label{simple-variables}

Variables in the \ac{DAP} have two forms. They are either base types or type
constructors. Base type, or simple, variables are similar to predefined
variables in procedural programming languages like C or Fortran (e.g.,
\texttt{int} or \texttt{integer*4}).

\begin{description}
\item [byte] an 8-bit byte;unsigned char in ANSI C\@.
\item [int16] a 16-bit signed integer.
\item [uint16] a 16-bit unsigned integer.
\item [int32] a 32-bit signed integer.
\item [uint32] a 32-bit unsigned integer.
\item [float32] the IEEE 32-bit floating point datatype (ANSI C's
  \texttt{float}). 
\item [float64] the IEEE 64-bit floating point datatype (ANSI C's
  \texttt{double}) .
\item [string] a sequence of bytes terminated by a null character.
\item [URL] represented as a string, but may be dereferenced in a \ac{CE};
  see Section~\ref{sec:ce}.
\end{description}

\begin{ttfamily}
\begin{center}
\begin{tabular}{lll}
simple-decl & = & simple-type id ";" \\
simple-type & = & "byte"| "int16" | "uint16" | "int32" | "uint32" \\
                & & | "float32" | "float64" | "string" | "url" \\
id & = & (ALPHA | "\_" | "\%" | ".") \\
       & & *(ALPHA | DIGIT | "/" | "\_" | "\%" | ".") \\
\end{tabular}
\end{center}
\end{ttfamily}

\subsection{Constructor Variables} 
\label{sec:ctor-vars}

Constructor variables describe the grouping of one or more variables or values
within a data set. These variables are used to describe different types of
relations between the variables that comprise the data set. There are five
classes of type constructor variables defined by the \ac{DAP}: lists, arrays,
structures, sequences, and grids. 

The types are defined in Sections~\ref{sec:list}~to~\ref{sec:grid}.

\subsubsection{List}
\label{sec:list}

 The \textbf{List} type constructor is used to hold lists of 0 or
  more items of one type. Lists of \texttt{byte}, \ldots, \texttt{grid} are
  specified using the keyword \texttt{list} before the variable's class. The
  list can be of any type \textbf{except \texttt{List}}.

\begin{ttfamily}
\begin{center}
\begin{tabular}{lll}
list-decl & = & "list" (simple-decl  | array-decl | structure-decl \\
          & & | sequence-decl | grid-decl) \\
\end{tabular}
\end{center}
\end{ttfamily}

Examples:
\begin{quote}
\begin{vcode}{t}
list int32 heights;
List Float64 x[10][10];
\end{vcode}
\end{quote}

\subsubsection{Array}
An \textbf{Array} is a one dimensional indexed data structure as defined by
ANSI \C. Multidimensional arrays are defined as arrays of arrays. The size of
each array's dimensions must be given. Each dimension of an array may also be
named.

Multi-dimensional arrays are stored in \gloss[word]{row-major} order (as is
the case with ANSI \C or \Cpp). Any array stored in row-major order is stored
so that the last dimension varies most quickly. For example suppose the two
dimensional array \texttt{String letter[2][3]} has two rows of three columns
each and looks like:
\begin{displaymath}
\begin{array}{ccc}
A & B & C \\
D & E & F \\
\end{array}
\end{displaymath}
the values would be stored as $A~B~C~D~E~F$ in memory.

% Define whether multi-dimensional arrays are row-major or not.  Don't
% use the phrase ``row-major'' without also giving an example; people
% don't ever remember which is which.  (Except perhaps for you.)
\begin{ttfamily}
\begin{center}
\begin{tabular}{lll}
array-decl & = & array-types id array-dims ";" \\
array-types & = & simple-decl | structure-decl | sequence-decl | grid-decl \\
array-dims & = & array-dim | array-dim array-dims \\
array-dim & = & "[" [ name "=" ] DIGIT *DIGIT "]" \\
\end{tabular}
\end{center}
\end{ttfamily}

\subsubsection{Structure}
 A structure groups variables so that the collection can be
  manipulated as a single item. The variables can be of any type.

\begin{ttfamily}
\begin{center}
\begin{tabular}{lll}
structure-type & = & "structure" "\{" *structure-types "\}" ";" \\
structure-types & = & simple-type | list-type | array-type \\
                & & | structure-type | sequence-type | grid-type \\
\end{tabular}
\end{center}
\end{ttfamily}

\subsubsection{Sequence}
 A sequence is an ordered set of $N$ variables which has
  several instantiations (or values). Variables in a sequence may be of
  different types.  Each instance of a sequence is one instantiation of the
  variables. Thus a sequence can be represented as:

\begin{displaymath}
\begin{array}{ccc}
  s_{0 0} & \cdots & s_{0 n} \\
  \vdots & \ddots & \vdots \\
  s_{i 0} & \cdots & s_{i n}
\end{array}
\end{displaymath}

\noindent Every instance of sequence $S$ has the same number, order, and
type of variables. Thus in a sequence which contains an array, each instance
of the array MUST be the same size.\footnote{But a sequence may contain a
  list and each instance of the list may have a different number of elements.
  This is because arrays must have their size declared while lists do not.} A
sequence implies that each of the variables is related to each other in some
logical way. A sequence is different from a structure because its constituent
variables have several instances while a structure's variables have only one
instance (or value).

\begin{ttfamily}
\begin{center}
\begin{tabular}{lll}
sequence-decl & = & "sequence" "\{" *sequence-types "\}" ";" \\
sequence-types & = & simple-type | list-type | array-type \\
                & & | structure-type | sequence-type | grid-type \\
\end{tabular}
\end{center}
\end{ttfamily}

\subsubsection{Grid}
\label{sec:grid}
 A grid is an association of an $N$ dimensional array with $N$
  named vectors, each of which has the same number of elements as the
  corresponding dimension of the array. Each vector is used to map indices of
  one of the array's dimensions to a set of values which are normally
  non-integral (e.g., floating point values). The $N$ (map) vectors may be
  different types. Grids are similar to arrays, but add named dimensions and
  maps for each of those dimensions.
  \label{page:grids} 

\begin{ttfamily}
\begin{center}
\begin{tabular}{lll}
grid-decl & = & "grid" "\{" "array:" *array-decl "maps:" *array-decl "\}" ";" \\
\end{tabular}
\end{center}
\end{ttfamily}
