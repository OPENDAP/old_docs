%
% Specification for an HTTP implementation of the DAP
%

\documentclass[justify]{dods-paper}
%\usepackage{acronym}
\usepackage{xspace}
%\usepackage{gloss}
\rcsInfo $Id$

% latex and HTML macros. Some latex commands become nops for HTML. 4/10/2001
% jhrg 
\T\newcommand{\Cpp}{\rm {\small C}\raise.5ex\hbox{\footnotesize++}\xspace}
\T\newcommand{\C}{\rm {\small C}\xspace}
\W\newcommand{\Cpp}{C++}
\W\newcommand{\C}{C\xspace}
\W\newcommand{\cdots}{...}
\W\newcommand{\ddots}{}
\W\newcommand{\vdots}{.}
\W\newcommand{\pm}{+/-}
\W\newcommand{\times}{*}
\W\newcommand{\uppercase}[1]{\textsc{#1}}
\T\newcommand{\qt}{\lit{\char127}}
\W\newcommand{\qt}{"}

\texorhtml{\def\rearrangedate#1/#2/#3#4{\ifcase#2\or January\or February\or
  March\or April\or May\or June\or July\or August\or September\or
  October\or November\or December\fi\ \ifx0#3\relax\else#3\fi#4, #1}
\def\rcsdocumentdate{\expandafter\rearrangedate\rcsInfoDate}}%
{\HlxEval{
(put 'rearrangedate       'hyperlatex 'hyperlatex-ts-rearrange-date)

(defun hyperlatex-ts-rearrange-date ()
  (let ((date-string (hyperlatex-evaluate-string 
                       (hyperlatex-parse-required-argument))))
    (let ((year-string (substring date-string 0 4))
          (month-string (substring date-string 5 7))
          (day-string (substring date-string 8 10))
          (month-list '("January" "February" "March" "April"
                        "May" "June" "July" "August" 
                        "September" "October" "November" "December")))
       (insert
         (concat (elt month-list (1- (string-to-number month-string)))
                 " " (int-to-string (string-to-number day-string))
                 ", " year-string)))))
}
\newcommand{\rcsdocumentdate}{\rearrangedate{\rcsInfoDate}}}

\newcommand{\dapversion}{Version 4.0\xspace}
\newcommand{\opendap}{OPeNDAP\xspace}
\newcommand{\DAP}{DAP\xspace}
\newcommand{\DODS}{DODS\xspace}
\newcommand{\NVODS}{NVODS\xspace}
\newcommand{\CE}{constraint expression\xspace}
\newcommand{\CEs}{constraint expressions\xspace}
\newcommand{\ErrorX}{ErrorX\xspace}
\newcommand{\CapX}{Server Capabilities Document\xspace}
\newcommand{\Blob}{Blob\xspace}
\newcommand{\DDX}{DDX\xspace}
\newcommand{\DAX}{DAX\xspace}
\newcommand{\DDS}{DDS\xspace}
\newcommand{\DAS}{DAS\xspace}
\newcommand{\URI}{URI\xspace}
\newcommand{\DataDDS}{DataDDS\xspace}

\newcommand{\type}[1]{\emph{#1}}
\newcommand{\Alias}{\type{Alias}\xspace}
\newcommand{\Array}{\type{Array}\xspace}
\newcommand{\Attribute}{\type{Attribute}\xspace}
\newcommand{\ProcAttribute}{\type{Processing Attribute}\xspace}
\newcommand{\Map}{\type{Map}\xspace}
\newcommand{\Target}{\type{Target}\xspace}
\newcommand{\FQN}{fully qualified name\xspace}
\newcommand{\Grid}{\type{Grid}\xspace}
\newcommand{\Structure}{\type{Structure}\xspace}
\newcommand{\Dataset}{\type{Dataset}\xspace}
\newcommand{\Sequence}{\type{Sequence}\xspace}
\newcommand{\Container}{\type{Container}\xspace}
\newcommand{\Bi}{\type{Binary Image}\xspace}
\newcommand{\String}{\type{String}\xspace}
\newcommand{\URL}{\type{URL}\xspace}
\newcommand{\Boolean}{\type{Boolean}\xspace}
\newcommand{\Byte}{\type{Byte}\xspace}
\newcommand{\Enum}{\type{Enumeration}\xspace}
\newcommand{\Time}{\type{Time}\xspace}
\newcommand{\Function}{\type{Function}\xspace}
\newcommand{\Description}{\type{Description}\xspace}
\newcommand{\Parameter}{\type{Parameter}\xspace}
\newcommand{\Constraint}{\type{Constraint}\xspace}
\newcommand{\NoAttributes}{\type{NoAttributes}\xspace}
\newcommand{\Project}{\type{Project}\xspace}
\newcommand{\Select}{\type{Select}\xspace}
\newcommand{\Hyperslab}{\type{Hyperslab}\xspace}

\newcommand{\Aliases}{\type{Aliases}\xspace}
\newcommand{\Arrays}{\type{Arrays}\xspace}
\newcommand{\Attributes}{\type{Attributes}\xspace}
\newcommand{\ProcAttributes}{\type{Processing Attributes}\xspace}
\newcommand{\Maps}{\type{Maps}\xspace}
\newcommand{\Targets}{\type{Targets}\xspace}
\newcommand{\FQNs}{fully qualified names\xspace}
\newcommand{\Grids}{\type{Grids}\xspace}
\newcommand{\Structures}{\type{Structures}\xspace}
\newcommand{\Sequences}{\type{Sequences}\xspace}
\newcommand{\Containers}{\type{Containers}\xspace}
\newcommand{\Bis}{\type{Binary Images}\xspace}
\newcommand{\Strings}{\type{Strings}\xspace}
\newcommand{\URLs}{\type{URLs}\xspace}
\newcommand{\Booleans}{\type{Booleans}\xspace}
\newcommand{\Bytes}{\type{Bytes}\xspace}
\newcommand{\Enums}{\type{Enumerations}\xspace}
\newcommand{\Times}{\type{Times}\xspace}
\newcommand{\CSVs}{comma-separated values\xspace}
\newcommand{\Functions}{\type{Functions}\xspace}
\newcommand{\Descriptions}{\type{Descriptions}\xspace}
\newcommand{\Parameters}{\type{Parameters}\xspace}
\newcommand{\Constraints}{\type{Constraints}\xspace}
\newcommand{\Projects}{\type{Projects}\xspace}
\newcommand{\Selects}{\type{Selects}\xspace}
\newcommand{\Hyperslabs}{\type{Hyperslabs}\xspace}

\newcommand{\DIR}{\textbf{Directory}\xspace}
\newcommand{\TEXT}{\textbf{Text}\xspace}
\newcommand{\HTML}{\textbf{HTML}\xspace}
\newcommand{\HELP}{\textbf{Help}\xspace}
\newcommand{\VER}{\textbf{Version}\xspace}
\newcommand{\INFO}{\textbf{Info}\xspace}

%%%%%%%%%%%%%% Web Services Paper
\newcommand{\GetDDX}{\textbf{GetDDX}\xspace}
\newcommand{\GetData}{\textbf{GetData}\xspace}
\newcommand{\GetBlobData}{\textbf{GetBlobData}\xspace}
\newcommand{\GetBlob}{\textbf{GetBlob}\xspace}
\newcommand{\GetDir}{\textbf{GetDir}\xspace}
\newcommand{\GetInfo}{\textbf{GetInfo}\xspace}

\newcommand{\FSs}{\type{Foundation Services}\xspace}
\newcommand{\FS}{\type{Foundation Service}\xspace}
\newcommand{\ODSN}{\type{OPeNDAP}\xspace}

\newcommand{\blobdataelement}{\texttt{BlobData} element\xspace}
\newcommand{\blobelement}{\texttt{Blob} element\xspace}



%\setcounter{secnumdepth}{4}
%\setcounter{tocdepth}{4}

\newcommand{\Tableref}[1]{Table~\ref{#1}}%
\newcommand{\Figureref}[1]{Figure~\ref{#1}}%
\W\begin{iftex}
\newcommand{\Sectionref}[2]{Section~\ref{#1}%
  \ifx#2)%
    \ on page~\pageref{#1}%
  \else% 
    \ (page~\pageref{#1})\xspace%
  \fi\ifx#2\space\ \else #2\fi}%
\W\end{iftex}
\W\newcommand{\Sectionref}[1]{Section~\ref{#1}}
\W\newcommand{\raggedright}{}


%% Conveniences for documenting XML
\newcommand{\tag}[1]{\emph{#1}}
\newcommand{\element}[1]{\link{\tag{#1}}{sec-xml-#1}}
\newcommand{\attribute}[1]{\emph{#1}}
\newcommand{\currentelement}{}
\newcommand{\ELEMENT}[1]{\renewcommand{\currentelement}{#1 element}%
  \subsubsection{#1}\label{sec-xml-#1}\indc{\currentelement}%
  \indc{catalog tag!#1}\indc{aggregation tag!#1}%
  \indc{XML!#1 element}}
\newcommand{\ATTRIBUTE}[1]{\item{\lit{#1}}\indc{\currentelement!#1}
    \indc{#1 attribute!of \currentelement}%
    \indc{XML!#1 attribute}}

% Conveniences for examples
\newcounter{exampleno}
\setcounter{exampleno}{0}
\newcounter{examplerefno}
\setcounter{examplerefno}{0}
\newcommand{\examplelabel}[1]{\refstepcounter{exampleno}\label{#1}%
  \medskip Example \theexampleno :\smallskip}
\newcommand{\exampleref}[1]{\texorhtml{Example~\ref{#1}%
    \refstepcounter{examplerefno}\label{exref\theexamplerefno}%
    % This is a test whether r@exref... is defined.  If not, skip
    % anything with the \pageref macro.
    \catcode`\@=11%
    \expandafter\ifx\csname r@exref\theexamplerefno\endcsname\relax\else%
    \expandafter\ifx\csname r@#1\endcsname\relax\else%
    \bgroup\count100=\pageref{exref\theexamplerefno}%
    \count101=\pageref{#1}\ifnum\count100=\count101\else~%
    on page~\pageref{#1}\fi\egroup\fi\fi\xspace}%
  {\link{Example \ref{#1}}{#1}}}%
\T\setlength{\vcodeindent}{10pt}
\texorhtml{
%% LaTeX version
\newenvironment{textoutput}[1]{\ifx #1\relax%
  \medskip Output:\vspace{-\medskipamount}\else%
  \medskip #1\vspace{-\medskipamount}\fi%
  \begin{list}{}{\setlength{\leftmargin}{\vcodeindent}}\begin{ttfamily}\item}%
 {\end{ttfamily}\end{list}} %
}{%% Hyperlatex version
\newenvironment{textoutput}[1]{\xml{blockquote}Output:\\ \\ \xml{tt}}%
  {\xml{/tt}\xml{/blockquote}}%
\newenvironment{minipage}[4]{}{}
\newenvironment{ttfamily}{\xml{blockquote}\xml{tt}}%
  {\xml{/tt}\xml{/blockquote}}}

\newcommand{\DAPOverviewTitle}{DAP Specification Overview}
\newcommand{\DAPOverview}{\xlink{\textbf{\textit{\DAPOverviewTitle}}}%
  {dap.html}\xspace}

% Old macros; new values
\newcommand{\DAPObjectsTitle}{The Data Access Protocol---DAP 2.0}
\newcommand{\DAPObjects}{\xlink{\textbf{\textit{\DAPObjectsTitle}}}%
  {http://www.opendap.org/pdf/ESE-RFC-004v0.06.pdf}\xspace}
  
\newcommand{\DAPHTTPTitle}{Using DAP 2.0 with HTTP}
\newcommand{\DAPHTTP}{\xlink{\textbf{\textit{\DAPHTTPTitle}}}%
  {daph.html}\xspace}

% New macros.
\newcommand{\DAPDataModelTitle}{DAP Data Model Specification}
\newcommand{\DAPDataModel}{\xlink{\textbf{\textit{\DAPDataModelTitle}}}%
  {dapo.html}\xspace}
  
\newcommand{\DAPWebTitle}{DAP Web Services Specification}
\newcommand{\DAPWeb}{\xlink{\textbf{\textit{\DAPWebTitle}}}%
  {daph.html}\xspace}
  

\newcommand{\DAPASCIITitle}{DAP Formatted Data Specification}
\newcommand{\DAPASCII}{\xlink{\textbf{\textit{\DAPASCIITitle}}}%
  {dapa.html}\xspace}
\newcommand{\DAPHTMLTitle}{DAP HTTP Query Specification}
\newcommand{\DAPHTML}{\xlink{\textbf{\textit{\DAPHTMLTitle}}}%
  {dapm.html}\xspace}

% It probably doesn't matter what we call the macro, but SOAP does not have
% to run over HTTP and I think that's going to be important for other groups.
% 10/27/03 jhrg
\newcommand{\DAPServicesTitle}{DAP HTTP Services Specification}
\newcommand{\DAPServices}{\xlink{\textbf{\textit{\DAPServicesTitle}}}%
  {daps.html}\xspace}

\newcommand{\thirtytwobitlimit}[1]{$4,294,967,296$ #1 ($2^{32}$)}
%%% Local Variables: 
%%% mode: latex
%%% TeX-master: t
%%% End: 


\title{\DAPHTTPTitle\\ DRAFT}
\htmltitle{\DAPHTTPTitle\ -- DRAFT}
\author{James Gallagher\thanks{OPeNDAP, Inc., jgallagher@OPeNDAP.org}
\and Tom Sgouros}
\date{\rcsInfoDate \\ Revision \rcsInfoRevision}
\htmladdress{James Gallagher <jgallagher@OPeNDAP.org>, 
  \rcsInfoDate, Revision: \rcsInfoRevision}
\htmldirectory{html}
\htmlname{daph}

\begin{document}

\maketitle
\T\tableofcontents

\section{Introduction}

\note[10/14/05 jhrg]{This paper was originally part of the DAP4 suite
  of papers. The DAP 2.0 RFC icludes all of the HTTP header info
  because DAP 2.0 is dependent on HTTP. This paper could be reworked
  to be a 'How to write a DAP 2.0 Server' paper by including all of
  the stuff that needs to be done that's is not part of the DAP 2.0
  RFC. For example, ASCII responses, the Info response, \ldots See the
  DAP2 RFC paper's section 7.2 and document all the responses that are
  not listed there.}

The \opendap system may be implemented using a wide variety of
communication protocols. \DAP servers and clients may communicate
using HTTP, FTP, GridFTP, or any other modern socket-based protocol.
No matter what communication protocol is used, all implementations of
the \DAP must adhere to the specification given in \DAPObjects.

This document specifies the behavior of only the HTTP implementation
of the \DAP. That is, \DAP servers and clients that communicate using
HTTP must adhere to the specifications in this document. The OPeNDAP
project supports only this protocol specification. Other protocol
specifications may be published by the groups publishing the first
implementation using that protocol.

In the remainder of this document, we use \DAP to signify the HTTP
implementation of the \DAP. This is strictly a convenience, and is not
meant to imply anything about other possible implementations.

The \DAP works like any other client/server system.  A client sends
requests to a server, and the server responds to those requests with
one of the \DAP objects or with a service response.  The data objects
and the services that are to be supported by all \DAP implementations
are described in the \DAPObjects document.

In the next two sections, we look in detail at the form of
the requests and the responses for the HTTP implementation of the
\DAP. 

\section{Requests}
\label{sec-requests}

A \DAP client sends a request to a server using HTTP. The request
consists of an HTTP GET request method, a \URI~\cite{rfc2396} that
encodes information specific to the \DAP and an HTTP protocol version
number followed by a MIME-like message containing various headers that
further describe the request. In practice, DAP clients use a
third-party library implementation of HTTP/1.1 so the GET request,
\URI and HTTP version information are hidden from the client; it sees
only the \DAP URL and some of the request headers.

\subsection{The URL}
\label{sec-url}

A URL is simply a unique name for some Internet resource.
\Figureref{fig-url-parts} shows the parts of a typical \DAP URL.

\begin{figure}[h]
\texorhtml
{\begin{center}\small
${}\overbrace{\tt http}^{\textnormal{Protocol}}://
\overbrace{\tt dods.gso.uri.edu}^{\textnormal{Machine Name}}/
\overbrace{\tt cgi-bin/nph-nc}^{\textnormal{Server}}/
\overbrace{\tt data}^{\textnormal{Directory}}/
\overbrace{\tt fnoc1.nc}^{\textnormal{Filename}}
\overbrace{\tt .dods}^{\textnormal{URL Suffix}}?
\overbrace{\tt sst[1:5][1:5]}^{\textnormal{Constraint}}$
\end{center}}
{\begin{vcode}{cb}
http://dods.gso.uri.edu/cgi-bin/nph-nc/data/fnoc1.nc.dods?sst[1:5][1:5]
   ^            ^               ^      ^    ^        ^     ^
   |            |               |      |    |        |     |
Protocol        |               |      |    |        |     |
Machine Name-----               |      |    |        |     |
Server---------------------------      |    |        |     |
Directory-------------------------------    |        |     |
Filename-------------------------------------        |     |
URL Suffix--------------------------------------------     |
Constraint expression---------------------------------------
\end{vcode}}
\caption{Parts of a \DAP URL}
\label{fig-url-parts}
\end{figure}

The parts of the URL are:

\begin{description}
  
\item[Protocol] The communication protocol between the client and the
  server.  For the HTTP implementation of the \DAP, this is always
  \lit{http}.

\item[Host] The IP number or name of the host machine running the \DAP
  server.
  
\item[Server] Many \DAP servers are simply a set of CGI scripts
  executed on demand by the \lit{httpd} server.  Here, the server is
  represented by a CGI script called \lit{nph-nc}.  Many HTTP servers
  parse the headers of incoming CGI requests.  The HTTP specification
  dictates that CGI programs whose name begins with \lit{nph-} will
  receive their data intact.  (See note on page~\pageref{note-1}.)
  Many \DAP implementations take advantage of this feature, since it
  removes dependence on any particular HTTP server software.
  
\item[Directory \& Filename] The \DAP is fundamentally a file-based
  protocol.  To the right of the server (CGI) name comes a directory
  and file name identifying the file containing the requested data.
  These together make up the \new{Dataset ID}.  \DAP implementations
  that serve data not stored in files can omit the directory names, or
  use them to organize their data.

\item[URL suffix] The \DAP uses the suffix on the file name to
  determine what sort of request this is.  \Tableref{tab-url-suffix}
  has a list of the suffixes and the kinds of responses they 
  summon.  There are a variety of services available from most \DAP
  HTTP servers.  These include getting the data as formatted text,
  getting information about the server and data available from it, 
  browsing available data files, and others.   Requests for these are
  indicated by different URL suffixes.  See \DAPASCII, \DAPHTML, or
  \Sectionref{sec-help} of this document for information about these
  services. 

  \begin{table}[htbp]
    \begin{center}
      \begin{tabular}[t]{lp{3in}}
        \tblhd{Suffix} & \tblhd{Meaning} \\ \hline
        \lit{.info} & Information about data and, optionally, the \DAP server\\
                    & returned as HTML \\ \hline
        \lit{.asc}  & ASCII Formatted data (\DAPASCII) returned as Text\\ \hline
        \lit{.html} & Data review form (\DAPHTML) returned as HTML\\ \hline
        \lit{.ver}  & Version information (\Sectionref{sec-version})
                      returned as Text\\ \hline
        \lit{.dods}, \lit{.das}, \lit{.dds} & DAP 2.0 responses returned as
                     defined in \DAPObjects \\ \hline
      \end{tabular}
      \caption{URL suffixes and their meaning}
      \label{tab-url-suffix}
    \end{center}
  \end{table}
  
  Most dedicated \DAP clients hide the suffixes from the user.  A user
  would enter the \DAP URL up to, but not including, the suffix, and
  the client software appends the suffix before sending it to the
  server.  It is convenient to know the suffixes, however, because
  of the \DAP services that provide responses that can be displayed by a
  standard web server.

\item[Constraint] The constraint expression is a way to subsample a
  \DAP dataset.  Constraint expression basics are covered in
  \DAPObjects, and the way to express a constraint expression in a
  request to an HTTP \DAP server is in \Sectionref{sec-ce-http}.

\end{description}

The URL in \Figureref{fig-url-parts} shows a client
request to the \lit{httpd} server on the machine
\lit{dods.gso.uri.edu}, for a netCDF dataset (specified by the
\lit{nph-nc} in the \lit{cgi-bin} directory) contained in a file
called \lit{fnoc1.nc}.  Upon receiving this URL, the \lit{httpd}
server executes the specified \DAP server module (\lit{nph-nc}), which
retrieves the file \lit{fnoc.nc} in a directory called \lit{data} relative to
wherever the \lit{httpd} server looks for its data.

\label{note-1}
\note{The only part of the URL whose spelling is not at the discretion of
the administrator of the host machine is the \lit{http}, and the
\lit{nph-} at the beginning of the CGI script name. Even the \lit{nc},
indicating netCDF, can be changed, although for clarity's sake, we
hope people won't do so.  Incidentally, the \lit{nph-} is a relic,
dating from the early days of the World Wide Web and the first
hypertext protocol standards.  It stands for ``Non-Parsing Header''
(See the CGI 1.1 Standard for more information.), and is the only way
to pass data through many httpd servers unparsed.  The \opendap
software uses this to make the server software portable between HTTP
servers.  If this is not important to you, and you know how to keep
your server from trying to parse the data passing through it, then
this is not essential.}



\subsection{Request Headers}


The headers described in this section MUST be handled as described.
Other headers which are part of HTTP 1.1 MAY be included in the
request and MAY be honored by a \DAP server.

\subsubsection{Accept-Encoding}
\label{sec-accept-encoding}

The \lit{Accept-Encoding} request-header is used by a \DAP client to
tell a server that it can accept compressed responses. See RFC
2616~\cite{rfc2616} for this header's grammar.  Values for encodings are
\lit{deflate}, \lit{gzip} and \lit{compress}. This header is
OPTIONAL. When a client includes this header it is effectively asking the
\DAP server to encode the response using the given scheme. The server is
under no obligation to use the requested encoding. However, a server MUST NOT
use an encoding when a client has not requested it. In addition, a client
MUST supply the header with every request for which it desires a special
encoding.

\subsubsection{Host}
\label{sec-host}

The \lit{Host} request-header is used by a \DAP client to provide its
IP address or DNS name to the \DAP server. See RFC 2616~\cite{rfc2616}
for this header's grammar. This header MUST be included with every request.

\subsubsection{User-Agent}
\label{sec-user-agent}

The \lit{User-Agent} request-header is used by a \DAP client to
provide specific information about the client software to the \DAP
server. See RFC 2616~\cite{rfc2616} for this header's grammar. This header is
RECOMMENDED. \DAP servers MAY log this information.

\subsection{Help Requests}
\label{sec-help}

There are two more required request forms for the HTTP implementation
of the \DAP: the \INFO request and the \HELP request.  

A \HELP request is intended to evoke information about the function of
a \DAP server, while an \INFO request includes that information with
information about a particular dataset served by that server.  The
syntax of a \HELP request is shown in \Tableref{tab:help}.

The \HELP response MUST be returned when the server receives a URL
whose Dataset ID portion is \lit{help}.  It MAY reply with the \HELP
response when it receives a URL with no extension (i.e., a URL with no
Dataset ID at all).  This second option can interfere with the \DIR
response (see \DAPHTML) if your server keeps datasets in
the root-level directory.  In this case, the \DIR response MUST be
returned instead of the \HELP response in reply to a URL without a
Dataset ID.

\begin{table}[!h]
\label{tab:help}
\caption{Syntax of a \HELP request}
\begin{center}
\begin{tabular}{lll}
\var{abs\_path} & = & \var{server\_path}/\var{dataset\_id} \\
\var{server\_path} & = & name of DAP server (including CGI, if any)\\
\var{dataset\_id} & = & \lit{help}
\end{tabular}
\end{center}
\end{table}

The syntax of an \INFO request is shown in \Tableref{tab:info}.

\begin{table}[!h]
\label{tab:info}
\caption{Syntax of an \INFO request}
\begin{center}
\begin{tabular}{lll}
\var{abs\_path} & = & \var{server\_path}/\var{dataset\_id} \lit{.} \var{ext} \\
\var{server\_path} & = & name of DAP server (including CGI, if any)\\
\var{dataset\_id} & = & directory and filename of some data set \\
\var{ext} & = & \lit{info} \\
\end{tabular}
\end{center}
\end{table}

\Sectionref{sec-help-response} describes the data returned by these
two requests

\subsection{Other Requests}

The requests (and their associated responses) described so far are the
ones that are either required or commonly available among \DAP
servers.  Two other responses are sometimes available, and the
\opendap project provides specifications for server implementators who
wish to provide those services.  These are the formatted data response
(also called the ASCII response), described in \DAPASCII, and the HTML
response, described in \DAPHTML.  Please refer to those documents for
information about those services.

\section{Responses}
\label{sec-responses}

A valid \DAP response has the same form as a valid HTTP response.  The
first line contains the HTTP protocol version, a status code and
reason phrase~\cite{rfc2616}. Following this are the response headers
which vary depending on the request and payload of the response (see
\Sectionref{sec-resp-headers} for a description of the headers). As
described in \cite{rfc822}, the HTTP response status line and headers
are separated from the response's payload by an extra set of CRLF
characters which make a blank line.

The \DAP response is the payload of the MIME-like HTTP response.  The
data objects that may be returned are described in detail in the
\DAPObjects.

\tbd{[Add some text describing how the three basic responses can be thought
  of as objects]}

\subsection{Response Headers}
\label{sec-resp-headers}

Like the requests, the HTTP response headers MAY contain all the
header fields specified in RFC 2616~\cite{rfc2616}.  Following is a
list of the header fields about which there is some \DAP{}-specific
behavior.  It is not a list of all the header fields used by the
\DAP.  In particular, servers are highly encouraged to implement the
\lit{Last-Modified} header field.

\subsubsection{Content-Description}

The \lit{Content-Description} header is used to tell clients which of
the different basic responses is being returned or if an error message
is being returned. For any of the basic responses (\DDX, \DAX, \Blob,
or the older \DDS, \DAS, or \DataDDS responses) or the error response
(\ErrorX object), this header MUST be included. This header SHOULD NOT
be included by the \TEXT, \HTML, \INFO, version or directory responses.
See RFC 2045~\cite{rfc2045} for information about this header.

\begin{ttfamily}
\begin{center}
\begin{tabular}{lll}
Content-Description & = & "Content-Description" ":" tag \\
tag & = & | "dap-ddx" | "dap-dax" | "dap-blob" | "dap-errorx" \\
 & & | "dods-dds" | "dods-das" | "dods-data" | "dods-error" \\
\end{tabular}
\end{center}
\end{ttfamily}

Example:

\begin{vcode}{it}
Content-Description: dods-error
\end{vcode}

\subsubsection{Content-Encoding}

If a \DAP server applies an encoding to an entity, it MUST include the
\lit{Content-Encoding} header in the response. See RFC
2616~\cite{rfc2616} for this header's grammar. The sole recognized
encoding for the \DAP is \lit{deflate}.

Example:

\begin{vcode}{it}
Content-Encoding: deflate
\end{vcode}

\subsubsection{Content-Type}

The \lit{Content-Type} header MUST be included in any response from a
\DAP server. Valid content types for \DAP responses are:
\lit{text/plain}, \lit{text/html} and \lit{application/octet}.  See
RFC 2616~\cite{rfc2616} for this header's grammar.

\emph{It would be better to use a multipart
  document in place of the \lit{application/octet}.} 

Example:

\begin{vcode}{it}
Content-Type: application/octet
\end{vcode}

\subsubsection{Date}

The \lit{Date} header provides a time stamp for the response. This header
is needed for servers that support caching. See RFC 2616~\cite{rfc2616}
for this header's grammar. Servers MUST provide this header.

Example:

\begin{vcode}{it}
Date: Fri, 09 Feb 2003 18:54:55 GMT
\end{vcode}

\subsubsection{Keep-Alive}

A \DAP server (or an underlying HTTP server if one is used to
implement the \DAP server) MAY return a \lit{Keep-Alive} header for an
authenticate (code 401) response. For all other responses, the \DAP
server MUST NOT return this header. See
RFC 2616~\cite{rfc2616} for this header's grammar.

\emph{This is a shortcoming of the \DAP. It should support HTTP/1.1's
  persistent connections. However, to do requires that the responses
  also return Content-Length. Since none of our servers do this, I've
  got no experience with persistent connections.}

\subsubsection{Server}

The \lit{Server} header provides information about the server used to
process the request. In this case the \emph{server} MAY be either the
\DAP server or an underlying HTTP server if the \DAP server uses
that as part of its implementation. See RFC 2616~\cite{rfc2616}
for this header's grammar. This header is OPTIONAL.

Example:

\begin{vcode}{it}
Server: Apache/1.3.12 (Unix)  (Red Hat/Linux) PHP/3.0.15 mod_perl/1.21
\end{vcode}

\subsubsection{WWW-Authentication}

The \lit{WWW-Authenticate} header MUST be included in a message that has a
response code of 401. That is, when the \DAP server is asked to provide
access to a resource that is restricted and the request does not include
authentication information (see ``HTTP Authentication: Basic and Digest
Access Authentication''~\cite{rfc2617}). then it must return with a response
code of 401 and include the \lit{WWW-Authenticate} header. See RFC
2616~\cite{rfc2616} for this header's grammar.

Example:

\begin{vcode}{it}
WWW-Authenticate: Basic realm="special directory, with CGIs"
\end{vcode}

\subsubsection{XDODS-Server}

The \lit{XDODS-Server} header is used to return a \DAP server's
implementation version information to the client program.  This header
MUST be included in every response. 

\begin{ttfamily}
\begin{center}
\begin{tabular}{lll}
XDODS-Server & = & "XDODS-Server" ":" "dods/" version [, text] \\
version & = & DIGIT "." DIGIT [ "." DIGIT ] \\
text & = & software implementation information, as a single line of text
\end{tabular}
\end{center}
\end{ttfamily}

Example:

\begin{vcode}{it}
XDODS-Server: dods/4.0.1, netCDF server 4.2.2
\end{vcode}

\subsection{Help Responses}
\label{sec-help-response}

There are two more required request forms for the HTTP implementation
of the \DAP: the \INFO request and the \HELP request.  

\subsubsection{\INFO}
\label{sec-info}

The \INFO response MUST present the user with an HTML document that
contains the dataset-specific information in the dataset's \DDX, as
well as the server-specific information in the \VER response and the
\CapX.  The intent is to present this information in a way that can be
rendered by a WWW browser.  Structural information about the dataset
SHOULD be preserved so that users can build \DAP URLs by hand.

The \INFO response MAY also return other information. If server
installers and/or dataset maintainers add HTML which describes the
dataset or the server, this information MUST be merged into the HTML
document returned in response to the \INFO request.

The purpose of the \INFO response is to supply information about the
dataset in a form that is easy for people to read. It should be
structured so that after only a little experience people can easily
assess a dataset using the document. In addition, the \INFO response
may be used in creating various user interfaces which access data
using \DAP servers.

The \INFO response should provide:

\begin{enumerate}
\item The hierarchical relation of container variables.
\item Each variable's datatype.
\item Each variable's attributes.
\item Any global attributes that the dataset contains.
\item Extra information supplied by the dataset creator/maintainer.
\item \CE functions available on this server.
\item The information from the \VER response.
\end{enumerate}

One implementation of the \INFO response uses HTML text in files
corresponding to each dataset served by a server, as well as HTML
documenting the server as a whole.  These documents are merged on the
fly to create the \INFO response, providing information both about the
server and the specific datasets in question.  

The \INFO response is by URLs with the syntax shown in
\Sectionref{sec-help}.  The response headers are as follows:

\begin{textoutput}{Required Headers:}
XDODS-Server: dods/4.0\\
Content-Type: text/html\\
Date: \emph{date}
\end{textoutput}


\subsubsection{\HELP}

The \HELP response MUST return an HTML document which lists the
extensions recognized by the server.  The response MAY return other
information as well.  The intent is to provide a human-readable
document, to be displayed in a WWW browser, that provides instruction
in the use of the server that received the request.

\begin{textoutput}{Required Headers:}
XDODS-Server: dods/4.0\\
Content-Type: text/html\\
Date: \emph{date}
\end{textoutput}


\subsection{Other Responses}

To learn about the responses to HTML or formatted data (ASCII)
requests, see \DAPHTML or \DAPASCII. 


\section{Authentication}

\note{Expand this section.}

The HTTP implementation of the \DAP should provide for authentication.
A \DAP implementation must implement HTTP/1.1 authorization features.

A client can include the username and password directly in the \DAP
URL, like this:

\begin{vcode}{it}
http://username@password:machine.uri.edu/nph-nc/data/file.nc  
\end{vcode}



\appendix

\bibliographystyle{plain}
\T\addcontentsline{toc}{section}{References}
\T\raggedright
\bibliography{../../../boiler/dods}

\end{document}
