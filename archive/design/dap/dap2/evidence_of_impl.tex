%
% Evidence of Implementation as outlined in NASA/ESE RFC 002. 06/29/04 jhrg

\documentclass[justify]{dods-paper}
\usepackage{xspace}
\rcsInfo $Id$

%
% These are html links which are used often enough in writing about DODS to
% merit an input file.
% jhrg. 4/17/94
%
% File rationalized and updated while writing the DODS User
% Guide. Also includes other useful abbreviations.
% tomfool 3/15/96
%
% Moved to dods-def.tex so I can remove links to documents that no
% longer reflect reality.
% tomfool 2/13/98
%
% $Id$
%
% Make sure to include layout.tex *before* using this file.

%% NOTE NOTE NOTE NOTE NOTE NOTE NOTE NOTE NOTE NOTE NOTE NOTE NOTE NOTE 
%%
%% If this file causes problems when running latex, you may have to edit your
%% texmf.cnf file. Here's a meesage from Tom:
%% > Are these references that use relative addresses (like
%% > ../boiler/blah.tex)?  If they are, you should look for the texmf.cnf
%% > file.  (It's often at /usr/share/texmf/web2c/texmf.cnf, and look for
%% > the openout_any parameter.  Check there anyway; there were some recent
%% > (i.e. in the early '90's) security fixes to TeX.
%%
%%%%%%%%%%%%%%%%%%%%%%%%%%%%%%%%%%%%%%%%%%%%%%%%%%%%%%%%%%%%%%%%%%%%%%%%%%%

%%% These are some DODS-specific convenience commands.
\newcommand{\DODSroot}{\lit{\$DODS\_ROOT}}     
% $

\newcommand{\opendap}{OPeNDAP\xspace}

%%% The OPD books and reference material
\newcommand{\OPDDoc}{http://opendap.org/support/docs.html}
\newcommand{\DODSDoc}{http://opendap.org/support/docs.html}

% \newcommand{\OPDDoc}{http://www.unidata.ucar.edu/packages/dods}
% \newcommand{\DODSDoc}{http://www.unidata.ucar.edu/packages/dods}

\newcommand{\OPDhomeUrl}%
  {http://opendap.org}
\newcommand{\OPDexampleUrl}%
  {BROKEN--FIX ME!}%\OPDDoc/examples}
% \newcommand{\OPDftpUrl}%
%  {ftp://dods.gso.uri.edu/pub/dods}
\newcommand{\OPDftpUrl}%
  {ftp://ftp.unidata.ucar.edu/pub/opendap/}
\newcommand{\OPDuserUrl}%
  {\OPDhomeUrl/user/guide-html/}
\newcommand{\OPDmguiUrl}%
  {\l/user/mgui-html/}
\newcommand{\OPDapiUrl}%
  {\OPDhomeUrl/api/pguide-html/}
\newcommand{\OPDapirefUrl}%
  {\OPDhomeUrl/api/pref/html/}
\newcommand{\OPDffUrl}%
  {\OPDhomeUrl/user/servers/dff-html/}
\newcommand{\OPDquickUrl}%
  {\OPDhomeUrl/user/quick-html/}
\newcommand{\OPDinstallUrl}%
  {\OPDhomeUrl/server/install-html}%
\newcommand{\OPDregexUrl}%
  {\OPDhomeUrl/user/regex-html}%
\newcommand{\OPDjavaUrl}%
  {\OPDhomeUrl/home/swJava1.1/}
\newcommand{\OPDwclientUrl}%
  {\OPDhomeUrl/api/wc-html/}
\newcommand{\OPDwserverUrl}%
  {\OPDhomeUrl/api/ws-html/}
\newcommand{\OPDaggUrl}%
  {\OPDhomeUrl/server/agg-html/}

\newcommand{\OPDhome}{\xlink{OPeNDAP Home page}{\OPDhomeUrl}}
\newcommand{\OPDjava}{\xlink{OPeNDAP Java home page}{\OPDjavaUrl}}
\newcommand{\OPDexamples}{\xlink{OPeNDAP examples page}{\OPDexampleUrl}}
\newcommand{\OPDftp}{\xlink{OPeNDAP ftp site}{\OPDftpUrl}}
%% Book titles do *not* contain the article.
\newcommand{\OPDuser}[1][]{\xlink%
  {\cit{OPeNDAP User Guide}}{\OPDuserUrl{}#1}}
\newcommand{\OPDmgui}{\xlink%
  {\cit{OPeNDAP Matlab GUI}}{\OPDmguiUrl}}
\newcommand{\OPDapi}{\xlink%
  {\cit{OPeNDAP Toolkit Programmer's Guide}}{\OPDapiUrl}}
\newcommand{\OPDapiref}{\xlink%
  {\cit{OPeNDAP Toolkit Reference}}{\OPDapirefUrl}}
\newcommand{\OPDffbook}{\xlink%
  {\cit{OPeNDAP Freeform ND Server Manual}}{\OPDffUrl}}
\newcommand{\OPDquick}{\xlink%
  {\cit{OPeNDAP Quick Start Guide}}{\OPDquickUrl}}
\newcommand{\OPDinstall}{\xlink%
  {\cit{OPeNDAP Server Installation Guide}}{\OPDinstallUrl}}
\newcommand{\OPDregex}{\xlink%
  {\cit{Introduction to Regular Expressions}}{\OPDregexUrl}}
\newcommand{\OPDagg}{\xlink%
  {\cit{OPeNDAP Aggregation Server Guide}}{\OPDaggUrl}}
\newcommand{\OPDwclient}{\xlink%
  {\cit{Writing an OPeNDAP Client}}{\OPDwclientUrl}}
\newcommand{\OPDwserver}{\xlink%
  {\cit{Writing an OPeNDAP Server}}{\OPDwclientUrl}}

\newcommand{\OPDffs}{OPeNDAP Freeform ND Server}

%%% Other DODS links.
\newcommand{\homepage}% For hyperlatex
  {\OPDDoc/}
\newcommand{\OPDsupport}{\xlink{support@unidata.ucar.edu}{mailto:support@unidata.ucar.edu}}
\newcommand{\DODSsupport}{\xlink{support@unidata.ucar.edu}{mailto:support@unidata.ucar.edu}}
\newcommand{\DODS}{\xlink{Distributed Oceanographic Data System}{\OPDhomeUrl}}
\newcommand{\OPD}{\xlink{Open Source Project for a Data Access Protocol}{\OPDhomeUrl}}
\newcommand{\OPDtechList}{\xlink{OPeNDAP Mailing Lists}{\OPDhomeUrl/mailLists/}}

%%% DODS versions
%% This has been removed.  Documents should not have an automatic
%% version number, because then it appears as if they have been
%% updated when they haven't.  Put the relevant version number to
%% whatever software is being described into each document's preface. 

\newcommand{\ifh}{WWW Interface}

% external refs for DODS documents

\newcommand{\CGI}{\xlink{Common Gateway Interface}
  {http://hoohoo.ncsa.uiuc.edu/cgi/overview.html}}

\newcommand{\MIME}{\xlink{Multipurpose Internet Mail Extensions}
  {http://www.cis.ohio-state.edu/htbin/rfc/rfc1590.html}}

\newcommand{\netcdf}{\xlink{NetCDF}
  {http://www.unidata.ucar.edu/packages/netcdf/guide.txn_toc.html}}

\newcommand{\JGOFS}{\xlink{Joint Geophysical Ocean Flux Study}
  {http://www1.whoi.edu/jgofs.html}}

\newcommand{\jgofs}{\xlink{JGOFS}
  {http://www1.whoi.edu/jgofs.html}}

\newcommand{\hdf}{\xlink{HDF}
  {http://www.ncsa.uiuc.edu/SDG/Software/HDF/HDFIntro.html}}

\newcommand{\ffnd}{FreeForm ND}

\newcommand{\Cpp}{\texorhtml
  {{\rm {\small C}\raise.5ex\hbox{\footnotesize ++}}}
  {C\htmlsym{##43}\htmlsym{##43}}}

% Commands

% Use pdflink instead. jhrg 8/4/2006
% \newcommand{\pslink}[1]{\small
% \begin{quote}
%   A \xlink{PDF version}{#1} of this document is available.
% \end{quote}
% \normalsize
% }

% Use the copy of this in dods-paper.hlx/cls or cut and paste this on
% a document-by-document basis. This version conflicts with the
% version in the class. jhrg 8/4/2006
% \newcommand{\pdflink}[1]{\small
% \begin{quote}
%   A \xlink{PDF version}{#1} of this document is available.
% \end{quote}
% \normalsize
% }

%%%%%%%%%%%%%% DODS macros
%
% These are here so that older latex files will compile. Someday remove these
% and fix the files. 04/13/04 jhrg

\newcommand{\DODShomeUrl}%
  {\OPDhomeUrl}
\newcommand{\DODSexampleUrl}%
  {\OPDDoc/examples}
% \newcommand{\OPDftpUrl}%
%  {ftp://dods.gso.uri.edu/pub/dods}
\newcommand{\DODSftpUrl}%
  {ftp://ftp.unidata.ucar.edu/pub/opendap/}
\newcommand{\DODSuserUrl}%
  {\OPDhomeUrl/user/guide-html/}
\newcommand{\DODSmguiUrl}%
  {\OPDhomeUrl/user/mgui-html/}
\newcommand{\DODSapiUrl}%
  {\OPDhomeUrl/api/pguide-html/}
\newcommand{\DODSapirefUrl}%
  {\OPDhomeUrl/api/pref/html/}
\newcommand{\DODSffUrl}%
  {\OPDhomeUrl/user/servers/dff-html/}
\newcommand{\DODSquickUrl}%
  {\OPDhomeUrl/user/quick-html/}
\newcommand{\DODSinstallUrl}%
  {\OPDhomeUrl/server/install-html}%
\newcommand{\DODSregexUrl}%
  {\OPDhomeUrl/user/regex-html}%
\newcommand{\DODSjavaUrl}%
  {\OPDhomeUrl/home/swJava1.1/}
\newcommand{\DODSwclientUrl}%
  {\OPDhomeUrl/api/wc-html/}
\newcommand{\DODSwserverUrl}%
  {\OPDhomeUrl/api/ws-html/}
\newcommand{\DODSaggUrl}%
  {\OPDhomeUrl/server/agg-html/}

\newcommand{\DODShome}{\xlink{OPeNDAP Home page}{\DODShomeUrl}}
\newcommand{\DODSjava}{\xlink{OPeNDAP Java home page}{\DODSjavaUrl}}
\newcommand{\DODSexamples}{\xlink{OPeNDAP examples page}{\DODSexampleUrl}}
\newcommand{\DODSftp}{\xlink{OPeNDAP ftp site}{\DODSftpUrl}}
\newcommand{\DODSuser}[1][]{\xlink%
  {\cit{The OPeNDAP User Guide}}{\DODSuserUrl{}#1}}
\newcommand{\DODSmgui}{\xlink%
  {\cit{The OPeNDAP Matlab GUI}}{\DODSmguiUrl}}
\newcommand{\DODSapi}{\xlink%
  {\cit{The DODS Toolkit Programmer's Guide}}{\DODSapiUrl}}
\newcommand{\DODSapiref}{\xlink%
  {\cit{The DODS Toolkit Reference}}{\DODSapirefUrl}}
\newcommand{\DODSffbook}{\xlink%
  {\cit{The DODS Freeform ND Server Manual}}{\DODSffUrl}}
\newcommand{\DODSquick}{\xlink%
  {\cit{The DODS Quick Start Guide}}{\DODSquickUrl}}
\newcommand{\DODSinstall}{\xlink%
  {\cit{The DODS Server Installation Guide}}{\DODSinstallUrl}}
\newcommand{\DODSregex}{\xlink%
  {\cit{Introduction to Regular Expressions}}{\DODSregexUrl}}
\newcommand{\DODSagg}{\xlink%
  {\cit{OPeNDAP Aggregation Server Guide}}{\DODSaggUrl}}
\newcommand{\DODSwclient}{\xlink%
  {\cit{Writing an OPeNDAP Client}}{\DODSwclientUrl}}
\newcommand{\DODSwserver}{\xlink%
  {\cit{Writing an OPeNDAP Server}}{\DODSwclientUrl}}

\newcommand{\DODSffs}{DODS Freeform ND Server}

% $Log: dods-def.tex,v $
% Revision 1.24  2004/12/21 22:30:04  jimg
% Fixed pslink; Added pdflink.
%
% Revision 1.23  2004/12/14 05:19:17  tomfool
% restored fix to pslink
%
% Revision 1.22  2004/12/09 21:01:58  tomfool
% excised test.dods.org
%
% Revision 1.21  2004/12/09 18:50:21  tomfool
% de-dodsifying
%
% Revision 1.15  2004/04/24 21:37:22  jimg
% I added every directory in preparation for adding everyting. This is
% part of getting the opendap web pages going...
%
% Revision 1.14  2004/02/12 16:05:50  jimg
% Moved the log to the end of the file.
%
% Revision 1.13  2004/01/16 18:05:31  jimg
% Added a note from Tom about setting texmf.cnf to allow \include to process
% files with ../ in their pathnames. You can also change the include to input,
% but I think include may offer some advantages for bigger/complex things like
% the Guides.
%
% Revision 1.12  2003/12/28 21:48:22  tom
% added newer books
%
% Revision 1.11  2003/12/08 19:04:43  tom
% little adjustments for DODS->opendap
%
% Revision 1.10  2003/12/08 18:53:30  tom
% DODS->OPeNDAP
%
% Revision 1.9  2002/07/15 17:49:55  tom
% added \DODSDoc
%
% Revision 1.8  2001/05/04 15:07:45  tom
% fixed pslink to include pdf files
%
% Revision 1.7  2001/02/19 20:39:13  tom
% added links to the new regex intro.
%
% Revision 1.6  2000/03/23 18:27:52  tom
% added abbreviations
%
% Revision 1.5  1999/07/01 16:00:19  tom
% added a couple of web page references
%
% Revision 1.4  1999/05/25 20:49:34  tom
% changed version numbers to 3.0
%
% Revision 1.3  1999/02/04 17:27:08  tom
% adjusted for hyperlatex and dods-book.cls
%
%

%%% Local Variables: 
%%% mode: latex
%%% TeX-master: t
%%% TeX-master: t
%%% End: 


\title{Evidence of Implementation for DAP 2.0}
\htmltitle{Evidence of Implementation and Significant Operational Experience for DAP 2.0}
\author{James Gallagher\thanks{OPeNDAP, Inc., $<j.gallagher@opendap.org>$}}
\date{\rcsInfoDate \\ Revision \rcsInfoRevision}
\htmladdress{James Gallagher <j.gallagher@opendap.org>, 
  \rcsInfoDate, Revision: \rcsInfoRevision}
\htmldirectory{evidence-html}
\htmlname{evidence}

\begin{document}

\maketitle

\section{Implementations}

The following list of implementations and operational sites is correct as of
8 August 2004. We know of other implementations and sites, but finding
accurate information about them can be challenging since there is no
requirement that users of the DAP communicate with us. Also, as is often the
case with an open-source protocol, implementations sometimes come to light
only after they have been developed and are used for some time.

The list below includes only those groups which have developed DAP software,
as opposed to those groups that have developed software which uses the DAP.
For the latter, see the list of `Other groups that distribute DODS/OPeNDAP
software'\footnote{http://www.opendap.org/developers/third\_party\_software.html}
at www.opendap.org. Also note that information about servers is only included
where the site has developed its own implementation of the DAP. For lists of
known DAP-compliant servers, see `OPeNDAP
Datasets'\footnote{http://www.opendap.org/data/index.html} at the OPeNDAP web
page. Section~\ref{opendap-servers} discusses OPeNDAP's experience with its
servers. 

Because the DAP is intended to be language-neutral, it is important that
there be implementations in different programming languages. The list below
includes that information.

The Data Access Protocol (DAP) has both a client- and a server-side
component, in the same sense that other protocols such as FTP or HTTP do.
Because most people are interested in solely one or the other component, most
of the implementations originating outside our group are for just the client
or server part.

\begin{itemize}
\item The reference implementation for the DAP is maintained by OPeNDAP. It
  was originally written as part of the Distributed Oceanographic Data System
  (DODS) project and then became part of the National Virtual Ocean Data
  System (NVODS) project. The implementation includes a library with a
  complete implementation of both the client- and server-components of the
  protocol as well as several client applications and a server capable of
  handling data stored/accessed in six different formats/APIs. Many
  government and university sites use this software; a partial list is
  available from the same location as the software. This software is written
  in \Cpp\ and is available at http://www.opendap.org/.

\item A second implementation of the DAP, written in Java, was started by
  personnel at JPL, working under subcontract as part of the DODS project.
  They built an implementation of the DAP for use in client development.
  Later, the implementation was extended to support servers as well, under
  subcontract with Oregon State University (OSU). Using the library,
  personnel at OSU developed a server for data held in relational (SQL)
  databases. Many government and university sites use this software; a
  partial list is available from the same location as the software. This
  software is written in Java and is available at http://www.opendap.org/.
  Note that this software is completely independent of the \Cpp\ reference
  implementation.

\item One of the software `products' we have built using the \Cpp\ DAP
  implementation is the `NetCDF Client Library.' This is a library which uses
  the netCDF file access API to read data from DAP-compliant servers. This is
  piece of software has been very successful and is used by many groups to
  make software they have already developed `network aware.' In addition,
  Unidata, the developers of the original NetCDF software, have used the
  Java-DAP implementation to add a similar capability to the Java netCDF
  library software they distribute. This software has also been very
  successful and many Pure Java clients use it to access data from
  DAP-compliant servers. Some of the client applications which use one of
  these two Client Libraries are listed on the Software
  page\footnote{http://www.opendap.org/developers/third\_party\_software.html}
  mentioned above.

\item The GrADS Data Server (GDS) and OPeNDAP-enabled GrADS application are
  both distributed by the Center for Ocean-Land-Atmosphere Studies
  (COLA)\footnote{http://cola.iges.org/}. The GDS is a DAP-compliant server
  which has been developed at COLA for internal use. However, they have
  decided to distribute the server under an open-source license. The
  OPeNDAP-enabled GrADS analysis program is also available from COLA. Both of
  these use software developed in-house at COLA as well as our software.
  Contact Brian Doty ($<doty@cola.iges.org>$) for more information.

\item The LDEO/IRI Climate Data
  Library\footnote{http://ingrid.ldeo.columbia.edu/} uses its own
  implementation of the DAP to read data from servers. The software is
  written in C. Contact Benno Blumenthal ($<benno@iri.columbia.edu>$) for
  more information.

\item A Python library for creating
  DODS/OPeNDAP\footnote{http://opendap.oceanografia.org/} clients has been
  developed at the Universidade~de~S\~ao~Paulo. Contact Roberto AF De Almeida
  ($<roberto@dealmeida.net>$) for more information.

\item The High-Altitude Observatory (HAO) at UCAR has developed a server for
  CEDAR data. This software has been written in \Cpp\ and used our \Cpp\ DAP
  implementation as a starting point but differs in several significant ways.
  First, the server supports both HTTP and GridFTP access to data (the latter
  is part of the Earth System Grid II project). Secondly, because CEDAR data
  is stored and accessed in ways that differ considerably from those of the
  data types our server was built to handle, the HAO group has made
  significant extensions to our \Cpp\ class library. Many of those extensions
  have been, or will be, incorporated into the library we distribute, but
  they have their origin out side our group. Contact Jose Garcia
  ($<jgarcia@ucar,edu>$) or Peter Fox ($<pfox@ucar.edu>$) for information
  about this software.

\item In the past various groups have talked to us about using the DAP as the
  backbone for an internal data system. It is very hard to track these
  developments because they mostly happen within the organizations. However,
  one such group which also provides access from outside their laboratory is
  at the Institut Pierre Simon Laplace (IPSL) in France. Contact `dodsipsl'
  ($<dodsipsl@ipsl.jussieu.fr>$) with the subject ``DODS-IPSL" for more
  information.

\end{itemize}

An important point to keep in mind when looking at web references to the DAP
is that in the past we did not make a strong distinction between the DODS and
NVODS projects and DAP. It is common to find references to the DAP which use
the words `DODS,' `NVODS,' or `OPeNDAP' instead.

\section{Operational Experience}

Operational experience with the DAP can be divided into two categories:
experience with software developed by OPeNDAP and experience with software
developed by other groups.

\subsection{OPeNDAP software}
\label{opendap-servers}

The OPeNDAP group has deployed data servers in use at more than forty
locations which range from university sites (e.g., University of Rhode
Island, University of Miami), government sites (e.g., NASA GSFC, NASA JPL,
NOAA PMEL) to corporate sites (e.g., SAIC, UCAR).

The DAP is used to serve data stored at these sites and is used by clients to
read data from those servers. The servers provide access to data stored in
netCDF, HDF4, Matlab, and DSP files as well as data which can be accessed
using either FreeForm or JGOFS (FreeForm is an API which can be used to
describe ad hoc data file formats; JGOFS is a data system developed for the
JGOFS project). Client applications include Ferret (using out netCDF client
library), Matlab (using our command-line client tool) and the OPeNDAP Data
Connector as well as significant set of other analysis programs. 

The data made accessible using the OPeNDAP servers primarily fall under the
oceanographic discipline, although not exclusively so. There are data sources
which house satellite imagery, model output, and climatologies as well as
point data from XBT, CDT and ships of opportunity. Estimates conservatively
put the data volume at 10 TeraBytes.\footnote{More realistic estimates are
in the range of 20 to 30 TeraBytes, but but constantly changing data sources
such as those as GSFC and JPL make this a very hard figure to pin down.}

In some cases data are available using other protocols in addition to the
DAP. In most cases where another access protocol is supported, it is FTP;
files which are used by our servers are also easily made available for access
using FTP.

The software implementing the DAP is maintained by OPeNDAP, a Rhode Island
not-for-profit specifically created to main and extend the DAP and other
related discipline-independent network data access technology.

For more information on OPeNDAP and the DODS/NVODS servers, contact Peter
Cornillon ($<pcornillon@opendap.org>$, 401.874.6283).

\subsection{COLA/GDS/GrADS}

The Center for Ocean-Land-Atmosphere Studies (COLA) provides access to
climate and weather data using the DAP. Almost all of the data is gridded
analysis and/or model output of atmospheric/ocean/land data, usually global.
A small amount is in-situ or "station" data. About half is NCEP weather
forecast model output, the other half is climate data and/or model output.
Total data volume is estimated at in excess of 3 TeraBytes. COLA began
serving data using the DAP in Sept of 2001. Access to data from the servers
maintained by COLA is generally increasing; for the month of July 2004, 1500
unique IP address requested a total of 176 GigaBytes.

COLA has developed the GDS, a server built using both their own software and
our software. The GDS adds extra capabilities in the form of server-side
analysis. COLA provides this server freely to other sites for their own use.

COLA has combined their GrADS analysis program with our netCDF
client library to create a client application which can read data from their
servers (it can also read data for any DAP-compatible server). 

All aspects of the DAP are used, with some extensions to the Constraint
Expression syntax to facilitate the extra analysis capabilities of the GDS
server. 

COLA is committed to tracking changes in the DAP and maintain their own
software. It should be pointed out that GrADS is a successful open-source
analysis tool in its own right and the developers at COLA have a demonstrated
track record of supporting that software for their user community.

For more information on COLA's use of the DAP, contact Brian Doty
($<doty@cola.iges.org>$, 301.595.7000).

\subsection{LAS}

The NOAA Pacific Marine Research Laboratory (PMEL) has developed a web portal
software package called Live Access Server (LAS) which uses the DAP to build
and interconnect data-access web portals. Both data and visualizations are
available from the sites. LAS makes available three to five TeraBytes of
model, climate, oceanography, satellite, and real time tides-winds-wave data.
PMEL is aware of 30+ institutions with publicly available LAS sites serving
up data. Typical recent accesses are on the order of 400 Mbytes per month
from the LAS operated at PMEL and estimated\footnote{Jonathan Callahan,
private communication, 8 Aug 2004.} at 2GB per month for the sum total of
all LAS sites in operation.

Developers and researcher at PMEL have developed the LAS software and are in
the process of building a server with analysis capabilites. In addition to
LAS and the server under development, PMEL also distributes a DAP-enable
version of their Ferret analysis tool. Their involvement with the software
using the DAP begain in 1999.

All aspects of the DAP are used by the combination of DAP-enabled Ferret, LAS
and the in-progress server.

PMEL is commited to tracking changes in the DAP.

For more information on PMEL's use of the DAP, contact Steve Hankin at PMEL
($<Steven.C.Hankin@noaa.gov>$, 206.526.6080).

\subsection{LEDO/Ingrid}

The Lamont-Doherty Earth Observatory (LDEO) at Columbia University hosts the
IRI/LDEO Climate Data Library which contains over 300 datasets from a variety
of earth science disciplines and climate-related topics.

The IRI/LDEO Climate Data Library (aka Ingrid) is implemented as a web portal
which researchers use to browse and access data. Data are delivered both as
files and using the DAP. Furthermore, Ingrid also reads data from many of the
NVODS servers, making it a client for those servers. The implementation is
entirely in-house.

For more information on Ingrid, contact Benno Blumenthal
($<benno@iri.columbia.edu>$).

\subsection{Overlap in the Data Provided}

In the previous section, the data volumes contain some overlap. That is, some
of the 3 TeraBytes of data accessible from COLA's GDS is also part of the
NVODS data source collection and some of Ingrid's data also appear as part of
NVODS and LAS. An effort to eliminate redundancy has been made, but it should
be stressed that these numbers are estimates.

\end{document}