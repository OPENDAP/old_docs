%
% Specification for the miscellaneous HTTP services that have to do
% with the server itself (and all the data it serves) rather than with
% any particular data set.
%
\documentclass[justify]{dods-paper}
\usepackage{acronym}
\usepackage{xspace}
\usepackage{gloss}
\rcsInfo $Id$

% latex and HTML macros. Some latex commands become nops for HTML. 4/10/2001
% jhrg 
\T\newcommand{\Cpp}{\rm {\small C}\raise.5ex\hbox{\footnotesize++}\xspace}
\T\newcommand{\C}{\rm {\small C}\xspace}
\W\newcommand{\Cpp}{C++}
\W\newcommand{\C}{C\xspace}
\W\newcommand{\cdots}{...}
\W\newcommand{\ddots}{}
\W\newcommand{\vdots}{.}
\W\newcommand{\pm}{+/-}
\W\newcommand{\times}{*}
\W\newcommand{\uppercase}[1]{\textsc{#1}}
\T\newcommand{\qt}{\lit{\char127}}
\W\newcommand{\qt}{"}

\texorhtml{\def\rearrangedate#1/#2/#3#4{\ifcase#2\or January\or February\or
  March\or April\or May\or June\or July\or August\or September\or
  October\or November\or December\fi\ \ifx0#3\relax\else#3\fi#4, #1}
\def\rcsdocumentdate{\expandafter\rearrangedate\rcsInfoDate}}%
{\HlxEval{
(put 'rearrangedate       'hyperlatex 'hyperlatex-ts-rearrange-date)

(defun hyperlatex-ts-rearrange-date ()
  (let ((date-string (hyperlatex-evaluate-string 
                       (hyperlatex-parse-required-argument))))
    (let ((year-string (substring date-string 0 4))
          (month-string (substring date-string 5 7))
          (day-string (substring date-string 8 10))
          (month-list '("January" "February" "March" "April"
                        "May" "June" "July" "August" 
                        "September" "October" "November" "December")))
       (insert
         (concat (elt month-list (1- (string-to-number month-string)))
                 " " (int-to-string (string-to-number day-string))
                 ", " year-string)))))
}
\newcommand{\rcsdocumentdate}{\rearrangedate{\rcsInfoDate}}}

\newcommand{\dapversion}{Version 4.0\xspace}
\newcommand{\opendap}{OPeNDAP\xspace}
\newcommand{\DAP}{DAP\xspace}
\newcommand{\DODS}{DODS\xspace}
\newcommand{\NVODS}{NVODS\xspace}
\newcommand{\CE}{constraint expression\xspace}
\newcommand{\CEs}{constraint expressions\xspace}
\newcommand{\ErrorX}{ErrorX\xspace}
\newcommand{\CapX}{Server Capabilities Document\xspace}
\newcommand{\Blob}{Blob\xspace}
\newcommand{\DDX}{DDX\xspace}
\newcommand{\DAX}{DAX\xspace}
\newcommand{\DDS}{DDS\xspace}
\newcommand{\DAS}{DAS\xspace}
\newcommand{\URI}{URI\xspace}
\newcommand{\DataDDS}{DataDDS\xspace}

\newcommand{\type}[1]{\emph{#1}}
\newcommand{\Alias}{\type{Alias}\xspace}
\newcommand{\Array}{\type{Array}\xspace}
\newcommand{\Attribute}{\type{Attribute}\xspace}
\newcommand{\ProcAttribute}{\type{Processing Attribute}\xspace}
\newcommand{\Map}{\type{Map}\xspace}
\newcommand{\Target}{\type{Target}\xspace}
\newcommand{\FQN}{fully qualified name\xspace}
\newcommand{\Grid}{\type{Grid}\xspace}
\newcommand{\Structure}{\type{Structure}\xspace}
\newcommand{\Dataset}{\type{Dataset}\xspace}
\newcommand{\Sequence}{\type{Sequence}\xspace}
\newcommand{\Container}{\type{Container}\xspace}
\newcommand{\Bi}{\type{Binary Image}\xspace}
\newcommand{\String}{\type{String}\xspace}
\newcommand{\URL}{\type{URL}\xspace}
\newcommand{\Boolean}{\type{Boolean}\xspace}
\newcommand{\Byte}{\type{Byte}\xspace}
\newcommand{\Enum}{\type{Enumeration}\xspace}
\newcommand{\Time}{\type{Time}\xspace}
\newcommand{\Function}{\type{Function}\xspace}
\newcommand{\Description}{\type{Description}\xspace}
\newcommand{\Parameter}{\type{Parameter}\xspace}
\newcommand{\Constraint}{\type{Constraint}\xspace}
\newcommand{\NoAttributes}{\type{NoAttributes}\xspace}
\newcommand{\Project}{\type{Project}\xspace}
\newcommand{\Select}{\type{Select}\xspace}
\newcommand{\Hyperslab}{\type{Hyperslab}\xspace}

\newcommand{\Aliases}{\type{Aliases}\xspace}
\newcommand{\Arrays}{\type{Arrays}\xspace}
\newcommand{\Attributes}{\type{Attributes}\xspace}
\newcommand{\ProcAttributes}{\type{Processing Attributes}\xspace}
\newcommand{\Maps}{\type{Maps}\xspace}
\newcommand{\Targets}{\type{Targets}\xspace}
\newcommand{\FQNs}{fully qualified names\xspace}
\newcommand{\Grids}{\type{Grids}\xspace}
\newcommand{\Structures}{\type{Structures}\xspace}
\newcommand{\Sequences}{\type{Sequences}\xspace}
\newcommand{\Containers}{\type{Containers}\xspace}
\newcommand{\Bis}{\type{Binary Images}\xspace}
\newcommand{\Strings}{\type{Strings}\xspace}
\newcommand{\URLs}{\type{URLs}\xspace}
\newcommand{\Booleans}{\type{Booleans}\xspace}
\newcommand{\Bytes}{\type{Bytes}\xspace}
\newcommand{\Enums}{\type{Enumerations}\xspace}
\newcommand{\Times}{\type{Times}\xspace}
\newcommand{\CSVs}{comma-separated values\xspace}
\newcommand{\Functions}{\type{Functions}\xspace}
\newcommand{\Descriptions}{\type{Descriptions}\xspace}
\newcommand{\Parameters}{\type{Parameters}\xspace}
\newcommand{\Constraints}{\type{Constraints}\xspace}
\newcommand{\Projects}{\type{Projects}\xspace}
\newcommand{\Selects}{\type{Selects}\xspace}
\newcommand{\Hyperslabs}{\type{Hyperslabs}\xspace}

\newcommand{\DIR}{\textbf{Directory}\xspace}
\newcommand{\TEXT}{\textbf{Text}\xspace}
\newcommand{\HTML}{\textbf{HTML}\xspace}
\newcommand{\HELP}{\textbf{Help}\xspace}
\newcommand{\VER}{\textbf{Version}\xspace}
\newcommand{\INFO}{\textbf{Info}\xspace}

%%%%%%%%%%%%%% Web Services Paper
\newcommand{\GetDDX}{\textbf{GetDDX}\xspace}
\newcommand{\GetData}{\textbf{GetData}\xspace}
\newcommand{\GetBlobData}{\textbf{GetBlobData}\xspace}
\newcommand{\GetBlob}{\textbf{GetBlob}\xspace}
\newcommand{\GetDir}{\textbf{GetDir}\xspace}
\newcommand{\GetInfo}{\textbf{GetInfo}\xspace}

\newcommand{\FSs}{\type{Foundation Services}\xspace}
\newcommand{\FS}{\type{Foundation Service}\xspace}
\newcommand{\ODSN}{\type{OPeNDAP}\xspace}

\newcommand{\blobdataelement}{\texttt{BlobData} element\xspace}
\newcommand{\blobelement}{\texttt{Blob} element\xspace}



%\setcounter{secnumdepth}{4}
%\setcounter{tocdepth}{4}

\newcommand{\Tableref}[1]{Table~\ref{#1}}%
\newcommand{\Figureref}[1]{Figure~\ref{#1}}%
\W\begin{iftex}
\newcommand{\Sectionref}[2]{Section~\ref{#1}%
  \ifx#2)%
    \ on page~\pageref{#1}%
  \else% 
    \ (page~\pageref{#1})\xspace%
  \fi\ifx#2\space\ \else #2\fi}%
\W\end{iftex}
\W\newcommand{\Sectionref}[1]{Section~\ref{#1}}
\W\newcommand{\raggedright}{}


%% Conveniences for documenting XML
\newcommand{\tag}[1]{\emph{#1}}
\newcommand{\element}[1]{\link{\tag{#1}}{sec-xml-#1}}
\newcommand{\attribute}[1]{\emph{#1}}
\newcommand{\currentelement}{}
\newcommand{\ELEMENT}[1]{\renewcommand{\currentelement}{#1 element}%
  \subsubsection{#1}\label{sec-xml-#1}\indc{\currentelement}%
  \indc{catalog tag!#1}\indc{aggregation tag!#1}%
  \indc{XML!#1 element}}
\newcommand{\ATTRIBUTE}[1]{\item{\lit{#1}}\indc{\currentelement!#1}
    \indc{#1 attribute!of \currentelement}%
    \indc{XML!#1 attribute}}

% Conveniences for examples
\newcounter{exampleno}
\setcounter{exampleno}{0}
\newcounter{examplerefno}
\setcounter{examplerefno}{0}
\newcommand{\examplelabel}[1]{\refstepcounter{exampleno}\label{#1}%
  \medskip Example \theexampleno :\smallskip}
\newcommand{\exampleref}[1]{\texorhtml{Example~\ref{#1}%
    \refstepcounter{examplerefno}\label{exref\theexamplerefno}%
    % This is a test whether r@exref... is defined.  If not, skip
    % anything with the \pageref macro.
    \catcode`\@=11%
    \expandafter\ifx\csname r@exref\theexamplerefno\endcsname\relax\else%
    \expandafter\ifx\csname r@#1\endcsname\relax\else%
    \bgroup\count100=\pageref{exref\theexamplerefno}%
    \count101=\pageref{#1}\ifnum\count100=\count101\else~%
    on page~\pageref{#1}\fi\egroup\fi\fi\xspace}%
  {\link{Example \ref{#1}}{#1}}}%
\T\setlength{\vcodeindent}{10pt}
\texorhtml{
%% LaTeX version
\newenvironment{textoutput}[1]{\ifx #1\relax%
  \medskip Output:\vspace{-\medskipamount}\else%
  \medskip #1\vspace{-\medskipamount}\fi%
  \begin{list}{}{\setlength{\leftmargin}{\vcodeindent}}\begin{ttfamily}\item}%
 {\end{ttfamily}\end{list}} %
}{%% Hyperlatex version
\newenvironment{textoutput}[1]{\xml{blockquote}Output:\\ \\ \xml{tt}}%
  {\xml{/tt}\xml{/blockquote}}%
\newenvironment{minipage}[4]{}{}
\newenvironment{ttfamily}{\xml{blockquote}\xml{tt}}%
  {\xml{/tt}\xml{/blockquote}}}

\newcommand{\DAPOverviewTitle}{DAP Specification Overview}
\newcommand{\DAPOverview}{\xlink{\textbf{\textit{\DAPOverviewTitle}}}%
  {dap.html}\xspace}

% Old macros; new values
\newcommand{\DAPObjectsTitle}{The Data Access Protocol---DAP 2.0}
\newcommand{\DAPObjects}{\xlink{\textbf{\textit{\DAPObjectsTitle}}}%
  {http://www.opendap.org/pdf/ESE-RFC-004v0.06.pdf}\xspace}
  
\newcommand{\DAPHTTPTitle}{Using DAP 2.0 with HTTP}
\newcommand{\DAPHTTP}{\xlink{\textbf{\textit{\DAPHTTPTitle}}}%
  {daph.html}\xspace}

% New macros.
\newcommand{\DAPDataModelTitle}{DAP Data Model Specification}
\newcommand{\DAPDataModel}{\xlink{\textbf{\textit{\DAPDataModelTitle}}}%
  {dapo.html}\xspace}
  
\newcommand{\DAPWebTitle}{DAP Web Services Specification}
\newcommand{\DAPWeb}{\xlink{\textbf{\textit{\DAPWebTitle}}}%
  {daph.html}\xspace}
  

\newcommand{\DAPASCIITitle}{DAP Formatted Data Specification}
\newcommand{\DAPASCII}{\xlink{\textbf{\textit{\DAPASCIITitle}}}%
  {dapa.html}\xspace}
\newcommand{\DAPHTMLTitle}{DAP HTTP Query Specification}
\newcommand{\DAPHTML}{\xlink{\textbf{\textit{\DAPHTMLTitle}}}%
  {dapm.html}\xspace}

% It probably doesn't matter what we call the macro, but SOAP does not have
% to run over HTTP and I think that's going to be important for other groups.
% 10/27/03 jhrg
\newcommand{\DAPServicesTitle}{DAP HTTP Services Specification}
\newcommand{\DAPServices}{\xlink{\textbf{\textit{\DAPServicesTitle}}}%
  {daps.html}\xspace}

\newcommand{\thirtytwobitlimit}[1]{$4,294,967,296$ #1 ($2^{32}$)}
%%% Local Variables: 
%%% mode: latex
%%% TeX-master: t
%%% End: 

\figpath{dap_figs}

\title{\DAPServicesTitle\\ DRAFT}
\htmltitle{\DAPServicesTitle\ -- DRAFT}
\author{James Gallagher\thanks{The University of Rhode Island,
    jgallagher@gso.uri.edu}, Tom Sgouros}
\date{Printed \today \\ Revision \rcsInfoRevision}
\htmladdress{James Gallagher <jgallagher@gso.uri.edu>, 
  \rcsInfoDate, Revision: \rcsInfoRevision}
\htmldirectory{html}
\htmlname{daps}

\begin{document}

\bibliographystyle{plain}

\maketitle
\T\tableofcontents

\section{HTTP Services}


\section{Requests}

\subsection{Responses}
\label{sec-resp-bodies}


The \INFO, \HTML or \DIR services all return HTML or plain
text documents. The \TEXT service returns data in a plain text document so
that it may be easily read into a spreadsheet or similar program. The \INFO
service formats information contained in the \DDS and \DAS objects
using HTML so that it may be displayed in a WWW browser. The \HTML
service returns a dynamic WWW form that can be used to request specific
variables from the data source. This service provides a low-level user
interface to a data source. The \DIR service provides a rudimentary form
of navigation for data sources which are composed of multiple files.

\subsubsection{\INFO}
\label{sec-info}

The \INFO response MUST present the user with an HTML document that
contains the dataset-specific information in the \DAX, as well as the
server-specific information in the \VER response and the \CapX.  This
intent is to present this information in a way that can be rendered by
a WWW browser.  Structural information about the dataset SHOULD be
preserved so that users can build \DAP URLs by hand.

The \INFO response MAY also return other information. If server
installers and/or dataset maintainers add HTML which describes the
dataset or the server, this information MUST be merged into the HTML
document returned in response to the \INFO request.

The purpose of the \INFO response is to supply information about the
dataset in a form that is easy for people to read. It should be
structured so that after only a little experience people can easily
assess a dataset using the document. In addition, the \INFO response
may be used in creating various user interfaces which access data
using \DAP servers.

The \INFO response should provide:

\begin{enumerate}
\item The hierarchical relation of container variables.
\item Each variable's datatype.
\item Each variable's attributes.
\item Any global attributes that the dataset contains.
\item Extra information supplied by the dataset creator/maintainer.
\item \CE functions available on this server.
\item The information from the \VER response.
\end{enumerate}

One implementation of the \INFO response uses HTML text in files
corresponding to each dataset served by a server, as well as HTML
documenting the server as a whole.  These documents are merged on the
fly to create the \INFO response, providing information both about the
server and the specific datasets in question.  

The \INFO response is triggered by a client using \lit{.info} as the
URL suffix.

\begin{textoutput}{Required Headers:}
XDODS-Server: dods/4.0\\
Content-Type: text/html\\
Date: \emph{date}
\end{textoutput}


\subsubsection{\HELP}
\label{sec-help}

The \HELP response MUST be returned when the server receives a URL
whose Dataset ID portion is \lit{help}.  It MAY reply with the \HELP
response when it receives a URL with no extension (i.e., a URL with no
Dataset ID at all).  This second option can interfere with the \DIR
response (see \Sectionref{sec-dir}) if your server keeps datasets in
the root-level directory.  In this case, the \DIR response MUST be
returned instead of the \HELP response in reply to a URL without a
Dataset ID.

\begin{ttfamily}
\begin{center}
\begin{tabular}{lll}
abs\_path & = & server\_path dataset\_id "." ext [ "?" query ] \\
server\_path & = & <name of DAP server> \\
dataset\_id & = & "help" \\
ext & = & "" \\
\end{tabular}
\end{center}
\end{ttfamily}

The \HELP response MUST return an HTML document which lists
the extensions recognized by the server. The response MAY return other
information as well.

\begin{textoutput}{Required Headers:}
XDODS-Server: dods/4.0\\
Content-Type: text/html\\
Date: \emph{date}
\end{textoutput}


%\printgloss{dods-glossary}

\T\addcontentsline{toc}{section}{References}
\T\raggedright
\bibliography{../../../boiler/dods}

\appendix

\section{Notational Conventions and Generic Grammar}
\label{app:grammar}

% This was taken verbatim from rfc2616. The original section title was
% `Notational Conventions and Generic Grammar' 3/21/2001 jhrg

\subsection{Augmented BNF}
All of the mechanisms specified in this document are described in both prose
and an augmented Backus-Naur Form (BNF) similar to that used by RFC
822~\cite{rfc822}. Implementors will need to be familiar with the notation in
order to understand this specification. The augmented BNF includes the
following constructs:

\begin{description}
  
\item [\texttt{name = definition}] The name of a rule is simply the name
  itself (without any enclosing \texttt{"$<$"} and \texttt{"$>$"}) and is
  separated from its definition by the equal \texttt{"="} character. White
  space is only significant in that indentation of continuation lines is used
  to indicate a rule definition that spans more than one line. Certain basic
  rules are in uppercase, such as SP, LWS, HT, CRLF, DIGIT, ALPHA, etc. Angle
  brackets are used within definitions whenever their presence will
  facilitate discerning the use of rule names.
  
\item [\texttt{"literal"}] Quotation marks surround literal text. Unless
  stated otherwise, the text is case-insensitive.
      
\item [\texttt{rule1 | rule2}] Elements separated by a bar (\texttt{"|"}) are
  alternatives, e.g., \texttt{"yes | no"} will accept \texttt{yes} or
  \texttt{no}.
  
\item [\texttt{(rule1 rule2)}] Elements enclosed in parentheses are treated
  as a single element.  Thus, \texttt{"(elem (foo | bar) elem)"} allows the
  token sequences \texttt{"elem foo elem"} and \texttt{"elem bar elem"}.
  
\item [\texttt{*rule}] The character \texttt{"*"} preceding an element
  indicates repetition. The full form is \texttt{"$<$n$>$*$<$m$>$element"}
  indicating at least \texttt{$<$n$>$} and at most \texttt{$<$m$>$}
  occurrences of element. Default values are 0 and infinity so that
  \texttt{"*(element)"} allows any number, including zero;
  \texttt{"1*element"} requires at least one; and \texttt{"1*2element"}
  allows one or two.
  
\item [\texttt{[rule]}] Square brackets enclose optional elements;
  \texttt{"[foo bar]"} is equivalent to \texttt{"*1(foo bar)"}.
  
\item [\texttt{N rule}] Specific repetition: \texttt{"$<$n$>$(element)"} is
  equivalent to \texttt{"$<$n>*$<$n$>$(element)"}; that is, exactly
  \texttt{$<$n$>$} occurrences of (element).  Thus 2DIGIT is a 2-digit
  number, and 3ALPHA is a string of three alphabetic characters.
  
\item [\texttt{\#rule}] A construct \texttt{"\#"} is defined, similar to
  \texttt{"*"}, for defining lists of elements. The full form is
  \texttt{"$<$n$>$\#$<$m$>$element"} indicating at least \texttt{$<$n$>$} and
  at most \texttt{$<$m$>$} elements, each separated by one or more commas
  (\texttt{","}) and OPTIONAL linear white space (LWS). This makes the usual
  form of lists very easy; a rule such as \texttt{( *LWS element *( *LWS ","
    *LWS element ))} can be shown as \texttt{1\#element} Wherever this
  construct is used, null elements are allowed, but do not contribute to the
  count of elements present. That is, \texttt{"(element), , (element) "} is
  permitted, but counts as only two elements. Therefore, where at least one
  element is required, at least one non-null element MUST be present. Default
  values are 0 and infinity so that \texttt{"\#element"} allows any number,
  including zero; \texttt{"1\#element"} requires at least one; and
  \texttt{"1\#2element"} allows one or two.
  
\item [\texttt{;} comment] A semi-colon, set off some distance to the right
  of rule text, starts a comment that continues to the end of line. This is a
  simple way of including useful notes in parallel with the specifications.
  
\item [implied \texttt{*LWS}] The grammar described by this specification is
  word-based. Except where noted otherwise, linear white space (LWS) can be
  included between any two adjacent words (token or quoted-string), and
  between adjacent words and separators, without changing the interpretation
  of a field. At least one delimiter (LWS and/or separators) MUST exist
  between any two tokens (for the definition of "token" below), since they
  would otherwise be interpreted as a single token.
\end{description}

\subsection{Basic Rules}

   The following rules are used throughout this specification to
   describe basic parsing constructs. The US-ASCII coded character set
   is defined by ANSI X3.4-1986~\cite{ANSI:US-ASCII}.

\begin{vcode}{it}
       OCTET          = <any 8-bit sequence of data>
       CHAR           = <any US-ASCII character (octets 0 - 127)>
       UPALPHA        = <any US-ASCII uppercase letter "A".."Z">
       LOALPHA        = <any US-ASCII lowercase letter "a".."z">
       ALPHA          = UPALPHA | LOALPHA
       DIGIT          = <any US-ASCII digit "0".."9">
       CTL            = <any US-ASCII control character
                        (octets 0 - 31) and DEL (127)>
       CR             = <US-ASCII CR, carriage return (13)>
       LF             = <US-ASCII LF, linefeed (10)>
       SP             = <US-ASCII SP, space (32)>
       HT             = <US-ASCII HT, horizontal-tab (9)>
       <">            = <US-ASCII double-quote mark (34)>
\end{vcode}

HTTP/1.1 defines the sequence CR LF as the end-of-line marker for all
protocol elements except the entity-body (see Appendix 19.3 of RFC
2616[9] for tolerant applications). The end-of-line marker within an
entity-body is defined by its associated media type, as described in
Section 3.7 of RFC 2616[9].

% My original text:
% except the entity-body (see appendix
%    19.3\footnote{In RFC 2616\cite{rfc2616}.} for
%    tolerant applications). The end-of-line marker within an entity-body
%    is defined by its associated media type, as described in section
%    3.7.\footnote{In RFC 2616\cite{rfc2616}.}
% The replacement text above was suggested by Allan Doyle, 20 April
% 2005.

\begin{vcode}{it}
       CRLF           = CR LF
\end{vcode}

   HTTP/1.1 header field values can be folded onto multiple lines if the
   continuation line begins with a space or horizontal tab. All linear
   white space, including folding, has the same semantics as SP. A
   recipient MAY replace any linear white space with a single SP before
   interpreting the field value or forwarding the message downstream.

\begin{vcode}{it}
       LWS            = [CRLF] 1*( SP | HT )
\end{vcode}

   The TEXT rule is only used for descriptive field contents and values
   that are not intended to be interpreted by the message parser. Words
   of *TEXT MAY contain characters from character sets other than ISO-
   8859-1 [22] only when encoded according to the rules of RFC 2047
   [14].

\begin{vcode}{it}
       TEXT           = <any OCTET except CTLs,
                        but including LWS>
\end{vcode}

   A CRLF is allowed in the definition of TEXT only as part of a header
   field continuation. It is expected that the folding LWS will be
   replaced with a single SP before interpretation of the TEXT value.

   Hexadecimal numeric characters are used in several protocol elements.

\begin{vcode}{it}
       HEX            = "A" | "B" | "C" | "D" | "E" | "F"
                      | "a" | "b" | "c" | "d" | "e" | "f" | DIGIT
\end{vcode}

   Many HTTP/1.1 header field values consist of words separated by LWS
   or special characters. These special characters MUST be in a quoted
   string to be used within a parameter value (as defined in section
   3.6).

\begin{vcode}{it}
       token          = 1*<any CHAR except CTLs or separators>
       separators     = "(" | ")" | "<" | ">" | "@"
                      | "," | ";" | ":" | "\" | <">
                      | "/" | "[" | "]" | "?" | "="
                      | "{" | "}" | SP | HT
\end{vcode}

   Comments can be included in some HTTP header fields by surrounding
   the comment text with parentheses. Comments are only allowed in
   fields containing "comment" as part of their field value definition.
   In all other fields, parentheses are considered part of the field
   value.

\begin{vcode}{it}
       comment        = "(" *( ctext | quoted-pair | comment ) ")"
       ctext          = <any TEXT excluding "(" and ")">
\end{vcode}

   A string of text is parsed as a single word if it is quoted using
   double-quote marks.

\begin{vcode}{it}
       quoted-string  = ( <"> *(qdtext | quoted-pair ) <"> )
       qdtext         = <any TEXT except <">>
\end{vcode}

   The backslash character ("\verb+\+") MAY be used as a single-character
   quoting mechanism only within quoted-string and comment constructs.

\begin{vcode}{it}
       quoted-pair    = "\" CHAR
\end{vcode}

\begin{quote}
This appendix was copied from RFC 2616~\cite{rfc2616}. The copyright
from that document reads:

\begin{quote}
  Copyright (C) The Internet Society (1999).  All Rights Reserved.

   This document and translations of it may be copied and furnished to
   others, and derivative works that comment on or otherwise explain it
   or assist in its implementation may be prepared, copied, published
   and distributed, in whole or in part, without restriction of any
   kind, provided that the above copyright notice and this paragraph are
   included on all such copies and derivative works.  However, this
   document itself may not be modified in any way, such as by removing
   the copyright notice or references to the Internet Society or other
   Internet organizations, except as needed for the purpose of
   developing Internet standards in which case the procedures for
   copyrights defined in the Internet Standards process must be
   followed, or as required to translate it into languages other than
   English.
\end{quote}

\end{quote}


\section{Acronyms and Abbreviations}
\begin{acronym}
%
% Make one entry per line, even if they are long lines that wrap and look
% ugly. This makes it simple to sort the list using emacs' sort-lines
% command. 3/27/2000 jhrg
%
% $Id$
\acro{AS}{Aggregation Server}
\acro{BNF} {Backus-Naur Form}
\acro{CE}{Constraint Expression}
\acro{CGI}{Common Gateway Interface}
\acro{COARDS}{Cooperative Ocean/Atmosphere Research Data Service}
\acro{CSV}{Comma Separated Values}
\acro{DAP}{Data Access Protocol}
\acro{DAS}{Dataset Attribute Structure}
\acro{DDS}{Dataset Descriptor Structure}
\acro{DODS}{Distributed Oceanographic Data System}, See the DODS home page: \texttt{http://\-unidata.ucar.edu\-/packages\-/dods/}
\acro{DataDDS}{Data Dataset Descriptor Structure}
\acro{FGDC}{Federal Geographic Data Community}
\acro{HTML}{Hypertext Markup Language}
\acro{HTTP}{HyperText Transfer Protocol}
\acro{MIME}{Multimedia Internet Mail Extensions}
\acro{SRS}{Software Requirements Specification}, See IEEE 830--1998
\acro{URI}{Uniform Resource Identifiers}
\acro{URL}{Uniform Resource Locator}
\acro{W3C}{The World Wide Web Consortium}, See http://www.w3c.org/
\acro{WWW}{The World Wide Web}
\acro{XDR}{External Data Representation}
\acro{XML}{Extensible Markup Language}
%%% Local Variables: 
%%% mode: latex
%%% TeX-master: t
%%% End: 

\end{acronym}

\section{Change log}

\begin{verbatim}
$Log: dap_services.tex,v $
Revision 1.8  2003/07/24 22:32:12  tom
excised dap_services document references

Revision 1.7  2003/07/16 01:06:08  tom
progress on comments, fixed titles

Revision 1.6  2003/05/23 21:50:52  tom
progress made

Revision 1.5  2003/05/23 19:27:42  tom
new files, rearranging DAP spec into separate documents

Revision 1.4  2003/03/19 21:58:53  tom
loosened auth requirements

Revision 1.3  2003/03/19 21:48:06  tom
progress made, ready for the March 03 DODS mtg

Revision 1.2  2003/03/17 17:45:10  tom
progress made.  draft for discussion 3/18/03

Revision 1.1  2003/01/14 19:55:31  jimg
Added.

\end{verbatim}

\end{document}

% \section{Encoding Values}
% \label{sec-rep-of-values}

% \emph{[Describe different types of variables as three groups: scalar, vector
%   and aggregate]}

% The \DAP uses Sun Microsystems' XDR protocol~\cite{xdr} for the external
% representation of all of the base type variables. Table~\ref{tab-base-xdr}
% shows the XDR types used to represent the various base type
% variables.

% \begin{table}
% \caption{The XDR data types used by the DAP as the external representations
%   of base-type variables}
% \label{tab-base-xdr}
% \begin{center}
% \begin{tabular}{|l|l|} \hline
% \multicolumn{1}{|c}{\textsc{Base Type}} & \multicolumn{1}{c|}{\textsc{XDR Type}} \\
% \hline \hline
% \lit{byte} & \lit{xdr byte} \\ \hline
% \lit{int16} & \lit{xdr short} \\ \hline
% \lit{uint16} & \lit{xdr unsigned short} \\ \hline
% \lit{int32} & \lit{xdr long} \\ \hline
% \lit{uint32} & \lit{xdr unsigned long} \\ \hline
% \lit{float32} & \lit{xdr float} \\ \hline
% \lit{float64} & \lit{xdr double} \\ \hline
% \lit{string} & \lit{xdr string} \\ \hline
% \lit{URL} & \lit{xdr string} \\ \hline
% \end{tabular}
% \end{center}
% \end{table}

% In order to transmit constructor type variables, the \DAP defines how the
% various base type variables, which comprise the constructor type variable,
% are transmitted. Any constructor type variable may be subject to a constraint
% expression which changes the amount of data transmitted for the variable (see
% Section~\ref{sec-ce}). For each of the six constructor types these
% definitions are:

% \begin{description}
  
% \item [Array] An array id first sent by sending the number of elements in the
%   array twice.\footnote{This is an artifact of the first implementation of
%     the DAP and XDR. The DAP software needed length information to allocate
%     memory for the array so it sent the array length. However, XDR also sends
%     the array length for its own purposes. This could be fixed but it is more
%     of an annoyance than anything else.} The array lengths are 32-bit
%   integers encoded using \lit{xdr\_long}.

%   Following the length information, each array element is encoded in
%   succession. Arrays of bytes are handled differently than other arrays:
% \begin{enumerate}
% \item An array of bytes: These are encoded as is and are padded to a
%   four-byte boundary. Thus an array of five bytes will be encoded as eight
%   bytes.
  
% \item One dimensional arrays of all other types are encoded by encoding each
%   element of the array in the order they appear.

% \item Multi0dimensional arrays are encoded by encoding the elements using
%   row-major ordering.
% \end{enumerate}

% \begin{ttfamily}
% \begin{center}
% \begin{tabular}{lll}
% Array & = & length length values \\
% length & = & <32-bit integer, signed, big endian> \\
% values & = & bytes | other-values \\
% bytes & = & <8-bit bytes padded to a four-byte boundary> \\
% other-values & = & numeric-values | strings | aggregates \\
% \end{tabular}
% \end{center}
% \end{ttfamily}

% \item [List] A list is sent as if it were an array. Even though the length of
%   a list is not declared, at the time the list's value(s) are to be sent, its
%   length must be known. Thus it is possible to think of a list as a vector of
%   values and hence use the same encoding for those values as would be used
%   for an equivalent array.

% \item [Structure] A structure is sent by encoding each field in the order
%   those fields are declared in the structure. For example, the structure:

% \begin{vcode}{it}
% Structure {
%     int32 x;
%     float64 y;
% } a;
% \end{vcode}

% Would be sent by encoding the int32 \lit{x} and then the float64
% \lit{y}. 

% Nested structures are sent by encoding their `leaf nodes' as visited in a
% depth first traversal. For example:

% \begin{vcode}{it}
% Structure {
%     int32 x;
%     Structure {
%         String name;
%         Byte image[512][512];
%     } picture;
%     float64 y;
% } a;
% \end{vcode}

% Would be sent by encoding \lit{x}, then \lit{name}, \lit{image} and
%   finally \lit{y}.

% \item [Sequence] A Sequence is transmitted by encoding each instance as for a
%   structure and sending one after the other, in the order of their occurrence
%   in the data set. The entire sequence is sent, subject to the constraint
%   expression. In other words, if no constraint expression is supplied then
%   the entire sequence is sent. However, if a constraint expression is given
%   all the records in the sequence that satisfy the expression are
%   sent

%   Because a sequence does \emph{not} have a length count, each instance
%   is prefixed by a \lit{start of sequence} marker. Also, to accommodate
%   nested sequences, then end of each sequence as a whole is marked by a
%   \lit{end of sequence} marker.

% \begin{ttfamily}
% \begin{center}
% \begin{tabular}{lll}
% sequence & = & instances end-of-seq \\
% instances & = & start-of-inst instance-values \\
% end-of-seq & = & <byte value 0xA5> \\
% start-of-inst & = & <byte value 0x5A> \\
% \end{tabular}
% \end{center}
% \end{ttfamily}

% \item [Grid] A grid is encoded as if it is a Structure (one component
%   after the other, in the order of their declaration).

% \end{description}
