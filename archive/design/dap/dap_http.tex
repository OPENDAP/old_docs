%
% Specification for an HTTP implementation of the DAP
%

\documentclass[justify]{dods-paper}
\usepackage{acronym}
\usepackage{xspace}
\usepackage{gloss}
\rcsInfo $Id$

% latex and HTML macros. Some latex commands become nops for HTML. 4/10/2001
% jhrg 
\T\newcommand{\Cpp}{\rm {\small C}\raise.5ex\hbox{\footnotesize++}\xspace}
\T\newcommand{\C}{\rm {\small C}\xspace}
\W\newcommand{\Cpp}{C++}
\W\newcommand{\C}{C\xspace}
\W\newcommand{\cdots}{...}
\W\newcommand{\ddots}{}
\W\newcommand{\vdots}{.}
\W\newcommand{\pm}{+/-}
\W\newcommand{\times}{*}
\W\newcommand{\uppercase}[1]{\textsc{#1}}
\T\newcommand{\qt}{\lit{\char127}}
\W\newcommand{\qt}{"}

\texorhtml{\def\rearrangedate#1/#2/#3#4{\ifcase#2\or January\or February\or
  March\or April\or May\or June\or July\or August\or September\or
  October\or November\or December\fi\ \ifx0#3\relax\else#3\fi#4, #1}
\def\rcsdocumentdate{\expandafter\rearrangedate\rcsInfoDate}}%
{\HlxEval{
(put 'rearrangedate       'hyperlatex 'hyperlatex-ts-rearrange-date)

(defun hyperlatex-ts-rearrange-date ()
  (let ((date-string (hyperlatex-evaluate-string 
                       (hyperlatex-parse-required-argument))))
    (let ((year-string (substring date-string 0 4))
          (month-string (substring date-string 5 7))
          (day-string (substring date-string 8 10))
          (month-list '("January" "February" "March" "April"
                        "May" "June" "July" "August" 
                        "September" "October" "November" "December")))
       (insert
         (concat (elt month-list (1- (string-to-number month-string)))
                 " " (int-to-string (string-to-number day-string))
                 ", " year-string)))))
}
\newcommand{\rcsdocumentdate}{\rearrangedate{\rcsInfoDate}}}

\newcommand{\dapversion}{Version 4.0\xspace}
\newcommand{\opendap}{OPeNDAP\xspace}
\newcommand{\DAP}{DAP\xspace}
\newcommand{\DODS}{DODS\xspace}
\newcommand{\NVODS}{NVODS\xspace}
\newcommand{\CE}{constraint expression\xspace}
\newcommand{\CEs}{constraint expressions\xspace}
\newcommand{\ErrorX}{ErrorX\xspace}
\newcommand{\CapX}{Server Capabilities Document\xspace}
\newcommand{\Blob}{Blob\xspace}
\newcommand{\DDX}{DDX\xspace}
\newcommand{\DAX}{DAX\xspace}
\newcommand{\DDS}{DDS\xspace}
\newcommand{\DAS}{DAS\xspace}
\newcommand{\URI}{URI\xspace}
\newcommand{\DataDDS}{DataDDS\xspace}

\newcommand{\type}[1]{\emph{#1}}
\newcommand{\Alias}{\type{Alias}\xspace}
\newcommand{\Array}{\type{Array}\xspace}
\newcommand{\Attribute}{\type{Attribute}\xspace}
\newcommand{\ProcAttribute}{\type{Processing Attribute}\xspace}
\newcommand{\Map}{\type{Map}\xspace}
\newcommand{\Target}{\type{Target}\xspace}
\newcommand{\FQN}{fully qualified name\xspace}
\newcommand{\Grid}{\type{Grid}\xspace}
\newcommand{\Structure}{\type{Structure}\xspace}
\newcommand{\Dataset}{\type{Dataset}\xspace}
\newcommand{\Sequence}{\type{Sequence}\xspace}
\newcommand{\Container}{\type{Container}\xspace}
\newcommand{\Bi}{\type{Binary Image}\xspace}
\newcommand{\String}{\type{String}\xspace}
\newcommand{\URL}{\type{URL}\xspace}
\newcommand{\Boolean}{\type{Boolean}\xspace}
\newcommand{\Byte}{\type{Byte}\xspace}
\newcommand{\Enum}{\type{Enumeration}\xspace}
\newcommand{\Time}{\type{Time}\xspace}
\newcommand{\Function}{\type{Function}\xspace}
\newcommand{\Description}{\type{Description}\xspace}
\newcommand{\Parameter}{\type{Parameter}\xspace}
\newcommand{\Constraint}{\type{Constraint}\xspace}
\newcommand{\NoAttributes}{\type{NoAttributes}\xspace}
\newcommand{\Project}{\type{Project}\xspace}
\newcommand{\Select}{\type{Select}\xspace}
\newcommand{\Hyperslab}{\type{Hyperslab}\xspace}

\newcommand{\Aliases}{\type{Aliases}\xspace}
\newcommand{\Arrays}{\type{Arrays}\xspace}
\newcommand{\Attributes}{\type{Attributes}\xspace}
\newcommand{\ProcAttributes}{\type{Processing Attributes}\xspace}
\newcommand{\Maps}{\type{Maps}\xspace}
\newcommand{\Targets}{\type{Targets}\xspace}
\newcommand{\FQNs}{fully qualified names\xspace}
\newcommand{\Grids}{\type{Grids}\xspace}
\newcommand{\Structures}{\type{Structures}\xspace}
\newcommand{\Sequences}{\type{Sequences}\xspace}
\newcommand{\Containers}{\type{Containers}\xspace}
\newcommand{\Bis}{\type{Binary Images}\xspace}
\newcommand{\Strings}{\type{Strings}\xspace}
\newcommand{\URLs}{\type{URLs}\xspace}
\newcommand{\Booleans}{\type{Booleans}\xspace}
\newcommand{\Bytes}{\type{Bytes}\xspace}
\newcommand{\Enums}{\type{Enumerations}\xspace}
\newcommand{\Times}{\type{Times}\xspace}
\newcommand{\CSVs}{comma-separated values\xspace}
\newcommand{\Functions}{\type{Functions}\xspace}
\newcommand{\Descriptions}{\type{Descriptions}\xspace}
\newcommand{\Parameters}{\type{Parameters}\xspace}
\newcommand{\Constraints}{\type{Constraints}\xspace}
\newcommand{\Projects}{\type{Projects}\xspace}
\newcommand{\Selects}{\type{Selects}\xspace}
\newcommand{\Hyperslabs}{\type{Hyperslabs}\xspace}

\newcommand{\DIR}{\textbf{Directory}\xspace}
\newcommand{\TEXT}{\textbf{Text}\xspace}
\newcommand{\HTML}{\textbf{HTML}\xspace}
\newcommand{\HELP}{\textbf{Help}\xspace}
\newcommand{\VER}{\textbf{Version}\xspace}
\newcommand{\INFO}{\textbf{Info}\xspace}

%%%%%%%%%%%%%% Web Services Paper
\newcommand{\GetDDX}{\textbf{GetDDX}\xspace}
\newcommand{\GetData}{\textbf{GetData}\xspace}
\newcommand{\GetBlobData}{\textbf{GetBlobData}\xspace}
\newcommand{\GetBlob}{\textbf{GetBlob}\xspace}
\newcommand{\GetDir}{\textbf{GetDir}\xspace}
\newcommand{\GetInfo}{\textbf{GetInfo}\xspace}

\newcommand{\FSs}{\type{Foundation Services}\xspace}
\newcommand{\FS}{\type{Foundation Service}\xspace}
\newcommand{\ODSN}{\type{OPeNDAP}\xspace}

\newcommand{\blobdataelement}{\texttt{BlobData} element\xspace}
\newcommand{\blobelement}{\texttt{Blob} element\xspace}



%\setcounter{secnumdepth}{4}
%\setcounter{tocdepth}{4}

\newcommand{\Tableref}[1]{Table~\ref{#1}}%
\newcommand{\Figureref}[1]{Figure~\ref{#1}}%
\W\begin{iftex}
\newcommand{\Sectionref}[2]{Section~\ref{#1}%
  \ifx#2)%
    \ on page~\pageref{#1}%
  \else% 
    \ (page~\pageref{#1})\xspace%
  \fi\ifx#2\space\ \else #2\fi}%
\W\end{iftex}
\W\newcommand{\Sectionref}[1]{Section~\ref{#1}}
\W\newcommand{\raggedright}{}


%% Conveniences for documenting XML
\newcommand{\tag}[1]{\emph{#1}}
\newcommand{\element}[1]{\link{\tag{#1}}{sec-xml-#1}}
\newcommand{\attribute}[1]{\emph{#1}}
\newcommand{\currentelement}{}
\newcommand{\ELEMENT}[1]{\renewcommand{\currentelement}{#1 element}%
  \subsubsection{#1}\label{sec-xml-#1}\indc{\currentelement}%
  \indc{catalog tag!#1}\indc{aggregation tag!#1}%
  \indc{XML!#1 element}}
\newcommand{\ATTRIBUTE}[1]{\item{\lit{#1}}\indc{\currentelement!#1}
    \indc{#1 attribute!of \currentelement}%
    \indc{XML!#1 attribute}}

% Conveniences for examples
\newcounter{exampleno}
\setcounter{exampleno}{0}
\newcounter{examplerefno}
\setcounter{examplerefno}{0}
\newcommand{\examplelabel}[1]{\refstepcounter{exampleno}\label{#1}%
  \medskip Example \theexampleno :\smallskip}
\newcommand{\exampleref}[1]{\texorhtml{Example~\ref{#1}%
    \refstepcounter{examplerefno}\label{exref\theexamplerefno}%
    % This is a test whether r@exref... is defined.  If not, skip
    % anything with the \pageref macro.
    \catcode`\@=11%
    \expandafter\ifx\csname r@exref\theexamplerefno\endcsname\relax\else%
    \expandafter\ifx\csname r@#1\endcsname\relax\else%
    \bgroup\count100=\pageref{exref\theexamplerefno}%
    \count101=\pageref{#1}\ifnum\count100=\count101\else~%
    on page~\pageref{#1}\fi\egroup\fi\fi\xspace}%
  {\link{Example \ref{#1}}{#1}}}%
\T\setlength{\vcodeindent}{10pt}
\texorhtml{
%% LaTeX version
\newenvironment{textoutput}[1]{\ifx #1\relax%
  \medskip Output:\vspace{-\medskipamount}\else%
  \medskip #1\vspace{-\medskipamount}\fi%
  \begin{list}{}{\setlength{\leftmargin}{\vcodeindent}}\begin{ttfamily}\item}%
 {\end{ttfamily}\end{list}} %
}{%% Hyperlatex version
\newenvironment{textoutput}[1]{\xml{blockquote}Output:\\ \\ \xml{tt}}%
  {\xml{/tt}\xml{/blockquote}}%
\newenvironment{minipage}[4]{}{}
\newenvironment{ttfamily}{\xml{blockquote}\xml{tt}}%
  {\xml{/tt}\xml{/blockquote}}}

\newcommand{\DAPOverviewTitle}{DAP Specification Overview}
\newcommand{\DAPOverview}{\xlink{\textbf{\textit{\DAPOverviewTitle}}}%
  {dap.html}\xspace}

% Old macros; new values
\newcommand{\DAPObjectsTitle}{The Data Access Protocol---DAP 2.0}
\newcommand{\DAPObjects}{\xlink{\textbf{\textit{\DAPObjectsTitle}}}%
  {http://www.opendap.org/pdf/ESE-RFC-004v0.06.pdf}\xspace}
  
\newcommand{\DAPHTTPTitle}{Using DAP 2.0 with HTTP}
\newcommand{\DAPHTTP}{\xlink{\textbf{\textit{\DAPHTTPTitle}}}%
  {daph.html}\xspace}

% New macros.
\newcommand{\DAPDataModelTitle}{DAP Data Model Specification}
\newcommand{\DAPDataModel}{\xlink{\textbf{\textit{\DAPDataModelTitle}}}%
  {dapo.html}\xspace}
  
\newcommand{\DAPWebTitle}{DAP Web Services Specification}
\newcommand{\DAPWeb}{\xlink{\textbf{\textit{\DAPWebTitle}}}%
  {daph.html}\xspace}
  

\newcommand{\DAPASCIITitle}{DAP Formatted Data Specification}
\newcommand{\DAPASCII}{\xlink{\textbf{\textit{\DAPASCIITitle}}}%
  {dapa.html}\xspace}
\newcommand{\DAPHTMLTitle}{DAP HTTP Query Specification}
\newcommand{\DAPHTML}{\xlink{\textbf{\textit{\DAPHTMLTitle}}}%
  {dapm.html}\xspace}

% It probably doesn't matter what we call the macro, but SOAP does not have
% to run over HTTP and I think that's going to be important for other groups.
% 10/27/03 jhrg
\newcommand{\DAPServicesTitle}{DAP HTTP Services Specification}
\newcommand{\DAPServices}{\xlink{\textbf{\textit{\DAPServicesTitle}}}%
  {daps.html}\xspace}

\newcommand{\thirtytwobitlimit}[1]{$4,294,967,296$ #1 ($2^{32}$)}
%%% Local Variables: 
%%% mode: latex
%%% TeX-master: t
%%% End: 


\title{\DAPHTTPTitle\\ DRAFT}
\htmltitle{\DAPHTTPTitle\ -- DRAFT}
\author{James Gallagher\thanks{The University of Rhode Island,
    jgallagher@gso.uri.edu}, Tom Sgouros}
\date{Printed \today \\ Revision \rcsInfoRevision}
\htmladdress{James Gallagher <jgallagher@gso.uri.edu>, 
  \rcsInfoDate, Revision: \rcsInfoRevision}
\htmldirectory{html}
\htmlname{daph}

\begin{document}

\maketitle
\T\tableofcontents

\section{Introduction}

The \opendap system may be implemented using a wide variety of
communication protocols.  \DAP servers and clients may communicate
using HTTP, FTP, GridFTP, or any other modern socket protocol.  No
matter what communication protocol is used, all implementations of the
\DAP must adhere to the specification given in \DAPObjects.  

The current document specifies the behavior of only the HTTP
implementation of the \DAP.  That is, \DAP servers and clients that
communicate using HTTP must adhere to the specifications in this
document.  The \opendap project supports only this protocol
specification.  Other protocol specifications may be published by the
groups publishing the first implementation using that protocol.

In the remainder of this document, we use \DAP to signify the HTTP
implementation of the \DAP.  This is strictly a convenience, and is
not meant to imply anything about other possible implementations.

The \DAP works like any other client/server system.  A client sends
requests to a server, and the server responds to those requests with
one of the \DAP objects or with a service response.  The data objects
and the services that are to be supported by all \DAP implementations
are described in the \DAPObjects document.

In the next two sections, we look in detail at the form of
the requests and the responses for the HTTP implementation of the
\DAP. 


\section{Requests}
\label{sec-requests}

A \DAP client sends a request to a server using HTTP.  The request
consists of an HTTP GET request method, a \URI~\cite{rfc2396} that
encodes information specific to the \DAP and an HTTP protocol version
number followed by a MIME-like message containing various headers that
further describe the request. In practice, DAP clients use a
third-party library implementation of HTTP/1.1 so the GET request,
\URI and HTTP version information are hidden from the client; it sees
only the \DAP URL and some of the request headers.  

\subsection{The URL}
\label{sec-url}

A URL is simply a unique name for some Internet resource.
\Figureref{fig-url-parts} shows the parts of a typical \DAP URL.

\begin{figure}[h]
\texorhtml
{\begin{center}\small
${}\overbrace{\tt http}^{\textnormal{Protocol}}://
\overbrace{\tt dods.gso.uri.edu}^{\textnormal{Machine Name}}/
\overbrace{\tt cgi-bin/nph-nc}^{\textnormal{Server}}/
\overbrace{\tt data}^{\textnormal{Directory}}/
\overbrace{\tt fnoc1.nc}^{\textnormal{Filename}}
\overbrace{\tt .dods}^{\textnormal{URL Suffix}}?
\overbrace{\tt sst[1:5][1:5]}^{\textnormal{Constraint}}$
\end{center}}
{\begin{vcode}{cb}
http://dods.gso.uri.edu/cgi-bin/nph-nc/data/fnoc1.nc.dods?sst[1:5][1:5]
   ^            ^               ^      ^    ^        ^     ^
   |            |               |      |    |        |     |
Protocol        |               |      |    |        |     |
Machine Name-----               |      |    |        |     |
Server---------------------------      |    |        |     |
Directory-------------------------------    |        |     |
Filename-------------------------------------        |     |
URL Suffix--------------------------------------------     |
Constraint expression---------------------------------------
\end{vcode}}
\caption{Parts of a \DAP URL}
\label{fig-url-parts}
\end{figure}

The parts of the URL are:

\begin{description}
  
\item[Protocol] The communication protocol between the client and the
  server.  For the HTTP implementation of the \DAP, this is always
  \lit{http}.

\item[Host] The IP number or name of the host machine running the \DAP
  server.
  
\item[Server] Many \DAP servers are simply a set of CGI scripts
  executed on demand by the \lit{httpd} server.  Here, the server is
  represented by a CGI script called \lit{nph-nc}.  Many HTTP servers
  parse the headers of incoming CGI requests.  The HTTP specification
  dictates that CGI programs whose name begins with \lit{nph-} will
  receive their data intact.  (See note on page~\pageref{note-1}.)
  Many \DAP implementations take advantage of this feature, since it
  removes dependence on any particular HTTP server software.
  
\item[Directory \& Filename] The \DAP is fundamentally a file-based
  protocol.  To the right of the server (CGI) name comes a directory
  and file name identifying the file containing the requested data.
  These together make up the \new{Dataset ID}.  \DAP implementations
  that serve data not stored in files can omit the directory names, or
  use them to organize their data.

\item[URL suffix] The \DAP uses the suffix on the file name to
  determine what sort of request this is.  \Tableref{tab-url-suffix}
  has a list of the suffixes and the kinds of responses they 
  summon.  There are a variety of services available from most \DAP
  HTTP servers.  These include getting the data as formatted text,
  getting information about the server and data available from it, 
  browsing available data files, and others.   Requests for these are
  indicated by different URL suffixes.  See \DAPASCII, \DAPHTML, or
  \Sectionref{sec-help} of this document for information about these
  services. 

  \begin{table}[htbp]
    \begin{center}
      \begin{tabular}[t]{lp{3in}}
        \tblhd{Suffix} & \tblhd{Meaning} \\ \hline
        \lit{.info} & Information about a \DAP server and its data \\ \hline
        \lit{.asc}  & Formatted data (\DAPASCII.)\\ \hline
        \lit{.html} & Data review form (\DAPHTML.) \\ \hline
        \lit{.ver} & Version information (\Sectionref{sec-version})\\ \hline
        \lit{.dods} & Data (\DDX and \Blob) \\ \hline
        \lit{.das}, \lit{.dds} & \DDX \\ \hline
      \end{tabular}
      \caption{URL suffixes and their meaning}
      \label{tab-url-suffix}
    \end{center}
  \end{table}
  
  Most dedicated \DAP clients hide the suffixes from the user.  A user
  would enter the \DAP URL up to, but not including, the suffix, and
  the client software appends the suffix before sending it to the
  server.  It is convenient to know the suffixes, however, because
  of the \DAP services that provide responses that can be displayed by a
  standard web server.

\item[Constraint] The constraint expression is a way to subsample a
  \DAP dataset.  Constraint expression basics are covered in
  \DAPObjects, and the way to express a constraint expression in a
  request to an HTTP \DAP server is in \Sectionref{sec-ce-http}.

\end{description}

The URL in \Figureref{fig-url-parts} shows a client
request to the \lit{httpd} server on the machine
\lit{dods.gso.uri.edu}, for a netCDF dataset (specified by the
\lit{nph-nc} in the \lit{cgi-bin} directory) contained in a file
called \lit{fnoc1.nc}.  Upon receiving this URL, the \lit{httpd}
server executes the specified \DAP server module (\lit{nph-nc}), which
retrieves the file \lit{fnoc.nc} in a directory called \lit{data} relative to
wherever the \lit{httpd} server looks for its data.

\label{note-1}
\note{The only part of the URL whose spelling is not at the discretion of
the administrator of the host machine is the \lit{http}, and the
\lit{nph-} at the beginning of the CGI script name. Even the \lit{nc},
indicating netCDF, can be changed, although for clarity's sake, we
hope people won't do so.  Incidentally, the \lit{nph-} is a relic,
dating from the early days of the World Wide Web and the first
hypertext protocol standards.  It stands for ``Non-Parsing Header''
(See the CGI 1.1 Standard for more information.), and is the only way
to pass data through many httpd servers unparsed.  The \opendap
software uses this to make the server software portable between HTTP
servers.  If this is not important to you, and you know how to keep
your server from trying to parse the data passing through it, then
this is not essential.}



\subsection{Request Headers}


The headers described in this section MUST be handled as described.
Other headers which are part of HTTP 1.1 MAY be included in the
request and MAY be honored by a \DAP server.

\subsubsection{Accept-Encoding}
\label{sec-accept-encoding}

The \lit{Accept-Encoding} request-header is used by a \DAP client to
tell a server that it can accept compressed responses. See RFC
2616~\cite{rfc2616} for this header's grammar.  Values for encodings are
\lit{deflate}, \lit{gzip} and \lit{compress}. This header is
OPTIONAL. When a client includes this header it is effectively asking the
\DAP server to encode the response using the given scheme. The server is
under no obligation to use the requested encoding. However, a server MUST NOT
use an encoding when a client has not requested it. In addition, a client
MUST supply the header with every request for which it desires a special
encoding.

\subsubsection{Host}
\label{sec-host}

The \lit{Host} request-header is used by a \DAP client to provide its
IP address or DNS name to the \DAP server. See RFC 2616~\cite{rfc2616}
for this header's grammar. This header MUST be included with every request.

\subsubsection{User-Agent}
\label{sec-user-agent}

The \lit{User-Agent} request-header is used by a \DAP client to
provide specific information about the client software to the \DAP
server. See RFC 2616~\cite{rfc2616} for this header's grammar. This header is
RECOMMENDED. \DAP servers MAY log this information.

\subsubsection{XDODS-Accept-Types}
\label{sec-accept-types}

The \lit{XDODS-Accept-Types} experimental request-header is used by a
\DAP client to tell a \DAP server which of the standard \DAP
datatypes the client can understand. The server SHOULD NOT return datatypes
to the client that it does not understand. This header is OPTIONAL; if absent
a server can assume that a client understands all the standard datatypes. 

\begin{ttfamily}
\begin{center}
\begin{tabular}{lll}
XDODS-Accept-Types & = & [ "!" ] type\_list | all \\
type\_list & = & type | type "," type\_list \\
all & = & "All" \\
type & = & "Byte" | "Int16" | "UInt16" | "Int32" \\
       & & | "UInt32" | "Float32" | "Float64" \\
       & & | "String" | "URL" | "Array" | "List" \\ 
       & & | "Structure" | "Sequence" | "Grid" \\
\end{tabular}
\end{center}
\end{ttfamily}

A \DAP server determines which datatypes are understood by a client using
the following rules:
\begin{enumerate}
\item If the value of the header is \lit{All}, then the client understands
  all the datatypes.
\item If a datatype is listed in the value-string, then that type is
  understood, otherwise it is not understood.
\item If the value-string starts with the literal \lit{!} then the type(s)
  listed are \emph{not} understood by the client and the remaining types are
  understood. 
\end{enumerate}

\subsubsection{XDODS-Accept-Endian}
\label{sec-accept-endian}

This header is used to indicate to the server whether the client uses
big-endian or little-endian byte order.  The possible values are
\lit{big} and \lit{little}.  The \DAP specifies that transmission of
data is to be in big-endian format (see \DAPObjects), but
this header allows for more efficient transmission in the event that
both the client and the server are on machines with little-endian
architecture. 


\subsubsection{XDODS-Accept-FP}
\label{sec-accept-fp}

This header is used to specify that the client uses a specific
floating-point number encoding.  The \DAP specifies that transmission
of floating-point data use the IEEE 754 standard method.  But in the
event that both the client and server are on machines that use a
different encoding method, this header can allow for more efficient
transmission.  

\subsection{Constraint Expression Syntax}
\label{sec-ce-http}

The \CE, described in \DAPObjects, is an XML document containing
instructions for subsetting a data set served by a \DAP server.  Since
the HTTP implementation of the \DAP uses a GET request instead of a
POST, the \CE must actually be part of the \URL, instead of appended in
a data payload.  To that end, there is a translation of the XML \CE
syntax into a form containing no newlines or other forbidden characters.
A schematic representation of the syntax of an HTTP \CE is shown in
\Tableref{tab:ce}.

A \CE has two parts, the projection clause and the selection
clause(s). A \new{projection} lists the variables to be returned by
the \DAP server, and a \new{selection} selects values of those
variables from the complete set stored at the server.  For example, a
projection clause might specify that temperature is to be returned,
and a selection clause would specify that only \Sequence entries with
dates later than 1999 are to be examined.  The result returned from a
request like this would be a \Sequence of temperature measurements
taken after 1999.

A projection clause may specify more than one variable, in which case
the variable names are specified in a comma-separated list.  A
projection may also contain invocations of server-side functions,
which are to be executed by the server, and whose data will be
returned to the client in the response document.  If the
projection is omitted, the server will assume that all the variables
in the dataset are to be returned.

A selection clause is a series of conditions to be satisfied by the
data returned.  For example, one might request all the data taken by a
particular ship, or a particular scientist.  Selections only apply to
\Sequence data, but \Grid data can also be selected in the projection
clause.


\begin{table}[!t]
\label{tab:ce}
\caption{The syntax of a constraint expression.}
\begin{center}
\begin{tabular}{llp{3in}}
CE               & = & [\var{projection}] [\lit{\&} \var{selection}] \\ \\
\var{projection} & = & \var{variables} \\
\var{variables}  & = & \var{variable} $|$ \var{variable}\lit{,}\var{variables} \\
\var{variable}   & = & \var{name} $|$ \var{hyperslab} $|$ \var{function} \\
\var{function}   & = & \var{fname}\lit{(}\var{args}\lit{)} \\
\var{args}       & = & \var{arg} $|$ \var{arg}\lit{,}\var{args} \\
\var{arg}        & = & \var{name} $|$ \var{constant} \\ \\

\var{hyperslab}  & = & \var{name}\lit{[}\var{array-dim}\lit{]} \\
\var{array-dim}  & = & \var{start\lit{:}stride\lit{:}stop} $|$ \var{start\lit{:}stop} \\
\var{start, stride, end} & = & \var{integer} \\ \\

\var{selection}  & = & \var{conditions} \\
\var{conditions} & = & \var{condition} $|$ \var{condition}\lit{\&}\var{conditions} \\
\var{condition}  & = & \var{variable} \var{rel\_op} \var{constant} \\
\var{rel\_op}    & = & $\texttt{<}|\texttt{>}|\texttt{=}|\texttt{!=}$ \\ \\
\var{name}       & = & The name of a variable in the data set. \\
\var{fname}      & = & The name of a function provided by the server.
                       See the server's \CapX. \\
\var{constant}   & = & A signed numeral constant, may or may not
                       include a decimal point. \\ 
\var{integer}    & = & An unsigned numeral without a decimal point. \\
\end{tabular}
\end{center}
\end{table}

See \DAPObjects for a description of the behavior of a \CE, and
limitations on these constraint forms for certain data types.

\note{Due to the restrictions of URL syntax, there must not be any
  space characters in a URL.  If you want to use a space, say in a
  quoted function argument, you can use \lit{\%20} to substitute for
  the space character.  Similar escape expressions may be used for
  other non-printing characters.}

Following are \CE examples from \DAPObjects, accompanied by their
translation into the single-line http format.


First is a simple selection from a \Sequence.

\examplelabel{ce-trans-1}
\begin{vcode}{it}
<Constraint name="013.214">
  <Project variable="temp"/>
  <Project variable="salt"/>
  <Select variable="salt" condition=">" constant="34.0"/>
</Constraint>

...?temp,salt&salt>34.0
\end{vcode}

The hyperslab method allows you to reference segments of a \Grid,
\Array, or \Sequence by index.

\examplelabel{ce-trans-2}
\begin{vcode}{it}
<Constraint name="a347">
  <Project variable="sst">
    <Hyperslab dimension="lat" start="1" stop="10" stride="2"/>
    <Hyperslab dimension="lon" start="20" stop="40"/>
  </Project>
</Constraint>

...?sst[1:2:10][20:40]
\end{vcode}

A selection may also be applied to a \Grid, so long as the selections
correspond to the \Grid dimensions.

\examplelabel{ce-trans-3}
\begin{vcode}{it}
<Constraint name="a348">
  <Project variable="sst"/>
  <Select variable="lat" condition=">" constant="24.5"/>
  <Select variable="lon" condition="<" constant="-50.5"/>
</Constraint>

...?sst&lat>24.5&lon<-50.5
\end{vcode}

Multiple \Grids may be selected if they share some dimensions.

\examplelabel{ce-trans-3a}
\begin{vcode}{it}
<Constraint name="a348">
  <Project variable="temp"/>
  <Project variable="oxygen"/>
  <Select variable="lat" condition=">" constant="24.5"/>
  <Select variable="lon" condition="<" constant="-50.5"/>
</Constraint>

...?temp,oxygen&lat>24.5&lon<-50.5
\end{vcode}

Functions may appear in the projection or selection clause.  If they
appear as part of a selection, they must return a single (scalar) value.

\examplelabel{ce-trans-4}
\begin{vcode}{it}
<Constraint name="ralph">
  <Project function="make-sst">
    <Argument variable="raw-count"/>
    <Argument constant="223"/>
  </Project>
  <Select function="make-sst" condition=">" constant="25.0">
    <Argument variable="raw-count"/>
    <Argument constant="223"/>
  </Select>
</Constraint>

...?make-sst(223)&make-sst(223)>25.0
\end{vcode}

\subsection{Version Request}
\label{sec-version}

The \VER response returns information about the \DAP version, server
version and may return information about a dataset's version.  The
response may be requested two ways: by using the string \lit{version}
as the filename (omitting the directory) or by appending the extension
\lit{ver} to the dataset name (see \Sectionref{sec-url}).
\Tableref{tab:version} contains a description of the request syntax.

\begin{table}[!h]
\label{tab:version}
\caption{Syntax of a \VER request}
\begin{center}
\begin{tabular}{lll}
\var{abs\_path} & = & \var{server\_path}/\var{dataset\_id} \lit{.} \var{ext} \\
\var{server\_path} & = & name of DAP server (including CGI, if any)\\
\var{dataset\_id} & = & \lit{version} \\
\var{ext} & = & \lit{ver} \\
\end{tabular}
\end{center}
\end{table}

See \Sectionref{sec-version-response} for a description of the HTTP
\VER reply.

\subsection{Help Requests}
\label{sec-help}

There are two more required request forms for the HTTP implementation
of the \DAP: the \INFO request and the \HELP request.  

A \HELP request is intended to evoke information about the function of
a \DAP server, while an \INFO request includes that information with
information about a particular dataset served by that server.  The
syntax of a \HELP request is shown in \Tableref{tab:help}.

The \HELP response MUST be returned when the server receives a URL
whose Dataset ID portion is \lit{help}.  It MAY reply with the \HELP
response when it receives a URL with no extension (i.e., a URL with no
Dataset ID at all).  This second option can interfere with the \DIR
response (see \DAPHTML) if your server keeps datasets in
the root-level directory.  In this case, the \DIR response MUST be
returned instead of the \HELP response in reply to a URL without a
Dataset ID.

\begin{table}[!h]
\label{tab:help}
\caption{Syntax of a \HELP request}
\begin{center}
\begin{tabular}{lll}
\var{abs\_path} & = & \var{server\_path}/\var{dataset\_id} \\
\var{server\_path} & = & name of DAP server (including CGI, if any)\\
\var{dataset\_id} & = & \lit{help}
\end{tabular}
\end{center}
\end{table}

The syntax of an \INFO request is shown in \Tableref{tab:info}.

\begin{table}[!h]
\label{tab:info}
\caption{Syntax of an \INFO request}
\begin{center}
\begin{tabular}{lll}
\var{abs\_path} & = & \var{server\_path}/\var{dataset\_id} \lit{.} \var{ext} \\
\var{server\_path} & = & name of DAP server (including CGI, if any)\\
\var{dataset\_id} & = & directory and filename of some data set \\
\var{ext} & = & \lit{info} \\
\end{tabular}
\end{center}
\end{table}

\Sectionref{sec-help-response} describes the data returned by these
two requests

\subsection{Other Requests}

The requests (and their associated responses) described so far are the
ones that are either required or commonly available among \DAP
servers.  Two other responses are sometimes available, and the
\opendap project provides specifications for server implementators who
wish to provide those services.  These are the formatted data response
(also called the ASCII response), described in \DAPASCII, and the HTML
response, described in \DAPHTML.  Please refer to those documents for
information about those services.

\section{Responses}
\label{sec-responses}

A valid \DAP response has the same form as a valid HTTP response.  The
first line contains the HTTP protocol version, a status code and
reason phrase~\cite{rfc2616}. Following this are the response headers
which vary depending on the request and payload of the response (see
\Sectionref{sec-resp-headers} for a description of the headers). As
described in \cite{rfc822}, the HTTP response status line and headers
are separated from the response's payload by an extra set of CRLF
characters which make a blank line.

The \DAP response is the payload of the MIME-like HTTP response.  The
data objects that may be returned are described in detail in the
\DAPObjects.

\tbd{[Add some text describing how the three basic responses can be thought
  of as objects]}

\subsection{Response Headers}
\label{sec-resp-headers}

Like the requests, the HTTP response headers MAY contain all the
header fields specified in RFC 2616~\cite{rfc2616}.  Following is a
list of the header fields about which there is some \DAP{}-specific
behavior.  It is not a list of all the header fields used by the
\DAP.  In particular, servers are highly encouraged to implement the
\lit{Last-Modified} header field.

\subsubsection{Content-Description}

The \lit{Content-Description} header is used to tell clients which of
the different basic responses is being returned or if an error message
is being returned. For any of the basic responses (\DDX, \DAX, \Blob,
or the older \DDS, \DAS, or \DataDDS responses) or the error response
(\ErrorX object), this header MUST be included. This header SHOULD NOT
be included by the \TEXT, \HTML, \INFO, version or directory responses.
See RFC 2045~\cite{rfc2045} for information about this header.

\begin{ttfamily}
\begin{center}
\begin{tabular}{lll}
Content-Description & = & "Content-Description" ":" tag \\
tag & = & | "dap-ddx" | "dap-dax" | "dap-blob" | "dap-errorx" \\
 & & | "dods-dds" | "dods-das" | "dods-data" | "dods-error" \\
\end{tabular}
\end{center}
\end{ttfamily}

Example:

\begin{vcode}{it}
Content-Description: dods-error
\end{vcode}

\subsubsection{Content-Encoding}

If a \DAP server applies an encoding to an entity, it MUST include the
\lit{Content-Encoding} header in the response. See RFC
2616~\cite{rfc2616} for this header's grammar. The sole recognized
encoding for the \DAP is \lit{deflate}.

Example:

\begin{vcode}{it}
Content-Encoding: deflate
\end{vcode}

\subsubsection{Content-Type}

The \lit{Content-Type} header MUST be included in any response from a
\DAP server. Valid content types for \DAP responses are:
\lit{text/plain}, \lit{text/html} and \lit{application/octet}.  See
RFC 2616~\cite{rfc2616} for this header's grammar.

\emph{It would be better to use a multipart
  document in place of the \lit{application/octet}.} 

Example:

\begin{vcode}{it}
Content-Type: application/octet
\end{vcode}

\subsubsection{Date}

The \lit{Date} header provides a time stamp for the response. This header
is needed for servers that support caching. See RFC 2616~\cite{rfc2616}
for this header's grammar. Servers MUST provide this header.

Example:

\begin{vcode}{it}
Date: Fri, 09 Feb 2003 18:54:55 GMT
\end{vcode}

\subsubsection{Keep-Alive}

A \DAP server (or an underlying HTTP server if one is used to
implement the \DAP server) MAY return a \lit{Keep-Alive} header for an
authenticate (code 401) response. For all other responses, the \DAP
server MUST NOT return this header. See
RFC 2616~\cite{rfc2616} for this header's grammar.

\emph{This is a shortcoming of the \DAP. It should support HTTP/1.1's
  persistent connections. However, to do requires that the responses
  also return Content-Length. Since none of our servers do this, I've
  got no experience with persistent connections.}

\subsubsection{Server}

The \lit{Server} header provides information about the server used to
process the request. In this case the \emph{server} MAY be either the
\DAP server or an underlying HTTP server if the \DAP server uses
that as part of its implementation. See RFC 2616~\cite{rfc2616}
for this header's grammar. This header is OPTIONAL.

Example:

\begin{vcode}{it}
Server: Apache/1.3.12 (Unix)  (Red Hat/Linux) PHP/3.0.15 mod_perl/1.21
\end{vcode}

\subsubsection{WWW-Authentication}

The \lit{WWW-Authenticate} header MUST be included in a message that has a
response code of 401. That is, when the \DAP server is asked to provide
access to a resource that is restricted and the request does not include
authentication information (see ``HTTP Authentication: Basic and Digest
Access Authentication''~\cite{rfc2617}). then it must return with a response
code of 401 and include the \lit{WWW-Authenticate} header. See RFC
2616~\cite{rfc2616} for this header's grammar.

Example:

\begin{vcode}{it}
WWW-Authenticate: Basic realm="special directory, with CGIs"
\end{vcode}

\subsubsection{XDODS-Server}

The \lit{XDODS-Server} header is used to return a \DAP server's
implementation version information to the client program.  This header
MUST be included in every response. 

\begin{ttfamily}
\begin{center}
\begin{tabular}{lll}
XDODS-Server & = & "XDODS-Server" ":" "dods/" version [, text] \\
version & = & DIGIT "." DIGIT [ "." DIGIT ] \\
text & = & software implementation information, as a single line of text
\end{tabular}
\end{center}
\end{ttfamily}

Example:

\begin{vcode}{it}
XDODS-Server: dods/4.0.1, netCDF server 4.2.2
\end{vcode}

\subsection{Version Response}
\label{sec-version-response}

If a \DAP server receives a \lit{version} request (see
\Sectionref{sec-version}), it MUST return \DAP version
information. If the request is made using the \lit{ver} extension to a
dataset ID then the server MUST return the \DAP version and server
version information. It MAY also return a dataset version.

Version information should be returned as plain text in the payload of the
response. This version information may be essentially the same as the
information in the XDODS-Server header. The intent is to present users and
system maintainers with information about servers that can be used to track
down problems or determine if a server can be upgraded to a newer version to
fix a particular problem.

\begin{ttfamily}
\begin{center}
\begin{tabular}{lll}
version-response & = & dap-version CRLF server-version \\
                 & & [ CRLF dataset-version ] \\
dap-version & = & "Core version:" token "/" version-number \\
server-version & = & "Server version:" token "/" version-number \\
dataset-version & = & "Dataset version:" token "/" version-number \\
token & = & 1*<any CHAR except CTLs or separators> \\
version-number & = & 1*DIGIT "." 1*DIGIT "." 1*DIGIT \\
\end{tabular}
\end{center}
\end{ttfamily}

\begin{textoutput}{Required Headers:}
XDODS-Server: dods/4.0\\
Content-Type: text/plain\\
Date: \emph{date}
\end{textoutput}


\subsection{Help Responses}
\label{sec-help-response}

There are two more required request forms for the HTTP implementation
of the \DAP: the \INFO request and the \HELP request.  

\subsubsection{\INFO}
\label{sec-info}

The \INFO response MUST present the user with an HTML document that
contains the dataset-specific information in the dataset's \DDX, as
well as the server-specific information in the \VER response and the
\CapX.  The intent is to present this information in a way that can be
rendered by a WWW browser.  Structural information about the dataset
SHOULD be preserved so that users can build \DAP URLs by hand.

The \INFO response MAY also return other information. If server
installers and/or dataset maintainers add HTML which describes the
dataset or the server, this information MUST be merged into the HTML
document returned in response to the \INFO request.

The purpose of the \INFO response is to supply information about the
dataset in a form that is easy for people to read. It should be
structured so that after only a little experience people can easily
assess a dataset using the document. In addition, the \INFO response
may be used in creating various user interfaces which access data
using \DAP servers.

The \INFO response should provide:

\begin{enumerate}
\item The hierarchical relation of container variables.
\item Each variable's datatype.
\item Each variable's attributes.
\item Any global attributes that the dataset contains.
\item Extra information supplied by the dataset creator/maintainer.
\item \CE functions available on this server.
\item The information from the \VER response.
\end{enumerate}

One implementation of the \INFO response uses HTML text in files
corresponding to each dataset served by a server, as well as HTML
documenting the server as a whole.  These documents are merged on the
fly to create the \INFO response, providing information both about the
server and the specific datasets in question.  

The \INFO response is by URLs with the syntax shown in
\Sectionref{sec-help}.  The response headers are as follows:

\begin{textoutput}{Required Headers:}
XDODS-Server: dods/4.0\\
Content-Type: text/html\\
Date: \emph{date}
\end{textoutput}


\subsubsection{\HELP}

The \HELP response MUST return an HTML document which lists the
extensions recognized by the server.  The response MAY return other
information as well.  The intent is to provide a human-readable
document, to be displayed in a WWW browser, that provides instruction
in the use of the server that received the request.

\begin{textoutput}{Required Headers:}
XDODS-Server: dods/4.0\\
Content-Type: text/html\\
Date: \emph{date}
\end{textoutput}


\subsection{Other Responses}

To learn about the responses to HTML or formatted data (ASCII)
requests, see \DAPHTML or \DAPASCII. 


\subsection{Alternate Response Bodies}
\label{sec-resp-alt}

Earlier versions of the \DAP did not use the \DDX, \DAX, \Blob, and
\ErrorX objects in their current form.  Instead, the \DAP (then called
\DODS) used a C-style syntax in objects called the \DDS, \DAS,
\DataDDS.  The correspondence between the earlier formats and the
current XML descriptions is not exact.  The \DataDDS contained the
information now in the \DDX and \Blob, but without any of the
attribute information for the variables.  The \DDS and \DAS together
contained the information now in the \DAX.

There were a large number of \DAP clients released in these earlier
versions, and so an HTTP implementation of the \DAP SHOULD provide the
responses in this section.

\subsubsection{DAS}
\label{sec-das}
The \DAS entity is returned as the payload of a message whose
\lit{Content-Type} header MUST be \lit{text/plain}. The body of the
response contains text which holds all of the attribute containers and
values.

The \DAS is used to store attributes for both the entire dataset and
variables in the dataset. The \DAS consists of a number of
\emph{containers} each of which hold zero or more attributes. Each attribute
is a name-datatype-value tuple. Values may be either scalar or vector. Note
that two, three, \ldots, dimensional attribute values are \emph{not}
supported.  The name of the attribute container MUST be the same as
the name of the variable to which its attributes refer.
% added a MUST.  ts.

A \DAS MUST have a container for each variable in the dataset. It MAY
contain any number of extra containers.

\begin{ttfamily}
\begin{center}
\begin{tabular}{lll}
das-doc & = & "attributes" "{" *attribute-cont "}" \\
attribute-cont & = & attribute-cont | attribute \\
attribute & = & simple-decl id value *(, value) ";" \\
value & = & <Float> | <int> | id | <quoted string> \\
\end{tabular}
\end{center}
\end{ttfamily}

The purpose of attributes is to hold additional information beyond the name,
datatype and value of a variable and/or to hold extra information about a
dataset as a whole. This extra information can make the contents of the
dataset much easier to use. For example, extra information contained in the
\DAS might provide unit names, scaling factors or the missing-value
values. Additional information about an entire dataset might contain
information about who collected the information, under what circumstances
\emph{et cetera}. 

Many systems rely on attributes to store extra information that is necessary
to perform certain manipulations with data. In effect, attributes are used to
store information that is used `by convention' rather than `by design'.
\ac{DODS} can effectively support these conventions by passing the attributes
from data set to user program via the \DAS. Of course, \ac{DODS} cannot
enforce conventions in data sets where they were not followed in the first
place.

An example \DAS is shown in \Figureref{fig-das}.

\begin{figure}
\begin{vcode}{cb}
attributes {
   catalog_number {
   }
   casts {
      experimenter {
      }
      time {
         string units "hour since 0000-01-01 00:00:00";
         string time_origin "1-JAN-0000 00:00:00";
      }
      location {
         lat {
            string long_name "Latitude";
            string units "degrees_north";
         }
         lon {
            string long_name "Longitude";
            string units "degrees_east";
         }
      }
      xbt {
         depth {
            string units "meters";
         }
         t {
            float32 missing_value -9.99999979e+33;
            float32 _fillvalue -9.99999979e+33;
            string history "From coads_climatology";
            string units "Deg C";
         }
      }
   }
}
\end{vcode}
\caption{Example Dataset Attribute Entry. This example matches the DDS shown
   in \Figureref{fig-dds}. Some of the variables in this fictional dataset
   (e.g., \lit{catalog\_number}) have no attributes. }
\label{fig-das}
\end{figure}

\subsubsection{DDS}
\label{sec-dds}
The \DDS entity is returned as the payload of a message whose
\lit{Content-Type} header MUST be \lit{text/plain}. The body of the
response contains text which holds all of the variables, their names and
datatypes.

The DDS is a textual description of the variables and their names and
types that compose the entire data set. The data set descriptor syntax
is similar to the variable declaration syntax of C and \Cpp. A
variable that is a member of one of the base type classes is declared
by by writing the class name followed by the variable name. The type
constructor classes are declared using C's brace notation.

\begin{ttfamily}
\begin{center}
\begin{tabular}{lll}
dds-doc & = & "dataset" "\{" *type-decl "\}" id ";" \\
type-decl & = & simple-decl | list-decl | array-decl \\
          & & | structure-decl | sequence-decl | grid-decl \\
\end{tabular}
\end{center}
\end{ttfamily}

The \lit{dataset} keyword has the same syntactic function as
\lit{structure} but is used for the specific job of enclosing the entire
dataset even when it does not technically need an enclosing element (because
at the outermost level it is a single element such as a structure or
sequence).

An example DDS is shown in \Figureref{fig-dds}.

\begin{figure}
\begin{vcode}{cb}
dataset {
   int catalog_number;
   sequence {
      string experimenter;
      int32 time;
      structure {
         float64 latitude;
         float64 longitude;
      } location;
      sequence {
         float depth;
         float temperature;
      } xbt;
   } casts;
} data;
\end{vcode}
\caption{Example Dataset Descriptor Entry.}
\label{fig-dds}
\end{figure}


\subsubsection{DODS}
\label{sec-dods}

The \ac{DataDDS} entity is returned as the payload of a message whose
\lit{Content-Type} header MUST be \lit{application/octet}.  The body
of the response contains both text which holds a \DDS which describes
the variables listed in the request and the values for those variables
encoded using XDR\cite{xdr}. The text \DDS and the binary data are
separated in the response entity by the literal \lit{Data:}.

\begin{ttfamily}
\begin{center}
\begin{tabular}{lll}
DataDDS & = & DDS CR "Data:" CR <Binary values>
\end{tabular}
\end{center}
\end{ttfamily}

\begin{ttfamily}
\begin{center}
\begin{tabular}{lll}
simple-decl & = & simple-type id ";" \\
simple-type & = & "byte"| "int16" | "uint16" | "int32" | "uint32" \\
                & & | "float32" | "float64" | "string" | "url" \\
id & = & (ALPHA | "\_" | "\%" | ".") \\
       & & *(ALPHA | DIGIT | "/" | "\_" | "\%" | ".") \\
\end{tabular}
\end{center}
\end{ttfamily}


Clients MAY supply a constraint expression (see \DAPObjects) with
any \lit{DataDDS} request. The \DDS in the \lit{DataDDS} response
describes the variables returned. The order that the variables are listed in
the \DDS MUST match the order of the values in the binary section of the
\ac{DataDDS} response. If the response contains constructor types, then the
variables are sent in the order they would be visited in a depth-first
traversal of the accompanying \DDS.

\subsubsection{ERROR}
\label{sec-error}
When a server encounters an error, either in its software or in the users
request, it MUST return an error response. The body of the response contains
an error code along with text that provides a description of the problem
encountered. Server writers are encouraged to provide text that describes the
problem with enough information to enable a user to correct the problem or
submit a meaningful bug report to the server's maintainer.

\note[Ed note]{Error objects can, in theory, contain a short Java or
  TCL/Tk program to help users resubmit URLs, et cetera. However,
  we've never used this feature.}

\begin{ttfamily}
\begin{center}
\begin{tabular}{lll}
Error & = & "Error" "\{" "code=" error-code ";" \\
      & & "message=" error-msg ";" "\}" \\
error-code & = & 1*DIGIT \\
error-msg & = & quoted-string \\
\end{tabular}
\end{center}
\end{ttfamily}

\section{Authentication and Authorization}

The HTTP implementation of the \DAP should provide for access control to
individual files.  Access can be controlled by the individual file, or
the directory.  See \Figureref{fig-url-parts}.  The \DAP server must
be able to query the client for the username and password as needed.
In the case of nested protections, where a file inside a protected
directory is itself is protected, the \DAP server must be able to
prompt the client multiple times.

When there is only one level of access control, a client can include
the username and password directly in the \DAP URL, like this:

\begin{vcode}{it}
http://username@password:machine.uri.edu/nph-nc/data/file.nc  
\end{vcode}

It is also convenient to be able to restrict access to data resources
by excluding requests from specific IP numbers, or ranges of IP
numbers.  The \DAP does not require this form of access, but it is
suggested. 



\appendix

\bibliographystyle{plain}
\T\addcontentsline{toc}{section}{References}
\T\raggedright
\bibliography{../../../boiler/dods}

\section{Change log}

\begin{verbatim}
$Log: dap_http.tex,v $
Revision 1.14  2003/07/24 22:32:12  tom
excised dap_services document references

Revision 1.13  2003/07/18 21:33:02  tom
re-added material about help and info services.  I'm going to delete dap_services

Revision 1.12  2003/07/18 20:09:39  tom
moved version response information in

Revision 1.11  2003/07/16 20:30:17  tom
comments incorporated.  needs detail about errors.

Revision 1.10  2003/07/16 01:06:08  tom
progress on comments, fixed titles

Revision 1.9  2003/06/10 16:13:53  tom
incorporated suggestions from March meeting

Revision 1.8  2003/06/10 01:02:57  tom
CE syntax fixes

Revision 1.7  2003/06/05 21:00:58  tom
changed CE syntax

Revision 1.6  2003/06/04 21:17:39  tom
little changes about CEs

Revision 1.5  2003/05/31 01:30:18  tom
fixed constraint expression description in dap_objects to be XML

Revision 1.4  2003/05/28 21:06:50  tom
progress 5/28

Revision 1.3  2003/05/23 22:08:30  tom
makefile repairs, bibliography

Revision 1.2  2003/05/23 21:50:52  tom
progress made

Revision 1.1  2003/05/23 19:27:42  tom
new files, rearranging DAP spec into separate documents


\end{verbatim}

\end{document}
