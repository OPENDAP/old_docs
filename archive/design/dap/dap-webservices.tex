
\documentclass[justify]{dods-paper}
\usepackage{longtable}
\usepackage{acronym}
\usepackage{xspace}
%\usepackage{gloss}
%\usepackage{changebar}
\rcsInfo $Id$

% latex and HTML macros. Some latex commands become nops for HTML. 4/10/2001
% jhrg 
\T\newcommand{\Cpp}{\rm {\small C}\raise.5ex\hbox{\footnotesize++}\xspace}
\T\newcommand{\C}{\rm {\small C}\xspace}
\W\newcommand{\Cpp}{C++}
\W\newcommand{\C}{C\xspace}
\W\newcommand{\cdots}{...}
\W\newcommand{\ddots}{}
\W\newcommand{\vdots}{.}
\W\newcommand{\pm}{+/-}
\W\newcommand{\times}{*}
\W\newcommand{\uppercase}[1]{\textsc{#1}}
\T\newcommand{\qt}{\lit{\char127}}
\W\newcommand{\qt}{"}

\texorhtml{\def\rearrangedate#1/#2/#3#4{\ifcase#2\or January\or February\or
  March\or April\or May\or June\or July\or August\or September\or
  October\or November\or December\fi\ \ifx0#3\relax\else#3\fi#4, #1}
\def\rcsdocumentdate{\expandafter\rearrangedate\rcsInfoDate}}%
{\HlxEval{
(put 'rearrangedate       'hyperlatex 'hyperlatex-ts-rearrange-date)

(defun hyperlatex-ts-rearrange-date ()
  (let ((date-string (hyperlatex-evaluate-string 
                       (hyperlatex-parse-required-argument))))
    (let ((year-string (substring date-string 0 4))
          (month-string (substring date-string 5 7))
          (day-string (substring date-string 8 10))
          (month-list '("January" "February" "March" "April"
                        "May" "June" "July" "August" 
                        "September" "October" "November" "December")))
       (insert
         (concat (elt month-list (1- (string-to-number month-string)))
                 " " (int-to-string (string-to-number day-string))
                 ", " year-string)))))
}
\newcommand{\rcsdocumentdate}{\rearrangedate{\rcsInfoDate}}}

\newcommand{\dapversion}{Version 4.0\xspace}
\newcommand{\opendap}{OPeNDAP\xspace}
\newcommand{\DAP}{DAP\xspace}
\newcommand{\DODS}{DODS\xspace}
\newcommand{\NVODS}{NVODS\xspace}
\newcommand{\CE}{constraint expression\xspace}
\newcommand{\CEs}{constraint expressions\xspace}
\newcommand{\ErrorX}{ErrorX\xspace}
\newcommand{\CapX}{Server Capabilities Document\xspace}
\newcommand{\Blob}{Blob\xspace}
\newcommand{\DDX}{DDX\xspace}
\newcommand{\DAX}{DAX\xspace}
\newcommand{\DDS}{DDS\xspace}
\newcommand{\DAS}{DAS\xspace}
\newcommand{\URI}{URI\xspace}
\newcommand{\DataDDS}{DataDDS\xspace}

\newcommand{\type}[1]{\emph{#1}}
\newcommand{\Alias}{\type{Alias}\xspace}
\newcommand{\Array}{\type{Array}\xspace}
\newcommand{\Attribute}{\type{Attribute}\xspace}
\newcommand{\ProcAttribute}{\type{Processing Attribute}\xspace}
\newcommand{\Map}{\type{Map}\xspace}
\newcommand{\Target}{\type{Target}\xspace}
\newcommand{\FQN}{fully qualified name\xspace}
\newcommand{\Grid}{\type{Grid}\xspace}
\newcommand{\Structure}{\type{Structure}\xspace}
\newcommand{\Dataset}{\type{Dataset}\xspace}
\newcommand{\Sequence}{\type{Sequence}\xspace}
\newcommand{\Container}{\type{Container}\xspace}
\newcommand{\Bi}{\type{Binary Image}\xspace}
\newcommand{\String}{\type{String}\xspace}
\newcommand{\URL}{\type{URL}\xspace}
\newcommand{\Boolean}{\type{Boolean}\xspace}
\newcommand{\Byte}{\type{Byte}\xspace}
\newcommand{\Enum}{\type{Enumeration}\xspace}
\newcommand{\Time}{\type{Time}\xspace}
\newcommand{\Function}{\type{Function}\xspace}
\newcommand{\Description}{\type{Description}\xspace}
\newcommand{\Parameter}{\type{Parameter}\xspace}
\newcommand{\Constraint}{\type{Constraint}\xspace}
\newcommand{\NoAttributes}{\type{NoAttributes}\xspace}
\newcommand{\Project}{\type{Project}\xspace}
\newcommand{\Select}{\type{Select}\xspace}
\newcommand{\Hyperslab}{\type{Hyperslab}\xspace}

\newcommand{\Aliases}{\type{Aliases}\xspace}
\newcommand{\Arrays}{\type{Arrays}\xspace}
\newcommand{\Attributes}{\type{Attributes}\xspace}
\newcommand{\ProcAttributes}{\type{Processing Attributes}\xspace}
\newcommand{\Maps}{\type{Maps}\xspace}
\newcommand{\Targets}{\type{Targets}\xspace}
\newcommand{\FQNs}{fully qualified names\xspace}
\newcommand{\Grids}{\type{Grids}\xspace}
\newcommand{\Structures}{\type{Structures}\xspace}
\newcommand{\Sequences}{\type{Sequences}\xspace}
\newcommand{\Containers}{\type{Containers}\xspace}
\newcommand{\Bis}{\type{Binary Images}\xspace}
\newcommand{\Strings}{\type{Strings}\xspace}
\newcommand{\URLs}{\type{URLs}\xspace}
\newcommand{\Booleans}{\type{Booleans}\xspace}
\newcommand{\Bytes}{\type{Bytes}\xspace}
\newcommand{\Enums}{\type{Enumerations}\xspace}
\newcommand{\Times}{\type{Times}\xspace}
\newcommand{\CSVs}{comma-separated values\xspace}
\newcommand{\Functions}{\type{Functions}\xspace}
\newcommand{\Descriptions}{\type{Descriptions}\xspace}
\newcommand{\Parameters}{\type{Parameters}\xspace}
\newcommand{\Constraints}{\type{Constraints}\xspace}
\newcommand{\Projects}{\type{Projects}\xspace}
\newcommand{\Selects}{\type{Selects}\xspace}
\newcommand{\Hyperslabs}{\type{Hyperslabs}\xspace}

\newcommand{\DIR}{\textbf{Directory}\xspace}
\newcommand{\TEXT}{\textbf{Text}\xspace}
\newcommand{\HTML}{\textbf{HTML}\xspace}
\newcommand{\HELP}{\textbf{Help}\xspace}
\newcommand{\VER}{\textbf{Version}\xspace}
\newcommand{\INFO}{\textbf{Info}\xspace}

%%%%%%%%%%%%%% Web Services Paper
\newcommand{\GetDDX}{\textbf{GetDDX}\xspace}
\newcommand{\GetData}{\textbf{GetData}\xspace}
\newcommand{\GetBlobData}{\textbf{GetBlobData}\xspace}
\newcommand{\GetBlob}{\textbf{GetBlob}\xspace}
\newcommand{\GetDir}{\textbf{GetDir}\xspace}
\newcommand{\GetInfo}{\textbf{GetInfo}\xspace}

\newcommand{\FSs}{\type{Foundation Services}\xspace}
\newcommand{\FS}{\type{Foundation Service}\xspace}
\newcommand{\ODSN}{\type{OPeNDAP}\xspace}

\newcommand{\blobdataelement}{\texttt{BlobData} element\xspace}
\newcommand{\blobelement}{\texttt{Blob} element\xspace}



%\setcounter{secnumdepth}{4}
%\setcounter{tocdepth}{4}

\newcommand{\Tableref}[1]{Table~\ref{#1}}%
\newcommand{\Figureref}[1]{Figure~\ref{#1}}%
\W\begin{iftex}
\newcommand{\Sectionref}[2]{Section~\ref{#1}%
  \ifx#2)%
    \ on page~\pageref{#1}%
  \else% 
    \ (page~\pageref{#1})\xspace%
  \fi\ifx#2\space\ \else #2\fi}%
\W\end{iftex}
\W\newcommand{\Sectionref}[1]{Section~\ref{#1}}
\W\newcommand{\raggedright}{}


%% Conveniences for documenting XML
\newcommand{\tag}[1]{\emph{#1}}
\newcommand{\element}[1]{\link{\tag{#1}}{sec-xml-#1}}
\newcommand{\attribute}[1]{\emph{#1}}
\newcommand{\currentelement}{}
\newcommand{\ELEMENT}[1]{\renewcommand{\currentelement}{#1 element}%
  \subsubsection{#1}\label{sec-xml-#1}\indc{\currentelement}%
  \indc{catalog tag!#1}\indc{aggregation tag!#1}%
  \indc{XML!#1 element}}
\newcommand{\ATTRIBUTE}[1]{\item{\lit{#1}}\indc{\currentelement!#1}
    \indc{#1 attribute!of \currentelement}%
    \indc{XML!#1 attribute}}

% Conveniences for examples
\newcounter{exampleno}
\setcounter{exampleno}{0}
\newcounter{examplerefno}
\setcounter{examplerefno}{0}
\newcommand{\examplelabel}[1]{\refstepcounter{exampleno}\label{#1}%
  \medskip Example \theexampleno :\smallskip}
\newcommand{\exampleref}[1]{\texorhtml{Example~\ref{#1}%
    \refstepcounter{examplerefno}\label{exref\theexamplerefno}%
    % This is a test whether r@exref... is defined.  If not, skip
    % anything with the \pageref macro.
    \catcode`\@=11%
    \expandafter\ifx\csname r@exref\theexamplerefno\endcsname\relax\else%
    \expandafter\ifx\csname r@#1\endcsname\relax\else%
    \bgroup\count100=\pageref{exref\theexamplerefno}%
    \count101=\pageref{#1}\ifnum\count100=\count101\else~%
    on page~\pageref{#1}\fi\egroup\fi\fi\xspace}%
  {\link{Example \ref{#1}}{#1}}}%
\T\setlength{\vcodeindent}{10pt}
\texorhtml{
%% LaTeX version
\newenvironment{textoutput}[1]{\ifx #1\relax%
  \medskip Output:\vspace{-\medskipamount}\else%
  \medskip #1\vspace{-\medskipamount}\fi%
  \begin{list}{}{\setlength{\leftmargin}{\vcodeindent}}\begin{ttfamily}\item}%
 {\end{ttfamily}\end{list}} %
}{%% Hyperlatex version
\newenvironment{textoutput}[1]{\xml{blockquote}Output:\\ \\ \xml{tt}}%
  {\xml{/tt}\xml{/blockquote}}%
\newenvironment{minipage}[4]{}{}
\newenvironment{ttfamily}{\xml{blockquote}\xml{tt}}%
  {\xml{/tt}\xml{/blockquote}}}

\newcommand{\DAPOverviewTitle}{DAP Specification Overview}
\newcommand{\DAPOverview}{\xlink{\textbf{\textit{\DAPOverviewTitle}}}%
  {dap.html}\xspace}

% Old macros; new values
\newcommand{\DAPObjectsTitle}{The Data Access Protocol---DAP 2.0}
\newcommand{\DAPObjects}{\xlink{\textbf{\textit{\DAPObjectsTitle}}}%
  {http://www.opendap.org/pdf/ESE-RFC-004v0.06.pdf}\xspace}
  
\newcommand{\DAPHTTPTitle}{Using DAP 2.0 with HTTP}
\newcommand{\DAPHTTP}{\xlink{\textbf{\textit{\DAPHTTPTitle}}}%
  {daph.html}\xspace}

% New macros.
\newcommand{\DAPDataModelTitle}{DAP Data Model Specification}
\newcommand{\DAPDataModel}{\xlink{\textbf{\textit{\DAPDataModelTitle}}}%
  {dapo.html}\xspace}
  
\newcommand{\DAPWebTitle}{DAP Web Services Specification}
\newcommand{\DAPWeb}{\xlink{\textbf{\textit{\DAPWebTitle}}}%
  {daph.html}\xspace}
  

\newcommand{\DAPASCIITitle}{DAP Formatted Data Specification}
\newcommand{\DAPASCII}{\xlink{\textbf{\textit{\DAPASCIITitle}}}%
  {dapa.html}\xspace}
\newcommand{\DAPHTMLTitle}{DAP HTTP Query Specification}
\newcommand{\DAPHTML}{\xlink{\textbf{\textit{\DAPHTMLTitle}}}%
  {dapm.html}\xspace}

% It probably doesn't matter what we call the macro, but SOAP does not have
% to run over HTTP and I think that's going to be important for other groups.
% 10/27/03 jhrg
\newcommand{\DAPServicesTitle}{DAP HTTP Services Specification}
\newcommand{\DAPServices}{\xlink{\textbf{\textit{\DAPServicesTitle}}}%
  {daps.html}\xspace}

\newcommand{\thirtytwobitlimit}[1]{$4,294,967,296$ #1 ($2^{32}$)}
%%% Local Variables: 
%%% mode: latex
%%% TeX-master: t
%%% End: 


% Note: to get the glossary to work, run bibtex on the *.gls.aux file,
% then latex the file, then bibtex *.gls, then latex again. Also, make
% sure to set your BST and BIBINPUTS environment variables so that the
% BST and BIB files will be found.
% \makegloss

\title{\DAPWebTitle\\ DRAFT}
\htmltitle{\DAPWebTitle\ -- DRAFT}
\author{James Gallagher\thanks{OPeNDAP, Inc. jgallagher@opendap.org}
\and Nathan Potter\thanks{Oregon State University,
  ndp@coas.oregonstate.edu}
\and  John Chamberlin\thanks{OPeNDAP, Inc. jchamberlain@OPeNDAP.org}}
\date{\rcsInfoDate \\ Revision: \rcsInfoRevision}
\htmladdress{James Gallagher <jgallagher@opendap.org>, \rcsInfoDate, 
  Revision: \rcsInfoRevision}
\htmldirectory{dap-webservices}
\htmlname{dap-webservices}

\begin{document}

\maketitle
%\T\tableofcontents


%%%%%%%%%%%%%%%%%%%%%%%% Introduction %%%%%%%%%%%%%%%%%%%%%%%%

\section{Introduction}
The purpose of this paper is to outline different web service design
choices being considered by OPeNDAP, to provide a rationale for
their different characteristics and to solicit comments about them.

The phrase Web services is used to describe a range of network-based
and distributed computing systems. Two characteristics common to
virtually all such systems are that clients and servers on different
computers interact to perform some computation and that the World Wide
Web infrastructure supports that interaction.

Two classes of web services have emerged. The first type of web
service system uses the REST architecture where \emph{resources} are
represented using URI and HTTP is used to return the current
\emph{representation} of that resource. These systems are based on
HTTP, XML and URI and is described by Fielding and others in a variety
of documents.\cite{rfc2616}\cite{Fielding:diss} In these systems, XML
is typically used to encode the representation (although for the web
in general, HTML is).\footnote{XML or HTML are not the only syntaxes,
  but they are the standard ones described in the bulk of the
  documentation.} Such systems are also often called `network-based'
because the existence of network operations is made explicit.

The second type of system follows the Remote Procedure Call model and
uses the Simple Object Access Protocol (SOAP) along with other
technologies such as WSDL and UDDI and builds a distributed system
(where the network operations are hidden from client programs). Unlike
HTTP/URI systems, using SOAP/WSDL/UDDI it is possible to build a
system where clients can access machine readable interface
descriptions (using WSDL records which can themselves be located using
UDDI) and then use the RPCs in those descriptions to access a server.
SOAP and its related protocols and technologies are described in great
detail by the W3C as well as in numerous books.

It's important to note that REST describes an architecture, as does
RPC. SOAP, HTTP, et c., are technologies which can be used to
implement systems designed using one of these architectures.

These two approaches to building systems both have strong points. In
the past, RPC-like systems have not scaled well beyond organizational
boundaries, but SOAP may provide an additional infrastructure which
allows it to overcome this limitation. At the same time the HTTP/URI
based systems have been successful to date but all require `hand
coding' of clients. This appears to be an inherent characteristic of
such a system, but that also may not be a drawback because completely
automated discovery and use of web services using UDDI/WSDL/SOAP may
not be realistic when dealing with the typical Earth-science data
objects.

We provide more information on the SOAP/WSDL technology and the REST
architecture in Appendices~\ref{ws+soap} and~\ref{rest} of this paper.

The remainder of this paper describes a SOAP messaging interface and a
HTTP/GET interface for the OPeNDAP server. In both cases we break
the server's interface down into two groups of services: Foundation
Services and Presentation Services.

\section {Foundation Services implemented using SOAP}
\label{FndSrvcs}

The core services in the OPeNDAP hierarchy are called the \FSs. There
are three of these services: \GetDDX; \GetData; and \GetBlob. The
\GetDDX and \GetData services are based on SOAP
messaging.\footnote{\ldots as opposed to SOAP RPC.
  See Appendix~\ref{ws+soap}.} The \GetBlob service does not use SOAP.
Instead it provides access to a serialized binary encoded data stream
in an out-of-band manner.

The \FSs are all exposed through the single service name: \ODSN.
This was done, in part, to allow clients to pool requests. A client may
send a single message to the \ODSN service that contains multiple
requests (see \Sectionref{pooled}).

\subsection {\GetDDX Service}
\label {getDDX}

The \GetDDX service is used to provide client access to the \DDX
representation of a \Dataset (See \DAPObjectsTitle). The client must
send a SOAP message containing a valid \CE\footnote{As is the case
  with DAP2, a valid \CE is the empty string, which provides no
  constraint on the returned data (i.e., it instructs the server to
  return all of the data in the data source).} to the service. In
response the service will return the \DDX of the \Dataset, constrained
as indicated in the \CE. This \DDX will contain the message and
service URL for the \GetBlob service call that will return the \Blob
data. This service is the functional equivalent of the old DODS URL
\lit{.dds} response but note that the DDX also includes all of the
information that was separated in the DAS response in DAP2, so it also
provides the same information that DAP2 provides using the \lit{.das}
response. 

For example, the SOAP body of a \GetDDX message might look like:

\begin{vcode}{t}

    <soap:Body>
        <GetDDX name="data.set.name">
            <Constraint>
                .
                .
                .
            </Constraint>
        </GetDDX>
    </soap:Body>
    
\end{vcode}

The SOAP body of the returned message might look like:

\begin{vcode}{t}

    <soap:Body>
        <Dataset name="data.set.name">
            <Array name="sst">
                .
                .
                .
            </Array>
            <BlobRequest URL="http://hostname/OPeNDAP/Blob/">
                <GetBlob name="data set pathname">
                    <Constraint>
                       .
                       .
                       .
                    </Constraint>
                </GetBlob>
            </BlobRequest>
        </Dataset>
    </soap:Body>
    
\end{vcode}

\subsubsection{WSDL for the \GetDDX service}

% Don't add much to this example since it fills up the page as it is.
\begin{vcode}{t}
<?xml version="1.0" encoding="UTF-8"?>
<definitions name="GetDDX"
   targetNamespace="http://xml.opendap.org/wsdl/getDDX.wsdl"
   xmlns="http://schemas.xmlsoap.org/wsdl/"
   xmlns:soap="http://schemas.xmlsoap.org/wsdl/soap/"
   xmlns:tns="http://xml.opendap.org/wsdl/GetDDX.wsdl"
   xmlns:xsd="http://www.w3.org/2001/XMLSchema">
 
   <message name="GetDDXRequest">
      <part name="data_source" type="xsd:string"/>
      <!-- It's really an XML fragment... -->
      <part name="constraint" type="xsd:string"/>
   </message>
   <message name="GetDDXResponse">
      <!-- Is this the best way to describe a returned XML document? -->
      <part name="DDX" type="xsd:string"/>
   </message>
 
   <portType name="DDX_PortType">
      <operation name="getDDX">
         <input message="tns:GetDDXRequest"/>
         <output message="tns:GetDDXResponse"/>
      </operation>
   </portType>

   <binding name="DDX_Binding" type="tns:DDX_PortType">
      <soap:binding style="document"
                    transport="http://schemas.xmlsoap.org/soap/http" />
      <operation>
         <soap:operation soapAction="http://test.opendap.org/opendap"/>
         <input>
            <soap:body use="literal"/>
         </input>
         <output>
            <soap:body use="literal"/>
         </output>
      </operation>
   </binding>
 
   <service name="OPeNDAP">
      <documentation>WSDL File for OPeNDAP SOAP Web Services</documentation>
        <port name="ddx" binding="tns:DDX_Binding">
           <http:address location="http://test.opendap.org/opendap/"/>
        </port>
   </service>
</definitions>
\end{vcode}


\subsection {\GetData Service}
\label {getData}

The \GetData service is used to provide client access (via SOAP) to
the \DDX representation of a \Dataset (see DAP Objects paper) bundled
with the \Blob that contains the serialized binary data content of the
returned \DDX. That is, this service provides a `SOAP way' to get the
data values. This paper also describes an out-of-band data access
`pseudo-service' which has some distinct advantages over this service
(See~\Sectionref{getBlob}). The client must send a SOAP message
containing a valid \CE to the service. In response the service will
return the \DDX of the \Dataset, constrained as indicated in the \CE.
This will be followed, in the same message, by a \blobdataelement
whose content is the Base64\cite{ietf:rfc2045} encoded serialized
binary data content of the returned \DDX constrained as specified in
the \CE. As always, the \DDX will contain the both the service URL for
the \GetBlob service call and the message to send to the service URL
that will return the \Blob data. This service is the functional
equivalent of the old DODS URL \lit{.dods} response.

An example call to the \GetData service might look like:

\begin{vcode}{t}

    <soap:Body>
        <GetData name="data.set.name">
            <Constraint>
                .
                .
                .
            </Constraint>
        </GetData >
    </soap:Body>
    
    
\end{vcode}

The SOAP body of the returned message might look (with liberties taken
with the XML formatting for readability) something like:

\begin{vcode}{t}

    <soap:Body>
        <Dataset name="data.set.name">
            <Array name="sst">
                .
                .
                .
            </Array>
            <BlobRequestMsg SrvcURL="hostname:12001/OPeNDAP/BSrvc">
                <GetBlob name="data.set.name">
                    <Constraint>
                       .
                       .
                       .
                    </Constraint>
                </GetBlob>
            </GetBlobRequestMsg>
        </Dataset>
        <BlobData>
        7xUzYGtGWgAAABkAAAAkVGhpcyBpcyBhIGRhdGEgdGVzdCBzdHJpbmcgKHBhc3MgMCk
        uAAAAJFRoaXMgaXMgYSBkYXRhIHRlc3Qgc3RyaW5nIChwYXNzIDEpLgAAACRUaGlzIGl
        zIGEgZGF0YSB0ZXN0IHN0cmluZyAocGFzcyAyKS4AAAAkVGhpcyBpcyBhIGRhdGEgdGV
        zdCBzdHJpbmcgKHBhc3MgMykuAAAAJFRoaXMgaXMgYSBkYXRhIHRlc3Qgc3RyaW5nICh
        wYXNzIDQpLgAAACRUaGlzIGlzIGEgZGF0YSB0ZXN0IHN0cmluZyAocGFzcyA1KS4AAAA
        kVGhpcyBpcyBhIGRhdGEgdGVzdCBzdHJpbmcgKHBhc3MgNikuAAAAJFRoaXMgaXMgYSB
        kYXRhIHRlc3Qgc3RyaW5nIChwYXNzIDcpLgAAACRUaGlzIGlzIGEgZGF0YSB0ZXN0IHN
        0cmluZyAocGFzcyA4KS4AAAAkVGhpcyBpcyBhIGRhdGEgdGVzdCBzdHJpbmcgKHBhc3M
        gOSkuAAAAJVRoaXMgaXMgYSBkYXRhIHRlc3Qgc3RyaW5nIChwYXNzIDEwKS4AAAAAAAA
        lVGhpcyBpcyBhIGRhdGEgdGVzdCBzdHJpbmcgKHBhc3MgMTEpLgAAAAAAACVUaGlzIGl
        IGEgZGF0YSB0ZXN0IHN0cmluZyAocGFzcyAxMikuAAAAAAAAJVRoaXMgaXMgYSBkYXRh
        </BlobData>
          
    </soap:Body>
    
\end{vcode}

Although the \GetData service provides a full SOAP based interface to
data values stored in an \opendap server, it is strongly recommend
that clients combine the use of the \GetDDX service and the \GetBlob
service instead. Since the \GetData service uses SOAP to carry binary
data content the Base64 encoding of the data is necessary to allow the
data to be transmitted in an XML document. This has (at least) two
unfortunate ramifications: First, the number of bytes transmitted is
increased by a factor of 33 percent. Secondly, the entire response
must be sent in the document, which eliminates the possibility of
streaming the response to the client so that the client can process it
on the fly.

\subsection{Alternatives to Base64 Encoding}

An alternative to providing data using Base64 encoding is to use SOAP
with Attachments (SwA).\cite{w3c:SwA} This provides a way to bundle
binary data without incurring the size penalty of Base64 encoding.
Unfortunately, SwA, like Base64 encoding, does not support data
streaming.

Other documents that relate to SOAP with Attachments are the
XML-binary Optimized Packaging\cite{w3c:XOP} (XOP) and SOAP Message
Transmission Optimization Mechanism\cite{w3c:SMTOM} (SMTOM). At one
point Microsoft and others were developing a MIME-like document format
which they called DIME that could support the SwA solution \emph{and}
streaming responses. However, the references to DIME from Microsoft
now describe this technology as superseded by SMTOM. It's not clear if
SMTOM and XOP together will support streamed responses. It's also
important that these rely on SOAP 1.2, which may have poor support in
a wide variety of languages for some time.

It seems that, using the current technology, SwA improves on Base64
encoding even though it does not suppport streaming responses. It
saves on the chore of performing the encoding and decoding, does not
incur the size penalty of Base64 and the (encoded) data values are not
processed by the XML parser.

Here's the previous response using SOAP with Attachments:

\begin{vcode}{t}

MIME-Version: 1.0
Content-Type: Multipart/Related; boundary=MIME_boundary; type=text/xml;
        start="<data.set.name.blob@hostname>"
Content-Description: This is the optional message description.

--MIME_boundary
Content-Type: text/xml; charset=UTF-8
Content-Transfer-Encoding: 8bit
Content-ID: <data.set.name.blob@hostname>

<?xml version='1.0' ?>
<SOAP-ENV:Envelope xmlns:SOAP-ENV="http://schemas.xmlsoap.org/soap/envelope/">
<SOAP-ENV:Body>
    <Dataset name="data.set.name">
        <Array name="sst">
            .
            .
            .
        </Array>
        <BlobRequestMsg SrvcURL="hostname:12001/OPeNDAP/BSrvc">
            <GetBlob name="data.set.name">
                <Constraint>
                   .
                   .
                   .
                </Constraint>
            </GetBlob>
        </GetBlobRequestMsg>
    </Dataset>
    <BlobData href="cid:data.set.name.blob@hostname"/>
</SOAP-ENV:Body>
</SOAP-ENV:Envelope>

--MIME_boundary
Content-Type: application/octet-stream
Content-Transfer-Encoding: binary
Content-ID: <data.set.name.blob@hostname>

...binary XDR encoded data...
--MIME_boundary--

    
\end{vcode}

% \subsection {\GetBlobData Service}
% \label {getBlobData}

% The \GetBlobData service is used to provide client access (via SOAP)
% to the \Blob that contains the serialized binary data content of the
% \Dataset as detailed in the request message. The client must send a
% SOAP message containing a valid \CE to the service. In response the
% service will return a \blobdataelement whose content is the Base64
% encoded serialized binary data content of the returned \DDX
% constrained as specified in the \CE.

% This service is the functional equivalent of the old DODS URL .blob
% response, with the caveat that the binary data has been Base64
% encoded.

% \begin{vcode}{t}

%     <soap:Body>
%         <GetBlobData name="data.set.name">
%             <Constraint>
%                 .
%                 .
%                 .
%             </Constraint>
%         </GetBlobData >
%     </soap:Body>
    
% \end{vcode}

% The SOAP body of the returned message might look (with liberties taken
% with the XML formatting for readability) something like:

% \begin{vcode}{t}

%     <soap:Body>
%         <BlobData name="data.set.name">
%         7xUzYGtGWgAAABkAAAAkVGhpcyBpcyBhIGRhdGEgdGVzdCBzdHJpbmcgKHBhc3MgMCk
%         uAAAAJFRoaXMgaXMgYSBkYXRhIHRlc3Qgc3RyaW5nIChwYXNzIDEpLgAAACRUaGlzIGl
%         zIGEgZGF0YSB0ZXN0IHN0cmluZyAocGFzcyAyKS4AAAAkVGhpcyBpcyBhIGRhdGEgdGV
%         zdCBzdHJpbmcgKHBhc3MgMykuAAAAJFRoaXMgaXMgYSBkYXRhIHRlc3Qgc3RyaW5nICh
%         wYXNzIDQpLgAAACRUaGlzIGlzIGEgZGF0YSB0ZXN0IHN0cmluZyAocGFzcyA1KS4AAAA
%         kVGhpcyBpcyBhIGRhdGEgdGVzdCBzdHJpbmcgKHBhc3MgNikuAAAAJFRoaXMgaXMgYSB
%         kYXRhIHRlc3Qgc3RyaW5nIChwYXNzIDcpLgAAACRUaGlzIGlzIGEgZGF0YSB0ZXN0IHN
%         0cmluZyAocGFzcyA4KS4AAAAkVGhpcyBpcyBhIGRhdGEgdGVzdCBzdHJpbmcgKHBhc3M
%         gOSkuAAAAJVRoaXMgaXMgYSBkYXRhIHRlc3Qgc3RyaW5nIChwYXNzIDEwKS4AAAAAAAA
%         lVGhpcyBpcyBhIGRhdGEgdGVzdCBzdHJpbmcgKHBhc3MgMTEpLgAAAAAAACVUaGlzIGl
%         IGEgZGF0YSB0ZXN0IHN0cmluZyAocGFzcyAxMikuAAAAAAAAJVRoaXMgaXMgYSBkYXRh
%         </BlobData>
          
%     </soap:Body>
   
    
% \end{vcode}

% Although \GetBlobData service provides a full SOAP based interface to
% data values stored in an \opendap server, it is strongly recommend
% that clients combine the use of the \GetDDX service and the \GetBlob
% service instead. Since the \GetBlobData service uses SOAP to carry the
% binary data content the Base64 encoding of the serialized binary data
% is necessary to allow the data to be transmitted in an XML document.
% This has (at least) two unfortunate ramifications: First, the number
% of bytes transmitted is increased by a factor of 33 percent. (CHECK
% THIS CLAIM!) Secondly, the entire response must be sent in the
% document, which eliminates the possibility of streaming the response
% to the client so that the client can process it on the fly.

\subsection {\GetBlob Service}
\label {getBlob}

The \GetBlob service is NOT a SOAP interface. This foundation service
is used by clients to retrieve the serialized binary content of a
\Dataset. To invoke this service a client connects to the correct port
on an OPeNDAP server and sends an XML message to the server. The
OPeNDAP server replies with the serialized binary data content of the
\Dataset described in the request as described in the DAP Objects
paper.

It is important to provide a way for the receiver of the \GetBlob
response to discern error messages that might appear at any point in
the stream, since the response returned by \GetBlob is not a contained
within a SOAP response, but instead is a stream of data values.
Suppose, for example, that several variables are to be returned and
the server (i.e., the \GetBlob response generator) reads and
serializes one, then the next and so on. An error accessing or
returning the data for the second variable might not be seen until the
values for the first have already been sent. To accommodate such a
situation, the stream returned by \GetBlob will be `chunked,' in a way
similar to HTTP/1.1 chunked responses, so that a premature end of data
and the start of information about an error (e.g., an ErrorX response)
can be detected.

An example \GetBlob message might look like:

\begin{vcode}{t}

    <GetBlob name="data set pathname">
        <Constraint>
            .
            .
            .
        </Constraint>
    </GetBlob >
    
    
\end{vcode}

In response the server will return a multi-part MIME document which
contains a brief description of the source of the data and the data
themselves, encoded using XDR.\footnote{Proposed is an optimization of
  the `always use XDR' approach where the client announces to the
  server that it uses either big- or little-endian representation and
  the server then responds either using the XDR-based big-endian (as
  default) or matches the client's representation, thus eliminating,
  in the worst case scenario, the need to transform the data values on
  both ends.}

\subsection {Pooled Requests}
\label {pooled}

The \ODSN service supports pooling requests. A client may collect
multiple \GetDDX and \GetData service requests and send them to a
server in a single request. The server will return the appropriate
items in the same order that they were requested. All of the SOAP
based \FSs services may be pooled. The \GetBlob service call MAY NOT
be pooled.

For example, a client could request 3 DDX's in a single pooled request:


\begin{vcode}{t}

    <soap:Body>
        <GetDDX name="data.set.01">
            <Constraint/>
        </GetDDX>
        <GetDDX name="data.set.02">
            <Constraint/>
        </GetDDX>
        <GetDDX name="data.set.03">
            <Constraint/>
        </GetDDX>
    </soap:Body>
    
    
\end{vcode}


And the server would return:

\begin{vcode}{t}


    <soap:Body>
        <Dataset name="data.set.01">
            .
            .
            .
        </Dataset>
        <Dataset name="data.set.02">
            .
            .
            .
        </Dataset>
        <Dataset name="data.set.03">
            .
            .
            .
        </Dataset>
    </soap:Body>
    
\end{vcode}

 The pooled requests need not be homogeneous. Consider:
 
\begin{vcode}{t}

    <soap:Body>
        <GetDDX name="data.set.01">
            <Constraint/>
        </GetDDX>
        <GetData name="data.set.02">
            <Constraint/>
        </GetDDX>
        <GetDDX name="data.set.03">
            <Constraint/>
        </GetDDX>
        <GetBlobData name="data.set.04">
            <Constraint/>
        </GetBlobData >
    </soap:Body>
    
    
\end{vcode}


And the server would return:

\begin{vcode}{t}


    <soap:Body>
        <Dataset name="data.set.01">
            .
            .
            .
        </Dataset>
        <Dataset name="data.set.02">
            .
            .
            .
        </Dataset>
        <BlobData>
            .
            .
            .
        </BlobData>
        <Dataset name="data.set.03">
            .
            .
            .
        </Dataset>
        <BlobData name="data.set.04">
            .
            .
            .
        </BlobData>

    </soap:Body>
    
\end{vcode}

\section {Presentation Services using SOAP}
\label {SNGHSrvcs}

The Presentation services, such as DAP2's Info response, will be
implemented using SOAP RPC. These higher level services provide
the kind of functionality that will allow our users to use UDDI as a
discovery mechanism, and should be kept in the SOAP RPC domain.

Presentations Services to be described: Info, Version, HTML (a form
interface).

Note that the development of `Server4' will introduce software which
uses THREDDS to encode directories and catalogs in XML. I assume that
a SOAP RPC interface will be developed for these and so they will fall
into the Presentation Services. 

The Presentation Services are responses which the OPeNDAP servers will
support that are not, strictly speaking, DAP responses.

%%%%%%%%%%%%%%%%%%%%%%%% WSDL %%%%%%%%%%%%%%%%%%%%%%%%
% \section{WSDL}

%%%%%%%%%%%%%%%%%%%%%%%% UDDI %%%%%%%%%%%%%%%%%%%%%%%%
%\section{UDDI}

\section{Foundation Services Using HTTP/GET}

The current server produced by OPeNDAP provides a HTTP/GET interface
for the \FSs (and the Presentation services, too). Different responses
are returned by adding a suffix to the data source URI. This interface
has been documented extensively elsewhere so, rather than repeat that
in this paper we will describe how the DAP4 features will alter this
already-documented interface.

The DAS and DDS responses are combined in one document in DAP4. This
new response is called the DDX. In the DDX each variable's attributes
are integral to the variable itself. Data sources themselves can still
have attributes independent of the variables' attributes (i.e., global
attributes) and, as with DAP2's DAS, there can be any number of
additional  attribute containers.

In addition to the DDX replacing the DAS/DDS pair, DAP4 bundles the
\emph{BLOB URI} in the DDX. Every DDX contains the BLOB URI which
references the data described by that DDX. Of course, clients are
under no obligation to dereference these BLOB URI, but each DDX
response will now contain all the information needed to access the
data it describes. When a DDX is requested using a \emph{Constraint
  Expression} (CE) the BLOB URI returned with the DDX response will
return those values as constrained by the CE.

The DDX replaces the DAS/DDS and also the DataDDS responses. At the
same time the DDX response also frees the client from remembering how
to form a URI for data since each DDX contains the BLOB URI which
references the data described by the DDX. This design also frees the
server from using the same URI base for both the metadata (DDX) and
data (BLOB URI). In fact the BLOB URI might reference data stored on a
completely different machine.\footnote{The SOAP interface described
  before is also capable of this.}

The adoption of XML as the syntax for the DDX frees clients and
servers from relying on parsers specific to the DAP, at least in part.
Because the DAP is typically used to describe fairly complex data
sources, most of the potential clients will need some form of
DAP-specific knowledge. 

\note{The OPeNDAP Data Server version 3.5 supports an experimental
  version of the DDX response. Clients, including (or maybe
  especially) web browsers, can access this by appending the suffix
  \lit{.ddx} to a DAP URL passed to a browser. The OPeNDAP test data
  site will support this once it's online. Its URL is:
  \lit{http://test.opendap.org/OPeNDAP-3.5/nph-dods/data/}.
  From there you can navigate to a data source and replace the
  \lit{.html} suffix with .ddx. In addition, \lit{libdap} version
  3.5.3 contains client side tools to use the DDX response.}

\subsection{HTTP/GET and the Request Syntax}

In the HTTP/GET interface, the query string is still used to pass the
CE from the client to the server. The disadvantage to this is that the
current CE requires a fairly sophisticated parser while simpler uses
of the query string use a parameter-value pair which is almost trivial
to read. We will investigate using the more common query string syntax
with the DAP4 servers.\footnote{For the sake of backward compatibility, we will
support the DAP2 CE grammar in DAP4.} This should avail DAP4 CEs some of
same benefits that the DDX will get from use of XML; that off the
shelf toolkits will be able to parse the CEs.

\subsection{WSDL and the HTTP/GET interface}

The HTTP/GET interface can be described using WSDL in a way very
similar to the SOAP interface. See Web Services Description Language (WSDL)
1.1\cite{christensen:wsdl1.1} for more information. 

Here is an example WSDL description for the DDX service:
% Don't add much to this example since it fills up the page as it is.
\begin{vcode}{t}
<?xml version="1.0" encoding="UTF-8"?>
<definitions name="DDXService"
   targetNamespace="http://xml.OPeNDAP.org/wsdl/DDXService.wsdl"
   xmlns="http://schemas.xmlsoap.org/wsdl/"
   xmlns:soap="http://schemas.xmlsoap.org/wsdl/soap/"
   xmlns:tns="http://xml.OPeNDAP.org/wsdl/DDXService.wsdl"
   xmlns:xsd="http://www.w3.org/2001/XMLSchema">
 
   <message name="DDXRequest">
      <part name="data_source" type="xsd:string"/>
      <part name="proj" type="xsd:string"/>
      <part name="sel" type="xsd:string"/>
   </message>
   <message name="DDXResponse">
      <!-- Is this the best way to describe a returned XML document? -->
      <part name="DDX" type="xsd:string"/>
   </message>
 
   <portType name="DDX_PortType">
      <operation name="getDDX">
         <input message="tns:DDXRequest"/>
         <output message="tns:DDXResponse"/>
      </operation>
   </portType>

   <binding name="DDX_Binding" type="tns:DDX_PortType">
      <http:binding verb="GET"/>
       <operation name="getDDX">
           <!-- two options below, one commented out -->
           <http:operation location="nph-dods/(data_source).ddx?proj=(proj)&sel=(sel)"/>
           <!-- http:operation location="nph-dods/(data_source)/ddx/(proj)/(sel)"/ -->
           <input>
               <http:urlReplacement/>
           </input>
           <output>
               <!-- Should this be application/xml? -->
               <mime:content type="text/xml"/>
           </output>
       </operation>
   </binding>
 
   <service name="DDX_Service">
      <documentation>WSDL File for DDXService</documentation>
        <port name="ddx" binding="tns:DDX_Binding">
           <http:address location="http://test.opendap.org/opendap/"/>
        </port>
   </service>
</definitions>
\end{vcode}

%%%%%%%%%%%%%%%%%%%%%%%% APPENDIX %%%%%%%%%%%%%%%%%%%%%%%%
\appendix

\section {Web Services and SOAP}
\label{ws+soap}

SOAP is used in two basic ways, as a tool for doing RPC on remote
systems, and as a messaging service.

When SOAP is used to perform RPC the parameters passed and the return
values must be consistent with the SOAP data model (unless custom
encodings are used, more on that later). This model is relatively
simplistic, it contains a collection of atomic types, arrays, and
structures. In the SOAP data model values must be associated with each
instance of an atomic type, this includes members of an array or
structure.

Given the type constraints and the embedded value requirements of the
SOAP data model it is not practical to try to shoe-horn the OPeNDAP
data model directly into a SOAP data model representation model.

In order pass our DAP objects via SOAP methods we have 2 choices:
Extend the SOAP data model to cover our custom data types (so that our
types can be used in an RPC request) or rely on SOAP messaging.

Custom data types require the use of custom serialization and
de-serialization classes that must be integrated into the SOAP server
(and client). These classes are tightly coupled to the particular
implementation of the SOAP engine (such as Apache AXIS, GLUE, etc.)

Messaging allows us to pass W3C DOM trees back and forth between
server and client. Each side can then act upon them as necessary
(parse them into a Java/C++ class in memory, extract pertinent
information directly from the DOM elements, what have you)

Using a SOAP messaging scheme will cut us off from the automatic WSDL
generation tools that have been developed for the SOAP RPC model.
However, this may not be such a large issue. Ultimately, having a
custom data type in an RPC call only buys you the information (at the
WSDL level) that this method requires (for example) a "DDS" or a
"ConstraintExpression" but probably doesn't lead you to an
implementation of such a beast.

\section{Ethan Davis Had this comment:}

I like the single 'OPeNDAP' service idea with the requested foundation
service encoded in the request message. I wonder if the DDX should
contain a more generic request message rather than one specific for
the Blob, e.g.,

\begin{verbatim}

</Dataset>
    ...
    <RequestMsg SrvcURL="http://hostname/axis/services/OPeNDAP"
                dataSetName="data.set.name">
            <Constraint>
                ...
            </Constraint>
    </RequestMsg>
</Dataset>

\end{verbatim}

That way it doesn't limit the type of request a user can build from
the DDX info (or rather, it doesn't give the appearance of limiting
the users request).

\section{REST: Representational State Transfer}
\label{rest}

\begin{quote}
\emph{This summary of REST was copied from
  webservices.sml.com\footnote{http://webservices.xml.com/pub/a/ws/2002/02/20/rest.html}
  I've added my own editorial comments (in italics) where they seem
  relevant. jhrg.}
\end{quote}

REST is a model for distributed computing. It is the one used by the
world's biggest distributed computing application, the Web. When
applied to web services technologies, it usually depends on a trio of
technologies designed to be extremely extensible: XML, URIs, and HTTP.
XML's extensibility should be obvious to most, but the other two may
not be.

\begin{quote}
\emph{The acronym `REST' was coined by Roy Fielding in his Ph. D.
  Dissertation.\cite{Fielding:diss} Fielding describes REST as ``a
  hybrid style derived from several \ldots network-based architectural
  styles \ldots combined with additional constraints that define a
  uniform connector interface.''\footnote{Fielding, p 76.} He also
  points out that most (previously) published information on software
  architecture excludes data as an ``important architectural
  element''\footnote{Fielding, p 23.} This fits nicely with the discussion
  below of URIs used to refer to `resources' which are typically data
  elements.}
\end{quote}

URIs are also extensible: there are an infinite number of possible
URIs. More importantly, they can apply to an infinite number of
logical entities called "resources." URIs are just the names and
addresses of resources. Some REST advocates call the process of
bringing your applications into this model "resource modeling." This
process is not yet as formal as object oriented modeling or
entity-relation modeling, but it is related.

The strength and flexibility of REST comes from the pervasive use of
URIs. This point cannot be over-emphasized. When the Web was invented
it had three components: HTML, which was about the worst markup
language of its day (other than being simple); HTTP, which was the
most primitive protocol of its day (other than being simple), and URIs
(then called URLs), which were the only generalized, universal naming
and addressing mechanism in use on the Internet. Why did the Web
succeed? Not because of HTML and not because of HTTP. Those standards
were merely shells for URIs.

\begin{quote}
  \emph{Fielding places significant emphasis on URI, stating that the
    architecture's goals are ``\ldots achieved by placing constraints
    on connector semantics where other styles have focused on
    component semantics.''\footnote{Fielding, p 148.} He also states that
    URI `` are both the simplest element of the Web architecture and
    the most important.''\footnote{Fielding, p 109.} They map the concept
    represented by a resource to the concrete representation of that
    resource at a given time. REST defines ``a resource to be the
    semantics of what the author intends to identify, rather than the
    value corresponding to those semantics at the time the reference
    is created.''\footnote{Fielding, p 111.} Late binding of
    references is crucial for cross-organizational systems.}
\end{quote}

HTTP's extensibility stems primarily from the ability to distribute
any payload with headers, using predefined or (occasionally) new
methods. What makes HTTP really special among all protocols, however,
is its built-in support for URIs and resources. URIs are the defining
characteristic of the Web: the mojo that makes it work and scale. HTTP
as a protocol keeps them front and center by defining all methods as
operations on URI-addressed resources. 

The most decisive difference between web services and previous
distributed computing problems is that web services must be designed
to work across organizational boundaries. Of course, this is also one
of the defining characteristics of the Web. This constraint has
serious implications with respect to security, auditing, and
performance.

REST first benefits security in a sort of sociological manner. Where
RPC protocols try as hard as possible to make the network look as if
it is not there, REST requires you to design a network interface in
terms of URIs and resources (increasingly XML resources). REST says:
``network programming is different than desktop programming -- deal
with it!'' [\emph{This idea resurfaces every so often. The first place
  I (jhrg) saw it was in a paper about Sun's RPC technology written by a
  group within Sun.\cite{waldo:dist-comp} Fielding also cites a
  reference to Tanenbaum and van Renesse that makes the same point
  when describing the distinction between `network-based' and
  `distributed' systems (that the former is not transparent to the
  user while the latter is).\footnote{Fielding, p 24.}}]
Whereas RPC interfaces
encourage you to view incoming messages as method parameters to be
passed directly and automatically to programs, REST requires a certain
disconnect between the interface (which is REST-oriented) and the
implementation (which is usually object-oriented).

\begin{quote}
  \emph{What makes HTTP [in the text, HTTP/1.1 is described as being
    designed using REST. jhrg] significantly different from RPC is
    that the requests are directed to resources using a generic
    interface with standard semantics that can be interpreted by
    intermediaries almost as well as by the machines that originate
    services. The result is an application that allows for layers of
    transformation and indirection that are independent of the
    information origin, which is very useful for an Internet-scale,
    multi-organization, anarchically scalable information
    system.\footnote{Fielding, p 142.}}
\end{quote}

\bibliographystyle{plain}
\T\addcontentsline{toc}{section}{References}
\T\raggedright
\bibliography{../../../boiler/dods}

\end{document}
