%
% $Id$
%
% Documentation for the DAP. Intended to be like an RFC document.
%

\documentclass{article}
%\usepackage{html}
%\usepackage{epsfig}
\usepackage{changebar}
\usepackage{hyperlatex}
\usepackage{verbatim}
\usepackage{vcode}
\usepackage{acronym}
\usepackage{xspace}
\usepackage{gloss}
\W\usepackage{sequential}
\W\usepackage{dods-look}

%\W\renewcommand{\HlxIcons}{http://dcz.dods.org/images/}
\W\newenvironment{ttfamily}{\xml{tt}}{\xml{/tt}}

% latex and HTML macros. Some latex commands become nops for HTML. 4/10/2001
% jhrg 
\T\newcommand{\Cpp}{\rm {\small C}\raise.5ex\hbox{\footnotesize++}\xspace}
\T\newcommand{\C}{\rm {\small C}\xspace}

\W\newcommand{\Cpp}{C++}
\W\newcommand{\C}{C}
\W\newcommand{\vdots}{}
\W\newcommand{\ddots}{}

% Note: to get the glossary to work, run bibtex on the *.gls.aux file,
% then latex the file, then bibtex *.gls, then latex again. Also, make
% sure to set your BST and BIBINPUTS environment variables so that the
% BST and BIB files will be found.
\makegloss

% Change paragraph typesetting; eliminate indenting and add more space between
% paragraphs. 2/15/2000 jhrg
\T\setlength{\parindent}{0em}     % Amount of indentation
\T\addtolength{\parskip}{1ex}     % Vertical separation
\T\setcounter{secnumdepth}{4}
\T\setcounter{tocdepth}{4}

\htmltitle{DODS DAP 2.0 -- DRAFT}
\htmladdress{James Gallagher <jgallagher@gso.uri.edu>, $Date$, $Revision$}
\htmldirectory{htmlrfc}
\htmlname{daprfc}

\begin{document}

\title{The DODS Data Access Protocol---DAP 2.0 \\ DRAFT}
\author{James Gallagher\thanks{The University of Rhode Island,
    jgallagher@gso.uri.edu}}
\date{$Date$ \\ $Revision$}

\bibliographystyle{plain}

\maketitle
\T\tableofcontents

\section{Introduction}

\subsection{Purpose}

This specification defines the protocol reffered to as ``DAP/2.0.'' 

The \ac{DAP} is a presentation- and application-level protocol for
distributed access to data organized as name-datatype-value tuples. The
protocol has been used by \ac{DODS} since 1996. The first version of the
protocol delt only with the three psuedo-objects the DAP uses to describe
information sources. Version 2.0 added constraint expressions that provide a
way to extract part of a data source. In addition, version 2.0 added services
that integrate the \ac{DAP} more with the rest of the \ac{WWW}.

While the name-datatype-value model is a nearly universal \emph{conceptual}
organization of data, the actual organization takes nearly as many forms as
there are individual collections. Information in a computer
can often be broken down into a set of variables that have names, datatypes
and values. For example, data from a satellite might be stored in a named
array of bytes. However, there are many different file formats, APIs and
file/directory organizations used to house data.

The \ac{DAP} was designed to hide the implementation of different collections
of data behind a simple language-like interface based on the
name-datatype-value conceptual model. In practice this has worked well.
However, as the number of different implementations of the \ac{DAP}
increases, it becomes more important that they all implement the same
protocol. 

There are now several groups which have built servers
\gloss[nocite]{dap-server} \gloss[nocite]{dap-client} supporting the
\ac{DAP}. This document provides a concise description of the \ac{DAP} to
simplify this task and to ensure that all the servers implement the same
protocol.

As people apply the protocol to a wider variety of data sources, it also
becomes more likely that deficiencies in the protocol will appear. If the
\ac{DAP} is found to fall short in some way, it should be extended and once
implemented, those extensions should be documented by amending this document
and incrementing the protocol's version number.

\subsection{Requirements}

The key words ``MUST'', ``MUST NOT'', ``REQUIRED'', ``SHALL'', ``SHALL NOT'',
``SHOULD'', ``SHOULD NOT'', ``RECOMENDED'', ``MAY'' and ``OPTIONAL'' in this
document are to be interpreted as described in RFC 2119~\cite{rfc2119}.

An implementation is not compliant if it fails to satisfy one or more of the
``MUST'' or ``REQUIRED'' level requirements for the protocols it implements.
An implementation that satisfies all of the ``MUST'' or ``REQUIRED'' level
and all of the ``SHOULD'' level requirements for its protocols is said to be
`unconditionally compliant'; one that satisfies all the ``MUST'' level
requirements but not all the ``SHOULD'' level requirements for its protocols
is said to be `conditionally compliant.'

\subsection{Overall Operation}

The \ac{DAP} uses the request-response
paradigm. A client program issues a request and a named server returns a
response.  Once the server has returned a response, the interaction ends
until a new request is made. The \ac{DAP} uses the \ac{HTTP} to frame the
request-response pair. 

The \ac{DAP} contains elements of both a presentation- and an
application-level protocol. The \ac{DAP} defines ten different responses,
four of which are at the presentation-level; the remaining responses are
application-level.

% The \ac{DAP} is a request-response application-level protocol. It uses
% \ac{HTTP}~\cite{rfc2616} as a session-level protocol~\cite{stevens:unp}. The
% \ac{DAP} uses \ac{HTTP}'s \ac{URL} and MIME-like request and response
% documents. However, it supports only a subset of all the possible \ac{HTTP}
% responses and places further restrictions on both the \ac{URL} and
% request/response headers.

To represent a \gloss[word]{dods-dataset} the \ac{DAP} uses three responses.
Two of these responses, the \ac{DDS} and \ac{DAS}, characterize the
variables, their datatypes, names and \gloss[word]{dods-attributes}. The
third response, the \ac{DataDDS}, holds data values along with name and
datatype information.  The three responses are complete in and of themselves
so that, for example, the data response can be used by a client without ever
requesting either of the two other responses.

The \ac{DAP} sends error information using an Error object. If a request for
any of the three basic responses cannot be returned, an Error response is
returned in its place.

In addition to the three basic responses and the error response, the \ac{DAP}
also defines several other responses which tie \ac{DAP} servers closely to
the \ac{WWW}. These additional responses build \ac{HTML} or text documents
that can be interpreted by \ac{WWW} browsers, \ac{HTML}-aware spreadsheets,
\emph{et cetera}. In most cases these documents are are created by first
accessing one or more of the three basic responses and interpreting their
contents.  Section~\ref{sec:responses} describes the different types of
responses.

\begin{quote}
  \textbf{Note:} The first implementation of the \ac{DAP} was written in \Cpp
  and the three basic responses correspond with objects in that
  implementation. For this reason these responses are referred to as
  `objects' in some of the \ac{DODS} documentation. In some cases it is
  easier to think of these responses as objects and, in those cases, we will
  use that term in this paper, too. See Section~\ref{sec:responses} for a
  discussion of the object/response duality.
\end{quote}

Operationally, a \ac{DAP} client sends a request to a server using \ac{HTTP}.
The request consists of a \ac{HTTP} GET request method, a
\ac{URI}~\cite{rfc2396} that encodes information specific to DODS (see
Section~\ref{sec:url-syntax}) and an \ac{HTTP} protocol version number
followed by a MIME-like message containing various headers that further
describe the request. In practice, DAP clients use a third-party library
implementation of \ac{HTTP}/1.1 so the GET request, \ac{URI} and \ac{HTTP}
version information are hidden from the client; it sees only the \ac{DAP}
\ac{URL} and some of the request headers. The \ac{DAP} server responds with a
status line that includes the \ac{HTTP} protocol version and an error or
success code, followed by a MIME-like message containing information about
the response and the response itself. The \ac{DAP} response is the payload of
the MIME-like \ac{HTTP} response.

\section{Characterization of a Dataset}

The \ac{DAP} characterizes a dataset as a collection of variables and a
collection of attributes. Variables contain the names, datatypes and vlaues
of the information in the dataset. A dataset may also have associated with
each variable zero or more attributes. Each attribute has a name, datatype
and value. 

A dataset's variables are described by two of the three basic responses. The
\ac{DDS} response holds the name and datatype of each variable in the
dataset. It is essentially a declaration of the variables in the dataset.
Sections~\ref{simple-variables} and~\ref{sec:ctor-vars} describe the diffent
types of variables in the \ac{DAP}. 

The \ac{DAS} response holds all of the attribute information for the dataset.
Attributes are a way to bind extra information to a variable and to an entire
dataset. While the datatypes supported for attributes are limited to simple
types and vectors of simple types, the actual collection of attributes is
completely unconstrained. Furthermore, the \ac{DAP} does not have any
standard set of attributes it requires for variables or datasets. See
Section~\ref{sec:das} for more information about the \ac{DAS} response.

Both the \ac{DDS} and \ac{DAS} responses describe the dataset and the
variables it contains. To request data, a client asks a server for a
\ac{DataDDS} object. The \ac{DataDDS} class is a subclass of \ac{DDS} and as
such it holds information about the names and types of variables in the data
source. In addition, a \ac{DataDDS} object holds values for those variables.
See Section~\ref{sec:rep-of-values} for information about encoding values in
a \ac{DataDDS} response.

\subsection{Simple Variables}
\label{simple-variables}

Variables in the \ac{DAP} have two forms. They are either base types or type
constructors. Base type, or simple, variables are similar to predefined
variables in procedural programming languages like C or Fortran (e.g.,
\texttt{int} or \texttt{integer*4}).

\begin{description}
\item [byte] an 8-bit byte;unsigned char in ANSI C\@.
\item [int16] a 16-bit signed integer.
\item [uint16] a 16-bit unsigned integer.
\item [int32] a 32-bit signed integer.
\item [uint32] a 32-bit unsigned integer.
\item [float32] the IEEE 32-bit floating point datatype (ANSI C's
  \texttt{float}). 
\item [float64] the IEEE 64-bit floating point datatype (ANSI C's
  \texttt{double}) .
\item [string] a sequence of bytes terminated by a null character.
\item [URL] represented as a string, but may be dereferenced in a \ac{CE};
  see Section~\ref{sec:ce}.
\end{description}

\begin{ttfamily}
\begin{center}
\begin{tabular}{lll}
simple-decl & = & simple-type id ";" \\
simple-type & = & "byte"| "int16" | "uint16" | "int32" | "uint32" \\
                & & | "float32" | "float64" | "string" | "url" \\
id & = & (ALPHA | "\_" | "\%" | ".") \\
       & & *(ALPHA | DIGIT | "/" | "\_" | "\%" | ".") \\
\end{tabular}
\end{center}
\end{ttfamily}

\subsection{Constructor Variables} 
\label{sec:ctor-vars}

Constructor variables describe the grouping of one or more variables or values
within a data set. These variables are used to describe different types of
relations between the variables that comprise the data set. There are five
classes of type constructor variables defined by the \ac{DAP}: lists, arrays,
structures, sequences, and grids. 

The types are defined in Sections~\ref{sec:list}~to~\ref{sec:grid}.

\subsubsection{List}
\label{sec:list}

 The \textbf{List} type constructor is used to hold lists of 0 or
  more items of one type. Lists of \texttt{byte}, \ldots, \texttt{grid} are
  specified using the keyword \texttt{list} before the variable's class. The
  list can be of any type \textbf{except \texttt{List}}.

\begin{ttfamily}
\begin{center}
\begin{tabular}{lll}
list-decl & = & "list" (simple-decl  | array-decl | structure-decl \\
          & & | sequence-decl | grid-decl) \\
\end{tabular}
\end{center}
\end{ttfamily}

Examples:
\begin{quote}
\begin{vcode}{t}
list int32 heights;
List Float64 x[10][10];
\end{vcode}
\end{quote}

\subsubsection{Array}
An \textbf{Array} is a one dimensional indexed data structure as defined by
ANSI \C. Multidimensional arrays are defined as arrays of arrays. The size of
each array's dimensions must be given. Each dimension of an array may also be
named.

Multi-dimensional arrays are stored in \gloss[word]{row-major} order (as is
the case with ANSI \C or \Cpp). Any array stored in row-major order is stored
so that the last dimension varies most quickly. For example suppose the two
dimensional array \texttt{String letter[2][3]} has two rows of three columns
each and looks like:
\begin{displaymath}
\begin{array}{ccc}
A & B & C \\
D & E & F \\
\end{array}
\end{displaymath}
the values would be stored as $A~B~C~D~E~F$ in memory.

% Define whether multi-dimensional arrays are row-major or not.  Don't
% use the phrase ``row-major'' without also giving an example; people
% don't ever remember which is which.  (Except perhaps for you.)
\begin{ttfamily}
\begin{center}
\begin{tabular}{lll}
array-decl & = & array-types id array-dims ";" \\
array-types & = & simple-decl | structure-decl | sequence-decl | grid-decl \\
array-dims & = & array-dim | array-dim array-dims \\
array-dim & = & "[" [ name "=" ] DIGIT *DIGIT "]" \\
\end{tabular}
\end{center}
\end{ttfamily}

\subsubsection{Structure}
 A structure groups variables so that the collection can be
  manipulated as a single item. The variables can be of any type.

\begin{ttfamily}
\begin{center}
\begin{tabular}{lll}
structure-type & = & "structure" "\{" *structure-types "\}" ";" \\
structure-types & = & simple-type | list-type | array-type \\
                & & | structure-type | sequence-type | grid-type \\
\end{tabular}
\end{center}
\end{ttfamily}

\subsubsection{Sequence}
 A sequence is an ordered set of $N$ variables which has
  several instantiations (or values). Variables in a sequence may be of
  different types.  Each instance of a sequence is one instantiation of the
  variables. Thus a sequence can be represented as:

\begin{displaymath}
\begin{array}{ccc}
  s_{0 0} & \cdots & s_{0 n} \\
  \vdots & \ddots & \vdots \\
  s_{i 0} & \cdots & s_{i n}
\end{array}
\end{displaymath}

\noindent Every instance of sequence $S$ has the same number, order, and
type of variables. Thus in a sequence which contains an array, each instance
of the array MUST be the same size.\footnote{But a sequence may contain a
  list and each instance of the list may have a different number of elements.
  This is because arrays must have their size declared while lists do not.} A
sequence implies that each of the variables is related to each other in some
logical way. A sequence is different from a structure because its constituent
variables have several instances while a structure's variables have only one
instance (or value).

\begin{ttfamily}
\begin{center}
\begin{tabular}{lll}
sequence-decl & = & "sequence" "\{" *sequence-types "\}" ";" \\
sequence-types & = & simple-type | list-type | array-type \\
                & & | structure-type | sequence-type | grid-type \\
\end{tabular}
\end{center}
\end{ttfamily}

\subsubsection{Grid}
\label{sec:grid}
 A grid is an association of an $N$ dimensional array with $N$
  named vectors, each of which has the same number of elements as the
  corresponding dimension of the array. Each vector is used to map indices of
  one of the array's dimensions to a set of values which are normally
  non-integral (e.g., floating point values). The $N$ (map) vectors may be
  different types. Grids are similar to arrays, but add named dimensions and
  maps for each of those dimensions.
  \label{page:grids} 

\begin{ttfamily}
\begin{center}
\begin{tabular}{lll}
grid-decl & = & "grid" "\{" "array:" *array-decl "maps:" *array-decl "\}" ";" \\
\end{tabular}
\end{center}
\end{ttfamily}

\section{Requests}

\subsection{URL Syntax}
\label{sec:url-syntax}
A \ac{DAP} \ac{URL} is essentially an \ac{HTTP} \ac{URL}~\cite{rfc2616} with
additional restrictions placed on the \texttt{abs\_path} component.

\begin{ttfamily}
\begin{center}
\begin{tabular}{lll}
DAP\_URL & = & "http:" "//" host [ ":" port ] [ abs\_path ] \\
abs\_path & = & server\_path dataset\_id "." ext [ "?" query ] \\
server\_path & = & \\
dataset\_id &= & \\
ext & = & "das" | "dds" | "dods" | "ver" | "html" | "info" | "asc" \\
\end{tabular}
\end{center}
\end{ttfamily}

The \ac{DAP} uses \ac{HTTP} as its session protocol~\cite{stevens:unp}(need a
better reference), so every \ac{DAP} \ac{URL} starts with the scheme
\texttt{http:}. The \texttt{host} and optional \texttt{port} name a host and
TCP port of an \ac{HTTP} server that will handle the session. The
\texttt{host} may also contain authentication information as described in RFC
2617~\cite{rfc2617}.

The \texttt{abs\_path} portion of the \texttt{DAP\_URL} is composed of four
parts:
\begin{description}
\item [server\_path] A pathname which identifies the \ac{DAP} server to
  handle the request. The servers may be implemented as \ac{CGI} programs or
  they may use another equivalent scheme (e.g., the Apache \ac{HTTP} daemon's
  module system).
\item [dataset\_id] A string passed to the server named by
  \texttt{server\_path} that uniquely identifies the source of data on
  \texttt{host}. The \texttt{dataset\_id} may take the form of a pathname
  within the \ac{HTTP} server's document root directory, or it may name the
  data source in some other way (e.g., the \ac{DAP} server might maintain a
  table of names mapped to tables in a relational database).
  
  Two special \texttt{dataset\_id}s SHOULD be recognized by a \ac{DAP}
  server. They are \texttt{version} and \texttt{help}. When a \ac{DAP} server
  receives the \texttt{dataset\_id} \texttt{version} it SHOULD respond
  with version information (see Section~\ref{sec:version}). When a \ac{DAP}
  server receives the \texttt{dataset\_id} \texttt{help} it SHOULD respond
  with a help message (see Section~\ref{sec:help}).
\item [ext] The \texttt{ext} part of the \texttt{abs\_path} tells the
  \ac{DAP} server which type of response to return. Each response has a three
  letter string that is used by the requester. See
  Section~\ref{sec:responses} for a description of the responses and the
  \texttt{ext} strings used to request them.
\item [query] The optional \texttt{query} part of the \texttt{abs\_path} is
  used with ASCII and binary data requests to limit those requests to
  specific variables or values within the data source. See
  Section~\ref{sec:ce}.
\end{description}

\subsubsection{Constraint expressions}
\label{sec:ce}

A \ac{CE} provides a way for clients to request certain
variables, or parts of certain variables, from a dataset. Many datasets are
large and many variables in datasets are also large. Clients are often
interested in only a small number of values from the entire dataset.
Constraint expressions provide a way for clients to tell a server which
variables, and in many cases, parts of those variables, thy would like.

Constraint expressions have the following syntax:

\begin{ttfamily}
\begin{center}
\begin{tabular}{lll}
CE & = & *(projection) *(\& selection) \\
projection & = & ids | function \\
ids & = & id | id , ids \\
function & = & id ( args ) \\
args & = & arg | args arg \\
arg & = & id | <quoted string> | <int> | <float> | [ deref ] URL \\
\end{tabular}
\end{center}
\end{ttfamily}

A \ac{CE} has two parts, the projection and the selection. A project lists the
variables to be returned by the \ac{DAP} server. If more than one variable is
to be returned, then the projection is a comma separated list of variables.
Leaving the projection part of the The \ac{CE} empty is shorthand for
requesting all the variables in the dataset.

To request one field of a constructor type, set \texttt{id} to the name of
the constructor, followed by a dot (\texttt{.}) and the field name. To
request all of the fields in a constructor, set \texttt{id} to the name of
the constructor.

An array may be accessed using only its name to return the entire array or
using a hyperslab (\texttt{[]}) operator to return a rectangular section of
the array. In the later case, the hyperslab is defined for each dimension by
a starting index, and ending index, and a stride value. \cbstart An Array or
Grid variable must either be unconstrained or have a hyperslab constraint for
each of its dimensions. It is an error for an $N$ dimensional object to have
$N-1$ or fewer hyperslabs Obviously, it is also an error for such a variable
to have $N+1$ or greater hyperslabs. In either of these two cases the server
must return an \texttt{Error} response. \cbend Specifying a stride $>$ 1 will
cause the dimension to be subsampled by the stride value.

When an object of type URL is dereferenced using the \texttt{*} operator, the
server must open the data set referenced by that URL and read its value(s).
The value(s) is/are used in the constraint. If the URL returns a datatype
that is incompatible with the context in which it is used, the server must
return an \texttt{Error} response. If the URL is not explicitly dereferenced
then it is treated as a string literal (with the URL text as its value).

\begin{ttfamily}
\begin{center}
\begin{tabular}{lll}
array-dim & = & [ start : stride : stop ] \\
          & & [ start : stop ] ; stride is one \\
\end{tabular}
\end{center}
\end{ttfamily}

In addition to variable names, functions can be called in the projection
subexpression. There is not limit to what these function can do so long as
the \ac{DAP} is not violated. These are an experimental part of the \ac{DAP}
and are used to insert new variables into the \ac{DDS}. 

\begin{ttfamily}
\begin{center}
\begin{tabular}{lll}
selection & = & *relation | *function \\
relation & = & (id rel\_op id) | (value rel\_op id) \\
         & & | (id rel\_op value) \\
value & = & id | <quoted string> | <int> | <float> | [ deref ] URL \\
deref & = & "*" \\
URL & = & <any valid DODS URL> \\
\end{tabular}
\end{center}
\end{ttfamily}

All of the classes listed in section~\ref{simple-variables} are local to the
data set except the \texttt{URL} type. \ac{URL}s are handled as described
previously.

\subsection{Request Headers}

The headers described in Sections~\ref{sec:accept-encoding}
to~\ref{sec:accept-types} MUST be handled as described. Other headers which
are part of \ac{HTTP} 1.1 MAY be included in the request and MAY be honored
by a \ac{DAP} server.

\subsubsection{Accept-Encoding}
\label{sec:accept-encoding}
The \texttt{Accept-Encoding} request-header is used by a \ac{DAP} client to
tell a server that it can accept compressed responses. See RFC
2616~\cite{rfc2616} for this header's grammar.  Values for encodings are
\texttt{deflate}, \texttt{gzip} and \texttt{compress}. This header is
OPTIONAL. When a client includes this header it is effectively asking the
\ac{DAP} server to encode the response using the given scheme. The server is
under no obligation to use the requested encoding. However, a server MUST NOT
use an encoding when a client has not requested it. In addition, a client
MUST supply the header with every request for which it desires a special
encoding.

\subsubsection{Host}
\label{sec:host}
The \texttt{Host} request-header is used by a \ac{DAP} client to provide its
IP address or DNS name to the \ac{DAP} server. See RFC 2616~\cite{rfc2616}
for this header's grammar. This header MUST be included with every request.

\subsubsection{User-Agent}
\label{sec:user-agent}
The \texttt{User-Agent} request-header is used by a \ac{DAP} client to
provide specific information about the client software to the \ac{DAP}
server. See RFC 2616~\cite{rfc2616} for this header's grammar. This header is
RECOMMENDED. \ac{DAP} servers MAY log this information.

\subsubsection{XDODS-Accept-Types}
\label{sec:accept-types}
The \texttt{XDODS-Accept-Types} experimental request-header is used by a
\ac{DAP} client to tell a \ac{DAP} server which of the standard \ac{DAP}
datatypes the client can understand. The server SHOULD NOT return datatypes
to the client that it does not understand. This header is OPTIONAL; if absent
a server can assume that a client understands all the standard datatypes. 

\begin{ttfamily}
\begin{center}
\begin{tabular}{lll}
XDODS-Accept-Types & = & [ "!" ] type\_list | all \\
type\_list & = & type | type "," type\_list \\
all & = & "All" \\
type & = & "Byte" | "Int16" | "UInt16" | "Int32" \\
       & & | "UInt32" | "Float32" | "Float64" \\
       & & | "String" | "URL" | "Array" | "List" \\ 
       & & | "Structure" | "Sequence" | "Grid" \\
\end{tabular}
\end{center}
\end{ttfamily}

A \ac{DAP} server determines which datatypes are understood by a client using
the following rules:
\begin{enumerate}
\item If the value of the header is \texttt{All}, then the client understands
  all the datatypes.
\item If a datatype is listed in the value-string, then that type is
  understood, otherwise it is not understood.
\item If the value-string starts with the literal \texttt{!} then the type(s)
  listed are \emph{not} understood by the client and the remaining types are
  understood. 
\end{enumerate}

\section{Responses}
\label{sec:responses}

A valid \ac{DAP} response has the same form as a valid \ac{HTTP} response.
The first line contains the \ac{HTTP} protocol version, a status code and
reason phrase~\cite{rfc2616}. Following this are the response headers which
vary depending on the request and payload of the response (see
Section~\ref{sec:resp-headers} for a description of the headers). As
described in RFC 822~\cite{rfc822}, the \ac{HTTP} response status line and
headers are separated from the response's payload by an extra set of CRLF
characters which make a blank line.

The ten response payloads defined by the \ac{DAP} are described in detail
in Section~\ref{sec:resp-bodies}. 

\emph{[Add some text describing how the three basic responses can be thought
  of as objects]}

\subsection{Response Headers}
\label{sec:resp-headers}

\subsubsection{Content-Description}
The \texttt{Content-Description} header is used to tell clients which of the
different basic responses is being returned or if an error message is being
returned. For any of the basic responses (\ac{DDS}, \ac{DAS}, or
\ac{DataDDS}) or the error response, this header MUST be included. This
header SHOULD NOT be included by the ASCII, HTML, INFO, version or directory
responses. See RFC 2045~\cite{rfc2045} for information about this header.

\begin{ttfamily}
\begin{center}
\begin{tabular}{lll}
Content-Description & = & "Content-Description" ":" tag \\
tag & = & "dods-dds" | "dods-das" | "dods-data" | "dods-error" \\
\end{tabular}
\end{center}
\end{ttfamily}

Example:
\begin{vcode}{ct}
Content-Description: dods-error
\end{vcode}

\subsubsection{Content-Encoding}
If a \ac{DAP} server applies an encoding to an entity, it MUST include the
\texttt{Content-Encoding} header in the response. See RFC 2616~\cite{rfc2616}
for this header's grammar. The sole recognized encoding for the \ac{DAP} is \texttt{deflate}.

Example:
\begin{vcode}{ct}
Content-Encoding: deflate
\end{vcode}

\subsubsection{Content-Type}
The \texttt{Content-Type} header MUST be included in any response from a
\ac{DAP} server. Valid content types for \ac{DAP} responses are:
\texttt{text/plain}, \texttt{text/html} and
\texttt{application/octet}.\footnote{It would be better to use a multipart
  document in place of the \texttt{application/octet}.} See RFC
2616~\cite{rfc2616} for this header's grammar.

Example:
\begin{vcode}{ct}
Content-Type: application/octet
\end{vcode}

\subsubsection{Date}
The \texttt{Date} header provides a time stamp for the response. This header
is needed for servers that support caching. See RFC 2616~\cite{rfc2616}
for this header's grammar. Servers MUST provide this header.

Example:
\begin{vcode}{ct}
Date: Fri, 09 Feb 2001 18:54:55 GMT
\end{vcode}

\subsubsection{Keep-Alive}
A \ac{DAP} server (or an underlying \ac{HTTP} server if one is used to
implement the \ac{DAP} server) MAY return a \texttt{Keep-Alive} header for an
authenticate (code 401) response. For all other responses, the \ac{DAP}
server MUST NOT return this header.\footnote{This is a shortcoming of the
  \ac{DAP}. It should support HTTP/1.1's persistent connections. However, to
  do requires that the responses also return Content-Length. Since none of our
  servers do this, I've got no experience with persistent connections.} See
RFC 2616~\cite{rfc2616} for this header's grammar.

\subsubsection{Server}
The \texttt{Server} header provides information about the server used to
process the request. In this case the \emph{server} MAY be either the
\ac{DAP} server or an underlying \ac{HTTP} server if the \ac{DAP} server uses
that as part of its implementation. See RFC 2616~\cite{rfc2616}
for this header's grammar. This header is OPTIONAL.

Example:

\begin{vcode}{ct}
Server: Apache/1.3.12 (Unix)  (Red Hat/Linux) PHP/3.0.15 mod\_perl/1.21
\end{vcode}

\subsubsection{WWW-Authentication}
The \texttt{WWW-Authenticate} header MUST be included in a message that has a
response code of 401. That is, when the \ac{DAP} server is asked to provide
access to a resource that is restricted and the request does not include
authentication information (see ``HTTP Authentication: Basic and Digest
Access Authentication''~\cite{rfc2617}). then it must return with a response
code of 401 and include the \texttt{WWW-Authenticate} header. See RFC
2616~\cite{rfc2616} for this header's grammar.

Example:

\begin{vcode}{ct}
WWW-Authenticate: Basic realm="special directory, with CGIs"
\end{vcode}

\subsubsection{XDODS-Server}
The \texttt{XDODS-Server} header is used to return \ac{DAP} server's
implementation version information to the client program.\footnote{The
  version information should be changed to reflect the version of the
  \ac{DAP}.} This header MUST be included in every response.

\begin{ttfamily}
\begin{center}
\begin{tabular}{lll}
XDODS-Server & = & "XDODS-Server" ":" "dods/" version \\
version & = & DIGIT "." DIGIT [ "." DIGIT ]
\end{tabular}
\end{center}
\end{ttfamily}

Example:
\begin{vcode}{ct}
XDODS-Server: dods/3.2.2
\end{vcode}

\subsection{Response Bodies}
\label{sec:resp-bodies}

\emph{[Add a paragraph about the DAS, DDS and DataDDS responses. Also, the
  Error response]}
\emph{[Make each response stand alone (list the special headers it must use
  plus say which extension triggers it.]}

The ASCII, INFO, HTML or Directory services all return \ac{HTML} or plain
text documents. The ASCII service returns data in a plain text document so
that it may be easily read into a spreadsheet of similar program. The INFO
service formats information contained in the \ac{DDS} and \ac{DAS} objects
using \ac{HTML} so that it may be displayed in a \ac{WWW} browser. The HTML
service returns a dynamic \ac{WWW} form that can be used to request specific
variables from the data source. This service provides a low-level user
interface to a data source. The Directory service provides a rudimentary form
of navigation for data sources which are composed of multiple files.

\subsubsection{DAS}
\label{sec:das}
The \ac{DAS} entity is returned as the payload of a message whose
\texttt{Content-Type} header MUST be \texttt{text/plain}. The body of the
response contains text which holds all of the attribute containers and
values.

The \ac{DAS} is used to store attributes for both the entire dataset and
variables in the dataset. The \ac{DAS} consists of a number of
\emph{containers} each of which hold zero or more attributes. Each attribute
is a name-datatype-value tuple. Values may be either scalar or vector. Note
that two, three, \ldots, dimensional attribute values are \emph{not}
supported.  The name of the attribute container MUST be the same as
the name of the variable to which its attributes refer.
% added a MUST.  ts.

A \ac{DAS} MUST have a container for each variable in the dataset. It MAY
contain any number of extra containers.

\begin{ttfamily}
\begin{center}
\begin{tabular}{lll}
das-doc & = & "attributes" "{" *attribute-cont "}" \\
attribute-cont & = & attribute-cont | attribute \\
attribute & = & simple-decl id value *(, value) ";" \\
value & = & <Float> | <int> | id | <quoted string> \\
\end{tabular}
\end{center}
\end{ttfamily}

The purpose of attributes is to hold additional information beyond the name,
datatype and value of a variable and/or to hold extra information about a
dataset as a whole. This extra information can make the contents of the
dataset much easier to use. For example, extra information contained in the
\ac{DAS} might provide unit names, scaling factors or the missing-value
values. Additional information about an entire dataset might contain
information about who collected the information, under what circumstances
\emph{et cetera}. 

Many systems rely on attributes to store extra information that is necessary
to perform certain manipulations with data. In effect, attributes are used to
store information that is used `by convention' rather than `by design'.
\ac{DODS} can effectively support these conventions by passing the attributes
from data set to user program via the \ac{DAS}. Of course, \ac{DODS} cannot
enforce conventions in data sets where they were not followed in the first
place.

An example \ac{DAS} is shown in Figure~\ref{fig:das}.

\begin{figure}
\begin{vcode}{cb}
attributes {
   catalog_number {
   }
   casts {
      experimenter {
      }
      time {
         string units "hour since 0000-01-01 00:00:00";
         string time_origin "1-JAN-0000 00:00:00";
      }
      location {
         lat {
            string long_name "Latitude";
            string units "degrees_north";
         }
         lon {
            string long_name "Longitude";
            string units "degrees_east";
         }
      }
      xbt {
         depth {
            string units "meters";
         }
         t {
            float32 missing_value -9.99999979e+33;
            float32 _fillvalue -9.99999979e+33;
            string history "From coads_climatology";
            string units "Deg C";
         }
      }
   }
}
\end{vcode}
\caption{Example Dataset Attribute Entry. This example matches the DDS shown
   in Figure~\ref{fig:dds}. Some of the variables in this fictional dataset
   (e.g., \texttt{catalog\_number}) have no attributes. }
\label{fig:das}
\end{figure}

\subsubsection{DDS}
\label{sec:dds}
The \ac{DDS} entity is returned as the payload of a message whose
\texttt{Content-Type} header MUST be \texttt{text/plain}. The body of the
response contains text which holds all of the variables, their names and
datatypes.

The DDS is a textual description of the variables and their names and types
that compose the entire data set. The data set descriptor syntax is similar
too the variable declaration/definition syntax of {\small C} and \Cpp. A
variable that is a member of one of the base type classes is declared by by
writing the class name followed by the variable name. The type constructor
classes are declared using {\small C}'s brace notation.

\begin{ttfamily}
\begin{center}
\begin{tabular}{lll}
dds-doc & = & "dataset" "\{" *type-decl "\}" id ";" \\
type-decl & = & simple-decl | list-decl | array-decl \\
          & & | structure-decl | sequence-decl | grid-decl \\
\end{tabular}
\end{center}
\end{ttfamily}

The \texttt{dataset} keyword has the same syntactic function as
\texttt{structure} but is used for the specific job of enclosing the entire
dataset even when it does not technically need an enclosing element (because
at the outermost level it is a single element such as a structure or
sequence).

An example DDS is shown in Figure~\ref{fig:dds}.

\begin{figure}
\begin{vcode}{cb}
dataset {
   int catalog_number;
   sequence {
      string experimenter;
      int32 time;
      structure {
         float64 latitude;
         float64 longitude;
      } location;
      sequence {
         float depth;
         float temperature;
      } xbt;
   } casts;
} data;
\end{vcode}
\caption{Example Dataset Descriptor Entry.}
\label{fig:dds}
\end{figure}


\subsubsection{DODS}
\label{sec:dods}
The \ac{DataDDS} entity is returned as the payload of a message whose
\texttt{Content-Type} header MUST be
\texttt{application/octet}.\footnote{This should be multipart/binary.} The
body of the response contains both text which holds a \ac{DDS} which
describes the variables listed in the request and the values for those
variables encoded using XDR\cite{xdr}. The text \ac{DDS} and the binary data
are separated in the response entity by the literal \texttt{Data:}.
\begin{ttfamily}
\begin{center}
\begin{tabular}{lll}
DataDDS & = & DDS CR "Data:" CR <Binary values>
\end{tabular}
\end{center}
\end{ttfamily}

Clients MAY supply a constraint expression (see Section~\ref{sec:ce}) with
any \texttt{DataDDS} request. The \ac{DDS} in the \texttt{DataDDS} response
describes the variables returned. The order that the variables are listed in
the \ac{DDS} MUST match the order of the values in the binary section of the
\ac{DataDDS} response. If the response contains constructor types, then the
variables are sent in the order they would be visited in a depth-first
traversal of the accompanying \ac{DDS}.

\paragraph{Encoding Values}
\label{sec:rep-of-values}

\emph{[Describe different types of variables as three groups: scalar, vector
  and aggregate]}

The \ac{DAP} uses Sun Microsystems' XDR protocol~\cite{xdr} for the external
representation of all of the base type variables. Table~\ref{tab:base-xdr}
shows the XDR types used to represent the various base type
variables.

\begin{table}
\caption{The XDR data types used by the DAP as the external representations
  of base-type variables}
\label{tab:base-xdr}
\begin{center}
\begin{tabular}{|l|l|} \hline
\multicolumn{1}{|c}{\textsc{Base Type}} & \multicolumn{1}{c|}{\textsc{XDR Type}} \\
\hline \hline
\texttt{byte} & \texttt{xdr byte} \\ \hline
\texttt{int16} & \texttt{xdr short} \\ \hline
\texttt{uint16} & \texttt{xdr unsigned short} \\ \hline
\texttt{int32} & \texttt{xdr long} \\ \hline
\texttt{uint32} & \texttt{xdr unsigned long} \\ \hline
\texttt{float32} & \texttt{xdr float} \\ \hline
\texttt{float64} & \texttt{xdr double} \\ \hline
\texttt{string} & \texttt{xdr string} \\ \hline
\texttt{URL} & \texttt{xdr string} \\ \hline
\end{tabular}
\end{center}
\end{table}

In order to transmit constructor type variables, the \ac{DAP} defines how the
various base type variables, which comprise the constructor type variable,
are transmitted. Any constructor type variable may be subject to a constraint
expression which changes the amount of data transmitted for the variable (see
Section~\ref{sec:ce}). For each of the six constructor types these
definitions are:

\begin{description}
  
\item [Array] An array id first sent by sending the number of elements in the
  array twice.\footnote{This is an artifact of the first implementation of
    the DAP and XDR. The DAP software needed length information to allocate
    memory for the array so it sent the array length. However, XDR also sends
    the array length for its own purposes. This could be fixed but it is more
    of an annoyance than anything else.} The array lengths are 32-bit
  integers encoded using \texttt{xdr\_long}.

  Following the length information, each array element is encoded in
  succession. Arrays of bytes are handled differently than other arrays:
\begin{enumerate}
\item An array of bytes: These are encoded as is and are padded to a
  four-byte boundary. Thus an array of five bytes will be encoded as eight
  bytes.
  
\item One dimensional arrays of all other types are encoded by encoding each
  element of the array in the order they appear.

\item Multi0dimensional arrays are encoded by encoding the elements using
  row-major ordering.
\end{enumerate}

\begin{ttfamily}
\begin{center}
\begin{tabular}{lll}
Array & = & length length values \\
length & = & <32-bit integer, signed, big endian> \\
values & = & bytes | other-values \\
bytes & = & <8-bit bytes padded to a four-byte boundary> \\
other-values & = & numeric-values | strings | aggregates \\
\end{tabular}
\end{center}
\end{ttfamily}

\item [List] A list is sent as if it were an array. Even though the length of
  a list is not declared, at the time the list's value(s) are to be sent, its
  length must be known. Thus it is possible to think of a list as a vector of
  values and hence use the same encoding for those values as would be used
  for an equivalent array.

\item [Structure] A structure is sent by encoding each field in the order
  those fields are declared in the structure. For example, the structure:

\begin{vcode}{it}
Structure {
    int32 x;
    float64 y;
} a;
\end{vcode}

Would be sent by encoding the int32 \texttt{x} and then the float64
\texttt{y}. 

Nested structures are sent by encoding their `leaf nodes' as visited in a
depth first traversal. For example:

\begin{vcode}{it}
Structure {
    int32 x;
    Structure {
        String name;
        Byte image[512][512];
    } picture;
    float64 y;
} a;
\end{vcode}

Would be sent by encoding \texttt{x}, then \texttt{name}, \texttt{image} and
  finally \texttt{y}.

\item [Sequence] A Sequence is transmitted by encoding each instance as for a
  structure and sending one after the other, in the order of their occurrence
  in the data set. The entire sequence is sent, subject to the constraint
  expression. In other words, if no constraint expression is supplied then
  the entire sequence is sent. However, if a constraint expression is given
  all the records in the sequence that satisfy the expression are
  sent

  Because a sequence does \emph{not} have a length count, each instance
  is prefixed by a \texttt{start of sequence} marker. Also, to accommodate
  nested sequences, then end of each sequence as a whole is marked by a
  \texttt{end of sequence} marker.

\begin{ttfamily}
\begin{center}
\begin{tabular}{lll}
sequence & = & instances end-of-seq \\
instances & = & start-of-inst instance-values \\
end-of-seq & = & <byte value 0xA5> \\
start-of-inst & = & <byte value 0x5A> \\
\end{tabular}
\end{center}
\end{ttfamily}

\item [Grid] A grid is encoded as if it is a Structure (one component
  after the other, in the order of their declaration).

\end{description}

\subsubsection{ERROR}
\label{sec:error}
When a server encounters an error, either in its software or in the users
request, it MUST return an error response. The body of the response contains
an error code along with text that provides a description of the problem
encountered. Server writers are encouraged to provide text that describes the
problem with enough information to enable a user to correct the problem or
submit a meaningful bug report to the server's maintainer.

\emph{[Error objects can, in theory, contain a short Java or TCL/Tk program
  to help users resubmit URLs, et cetera. However, we've never used this
  feature]}

\begin{ttfamily}
\begin{center}
\begin{tabular}{lll}
Error & = & "Error" "\{" "code=" error-code ";" \\
      & & "message=" error-msg ";" "\}" \\
error-code & = & 1*DIGIT \\
error-msg & = & quoted-string \\
\end{tabular}
\end{center}
\end{ttfamily}

\subsubsection{Version}
\label{sec:version}
The \texttt{version} response returns information about the \ac{DAP} version,
server version and may return information about a dataset's version. The
response may be requested two ways: by using the string \texttt{version}
as the \texttt{dataset\_id} or by appending the extension
\texttt{ver} to the dataset name (see Section~\ref{sec:url-syntax}). 

\begin{ttfamily}
\begin{center}
\begin{tabular}{lll}
abs\_path & = & server\_path dataset\_id "." ext [ "?" query ] \\
server\_path & = & <name of DAP server> \\
dataset\_id & = & "version" \\
ext & = & "ver" \\
\end{tabular}
\end{center}
\end{ttfamily}

If a \ac{DAP} server receives a \texttt{version} request, it MUST return
return \ac{DAP} version version information. If the request is made using the
\texttt{ver} extension to a \texttt{dataset\_id} then the server MUST return
the \ac{DAP} version and server version information. It MAY also return a
dataset version.

Version information should be returned as plain text in the payload of the
response. This version information may be essentially that sane as the
information in the XDODS-Server header. The intent is to present users and
system maintainers with information about servers that can be used to track
down problems or determine if a server can be upgraded to a newer version to
fix a particular problem.

\begin{ttfamily}
\begin{center}
\begin{tabular}{lll}
version-response & = & dap-version CRLF server-version \\
                 & & [ CRLF dataset-version ] \\
dap-version & = & "Core version:" token "/" version-number \\
server-version & = & "Server version:" token "/" version-number \\
dataset-version & = & "Dataset version:" token "/" version-number \\
token & = & 1*<any CHAR except CTLs or separators> \\
version-number & = & 1*DIGIT "." 1*DIGIT "." 1*DIGIT \\
\end{tabular}
\end{center}
\end{ttfamily}

\subsubsection{ASCII}
\label{sec:ascii}
The \texttt{ASCII} response provides a way for a client to get data from a
\ac{DAP} server formatted as ASCII text. Server implementors may develop
custom ASCII representations for their data as long as they meet the
requirements given here:

\begin{enumerate}
\item The formatting MUST use \ac{CSV} with text that associates a name with
  a value or group of values.
\item The response body SHOULD be easy to incorporate into commonly available
  spreadsheet programs.
\item There MUST be enough structural information in the response to remove
  any ambiguity about which variable(s) it contains.
\end{enumerate}

\paragraph{Suggested ASCII representations}

In this section variables are classified as either simple types or complex
types. Simple types are the elemental datatypes in DODS (Byte, \ldots, Url).
Complex types are Arrays, Lists and the constructor types Structure, Sequence
and Grid. Since a dataset groups together a collection of otherwise unrelated
variables, it will be considered a Structure in this section. That is, each
dataset has an implicit (anonymous) Structure as its outermost variable.

Simple datatypes are represented in ASCII using common print representations
for those those types. Integer and floating point variables can be converted
using whatever language and/or library facilities are available to the
server.

The ASCII representations for complex datatypes are given below. In general,
complex datatypes each have two different representations, one when they
contain only other simple types and another when they contain other complex
types. 

\begin{enumerate}
\item Grid
\begin{enumerate}
\item A Grid with only a single dimension will be represented as a table of
  values. The values of the sole map vector will be printed in a row,
  separated by commas and prefixed by the map vector's name. The values of
  the Grid's Array will be prefixed by its name and printed on the following
  row, also as CSVs. See Example ??.
\item A Grid with two or more map vectors will be printed as follows. The
  rightmost dimension/map will be printed on one line as CSVs and prefixed
  by the map vector's name as is the case for a Grid with only a single map.
  The Array-part's values will be printed on subsequent lines which enumerate
  all of the indices of the other dimensions/maps so that the rightmost
  varies fastest. Each row of values will be prefixed by the array-part's
  name and the list of bracketed map vector values which describe the given
  row's data. See Examples ??--??.
\end{enumerate}

\T\begin{minipage}{5in}
Example :
\T\medskip

\begin{vcode}{ti}
Dataset {
  Grid {
    Array:
      Byte x[5];
    Maps:
      Float64 lat[5];
  } values;
} ex14;
\end{vcode}
\T\medskip

Output:\\
\begin{ttfamily}
Dataset: $dataset name$\\
values.lat, $values.lat[0]$, \ldots, $values.lat[4]$\\
values.x, $values.x[0]$, \ldots, $values.x[4]$
\end{ttfamily}
\T\end{minipage}
\T\bigskip

\T\begin{minipage}{5in}
Example :
\T\medskip

\begin{vcode}{it}
Dataset {
  Grid {
    Array: 
      Byte temp[lat=5][lon=5];
    Maps:
      Float64 lat[5];
      Float64 lon[5];
  } v;
} ex15;
\end{vcode}
\T\medskip

Output:\\
\begin{ttfamily}
Dataset: $dataset name$\\
v.lon, $v.lon[0]$, \ldots, $v.lon[4]$\\
v.temp[v.lat=$v.lat[0]$], $v.temp[0][0]$, \ldots, $v.temp[0][4]$\\
\vdots\\
v.temp[v.lat=$v.lat[4]$], $v.temp[4][0]$, \ldots, $v.temp[4][4]$
\end{ttfamily}
\T\end{minipage}
\T\bigskip

\T\begin{minipage}{5in}
Example :
\T\medskip

\begin{vcode}{it}
Dataset {
  Grid {
    Array: 
      Byte temp[t=4][lat=5][lon=5];
    Maps:
      Int32 t[4];
      Float64 lat[5];
      Float64 lon[5];
  } v;
} ex16;
\end{vcode}
\T\medskip

Output:\\
\begin{ttfamily}
Dataset: $dataset name$\\
v.lon, $v.lon[0]$, \ldots, $v.lon[4]$\\
v.temp[v.t=$v.t[0]$][v.lat=$v.lat[0]$], $temp[0][0][0]$, \ldots, $temp[0][0][4]$\\
\vdots\\
v.temp[v.t=$v.t[0]$][v.lat=$v.lat[4]$], $temp[0][4][0]$, \ldots, $temp[0][4][4]$\\
\vdots\\
v.temp[v.t=$v.t[3]$][v.lat=$v.lat[0]$], $temp[3][0][0]$, \ldots, $temp[3][0][4]$\\
\vdots\\
v.temp[v.t=$v.t[3]$][v.lat=$v.lat[4]$], $temp[3][4][0]$, \ldots, $temp[3][4][4]$\\
\end{ttfamily}
\T\end{minipage}
\T\bigskip

\item Structure
\begin{enumerate}
\item Structures that contain only simple types are printed as a table with a
  single row. That is, a row of comma separated field names followed by a row
  of comma separated values. See Example ??.
\item A Structure contains simple types and/or other structures which
  themselves contain only simple types will be flattened and printed as if it
  were a single structure composed of solely of simple types. See Example ??.
\item A Structure that contains one or more Array, Sequence or Grid variables
  uses a vertical ordering of its fields. Each field is printed by writing
  its print representation on successive rows, starting with the first field
  and going in order to the last. For each row the type of variable
  determines the specific format of the output. Example ?? shows a structure
  which contains a String and a Grid.
\end{enumerate}

\T\begin{minipage}{5in}
Example :
\T\medskip

\begin{vcode}{it}
Dataset {
  Structure {
    String name;
    String date;
    Float64 lat;
    Float64 lon;
  } drop;
} ex6;
\end{vcode}
\T\medskip

Output:\\
\begin{ttfamily}
Dataset: $dataset name$\\
drop.name, drop.date, drop.lat, drop.lon\\
$drop.name$, $drop.date$, $drop.lat$, $drop.lon$
\end{ttfamily}
\T\end{minipage}
\T\bigskip

\T\begin{minipage}{5in}
Example :
\T\medskip

\begin{vcode}{it}
Dataset {
  Structure {
    String name;
    String date;
    Structure {
      Float64 lat;
      Float64 lon;
    } point;
  } drop;
} ex7;
\end{vcode}
\T\medskip

Output:\\
\begin{ttfamily}
Dataset: $dataset name$\\
drop.name, drop.date, drop.point.lat, drop.point.lon\\
$drop.name$, $drop.date$, $drop.point.lat$, $drop.point.lon$
\end{ttfamily}
\T\end{minipage}
\T\bigskip

\T\begin{minipage}{5in}
Example :
\T\medskip

\begin{vcode}{it}
Dataset {
  Structure {
    String date;
    Grid {
      Array: 
      Byte temp[lat=5][lon=5];
      Maps:
      Float64 lat[5];
      Float64 lon[5];
    } values;
  } test;
} ex8;
\end{vcode}
\T\medskip

Output:\\
\begin{ttfamily}
Dataset: $dataset name$\\
test.date, $test.date$\\
test.v.lon, $test.v.lon[0]$, \ldots, $temp.lon[4]$\\
test.v.temp[$test.v.lat[0]$]: $test.v.temp[0][0]$, \ldots, $test.v.temp[0][4]$\\
\vdots\\
test.v.temp[$test.v.lat[4]$]: $test.v.temp[4][0]$, \ldots, $test.v.temp[4][4]$
\end{ttfamily}
\T\end{minipage}
\T\bigskip

\item Sequence
\begin{enumerate}
\item Sequences containing only simple types will be represented as tables.
  The first row of output will list the Sequence's field names separated by
  commas. Each instance of the Sequence will be listed on subsequent rows as
  comma separated values.
\item Sequences containing simple types and/or other Sequences or Structures
  which themselves contain only simple types will be flattend and printed as
  a table.
\item Sequences which contain either, directly or indirectly, Arrays or Grids
  will be printed with each field on its own row. Successive rows of the
  Sequence will be printed as successive blocks of rows. To print each field
  its type-dependent print representation will be used. 
\end{enumerate}

\T\begin{minipage}{5in}
Example :
\T\medskip

\begin{vcode}{it}
Dataset {
  Sequence {
    String name;
    Float64 lat;
    Float64 lon;
  } drop;
} ex8;
\end{vcode}
\T\medskip

Output:\\
\begin{ttfamily}
Dataset: $dataset name$\\
drop.name, drop.lat, drop.lon\\
$drop.name$, $drop.lat$, $drop.lon$\\
\vdots
\end{ttfamily}
\T\end{minipage}
\T\bigskip

\T\begin{minipage}{5in}
Example :
\T\medskip

\begin{vcode}{it}
Dataset {
  Sequence {
    String name;
    Sequence {
      Float64 lat;
      Float64 lon;
    } loc;
  } drop;
} ex10;
\end{vcode}
\T\medskip

Output:\\
\begin{ttfamily}
Dataset: $dataset name$\\
drop.name, drop.loc.lat, drop.loc.lon\\
$drop.name$, $drop.loc.lat$, $drop.loc.lon$\\
\vdots
\end{ttfamily}
\T\end{minipage}
\T\bigskip

\T\begin{minipage}{5in}
Example :
\T\medskip

\begin{vcode}{it}
Dataset {
  Sequence {
    String name;
    Structure {
      Float64 lat;
      Float64 lon;
    } loc;
  } drop;
} ex11;
\end{vcode}
\T\medskip

Output:\\
\begin{ttfamily}  
Dataset: $dataset name$\\
drop.name, drop.loc.lat, drop.loc.lon\\
$drop.name$, $drop.loc.lat$, $drop.loc.lon$\\
\vdots
\end{ttfamily}
\T\end{minipage}
\T\bigskip

\T\begin{minipage}{5in}
Example :
\T\medskip

\begin{vcode}{it}
Dataset {
  Sequence {
    String name;
    Array temps[5];
  } data;
} ex12;
\end{vcode}
\T\medskip

Output:\\
\begin{ttfamily}
Dataset: $dataset name$\\
data.name, $data.name$\\
data.temps, $data.temps[0]$, \ldots, $data.temps[4]$\\
\vdots
\end{ttfamily}
\T\end{minipage}
\T\bigskip

\T\begin{minipage}{5in}
Example :
\T\medskip

\begin{vcode}{it}
Dataset {
  Sequence {
    String name;
    Array temps[5][5];
  } data;
}ex13;
\end{vcode}
\T\medskip

Output:\\
\begin{ttfamily}
Dataset: $dataset name$\\
data.name, $data.name_0$\footnote{The subscripts are here to reinforce that a
  sequence consists of $N$ values (or instances) and thus there are $N$
  values for both \texttt{name} and \texttt{temps}.}\\
data.temps[0], $data.temps[0][0]_0$, \ldots, $data.temps[0][4]_0$\\
\vdots\\
data.temps[4], $data.temps[4][0]_0$, \ldots, $data.temps[4][4]_0$\\
\vdots\\
data.name, $data.name_N$\\
data.temps[0], $data.temps[0][0]_N$, \ldots, $data.temps[0][4]_N$\\
\vdots\\
data.temps[4], $data.temps[4][0]_N$, \ldots, $data.temps[4][4]_N$\\
\end{ttfamily}
\T\end{minipage}
\T\bigskip

\item Array 
\begin{enumerate}
\item An one-dimensional array of simple types is printed as a name and comma
  separated list of values of the vector, all on a single line. See Example
  ??.
\item A multidimensional array of simple types of rank $N$ is represented by
  writing out, for each of its $N-1$ dimensions' indices, the name, the index
  and a row of values. See Examples ??--??.
\item Each element of an array of complex types is printed on a group of
  lines. For an array of complex types of rank $N$, each row will be prefixed
  by the array name and the index for that row. 
\end{enumerate}

\T\begin{minipage}{5in}
Example 1:
\T\medskip

\begin{vcode}{it}
Dataset {
    Int32 x[10];
} ex1;
\end{vcode}
\T\medskip

Output:\\
\begin{ttfamily}
Dataset: $dataset name$\\
x, $x[0]$, \ldots, $x[9]$
\end{ttfamily}
\T\end{minipage}
\T\bigskip

\T\begin{minipage}{5in}
Example 3:
\T\medskip

\begin{vcode}{it}
Dataset {
    Int32 x[20][10];
} ex3;
\end{vcode}
\T\medskip

Output:\\
\begin{ttfamily}
Dataset: $dataset name$\\
x[0], $x[0][0]$, \ldots, $x[0][9]$\\
\vdots\\
x[19], $x[19][0]$, \ldots, $x[19][9]$\\
\end{ttfamily}
\T\end{minipage}
\T\bigskip

\T\begin{minipage}{5in}
Example 4:
\T\medskip

\begin{vcode}{it}
Dataset {
    Int32 x[5][10][15];
} ex4;
\end{vcode}
\T\medskip

Output:\\
\begin{ttfamily}
Dataset: $dataset name$\\
x[0][0], $x[0][0][0]$, \ldots, $x[0][0][14]$\\
\vdots\\
x[0][9], $x[0][9][0]$, \ldots, $x[0][9][14]$\\
\vdots\\
x[4][0], $x[4][0][0]$, \ldots, $x[4][0][14]$\\
\vdots\\
x[4][9], $x[4][9][0]$, \ldots, $x[4][9][14]$\\
\end{ttfamily}
\T\end{minipage}
\T\bigskip

\T\begin{minipage}{5in}
Example 2:
\T\medskip

\begin{vcode}{it}
Dataset {
    Structure {
        Int32 x;
        String name;
    } pair[10];
} ex2;
\end{vcode}
\T\medskip

Output:\\
\begin{ttfamily}
Dataset: $dataset name$\\
pair[0], $pair[0].x$, $pair[0].name$\\
\vdots\\
pair[9], $pair[9].x$, $pair[9].name$
\end{ttfamily}
\T\end{minipage}
\T\bigskip

\T\begin{minipage}{5in}
Example 5:
\T\medskip

\begin{vcode}{it}
Dataset {
    Structure {
        Int32 x;
        String name;
    } pair[5][10];
} ex5;
\end{vcode}
\T\medskip

Output:\\
\begin{ttfamily}
Dataset: $dataset name$\\
pair[0][0], $pair[0][0].x$, $pair[0][0].name$\\
\vdots\\
pair[0][9], $pair[0][9].x$, $pair[0][9].name$\\
\vdots\\
pair[4][0], $pair[4][0].x$, $pair[4][0].name$\\
\vdots\\
pair[4][9], $pair[4][9].x$, $pair[4][9].name$\\
\end{ttfamily}
\T\end{minipage}
\T\bigskip

\end{enumerate}

\subsubsection{HTML}
\label{sec:html}
\emph{[Add text \ldots]}

The \texttt{HTML} response MUST present the user with an \ac{HTML} document
that can be used to access values from the dataset. The document MUST show
all of the variables in the dataset along with their attributes and it MUST
show the dataset's attributes. Structural information SHOULD be preserved. 

The intent of the \texttt{HTML} response it to provide an interface to the
dataset that can be used with a \ac{WWW} browser. Users should be able to
look at the variables in the dataset, read attributes and request values from
one \ac{WWW} location without making multiple requests. 

Because \ac{DAP} \ac{URL}s are somewhat confusing for many people, this
interface should provide a point-and-click way to build those \ac{URL}s.

\subsubsection{INFO}
\label{sec:info}
\emph{[Add text \ldots]}

The \texttt{info} response MUST present the user with an \ac{HTML} document
that contains the information in the \ac{DDS} and \ac{DAS}. This intent is to
present this information in a way that can be rendered by a \ac{WWW}
browser. Structural information about the dataset SHOULD be preserved so that
users can build URLs by hand.

The \texttt{info} response MAY also return other information. If server
installers and/or dataset maintainers add \ac{HTML} which describes the
dataset or the server, this information MUST be merged into the \ac{HTML}
document returned in response to the \texttt{info} request.

The purpose of the \texttt{info} response is to supply information about the
dataset in a form that is easy for people to read. It should be structured so
that after only a little experience people can easily assess a dataset using
the document. In addition, the \texttt{info} response may be used in creating
various user interfaces which access data using \ac{DAP} servers.

The usage server should provide:
\begin{enumerate}
\item The hierarchical relation of container variables.
\item Each variable's datatype.
\item Each variable's attributes.
\item Any global attributes that the dataset contains.
\item Extra information supplied by the dataset creator/maintainer.
\end{enumerate}

Mechanism:
\begin{enumerate}
\item Access to the HTML version of the usage information will be via a URL
  with the suffix \texttt{.info}.
  
\item Dataset creators/maintainers will be able to supply a set
  of pseudo-HTML documents that will be concatenated with the information
  derived from the DAS and DDS objects. 

\begin{vcode}{i}
Here in a vcode temporarily...
Per-dataset HTML*:


    <h3>
    About the Fleet Numerical Ocean Winds dataset
    </h3>

    The FNOC dataset is used as a sample dataset for testing of DODS. It is a
    `real' scientific dataset, but I really don't understand what it means.

    <p>

NB: This file must be in the same directory as the dataset. Note that the
same <basename><digits>.html scheme used for the override file is used
here.

Per-server HTML*:

    <h3>
    Server Functions:
    </h3>
    <dl>
    <dt>geolocate(variable, lat1, lat2, lon1, lon2)
    <dd>Returns the elements of <em>variable</em> that fall within the box
    created by <em>lat1</em>, <em>lon1</em> and <em>lat2</em>, <em>lon2</em>.

    <p>

    <dt>time(variable, start\_time, stop\_time)
    <dd>Returns the elements of <em>variable</em> that fall within the time
    interval <em>start\_time</em> and <em>stop\_time</em>.
    </dl>

    <p>

NB: This file must reside with the server. For the netcdf server (which is
made up of nph-nc, nc\_das, nc\_dds adn nc\_dods) this file is called nc.html).
\end{vcode}

\item There will be an override mode where only the user supplied HTML is
  returned. Thus if the automatically generated part of the server fails for
  a particular dataset the creator/maintainer can work around that problem.

  If an override file is present, the \texttt{info} response MUST return that
  file INSTEAD of the automatically generated \ac{HTML} document. The
  contents of the override file MUST be well formed \ac{HTML}.

  An override file is a file who's extension is \texttt{.ovr} and who's
  basename\footnote{Define basename as the part of the filename before the
  last dot (\texttt{.}).} matches that of the dataset of which the
  \texttt{info} request was made. This requirement is amended in one way; if
  a dataset's basename ends in a number, an override document need not
  include that number in its basename. In this case, the override document
  SHALL be applicable to all of the datasets which have a common basename not
  including the number if and only if an override file with a name which
  matches exactly is not present.

  Some example dataset names and override files:
  \begin{itemize}
    \item [temp.ovr] Matches: temp.dat.
    \item [fnoc.ovr] Matches: fnoc1.nc, fnoc2.nc, \ldots, fnoc10.nc. It does
      not match fnocx.nc.
    \item [fnoc1.ovr] Matches: fnoc1.nc and fnoc10.nc but not fnoc2.nc. But
      if a fnoc10.ovr file is present that must take precedent.
  \end{itemize}

\end{enumerate}

\emph{[Insert information from usage-server.txt]}

\subsubsection{Help}
\label{sec:help}
The \texttt{help} response MUST be returned when either the server receives a
\ac{URL} with no extension (i.e., a \ac{URL} which asks for no object) or
when the \texttt{dataset\_id} portion of the \ac{URL} is \texttt{help}.

\begin{ttfamily}
\begin{center}
\begin{tabular}{lll}
abs\_path & = & server\_path dataset\_id "." ext [ "?" query ] \\
server\_path & = & <name of DAP server> \\
dataset\_id & = & "help" \\
\end{tabular}
\end{center}
\end{ttfamily}

The second way of requesting the \texttt{help} response is analogous to one
of the ways of requesting the \texttt{version} response.

The \texttt{help} response MUST return an \ac{HTML} document which lists
the extensions recognized by the server. The response MAY return other
information as well.

\subsubsection{Directory}
\label{sec:dir}
\emph{[Needs significant design work \ldots]}

The Directory response is an \ac{HTML} document which provide a simple way
for users to navigate among different datasets that may be accessed with the
server. The actual presentation mechanism is up to the server implementor.
However, when datasets are represented in the response there must be a
\ac{URL} which references the \texttt{HTML} response for that dataset.

\printgloss{dods-glossary}

\T\addcontentsline{toc}{section}{References}
\T\raggedright
\bibliography{dods}

\T\appendix

\section{Notational Conventions and Generic Grammar}
\label{app:grammar}

% This was taken verbatim from rfc2616. The original section title was
% `Notational Conventions and Generic Grammar' 3/21/2001 jhrg

\subsection{Augmented BNF}
All of the mechanisms specified in this document are described in both prose
and an augmented Backus-Naur Form (BNF) similar to that used by RFC
822~\cite{rfc822}. Implementors will need to be familiar with the notation in
order to understand this specification. The augmented BNF includes the
following constructs:

\begin{description}
  
\item [\texttt{name = definition}] The name of a rule is simply the name
  itself (without any enclosing \texttt{"$<$"} and \texttt{"$>$"}) and is
  separated from its definition by the equal \texttt{"="} character. White
  space is only significant in that indentation of continuation lines is used
  to indicate a rule definition that spans more than one line. Certain basic
  rules are in uppercase, such as SP, LWS, HT, CRLF, DIGIT, ALPHA, etc. Angle
  brackets are used within definitions whenever their presence will
  facilitate discerning the use of rule names.
  
\item [\texttt{"literal"}] Quotation marks surround literal text. Unless
  stated otherwise, the text is case-insensitive.
      
\item [\texttt{rule1 | rule2}] Elements separated by a bar (\texttt{"|"}) are
  alternatives, e.g., \texttt{"yes | no"} will accept \texttt{yes} or
  \texttt{no}.
  
\item [\texttt{(rule1 rule2)}] Elements enclosed in parentheses are treated
  as a single element.  Thus, \texttt{"(elem (foo | bar) elem)"} allows the
  token sequences \texttt{"elem foo elem"} and \texttt{"elem bar elem"}.
  
\item [\texttt{*rule}] The character \texttt{"*"} preceding an element
  indicates repetition. The full form is \texttt{"$<$n$>$*$<$m$>$element"}
  indicating at least \texttt{$<$n$>$} and at most \texttt{$<$m$>$}
  occurrences of element. Default values are 0 and infinity so that
  \texttt{"*(element)"} allows any number, including zero;
  \texttt{"1*element"} requires at least one; and \texttt{"1*2element"}
  allows one or two.
  
\item [\texttt{[rule]}] Square brackets enclose optional elements;
  \texttt{"[foo bar]"} is equivalent to \texttt{"*1(foo bar)"}.
  
\item [\texttt{N rule}] Specific repetition: \texttt{"$<$n$>$(element)"} is
  equivalent to \texttt{"$<$n>*$<$n$>$(element)"}; that is, exactly
  \texttt{$<$n$>$} occurrences of (element).  Thus 2DIGIT is a 2-digit
  number, and 3ALPHA is a string of three alphabetic characters.
  
\item [\texttt{\#rule}] A construct \texttt{"\#"} is defined, similar to
  \texttt{"*"}, for defining lists of elements. The full form is
  \texttt{"$<$n$>$\#$<$m$>$element"} indicating at least \texttt{$<$n$>$} and
  at most \texttt{$<$m$>$} elements, each separated by one or more commas
  (\texttt{","}) and OPTIONAL linear white space (LWS). This makes the usual
  form of lists very easy; a rule such as \texttt{( *LWS element *( *LWS ","
    *LWS element ))} can be shown as \texttt{1\#element} Wherever this
  construct is used, null elements are allowed, but do not contribute to the
  count of elements present. That is, \texttt{"(element), , (element) "} is
  permitted, but counts as only two elements. Therefore, where at least one
  element is required, at least one non-null element MUST be present. Default
  values are 0 and infinity so that \texttt{"\#element"} allows any number,
  including zero; \texttt{"1\#element"} requires at least one; and
  \texttt{"1\#2element"} allows one or two.
  
\item [\texttt{;} comment] A semi-colon, set off some distance to the right
  of rule text, starts a comment that continues to the end of line. This is a
  simple way of including useful notes in parallel with the specifications.
  
\item [implied \texttt{*LWS}] The grammar described by this specification is
  word-based. Except where noted otherwise, linear white space (LWS) can be
  included between any two adjacent words (token or quoted-string), and
  between adjacent words and separators, without changing the interpretation
  of a field. At least one delimiter (LWS and/or separators) MUST exist
  between any two tokens (for the definition of "token" below), since they
  would otherwise be interpreted as a single token.
\end{description}

\subsection{Basic Rules}

   The following rules are used throughout this specification to
   describe basic parsing constructs. The US-ASCII coded character set
   is defined by ANSI X3.4-1986~\cite{ANSI:US-ASCII}.

\begin{vcode}{it}
       OCTET          = <any 8-bit sequence of data>
       CHAR           = <any US-ASCII character (octets 0 - 127)>
       UPALPHA        = <any US-ASCII uppercase letter "A".."Z">
       LOALPHA        = <any US-ASCII lowercase letter "a".."z">
       ALPHA          = UPALPHA | LOALPHA
       DIGIT          = <any US-ASCII digit "0".."9">
       CTL            = <any US-ASCII control character
                        (octets 0 - 31) and DEL (127)>
       CR             = <US-ASCII CR, carriage return (13)>
       LF             = <US-ASCII LF, linefeed (10)>
       SP             = <US-ASCII SP, space (32)>
       HT             = <US-ASCII HT, horizontal-tab (9)>
       <">            = <US-ASCII double-quote mark (34)>
\end{vcode}

HTTP/1.1 defines the sequence CR LF as the end-of-line marker for all
protocol elements except the entity-body (see Appendix 19.3 of RFC
2616[9] for tolerant applications). The end-of-line marker within an
entity-body is defined by its associated media type, as described in
Section 3.7 of RFC 2616[9].

% My original text:
% except the entity-body (see appendix
%    19.3\footnote{In RFC 2616\cite{rfc2616}.} for
%    tolerant applications). The end-of-line marker within an entity-body
%    is defined by its associated media type, as described in section
%    3.7.\footnote{In RFC 2616\cite{rfc2616}.}
% The replacement text above was suggested by Allan Doyle, 20 April
% 2005.

\begin{vcode}{it}
       CRLF           = CR LF
\end{vcode}

   HTTP/1.1 header field values can be folded onto multiple lines if the
   continuation line begins with a space or horizontal tab. All linear
   white space, including folding, has the same semantics as SP. A
   recipient MAY replace any linear white space with a single SP before
   interpreting the field value or forwarding the message downstream.

\begin{vcode}{it}
       LWS            = [CRLF] 1*( SP | HT )
\end{vcode}

   The TEXT rule is only used for descriptive field contents and values
   that are not intended to be interpreted by the message parser. Words
   of *TEXT MAY contain characters from character sets other than ISO-
   8859-1 [22] only when encoded according to the rules of RFC 2047
   [14].

\begin{vcode}{it}
       TEXT           = <any OCTET except CTLs,
                        but including LWS>
\end{vcode}

   A CRLF is allowed in the definition of TEXT only as part of a header
   field continuation. It is expected that the folding LWS will be
   replaced with a single SP before interpretation of the TEXT value.

   Hexadecimal numeric characters are used in several protocol elements.

\begin{vcode}{it}
       HEX            = "A" | "B" | "C" | "D" | "E" | "F"
                      | "a" | "b" | "c" | "d" | "e" | "f" | DIGIT
\end{vcode}

   Many HTTP/1.1 header field values consist of words separated by LWS
   or special characters. These special characters MUST be in a quoted
   string to be used within a parameter value (as defined in section
   3.6).

\begin{vcode}{it}
       token          = 1*<any CHAR except CTLs or separators>
       separators     = "(" | ")" | "<" | ">" | "@"
                      | "," | ";" | ":" | "\" | <">
                      | "/" | "[" | "]" | "?" | "="
                      | "{" | "}" | SP | HT
\end{vcode}

   Comments can be included in some HTTP header fields by surrounding
   the comment text with parentheses. Comments are only allowed in
   fields containing "comment" as part of their field value definition.
   In all other fields, parentheses are considered part of the field
   value.

\begin{vcode}{it}
       comment        = "(" *( ctext | quoted-pair | comment ) ")"
       ctext          = <any TEXT excluding "(" and ")">
\end{vcode}

   A string of text is parsed as a single word if it is quoted using
   double-quote marks.

\begin{vcode}{it}
       quoted-string  = ( <"> *(qdtext | quoted-pair ) <"> )
       qdtext         = <any TEXT except <">>
\end{vcode}

   The backslash character ("\verb+\+") MAY be used as a single-character
   quoting mechanism only within quoted-string and comment constructs.

\begin{vcode}{it}
       quoted-pair    = "\" CHAR
\end{vcode}

\begin{quote}
This appendix was copied from RFC 2616~\cite{rfc2616}. The copyright
from that document reads:

\begin{quote}
  Copyright (C) The Internet Society (1999).  All Rights Reserved.

   This document and translations of it may be copied and furnished to
   others, and derivative works that comment on or otherwise explain it
   or assist in its implementation may be prepared, copied, published
   and distributed, in whole or in part, without restriction of any
   kind, provided that the above copyright notice and this paragraph are
   included on all such copies and derivative works.  However, this
   document itself may not be modified in any way, such as by removing
   the copyright notice or references to the Internet Society or other
   Internet organizations, except as needed for the purpose of
   developing Internet standards in which case the procedures for
   copyrights defined in the Internet Standards process must be
   followed, or as required to translate it into languages other than
   English.
\end{quote}

\end{quote}


\section{Acronyms and Abbreviations}
\begin{acronym}
%
% These are html links which are used often enough in writing about DODS to
% merit an input file.
% jhrg. 4/17/94
%
% File rationalized and updated while writing the DODS User
% Guide. Also includes other useful abbreviations.
% tomfool 3/15/96
%
% Moved to dods-def.tex so I can remove links to documents that no
% longer reflect reality.
% tomfool 2/13/98
%
% $Id$
%
% Make sure to include layout.tex *before* using this file.

%% NOTE NOTE NOTE NOTE NOTE NOTE NOTE NOTE NOTE NOTE NOTE NOTE NOTE NOTE 
%%
%% If this file causes problems when running latex, you may have to edit your
%% texmf.cnf file. Here's a meesage from Tom:
%% > Are these references that use relative addresses (like
%% > ../boiler/blah.tex)?  If they are, you should look for the texmf.cnf
%% > file.  (It's often at /usr/share/texmf/web2c/texmf.cnf, and look for
%% > the openout_any parameter.  Check there anyway; there were some recent
%% > (i.e. in the early '90's) security fixes to TeX.
%%
%%%%%%%%%%%%%%%%%%%%%%%%%%%%%%%%%%%%%%%%%%%%%%%%%%%%%%%%%%%%%%%%%%%%%%%%%%%

%%% These are some DODS-specific convenience commands.
\newcommand{\DODSroot}{\lit{\$DODS\_ROOT}}     
% $

\newcommand{\opendap}{OPeNDAP\xspace}

%%% The OPD books and reference material
\newcommand{\OPDDoc}{http://opendap.org/support/docs.html}
\newcommand{\DODSDoc}{http://opendap.org/support/docs.html}

% \newcommand{\OPDDoc}{http://www.unidata.ucar.edu/packages/dods}
% \newcommand{\DODSDoc}{http://www.unidata.ucar.edu/packages/dods}

\newcommand{\OPDhomeUrl}%
  {http://opendap.org}
\newcommand{\OPDexampleUrl}%
  {BROKEN--FIX ME!}%\OPDDoc/examples}
% \newcommand{\OPDftpUrl}%
%  {ftp://dods.gso.uri.edu/pub/dods}
\newcommand{\OPDftpUrl}%
  {ftp://ftp.unidata.ucar.edu/pub/opendap/}
\newcommand{\OPDuserUrl}%
  {\OPDhomeUrl/user/guide-html/}
\newcommand{\OPDmguiUrl}%
  {\l/user/mgui-html/}
\newcommand{\OPDapiUrl}%
  {\OPDhomeUrl/api/pguide-html/}
\newcommand{\OPDapirefUrl}%
  {\OPDhomeUrl/api/pref/html/}
\newcommand{\OPDffUrl}%
  {\OPDhomeUrl/user/servers/dff-html/}
\newcommand{\OPDquickUrl}%
  {\OPDhomeUrl/user/quick-html/}
\newcommand{\OPDinstallUrl}%
  {\OPDhomeUrl/server/install-html}%
\newcommand{\OPDregexUrl}%
  {\OPDhomeUrl/user/regex-html}%
\newcommand{\OPDjavaUrl}%
  {\OPDhomeUrl/home/swJava1.1/}
\newcommand{\OPDwclientUrl}%
  {\OPDhomeUrl/api/wc-html/}
\newcommand{\OPDwserverUrl}%
  {\OPDhomeUrl/api/ws-html/}
\newcommand{\OPDaggUrl}%
  {\OPDhomeUrl/server/agg-html/}

\newcommand{\OPDhome}{\xlink{OPeNDAP Home page}{\OPDhomeUrl}}
\newcommand{\OPDjava}{\xlink{OPeNDAP Java home page}{\OPDjavaUrl}}
\newcommand{\OPDexamples}{\xlink{OPeNDAP examples page}{\OPDexampleUrl}}
\newcommand{\OPDftp}{\xlink{OPeNDAP ftp site}{\OPDftpUrl}}
%% Book titles do *not* contain the article.
\newcommand{\OPDuser}[1][]{\xlink%
  {\cit{OPeNDAP User Guide}}{\OPDuserUrl{}#1}}
\newcommand{\OPDmgui}{\xlink%
  {\cit{OPeNDAP Matlab GUI}}{\OPDmguiUrl}}
\newcommand{\OPDapi}{\xlink%
  {\cit{OPeNDAP Toolkit Programmer's Guide}}{\OPDapiUrl}}
\newcommand{\OPDapiref}{\xlink%
  {\cit{OPeNDAP Toolkit Reference}}{\OPDapirefUrl}}
\newcommand{\OPDffbook}{\xlink%
  {\cit{OPeNDAP Freeform ND Server Manual}}{\OPDffUrl}}
\newcommand{\OPDquick}{\xlink%
  {\cit{OPeNDAP Quick Start Guide}}{\OPDquickUrl}}
\newcommand{\OPDinstall}{\xlink%
  {\cit{OPeNDAP Server Installation Guide}}{\OPDinstallUrl}}
\newcommand{\OPDregex}{\xlink%
  {\cit{Introduction to Regular Expressions}}{\OPDregexUrl}}
\newcommand{\OPDagg}{\xlink%
  {\cit{OPeNDAP Aggregation Server Guide}}{\OPDaggUrl}}
\newcommand{\OPDwclient}{\xlink%
  {\cit{Writing an OPeNDAP Client}}{\OPDwclientUrl}}
\newcommand{\OPDwserver}{\xlink%
  {\cit{Writing an OPeNDAP Server}}{\OPDwclientUrl}}

\newcommand{\OPDffs}{OPeNDAP Freeform ND Server}

%%% Other DODS links.
\newcommand{\homepage}% For hyperlatex
  {\OPDDoc/}
\newcommand{\OPDsupport}{\xlink{support@unidata.ucar.edu}{mailto:support@unidata.ucar.edu}}
\newcommand{\DODSsupport}{\xlink{support@unidata.ucar.edu}{mailto:support@unidata.ucar.edu}}
\newcommand{\DODS}{\xlink{Distributed Oceanographic Data System}{\OPDhomeUrl}}
\newcommand{\OPD}{\xlink{Open Source Project for a Data Access Protocol}{\OPDhomeUrl}}
\newcommand{\OPDtechList}{\xlink{OPeNDAP Mailing Lists}{\OPDhomeUrl/mailLists/}}

%%% DODS versions
%% This has been removed.  Documents should not have an automatic
%% version number, because then it appears as if they have been
%% updated when they haven't.  Put the relevant version number to
%% whatever software is being described into each document's preface. 

\newcommand{\ifh}{WWW Interface}

% external refs for DODS documents

\newcommand{\CGI}{\xlink{Common Gateway Interface}
  {http://hoohoo.ncsa.uiuc.edu/cgi/overview.html}}

\newcommand{\MIME}{\xlink{Multipurpose Internet Mail Extensions}
  {http://www.cis.ohio-state.edu/htbin/rfc/rfc1590.html}}

\newcommand{\netcdf}{\xlink{NetCDF}
  {http://www.unidata.ucar.edu/packages/netcdf/guide.txn_toc.html}}

\newcommand{\JGOFS}{\xlink{Joint Geophysical Ocean Flux Study}
  {http://www1.whoi.edu/jgofs.html}}

\newcommand{\jgofs}{\xlink{JGOFS}
  {http://www1.whoi.edu/jgofs.html}}

\newcommand{\hdf}{\xlink{HDF}
  {http://www.ncsa.uiuc.edu/SDG/Software/HDF/HDFIntro.html}}

\newcommand{\ffnd}{FreeForm ND}

\newcommand{\Cpp}{\texorhtml
  {{\rm {\small C}\raise.5ex\hbox{\footnotesize ++}}}
  {C\htmlsym{##43}\htmlsym{##43}}}

% Commands

% Use pdflink instead. jhrg 8/4/2006
% \newcommand{\pslink}[1]{\small
% \begin{quote}
%   A \xlink{PDF version}{#1} of this document is available.
% \end{quote}
% \normalsize
% }

% Use the copy of this in dods-paper.hlx/cls or cut and paste this on
% a document-by-document basis. This version conflicts with the
% version in the class. jhrg 8/4/2006
% \newcommand{\pdflink}[1]{\small
% \begin{quote}
%   A \xlink{PDF version}{#1} of this document is available.
% \end{quote}
% \normalsize
% }

%%%%%%%%%%%%%% DODS macros
%
% These are here so that older latex files will compile. Someday remove these
% and fix the files. 04/13/04 jhrg

\newcommand{\DODShomeUrl}%
  {\OPDhomeUrl}
\newcommand{\DODSexampleUrl}%
  {\OPDDoc/examples}
% \newcommand{\OPDftpUrl}%
%  {ftp://dods.gso.uri.edu/pub/dods}
\newcommand{\DODSftpUrl}%
  {ftp://ftp.unidata.ucar.edu/pub/opendap/}
\newcommand{\DODSuserUrl}%
  {\OPDhomeUrl/user/guide-html/}
\newcommand{\DODSmguiUrl}%
  {\OPDhomeUrl/user/mgui-html/}
\newcommand{\DODSapiUrl}%
  {\OPDhomeUrl/api/pguide-html/}
\newcommand{\DODSapirefUrl}%
  {\OPDhomeUrl/api/pref/html/}
\newcommand{\DODSffUrl}%
  {\OPDhomeUrl/user/servers/dff-html/}
\newcommand{\DODSquickUrl}%
  {\OPDhomeUrl/user/quick-html/}
\newcommand{\DODSinstallUrl}%
  {\OPDhomeUrl/server/install-html}%
\newcommand{\DODSregexUrl}%
  {\OPDhomeUrl/user/regex-html}%
\newcommand{\DODSjavaUrl}%
  {\OPDhomeUrl/home/swJava1.1/}
\newcommand{\DODSwclientUrl}%
  {\OPDhomeUrl/api/wc-html/}
\newcommand{\DODSwserverUrl}%
  {\OPDhomeUrl/api/ws-html/}
\newcommand{\DODSaggUrl}%
  {\OPDhomeUrl/server/agg-html/}

\newcommand{\DODShome}{\xlink{OPeNDAP Home page}{\DODShomeUrl}}
\newcommand{\DODSjava}{\xlink{OPeNDAP Java home page}{\DODSjavaUrl}}
\newcommand{\DODSexamples}{\xlink{OPeNDAP examples page}{\DODSexampleUrl}}
\newcommand{\DODSftp}{\xlink{OPeNDAP ftp site}{\DODSftpUrl}}
\newcommand{\DODSuser}[1][]{\xlink%
  {\cit{The OPeNDAP User Guide}}{\DODSuserUrl{}#1}}
\newcommand{\DODSmgui}{\xlink%
  {\cit{The OPeNDAP Matlab GUI}}{\DODSmguiUrl}}
\newcommand{\DODSapi}{\xlink%
  {\cit{The DODS Toolkit Programmer's Guide}}{\DODSapiUrl}}
\newcommand{\DODSapiref}{\xlink%
  {\cit{The DODS Toolkit Reference}}{\DODSapirefUrl}}
\newcommand{\DODSffbook}{\xlink%
  {\cit{The DODS Freeform ND Server Manual}}{\DODSffUrl}}
\newcommand{\DODSquick}{\xlink%
  {\cit{The DODS Quick Start Guide}}{\DODSquickUrl}}
\newcommand{\DODSinstall}{\xlink%
  {\cit{The DODS Server Installation Guide}}{\DODSinstallUrl}}
\newcommand{\DODSregex}{\xlink%
  {\cit{Introduction to Regular Expressions}}{\DODSregexUrl}}
\newcommand{\DODSagg}{\xlink%
  {\cit{OPeNDAP Aggregation Server Guide}}{\DODSaggUrl}}
\newcommand{\DODSwclient}{\xlink%
  {\cit{Writing an OPeNDAP Client}}{\DODSwclientUrl}}
\newcommand{\DODSwserver}{\xlink%
  {\cit{Writing an OPeNDAP Server}}{\DODSwclientUrl}}

\newcommand{\DODSffs}{DODS Freeform ND Server}

% $Log: dods-def.tex,v $
% Revision 1.24  2004/12/21 22:30:04  jimg
% Fixed pslink; Added pdflink.
%
% Revision 1.23  2004/12/14 05:19:17  tomfool
% restored fix to pslink
%
% Revision 1.22  2004/12/09 21:01:58  tomfool
% excised test.dods.org
%
% Revision 1.21  2004/12/09 18:50:21  tomfool
% de-dodsifying
%
% Revision 1.15  2004/04/24 21:37:22  jimg
% I added every directory in preparation for adding everyting. This is
% part of getting the opendap web pages going...
%
% Revision 1.14  2004/02/12 16:05:50  jimg
% Moved the log to the end of the file.
%
% Revision 1.13  2004/01/16 18:05:31  jimg
% Added a note from Tom about setting texmf.cnf to allow \include to process
% files with ../ in their pathnames. You can also change the include to input,
% but I think include may offer some advantages for bigger/complex things like
% the Guides.
%
% Revision 1.12  2003/12/28 21:48:22  tom
% added newer books
%
% Revision 1.11  2003/12/08 19:04:43  tom
% little adjustments for DODS->opendap
%
% Revision 1.10  2003/12/08 18:53:30  tom
% DODS->OPeNDAP
%
% Revision 1.9  2002/07/15 17:49:55  tom
% added \DODSDoc
%
% Revision 1.8  2001/05/04 15:07:45  tom
% fixed pslink to include pdf files
%
% Revision 1.7  2001/02/19 20:39:13  tom
% added links to the new regex intro.
%
% Revision 1.6  2000/03/23 18:27:52  tom
% added abbreviations
%
% Revision 1.5  1999/07/01 16:00:19  tom
% added a couple of web page references
%
% Revision 1.4  1999/05/25 20:49:34  tom
% changed version numbers to 3.0
%
% Revision 1.3  1999/02/04 17:27:08  tom
% adjusted for hyperlatex and dods-book.cls
%
%

%%% Local Variables: 
%%% mode: latex
%%% TeX-master: t
%%% TeX-master: t
%%% End: 

%
% Make one entry per line, even if they are long lines that wrap and look
% ugly. This makes it simple to sort the list using emacs' sort-lines
% command. 3/27/2000 jhrg
%
% $Id$
\acro{AS}{Aggregation Server}
\acro{BNF} {Backus-Naur Form}
\acro{CE}{Constraint Expression}
\acro{CGI}{Common Gateway Interface}
\acro{COARDS}{Cooperative Ocean/Atmosphere Research Data Service}
\acro{CSV}{Comma Separated Values}
\acro{DAP}{Data Access Protocol}
\acro{DAS}{Dataset Attribute Structure}
\acro{DDS}{Dataset Descriptor Structure}
\acro{DODS}{Distributed Oceanographic Data System}, See the DODS home page: \texttt{http://\-unidata.ucar.edu\-/packages\-/dods/}
\acro{DataDDS}{Data Dataset Descriptor Structure}
\acro{FGDC}{Federal Geographic Data Community}
\acro{HTML}{Hypertext Markup Language}
\acro{HTTP}{HyperText Transfer Protocol}
\acro{MIME}{Multimedia Internet Mail Extensions}
\acro{SRS}{Software Requirements Specification}, See IEEE 830--1998
\acro{URI}{Uniform Resource Identifiers}
\acro{URL}{Uniform Resource Locator}
\acro{W3C}{The World Wide Web Consortium}, See http://www.w3c.org/
\acro{WWW}{The World Wide Web}
\acro{XDR}{External Data Representation}
\acro{XML}{Extensible Markup Language}
%%% Local Variables: 
%%% mode: latex
%%% TeX-master: t
%%% End: 

\end{acronym}

\section{Change log}

\begin{verbatim}
$Log: dap_rfc.tex,v $
Revision 1.22  2004/04/24 21:37:22  jimg
I added every directory in preparation for adding everyting. This is
part of getting the opendap web pages going...

Revision 1.21  2003/05/23 19:27:02  tom
moved html output so not to interfere with the newer DAP spec docs

Revision 1.20  2003/01/14 19:50:33  jimg
Fixed text about Arrays and Grids.

Revision 1.19  2002/06/12 00:36:12  jimg
Added text about Array/Grid constrains. It's an error for either to be
partially constrained.

Revision 1.18  2001/09/18 18:28:50  jimg
Fixed some formatting problems with the ASCII output examples.
Changes a few words in that section, too.

Revision 1.17  2001/09/12 21:55:53  jimg
I've added a considerable amount of text describing how ASCII output should
be generated. Once I've implemented this I'll make changes and ask for
group-wide comments.

Revision 1.16  2001/07/19 21:47:09  jimg
Added text about the info service.

Revision 1.15  2001/05/17 00:09:01  jimg
Merged the revision at the tail end of the hlx-version branch back to the
text on the trunk. I think I did not undo stuff, but there are still three
conflicts I don't know how to resolve. So this revision won't make it through
latex.

Revision 1.14  2001/05/16 23:55:29  jimg
Added just a little support for hyperlatex

Revision 1.11.2.4  2001/05/16 15:02:43  tom
aligned with hyperlatex needs

Revision 1.13  2001/04/13 21:01:51  jimg
Ok. After all that. This version is the result of my backing out the hlx
changes from the latex on the trunk.

Revision 1.11  2001/03/26 19:53:26  jimg
Added a second round of changes suggested by Paul.

Revision 1.10  2001/03/23 01:53:48  jimg
Read over all but 4.2.4 ro 4.2.10. Added more examples, esp. in the section
on URL syntax.

Revision 1.9  2001/03/22 08:55:41  jimg
Added examples; most of Paul's and Tom's comments.

Revision 1.8  2001/03/20 23:46:57  jimg
Added examples.

Revision 1.7  2001/03/13 18:09:43  jimg
Started on Tom's second round of changes...

Revision 1.6  2001/03/12 03:13:55  tom
incorporated some of Paul's typo-finds.

Revision 1.5  2001/03/12 02:56:37  tom
fixed typos, rearranged a tiny bit

Revision 1.4  2001/03/08 23:05:19  jimg
Made various structural changes. Moved text from the Characterization...
section into the Responses section. Divided the paper in to three parts:
one that describes the characterization of a dataset as a collection of
variables, one that describes the requests and one that describes the
responses. There are other changes in the text.

Revision 1.3  2001/03/08 06:36:31  jimg
Made the first few changes suggested by Tom. Checkpoint here before making
major structural changes.

Revision 1.2  2001/03/06 00:49:09  jimg
Partially complete; Response section needs text.

Revision 1.1  2001/02/27 18:56:26  jimg
Added
\end{verbatim}

\end{document}
