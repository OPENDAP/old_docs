% latex and HTML macros. Some latex commands become nops for HTML. 4/10/2001
% jhrg 
\T\newcommand{\Cpp}{\rm {\small C}\raise.5ex\hbox{\footnotesize++}\xspace}
\T\newcommand{\C}{\rm {\small C}\xspace}
\W\newcommand{\Cpp}{C++}
\W\newcommand{\C}{C\xspace}
\W\newcommand{\cdots}{...}
\W\newcommand{\ddots}{}
\W\newcommand{\vdots}{.}
\W\newcommand{\pm}{+/-}
\W\newcommand{\times}{*}
\W\newcommand{\uppercase}[1]{\textsc{#1}}
\T\newcommand{\qt}{\lit{\char127}}
\W\newcommand{\qt}{"}

\texorhtml{\def\rearrangedate#1/#2/#3#4{\ifcase#2\or January\or February\or
  March\or April\or May\or June\or July\or August\or September\or
  October\or November\or December\fi\ \ifx0#3\relax\else#3\fi#4, #1}
\def\rcsdocumentdate{\expandafter\rearrangedate\rcsInfoDate}}%
{\HlxEval{
(put 'rearrangedate       'hyperlatex 'hyperlatex-ts-rearrange-date)

(defun hyperlatex-ts-rearrange-date ()
  (let ((date-string (hyperlatex-evaluate-string 
                       (hyperlatex-parse-required-argument))))
    (let ((year-string (substring date-string 0 4))
          (month-string (substring date-string 5 7))
          (day-string (substring date-string 8 10))
          (month-list '("January" "February" "March" "April"
                        "May" "June" "July" "August" 
                        "September" "October" "November" "December")))
       (insert
         (concat (elt month-list (1- (string-to-number month-string)))
                 " " (int-to-string (string-to-number day-string))
                 ", " year-string)))))
}
\newcommand{\rcsdocumentdate}{\rearrangedate{\rcsInfoDate}}}

\newcommand{\dapversion}{Version 4.0\xspace}
\newcommand{\opendap}{OPeNDAP\xspace}
\newcommand{\DAP}{DAP\xspace}
\newcommand{\DODS}{DODS\xspace}
\newcommand{\NVODS}{NVODS\xspace}
\newcommand{\CE}{constraint expression\xspace}
\newcommand{\CEs}{constraint expressions\xspace}
\newcommand{\ErrorX}{ErrorX\xspace}
\newcommand{\CapX}{Server Capabilities Document\xspace}
\newcommand{\Blob}{Blob\xspace}
\newcommand{\DDX}{DDX\xspace}
\newcommand{\DAX}{DAX\xspace}
\newcommand{\DDS}{DDS\xspace}
\newcommand{\DAS}{DAS\xspace}
\newcommand{\URI}{URI\xspace}
\newcommand{\DataDDS}{DataDDS\xspace}

\newcommand{\type}[1]{\emph{#1}}
\newcommand{\Alias}{\type{Alias}\xspace}
\newcommand{\Array}{\type{Array}\xspace}
\newcommand{\Attribute}{\type{Attribute}\xspace}
\newcommand{\ProcAttribute}{\type{Processing Attribute}\xspace}
\newcommand{\Map}{\type{Map}\xspace}
\newcommand{\Target}{\type{Target}\xspace}
\newcommand{\FQN}{fully qualified name\xspace}
\newcommand{\Grid}{\type{Grid}\xspace}
\newcommand{\Structure}{\type{Structure}\xspace}
\newcommand{\Dataset}{\type{Dataset}\xspace}
\newcommand{\Sequence}{\type{Sequence}\xspace}
\newcommand{\Container}{\type{Container}\xspace}
\newcommand{\Bi}{\type{Binary Image}\xspace}
\newcommand{\String}{\type{String}\xspace}
\newcommand{\URL}{\type{URL}\xspace}
\newcommand{\Boolean}{\type{Boolean}\xspace}
\newcommand{\Byte}{\type{Byte}\xspace}
\newcommand{\Enum}{\type{Enumeration}\xspace}
\newcommand{\Time}{\type{Time}\xspace}
\newcommand{\Function}{\type{Function}\xspace}
\newcommand{\Description}{\type{Description}\xspace}
\newcommand{\Parameter}{\type{Parameter}\xspace}
\newcommand{\Constraint}{\type{Constraint}\xspace}
\newcommand{\NoAttributes}{\type{NoAttributes}\xspace}
\newcommand{\Project}{\type{Project}\xspace}
\newcommand{\Select}{\type{Select}\xspace}
\newcommand{\Hyperslab}{\type{Hyperslab}\xspace}

\newcommand{\Aliases}{\type{Aliases}\xspace}
\newcommand{\Arrays}{\type{Arrays}\xspace}
\newcommand{\Attributes}{\type{Attributes}\xspace}
\newcommand{\ProcAttributes}{\type{Processing Attributes}\xspace}
\newcommand{\Maps}{\type{Maps}\xspace}
\newcommand{\Targets}{\type{Targets}\xspace}
\newcommand{\FQNs}{fully qualified names\xspace}
\newcommand{\Grids}{\type{Grids}\xspace}
\newcommand{\Structures}{\type{Structures}\xspace}
\newcommand{\Sequences}{\type{Sequences}\xspace}
\newcommand{\Containers}{\type{Containers}\xspace}
\newcommand{\Bis}{\type{Binary Images}\xspace}
\newcommand{\Strings}{\type{Strings}\xspace}
\newcommand{\URLs}{\type{URLs}\xspace}
\newcommand{\Booleans}{\type{Booleans}\xspace}
\newcommand{\Bytes}{\type{Bytes}\xspace}
\newcommand{\Enums}{\type{Enumerations}\xspace}
\newcommand{\Times}{\type{Times}\xspace}
\newcommand{\CSVs}{comma-separated values\xspace}
\newcommand{\Functions}{\type{Functions}\xspace}
\newcommand{\Descriptions}{\type{Descriptions}\xspace}
\newcommand{\Parameters}{\type{Parameters}\xspace}
\newcommand{\Constraints}{\type{Constraints}\xspace}
\newcommand{\Projects}{\type{Projects}\xspace}
\newcommand{\Selects}{\type{Selects}\xspace}
\newcommand{\Hyperslabs}{\type{Hyperslabs}\xspace}

\newcommand{\DIR}{\textbf{Directory}\xspace}
\newcommand{\TEXT}{\textbf{Text}\xspace}
\newcommand{\HTML}{\textbf{HTML}\xspace}
\newcommand{\HELP}{\textbf{Help}\xspace}
\newcommand{\VER}{\textbf{Version}\xspace}
\newcommand{\INFO}{\textbf{Info}\xspace}

%%%%%%%%%%%%%% Web Services Paper
\newcommand{\GetDDX}{\textbf{GetDDX}\xspace}
\newcommand{\GetData}{\textbf{GetData}\xspace}
\newcommand{\GetBlobData}{\textbf{GetBlobData}\xspace}
\newcommand{\GetBlob}{\textbf{GetBlob}\xspace}
\newcommand{\GetDir}{\textbf{GetDir}\xspace}
\newcommand{\GetInfo}{\textbf{GetInfo}\xspace}

\newcommand{\FSs}{\type{Foundation Services}\xspace}
\newcommand{\FS}{\type{Foundation Service}\xspace}
\newcommand{\ODSN}{\type{OPeNDAP}\xspace}

\newcommand{\blobdataelement}{\texttt{BlobData} element\xspace}
\newcommand{\blobelement}{\texttt{Blob} element\xspace}



%\setcounter{secnumdepth}{4}
%\setcounter{tocdepth}{4}

\newcommand{\Tableref}[1]{Table~\ref{#1}}%
\newcommand{\Figureref}[1]{Figure~\ref{#1}}%
\W\begin{iftex}
\newcommand{\Sectionref}[2]{Section~\ref{#1}%
  \ifx#2)%
    \ on page~\pageref{#1}%
  \else% 
    \ (page~\pageref{#1})\xspace%
  \fi\ifx#2\space\ \else #2\fi}%
\W\end{iftex}
\W\newcommand{\Sectionref}[1]{Section~\ref{#1}}
\W\newcommand{\raggedright}{}


%% Conveniences for documenting XML
\newcommand{\tag}[1]{\emph{#1}}
\newcommand{\element}[1]{\link{\tag{#1}}{sec-xml-#1}}
\newcommand{\attribute}[1]{\emph{#1}}
\newcommand{\currentelement}{}
\newcommand{\ELEMENT}[1]{\renewcommand{\currentelement}{#1 element}%
  \subsubsection{#1}\label{sec-xml-#1}\indc{\currentelement}%
  \indc{catalog tag!#1}\indc{aggregation tag!#1}%
  \indc{XML!#1 element}}
\newcommand{\ATTRIBUTE}[1]{\item{\lit{#1}}\indc{\currentelement!#1}
    \indc{#1 attribute!of \currentelement}%
    \indc{XML!#1 attribute}}

% Conveniences for examples
\newcounter{exampleno}
\setcounter{exampleno}{0}
\newcounter{examplerefno}
\setcounter{examplerefno}{0}
\newcommand{\examplelabel}[1]{\refstepcounter{exampleno}\label{#1}%
  \medskip Example \theexampleno :\smallskip}
\newcommand{\exampleref}[1]{\texorhtml{Example~\ref{#1}%
    \refstepcounter{examplerefno}\label{exref\theexamplerefno}%
    % This is a test whether r@exref... is defined.  If not, skip
    % anything with the \pageref macro.
    \catcode`\@=11%
    \expandafter\ifx\csname r@exref\theexamplerefno\endcsname\relax\else%
    \expandafter\ifx\csname r@#1\endcsname\relax\else%
    \bgroup\count100=\pageref{exref\theexamplerefno}%
    \count101=\pageref{#1}\ifnum\count100=\count101\else~%
    on page~\pageref{#1}\fi\egroup\fi\fi\xspace}%
  {\link{Example \ref{#1}}{#1}}}%
\T\setlength{\vcodeindent}{10pt}
\texorhtml{
%% LaTeX version
\newenvironment{textoutput}[1]{\ifx #1\relax%
  \medskip Output:\vspace{-\medskipamount}\else%
  \medskip #1\vspace{-\medskipamount}\fi%
  \begin{list}{}{\setlength{\leftmargin}{\vcodeindent}}\begin{ttfamily}\item}%
 {\end{ttfamily}\end{list}} %
}{%% Hyperlatex version
\newenvironment{textoutput}[1]{\xml{blockquote}Output:\\ \\ \xml{tt}}%
  {\xml{/tt}\xml{/blockquote}}%
\newenvironment{minipage}[4]{}{}
\newenvironment{ttfamily}{\xml{blockquote}\xml{tt}}%
  {\xml{/tt}\xml{/blockquote}}}

\newcommand{\DAPOverviewTitle}{DAP Specification Overview}
\newcommand{\DAPOverview}{\xlink{\textbf{\textit{\DAPOverviewTitle}}}%
  {dap.html}\xspace}

% Old macros; new values
\newcommand{\DAPObjectsTitle}{The Data Access Protocol---DAP 2.0}
\newcommand{\DAPObjects}{\xlink{\textbf{\textit{\DAPObjectsTitle}}}%
  {http://www.opendap.org/pdf/ESE-RFC-004v0.06.pdf}\xspace}
  
\newcommand{\DAPHTTPTitle}{Using DAP 2.0 with HTTP}
\newcommand{\DAPHTTP}{\xlink{\textbf{\textit{\DAPHTTPTitle}}}%
  {daph.html}\xspace}

% New macros.
\newcommand{\DAPDataModelTitle}{DAP Data Model Specification}
\newcommand{\DAPDataModel}{\xlink{\textbf{\textit{\DAPDataModelTitle}}}%
  {dapo.html}\xspace}
  
\newcommand{\DAPWebTitle}{DAP Web Services Specification}
\newcommand{\DAPWeb}{\xlink{\textbf{\textit{\DAPWebTitle}}}%
  {daph.html}\xspace}
  

\newcommand{\DAPASCIITitle}{DAP Formatted Data Specification}
\newcommand{\DAPASCII}{\xlink{\textbf{\textit{\DAPASCIITitle}}}%
  {dapa.html}\xspace}
\newcommand{\DAPHTMLTitle}{DAP HTTP Query Specification}
\newcommand{\DAPHTML}{\xlink{\textbf{\textit{\DAPHTMLTitle}}}%
  {dapm.html}\xspace}

% It probably doesn't matter what we call the macro, but SOAP does not have
% to run over HTTP and I think that's going to be important for other groups.
% 10/27/03 jhrg
\newcommand{\DAPServicesTitle}{DAP HTTP Services Specification}
\newcommand{\DAPServices}{\xlink{\textbf{\textit{\DAPServicesTitle}}}%
  {daps.html}\xspace}

\newcommand{\thirtytwobitlimit}[1]{$4,294,967,296$ #1 ($2^{32}$)}
%%% Local Variables: 
%%% mode: latex
%%% TeX-master: t
%%% End: 
