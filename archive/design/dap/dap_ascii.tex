%
% Specification for ASCII formatted data responses for the DAP
%

\documentclass[justify]{dods-paper}
\usepackage{acronym}
\usepackage{xspace}
\usepackage{gloss}
\rcsInfo $Id$

% latex and HTML macros. Some latex commands become nops for HTML. 4/10/2001
% jhrg 
\T\newcommand{\Cpp}{\rm {\small C}\raise.5ex\hbox{\footnotesize++}\xspace}
\T\newcommand{\C}{\rm {\small C}\xspace}
\W\newcommand{\Cpp}{C++}
\W\newcommand{\C}{C\xspace}
\W\newcommand{\cdots}{...}
\W\newcommand{\ddots}{}
\W\newcommand{\vdots}{.}
\W\newcommand{\pm}{+/-}
\W\newcommand{\times}{*}
\W\newcommand{\uppercase}[1]{\textsc{#1}}
\T\newcommand{\qt}{\lit{\char127}}
\W\newcommand{\qt}{"}

\texorhtml{\def\rearrangedate#1/#2/#3#4{\ifcase#2\or January\or February\or
  March\or April\or May\or June\or July\or August\or September\or
  October\or November\or December\fi\ \ifx0#3\relax\else#3\fi#4, #1}
\def\rcsdocumentdate{\expandafter\rearrangedate\rcsInfoDate}}%
{\HlxEval{
(put 'rearrangedate       'hyperlatex 'hyperlatex-ts-rearrange-date)

(defun hyperlatex-ts-rearrange-date ()
  (let ((date-string (hyperlatex-evaluate-string 
                       (hyperlatex-parse-required-argument))))
    (let ((year-string (substring date-string 0 4))
          (month-string (substring date-string 5 7))
          (day-string (substring date-string 8 10))
          (month-list '("January" "February" "March" "April"
                        "May" "June" "July" "August" 
                        "September" "October" "November" "December")))
       (insert
         (concat (elt month-list (1- (string-to-number month-string)))
                 " " (int-to-string (string-to-number day-string))
                 ", " year-string)))))
}
\newcommand{\rcsdocumentdate}{\rearrangedate{\rcsInfoDate}}}

\newcommand{\dapversion}{Version 4.0\xspace}
\newcommand{\opendap}{OPeNDAP\xspace}
\newcommand{\DAP}{DAP\xspace}
\newcommand{\DODS}{DODS\xspace}
\newcommand{\NVODS}{NVODS\xspace}
\newcommand{\CE}{constraint expression\xspace}
\newcommand{\CEs}{constraint expressions\xspace}
\newcommand{\ErrorX}{ErrorX\xspace}
\newcommand{\CapX}{Server Capabilities Document\xspace}
\newcommand{\Blob}{Blob\xspace}
\newcommand{\DDX}{DDX\xspace}
\newcommand{\DAX}{DAX\xspace}
\newcommand{\DDS}{DDS\xspace}
\newcommand{\DAS}{DAS\xspace}
\newcommand{\URI}{URI\xspace}
\newcommand{\DataDDS}{DataDDS\xspace}

\newcommand{\type}[1]{\emph{#1}}
\newcommand{\Alias}{\type{Alias}\xspace}
\newcommand{\Array}{\type{Array}\xspace}
\newcommand{\Attribute}{\type{Attribute}\xspace}
\newcommand{\ProcAttribute}{\type{Processing Attribute}\xspace}
\newcommand{\Map}{\type{Map}\xspace}
\newcommand{\Target}{\type{Target}\xspace}
\newcommand{\FQN}{fully qualified name\xspace}
\newcommand{\Grid}{\type{Grid}\xspace}
\newcommand{\Structure}{\type{Structure}\xspace}
\newcommand{\Dataset}{\type{Dataset}\xspace}
\newcommand{\Sequence}{\type{Sequence}\xspace}
\newcommand{\Container}{\type{Container}\xspace}
\newcommand{\Bi}{\type{Binary Image}\xspace}
\newcommand{\String}{\type{String}\xspace}
\newcommand{\URL}{\type{URL}\xspace}
\newcommand{\Boolean}{\type{Boolean}\xspace}
\newcommand{\Byte}{\type{Byte}\xspace}
\newcommand{\Enum}{\type{Enumeration}\xspace}
\newcommand{\Time}{\type{Time}\xspace}
\newcommand{\Function}{\type{Function}\xspace}
\newcommand{\Description}{\type{Description}\xspace}
\newcommand{\Parameter}{\type{Parameter}\xspace}
\newcommand{\Constraint}{\type{Constraint}\xspace}
\newcommand{\NoAttributes}{\type{NoAttributes}\xspace}
\newcommand{\Project}{\type{Project}\xspace}
\newcommand{\Select}{\type{Select}\xspace}
\newcommand{\Hyperslab}{\type{Hyperslab}\xspace}

\newcommand{\Aliases}{\type{Aliases}\xspace}
\newcommand{\Arrays}{\type{Arrays}\xspace}
\newcommand{\Attributes}{\type{Attributes}\xspace}
\newcommand{\ProcAttributes}{\type{Processing Attributes}\xspace}
\newcommand{\Maps}{\type{Maps}\xspace}
\newcommand{\Targets}{\type{Targets}\xspace}
\newcommand{\FQNs}{fully qualified names\xspace}
\newcommand{\Grids}{\type{Grids}\xspace}
\newcommand{\Structures}{\type{Structures}\xspace}
\newcommand{\Sequences}{\type{Sequences}\xspace}
\newcommand{\Containers}{\type{Containers}\xspace}
\newcommand{\Bis}{\type{Binary Images}\xspace}
\newcommand{\Strings}{\type{Strings}\xspace}
\newcommand{\URLs}{\type{URLs}\xspace}
\newcommand{\Booleans}{\type{Booleans}\xspace}
\newcommand{\Bytes}{\type{Bytes}\xspace}
\newcommand{\Enums}{\type{Enumerations}\xspace}
\newcommand{\Times}{\type{Times}\xspace}
\newcommand{\CSVs}{comma-separated values\xspace}
\newcommand{\Functions}{\type{Functions}\xspace}
\newcommand{\Descriptions}{\type{Descriptions}\xspace}
\newcommand{\Parameters}{\type{Parameters}\xspace}
\newcommand{\Constraints}{\type{Constraints}\xspace}
\newcommand{\Projects}{\type{Projects}\xspace}
\newcommand{\Selects}{\type{Selects}\xspace}
\newcommand{\Hyperslabs}{\type{Hyperslabs}\xspace}

\newcommand{\DIR}{\textbf{Directory}\xspace}
\newcommand{\TEXT}{\textbf{Text}\xspace}
\newcommand{\HTML}{\textbf{HTML}\xspace}
\newcommand{\HELP}{\textbf{Help}\xspace}
\newcommand{\VER}{\textbf{Version}\xspace}
\newcommand{\INFO}{\textbf{Info}\xspace}

%%%%%%%%%%%%%% Web Services Paper
\newcommand{\GetDDX}{\textbf{GetDDX}\xspace}
\newcommand{\GetData}{\textbf{GetData}\xspace}
\newcommand{\GetBlobData}{\textbf{GetBlobData}\xspace}
\newcommand{\GetBlob}{\textbf{GetBlob}\xspace}
\newcommand{\GetDir}{\textbf{GetDir}\xspace}
\newcommand{\GetInfo}{\textbf{GetInfo}\xspace}

\newcommand{\FSs}{\type{Foundation Services}\xspace}
\newcommand{\FS}{\type{Foundation Service}\xspace}
\newcommand{\ODSN}{\type{OPeNDAP}\xspace}

\newcommand{\blobdataelement}{\texttt{BlobData} element\xspace}
\newcommand{\blobelement}{\texttt{Blob} element\xspace}



%\setcounter{secnumdepth}{4}
%\setcounter{tocdepth}{4}

\newcommand{\Tableref}[1]{Table~\ref{#1}}%
\newcommand{\Figureref}[1]{Figure~\ref{#1}}%
\W\begin{iftex}
\newcommand{\Sectionref}[2]{Section~\ref{#1}%
  \ifx#2)%
    \ on page~\pageref{#1}%
  \else% 
    \ (page~\pageref{#1})\xspace%
  \fi\ifx#2\space\ \else #2\fi}%
\W\end{iftex}
\W\newcommand{\Sectionref}[1]{Section~\ref{#1}}
\W\newcommand{\raggedright}{}


%% Conveniences for documenting XML
\newcommand{\tag}[1]{\emph{#1}}
\newcommand{\element}[1]{\link{\tag{#1}}{sec-xml-#1}}
\newcommand{\attribute}[1]{\emph{#1}}
\newcommand{\currentelement}{}
\newcommand{\ELEMENT}[1]{\renewcommand{\currentelement}{#1 element}%
  \subsubsection{#1}\label{sec-xml-#1}\indc{\currentelement}%
  \indc{catalog tag!#1}\indc{aggregation tag!#1}%
  \indc{XML!#1 element}}
\newcommand{\ATTRIBUTE}[1]{\item{\lit{#1}}\indc{\currentelement!#1}
    \indc{#1 attribute!of \currentelement}%
    \indc{XML!#1 attribute}}

% Conveniences for examples
\newcounter{exampleno}
\setcounter{exampleno}{0}
\newcounter{examplerefno}
\setcounter{examplerefno}{0}
\newcommand{\examplelabel}[1]{\refstepcounter{exampleno}\label{#1}%
  \medskip Example \theexampleno :\smallskip}
\newcommand{\exampleref}[1]{\texorhtml{Example~\ref{#1}%
    \refstepcounter{examplerefno}\label{exref\theexamplerefno}%
    % This is a test whether r@exref... is defined.  If not, skip
    % anything with the \pageref macro.
    \catcode`\@=11%
    \expandafter\ifx\csname r@exref\theexamplerefno\endcsname\relax\else%
    \expandafter\ifx\csname r@#1\endcsname\relax\else%
    \bgroup\count100=\pageref{exref\theexamplerefno}%
    \count101=\pageref{#1}\ifnum\count100=\count101\else~%
    on page~\pageref{#1}\fi\egroup\fi\fi\xspace}%
  {\link{Example \ref{#1}}{#1}}}%
\T\setlength{\vcodeindent}{10pt}
\texorhtml{
%% LaTeX version
\newenvironment{textoutput}[1]{\ifx #1\relax%
  \medskip Output:\vspace{-\medskipamount}\else%
  \medskip #1\vspace{-\medskipamount}\fi%
  \begin{list}{}{\setlength{\leftmargin}{\vcodeindent}}\begin{ttfamily}\item}%
 {\end{ttfamily}\end{list}} %
}{%% Hyperlatex version
\newenvironment{textoutput}[1]{\xml{blockquote}Output:\\ \\ \xml{tt}}%
  {\xml{/tt}\xml{/blockquote}}%
\newenvironment{minipage}[4]{}{}
\newenvironment{ttfamily}{\xml{blockquote}\xml{tt}}%
  {\xml{/tt}\xml{/blockquote}}}

\newcommand{\DAPOverviewTitle}{DAP Specification Overview}
\newcommand{\DAPOverview}{\xlink{\textbf{\textit{\DAPOverviewTitle}}}%
  {dap.html}\xspace}

% Old macros; new values
\newcommand{\DAPObjectsTitle}{The Data Access Protocol---DAP 2.0}
\newcommand{\DAPObjects}{\xlink{\textbf{\textit{\DAPObjectsTitle}}}%
  {http://www.opendap.org/pdf/ESE-RFC-004v0.06.pdf}\xspace}
  
\newcommand{\DAPHTTPTitle}{Using DAP 2.0 with HTTP}
\newcommand{\DAPHTTP}{\xlink{\textbf{\textit{\DAPHTTPTitle}}}%
  {daph.html}\xspace}

% New macros.
\newcommand{\DAPDataModelTitle}{DAP Data Model Specification}
\newcommand{\DAPDataModel}{\xlink{\textbf{\textit{\DAPDataModelTitle}}}%
  {dapo.html}\xspace}
  
\newcommand{\DAPWebTitle}{DAP Web Services Specification}
\newcommand{\DAPWeb}{\xlink{\textbf{\textit{\DAPWebTitle}}}%
  {daph.html}\xspace}
  

\newcommand{\DAPASCIITitle}{DAP Formatted Data Specification}
\newcommand{\DAPASCII}{\xlink{\textbf{\textit{\DAPASCIITitle}}}%
  {dapa.html}\xspace}
\newcommand{\DAPHTMLTitle}{DAP HTTP Query Specification}
\newcommand{\DAPHTML}{\xlink{\textbf{\textit{\DAPHTMLTitle}}}%
  {dapm.html}\xspace}

% It probably doesn't matter what we call the macro, but SOAP does not have
% to run over HTTP and I think that's going to be important for other groups.
% 10/27/03 jhrg
\newcommand{\DAPServicesTitle}{DAP HTTP Services Specification}
\newcommand{\DAPServices}{\xlink{\textbf{\textit{\DAPServicesTitle}}}%
  {daps.html}\xspace}

\newcommand{\thirtytwobitlimit}[1]{$4,294,967,296$ #1 ($2^{32}$)}
%%% Local Variables: 
%%% mode: latex
%%% TeX-master: t
%%% End: 


\title{\DAPASCIITitle\\ DRAFT}
\htmltitle{\DAPASCIITitle\ -- DRAFT}
\author{James Gallagher\thanks{The University of Rhode Island,
    jgallagher@gso.uri.edu}, Tom Sgouros}
\date{Printed \today \\ Revision \rcsInfoRevision}
\htmladdress{James Gallagher <jgallagher@gso.uri.edu>, 
  \rcsInfoDate, Revision: \rcsInfoRevision}
\htmldirectory{html}
\htmlname{dapa}

\begin{document}

\maketitle
\T\tableofcontents

\section{Text Data Format}
\label{sec-ascii}

The text data response (also called the \lit{ASCII} response) provides
a way for a client to get data from a \DAP server formatted as plain
text. Server implementors may develop custom text representations for
their data as long as they meet the requirements given here:

\begin{enumerate}
\item The formatting must use comma-separated values with text that
  associates a name with a value or group of values.
\item The response body should be easy to incorporate into commonly
  available spreadsheet programs.
\item There must be enough structural information in the response to
  remove any ambiguity about which variable(s) it contains.  The
  intent of the text data object is that the client does not require a
  copy of the dataset \DDX to make sense of the data.\footnote{Variable
  attributes are not usually included in the text representation,
  though global ones often are.}
\end{enumerate}

The Text Data response from an \opendap\ server is an HTTP
service.  See \DAPHTTP\ for a description of the \DAP HTTP
implementation, and see \DAPObjects\ for a description of the \DAP\
overall.  
%\DAPServices contains descriptions of some of the other
%HTTP services. 

\subsection{Interaction}

The text response is triggered by a client using \lit{.asc} as the URL
suffix in a request to a \DAP HTTP server.  See \DAPHTTP for a
description of the \DAP URL components.

The reply must contain the following header:

\begin{textoutput}{Required Headers:}
XDODS-Server: dods/4.0\\
Content-Type: text/plain\\
Date: \emph{date}
\end{textoutput}


\section{ASCII Format Description}

The following sections present a suggested format for servers to
follow when returning text data.  In general, constructor datatypes
each have two different representations, one for when they contain
only atomic data types, and another for when they contain other
constructor data types.

Note that since the \Dataset type encloses all the variables in a
dataset, each text data message has an implicit (and anonymous)
\Structure as its outermost variable.


\subsection{Atomic Types}
\label{sec-ascii-atomic}

The \DAP atomic data types may be encoded as follows:

\begin{description}
\item[Integer types] Numbers without decimal point.

\item[Floating-Point types] Numbers with a decimal point.
  
\item[String types] Recorded as quoted strings using double quotes
  (\qt).  Double quote characters within the string are shown as
  \texorhtml{$\backslash$}{\back}\qt.
  
\item[Binary images] The base64 encoding of the binary image, enclosed
  in brackets.  (Base64 encoding uses only alphanumeric characters,
  plus \lit{+}, \lit{/}, and \lit{=}, and cannot produce commas,
  colons or brackets, so these may be included in documents with other
  data.)

\end{description}

A simple variable, whose only value is an atomic data type may be
presented with the variable name, followed by a colon, then the
value.  For example:

\begin{vcode}{ist}
ShipName: "Oceanus"
Latitude: 43.44
StationNumber: 378
GIF27: {R0lGODlhBwAHAIAAAAAAAP///ywAAAAABwAHAAACDERil7uHAFV7SlZUAAA7}
\end{vcode}

\subsection{\Grids}
\label{sec-ascii-grid}


\begin{enumerate}
\item A \Grid with only a single dimension will be represented as a table of
  values. The values of the sole map vector will be printed in a row,
  separated by commas and prefixed by the map vector's name. The values of
  the \Grid's \Array will be prefixed by its name and printed on the following
  row, also as \CSVs.  See \exampleref{ex-1}.

\begin{minipage}[t]{3in}
\examplelabel{ex-1}

\begin{vcode}{tis}
<Grid name="values">
  <Array name="x">
    <Byte/>
    <dimension name="lat" size="5"/>
  </Array>
  <Map name="lat">
    <Float64/>
    <dimension size="5"/>
  </Map>
</Grid>
\end{vcode}
\end{minipage}
\begin{minipage}[t]{5in}
\begin{textoutput}{Output}
Dataset: $dataset name$\\
values.lat, $values.lat[0]$, \ldots, $values.lat[4]$\\
values.x, $values.x[0]$, \ldots, $values.x[4]$
\end{textoutput}
\end{minipage}


\item A \Grid with two or more map vectors will be printed as follows. The
  rightmost dimension/map will be printed on one line as \CSVs and prefixed
  by the map vector's name as is the case for a \Grid with only a single map.
  The \Array's values will be printed on subsequent lines which enumerate
  all of the indices of the other dimensions/maps so that the rightmost
  varies fastest. Each row of values will be prefixed by the array-part's
  name and the list of bracketed map vector values which describe the given
  row's data. See \exampleref{ex-2} and \exampleref{ex-3}.


\begin{minipage}[t]{2.5in}
\examplelabel{ex-2}

\begin{vcode}{tis}
<Grid name="v">
  <Array name="temp">
    <Byte/>
    <dimension name="lat" size="5"/>
    <dimension name="lon" size="5"/>
  </Array>
  <Map name="lat">
    <Float64/>
    <dimension size="5"/>
  </Map>
  <Map name="lon">
    <Float64/>
    <dimension size="5"/>
  </Map>
</Grid>
\end{vcode}
\end{minipage}
\begin{minipage}[t]{5in}
\begin{textoutput}{Output}
Dataset: $dataset name$\\
v.lon, $v.lon[0]$, \ldots, $v.lon[4]$\\
v.temp[v.lat=$v.lat[0]$], $v.temp[0][0]$, \ldots, $v.temp[0][4]$\\
\vdots\\
v.temp[v.lat=$v.lat[4]$], $v.temp[4][0]$, \ldots, $v.temp[4][4]$
\end{textoutput}
\end{minipage}

\begin{minipage}[t]{2.5in}
\examplelabel{ex-3}

\begin{vcode}{its}
<Grid name="v">
  <Array name="temp">
    <Byte/>
    <dimension name="lat" size="5"/>
    <dimension name="lon" size="5"/>
  </Array>
  <Map name="t">
    <Int32/>
    <dimension size="4">
  </Map>
  <Map name="lat">
    <Float64/>
    <dimension size="5"/>
  </Map>
  <Map name="lon">
    <Float64/>
    <dimension size="5"/>
  </Map>
</Grid>
\end{vcode}
\end{minipage}
\\
\begin{minipage}[t]{5in}
\begin{textoutput}{Output}
Dataset: $dataset name$\\
v.lon, $v.lon[0]$, \ldots, $v.lon[4]$\\
v.temp[v.t=$v.t[0]$][v.lat=$v.lat[0]$], $temp[0][0][0]$, \ldots, $temp[0][0][4]$\\
\vdots\\
v.temp[v.t=$v.t[0]$][v.lat=$v.lat[4]$], $temp[0][4][0]$, \ldots, $temp[0][4][4]$\\
\vdots\\
v.temp[v.t=$v.t[3]$][v.lat=$v.lat[0]$], $temp[3][0][0]$, \ldots, $temp[3][0][4]$\\
\vdots\\
v.temp[v.t=$v.t[3]$][v.lat=$v.lat[4]$], $temp[3][4][0]$, \ldots, $temp[3][4][4]$\\
\end{textoutput}
\end{minipage}

\end{enumerate}

\subsection{\Structures}
\label{sec-ascii-structure}

\begin{enumerate}
\item \Structures that contain only simple types are printed as a table
  with a single row. That is, a row of comma separated field names
  followed by a row of comma separated values. See \exampleref{ex-4}.

\begin{minipage}[t]{2.25in}
\examplelabel{ex-4}

\begin{vcode}{it}
<Structure name="drop">
  <String name="name">
  <String name="date">
  <Float64 name="lat">
  <Float64 name="lon">
</Structure>
\end{vcode}
\end{minipage}
\begin{minipage}[t]{5in}
\begin{textoutput}{Output}
Dataset: $dataset name$\\
drop.name, drop.date, drop.lat, drop.lon\\
$drop.name$, $drop.date$, $drop.lat$, $drop.lon$
\end{textoutput}
\end{minipage}

\item A \Structure contains simple types and/or other \Structures which
  themselves contain only simple types will be flattened and printed
  as if it were a single \Structure composed of solely of simple types.
  See \exampleref{ex-5}.

\begin{minipage}[t]{2.25in}
\examplelabel{ex-5}

\begin{vcode}{sit}
<Structure name="drop">
  <String name="name"/>
  <String name="date"/>
  <Structure name="point">
    <Float64 name="lat"/>
    <Float64 name="lon"/>
  </Structure>
</Structure>
\end{vcode}
\end{minipage}
\begin{minipage}[t]{5in}
\begin{textoutput}{Output}
Dataset: $dataset name$\\
drop.name, drop.date, drop.point.lat, drop.point.lon\\
$drop.name$, $drop.date$, $drop.point.lat$, $drop.point.lon$
\end{textoutput}
\end{minipage}

\item A \Structure that contains one or more \Array, \Sequence or \Grid
  variables uses a vertical ordering of its fields. Each field is
  printed by writing its print representation on successive rows,
  starting with the first field and going in order to the last. For
  each row the type of variable determines the specific format of the
  output. \exampleref{ex-6} shows a \Structure which contains a \String
  and a \Grid.

\begin{minipage}[t]{2.5in}
\examplelabel{ex-6}

\begin{vcode}{xit}
<Structure name="test">
  <String name="date"/>
  <Grid name="values">
    <Array name="temp">
      <Byte/>
      <dimension name="lat" size="5"/>
      <dimension name="lon" size="5"/>
    </Array>
    <Map name="lat">
      <Float64/>
      <dimension size="5"/>
    </Map>
    <Map name="lon">
      <Float64/>
      <dimension size="5"/>
    </Map>
  </Grid>
</Structure>
\end{vcode}
\end{minipage}

\begin{minipage}[t]{5in}
\begin{textoutput}{Output}
Dataset: $dataset name$\\
test.date, $test.date$\\
test.v.lon, $test.v.lon[0]$, \ldots, $temp.lon[4]$\\
test.v.temp[$test.v.lat[0]$]: $test.v.temp[0][0]$, \ldots, $test.v.temp[0][4]$\\
\vdots\\
test.v.temp[$test.v.lat[4]$]: $test.v.temp[4][0]$, \ldots, $test.v.temp[4][4]$
\end{textoutput}
\end{minipage}

\end{enumerate}


\subsection{\Sequences}
\label{sec-ascii-sequence}

\begin{enumerate}
\item \Sequences containing only simple types will be represented as tables.
  The first row of output will list the \Sequence's field names separated by
  commas. Each instance of the \Sequence will be listed on subsequent rows as
  comma separated values.  See \exampleref{ex-7}.

\begin{minipage}[t]{2.5in}
\examplelabel{ex-7}

\begin{vcode}{it}
<Sequence name="drop">
  <String name="name"/>
  <Float64 name="lat"/>
  <Float64 name="lon"/>
</Sequence>
\end{vcode}
\end{minipage}
\begin{minipage}[t]{5in}
\begin{textoutput}{Output}
Dataset: $dataset name$\\
drop.name, drop.lat, drop.lon\\
$drop.name$, $drop.lat$, $drop.lon$\\
\vdots
\end{textoutput}
\end{minipage}

\item \Sequences containing simple types and/or other \Sequences or \Structures
  which themselves contain only simple types will be flattened and printed as
  a table.  See \exampleref{ex-8} and \exampleref{ex-9}.

\begin{minipage}[t]{2.5in}
\examplelabel{ex-8}

\begin{vcode}{its}
<Sequence name="drop">
  <String name="name"/>
  <Sequence name="loc">
    <Float64 name="lat"/>
    <Float64 name="lon"/>
  </Sequence>
</Sequence>
\end{vcode}
\end{minipage}
\begin{minipage}[t]{5in}
\begin{textoutput}{Output}
Dataset: $dataset name$\\
drop.name, drop.loc.lat, drop.loc.lon\\
$drop.name$, $drop.loc.lat$, $drop.loc.lon$\\
\vdots
\end{textoutput}
\end{minipage}

\begin{minipage}[t]{2.5in}
\examplelabel{ex-9}

\begin{vcode}{it}
<Sequence name="drop">
  <String name="name"/>
  <Structure name="loc">
    <Float64 name="lat"/>
    <Float64 name="lon"/>
  </Structure>
</Sequence>
\end{vcode}
\end{minipage}
\begin{minipage}[t]{5in}
\begin{textoutput}{Output}  
Dataset: $dataset name$\\
drop.name, drop.loc.lat, drop.loc.lon\\
$drop.name$, $drop.loc.lat$, $drop.loc.lon$\\
\vdots
\end{textoutput}
\end{minipage}

\item \Sequences that contain either, directly or indirectly, \Arrays or \Grids
  will be printed with each field on its own row. Successive rows of the
 \Sequence will be printed as successive blocks of rows. To print each field
  its type-dependent print representation will be used. See
  \exampleref{ex-10} and \exampleref{ex-11}.


\begin{minipage}[t]{2.5in}
\examplelabel{ex-10}

\begin{vcode}{it}
<Sequence name="data">
  <String name="name">
  <Array name="temps">
    <Byte/>
    <dimension size="5">
  </Array>
</Sequence>
\end{vcode}
\end{minipage}
\begin{minipage}[t]{5in}
\begin{textoutput}{Output}
Dataset: $dataset name$\\
data.name, $data.name$\\
data.temps, $data.temps[0]$, \ldots, $data.temps[4]$\\
\vdots
\end{textoutput}
\end{minipage}

\begin{minipage}[t]{2.25in}
\examplelabel{ex-11}

\begin{vcode}{it}
<Sequence name="data">
  <String name="name"/>
  <Array name="temps">
    <Byte/>
    <dimension size="5">
    <dimension size="5">
  </Array>
</Sequence>    
\end{vcode}
\end{minipage}
\begin{minipage}[t]{4in}
\begin{textoutput}{Output}
Dataset: $dataset name$\\
data.name, $data.name_0$\footnote{The subscripts are here to reinforce that a
 \Sequence consists of $N$ values (or instances) and thus there are $N$
  values for both \texttt{name} and \texttt{temps}.}\\
data.temps[0], $data.temps[0][0]_0$, \ldots, $data.temps[0][4]_0$\\
\vdots\\
data.temps[4], $data.temps[4][0]_0$, \ldots, $data.temps[4][4]_0$\\
\vdots\\
data.name, $data.name_N$\\
data.temps[0], $data.temps[0][0]_N$, \ldots, $data.temps[0][4]_N$\\
\vdots\\
data.temps[4], $data.temps[4][0]_N$, \ldots, $data.temps[4][4]_N$\\
\end{textoutput}
\end{minipage}

\end{enumerate}

\subsection{\Arrays}
\label{sec-ascii-array}

\begin{enumerate}
\item A one-dimensional \Array of simple types is printed as a name and comma
  separated list of values of the vector, all on a single line. See \exampleref{ex-12}.

\begin{minipage}[t]{2.5in}
\examplelabel{ex-12}

\begin{vcode}{it}
<Array name="x">
  <Int32/>
  <dimension size="10"/>
</Array>
\end{vcode}
\end{minipage}
\begin{minipage}[t]{5in}
\begin{textoutput}{Output}
Dataset: $dataset name$\\
x, $x[0]$, \ldots, $x[9]$
\end{textoutput}
\end{minipage}

\item A multidimensional \Array of simple types of rank $N$ is represented by
  writing out, for each of its $N-1$ dimensions' indices, the name, the index
  and a row of values. See \exampleref{ex-13} and \exampleref{ex-14}.

\begin{minipage}[t]{2.5in}
\examplelabel{ex-13}

\begin{vcode}{it}
<Array name="x">
  <Int32/>
  <dimension size="20"/>
  <dimension size="10"/>
</Array>
\end{vcode}
\end{minipage}
\begin{minipage}[t]{5in}
\begin{textoutput}{Output}
Dataset: $dataset name$\\
x[0], $x[0][0]$, \ldots, $x[0][9]$\\
\vdots\\
x[19], $x[19][0]$, \ldots, $x[19][9]$\\
\end{textoutput}
\end{minipage}

\begin{minipage}[t]{2.5in}
\examplelabel{ex-14}

\begin{vcode}{it}
<Array name="x">
  <Int32/>
  <dimension size="5"/>
  <dimension size="10"/>
  <dimension size="15"/>
</Array>
\end{vcode}
\end{minipage}
\begin{minipage}[t]{5in}
\begin{textoutput}{Output}
Dataset: $dataset name$\\
x[0][0], $x[0][0][0]$, \ldots, $x[0][0][14]$\\
\vdots\\
x[0][9], $x[0][9][0]$, \ldots, $x[0][9][14]$\\
\vdots\\
x[4][0], $x[4][0][0]$, \ldots, $x[4][0][14]$\\
\vdots\\
x[4][9], $x[4][9][0]$, \ldots, $x[4][9][14]$\\
\end{textoutput}
\end{minipage}

\item Each element of an \Array of complex types is printed on a group of
  lines. For an \Array of complex types of rank $N$, each row will be prefixed
  by the \Array name and the index for that row.  See
  \exampleref{ex-15} and \exampleref{ex-16}.

\begin{minipage}[t]{2.5in}
\examplelabel{ex-15}

\begin{vcode}{it}
<Array name="pair">
  <Structure>
    <Int32 name="x"/>
    <String name="name"/>
  </Structure>
  <dimension size="10"/>
</Array>
\end{vcode}
\end{minipage}
\begin{minipage}[t]{5in}
\begin{textoutput}{Output}
Dataset: $dataset name$\\
pair[0], $pair[0].x$, $pair[0].name$\\
\vdots\\
pair[9], $pair[9].x$, $pair[9].name$
\end{textoutput}
\end{minipage}

\begin{minipage}[t]{2.5in}
\examplelabel{ex-16}

\begin{vcode}{it}
<Array name="pair">
  <Structure>
    <Int32 name="x"/>
    <String name="name"/>
  </Structure>
  <dimension size="5"/>
  <dimension size="10"/>
</Array>
\end{vcode}
\end{minipage}
\begin{minipage}[t]{5in}
\begin{textoutput}{Output}
Dataset: $dataset name$\\
pair[0][0], $pair[0][0].x$, $pair[0][0].name$\\
\vdots\\
pair[0][9], $pair[0][9].x$, $pair[0][9].name$\\
\vdots\\
pair[4][0], $pair[4][0].x$, $pair[4][0].name$\\
\vdots\\
pair[4][9], $pair[4][9].x$, $pair[4][9].name$\\
\end{textoutput}
\end{minipage}

\end{enumerate}

\appendix

%\bibliographystyle{plain}
%\T\addcontentsline{toc}{section}{References}
%\T\raggedright
%\bibliography{../../../boiler/dods}

\section{Alternate Form}

The following DDS examples correspond to the XML examples that appear
earlier in this text.

\exampleref{ex-1}
\begin{vcode}{tis}
Dataset {
  Grid {
    Array:
      Byte x[5];
    Maps:
      Float64 lat[5];
  } values;
} ex1;
\end{vcode}

\exampleref{ex-2}
\begin{vcode}{tis}
Dataset {
  Grid {
    Array: 
      Byte temp[lat=5][lon=5];
    Maps:
      Float64 lat[5];
      Float64 lon[5];
  } v;
} ex2;
\end{vcode}

\exampleref{ex-3}
\begin{vcode}{tis}
Dataset {
  Grid {
    Array: 
      Byte temp[t=4][lat=5][lon=5];
    Maps:
      Int32 t[4];
      Float64 lat[5];
      Float64 lon[5];
  } v;
} ex3;
\end{vcode}

\exampleref{ex-4}
\begin{vcode}{tis}
Dataset {
  Structure {
    String name;
    String date;
    Float64 lat;
    Float64 lon;
  } drop;
} ex6;
\end{vcode}

\exampleref{ex-5}
\begin{vcode}{tis}
Dataset {
  Structure {
    String name;
    String date;
    Structure {
      Float64 lat;
      Float64 lon;
    } point;
  } drop;
} ex7;
\end{vcode}

\exampleref{ex-6}
\begin{vcode}{tis}
Dataset {
  Structure {
    String date;
    Grid {
      Array: 
      Byte temp[lat=5][lon=5];
      Maps:
      Float64 lat[5];
      Float64 lon[5];
    } values;
  } test;
} ex8;
\end{vcode}

\exampleref{ex-7}
\begin{vcode}{tis}
Dataset {
  Sequence {
    String name;
    Float64 lat;
    Float64 lon;
  } drop;
} ex8;
\end{vcode}

\exampleref{ex-8}
\begin{vcode}{tis}
Dataset {
  Sequence {
    String name;
    Sequence {
      Float64 lat;
      Float64 lon;
    } loc;
  } drop;
} ex10;
\end{vcode}

\exampleref{ex-9}
\begin{vcode}{tis}
Dataset {
  Sequence {
    String name;
    Structure {
      Float64 lat;
      Float64 lon;
    } loc;
  } drop;
} ex11;
\end{vcode}

\exampleref{ex-10}
\begin{vcode}{tis}
Dataset {
  Sequence {
    String name;
    Array temps[5];
  } data;
} ex12;
\end{vcode}

\exampleref{ex-11}
\begin{vcode}{tis}
Dataset {
  Sequence {
    String name;
    Array temps[5][5];
  } data;
}ex13;
\end{vcode}

\exampleref{ex-12}
\begin{vcode}{tis}
Dataset {
    Int32 x[10];
} ex1;
\end{vcode}

\exampleref{ex-13}
\begin{vcode}{tis}
Dataset {
    Int32 x[20][10];
} ex3;
\end{vcode}

\exampleref{ex-14}
\begin{vcode}{tis}
Dataset {
    Int32 x[5][10][15];
} ex4;
\end{vcode}

\exampleref{ex-15}
\begin{vcode}{tis}
Dataset {
    Structure {
        Int32 x;
        String name;
    } pair[10];
} ex2;
\end{vcode}

\exampleref{ex-16}
\begin{vcode}{tis}
Dataset {
    Structure {
        Int32 x;
        String name;
    } pair[5][10];
} ex5;
\end{vcode}



\section{Change log}

\begin{verbatim}
$Log: dap_ascii.tex,v $
Revision 1.6  2003/07/24 22:32:12  tom
excised dap_services document references

Revision 1.5  2003/07/18 18:57:58  tom
moved out all DDS forms

Revision 1.4  2003/07/16 01:06:08  tom
progress on comments, fixed titles

Revision 1.3  2003/05/23 22:08:30  tom
makefile repairs, bibliography

Revision 1.2  2003/05/23 21:50:52  tom
progress made

Revision 1.1  2003/05/23 19:27:42  tom
new files, rearranging DAP spec into separate documents


\end{verbatim}

\end{document}
