% A usage specification for DODS
%
%  Tom Sgouros, 3/98
%
%  $Id$


% Things to do:
%
%   Worry about the connection between the GUI and the analysis
% package  (loaddods? IPC? whatever?)
%   Worry about the hierarchy of attributes.
%   Incorporate James's comments.
%



\documentclass[10pt]{report}
\usepackage{psfig,html,ifthen,fancyhdr,code,changebar}
\pagestyle{fancyplain}

\newcommand{\cs}{Catalog Server}
\newcommand{\gui}{GUI}
\newcommand{\bigspace}{\vspace*{1.5in}}


\raggedbottom
\sloppy

% When `draft-document is 1, print marginalia.
\newcounter{draft-document}
\setcounter{draft-document}{0}
%%% Tex customizations and command definitions for DODS user
%%% guides, 2 October 1997 - tomfool
%%%
%%% $Id$

%%% $Log: layout.tex,v $
%%% Revision 1.5  1999/05/25 20:49:34  tom
%%% changed version numbers to 3.0
%%%
%%% Revision 1.4  1999/02/04 17:39:02  tom
%%% modified for use with dods-book.cls
%%%
%%% Revision 1.3  1998/03/13 20:43:26  tom
%%% writing of API manual.
%%%
%%% Revision 1.2  1998/02/12 15:47:25  tom
%%% updated for GUI doc
%%%
%%% Revision 1.1  1997/10/02 17:18:14  tom
%%% moved from user guide to boiler, made slightly more general,
%%% for use with other guides.
%%%

%%%%%%%%%%%%%%%%%%%%%%%%%%%%%%%%%%%%%%%%%%%%%%%%%%%%%%%%%%%%%%%%%%%%%
%%% This file is for TeX macros that are equally appropriate for the
%%% hard-copy (LaTeX) and html (hyperlatex) versions of the dods
%%% books.  If it's not appropriate for both, then it probably belongs
%%% in dods-book.cls or dods-book.hlx.

%%%%%%%%%%%%%%%%%%%%%%%%%%%%%%%%%%%%%%%%%%%%%%%%%%%%%%%%%%%%%%%%%%%%%
\NotSpecial{\do\_}% This removes the special meaning of `_', so for
                  % subscripts, there must be an `tex' environment
                  % around any diagram using subscripts, and an entire
                  % alternate html figure using <sub> tags.


%%%%%%%%%%%%%%%%%%%%%%%%%%%%%%%%%%%%%%%%%%%%%%%%%%%%%%%%%%%%%%%%%%%%%
%%% Different kinds of cross references.
%\newcommand{\chapterref}[1]{Chapter~\refl{#1} on page~\pagerefl{#1}}
\newcommand{\chapterref}[1]{\link{Chapter~\ref{#1}}{#1}}
\newcommand{\appref}[1]{\link{Appendix~\ref{#1}}%
  [~on page~\pageref{#1}]{#1}}
\newcommand{\sectionref}[1]{\link{Section~\ref{#1}}%
  [~on page~\pageref{#1}]{#1}} 
\newcommand{\pagexref}[1]{\link*{here}[page~\pageref{#1}]{#1}}
\newcommand{\tableref}[1]{\link{table~\ref{#1}}{#1}}
\newcommand{\Tableref}[1]{\link{Table~\ref{#1}}{#1}}
\newcommand{\figureref}[1]{\link{figure~\ref{#1}}{#1}}
\newcommand{\Figureref}[1]{\link{Figure~\ref{#1}}{#1}}

%%% A Prefatory list of the font conventions:
\newcommand{\listconventions} {
\section{Conventions}
\label{pref,conventions}

The \indn{typographic conventions} shown in
Table~\ref{typo-conventions} are followed in this guide and all the
other DODS documentation.

\begin{table}[htbp]
  \begin{center}
  \caption{Typographic Conventions}
  \label{typo-conventions}
  \begin{tabular}{|c|p{3in}|} \hline
    \lit{Literal text}  &  
         Typed by the computer, or in a code listing.\\ \hline
    \inp{User input}    &  
         Type this precisely as written.\\ \hline
    \var{Variables}     &   
         Used as a place holder for a user-specified or variable
         value. Choose an appropriate value and use that in place.\\
         \hline 
    \but{Button Text}\texonly{\rule{0pt}{2.5ex}}   
        &  Used to indicate text on a GUI button.\\ 
         \hline 
    \pdmenu{Menu Name}    &  This is the name of a GUI menu.\\ \hline 
  \end{tabular}
  \end{center}
\end{table}

When referring to a button in a menu, we will often use the notation:
\but{Menu,Button}. For example, \but{Options,Colors,Foreground} would
indicate the \but{Foreground} button in the \pdmenu{Colors} menu,
selected under the \pdmenu{Options} menu.
 }


%%% Local Variables: 
%%% mode: latex
%%% TeX-master: t
%%% End: 

\setlength{\changebarsep}{10pt}

\begin{document}

\title{A \emph{Usage} Specification for DODS}
\author{Tom Sgouros}
\date{\today}

\maketitle

\begin{abstract} 
  The DODS functionality is very close to complete.  Servers exist who
  can send data requested by clients, and clients exist who can ask
  for it.  There remain a few problems of data API interoperability,
  platform support and so on, but while knotty, they are entirely
  surmountable details.  The most substantial obstacle to widespread
  DODS use has to do with the lack of attention paid to the \emph{use}
  of the system.  In this paper I describe several usage problems,
  some widely agreed-upon, and others not, and attempt to suggest (and
  specify) design solutions to them.
\end{abstract}



\tableofcontents

\chapter{Introduction}
\label{sec:intro}

The DODS data retrieval system as it currently exists is a nearly
complete solution to many of the problems involved in accessing
remote earth science data.  With a few exceptions, DODS provides a way
to access data on remote machines with a minimum of fuss, independent
of the data format.  There are still several wrinkles to work out,
like specific data model translation issues, platform portability, and
error handling and so on, but these are details.  The design works.

Despite this success, there remain several serious obstacles to
widespread use of DODS in the science community.  (Besides overcoming
the dreaded ``not-invented-here'' syndrome.)  Most of these have to do
with data retrieval issues not addressed by DODS.  For example, DODS
deals well with issues of conflicting data models, but not at all with
conflicting choices of units and names.  DODS deals well with the
issue of decentralizing data archives, but not at all with how
researchers are to find the decentralized data.

The obstacles I've identified are these:

\begin{itemize}

\item Lack of a central source of data about DODS datasets.

\item Datasets rendered compatible by clever DODS software may still
  be incompatible because of differing units or names or usage
  conventions. 

\item Lack of a ``browser'' compatible with several different analysis
  packages.

\end{itemize}

The following sections contain design proposals for dealing with these
issues.  They may be read in any order, but are meant as a unified
proposal.

Several of the proposed designs would impose certain standards (for
attribute and variable names, for example) on datasets.  It is
important to note that DODS itself does not impose these standards.  A
researcher can use DODS to share his or her data with colleagues or
whomever without conforming to these proposed standards.  Datasets
that do conform, however, will be more widely accessible via the \cs\ 
and \gui\ proposed here.  Further, nothing here would preclude
\emph{another} similar standard from being created.\footnote{This is
  not a bad thing, since standards relevant for earth science data may
  not apply to genetic data, for example, or accounting data or
  whatever.  DODS is infrastructure for data exchange in the same way
  that roads are infrastructure for travel.  The same roads can be
  used by motorcycles, Bentleys, go-carts, and unicycles.  Similarly,
  because a community of earth science researchers establish certain
  standards for DODS earth science data does not mean that a community
  of particle physicists should feel bound by them.}


\chapter{Catalog Server}
\label{sec:catalog}

\cbstart
This proposal envisions an indefinite number of \cs s in the world.
Typically, there will be a \cs\ for any large collection of data,
whether it be one or many datasets.

The \cs\ has two functions:

\begin{itemize}
\item To help make sense of the internal organization of DODS
  datasets, and
\item To help a researcher survey the data available that might apply
  to a particular scientific problem.
\end{itemize}

\section{A Possible Catalog Structure}
\label{sec:dbstruct}

The internal organization problem was first identified in the case of
datasets that occupy multiple disk files.  The AVHRR archive, for
example, is organized into several thousand data files, each
containing a single image.  The \cs\ was envisioned as the solution to
the problem of linking each data file with the time value
corresponding to it.

Consider a relational database that links time values with images.  It
might be modeled with a table like the following:

\begin{table}[h] 
\begin{center}
\textbf{AVHRR Archive} \\
\begin{tabular}{|l|l|} \hline
Time        &Image \\ \hline
1400-4Apr98 &Img1 \\
1800-4Apr98 &Img2 \\
2200-4Apr98 &Img3 \\
0200-5Apr98 &Img4 \\
0600-5Apr98 &Img5 \\
1000-5Apr98 &Img6 \\ \hline
\end{tabular}
\end{center}
\caption{A Relational Table for the AVHRR Archive}
\label{fig:reltbl}
\end{table}

\noindent
The ``Img\emph{X}'' entries stand for an image object instead of a
simple scalar value.  This table associates the time attribute with
each image, so that a search for a specific time or range of times can
return a specific image or images.

Assume that the table and images are to be served by some server or
set of servers.  With the addition of a column for storing the URL of
a particular data file, our data catalog could look like this:

\begin{center}
\textbf{AVHRR Archive} \\
\begin{tabular}{|l|l|l|} \hline
Time        &Image &URL \\ \hline
1400-4Apr98 &      &URL1 \\
1800-4Apr98 &      &URL2 \\
2200-4Apr98 &      &URL3 \\
0200-5Apr98 &      &URL4 \\
0600-5Apr98 &      &URL5 \\
1000-5Apr98 &      &URL6 \\ \hline
\end{tabular} \\ {\huge\strut}
\textbf{URL1}
\begin{tabular}{|l|} \hline
Img1 \\ \hline
\end{tabular}
\hfill
\textbf{URL2}
\begin{tabular}{|l|} \hline
Img2 \\ \hline
\end{tabular}
\hfill
\textbf{URL3}
\begin{tabular}{|l|} \hline
Img3 \\ \hline
\end{tabular} \\ {\huge\strut} 
\textbf{URL4}
\begin{tabular}{|l|} \hline
Img4 \\ \hline
\end{tabular}
\hfill
\textbf{URL5}
\begin{tabular}{|l|} \hline
Img5 \\ \hline
\end{tabular}
\hfill
\textbf{URL6}
\begin{tabular}{|l|} \hline
Img6 \\ \hline
\end{tabular}
\end{center}

\noindent
The data here is distributed among seven different sources, which
could be different machines or just different URLs.  An energetic
client could use these seven servers to amass the data for a single
large table like the one in \tableref{fig:reltbl}.  Alternatively,
the primary server labeled ``AVHRR Archive'' could use the six
subordinate data servers to assemble the same table.

One thing is important to note: the table in \tableref{fig:reltbl}
exists nowhere.  It is entirely a dynamic construction, whose pieces
are constructed as needed.  The table can ``exist'' without all the
data being on one server \emph{if} the primary server is smart enough
to issue a subordinate request when the data it has is not enough to
fulfill some query.  So, if the primary server received a request for
all the images recorded in the morning of April 5, 1998, it would
issue subordinate requests for the data stored at URL4, URL5, and
URL6.  The information on the primary server is enough to determine
that the data at URL1, URL2, and URL3 need not be requested to satisfy
the query.

Some queries need not produce any subordinate requests.  For example,
a request for a list of times where there is a corresponding image
could be satisfied entirely by the primary server, without any
subordinate queries.

This structure is roughly the picture of the \cs\ that has been under
discussion in the DODS group.  The structure proposed in this paper is
simply an extension of that model that covers much of the other
presumed used for the \cs .  Some of the refinement is a software
issue; the \cs\ must be somewhat more sophisticated than was
previously envisioned.  However, much more functionality will be added
due to the organization and definition of information within the
framework defined by the \cs\ software.

\subsection{Refining the Picture}

Consider a slightly more complex example.  The following (vastly
simplified) relational table could be used for XBT data.  The \emph{T}
indicates a vector of temperature data, and the \emph{P} indicates a
corresponding vector of pressures.

\begin{center}
\textbf{XBT Data} \\
\begin{tabular}{|l|l|l|l|l|l|} \hline
Ship    &Time    &Lon  &Lat &Press &Temp  \\ \hline
Oceanus &4Apr98 &-40.3 &34.4&P1    &T1    \\
Knorr   &5Apr98 &-42.2 &32.3&P2    &T2    \\ \hline 
\end{tabular}
\end{center}

For a simple \cs , like in the previous example, all of the
information in the table except the pressure and temperature vectors
would be in the \cs\ and the vectors themselves would be in the data
files.  So this information could be arranged\footnote{The operative
  word is ``could.''  There are many ways to arrange this information,
  and what is shown is perhaps not among the likely alternatives.  But
  it is useful for illustrating the database structure.} on three
different servers like this (one \cs\ and two standard DODS servers):

\begin{center}
\textbf{\cs} \\
\begin{tabular}{|l|l|l|l|l|l|l|} \hline
Ship    &Time    &Lon  &Lat &Press &Temp &URL \\ \hline
Oceanus &4Apr98 &-40.3 &34.4&      &     &URL1 \\
Knorr   &5Apr98 &-42.2 &32.3&      &     &URL2 \\ \hline 
\end{tabular}

\strut \\ \textbf{URL1}\\
\begin{tabular}{|c|c|} \hline
P1   &T1 \\ \hline 
\end{tabular}

\strut \\ \textbf{URL2}\\
\begin{tabular}{|c|c|} \hline
P2   &T2 \\ \hline
\end{tabular}
\end{center}

To see how this would work, look at what would happen if the \cs\
receives a query.  Assume that the queries come in the form of a DODS
constraint expression, and imagine the \cs\ receiving an expression
like this:

\begin{code}{c}
?URL&Time=4Apr98&Ship="Oceanus"
\end{code}

or this:

\begin{code}{c}
?URL&Lon>-41&Lat>30
\end{code}

Both of these queries would return a list of URLs corresponding to the
conditions found in the constraint expression.\footnote{Note that the
  requested scalar in the example, ``URL'', must be turned into a
  List.}  Of course, if you are only browsing the dataset, rather than
trying to retrieve data, you could also make queries like these:

\begin{code}{c}
?Time&Lon>-41&Lat>30
\end{code}

or

\begin{code}{c}
?Ship&Time>1Apr98&Time<1May98
\end{code}

These requests would return lists of dates and ships corresponding to
the input constraints.  This provides a useful way for a user to see
the dimensions of a dataset.


\subsubsection{Subsidiary \cs s}

Now consider an archive of a large number of data files, sharing
several attributes.  The relational table for a \cs\ containing such
an archive is shown here:

\begin{center}
\textbf{Whole Shebang} \\
\begin{tabular}{|l|l|l|l|l|l|} \hline
Ship    &Time    &Lon  &Lat &Press &Temp \\ \hline
Oceanus &4Apr98 &-40.3 &34.4&P1A   &T1A  \\
Oceanus &5Apr98 &-40.3 &34.4&P1B   &T1B  \\
Oceanus &6Apr98 &-40.3 &34.4&P1C   &T1C  \\
Oceanus &7Apr98 &-40.3 &34.4&P1D   &T1D  \\
Knorr   &5Apr98 &-42.2 &32.3&P2    &T2   \\ \hline 
\end{tabular}
\end{center}

This data could be distributed among two \cs s in the following
manner:

\begin{center}
\textbf{Primary \cs} \\
\begin{tabular}{|l|l|l|l|l|l|l|} \hline
Ship    &Time    &Lon  &Lat &Press &Temp &URL \\ \hline
Oceanus &       &-40.3 &34.4&      &     &CS1 \\
Knorr   &5Apr98 &-42.2 &32.3&      &     &URL2 \\ \hline 
\end{tabular}

\strut \\ \textbf{CS1}\\
\begin{tabular}{|l|l|l|l|} \hline
Time    &Press &Temp &URL \\ \hline
4Apr98  &      &     &URL1A \\
5Apr98  &      &     &URL1B \\
6Apr98  &      &     &URL1C \\
7Apr98  &      &     &URL1D \\ \hline
\end{tabular} \\ {\huge\strut}
\textbf{URL1A}
\begin{tabular}{|c|c|} \hline
P1A &T1A \\ \hline
\end{tabular}
\hfill
\textbf{URL1B}
\begin{tabular}{|c|c|} \hline
P1B &T1B \\ \hline
\end{tabular} \\ {\huge\strut} 
\textbf{URL1C}
\begin{tabular}{|c|c|} \hline
P1C &T1C \\ \hline
\end{tabular} 
\hfill
\textbf{URL1D}
\begin{tabular}{|c|c|} \hline
P1D &T1D \\ \hline
\end{tabular} \\ {\huge\strut}
\textbf{URL2}
\begin{tabular}{|c|c|} \hline
P2   &T2 \\ \hline
\end{tabular}
\end{center}

The primary \cs\ contains all the data it needs to reply sensibly to
the following query:

\begin{code}{c}
?Ship&Lon>-41&Lat>30
\end{code}

However, a user might make a request like this:

\begin{code}{c}
?Time&Lon>-41&Lat>30
\end{code}

or

\begin{code}{c}
?Ship&Time>1Apr98&Time<1May98
\end{code}

In this case, the primary \cs\ does not contain all the information
necessary to respond to the request, and must seek the information it
needs from the subordinate \cs\ (\textbf{CS1}).  It does this by
stripping from the CE the clauses it can answer and passing them along
(if all the clauses it can answer are true).  The above two requests
to the primary \cs\ would become the following two requests to the
\textbf{CS1} server:

\begin{code}{c}
?Time
\end{code}

or

\begin{code}{c}
?&Time>1Apr98&Time<1May98
\end{code}

Note that the DODS constraint expression semantics would have to be
adjusted to make the second example here return a boolean value.

The data in the primary \cs\ and a subordinate \cs\ need not be
strictly complementary.  In the above example, there is no reason why
the \textbf{CS1} server could not also contain a value for the
``Ship'' field.  The only requirement is that the subordinate server
either be able to fill in the missing information needed by the
primary \cs , or be able to issue a subordinate query to another
server who can.

This structure will also allow direct examination of the \textbf{CS1}
server, which can be queried by date alone.  This would allow a local
\cs\ to be maintained, and be accessible to local users, while still
making the information available to users of the primary \cs .

\pagebreak
\subsection{Extending the Model}

Consider another table:

\begin{table}[h]
\begin{center}
\textbf{The Really Big Shebang} \\
\begin{tabular}{|l|l|l|l|l|l|l|l|l|l|} \hline
Platform&Time        &Lon   &Lat &Press &Temp &LonRange&LatRange&Image\\ \hline
Oceanus &1200-4Apr98 &-40.3 &34.4&P1A   &T1A  &        &        &     \\
Oceanus &1430-5Apr98 &-40.3 &34.4&P1B   &T1B  &        &        &     \\
Oceanus &1510-6Apr98 &-40.3 &34.4&P1C   &T1C  &        &        &     \\
Oceanus &2200-7Apr98 &-40.3 &34.4&P1D   &T1D  &        &        &     \\
Knorr   &2259-5Apr98 &-42.2 &32.3&P2    &T2   &        &        &     \\
AVHRR   &1400-4Apr98 &      &    &      &     &-77,-63 &34,45   &Img1 \\
AVHRR   &1800-4Apr98 &      &    &      &     &-77,-63 &34,45   &Img2 \\
AVHRR   &2200-4Apr98 &      &    &      &     &-77,-63 &34,45   &Img3 \\
AVHRR   &0200-5Apr98 &      &    &      &     &-77,-63 &34,45   &Img4 \\
AVHRR   &0600-5Apr98 &      &    &      &     &-77,-63 &34,45   &Img5 \\
AVHRR   &1000-5Apr98 &      &    &      &     &-77,-63 &34,45   &Img6 \\ \hline
\end{tabular}
\end{center}
\caption{A Table Including Multiple Datasets}
\label{fig:reltbl2}
\end{table}

This table could be broken up like this:

\begin{center}
\textbf{Primary \cs} \\
  \begin{tabular}{|l|l|l|l|} \hline
    Platform&LonRange&LatRange&URL\\ \hline
    Oceanus &-40.3,-40.3&34.4,34.4&CS1 \\
    Knorr   &           &         &CS2 \\
    AVHRR   &-77,-63    &34,45    &CS3 \\ \hline
  \end{tabular}

  \textbf{{\huge\strut}CS1} \\
  \begin{tabular}{|l|l|l|l|l|l|} \hline
    Time        &Lat   &Lon &Press &Temp &URL    \\ \hline
    1200-4Apr98 &-40.3 &34.4&      &     & URL1A \\
    1430-5Apr98 &-40.3 &34.4&      &     & URL1B \\
    1510-6Apr98 &-40.3 &34.4&      &     & URL1C \\
    2200-7Apr98 &-40.3 &34.4&      &     & URL1D \\ \hline
  \end{tabular}

  \textbf{{\huge\strut}CS2} \\
  \begin{tabular}{|l|l|l|l|l|l|} \hline
    Time        &Lat   &Lon &Press &Temp & URL   \\ \hline
    2259-5Apr98 &-42.2 &32.3&      &     & URL2  \\ \hline
  \end{tabular}

  \textbf{{\huge\strut}CS2} \\
  \begin{tabular}{|l|l|l|} \hline
    Time       &Img & URL   \\ \hline
   1400-4Apr98 &    & URL3A \\
   1800-4Apr98 &    & URL3A \\
   2200-4Apr98 &    & URL3A \\
   0200-5Apr98 &    & URL3A \\
   0600-5Apr98 &    & URL3A \\
   1000-5Apr98 &    & URL3A \\ \hline
  \end{tabular}
\end{center}

\noindent
The Img\emph{X}, P\emph{X}, and T\emph{X} variables are served by DODS
servers specified with the URL\emph{X} in the tables.

The following features are notable about the above example.

\begin{itemize}
\item To construct the (imaginary) table in \tableref{fig:reltbl2} you
  need not have headings for the empty columns in the primary \cs\ 
  tables.  If a \cs\ can make no sense of a clause in a constraint
  expression, it can simply pass it on to the next lower level.  The
\item Some data attributes, such as the spatial ranges in the
  \textbf{Primary \cs} can be considered optional.  That is, if they
  are present, they will make a data query simpler to answer.
  Consider a spatial query received by the \textbf{Primary \cs} in the
  above example.  That server has the information necessary to
  determine whether or not to make a query of the Oceanus or AVHRR
  servers, but not the Knorr.  A spatial query will, therefore,
  \emph{always} produce a subordinate query to the Knorr server, but
  will only generate a subordinate query to the Oceanus or AVHRR
  archives if there is likely to be data returned.
\item The proposed \cs\ structure can be used not only to address the
  problem of organizing multi-file datasets, but also to provide a way
  to organize a multi-dataset catalog.  After all, the problems of
  browsing a complex dataset and browsing a catalog of datasets are
  quite comparable.
\end{itemize}

This \cs\ structure is illustrated in \figureref{fig:cs-org}.  The \cs
s can be thought of as the branch nodes in a hierarchy of servers,
where the DODS data servers are the leaf nodes.  As in the figure, a
``naive'' client can direct its requests to the top level \cs .  The
``smart'' client, who knows many of the attributes served at the top
level, and directs its data queries at the lower level.  There can be
repetition of data among the layers; the only requirement is that
together, all the nodes create a unified table of values.

For example, consider the AVHRR archive.  Most of the dataset
attributes for the 16,000 data files are identical.  These would be
stored in a high-level \cs .  A user who is working every day with
this archive should be able to direct a catalog query to a \cs\ local
to the dataset server itself.  This user already knows most of the
dataset's attributes, and is only interested in the time and coverage
of any particular data file.  A user who is looking for satellite data
may not know all the attributes of the AVHRR dataset, so should have
access to the dataset attributes found in the higher-level \cs .

\begin{figure}[hp] 
\centerline{\psfig{figure=gui-figs/cs-org.ps}}
\caption{Structure of the \cs}
\label{fig:cs-org}
\end{figure}

There is only one version of the \cs\ software in this proposal;
though the database is distributed, each server installation is
essentially the same.  Of course, the database served by each server
is different, and might even be optimized with respect to different
variables.  That is, one \cs\ might have optimized searches on the
time variable, while another might optimize searches on spatial
variables.  However, the basic structure, represented by the set of
all possible queries and responses, remains constant across servers.
This means that client requests can be directed to any level of the
\cs\ structure.  However, as we have seen, a client is only guaranteed
to have access to the complete set of attributes for a given database
if it directs its request to the top level of the \cs\ structure.


\subsection{So What?}

The \cs\ software manages a relational database; several \cs s
together can form the distributed database DODS \cs\ structure
outlined in \figureref{fig:cs-org}.  However, the \emph{only}
assumptions made about the \cs\ software to satisfy the structure
defined here are these:

\begin{enumerate}
\item Dataset queries are made using DODS constraint expressions.
\item Responses to queries are made with a DODS data stream.
\item The \cs\ can edit constraint expressions, and make subordinate
  queries with an edited CE when it does not have enough information
  to answer a query.
\end{enumerate}

This means that the \cs\ code can be readily adapted from one of the
existing DODS servers that can deal with relational tables.  What must
be added is the module that can edit a constraint expression and
dereference a URL.\footnote{It may be that the existing DODS servers
  capable of handling relational tables are not the optimum solution.
  A better solution may be to use one of the many free relational
  DBMS systems available, and adapt it to use with DODS}

%Some datasets are organized into multiple files, while others occupy
%just a single file.  The rough correspondence between URLs and disk
%files gave rise to the concern that multiple-file datasets would
%``break'' the dataset-scanning capacity of DODS.  But even datasets
%contained in one disk file can have non-standard internal
%organizations.  Two different netCDF datasets containing arrays of
%some data variable might have the independent variables stored in two
%different orders, for example.
%

\subsection{What's in the \cs ?}
\label{sec:structure}

A \cs\ functions as a piece of a distributed relational database of
dataset ``attribute'' information.  This is the information typically
found in the DODS DAS.  (Unlike the DAS, however, the information in
the \cs\ can be computed.  That is, numbers are numbers and strings
are strings, unlike in a DODS DAS, where everything is a string.)  Each
entry in the \cs\ consists of dataset attributes, \emph{and} a URL to
which more detailed queries can be directed.

There is no fixed set of dataset attributes in each \cs\ record.
However, if an attribute required by a given query is not present in
a \cs\ record, a subordinate query must be issued to the corresponding
URL for the missing information.


\subsection{Problems with the \cs\ Structure}
\label{sec:prblems}

These are the known problems associated with the \cs\ structure
outlined in this chapter:

\begin{description}
\item{\bf Circular references}\\The \cs\ must be equipped to recognize
  when it is participating in a circular reference. 
  
  \textit{Solution}: The \cs\ could append its address to the CE when
  it makes a subordinate request.  When a \cs\ intends to make a
  subordinate query, it can compare the URL with the list of
  addresses.  If any match, the subordinate query would be skipped.  A
  query could also be submitted with a POST, which would allow this
  information to appear in the message body.
\item{\bf Duplicate references}\\Two \cs s might point to the same
  dataset, or to the same \cs .
  
  \textit{Solution}: The \cs\ must scan the list that results from a
  query to remove duplicate entries.  It is possible, however, that
  for a particular structure of datasets and a particular query, that
  two different datasets would produce identical entries in such a
  return list.
\item{\bf DODS Constraint Expression Syntax}\\The CE Syntax must be
  expanded to allow a List return when more than one value meets the
  CE criterion, and to provide for a boolean return value for CEs
  without a projection expression.

  Other possible extensions might include the server return address
  indicated above.
\item{\bf Data Volume}\\Considering the volume of data that may flow
  from a data request, a \cs\ must have a way to control its
  subordinate requests.  That is, if someone makes an AVHRR data query
  through the GCMD DODS server, the system can't allow megabytes of
  data to flow through the GCMD server.
  
  \textit{Solution}: By default, the result of a data query made
  through a \cs\ should be the list of URLs that \emph{would} have
  been dereferenced in that query.  In general, a client (either a
  user or another \cs ) should have a way to control whether a query
  can issue a subordinate query.  Like the return address idea above,
  this might take the form of a flag appended to the constraint
  expression.
\end{description}

\cbend

%
%\section{A Function Prototype}
%\label{sec:proto}
%
%From the point of view of a DODS \gui , a \cs\ can be considered a
%programming procedure, or function, with enumerable inputs and
%outputs.  The inputs would usually (though not exclusively) be
%constructed with the help of the \gui , which would also be available
%to display the output.  (As well as the subsequent look at the data
%itself.) 
%
%\subsection{Inputs}
%
%A query about datasets typically takes the form of a dataset name (or
%names) and a set of variables or dataset attributes to search for.
%Each variable may also have a data range.  For example, you might be
%only be interested in sites where the temperatures are above
%$29^{\circ}$ celsius, or only in data between two latitudes, or two
%dates.  In a search for datasets satisfying some criteria, the dataset
%name would be omitted.  (Its place might be taken by some other search
%criteria, like observing platform or principal scientist.)
%
%These queries are similar to the sorts of queries made of datasets
%themselves.  As was mentioned in section~\ref{sec:dbstruct} DODS has
%already developed a fairly sophisticated syntax for making these
%queries, the constraint expression.  With a small number of
%enhancements to the semantics of this form (see
%page~\pageref{sec:prblems}), the same expression can be used to query
%the \cs\ database as well.  This will give the \cs s great 
%
%Note that one may make a data request for some data range without
%asking for that variable.  That is, one might want all temperatures
%where the salinity is greater than 37 ppt.  This is not, however, a
%request the \cs\ can make any sense of, since it has only access to
%summary data (upper and lower bounds).  From data summaries alone, one
%can't tell which points have which combinations of values.  The
%esponses from a \cs\ are necessarily imperfect.  Only queries to the
%datasets themselves can produce complete responses.
%
%A query may refer to dataset attributes as well as to dataset
%variables.\footnote{In fact, inasmuch as dataset variables can be
%  considered to be attributes of the dataset, this amounts to the same
%  thing.  Some datasets reflect this hierarchy, which is designed into
%  the DAS structure, better than others.}  That is, one can request
%all data taken from a certain ship, for example, or all climatologies,
%\emph{if} there exists a consistently named attribute signalling ship
%names or climatologies.
%
%The input to the \cs\ can be written using the same constraint
%expression syntax as you might use to make a request to a DODS data
%server. 
%
%\subsection{Outputs}
%
%There are several different forms that could be taken by the return
%data from a \cs\ query, depending on the input constraint expression.
%What is important is that the return be a DODS data stream.  That is,
%whether the client is receiving data or attribute data, it should be
%transmitted using the DODS DAP.
%
%Some attention is due to the URL that may be transmitted for a
%dataset.  This attribute may take the form of a simple string:
%
%\begin{code}{cb}
%String URL "http://saci/cgi-bin/nph-nc/weekly.nc";
%\end{code}
%
%\noindent or it may be an interpreted function that must be
%executed in order to derive the URL(s).  The arguments of the function
%are taken from the query parameters.
%
%\begin{code}{cb}
%URL {
%   String Call "make_urls(query_time_min, query_time_max)";
%   String Def "List Structure make_urls(time_min, time_max) {
%                      ...
%   }";
%}
%\end{code}
%
%The client receiving this function must know how to execute it, so the
%query parameters must exist in a local name space with names
%standardized so that the URL function arguments can be properly
%specified. (See section~\ref{sec:functions}.)  It is not important how
%the function operates.  The function could be the encoding of an
%algorithm to create a file name from time values, for example, or it
%could just select from a list of file names.
%
%The return value of the URL function is a data structure that contains
%a list of URLs, and a list of the data variable values that identify
%them.  So, for example, a URL function might accept two times, and
%return a list of the URLs that fall between those two times.  That
%list would also have to be accompanied by a list of time values, one
%or each URL.  In addition, the data structure must somehow signify
%that the accompanying data is ``time.''
%
%A dataset's DAS received from a query to the \cs\ will differ from the
%DAS received from a query to the dataset itself only by the addition
%of the URL field, which is not required to be present in a ``native''
%DAS.
%
%The \cs\ returns an \emph{entire} DAS.  Even if the query range covers
%only a fraction of the DAS area, the whole DAS is sent back to the
%client.  The client is then free to modify the DAS before presenting
%it to the user.  That is, if a DAS describes a dataset temperature
%range as [-1.5, 29.6], and a query is for datasets whose range is
%between [10.0, 30.0], the server returns [-1.5, 29.6], and lets the
%\gui\ (or other client program) make the adjustment in the range to
%[10.0, 29.6].
%
%
%\section{Operating the \cs}
%\label{sec:operating}
%
%The \cs\ is run with a database consisting of a list of DODS Data
%Attribute Structures (DAS), slightly modified from the DAS stored at
%the site of the dataset itself.  The modification requires the
%addition of a URL field to the top-level attribute list,\tbd{You could
%allow URLs to appear at lower levels, as well.  For example, you might
%have a dataset where different variables occupied different files.
%What then?} and the conversion of units in the ranges to the DODS
%standards.  (See section~\ref{sec:units})
%
%The value of the URL field is either a string containing a single URL,
%or a function of the query arguments, to be evaluated by the
%client. (See section~\ref{sec:functions}.)
%
%The \cs\ returns the entire DAS for any dataset whose bounds overlap
%the query criteria.  The requesting program (generally the \gui ) is
%responsible for adjusting the bounds of the data variable attributes
%to accommodate the query criteria.
%
%\cbstart 
%
%An entry in the \cs\ database can consist of either a DAS or the URL
%of another DODS \cs .  If a database entry indicates another \cs , the
%first \cs\ forwards the search request to that subsidiary \cs\ and
%incorporates the return data into the data accumulated for the initial
%data request.
%
%This structure allows great flexibility in the number and structure of
%\cs s, and how DODS uses them, but it has some important implications:
%
%\begin{enumerate}
%\item A \cs\ must check the source of a request before searching its
%  database.  Otherwise a \cs\ could initiate a subordinate query to
%  a \cs , which could then initiate a subordinate query right back to
%  the first server.  A \cs\ must know when it is processing a query
%  for a client, and when it is processing a query for another \cs .
%  In the second case, there must be a record of all the servers that
%  have handled the request so far.
%\item The potential exists for creating serious network traffic
%  issues.  A client should be able to specify whether or not a \cs\
%  query should be subordinated to another server or not.
%\end{enumerate}
%
%\cbend

\cbstart

\section{Getting from Here to There}
\label{sec:getting}

The implementation of the \cs should proceed as follows:

\begin{enumerate}
\item Implement the \cs without redirection, but \emph{with}
  standardized attribute names.  This will allow the Matlab GUI to be
  rewritten to run without (or with substantially smaller) archive
  M-files.  This will require the creation of a \cs\ that might be
  larger than eventually will be necessary.
\item Implement redirection.  The \cs s can assume their proper size.
\end{enumerate}

\cbend

\chapter{Units, Names, and other Attributes}
\label{sec:attr}

The \cs\ design presented so far will work only insofar as the
attributes (DASs) of several different datasets are compatible with
one another.  Comparisons between one set of attributes and another
are not possible unless the two structures use comparable sets of
attributes, as well as compatible data variable names and units.

Here is a preliminary list of the essential attributes.  In the spirit
of trying to keep the burden on the data provider to a minimum, I have
tried to minimize the number of entries on the list.  They are
separated into a list of ``global'' attributes (apply to an entire
dataset) and a list for each data variable.

\begin{itemize}
\item Dataset name
\item Dataset description
\item URL
\end{itemize}

\noindent The following attributes must be present for each data variable in a
dataset: 

\begin{itemize}
\item Variable's DODS-standard name
\item Data range (This can be a minimum and maximum or an enumerated
  list of possibilities.)
\item Units
\item Unit Conversion Function (if necessary)
\item Period (if any) ``Granularity''
\end{itemize}

In addition, the data variable storage order must be recorded somehow.
For some data APIs (e.g. netCDF), this matters.  Therefore, the order
of the attributes in the DAS must correspond to the order of the
variables in the dataset.

\section{Names}
\label{sec:names}

The standardization of attributes also implies a concomitant
standardization of attribute and variable names.  To compare two
datasets, the simple attributes used in the comparisons must have
comparable names.  Similarly, compound (container) attributes must
have attributes in common.

Consider the following DAS:

\begin{code}{cb}
Attributes {
  String ShipName "Knorr";
  String ChiefScientist "Sally";

  salt {
    String DodsName "Ocean/Salinity";
  }

  temp {
    String DodsName "Ocean/Temperature";
  }
}
\end{code}

\noindent To compare the DAS above to the one below, it is simple 
for a computer to answer the question of whether the two datasets were
taken from the same ship, but more challenging to answer whether the
two ships were under the same scientist.

\begin{code}{cb}
Attributes {
  String ShipName "Endeavour";
  String PrincipalInvestigator "Peter";

  Salinity {
    String DodsName "Ocean/Salinity";
  }

  Temperature {
    String DodsName "Ocean/Temperature";
  }
}
\end{code}

\noindent On the other hand comparing salinities and temperatures 
between the two datasets is relatively simple, since in both datasets,
all the data variables are identified with a common attribute,
``DodsName.''  DODS needs standards for both metadata names \emph{and}
data variable names.

\subsection{GCMD Names}
\label{sec:gcmd-names}

The GCMD maintains a hierarchical list of valid keywords for earth
science data that would be compatible with use by the DODS project.
The names are somewhat unwieldy, but their hierarchical nature makes
clear several important differences between data variables that might
otherwise seem similar.  (e.g. Geologic time is distinguished from
Oceanographic time.)  Refer to
\htmladdnormallink{http://http://gcmd.gsfc.nasa.gov/cgi-bin/valids}{http://http://gcmd.gsfc.nasa.gov/cgi-bin/valids}.

The GCMD list has two important advantages to the DODS project.

\begin{itemize}
\item It is exhaustive, covering data ranging from Ocean Temperature to
  Poverty Indexes.
\item Someone else is maintaining it.
\end{itemize}

I suppose that a third advantage is that the NASA CAN seems to require
something related to the GCMD (this part is beyond my ken), and
adopting their standards will make it ever so much easier to cooperate
with them.

One major drawback is that the GCMD names are long and quite
unwieldy.  Using them will dictate the creation of a DAS authoring
tool, or else no one will spell the names without typos.


\section{Units}
\label{sec:units}

If DODS is to be useful, a user must perceive all the datasets in a
given \cs\ to have data in comparable units.  This does not mean that
all datasets must be kept in the same units.  In the same way that
DODS hides issues of data access API from the user, it can hide the
units issue.

Assume that DODS establishes a standard unit for each of the data
names in the GCMD catalog.\footnote{GCMD may already do this, but I'm
  not sure.}  Consider a data variable in any given dataset.  Either
it is stored in that unit or it is not.  If it is, there is no
problem, so long as this can be verified in the DAS.  If it is not,
the DAS must signal this, and supply a conversion function to convert
the data storage units into the DODS standard units.  The two
attribute containers below illustrate this.

\begin{code}{cb}
temp {
  String StorageUnit "degCelsius";
}

Temperature {
  String StorageUnit "Counts";
  String UnitConvert "C = convert(counts) {
    C = 24.55 + 133.26 * counts ;
  }";
}
\end{code}

This approach provides several advantages, and at least one
significant disadvantage.  The principal advantage is universality.
\emph{Any} unit conversion can be set up this way, even ones that
depend on other data.  The principal disadvantage is the limitations
placed on the formulation of constraint expressions.  This is
described below.

Many unit conversions are linear, so there is a temptation simply to
use the slope and offset instead of the entire function definition.
But several units in common use have non-linear conversions from other
common units.  One need only consider time to see the truth of this
(convert decimal years to julian days, for example), but other units,
like CTD machine units, also use non-linear conversions. \cbstart\
Since so many conversions are linear in nature, it would make sense to
have a ``cheat'' linear conversion method available, optimized for
this case.\cbend

The values stored as range boundaries in the \cs\ collection of DASs
must all be in the same units.  They are to be converted when the \cs\
database is compiled and updated.

This unit conversion scheme does not cover all the cases. Consider the
two cases of a data variable that is and is not stored in
DODS-standard units, and the two cases of a variable in a constraint
expression projection clause and a selection clause.  This makes the 
following four conditions:

\begin{center} 
\begin{tabular}{| c || c | c |} \hline 
 & Projection  & Selection \\ \hline \hline
DODS Units & A & B \\ 
non-DODS Units & C & D \\ \hline
\end{tabular} 
\end{center} 

In the ``DODS Units'' row, cases A and B, the dataset stores its units
in the DODS standard units for that data variable.  Therefore, there
is no conversion issue in selecting that variable, or constraining a
data query with it.  

For case C, the user will query the \cs , which is stored in
DODS-standard units.  The \cs\ provides a conversion function, which
the client keeps on hand.  When a query is made to that dataset and
variable, the stored conversion function can be used to transform the
incoming data points in memory.

\cbstart\ 
Case D, however, is not as readily subdued.  Here, the user wants to
make a query to the dataset (not the \cs ), selecting on a variable
not stored in DODS-standard units.  To do this in a way that doesn't
load the DODS server, the constraint expression must be converted
before being sent to the server.  

 Since the conversion functions are
stored as functions, they are not readily invertible.  In this
scenario, a user may have queried the \cs , and found some dataset
with interesting data.  When she then goes to formulate a query to the
dataset, the selection clause must be expressed in different units
than it was for the catalog query.  The \gui\ should alert the user to
the discrepancy, and provide the opportunity to use different values
for the variable.  Since many conversions are simple, the \gui\ could
provide guesses.
\cbend

Most instances of case D would be solved if the constraint expression
parser contained arithmetic capabilities.  

\subsection{Longitude}
\label{sec:longitude}

This quantity deserves a special mention since it is probably the only
periodic data type regularly encountered in earth-science datasets
that has more than one accepted coordinate system.\footnote{Calendar
  days are also periodic, for example, but everyone agrees (these
  days) that December 31 is always represented by a positive number.}

\section{Functions}
\label{sec:functions}

The concept of the client-executed interpreted functions deserves
amplification.\footnote{Plus, if it is decided that such a structure
  makes sense, choosing a language then becomes an issue.  The fact
  that Tcl is already embedded into DODS clients may make the choice
  simple or hard, depending on whether Tcl can be made to be
  appropriate to the task at hand.}  These functions would be
available to handle the construction of URLs, conversion of units,
creation of calculated data values (like density or potential
vorticity), and possibly even data type translation.

\subsubsection{Execution}

A function is not executed until (or unless) the user asks to see a
quantity whose value depends on that function.  For example, a URL
returned as a function will not be evaluated until the user asks to
see it.  A unit conversion function will not be evaluated until the
quantity that uses it is mapped or displayed.

\subsubsection{Environment}

The ``environment'' in which these functions are evaluated is made up
of the data and attributes of a dataset.  More technically, the
environment might simply be the DDS of a dataset, with data attributes
available through some extension of the `.' operator.  For example,
the long name (``DODS Name'') of a temperature variable called
\lit{temp} might be referenced as a string called
\lit{temp.DodsName}.\footnote{Since `.' is also used to reference
  Structure fields, another character (`:') might be less confusing.
  On the other hand, it isn't so bad to think of data attributes as
  members of a structure, so perhaps the consistency might be useful.}

The variables in the DDS would be the ``global'' variables in such a
space.  The DDS manipulation methods can be used to manage a
function's local variables.  The list of global variables also must
include the URL query parameters (components of the constraint
expression), so that URL-building functions can operate.

\cbstart
%\section{Local Caching of Attributes}
%\label{sec:cache}

%Local attributes would override those in the remote dataset.  They
%could be edited locally by the \gui\ or by hand for good or for evil.



%
%\section{Hierarchy of Attributes}
%\label{sec:hierarchy}
%
%Don't \emph{need}, but would be nice to have a hierarchy of
%attributes, so that, as you can currently define attributes for an
%entire dataset, you could define attributes for all temperatures, for
%all netCDF datasets, for all vectors, and so on.  This, combined with
%local attribute caching, would be a pretty cool thing.

\cbend

\chapter{GUI}
\label{sec:gui}

The DODS \gui\ allows a user graphic access to data retrieval
functions, as well as a limited amount of graphic data display
capacity.  The DODS \gui\ must allow a user to:

\begin{itemize}
\item Scan available remote DODS datasets, using information from one
  or more \cs s, displaying graphic information about the datasets
  and the data they contain.  The user should be able to control
  aspects of the display to answer questions about the datasets.
\item Import data from those datasets to the local machine.  The user
  should be able to control how that data gets imported, and whether
  it goes directly into an analysis package, or to a disk file.
\end{itemize}

The graphic display needs for the \gui\ are fairly limited.  Part of
the strength of the DODS analysis of science data usage is that its
designers made explicit the distinction between data retrieval and
data analysis.  DODS is a data retrieval system, \emph{not} an
analysis package, and it will be a mistake to try to emulate one (at
least until such time as the DODS team has the coding capacity of the
Mathworks, for example).

When using local data, the retrieval and analysis phases
are typically done by the same program.  Matlab, for example, has
``load'' functions to load data into the workspace, and a wealth of
other functions for the analysis.  However, though the same program
handles both phases, they are still distinct operations.  

\section{Relation to Analysis Software}
\label{sec:analysis}

The DODS \gui\ is a standalone application, capable of reading remote
DODS data, and writing it to a local disk file for later reading.
Alternatively, you can also run DODS from an analysis package, like
Matlab, IDL, Ferret, and so forth.  In this case, data can be written
to a file, or exported directly to the analysis package for display,
analysis, slicing, dicing or whatever.

Science data extracted from DODS datasets travels from the \gui\ 
to the analysis package when the user selects \but{File:Export} from
the menu bar.


\section{Relation to Catalog Server}
\label{sec:cat-rel}

This section describes the data sent back and forth from the \gui\ to
the \cs\ and back again.


\subsubsection{From the \gui\ to the \cs\ }

A data query uses the following data to specify a catalog search to be
done.

\begin{itemize}
\item The dependent variable name (or names) to search for.
\item The dataset name (or names) to search in.
\item The variable names to search with.  These independent variables
  define the bounding box of the search.  Note that these don't have
  to be geographic boundaries, just boundaries in the N-space defined
  by the dataset.
\item The values of the bounding box.  There should be a range
  specified for each of the variables in the independent variable
  list. 
\item The desired resolution, for searching gridded data.
\end{itemize}

\subsubsection{From the \cs\ to the \gui\ }

The results of a data query are given in a set of DODSS objects
returned from the \cs .  Conceptually speaking, the results are a list
of dataset URLs and descriptions of the data found by dereferencing
those URLs.

%More detail is provided in section~\ref{sec:proto}.

Some refinements of the data query can be done locally, with the data
returned by the \cs , without making another query to the \cs .  For
example, a user can eliminate a URL from the list of returned URLs.
For many other modifications, you must make another query to the \cs\ 
.

The \gui\ must be able to amalgamate the results of several dataset
queries so that a user can export to the analysis package the results
of several different queries.  This means that successive queries must
be able to be merged into a single data space.




\section{Display Capacity}
\label{sec:display}

the DODS \gui\ must be able to display simple two- and
three-dimensional graphs of input data, allowing a user to choose
among whatever number of data variables is returned by a data
request.  The graph types supported are:

\begin{itemize}
\item Scatter plots and Line graphs
\item Contour and pseudo-color maps
\end{itemize}

\section{Main Window}
\label{sec:main-window}

The main window of the \gui\ is shown in
\figureref{fig:gui-main-window}.  It contains the following areas:

\begin{description}
\item[Menu Bar] The menu bar, running along the top of the window,
  contains the pull-down menus used to control the \gui .
\item[Shortcut Buttons] A few of the most commonly used functions can
  be executed with shortcut buttons displayed in a column on the left
  of the main window.
\item[Scale] The bottom of the column on the left of the main window
  is used for a data scale.  In the case of a color plot, this is
  where the color scale is displayed.  In the case of a scatter plot
  or line graph, the point shapes and line weights are explained
  here. 
\item[Data Viewport] The data viewport contains windows used to
  specify data constraints, and one window used to display returned
  data.  
\end{description}

\begin{figure}[htbp] 
\centerline{\psfig{figure=gui-figs/gui-main.ps}}
\caption{The \gui\ Main Window}
\label{fig:gui-main-window}
\end{figure}

\noindent Most of these window parts will be familiar to any user who 
has used other UNIX or Windows graphical interfaces.  The Data
Viewport demands somewhat more explanation.

\subsubsection{Data Viewport}
\label{sec:window:viewport}

The Data Viewport contains a collection of windows used to specify
data constraints and to display data.  The viewport contains one
2-dimensional window and an arbitrary number of one-dimensional
windows.  The number of windows and the properties of each window are
controlled by the \but{Display:Variables} dialog box.  

The one-dimensional windows, or display scales, allow you to set a
range for the indicated variable.  Use the \but{Display:Set Data
  Range} to set the range with the mouse.  You can also set a range
with the \but{Query:Constraint} dialog, in which case, the range will
be displayed on the appropriate display scale.

The two-dimensional window is used both to set a range and to display
some of the returned data.  The returned variable can be displayed on
one of the window axes, in which case the data is displayed as a line
or scatter graph. If neither axis corresponds to the displayed.

Returned data is displayed in the 2-d window.  The two figures nearby
show the data viewport in two different states.  The first,
figure~\ref{fig:gui-main-window}, shows a view of a standard map,
while the second, figure~\ref{fig:gui-main-window-alt} shows a
slightly less orthodox, but still possible useful, view of latitude in
Temperature-Salinity space, bounded by longitude values to include
(roughly) the Atlantic.

\begin{figure}[htbp] 
\centerline{\psfig{figure=gui-figs/gui-main-alt.ps}}
\caption{The \gui\ Main Window, Another State}
\label{fig:gui-main-window-alt}
\end{figure}

\section{Using the GUI}
\label{sec:using-GUI}

Several steps:

\begin{enumerate}
\item Describe data desired, including variables and limits.
  \begin{enumerate}
  \item Describe variables desired.
  \item Describe limits
  \end{enumerate}
\item Find datasets with that data. (This is done implicitly with the
  modal Datasets and Variables menus in the Matlab \gui .)
\item Select a specific dataset.
\item Request data from that dataset.
\item Refine the request.
\item Import data into analysis package for further display.
\end{enumerate}


\subsection{Modes}
\label{sec:modes}

Several items may be selected by a user.  These \emph{settings} define
the \gui\ modes.  Several \gui\ functions depend on these setting.
The following settings have no default values, and functions dependent
on them should be grayed out until the selections are made.

\begin{itemize}
\item The dataset to be queried.
\item The variables to be retrieved.
\end{itemize}

\noindent The following settings have default values:

\begin{itemize}
\item The variables used as query criteria. (Default: geographic, plus
  time)
\item The resolution used when searching for gridded data. (Default:
  whatever corresponds to the stride value of one.)
\item The variables to be displayed. (Default: the first variable in
  the ``projection'' list.)
\item The query criteria. (Default: None.  The default gets all the
  data in the specified dataset.)
\end{itemize}


\subsubsection{Data Selection Modes}

There are two data selection modes: ``selection'' and
``subselection.''  The ``selection'' mode is used to specify a data
query, and allows you to set the necessary parameters to submit a data
query search to the \cs .

The ``subselection'' mode is used to refine a previously-made
selection.  When you refine a subselection, the query response can be
constructed from data already received.  You do not need to issue
another request to the \cs .

\subsection{Menu Reference}
\label{sec:menu}

The \gui\ menus are described below.

\subsubsection{File}
\label{sec:menu:file}
The File menu contains functions related to the overall \gui\
session.  It controls the connection to the \cs\ , and the
relationship with the analysis package in use.
\begin{description}
\item[]
  \begin{center}
    \begin{tabular}[c]{|l|} \hline
      \textbf{File} \\ \hline
      Open \\
      Save Data \\
      Save Session \\
      Export \\
      Import \\
      Print \\
      Quit \\ \hline
    \end{tabular}
  \end{center}
  
  \begin{description}
  \item[Open] Open a connection to a new \cs\ .
  \item[Save Data] Saves the data currently displayed in the \gui\ in
    a disk file.
  \item[Save Session] Save a record of the \gui\ session, so that
    interrupted work can be easily resumed.
  \item[Export] Sends data to an analysis package.
  \item[Import] Receives data from an analysis package.
  \item[Merge Data] Merge the results from two or more data queries
    so that it can be mapped or exported all at once.
  \item[Print] Creates a readable record of the current session,
    including the datasets queried and result summaries.
  \item[Clear All] Flush all the \gui\ memory.  (Equivalent to
    restarting it.)
  \item[Quit] Leave the \gui .
  \end{description}
\end{description}

\subsubsection{View}
\label{sec:menu:View}
The View menu contains functions used to control various
aspects of what is displayed.  This controls such aspects as the
dataset from which data is sought, and the variables requested, as
well as whether the window contains the text description of the
dataset or the graphic depiction of the request.
\begin{description}
\item[] 
  \begin{center}
    \begin{tabular}[c]{|l|} \hline
      \textbf{View} \\ \hline
      Variables \\
      Datasets \\
      Text \\ \hline
    \end{tabular}
  \end{center}
  \begin{description}
  \item[Variables] Displays a window containing the list of variables
    available in the currently selected datasets. This window is also
    available through a \but{Variables} button on the \gui\ main
    display.  You use this window to select the variables you want
    returned from your dataset query.
  \item[Datasets] Displays a window containing the list of datasets
    that may satisfy the search criteria.  This includes the list of
    desired variables, and whatever query conditions (e.g. location)
    are already set.  This window is also available through a
    \but{Datasets} button on the \gui\ main display.\footnote{The
      current \gui\ filters the dataset and variable lists on the fly,
      whenever a condition is added to the query.  You can leave these
      windows exposed and watch them change as conditions are added.
      This may not be appropriate for a substantially larger, or
      hierarchically organized list of datasets, where filtering the
      list may require substantial processing time.}
  \item[Text] Displays a window with the narrative information about
    the selected dataset, if any.
  \end{description}
\end{description}

\subsubsection{Query}
\label{sec:menu:query}
The Query menu controls the various query parameters used in a
data request.  The \but{Constraints} button controls the selection
clause of a constraint expression.  The projection clause is
controlled with the \but{View:Variables} menu button.
\begin{description}
\item[] 
  \begin{center}
    \begin{tabular}[c]{|l|} \hline
      \textbf{Query} \\ \hline
      Set Data Range \\
      Constraints \\ 
      Resolution \\
      Mode \\ \hline
    \end{tabular}
  \end{center}
  \begin{description}
  \item[Set Data Range] Use this to set a range in any of the display
    scales on the \gui\ main window.  There is also a \but{Set Data
    Range} button on the main window.
  \item[Constraints] Brings up a dialog box that allows you to select
    from a list of variables in the dataset those to use for the
    query. The same dialog box allows you to specify ranges for the
    variables.  You can use this dialog to add constraints to a data
    request beyond what can be specified in the \gui\ main window.
  \item[Resolution] Sets the desired resolution for requesting grid
    data. 
  \item[Mode] Allows you to toggle back and forth between selection
    modes.  ``Selection'' mode is for describing a request to be made
    to the \cs , while ``subselection'' mode is for refining a
    previously made request.  This allows you to weed out unwanted
    data without incurring network transmission overhead.
  \end{description}
\end{description}

\subsubsection{Get}
\label{sec:menu:get}
Once a dataset has been selected, and a query set up, these
functions are used to initiate the network requests.  The \gui\ 
makes requests both to \cs s and directly to data servers.
\begin{description}
\item[] 
  \begin{center}
    \begin{tabular}[c]{|l|} \hline
      \textbf{Get} \\ \hline
      Details \\
      Data \\ 
      Dataset List \\ \hline
    \end{tabular}
  \end{center}
  \begin{description}
  \item[Details] Retrieves summary information about the selected
    dataset with the given constraint.  This information includes the
    number of datapoints in the selection, and an estimate of the size
    of the returned file.  This request goes to the \cs\ .
  \item[Data] Retrieves the data from the selected dataset.  This is
    grayed out until a dataset and variables to retrieve are
    selected.  This request goes to the data server itself, although
    the \cs\ may be queried (if it hasn't been already) to determine
    the server's URL.
  \item[Dataset List] Queries the specified \cs\ (See
    \but{File:Open}) for an updated list of the available datasets.
  \end{description}
\end{description}

\subsubsection{Display}
\label{sec:menu:display}
This menu contains several functions that controls how the \gui\ 
displays its data, and what is shown in the main window.  These
functions do not control the data itself, which is done through the
\but{View} and \but{Query} menus.  Some display options that don't
have anything to do with the data are under the \but{Options} menu.
\begin{description}
\item[] 
  \begin{center}
    \begin{tabular}[c]{|l|} \hline
      \textbf{Display} \\ \hline
      Variables \\
      Zoom \\
      Data Display \\
      Data Colors \\
      Reset \\
      Clear \\ \hline
    \end{tabular}
  \end{center}
  \begin{description}
  \item[Variables] This brings up a dialog box allowing the user to
    specify:
    \begin{enumerate}
    \item The dependent variable to graph.  If there is only one
      variable selected for the data query, (See
      \but{View:Variables}), that variable will be graphed.
    \item The independent variables to display.  The user can indicate
      on the scales displayed in the main window constraints to add to
      the data query.  Use this dialog box to specify the limits of
      the variable scales, and the format of their labels.
    \end{enumerate}
  \item[Zoom] Use this to zoom in closer to one of the display scales
    on the \gui\ main window.  There is also a \but{Zoom} button on
    the main window.
  \item[Data Display] Use this to open a dialog box to select the type
    of graph to be displayed (if any).  The choices are:
    \begin{description}
    \item[Line] 
    \item[Scatter] 
    \item[Color] 
    \end{description}
  \item[Data Colors] Specify the color scale for a pseudo-color plot.
    This dialog box controls both the color scale and the limits.
  \item[Reset] Redraw the current main window, and all the plots in it.
  \item[Clear] Clear the plots in the current main window.
  \end{description}
\end{description}


\subsubsection{Options}
\label{sec:menu:options}
The functions in this menu are used to set communications and
display options.
\begin{description}
\item[] 
  \begin{center}
    \begin{tabular}[c]{|l|} \hline
      \textbf{Options} \\ \hline
      Data Size Threshold \\
      Colors \\
      Fonts \\
      Save Options \\ \hline
    \end{tabular}
  \end{center}
  \begin{description}
  \item[Data Size Threshold]  Data requests estimated to result in
    network transfers greater than this amount cause the \gui\ to
    prompt the user to be sure that the request is intentional.
  \item[Colors] Sets the colors of various parts of the \gui\ main
    window. 
  \item[Fonts]  Control the fonts used as labels in the \gui\ main
    window. 
  \item[Save Options] Saves the currently selected options (including
    display options set in the \but{Display} menu) in a disk file.
    These options will be read when the \gui\ starts up next time.
  \end{description}
\end{description}


\subsubsection{Help}
\label{sec:menu:help}
Displays documentation for the \gui\ and DODS.
\begin{description}
\item[] 
  \begin{center}
    \begin{tabular}[c]{|l|} \hline
      \textbf{Help} \\ \hline
      Contents \\
      Index \\
      About DODS \\
      DODS Web Page \\ \hline
    \end{tabular}
  \end{center}
  \begin{description}
  \item[Contents] The \gui\ documentation table of contents.
  \item[Index] The \gui\ documentation index.
  \item[About DODS] Whatever.
  \item[DODS Web Page] Starts a web browser pointing to the DODS home
    page. 
  \end{description}
\end{description}

\chapter{A Specification}
\label{sec:spec}

Conclusion
\bigspace






\end{document}
