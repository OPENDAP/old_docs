
% Architecture for the DODS data delivery mechanism. 
%
% $Id$

\documentstyle[12pt,psfig,html,margins]{article}

\psfigurepath{arch-figs}
%
% These are html links which are used often enough in writing about DODS to
% merit an input file.
% jhrg. 4/17/94
%
% File rationalized and updated while writing the DODS User
% Guide. Also includes other useful abbreviations.
% tomfool 3/15/96
%
% $Id$
%
% HTML references to DODS documents
%
% For most of the DODS documentation there are three html references defined:
% 1) The upper case \newcommand produces the title of the paper, an
%    html link in the online documentation and a footnote in the paper
%    version.  
% 2) A capitalized version produces a capitalized description of the
%    paper and a link in the online version (but no footnote in the
%    paper version). 
% 3) A lower case version which produces a lower case description of
%    the paper and a html link, but no footnote.

% External references to documents:
%
% In order for external references to work (via latex2html) the labels used
% throughout the documents must be unique. The text labels are all prefixed
% with a document identifier (e.g., ddd) with a colon separater. The
% \externalrefs below (way below...) includes the perl files html.sty uses to
% make the cross refs.

\newcommand{\DODS}{\htmladdnormallink{Distributed Oceanographic Data System}
  {http://www.unidata.ucar.edu/packages/dods/}}

\newcommand{\Dods}{\htmladdnormallink{DODS}{http://www.unidata.ucar.edu/packages/dods/}}

\newcommand{\wrkshp}{\htmladdnormallink{Report on the First Workshop for the
    Distributed Oceanographic Data System, Proposed System Architectures}
  {http://www.unidata.ucar.edu/packages/dods/archive/reports/workshop1/section3.13.html}}

% not DODS URLs, but useful all the same...

\newcommand{\www}{\htmladdnormallink{World Wide Web} 
  {http://www.w3.org/hypertext/WWW/TheProject.html}}
\newcommand{\url}{\htmladdnormallink{Uniform Resource Locator}
  {http://www.w3.org/hypertext/WWW/Addressing/URL/url-spec.html}}

\newcommand{\uri}{\htmladdnormallinkfoot{Uniform Resource Identifiers}
  {http://www.w3.org/hypertext/WWW/Addressing/URL/uri-spec.html}}

\newcommand{\urn}{\htmladdnormallinkfoot{Uniform Resource Name}
  {http://www.acl.lanl.gov/URI/archive/uri-archive.messages/1143.html}}

\newcommand{\HTTPD}{\htmladdnormallinkfoot{HTTPD}
  {http://hoohoo.ncsa.uiuc.edu/docs/Overview.html}}

\newcommand{\HTTP}{\htmladdnormallinkfoot{HyperText Transfer Protocol}
  {http://www.w3.org/hypertext/WWW/Protocols/HTTP/HTTP2.html}}

%\newcommand{\HTML}{\htmladdnormallinkfoot{HyperText Markup Language}
%  {http://www.w3.org/hypertext/WWW/MarkUp/MarkUp.html}}

% CERN used to be `Conseil Europeen pour la Recherche Nucleaire'
\newcommand{\CERN}
  {\htmladdnormallink{European Laboratory for Particle Physics}
  {http://www.cern.ch/}}

\newcommand{\NCSA}
  {\htmladdnormallink{National Center for Supercomputing Applications}
  {http://www.ncsa.uiuc.edu/}}

\newcommand{\WWWC}{\htmladdnormallink{World Widw Web Consortium}
  {http://www.w3.org/}}

\newcommand{\CGI}{\htmladdnormallinkfoot{Common Gateway Interface}
  {http://hoohoo.ncsa.uiuc.edu/cgi/overview.html}}

\newcommand{\MIME}{\htmladdnormallink{Multipurpose Internet Mail Extensions}
  {http://www.cis.ohio-state.edu/htbin/rfc/rfc1590.html}}

\newcommand{\netcdf}{\htmladdnormallink{NetCDF}
  {http://www.unidata.ucar.edu/packages/netcdf/guide.txn_toc.html}}

\newcommand{\JGOFS}{\htmladdnormallink{Joint Geophysical Ocean Flux Study}
  {http://www1.whoi.edu/jgofs.html}}

\newcommand{\jgofs}{\htmladdnormallink{JGOFS}
  {http://www1.whoi.edu/jgofs.html}}

\newcommand{\hdf}{\htmladdnormallink{HDF}
  {http://www.ncsa.uiuc.edu/SDG/Software/HDF/HDFIntro.html}}

\newcommand{\Cpp}{{\rm {\small C}\raise.5ex\hbox{\footnotesize ++}}}

% Commands

\newcommand{\pslink}[1]{\small
\begin{quote}
  A \htmladdnormallink{postscript}{#1} version of this document is
  available.  You may also use \htmladdnormallink{anonymous
  ftp}{ftp://ftp.unidata.ucar.edu/pub/dods/ps-docs/} to access postscript files
  of all of the DODS documentation.
\end{quote}
\normalsize
}

\newcommand{\declaration}[1]{\small
{\tt {#1}}
\normalsize
}

% All the links from here down are for the white papers. I'm not sure that any
% of these are still valid given the reorganization of the web site. 5/12/98
% jhrg.

\newcommand{\DDA}{\htmladdnormallinkfoot{DODS---Data Delivery Architecture}
  {http://www.unidata.ucar.edu/packages/dods/archive/design/data-delivery-arch/data-delivery-arch.html}}
\newcommand{\Dda}{\htmladdnormallink{Data Delivery Architecture}
  {http://www.unidata.ucar.edu/packages/dods/archive/design/data-delivery-arch/data-delivery-arch.html}}
\newcommand{\dda}{\htmladdnormallink{data delivery architecture}
  {http://www.unidata.ucar.edu/packages/dods/archive/design/data-delivery-arch/data-delivery-arch.html}}

\newcommand{\DDD}{\htmladdnormallinkfoot{DODS---Data Delivery Design}
  {http://www.unidata.ucar.edu/packages/dods/archive/design/data-delivery-design/data-delivery-design.html}}
\newcommand{\Ddd}{\htmladdnormallink{Data Delivery Design}
  {http://www.unidata.ucar.edu/packages/dods/archive/design/data-delivery-design/data-delivery-design.html}}
\newcommand{\ddd}{\htmladdnormallink{data delivery design}
  {http://www.unidata.ucar.edu/packages/dods/archive/design/data-delivery-design/data-delivery-design.html}}

\newcommand{\URL}{\htmladdnormallinkfoot{DODS---Uniform Resource Locator}
  {http://www.unidata.ucar.edu/packages/dods/archive/design/urls/urls.html}}
\newcommand{\Url}{\htmladdnormallink{Uniform Resource Locator}
  {http://www.unidata.ucar.edu/packages/dods/archive/design/urls/urls.html}}
\newcommand{\dodsurl}{\htmladdnormallink{uniform resource locator}
  {http://www.unidata.ucar.edu/packages/dods/archive/design/urls/urls.html}}

\newcommand{\DAP}{\htmladdnormallinkfoot{DODS---Data Access Protocol}
  {http://www.unidata.ucar.edu/packages/dods/archive/design/api/api.html}}
\newcommand{\Dap}{\htmladdnormallink{Data Access Protocol}
  {http://www.unidata.ucar.edu/packages/dods/archive/design/api/api.html}}
\newcommand{\dap}{\htmladdnormallink{data access protocol}
  {http://www.unidata.ucar.edu/packages/dods/archive/design/api/api.html}}

\newcommand{\SOFT}
        {\htmladdnormallinkfoot{DODS---Software Development Environment}
  {http://www.unidata.ucar.edu/packages/dods/archive/managment/software/software.html}}
\newcommand{\Soft}{\htmladdnormallink{Software Development Environment}
  {http://www.unidata.ucar.edu/packages/dods/archive/managment/software/software.html}}
\newcommand{\soft}{\htmladdnormallinkfoot{software development environment}
  {http://www.unidata.ucar.edu/packages/dods/archive/managment/software/software.html}}

% I used to call the DAP the API, then for a few days I called it the
% DTP. These def's were used then. Rather than find everywhere they are used
% (not that hard, but...) I just hacked the macros. NB: the source file for
% the DAP document is still called `api.tex'

\newcommand{\API}{\htmladdnormallinkfoot{DODS---Data Access Protocol}
  {http://www.unidata.ucar.edu/packages/dods/archive/design/api/api.html}}
\newcommand{\api}{\htmladdnormallink{data access protocol}
  {http://www.unidata.ucar.edu/packages/dods/archive/design/api/api.html}}

\newcommand{\DTP}{\htmladdnormallinkfoot{DODS---Data Access Protocol}
  {http://www.unidata.ucar.edu/packages/dods/archive/design/api/api.html}}
\newcommand{\dtp}{\htmladdnormallink{data access protocol}
  {http://www.unidata.ucar.edu/packages/dods/archive/design/api/api.html}}

\newcommand{\TOOLKIT}{\htmladdnormallinkfoot{DODS---Client and Server Toolkit}
  {http://www.unidata.ucar.edu/packages/dods/archive/implementation/toolkits/toolkits.html}}
\newcommand{\Toolkit}{\htmladdnormallink{Client and Server Toolkit}
  {http://www.unidata.ucar.edu/packages/dods/archive/implementation/toolkits/toolkits.html}}
\newcommand{\toolkit}{\htmladdnormallink{client and server toolkit}
  {http://www.unidata.ucar.edu/packages/dods/archive/implementation/toolkits/toolkits.html}}

\newcommand{\TKR}{\htmladdnormallinkfoot{DODS---DAP Toolkit Reference}
  {http://www.unidata.ucar.edu/packages/dods/archive/implementation/toolkits/toolkits.html}}
\newcommand{\Tkr}{\htmladdnormallink{DAP Toolkit Reference}
  {http://www.unidata.ucar.edu/packages/dods/archive/implementation/toolkits/toolkits.html}}
\newcommand{\tkr}{\htmladdnormallink{DAP toolkit reference}
  {http://www.unidata.ucar.edu/packages/dods/archive/implementation/toolkits/toolkits.html}}

% I know that these are wrong. The papers have been moved to the gso web
% site, but I'm not sure exactly where. 5/12/98 jhrg

\newcommand{\DM}{\htmladdnormallinkfoot{DODS---Data Model}
  {http://lake.mit.edu/dods-dir/dodsdm2.html}}
\newcommand{\Dm}{\htmladdnormallink{Data Model}
  {http://lake.mit.edu/dods-dir/dodsdm2.html}}
\newcommand{\dm}{\htmladdnormallink{data model}
  {http://lake.mit.edu/dods-dir/dodsdm2.html}}

\newcommand{\DD}{\htmladdnormallinkfoot{DODS---Data Delivery}
  {http://lake.mit.edu/dods-dir/dods-dd.html}}

% external refs for DODS documents

\externallabels{http://www.unidata.ucar.edu/packages/dods/design/api}
  {/home/www/packages/dods/archive/design/api/labels.pl}
\externallabels{http://www.unidata.ucar.edu/packages/dods/design/data-delivery-arch}
  {/home/www/packages/dods/archive/design/data-delivery-arch/labels.pl}
\externallabels{http://www.unidata.ucar.edu/packages/dods/design/data-delivery-design}
  {/home/www/packages/dods/archive/design/data-delivery-design/labels.pl}
\externallabels{http://www.unidata.ucar.edu/packages/dods/implementation/toolkits}
  {/home/www/packages/dods/archive/implementation/toolkits/labels.pl}

%\renewcommand{\externalref}[2]{#2}

%%% Local Variables: 
%%% mode: latex
%%% TeX-master: t
%%% End: 


\begin{document}

\title{DODS---Data Delivery Architecture}
\author{}
\date{23 August 1996} % from URI web version {\today}

\maketitle

\begin{abstract}

  This paper describes the architecture of the DODS data delivery mechanism.
  This mechanism is one of two components that make up the DODS system
  design, the other being the data locator. In this paper it is assumed that
  every data set is accessible using an application programmer interface
  (API). In addition to local data set access, the DODS data delivery
  architecture provides remote access by replacing existing API
  implementations (which provide only local access) with new implementations
  which access data through servers located on the Internet. For each API
  DODS supports, a client library and data server are created. The client
  library supports the same functions as the original implementation of the
  API, but uses a DODS data server to satisfy each function call rather than
  reading data from a file. The client library and data server communicate
  using an application level transmission protocol that facilitates
  translation between the supported APIs.  While the data delivery
  architecture of DODS is not inherently limited to the UNIX operating
  system, the DODS project has no plan to support other operating systems. It
  is also important to note that this architecture describes a method for
  direct access to scientific data.

\end{abstract}


% Biolerplate text warning that anything can (and almost certainly will)
% change. 
%
% jhrg 3/25/94
%
% $Id$

\small

\centerline{{\bf Note}} 

\begin{quotation}

\begin{htmlonly}
  This is a working document made available to solicit comments. To receive
  e-mail notice of new or significantly changed documents, send e-mail
  to \htmladdnormallink{majordomo@unidata.ucar.edu}
  {mailto:majordomo@unidata.ucar.edu} with 'subscribe dods' in the
  body of the message.
\end{htmlonly}

\begin{latexonly}
  This is a working document made available to solicit comments. To receive
  e-mail notice of new or significantly changed documents, send e-mail
  to majordomo@unidata.ucar.edu with 'subscribe dods' in the body of
  the message. 

  Hypertext references in the hardcopy version of this document are
  represented using footnotes that contain a Universal Resource Locator (URL)
  to the referenced document.  To read the online documentation, open the URL
  `http://www.unidata.ucar.edu/' with a World Wide Web browser.
\end{latexonly}

\end{quotation}

\normalsize

%
% Biolerplate text expaining that this document is intended for programmers.
%
% jhrg 4/17/94
%
% $Id$

\small

\begin{quotation}
  This document is written for people interested in implementing software
  that will be part of, or work in parallel with, the Distributed
  Oceanographic Data System. 
\end{quotation}

\normalsize

\begin{htmlonly}
\pslink{ftp://ftp.unidata.ucar.edu/pub/dods/ps-docs/data-delivery-arch.ps}
\end{htmlonly}

\clearpage

\tableofcontents

\clearpage

\section{Introduction}

This document is an introduction to the DODS data delivery system. It
presents an architectural view of the system and provides rationale for some
of the important decisions that lead up to this design. This document is not
a detailed description of the design; that can be found in \DDD\ and related
documents.

The data delivery system uses the client-server paradigm to provide access to
remote data sets. On each computer which hosts a data set, a data server must
be installed. Client programs which communicate with these servers can be
built either from scratch or from existing code using reimplementations of
commonly used data access Application Programmer Interfaces (API)\footnote{As
of 23 August 1996, % was \today
DODS supports two APIs: \netcdf\ and \jgofs.}. The API libraries
are reimplemented so that existing code can be re-linked {\em without\/}
modification. Thus the data delivery system provides a pathway for existing
code to migrate from local access to remote access without requiring
modification of that code.

DODS addresses access to on-line data only. It would be nice to provide
access to data that are not already on-line, but the problems associated with
that are outside the scope of DODS (e.g., some data collections are not
suited to on-line access because of their size or content). Thus, the data
delivery architecture addresses the problem of network access to data sets
that are already on-line.

Before describing the data delivery system, some important computer
networking and data management terms and concepts are discussed. Each of
these is first defined in general terms and then particular aspects of those
ideas that are important to DODS are discussed. These include networking
technologies (remote procedure calls and the hypertext transfer protocol) and
data access tools and concepts (application programmer interface, data model
and data format).

\section{Important Terms and Technologies}

Some important concepts appear frequently in the discussions about the DODS
data delivery system. In this section they are defined or described so that
readers will all have some common background when reading the documents that
are part of the data delivery system.

\subsection{Application Programming Interface---API}

An application program interface (API) is a formal method of access to a
computational resource such as a disk file, set of disk files or a device
such as the display screen.

% where should this paragraph go?
%
%In this paper those operations may be referred to as {\em API functions\/} or
%{\em API entry points\/} and the resources are data. The implementation of
%any given API that is the standard implementation of that API (e.g., the HDF
%library from NCSA) is referred to as the {\em library\/} or just the {\em
%  API\/} depending on context. When the documentation refers to the DODS
%implementation of an API it uses {\em surrogate library\/} or {\em client
%  library\/}.

The DODS data delivery architecture addresses access to data stored on-line
on computer systems. In this paper we assume that an API is used to access
data stored on a random access storage device such as a disk drive.  However,
it may be the case that some data are not actually accessed via disk files
but through a more complex mechanism such as specialized database hardware
systems. None-the-less, the architecture of the data delivery design is still
based on the {\em concept\/} of access to data through an API, stored in disk
files.

A specific API implementation consists of a set of operations used to access
an instance of a well defined resource.  The set of operations are typically
functions such as open, read, etc., that the system uses to provide access to
the resource.  API operations are made available to the application program
usually as function calls in a software library. Figure~\ref{fig:stand-alone}
shows a user program linked with an API that is used to access some data.

As is shown in Figure~\ref{fig:stand-alone}, an API consists of an external
interface and some internal implementation. The external interface is used by
the application program. This external interface is assumed to be very
constant throughout the life of the API\@. User programs can expect that future
implementations of the API may improve performance or provide additional
features, but will still present the original set of function calls, possibly
with some additions over time.

By basing data access on existing APIs, DODS is reusing software
systems for which stability and longevity were principle design goals.

\begin{figure}
\centerline{\psfig{figure=stand-alone3.ps,width=4.5in}}
\caption{Architecture of a Simple User Program---An API is Used to Access
  Data in a Disk File.}
\label{fig:stand-alone}
\end{figure}

\subsection{Remote Procedure Call---RPC} 
\label{dda-rpc}

Remote procedure calls (RPCs) provide a framework for implementing remote
access to a system.  They create a distributed computing environment that is
established and controlled at the procedure level within an application. 

In an earlier version of DODS, RPCs were the basis for all data transmission;
the data delivery design has since changed to HTTP as the foundation for data
transmission. Regardless, RPCs still merit discussion since they frame the
concepts that are central to the design of the data delivery component of
DODS\@.

RPC technology simplifies the design of distributed software systems by
providing a means to separate one program into two cooperating processes
using a procedural interface. This is straightforward because many programs
can naturally be broken into two sections along a procedural line separating
a core set of functions and some additional functions special to a particular
application. Figure~\ref{fig:prog2client-server} shows how RPCs can be used
to split a user program and API into a client program and remote server.  The
RPC client and server `stubs' now form the communication interface between
the application program and server. The client stubs encode their arguments
and sends them over the network to matching stubs in the server. The server
stubs extract the arguments from the network and compute results based on the
arguments. The server stubs return to the client the results of the
computations.

RPC client stubs can serve as replacements for the API functions invoked by
an application program. In this case the function of the networking portions
of the client and server software are no different than in the more general
case described in the preceding paragraph. Once arguments are passed into the
server stub it can `compute the result' by calling the API function for which
the client stub is a replacement. The server stub then returns as its result
the return value of the API call. From the vantage point of the user program,
there is no difference between this access and an access to a local file
using the standard API library.  This is true because the RPC client stubs
preserve exactly the semantics of the standard API since there is a
one-to-one mapping between function calls in the RPC client stubs (i.e.,
client library), the RPC server stubs (the transmission protocol) and the
standard implementation of the API\footnote{In effect, RPC is used to replace
the normal procedure call mechanism with network messages, but outside of the
argument processing there is no change to the API entry points so their
semantics are guaranteed to remain unchanged.}. While building a server in
this way is guaranteed not to change the semantics of the API, the
operational mode of the API has radically changed. Previously it read data
from a file, now it reads data from a data server which may be located
anywhere on a given computer network (e.g., the Internet).

RPCs are not without drawbacks, however. A major problem with RPC technology
is that it is based on a strict request-reply paradigm. While certain
data sets work well in that context, others do not. For example, \jgofs
provides access to data sets that are functionally relational databases, and
thus are of unknown length. A good data server for such a data set would begin
returning records to the user program {\em before\/} the server completes
sending the result of a given access. However, the request-reply nature of
RPCs makes this difficult.

Further complicating the use of RPCs is that all network interprocess
communications code is nominally contained in the RPC server. Thus the server
is responsible for all access logging and all security precautions. While
basic access to a network is fairly easy to provide, more advanced functions
quickly add to the complexity of the server and can easily dominate the time
required to design and build it as well as the effort required to support it. 

\begin{quote}
\small 

\centerline{{\bf Note}} During the early part of the DODS design effort, we
built two sample RPC client-stub/server-stub pairs: one for \netcdf\ and one
for \jgofs. While these worked well within the limitations of RPCs, it was
felt that for certain data sets, particularly those that contained relational
data, the strict request-reply nature of RPCs would need to be improved upon.
This combined with the additional work required to support development of
full-fledged data servers caused us to look at other transport mechanisms.

These implementations, however, did effectively demonstrate that we could
build surrogates for the DODS supported APIs and re-link third party programs
with surrogate libraries. The re-linked user programs could indeed access
data via the data servers without any modification whatsoever. These
RPC-based libraries, however, could not interoperate.

\normalsize
\end{quote}

\begin{figure}
\centerline{\psfig{figure=prog2client-server2.ps,width=4.5in}}
\caption{Implementation of Distributed Application---API Access Using
  RPCs.}
\label{fig:prog2client-server}
\end{figure}

\subsection{HyperText Transfer Protocol---HTTP}

A second data delivery mechanism that is, on the surface, very different from
RPCs is the \HTTP\footnote {HTTP is the communication protocol used for the
World Wide Web.}. This transport mechanism is primarily intended to move
hypertext documents (text, images, \ldots) over a network. Its basic unit of
transfer is a file. The protocol is closely coupled with existing HTTP
servers from \WWWC, \NCSA\ and commercial providers.  These servers support
an extension mechanism called the \CGI\ which can be used to transform
information on the server computer to a hypertext document. Once the
hypertext document has been prepared, it is passed back to the HTTP server
which then sends it to the client.

A crucial difference between RPC technology and HTTP is that RPC is used to
build an application-level communication protocol while HTTP {\em is\/} an
application level protocol. While RPC can be used to build either stateless
of stateful distributed systems, HTTP can only be used to build stateless
systems. Because HTTP's extensions typically take the form of filter programs
combined (using the CGI mechanism) with a server, HTTP is a poor choice for
supporting `fat' interfaces. In other words, HTTP/CGI is a good choice for
interfaces that are stateless and have a small number of entry points, but a
poor choice for stateful or large interfaces.

The widespread use of HTTP and the competition between various providers of
HTTP daemons has focused the attention of those server providers on important
considerations like system security and accounting. Thus, while providers of
HTTP software need to make high quality implementations of the HTTP protocol
available, there is also considerable effort expended towards additional
features which improve the overall function of any networked system. Because
using RPCs virtually necessitates rebuilding support features like access
control and logging, using HTTP can translate into significant savings during
the design, implementation and management phases of a project.

The widespread use of HTTP means that server programs for HTTP are likely to
exist on many of the would-be data-provider machines. Thus using HTTP will
remove one impediment from a user becoming a data provider, that is, to
become a data provider users will not necessarily have to install a data
server. The nature of RPC technology is such that data server installation is
typically considered complex by users and removing that hurdle improves the
accessibility of the system (since the system depends on, to a large extent,
the willingness of people to become data providers).

\begin{figure}
\centerline{\psfig{figure=http-interface.ps,height=2.5in}}
\caption{HTTP can be used to construct servers architecturally similar to
  those built with RPCs. (Figure~\protect\ref{fig:prog2client-server}).  To
  do so the {\em RPC server stub\/} portion of the server is implemented
  using HTTPD and one or more {\em Common Gateway Interfaces\/} as shown in
  this figure.}
\label{fig:http-interface}
\end{figure}

While there are fundamental differences between RPC and HTTP,
Figure~\ref{fig:http-interface} shows how the {\em RPC server stubs\/} layer
in Figure~\ref{fig:prog2client-server} can be constructed using one
implementation of HTTP and some additional software.  One crucial difference
between systems built with RPC and those built using HTTP is that the
RPC-based server can maintain state information while the HTTP-based server
cannot.  Thus, while servers based on RPC can use any convenient set of
functions as the server interface, the HTTP interface must be more carefully
chosen so as to insure that the system can function efficiently without state
information retained in the server.

\subsection{Data Model}
\label{data-models}

A data model describes the organization of data. It is composed of a defined
set of elements and structures along with valid operations which can be
performed on those elements and structures. All programming languages
incorporate a data model in their design. For example the C programming
language defines a set of types which can be used to build programs (int,
double, \ldots); that set of types and their associated operations is the
data model for C programs.  Similarly, while not full-fledged programming
languages, APIs define a data model for the resources to which they provide
access by defining a set of functions (which are analogous to operators in a
programming language) and defining types which are passed to, or returned
from, those functions.

Data models are important because they provide a way to describe the types of
information that can (or cannot) be described using a given programming
language or API\@. Thus a data model can be used to compare the applicability
of two such systems to a particular problem. In addition, if a single data
model is supported by two APIs then it is possible to translate information
represented using that data model from one API to an equivalent
representation in the other API\@. Knowing which data model or models are
supported by a set of APIs enables a scientist to choose the API best suited
to a given type of information. It also provides a means for programmers to
build translators for certain types of data represented using those APIs.

As is often the case with semantic descriptions, a formal description of a
data model is hard to produce for many existing systems. This is true because
many APIs do not rigorously define their operators. Instead they rely on
conventions with ill specified boundary conditions. Thus the exact semantics
of many data types are hard to formalize.

The lack of formal data model semantics is further complicated when new types
are added (ad hoc) to an API\@. The `data type' exists only in so much as users
of the API agree to be bound by convention. A user may abuse the convention
or they may ignore, out of ignorance, a widely used convention. Either case
creates data that is hard to use. This problem create special difficulties
for people writing data translators since important conventions have not been
followed. For this reason, it is impossible to write perfect translators even
for a subset of data types.

Lack of formal data model semantics makes it hard to build translators to or
from an API or programming language even for a small set of well defined
objects.  Thus for many APIs, while it is possible to build translators for
data represented using a core set of data models common to those APIs, it is
not in general possible to build translators for any arbitrary object
represented in one API to every other API\footnote{Furthermore, some data
models cannot be translated into other data types without solving a
computational problem that is known to be NP complete.}.

\subsection{Data Format}

It is important to distinguish a data format from a data model. The later is
the convention used to store information so that a program can access the
information. A data format defines only the form of the information, not the
operations that can be performed on it. In contrast, a data model often does
{\em not\/} define a particular format. For example, an integer data type
(which includes properties of the object like its range and the operations
which can be applied to it) does not specify exactly how such an object will
be stored in a computer. It is possible to build (in software) many different
implementations of a given data model which have a high degree of
interoperability but which all use different internal representations.

Many data formats imply a particular data model or set of data models. Thus
it is possible to translate information stored in one format to another when
the two formats share data models. However, the problems that plague many
APIs (lack of formalism, \ldots) are worse for data formats because the set
of operations, rather than being at least partially defined, is purely
conventional. Thus features of a particular format which may be in fact
consequences of a particular implementation often must be accounted for in
the definitions of the operations. Lossless translations between data formats
are very hard even in limited cases and, as with APIs, impossible in general.

\section{Data Delivery Architecture}

DODS uses the client-sever paradigm to provide access to data across the
Internet. Each computer with on-line data can provide access to any user with
network access by installing one or more DODS data servers.  Users then
employ client programs to access data from the data servers. The data
delivery architecture for the entire distributed data system is the set of
DODS data servers and clients in existence at any given time. Because there
is no central authority which controls clients or servers, the system is in a
constant state of flux. In the future there will be a system which enables
users to search for servers with a particular type of data, but participation
in such a central registry is voluntary and is not part of the data delivery
architecture proper, rather the data delivery architecture consists solely of
the data servers and necessary client software.

The data delivery architecture does not rely on a centralized archive of
data, instead various data providers make relatively small amounts of data
available using their computers and data server software. Each user of data
can access the sum total of data from all the people or groups making data
available much as each user of a World Wide Web browser can access any
document made available through HTTP servers. While the contribution from
each individual is small, the total pool of data that users can access is
potentially huge.  Furthermore, the architecture of the DODS data delivery
mechanism is both general enough and simple enough to be used by individual
scientists and large agencies or groups of scientists. Thus the pool of data
which comprise DODS can be contributed by a wide variety of providers.

In the data delivery mechanism, the DODS \dap\ is used to provide the
interface definition for a data server. However, each data server translates
the `calls' in the protocol to function calls in the API\@. These calls are
used to encode information contained in the data set.  Thus, DODS data access
is built around a family of data servers, defined by the set of data-access
APIs supported by DODS, all of which use the same data access
protocol\footnote{In previous versions of this and other documents, the \dap
was referred to as the DODS API; that named seemed to confuse more than
edify, so it has been changed.}.

An essential concept on which DODS bases its data delivery architecture is
that a significant percentage of existing data is stored in a documented
format or API\footnote{In the remainder of this paper (and elsewhere) we use
  the phrase `data API' to mean the API used to encode the data. From time to
  time, we abuse the phrase `data API' to mean `the format used to store the
  data'.}. This assumption ensures that there is a high return for time spent
developing data servers. If an API has been used to encode a large number of
different data sets, then a single type of data server (one for that API) can
be used to provide access to all those data sets. The assumption that the
interface to a data set (or type of data set) is well documented also reduces
the complexity of designing a data server.

A second essential concept is that many science-data APIs draw on a small set
of data models. These data models include relational sets, arrays,
hierarchies, etc. It should therefore be possible to construct data servers
which can translate some (or many) data sets from one API to another. This
will enable users of one of the DODS supported APIs to read most of the data
sets in any of the other supported APIs---not just data sets in one API\@.

The DODS data delivery architecture is flexible enough to support a wide
range of users. Some users will have a detailed understanding of the format
and content of a particular data set and should be able to use that
understanding to their advantage. Other users will have no prior knowledge of
format or content of the same data set and still should be able to access
that data in a meaningful way.

Supporting a wide range of data access, from data browsing to direct
access, means providing many different types of user applications which can
access data. In fact, the sum total of data browsing, visualization and
access software that even a single user might want is staggering.
Furthermore, much of that software has already been written, but typically
not with distributed access in mind and not for the specific format or API in
which the data a given user wants is encoded. Complicating the problem is the
wide range of data formats and APIs now available and the rate at which their
numbers are expanding. Trying to write a complete set of network tools would
be impossible. Before the tools were complete new formats/APIs
would need to be supported.

A solution to this problem that fits well with the idea of data servers
providing access based on APIs is to build replacement libraries for for
those APIs. These libraries present to a program the same set of functions
that the original implementation of the API presented, but replace the bodies
of those functions with code that access a data server rather than a local
resource. Any user program that was originally designed to work with one of
the supported APIs can be re-linked with the DODS implementation of that API
(called a `surrogate library' or a `client library') and access all of the
distributed resources of DODS\@. Thus, the parts of the data delivery
architecture that support data browsing, visualization and access are not
described here because DODS will rely on existing user programs to provide
those functions.

The DODS architecture supports data management by presenting the scientist
with a set of third-party tools for storing and organizing data. Each of
these tools provides support for representing information using one or more
data models. Furthermore, the inclusion of an API in DODS implies that the
API has been accepted by the scientific/oceanographic community and is thus
of use in organizing or accessing relevant data sets (i.e., we chose APIs
which are `popular' and feel that popularity implies acceptance). The set of
data models circumscribed by the set of supported APIs forms the set of data
models which the \dap\ uses to describe and provide access to data
sets. Because the set of models is finite, it is possible that the user
community will want to add an API which uses a data model DODS knows nothing
about. In this case the \dap\ must be extended to support the new data
model(s).

The complete data delivery architecture for DODS consists of data servers and
data sets located on computers throughout the Internet, combined with user
programs re-linked with DODS surrogate libraries. The set of data sets and
user programs are not limited to those that use DODS supported APIs and/or
formats. That a data set is accessible via a DODS data server and that a user
program is able to access data using the \dap\ is the sole criteria for
membership in DODS\@.

The intent of this architecture is to make it as simple as possible for
scientists to access remote data. Using DODS servers, it should be possible
for many scientists to use programs already written to access new and
existing data sets stored on remote computers. The only preconditions their
programs must meet are that they must have been written using an API which
DODS supports and that they can be re-linked with the DODS implementation of
that API\@. Because the interface to the API remains exactly the same the user
program does not need to be modified in any way to access the remote data
available through DODS\@.

\subsection{Two Prospective Models for Data Servers}

Typically, a user program can be decomposed into some user-written source
code and a library of functions which implement the data access API\@. This
library is typically written and maintained by some third party group (e.g.,
a commercial software vendor or a research organization). Users of a
particular API access DODS data servers by replacing the `standard' version
of that API with the DODS version. This new version presents the application
with the same functions (so that all their existing programs continue to work
just as before), but adds the capability to access remote data sets by
connecting to DODS data servers. The server and the user
program/client-library combination communicate over the Internet and thus may
reside on any two machines that are connected to the Internet. Each client
library for a particular API is built so that a user program which was
written for that API, with or without prior knowledge of DODS, will be able
to access any of the data available through DODS servers. The DODS client
library for a particular API will necessarily perform different operations
internally, but the interface it presents to the user program will be exactly
the same as the original implementation of the library.

There are two conceptual models for a DODS data server: one in which data is
accessed using a server which closely mirrors the function calls supported by
the API used to store that data, and one which supports an interface using a
different API\@. In the latter case, calls in the server's interface (i.e.,
transmission protocol) are translated into calls in the data API and the
results are translated back from the data API to the server interface for
transmission. Figures~\ref{fig:simple-server}~and~\ref{fig:dods-server} show
these two architectures.

If DODS were to only support a single API, then clearly the first model of
data server (Figure~\ref{fig:simple-server}) would be preferable since,
lacking the need to transform data representations, it is simpler to
implement. A set of client stubs for the API would serve as the
reimplementation of the library. User programs would link to this and the
internals of each function call would be replaced with code to marshal the
function's arguments and send them to the correct server stub. On the server
side, the corresponding stub would unmarshal the arguments, call the function
as originally implemented and arrange to return to the client the function's
result (i.e., the reimplementation of the library, linked with the user
program). Once constructed, simple optimizations could be incorporated to
such a system to improve network use.

However, DODS intends to support data access through many APIs since one goal
of DODS is to provide a means for different data sets to interoperate. One
possible server architecture would require DODS to provide software to
translate information in each of its supported APIs to each of the other $N$
supported APIs. If each data server provides access solely in terms of the
API used to encode its data, then each new API brought into the set of
supported APIs requires $N+1$ additional translators be constructed. In
effect, the set of APIs becomes closely coupled in this model, that is each
new API effects the necessary support software for {\em all\/} APIs and
future changes in the system require modifications to existing components. As
the number of supported APIs increase the work required to add an additional
API becomes intractable.

%However, DODS intends to support data access through many APIs---initially
%two very different APIs are to be supported, with others added in the future.
%One goal of DODS is to provide a means for data sets to interoperate. A user
%program written to access data via the netCDF API should be able, once
%re-linked with the DODS implementation of netCDF, to access data from {\em
%  both\/} netCDF and JGOFS data sets (assuming that the data sets are
%accessible via DODS data servers). Thus, DODS must build software to
%translate information in each of its supported APIs to each of the other $N$
%supported APIs. If each data server provides access solely in terms of the
%API used to encode the its data, then each new API brought into the set of
%supported APIs requires $N+1$ translators be constructed. In effect, the set
%of APIs become closely coupled in this model---each new API effects the
%necessary support software for {\em all\/} APIs and future changes in the
%system require modifications to existing components. As the number of
%supported APIs increase the work required to add an additional API becomes
%intractable.

A second server architecture uses a common transmission protocol for all data
servers and requires that each supported API be accessible using this
protocol. In this model, support for a new API does not require modification
or addition to the existing software for currently supported APIs. However,
the trade-off necessary to reduce the growth of overall system complexity
translates to increased cost to support each individual API because for each
API two translation modules must be built, one to go from the API functions
to the transmission protocol, and one to reverse that operation. Of course,
if the intent is to only provide access to DODS data servers using the new
API but, not to build servers for that API's data, then only one translation
unit need be built. However, in general both translators will be desirable.

A third architecture involves combining elements of the the preceding two
architectures; using servers which speak an API dependent protocol combined
with translators that provide access to those servers through a single
uniform interface. This design has one advantage over a design that builds
translation into the client library and the server---it provides a simple way
for an API to be partially added to DODS\@. However, in many ways this may not
be a good feature for the system. If APIs can be added without translators
this could significantly limit the functionality of the system. DODS would
`support' many APIs, but only provide translators for a small set of APIs.
Even though the set of APIs could grow, the set of data available to most of
the APIs would be limited to data sets created using that API\@.

To control the complexity of adding support for an API to DODS, the second
architecture was chosen. Each API is accessed using a single transmission
protocol (see \DTP). This protocol (which in the past was referred to as the
DODS API) is sufficiently flexible to represent information encoded with
either of the two currently supported APIs. The translation to and from the
user API and data API are performed by the client library and data server,
respectively. This design was chosen over a design with data servers that use
the data API as a transmission protocol because it was felt that
interoperability of all data sets was a central goal of DODS\footnote{As is
described in \DDD, a second benefit of this architecture is that certain
transmission optimizations with relational data sets are possible.}.

The client and server components shown in Figure~\ref{fig:dods-server}
indicate that to add an additional API to the set of APIs supported by DODS,
four modules are involved: the surrogate user API library, Client Network
Interface, Network Server Interface and Server Side Interface. Of these four
modules two, the client interface and the server interface, are unlikely to
change from implementation to implementation.  Thus, each additional API will
involve the construction of the surrogate client library and the server side
translator.

\begin{quote}
\small
\centerline{{\bf Note}}
In previous documentation, the {\em DODS Server\/} architecture was referred
to as a {\em Translating Server}. In effect all DODS data servers are now
translating servers. It is important to understand that a DODS server does
not translate the data---it translates the method of access to data.
\normalsize
\end{quote}

\begin{figure}
\centerline{\psfig{figure=simple-server.ps,height=7.0in}}
\caption{Data server using the API interface as a Network Protocol (via RPC
    technology)}
\label{fig:simple-server}
\end{figure}

\begin{figure}
\centerline{\psfig{figure=dods-server.ps,height=7.0in}}
\caption{DODS data server configuration employing client and server
    translators. The network protocol is not the same as the API interface;
    it is a special transmission protocol which facilitates the translation
    of calls made in one API to calls made in another.}
\label{fig:dods-server}
\end{figure}

\subsection{The Role of Ancillary Data in Translation}

In order to support different data access APIs, DODS must effectively
translate between data models (See Section~\ref{data-models} and the \DTP
sections titled ``Dataset Descriptor Structure''~\externalref{api:dds} and
``Dataset Attribute Structure''~\externalref{api:das}). It is often the case
that the information about a data set available from an API is not sufficient
to translate that data set's access to another API which implements a
different data model. What is needed is some additional information about the
organization and relationships of the variables within the data set.  This
information is contained in the {\em Ancillary Data\/} store of a data set
(Figure~\ref{fig:dods-server}).

The ancillary data for a given data set is stored in text files and is
completely separate from the data set itself. A user who wishes to provide
data to the DODS community can store the extra information in one of these
text files without modifying the original data set. 

There are two types of ancillary data supported by the \dtp. Information
about the data types of the variables contained in a data set may be provided
using the Dataset Descriptor Structure (DDS). This structure is used to
describe the name, type, shape, and structural relation of the variables in a
given data set. Whenever the client library needs to translate information
from its network form (i.e., the form the information takes when it is
encoded using the \dtp) it will use information in the DDS structure to
decide how to perform those transformations. Similarly, the Dataset Attribute
Structure can be used to affect the translation of variable attributes in a
given data set.

The ancillary data files contain knowledge about the variables for use by the
data server during translation.  Data set translation is effectively the
translation of each variable in a given data set and API combination to one
or more variable(s) in a different API\@. The translation of each variable is a
transformation of data type and that transformation is directed by the
knowledge the translating software has about the API and the specific types
and organization of a given data set. For a data server, the ancillary data
structures are the only source of knowledge available beyond the API and the
data set itself.

When the information reaches a client library, all content specific to a
given API is lost. All information from the data set that is not actual data
is coded in a structure that is the catenation of the ancillary data
structures and information that can be extracted from the data set itself.
This information is structured just like the ancillary data files stored on
the server side. These files are the only source of knowledge about the data
set the client library can access.

Ancillary information is a data type description language. Knowledge about
the data set not available to the data server must be stored in one of the
two ancillary data files if it is to be available to the server upon
translation. The data server combines information evident from the data set
with that in the ancillary files to create the two structures sent to the
server. These two structures have the same form as an ancillary data.

\section{Conclusion}

The DODS data delivery architecture is based on a set of loosely cooperating
data servers spread throughout host computers on the Internet. The extent of
these host's commitment to DODS is that they run the servers and permit people
with client software to connect and access data. This system
architecture follows in the footsteps of other client-server systems where
participation is wholly voluntary. 

To provide a basic level of server security and accounting, the data delivery
architecture prescribes that third-party HTTP servers be the principle
network interface of DODS\@. Using these servers as the basis for the data
servers frees the DODS project from the task of building secondary support
services into DODS serves. The extra work required to build servers robust
enough to be considered trustworthy is a considerable drawback to RPC
technology.

The DODS data delivery architecture is based heavily on data access through a
set of known, stable and documented APIs. Each of these APIs defines, to a
greater or lesser extent, the semantics of the data models which it uses and
thus the data it can be used to store/access. Given this and that most
science data APIs draw on a small set data models, it is reasonable to assume
that calls in these APIs can be translated into a common access protocol.
Since each client program will use this common protocol to access data
servers, DODS will facilitate interoperability between the set of APIs it
supports. Because the characteristics of APIs that make this possible are not
limited to a small set of APIs it is reasonable to assume that the set of
APIs supported by DODS can be extended without radical modifications to the
basic structure of the system.

In order to maximize the body of existing software that can be used with
DODS, the development of the data delivery system will concentrate on
building data servers and reimplementing API libraries. The data servers can
be used to provide access to data stored in any one of the supported APIs
data sets. The reimplemented libraries (i.e., client libraries) can be used
to re-link existing application programs providing the extended functionality
of DODS without necessitating any modification of those programs.

\end{document}



