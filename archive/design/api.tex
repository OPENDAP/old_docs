

% This file contains information on the DAP. It contains the rationale,
% requirements, and design for the API. The translating servers and
% translator processes are described in the DDA and DDD documents.
%
% $Id$

%\documentstyle[code,12pt,html,psfig,margins]{article}
\documentstyle[code,html,psfig]{article}

\psfigurepath{api-figs}

%
% These are html links which are used often enough in writing about DODS to
% merit an input file.
% jhrg. 4/17/94
%
% File rationalized and updated while writing the DODS User
% Guide. Also includes other useful abbreviations.
% tomfool 3/15/96
%
% $Id$
%
% HTML references to DODS documents
%
% For most of the DODS documentation there are three html references defined:
% 1) The upper case \newcommand produces the title of the paper, an
%    html link in the online documentation and a footnote in the paper
%    version.  
% 2) A capitalized version produces a capitalized description of the
%    paper and a link in the online version (but no footnote in the
%    paper version). 
% 3) A lower case version which produces a lower case description of
%    the paper and a html link, but no footnote.

% External references to documents:
%
% In order for external references to work (via latex2html) the labels used
% throughout the documents must be unique. The text labels are all prefixed
% with a document identifier (e.g., ddd) with a colon separater. The
% \externalrefs below (way below...) includes the perl files html.sty uses to
% make the cross refs.

\newcommand{\DODS}{\htmladdnormallink{Distributed Oceanographic Data System}
  {http://www.unidata.ucar.edu/packages/dods/}}

\newcommand{\Dods}{\htmladdnormallink{DODS}{http://www.unidata.ucar.edu/packages/dods/}}

\newcommand{\wrkshp}{\htmladdnormallink{Report on the First Workshop for the
    Distributed Oceanographic Data System, Proposed System Architectures}
  {http://www.unidata.ucar.edu/packages/dods/archive/reports/workshop1/section3.13.html}}

% not DODS URLs, but useful all the same...

\newcommand{\www}{\htmladdnormallink{World Wide Web} 
  {http://www.w3.org/hypertext/WWW/TheProject.html}}
\newcommand{\url}{\htmladdnormallink{Uniform Resource Locator}
  {http://www.w3.org/hypertext/WWW/Addressing/URL/url-spec.html}}

\newcommand{\uri}{\htmladdnormallinkfoot{Uniform Resource Identifiers}
  {http://www.w3.org/hypertext/WWW/Addressing/URL/uri-spec.html}}

\newcommand{\urn}{\htmladdnormallinkfoot{Uniform Resource Name}
  {http://www.acl.lanl.gov/URI/archive/uri-archive.messages/1143.html}}

\newcommand{\HTTPD}{\htmladdnormallinkfoot{HTTPD}
  {http://hoohoo.ncsa.uiuc.edu/docs/Overview.html}}

\newcommand{\HTTP}{\htmladdnormallinkfoot{HyperText Transfer Protocol}
  {http://www.w3.org/hypertext/WWW/Protocols/HTTP/HTTP2.html}}

%\newcommand{\HTML}{\htmladdnormallinkfoot{HyperText Markup Language}
%  {http://www.w3.org/hypertext/WWW/MarkUp/MarkUp.html}}

% CERN used to be `Conseil Europeen pour la Recherche Nucleaire'
\newcommand{\CERN}
  {\htmladdnormallink{European Laboratory for Particle Physics}
  {http://www.cern.ch/}}

\newcommand{\NCSA}
  {\htmladdnormallink{National Center for Supercomputing Applications}
  {http://www.ncsa.uiuc.edu/}}

\newcommand{\WWWC}{\htmladdnormallink{World Widw Web Consortium}
  {http://www.w3.org/}}

\newcommand{\CGI}{\htmladdnormallinkfoot{Common Gateway Interface}
  {http://hoohoo.ncsa.uiuc.edu/cgi/overview.html}}

\newcommand{\MIME}{\htmladdnormallink{Multipurpose Internet Mail Extensions}
  {http://www.cis.ohio-state.edu/htbin/rfc/rfc1590.html}}

\newcommand{\netcdf}{\htmladdnormallink{NetCDF}
  {http://www.unidata.ucar.edu/packages/netcdf/guide.txn_toc.html}}

\newcommand{\JGOFS}{\htmladdnormallink{Joint Geophysical Ocean Flux Study}
  {http://www1.whoi.edu/jgofs.html}}

\newcommand{\jgofs}{\htmladdnormallink{JGOFS}
  {http://www1.whoi.edu/jgofs.html}}

\newcommand{\hdf}{\htmladdnormallink{HDF}
  {http://www.ncsa.uiuc.edu/SDG/Software/HDF/HDFIntro.html}}

\newcommand{\Cpp}{{\rm {\small C}\raise.5ex\hbox{\footnotesize ++}}}

% Commands

\newcommand{\pslink}[1]{\small
\begin{quote}
  A \htmladdnormallink{postscript}{#1} version of this document is
  available.  You may also use \htmladdnormallink{anonymous
  ftp}{ftp://ftp.unidata.ucar.edu/pub/dods/ps-docs/} to access postscript files
  of all of the DODS documentation.
\end{quote}
\normalsize
}

\newcommand{\declaration}[1]{\small
{\tt {#1}}
\normalsize
}

% All the links from here down are for the white papers. I'm not sure that any
% of these are still valid given the reorganization of the web site. 5/12/98
% jhrg.

\newcommand{\DDA}{\htmladdnormallinkfoot{DODS---Data Delivery Architecture}
  {http://www.unidata.ucar.edu/packages/dods/archive/design/data-delivery-arch/data-delivery-arch.html}}
\newcommand{\Dda}{\htmladdnormallink{Data Delivery Architecture}
  {http://www.unidata.ucar.edu/packages/dods/archive/design/data-delivery-arch/data-delivery-arch.html}}
\newcommand{\dda}{\htmladdnormallink{data delivery architecture}
  {http://www.unidata.ucar.edu/packages/dods/archive/design/data-delivery-arch/data-delivery-arch.html}}

\newcommand{\DDD}{\htmladdnormallinkfoot{DODS---Data Delivery Design}
  {http://www.unidata.ucar.edu/packages/dods/archive/design/data-delivery-design/data-delivery-design.html}}
\newcommand{\Ddd}{\htmladdnormallink{Data Delivery Design}
  {http://www.unidata.ucar.edu/packages/dods/archive/design/data-delivery-design/data-delivery-design.html}}
\newcommand{\ddd}{\htmladdnormallink{data delivery design}
  {http://www.unidata.ucar.edu/packages/dods/archive/design/data-delivery-design/data-delivery-design.html}}

\newcommand{\URL}{\htmladdnormallinkfoot{DODS---Uniform Resource Locator}
  {http://www.unidata.ucar.edu/packages/dods/archive/design/urls/urls.html}}
\newcommand{\Url}{\htmladdnormallink{Uniform Resource Locator}
  {http://www.unidata.ucar.edu/packages/dods/archive/design/urls/urls.html}}
\newcommand{\dodsurl}{\htmladdnormallink{uniform resource locator}
  {http://www.unidata.ucar.edu/packages/dods/archive/design/urls/urls.html}}

\newcommand{\DAP}{\htmladdnormallinkfoot{DODS---Data Access Protocol}
  {http://www.unidata.ucar.edu/packages/dods/archive/design/api/api.html}}
\newcommand{\Dap}{\htmladdnormallink{Data Access Protocol}
  {http://www.unidata.ucar.edu/packages/dods/archive/design/api/api.html}}
\newcommand{\dap}{\htmladdnormallink{data access protocol}
  {http://www.unidata.ucar.edu/packages/dods/archive/design/api/api.html}}

\newcommand{\SOFT}
        {\htmladdnormallinkfoot{DODS---Software Development Environment}
  {http://www.unidata.ucar.edu/packages/dods/archive/managment/software/software.html}}
\newcommand{\Soft}{\htmladdnormallink{Software Development Environment}
  {http://www.unidata.ucar.edu/packages/dods/archive/managment/software/software.html}}
\newcommand{\soft}{\htmladdnormallinkfoot{software development environment}
  {http://www.unidata.ucar.edu/packages/dods/archive/managment/software/software.html}}

% I used to call the DAP the API, then for a few days I called it the
% DTP. These def's were used then. Rather than find everywhere they are used
% (not that hard, but...) I just hacked the macros. NB: the source file for
% the DAP document is still called `api.tex'

\newcommand{\API}{\htmladdnormallinkfoot{DODS---Data Access Protocol}
  {http://www.unidata.ucar.edu/packages/dods/archive/design/api/api.html}}
\newcommand{\api}{\htmladdnormallink{data access protocol}
  {http://www.unidata.ucar.edu/packages/dods/archive/design/api/api.html}}

\newcommand{\DTP}{\htmladdnormallinkfoot{DODS---Data Access Protocol}
  {http://www.unidata.ucar.edu/packages/dods/archive/design/api/api.html}}
\newcommand{\dtp}{\htmladdnormallink{data access protocol}
  {http://www.unidata.ucar.edu/packages/dods/archive/design/api/api.html}}

\newcommand{\TOOLKIT}{\htmladdnormallinkfoot{DODS---Client and Server Toolkit}
  {http://www.unidata.ucar.edu/packages/dods/archive/implementation/toolkits/toolkits.html}}
\newcommand{\Toolkit}{\htmladdnormallink{Client and Server Toolkit}
  {http://www.unidata.ucar.edu/packages/dods/archive/implementation/toolkits/toolkits.html}}
\newcommand{\toolkit}{\htmladdnormallink{client and server toolkit}
  {http://www.unidata.ucar.edu/packages/dods/archive/implementation/toolkits/toolkits.html}}

\newcommand{\TKR}{\htmladdnormallinkfoot{DODS---DAP Toolkit Reference}
  {http://www.unidata.ucar.edu/packages/dods/archive/implementation/toolkits/toolkits.html}}
\newcommand{\Tkr}{\htmladdnormallink{DAP Toolkit Reference}
  {http://www.unidata.ucar.edu/packages/dods/archive/implementation/toolkits/toolkits.html}}
\newcommand{\tkr}{\htmladdnormallink{DAP toolkit reference}
  {http://www.unidata.ucar.edu/packages/dods/archive/implementation/toolkits/toolkits.html}}

% I know that these are wrong. The papers have been moved to the gso web
% site, but I'm not sure exactly where. 5/12/98 jhrg

\newcommand{\DM}{\htmladdnormallinkfoot{DODS---Data Model}
  {http://lake.mit.edu/dods-dir/dodsdm2.html}}
\newcommand{\Dm}{\htmladdnormallink{Data Model}
  {http://lake.mit.edu/dods-dir/dodsdm2.html}}
\newcommand{\dm}{\htmladdnormallink{data model}
  {http://lake.mit.edu/dods-dir/dodsdm2.html}}

\newcommand{\DD}{\htmladdnormallinkfoot{DODS---Data Delivery}
  {http://lake.mit.edu/dods-dir/dods-dd.html}}

% external refs for DODS documents

\externallabels{http://www.unidata.ucar.edu/packages/dods/design/api}
  {/home/www/packages/dods/archive/design/api/labels.pl}
\externallabels{http://www.unidata.ucar.edu/packages/dods/design/data-delivery-arch}
  {/home/www/packages/dods/archive/design/data-delivery-arch/labels.pl}
\externallabels{http://www.unidata.ucar.edu/packages/dods/design/data-delivery-design}
  {/home/www/packages/dods/archive/design/data-delivery-design/labels.pl}
\externallabels{http://www.unidata.ucar.edu/packages/dods/implementation/toolkits}
  {/home/www/packages/dods/archive/implementation/toolkits/labels.pl}

%\renewcommand{\externalref}[2]{#2}

%%% Local Variables: 
%%% mode: latex
%%% TeX-master: t
%%% End: 


\begin{document}

\title{DODS---Data Access Protocol}
\author{}
\date{23 August 1996} %{\today}

\maketitle

\begin{abstract}

  This document describes a stateless data transmission protocol which can be
  used to access a wide variety of earth sciences data. The protocol is
  referred to as the Data Access Protocol in much of this document and it is
  intended to be used in the construction of system software, particularly
  for the Distributed Oceanographic Data System. This protocol for data
  access is flexible enough to access all the data sets which will be part of
  the system. However, it does not offer the rich features found in
  interfaces intended for user-program construction.  In addition to
  presenting the requirements and a design for the interface, the rationale
  for choosing this type of interface is also given.

\end{abstract}


% Biolerplate text warning that anything can (and almost certainly will)
% change. 
%
% jhrg 3/25/94
%
% $Id$

\small

\centerline{{\bf Note}} 

\begin{quotation}

\begin{htmlonly}
  This is a working document made available to solicit comments. To receive
  e-mail notice of new or significantly changed documents, send e-mail
  to \htmladdnormallink{majordomo@unidata.ucar.edu}
  {mailto:majordomo@unidata.ucar.edu} with 'subscribe dods' in the
  body of the message.
\end{htmlonly}

\begin{latexonly}
  This is a working document made available to solicit comments. To receive
  e-mail notice of new or significantly changed documents, send e-mail
  to majordomo@unidata.ucar.edu with 'subscribe dods' in the body of
  the message. 

  Hypertext references in the hardcopy version of this document are
  represented using footnotes that contain a Universal Resource Locator (URL)
  to the referenced document.  To read the online documentation, open the URL
  `http://www.unidata.ucar.edu/' with a World Wide Web browser.
\end{latexonly}

\end{quotation}

\normalsize

%
% Biolerplate text expaining that this document is intended for programmers.
%
% jhrg 4/17/94
%
% $Id$

\small

\begin{quotation}
  This document is written for people interested in implementing software
  that will be part of, or work in parallel with, the Distributed
  Oceanographic Data System. 
\end{quotation}

\normalsize

\begin{htmlonly}
\pslink{ftp://ftp.unidata.ucar.edu/pub/dods/ps-docs/api.ps}
\end{htmlonly}

\clearpage

\tableofcontents

\clearpage

\section{Introduction}
\label{introduction}

The \DODS\ (DODS) Data Access Protocol (DAP) is a {\em lingua franca\/} for
reading earth science data sets. It is used as an intermediate representation
for data which are nominally accessed using an established third-party
Application Programmer Interface (API) (e.g., netCDF). Because, in addition
to a method for data access, the DAP defines the content of ancillary
information about data sets, it provides the means to access data sets
through a single protocol using any of the DODS supported APIs. Because the
\dap\ serves as an intermediate representation for several different APIs, it
can be used to translate between any two of those APIs.

In \DDA, an architecture is described that uses third party APIs to access
data stored on remote computers. In that document, it is assumed that users
of those data write programs in which those data are accessed using an API\@.
For DODS, this assumption is advantageous because it limits the class of
programs which DODS must address to those which use a formally defined
interface for data access. For the user, it is advantageous to use an API
because the storage format of the data is both hidden from the user and
separated from the user program source code, resulting in programs which take
less time to write and modify and which are more likely to be correct.

Most data access APIs do not provide direct access to distributed
data\footnote{The phrase {\em distributed data\/} refers to data sets that
reside on different computers which are linked by a network such as
Internet. The computers may or may not be physically remote from each other.
The main point is that the computers manage their data resources
independently. In this paper the terms {\em remote\/} and {\em distributed\/}
are used to imply independently managed resources.}. The DODS data delivery
architecture describes how such a capability can be added to an API\@.
Reimplementing an API using the DODS software \toolkit\ provides the
necessary bridge between the API's function calls and the network
communication infrastructure necessary to allow the API to access distributed
data sets.

Once users have the power to access remotely stored data they realize that
not all data are available using the API that they use. In fact, many data
sets' storage formats are incompatible with third party APIs altogether
because they are stored using conventions independently developed at the site
where the data were first analyzed, collected, or archived. While those in
charge of the data at a remote site may be very willing to provide remote
access, they will probably not be willing to reformat their data for most
off-site users.  How do users access remote data when it is not stored in a
format or data model their programs understand?

The DODS Data Access Protocol provides a means to access data sets using any
one of a set of supported APIs by defining an intermediate representation for
data sets, and components of data sets, that is API and format independent.
Using the DODS DAP as an intermediate representation, software components can
be built which translate a user program's API calls into DAP calls. Data
servers can be built which accept DAP calls and, either by translating
into a third-party API or using data set-local access methods, read the data.
Thus software which uses any one of a set of APIs can be used to read data so
long as a DODS server for that data exists---even if the data is stored in an
API different from the one used by the display or analysis software.

\begin{center}
{\bf Note}
\begin{quote}
In a previous design, the DODS data delivery system contained
reimplementations of the supported APIs which used the API entry points as
the remote access interface definition; the set of calls to which a remote
data server responds. However, this design did not facilitate multi API
access (i.e., API translation) to data sets.
\end{quote}
\end{center}

\section{Rationale}

Most data access APIs are targeted at application writers---they provide
powerful tools to read, write, and process one specific type of scientific
data and are typically specialized so that software in one application domain
is easy to write. In order to fulfill this function, APIs are typically
`feature rich' (i.e., they provide many ways to perform similar tasks). This
characteristic causes APIs that maintain this level of flexibility over a
wide range of data types to become very complex. Because simplicity and
ease-of-use are two characteristics that make APIs attractive to application
programmers, most API designers try to limit the scope of an API so that it
can provide the features application programmers want, at the level of detail
they want, without becoming excessively large and thus burdensome to use.
The netCDF and JGOFS APIs are good examples of APIs designed for a specific
type of application.  Each is used to access earth-science data sets, but
netCDF is most suited to gridded model data, and JGOFS has direct support
only for relational data.

However, the netCDF API cannot be directly used to access data sets produced
by the JGOFS API and vice versa. An application using either one of these
APIs would need to be modified if a user wanted to integrate both netCDF and
JGOFS compliant data sets. Modifying a program so that the data access API it
uses matches with the API used to write a data set is an expensive process
and one which is exacerbated by the fact that more and more users do not
write their own software. Modifying someone else's software can be very
complex and costly at best; at worst it is not possible at all.  Moreover,
when the data sets are remote it may be impossible to convince their
maintainers to replicate them for another API since this, too, is very
costly.

One alternative to rewriting application software or replicating data is to
define an intermediate representation which can be used to access data stored
in any one of a number of APIs.  If the intermediate representation is based
on a data model that encompasses the abstractions of a certain set of APIs,
then the intermediate representation can be used to translate between those
APIs. The implementation of an intermediate representation interface need not
be an application interface itself; instead it can be a set of tools used to
build application-level components. These components can then be combined at
link-time and/or runtime with user-programs to access data.

An interface which is to be used as a translation mechanism cannot afford to
present the rich set of access functions that the individual APIs, which it
subsumes, present. It must control the size and complexity of its interface by
removing features while still providing the capability to transparently
interoperate between several data models.  The DODS DAP is such an interface.
It is designed to be used in the construction of system components which are
configured at runtime by users, maintainers or other software components.
Thus, the DODS DAP is not a separate addition to the set of interfaces formed
by JGOFS, netCDF, and others.  Instead it is an intermediate representation
which can be used to translate between those different APIs.

\section{Requirements}

This section presents the requirements that must be met for an intermediary
representation for any of the DODS supported APIs. Currently, two APIs are
supported by DODS: netCDF and JGOFS\@. As additional APIs are incorporated
into the system (e.g., HDF), the DAP may add support for the data types those
handle.

\begin{enumerate}

\item The DODS \dap (DAP) can be used to map data stored using one API to
  calls in a second API so long as both APIs are supported by DODS\@.

\item It must be possible to add new data types and operators to the DAP
  without changing the external interface it presents.

\item The DAP does not characterize the data set based on its native
  implementation. The DAP supports access to data sets that consist of a
  single file, several files, a relational database, a set of relational
  databases, or any other reasonable representation.

\item The DAP makes all data sets appear cohesive---they always appear as a
  single store of data regardless of their actual implementation. Not yet
  satisfied (23 August 1996). %\today).

\item The DAP supports data sets which consist of one or more variables.

\item When a data set contains two or more variables, the DAP makes their
  relationship explicit. The relations are precisely described using a syntax
  which other computer systems can analyze.

\item The DAP describes each variable in a data set by its:

\begin{enumerate}
\item Name
\item Type
\item Shape
\item Units. Not yet satisfied (23 August 1996) %\today)
\item Type-dependent characteristics
\item \ldots zero or more attribute-value pairs
\end{enumerate}

\item The characteristics of the different variable types are given by the
  \dm.

\item The DAP can perform the following operations on a data set:

\begin{enumerate}
\item Get data set structure description
\item Get data set attribute list
\item Send arguments to a data set
\item Get values from the data set
\end{enumerate}

\end{enumerate}

\section{Design}
\label{design}

The DODS DAP design contains three important parts: A data model which
describes data types that can be supported by the protocol and how they are
handled, the data set description and data set attribute structures which
describes the structure of data sets and the data they contain, and a small
set of messages that are used to access data. Each of these components are
described in the following subsections along with a summary of the runtime
process configuration needed for DODS to translate from a data set API to a
user program API\@. The discussion of the runtime process configuration
frames each of the three design components and is used to explain why they
are needed.

\subsection{Process and Module Configuration}
\label{process}

The set of processes and modules needed to access a data set is shown in
Figure~\ref{fig:structure}. In the figure two processes are shown. Process 1
contains the user program, a surrogate library implementation of the data
access API used by that program, and translator component which uses the DODS
DAP to request data used to satisfy calls from the user-program API calls.
This process, called the {\em user process\/} communicates with the second
process, the {\em translating server\/} process.  The translating server
contains two modules---one to recast the incoming DODS DAP calls into calls
in the the data set's API and a second module that implements that API\@. In
the figure all data is transmitted over the network using the DODS DAP
regardless of the API used by the user program or used to store the data,
even if the APIs are the same.

\begin{figure}
\centerline{\psfig{figure=structure.ps}}
\caption{Two processes are used to translate API Y calls from the user
  program to API X calls for the data set. These same two processes are the
  client and server in a distributed data system. Note that DODS does not
  supply either the complete client or the complete server---rather DODS
  software consists of modules that can be combined with existing software
  (user programs, third party libraries) to build distributed systems. In
  this figure, {\em Module\/} refers to a conceptual unit (e.g., a library),
  not just a compile-time unit.}
\label{fig:structure}
\end{figure}

The DODS DAP's main function is to facilitate translation between two
different data access APIs. In order to read data stored in API X as in
Figure~\ref{fig:structure} a server which provides access using the DAP can
use the native implementation of API X to read information from the data set.

In addition to the data accessible via the native API, each data set will
contain ancillary data. These data will be directly accessed by some of the
DAP calls. It will be used by both the local implementation of the DAP to aid
in translating the DAP calls into the data set's native API calls and by
the remote translation process.

In order to effectively translate the user program API calls into DODS DAP
calls, the translator module must have some knowledge of the source data set's
structure. This structural information will comprise part of the ancillary
data that is accessed directly by the DODS DAP\@. Based on this information,
which can be accessed using DAP calls, the translator can choose how to best
translate the user programs data-access calls. In effect, the translator must
map one data access API onto a data model to which it may not be well suited
(either because the implementors of the user program or the data set have
chosen an inappropriate interface).

Finally, in order to be useful by many user programs, particularly those
written by a third party development team for a group of users, the
translator must have some additional information about the representation of
certain data objects expected by the user program. This information, called
{\em Usage Data\/} in Figure~\ref{fig:structure}, will allow the user to
specify the format of dates and similar information which has many different
common forms but no universally accepted format. It is stored in a file
created by the user or developers of the {\em user program}. This
information is defined by DODS outside of the definition of API Y and will be
used by the translator module when data requests are made by the user program
or when data is returned to the user program in response to one of those
requests.

\subsection{The Data Model}

Data models provide a way to organize scientific data sets so that useful
relationships between individual datum are evident. Many data models have
been specifically designed to make using the data in a computer program
simpler. Examples of computationally oriented data models for scientific data
are hierarchical, sequential, and gridded data models. 

Data models are abstract, however, and to be used by a computer program they
must first be implemented by a programmer. Often this implementation takes
the form of an API---a library of functions which can read and write data
using a data model or models as guidance. Thus every data access API can be
viewed as implementing some data model, or in some cases several data models.

Because DODS needs to support several very different data models, it is
important to design it around a core set of concepts that can be applied
equally well to each of those data models. If that can be done, then
translation between data represented in those different models may be
possible.

Currently DODS supports two very different data access APIs: netCDF and
JGOFS\@. The netCDF API is designed for access to gridded data, but has some
limited capabilities to access sequence data (although not with all of its
supported programming language interfaces). The JGOFS API provides access to
relational or sequence data\footnote{In the remainder of this document, the
  phrase {\em sequence data}, or just {\em sequence}, will mean an ordered
  set of elements each of which contains one or more sub elements where all
  of the sub-elements of an element are explicitly related to each other.}.
Both APIs support access in several programming languages (at least C and
Fortran) and both provide extensive support for limiting the amount of data
retrieved.  For example a program accessing a gridded data set using netCDF
can extract a subsampled portion or {\em hyperslab\/} of that data.  Likewise,
the JGOFS API provides a powerful set of operators which can be used to
specify which type of sequence elements to extract (e.g., only those
corresponding to data captured between 12:01am and 11:59am) as well as
masking certain parameters from the returned elements so that only those
parameters needed by the program are returned.

The DODS DAP uses the concepts of variables and operators as the base for the
data model. Within the data model, a data set consists of one or more
variables where each variable is described formally by a number of
attributes.  Variables associate names with each component of a data set, and
those names are used to refer to the components of the data set. In addition
to their different attributes, it is possible to operate on individual
variables or named collections of variables. The principal operation is {\em
  access}, although this can be modified in a number ways.

\subsubsection{Base-Type Variables}
\label{base-type-variables}

Variables in the DODS DAP have two forms. They are either base types or type
constructors. Base type variables are similar to predefined variables in
procedural programming languages like C or Fortran (e.g., {\tt int} or {\tt
  integer*4}).  While these certainly have an internal structure, it is not
possible to access parts of that structure using the DAP\@. Base type
variables in the DAP have three predefined attributes (or characteristics):
Name, Type, and Unit.  They are defined as follows:

\begin{description}

\item [Name] A unique identifier that can be used to reference the part of
  the data set associated with the variable.

\item [Type] The data type contained by the variable. This can be one of {\tt
    byte}, {\tt int32}, {\tt float64}, {\tt string}, and {\tt URL}\@.
  Where:

\begin{description}
\item [byte] is the same as unsigned char in ANSI C\@.
\item [int32] is a 32 bit integer---it is synonymous with long in
  ANSI C when that type is implemented as 32 bits.
\item [float64] is the IEEE 64 bit floating point data type.
\item [string] is a sequence of bytes terminated by a null character.
\item [URL] is a string as defined in \URL\@.
\end{description}

\item [Unit] This contains the name of the units of the value contained in
  the variable.  Examples of typical units are degrees Celsius, degrees
  Kelvin, \ldots If a variable is unit less, then this is null.

\end{description}  

\subsubsection{Type Constructor Variables} 

Type constructor variables describe the grouping of one or more variables
within a data set. These classes are used to describe different types of
relations between the variables that comprise the data set. This information
can be useful to people who would like to understand more about the data set
than can be conveyed with implicit relations. It is also designed to be
useful to other programs/processes in the data access chain.  There are six
classes of type constructor variables defined by the DAP: lists, arrays,
structures, sequences, functions, and grids. The type constructor classes
besides structure provide information that is used in the translation of
subsetting operations (hyperslabbing or selections and projections in netCDF
or JGOFS parlance, respectively). The types are defined as:

\begin{description}

\item [List] The {\bf List} type constructor is used to hold lists of 0 or
  more items of one type. Lists of {\tt int32}, \ldots, {\tt grid} are
  specified using the keyword {\tt list} before the variable's class. Access
  to an element of a list is possible using one of the five operators given
  in Table~\ref{api:tab:class-ops}.

\item [Array] An {\bf Array} is a one dimensional indexed data structure as
  defined by ANSI C\@. Multidimensional arrays are defined as arrays of arrays.
  In addition to element access using subscripts enclosed in brackets ({\tt
    []}), an array may be accessed using only its name to return the entire
  array or using a hyperslab operator to return a rectangular section of the
  array. In the later case, the hyperslab is defined for each dimension by a
  starting index, and ending index, and a stride value.  Specifying a stride
  $>$ 1 will cause the dimension to be subsampled by the stride value.
  Table~\ref{api:tab:class-ops} shows the syntax for array accesses including
  hyperslabs.

  In addition to its magnitude, every dimension of an array may also have a
  name. It is possible to find the name for any given dimension (e.g., the
  $i^{th}$ dimension) and thus write software which access the $3^{rd}$
  element of the dimension {\em cast\/} (See \TOOLKIT).

\item [Structures] A structure is a class that conveys no relational
  information and may contain several variables of different classes. It is
  used to supply information to other parts of the data access and
  translation system that may be useful in optimizing the access or
  translation operations. The structure type can also be used to group a set
  of unrelated variables together into a single data set.

\item [Sequences] A sequence is an ordered set of $N$ variables which has
  several instantiations (or values). Variables in a sequence may be of
  differing classes.  Each instance of a sequence is one instantiation of the
  variables. Thus a sequence can be represented as:

\begin{displaymath}
\begin{array}{ccc}
  s_{0 0} & \cdots & s_{0 n} \\
  \vdots & \ddots & \vdots \\
  s_{i 0} & \cdots & s_{i n}
\end{array}
\end{displaymath}

\noindent Every instance of sequence $S$ has the same number, order, and
class of variables. A sequence implies that each of the variables is
related to each other in some logical way. A sequence is different from a
structure because its constituent variables have several instances while a
structure's variables have only one instance (or value). Because a sequence has
several values for each of its variables it has an implied {\em state}, in
addition to those values.

\item [Functions] Functions are a subclass of Sequences and are used to
  indicate that one set of variables has a functional relation to a second
  set of variables. Variables in a function may be of differing classes. The
  mathematical description of this functional relation is not
  specified. Instead the function type is used to indicate that one of the
  two sets constitute the independent variables and the other the dependent
  variables.  Typically, the variables defined by a function have more than
  one instance---functions are similar to sequences but have additional
  information about the functional dependency of variables.

\item [Grid] A grid is an association of an $N$ dimensional array with $N$
  named vectors, each of which has the same number of elements as the
  corresponding dimension of the array. Each vector is used to map indices of
  one of the array's dimensions to a set of values which are normally
  non-integral (e.g., floating point values). The $N$ (map) vectors may be
  members of different classes. Grids are similar to arrays of base type
  variables, but add named dimensions and maps for each of those dimensions.
  \label{page:grids} 

\end{description}

\subsubsection{Operators}
\label{api:operators}

The principal operation performed on any variable is to access that variable
and retrieve its value or values.  For a base type variable, access is the
implied operation and is achieved by passing the name of the variable to a
data server using the access protocol (see Section~\ref{entry}). For an
instance of a type constructor class, the variable name will access the
entire object and return it in an array (or C struct)\footnote{If a variable
  contains elements of different types then an instance of it {\em must\/} be
  returned in a {\tt struct} and not an array.  This implies that the
  receiving program must dynamically allocate storage for the variable and
  then correctly access the fields.  This implies sophisticated programming
  (beyond the `user' level) or an interpreter---or both.}. Fields of type
constructors may also be accessed using the dot ({\tt.}) operator or the
virtual file system syntax. If a structure {\tt s} has two fields {\tt time}
and {\tt temperature}, then those fields may be accessed using {\tt s.time}
and {\tt s.temperature} or as {\tt s/time} and {\tt s/temperature}.

All of the classes listed in section~\ref{base-type-variables} are local to
the data set except the {\tt URL} type. When an object of type URL is
accessed, the server must open the data set referenced by that URL and read
its values such that the structure of that referenced data set replaces the
URL in the current data set. In practice this is most useful in a constraint
expression when the value of a variable is compared witht he value of another
variable in a different data set. If the URL cannot be referenced, then the
type of the URL degrades to {\em string\/} and an access to that element of
the data set returns that string.

Access to variables can be modified using selection operators. Each type of
variable has its own set of selection and projection operators which can be
used to modify the result of accessing a variable of that type.
Table~\ref{api:tab:class-ops} summarizes the types and the operators applicable
to them. In the table, operators have the meaning defined by ANSI C except as
follows: the array hyperslab operators are as defined by netCDF\cite{netcdf}
(where {\tt a} is the start index, {\tt b} is the stride, and {\tt c} is the
end index), the string operators are as defined by AWK\cite{kern:upe}, and
the list operators are as defined by Common Lisp\cite{steele:clisp}.

\begin{table}
\caption{Classes and operators in the DAP\@.}
\label{api:tab:class-ops}
\begin{center}
\begin{tabular}{|| l | l ||} \hline
\multicolumn{1}{|| c}{\sc Class} & \multicolumn{1}{c ||}{\sc Operations} \\
\hline \hline
\multicolumn{2}{|| l ||}{\em Base Type} \\ \hline
{\tt byte}, {\tt int32}, {\tt float64} & {\tt < > = != <= >=} \\ \hline
{\tt string} & {\tt = != } $=\sim$ \\ \hline
{\tt URL} & {\tt *} \\ \hline
\multicolumn{2}{|| l ||}{\em Constructor\/} \\ \hline
{\tt array} & {\tt [start:stop] [start:stride:stop]} \\ \hline
{\tt list} & {\tt length, car, cdr, nth, member} \\ \hline
{\tt structure} & {\tt.} \\ \hline
{\tt sequence} & {\tt.} \\ \hline
{\tt function} & {\tt.} \\ \hline
{\tt grid} & {\tt .} \\ \hline
\end{tabular}
\end{center}
\end{table}

The hierarchy of the type constructor classes in the DAP is shown in
Figure~\ref{fig:class}. 

\begin{figure}
\centerline{\psfig{figure=class-hier.ps}}
\caption{The hierarchy of classes in the DAP\@.}
\label{fig:class}
\end{figure}

\subsection{The External Representation of Variables}
\label{api:external-rep}

Each of the base-type and type constructor variables has an external
representation defined by the \dap. This representation is used when an
object of the given type is transferred from one computer to another.
Defining a single external representation simplifies the translation of
variables from one computer to another when those computers use different
internal representations for those variable types\cite{xdr}. 

Constraint expressions, which are an importat part of the \dap, do not affect
{\em how\/} a base-type variable is transmitted. Constraint expressions only
determine if, given a particular value for a variable, that variable should
be transmitted. However, for constructor type variables contraint expressions
may be used to exclude portions of the variable. In these cases, the
constraint expressions can be used to change the way a particular variable is
transmitted. For example, if a constraint expression is used to select the
first three of six fields in a structure, the last three fields of that
structure are {\em not\/} transmitted by the server.

The \dap uses Sun Microsystems' XDR protocol\cite{xdr} for the external
representation of all of the base type variables. Table~\ref{tab:base-xdr}
shows the XDR types used to represent the various base type
variables.

\begin{table}
\caption{The XDR data types used by the \dap as the external representations
  of base-type variables}
\label{tab:base-xdr}
\begin{center}
\begin{tabular}{|| l | l ||} \hline
\multicolumn{1}{|| c}{\sc Base Type} & \multicolumn{1}{c ||}{\sc XDR Type} \\
\hline \hline
{\tt byte} & {\tt xdr byte} \\ \hline
{\tt int32} & {\tt xdr long} \\ \hline
{\tt float64} & {\tt xdr double} \\ \hline
{\tt string} & {\tt xdr string} \\ \hline
{\tt URL} & {\tt xdr string} \\ \hline
\end{tabular}
\end{center}
\end{table}

In order to transmit constructor type variables, the \dap defines how the
various base type variables, which comprise the constructor type variable,
are transmitted. Any constructor type variable may be subject to a constraint
expression which changes the amount of data transmitted for the variable (see
Section~\ref{api:constraints}). For each of the six constructor types these
definitions are:

\begin{description}

\item [Array] An array is sent using the {\tt xdr\_array} function. This means
  that an array of 100 {\tt Int32}s is sent as a single block of 100 {\tt
    xdr long}s, not 100 separate {\tt xdr long}s. 

\item [List] A list is sent as if it were an array.

\item [Structure] A structure is sent by encoding each field in the order
  those fields are declared in the DDS and transmitting the resulting block
  of bytes.

\item [Sequence] A Sequence is transmitted by encoding each instance as for a
  structure and sending one after the other, in the order of their occurrence
  in the data set. The entire sequence is sent, subject to the constraint
  expression. In other words, if no constraint expression is supplied then
  the entire sequence is sent. However, if a constraint expression is given
  all the records in the sequence that satisfy the expression are
  sent\footnote{The client process can limit the information received by
    either using a constraint expression or prematurely closing the I/O
    stream. In the later case the server will exit without sending the entire
    sequence.}.

\item [Function] A function is encoded as if it is a Sequence (one component
  after the other, in the order of their declaration).

\item [Grid] A grid is encoded as if it is a Structure (one component
  after the other, in the order of their declaration).

\end{description}

\subsection{Dataset Descriptor Structure}
\label{api:dds}

In order to translate from the user program's API to the data set's API, the
translator process must have some knowledge about the types of the variables,
and their semantics, that comprise the data set. It must also know something
about the relations of those variables---even those relations which are only
implicit in the data set's own API\@. This knowledge about the data set's
structure is contained in a text description of the data set called the {\em
  Dataset Description Structure}.

The data set description structure (DDS) does not describe how the information
in the data set is physically stored, nor does it describe how the data set's
API is used to access that data. Those pieces of information are contained in
the data set's API and in the translating server, respectively.  The
translating server uses the DDS to describe the structure of a particular
data set to a translator---the DDS contains knowledge about the data set
variables and the interrelations of those variables.  In addition, the DDS
can be used to satisfy some of the DODS supported APIs data set description
calls. For example, netCDF has a function which returns the names of all the
variables in a netCDF data file. The DDS can be used to get that information.

The DDS is a textual description of the variables and their classes that
comprise the entire data set. The data set descriptor syntax is based on the
variable declaration/definition syntax of C and C++. A variable that is a
member of one of the base type classes is declared by by writing the class
name followed by the variable name. The type constructor classes are declared
using C's brace notation. A grammar for the syntax is given in
Table~\ref{tab:DDS}. Each of the keywords for the type constructor and base
type classes have already been described in
section~\ref{base-type-variables}. The {\tt data set} keyword has the same
syntactic function as {\tt structure} but is used for the specific job of
enclosing the entire data set even when it does not technically need an
enclosing element (because at the outermost level it is a single element such
as a structure or sequence).

\begin{table}
\caption{Dataset Descriptor Structure Syntax}
\label{tab:DDS}
\small
\begin{center}
\begin{tabular}{ll} \hline
%
% This syntax comes straight from the bison source code
% (src/api.dds.y). Let's try *not* to change it, since reformatting is a
% royal pain. jhrg 11/9/94
%
{\em data sets\/} & {\em data set\/} \\
                 & {\em data sets} {\em data set\/} \\

{\em data set\/}  & {\tt dataset} {\tt \{} {\em declarations\/} {\tt \}} 
                   {\em  name\/} {\tt ;} \\

{\em declarations\/} & {\em declaration\/} \\
                 & {\em declarations} {\em  declaration\/} \\
                 &  $\delta$ \\

{\em declaration\/} &   {\em list} {\em declaration\/}  \\
                 & {\em base-type} {\em var\/} {\tt ;} \\
                 & {\em structure\/}  {\tt \{} {\em declarations\/} {\tt \}} 
                  {\em var\/} {\tt ;} \\
                 & {\em sequence\/} {\tt \{} {\em declarations\/} {\tt \}} 
                  {\em var\/} {\tt ;} \\
                 & {\em function\/} {\tt \{} {\tt independent} {\tt :}
                  {\em declarations} {\em dependent\/} {\tt :} 
                  {\em declarations\/} {\tt \}} 
                  {\em var\/} {\tt ;} \\
                 & {\em grid\/} {\tt \{} {\tt array} {\tt :} 
                  {\em declaration\/} {\tt maps} {\tt :} 
                  {\em declarations\/} {\tt \}} 
                  {\em var\/} {\tt ;} \\

{\em list\/}    & {\tt list} \\

{\em structure\/} & {\tt structure} \\

{\em sequence\/} & {\tt sequence} \\

{\em function\/} & {\tt function} \\

{\em grid\/}    & {\tt grid} \\

{\em base-type\/} & {\tt byte} \\ 
                  & {\tt int32} \\
                  & {\tt float64} \\
                  & {\tt string} \\
                  & {\tt url} \\

{\em var\/}     & {\tt id} \\
                & {\em var} {\em array-decl\/} \\

{\em array-decl\/} & {\tt $[$} {\tt integer} {\tt $]$} \\
                   & {\tt $[$} {\tt id} {\tt =} {\tt integer} {\tt $]$} \\

{\em name\/}    & {\tt id} \\
\end{tabular}
\end{center}
\normalsize
\end{table}

An example DDS entry is shown in Figure~\ref{fig:dds}. Suppose that three
experimenters have each performed temperature measurements at different
locations and at different times. This information could be held in a data set
consisting of a sequence of the experimenter's name, the time and location of
each measurement and the list of measurements themselves, and indicates that
there is a relation between the experimenter, location, time and temperature
called temp\_measurement.

\begin{figure}
\begin{code}{cb}
              data set {
                  int catalog_number;
                  function {
                    independent:
                      string experimenter;
                      int time;
                      structure {
                          float latitude;
                          float longitude;
                      } location;
                    dependent:
                      sequence {
                          float depth;
                          float temperature;
                      } temperature;
                  } temp_measurement;
              } data;
\end{code}
\caption{Example Dataset Descriptor Entry.}
\label{fig:dds}
\end{figure}

\subsection{Dataset Attribute Structure}
\label{api:das}

The Dataset Attribute Structure (DAS) is used to store attributes for
variables in the data set. An attribute is any piece of information about a
variabale that the creator wants to bind with that variable {\em excluding\/}
the characteristics type, shape, and units. The characteristics type, shape
and units are always defined for every variable; they are data type
information about the variable. Attributes, on the other hand, are intended
to store extra information about the data such as a paragraph describing how
it was collected or processed\footnote{To define attributes for the entire
  data set, create an entry for a variable with the same name as the data
  set.}. In principle attributes are not processed by software other than to
be displayed. However, many systems rely on attributes to store extra
information that is necessary to perform certain manipulations on data. In
effect, attributes are used to store information that is used `by convention'
rather than `by design'. DODS can effectively support these conventions by
passing the attributes from data set to user program via the DAS\@. Of course,
DODS cannot enforce conventions in data sets where they were not followed in
the first place.

The syntax for attributes is given in Table~\ref{tab:DAS}. Every attribute of
a variable is a triple: attribute name, type and value. Note that the
attributes specified using the DAS are different from the information
contained in the DDS\@. Each attribute is completely distinct from the name,
type and value of its associated variable. The name of an attribute is an
identifier, following the normal rules for an identifier in a programming
language with the addition that the `/' character may be used. The type of an
attribute may be one of: Byte, Int32, Float64, String or Url. An attribute
may be scalar or vector. In the later case the values of the vector are
separated by commas (,) in the textual representation of the DAS\@. 

When the \dap\ is used to read the attributes of a variable and that variable
contains other variables, only the attributes of the named variable are
returned. In other words, while the DDS is a hierarchical structure, the DAS
is {\em not\/}; it is similar to a flat-file database.

\begin{table}
\caption{Dataset Attribute Structure Syntax}
\label{tab:DAS}
\small
\begin{center}
\begin{tabular}{ll} \hline
{\em attributes\/} & {\em attribute\/} \\
                 & {\em attributes} {\em attribute\/} \\

{\em attribute\/} & {\tt attribute} {\tt \{} {\em var-attr-list\/} {\tt \}} \\

{\em var-attr-list\/} & {\em var-attr\/} \\
                 & {\em var-attr-list} {\em var-attr\/} \\
                 & $\delta$ \\

{\em var-attr\/} & {\em var-name} 
                {\tt \{} {\em attr-list\/} {\tt \}} \\

{\em attr-list\/} & {\em attr-pair\/} \\
                &  {\em attr-list} {\em attr-pair\/} \\
                & $\delta$ \\

{\em attr-pair\/} & {\em attr-type} {\em attr-name} {\em attr-val\/} 
                {\tt ;} \\

{\em var-name\/} & {\tt identifier} \\

{\em attr-name\/} & {\tt identifier} \\

{\em attr-val-vec\/} & {\em attr-val\/} \\
                     & {\em attr-val-vec\/} {\tt ,} {\em attr-val\/} \\

{\em attr-val\/} & {\tt value} \\
                 & {\tt identifier} \\
                 & {\tt string} \\

{\em attr-type\/} & {\tt Byte} \\
                  & {\tt Int32} \\
                  & {\tt Float64} \\
                  & {\tt String} \\
                  & {\tt Url} \\

\end{tabular}
\end{center}
\normalsize
\end{table}

\subsection{Data Access Protocol Entry Points}
\label{entry}

The DAP is a stateless protocol. Each of the DAP's entry points (i.e., the
messages a data server will respond to) does a single isolated job and they
can be issued in any order (although in many applications it will not make
sense to get the values for a variable before finding out the name of the
variable \ldots). The stateless nature of the DAP fits well within the
context of the data delivery system described in \DDA\@.  In that paper a
client-server architecture for remote access is described which relies on the
HTTPD/CGI mechanism to build a data server. One implementation of that
architecture is described in \DDD\ uses three CGI modules, one of each of the
three DAP entry points.

In this paper we talk about messages to the data server as if it is a
stateful server like {\tt ftpd}. However, that is merely a convient way to
phrase the discussion of the DAP---in fact the data server is a stateless
machine accessed by getting the value of a URL\@. 

Each DODS data server must respond to three URLs; one for each of the objects
(DAS, DDS and variable) which the server returns. These URLs are formed by
appending a suffix to the base URL which references the data set. In
addition, individual variables may be accessed using the constraint
expression mechanism described here. The paper \URL\ conatins more
information on this scheme.

Two messages are provided to access the data set descriptor structure (DDS)
and the data set attribute structure (DAS)\@. The response to these messages
is text formated using the respective grammars in
Tables~\ref{tab:DAS}~and~\ref{tab:DDS}. This text can then be parsed by the
caller to determine the structure of the data set, types and sizes of each of
its components and their attributes. These structures are derived both from
information contained in the data set and for ancillary information supplied
by the data set maintainers in separate text files (in the \ddd\ the origin
of these structures is described in detail). They provide information that is
often referred to as `metadata' and may be cached by the client system.
Future accesses to the same data set can then skip the retrieval of these
structures.

All variables are read from a data set individually using a single `read'
message. The message must include the name of the variable to read, and
optionally, may include an expression that describes the range of values
desired. No other information need be sent to the server. In response to this
message the value(s) of the variable(s) is/are then sent back to the client.
Base type variables return only a single value when accessed, but other types
may return more values. For example, a {\tt structure} with three {\tt int32}
and two {\tt float64} members will return five values (unless an expression
constrains the access).

One important capability of the DODS API, which it inherits from JGOFS, is
the ability to set constraints on variables (JGOFS calls this `using
selection and projection' operators).  Constraints are used to control the
values and/or members of constructor types that are returned when a variable
is accessed. For some data sets and some variables, constraints make little
or no difference in how the variables are accessed.  However, for certain
types of data, constraining access can vastly reduce the amount of data the
application needs to process and, in DODS case, transmit over the network.

\label{api:constraints}
Constraint expressions provide more flexibility in the way data is accessed.
In a data set with many interrelated variables, or with very large variables,
these expressions are a way for the user program to move some of the complex
logic needed to search data, in order to find those data with some desired
set of properties, away from the user program and into the data supply
system.  Different types of constraint expressions make sense for different
types of data. The grammar for constraint expressions is given in
Table~\ref{api:tab:expr}.  In the table {\em Operator\/} is one of the
variable-class dependent operators listed in Table~\ref{api:tab:class-ops},
{\tt !} is boolean {\em not\/} and {\tt \&} is boolean {\em and}.

\begin{table}
\caption{Constraint Expression Syntax}
\label{api:tab:expr}
\begin{center}
\begin{tabular}{ll} \hline

{\em expression\/} & {\em projection} {\em selection \/} \\

{\em projection\/} & {\em variable\/} \\
                   & {\em variable} {\tt ,} {\em projection\/} \\
                   & $\delta$ \\

{\em selection\/}  & {\em variable} {\em operator} {\em value\/} \\
                   & {\em variable} {\em operator} {\em variable\/} \\
                   & $\delta$ \\

{\em compound-sel\/} & {\em selction} {\em boolean-op} {\em compound-sel\/} \\ 
                     & {\tt !} {\em selection} {\em compound-sel\/} \\
                     & {\em selection\/} \\

{\em boolean-op\/} & {\tt \&} \\
\end{tabular}
\end{center}
\end{table}

\section{Conclusion}

The Data Access Protocol is used to access data sets within the DODS data
delivery system. It is not intended to be limited to DODS, but it is
specialized for the system. Among those specializations are the stateless
nature of the protocol. While the DAP could be used in many different
contexts, it is principally designed to be the transmission protocol used by
the data servers in the DODS data delivery system.

DODS supports access to data sets stored in a number of established, third
party APIs. In addition, it supports cross-access to those data sets. Thus
data in any one of the supported APIs is accessible using any one of the
supported APIs (at least in principle---some {\em possible\/} accesses may
wind up returning meaningless, or null, data). The DAP facilitates this cross
connections of user programs and data sets by providing a common
access protocol for each API\@. Each data server will provide access to data
using the DAP and each surrogate library (See \DDA\ and \DDD) will satisfy
the supported API's function calls with information obtained via the DAP\@.

To support its stateless interface, the DAP consists of two structures which
contain attribute and type information about the variables in the data set
and a single, programmable, mechanism for reading a variable. The attribute
and type information structures can be used to build powerful translators
which read variables in one type and transform them, on-the-fly, to types
suitable for the surrogate API\@. 

\clearpage

\addcontentsline{toc}{section}{Bibliography}
\bibliographystyle{plain}
\bibliography{../../boiler/dods}

\end{document}


