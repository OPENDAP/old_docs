
% IEEE 830-1998 SRS for the new DODS server.
% 1/11/2000 jhrg
%
% $Id$

\documentclass{article}

\usepackage{changebar}
\usepackage{acronym}
\usepackage{gloss}
\usepackage{xspace}
\usepackage{epsfig}

\newcommand{\na}{{\sc N/A}\xspace}

% Note: to get the gloosary to work, run bibtex in the *.gls.aux file,
% then latex registry, then bibtex *.gls, then latex. Also, make sure to set
% your BST and BIBINPUTS environment variables so that the BST and BIB files
% will be found.
\makegloss

% Change paragraph typesetting; eliminate indenting and add more space between
% paragraphs. 2/15/2000 jhrg
\setlength{\parindent}{0em}     % Amount of indentation
\addtolength{\parskip}{1ex}     % Vertical separation

\begin{document}

\title{DODS Server SRS}
\author{James Gallagher\thanks{The University of Rhode Island, jgallagher@gso.uri.edu}}
\date{\today \\ $Revision$ }

\bibliographystyle{plain}

\maketitle
\tableofcontents

%%%%%%%%%%%%%%%%%%%%%%%%%%%%%%%  Introdunction  %%%%%%%%%%%%%%%%%%%%%%%%%%%%

\section{Introduction}
\emph{[What is this document about, in very general terms.]}

This document conforms to the IEEE~830-1998 \ac{SRS} recommended practice.
Since the recommended practice covers a wide range of possible projects, some
of the information in it is not appropriate for this part of \acs{DODS}.
Where that is the case, or I think it is the case, I have marked the section
N/A.

\textbf{Bold face} type is used to indicate a word or phrase that may be
found in the glossary.

\emph{Emphasized} text contained in square brackets ([]) is used to indicate
an editorial comment about the information that should be provided in a part
of the \ac{SRS}.

\subsection{Purpose}
\emph{[Who is this document for?]}

\subsection{Scope}
\emph{[What part of the project does this SRS cover.]}

\subsection{Overview}

The remainder of this document is organized as follows:
\begin{enumerate}
\item Section~\ref{sec:overall} provides background for the specific
requirements and relates those requirements to the rest of DODS.
\item Section~\ref{sec:specific} lists the specific requirements for the
  Cache.
\item Following Section~\ref{sec:specific} are a list of acronyms and
  abbreviations, a change log, a glossary and references.
\end{enumerate}

%%%%%%%%%%%%%%%%%%%%%%%%%  Overall Description  %%%%%%%%%%%%%%%%%%%%%%%%%%%%%

\section{Overall Description}
\label{sec:overall}

\emph{[This section of the SRS should describe the general factors that
  affect the product and its requirements. This section should not state
  specific requirements. Instead, it provides a background for those
  requirements, which should be defined in Section 3.]}

\subsection{Product perspective}
DODS servers are currently implemented using a separate CGI for each
different type of data file or data source. If the server's
architecture is changed so that all the various data sources and/or
file types are accessed from the same CGI, the servers will be simpler
to manage and the common services of the servers will be simpler. 

\subsubsection{System interfaces}
\na
\subsubsection{User interfaces}
\paragraph{URLs}
The primary way users interact with DODS servers is though a
URL. Every DODS dataset is is uniquely refered to by its URL, known as
its \textbf{DODS URL}\onlygloss{dods-url}. Given a dataset contained
in a file whose pathname is \texttt{$<htdocs>$/data/nc/fnoc1.nc},
where \texttt{$<htdocs>$} is the document root directory of the host's
web server, and the DODS server's CGI script is installed in
\texttt{$<server root>$/cgi-bin/nph-dods}, the DODS URL for that
dataset is \texttt{http://$<machine>$/cgi-bin/nph-dods/data/nc/fnoc1.nc}.

In the current implementation of DODS servers, each different data
source uses a separate dispatch CGI, however, there's no reason for
this other than history. By changing the architecture of the servers
so that they use a single dispatch, every DODS dataset on a given
machine will have a common pathname from the protocol specification
down to the file component. This change will simplify the directory
service, making it much easier for users to navigate between datasets
using a web browser.

\paragraph{Coalesce the multiple components into one executable}
Currently DODS servers are comprised of a dispatch script and a
collection of separate executables, one for each possible server
response. Each of the exectuables contains mostly DAP library code;
only about 10\% of each executable is unique. If the functions of each
of the programs are combined into a single program, the size and
complexity of a DODS server will be reduced. 

While it is possible to load \emph{all} of the possible format
handlers into a single exectuable, the resulting program would be
huge. In addition, providers server data in all of our supported
formats. A compromise between combining all of the format handlers
versus keeping the current separate architecture is to combine the
three distinct handlers on a format by format basis. The executables
common to all of the servers should be combined into a single
executable as well.

Using this architecture, a DODS server will consist of one dispatch
program, one program to handle the responses common to all servers and
one exectuable for each format to which the server provides access.

Question: Can we use shared libraries and/or dynamic loadding on all
of our supported platforms? If so, how about combining the dispatch
with common response generation and loadding in the format handlers at
run time? This would cut a DODS server down to a single fork/exec.

\paragraph{The server configuration file}
Instead of using a specific CGI dispatch for each format, the server
should use a configuration file which maps regular expressions to
formats. The server will use this to decide which handler ought to be
used to process the dataset. 

\paragraph{HTML Directory page for the server}
Each server should have an HTML page that is stored in the $<htdocs>$
directory and that lists each dataset, matching a title and URL. The
intent of this HTML file is two-fold; to provide users with a simple
interface to the datasets held by a particular machine and to provide
web robots with a toe hold for the datasets. Commercial web robots
will probably make scant use of this, but a specialized robot could
use it to dig into the datasets and read variable names, types,
attributes and possibly some data values (to compute ancillary data
otherwise missing).

\subsubsection{Hardware interfaces}
\na
\subsubsection{Software interfaces}
CGI~1.1

HTTP~1.1

\subsubsection{Communications interfaces}
\na
\subsubsection{Memory}
\na
\subsubsection{Operations}
\na
\subsubsection{Site adaptation requirements}
As new file types are added, the server configuration file will have
to be edited. If the server maintainer forgets to do this, then the
server will not work with those new file types. This could confuse
users of the server as well as the dataset and server maintainers. To
counter this, the server must send an error message to the HTTPd error
log. Obviously, the message should be explicit enough so that the server
maintainer will understand how to fix the problem.

\subsection{Product functions}

\subsection{User characteristics}

\subsection{Constraints}
\emph{[This section lists any other items that may limit the developer's
  options.]}

\subsubsection{Regulatory policies}
\subsubsection{Hardware limitations}
\subsubsection{Interfaces to other applications}
\subsubsection{Parallel operations}
\subsubsection{Audit functions}
\subsubsection{Control functions}
\subsubsection{Higher-order language requirements}
\subsubsection{Signal handshake protocols}
\subsubsection{Reliability requirements}
\subsubsection{Criticality of the application}
\subsubsection{Safety and security considerations}

\subsection{Assumptions and dependencies}

%%%%%%%%%%%%%%%%%%%%%%%%%%%  Specific Requirements %%%%%%%%%%%%%%%%%%%%%%%%%

\section{Specific Requirements}
\label{sec:specific}
\emph{[This section of the \ac{SRS} lists all of the software requirements to
  a level of detail sufficient to enable designers to design a system to
  satisfy those requirements, and testers to test that the systems satisfies
  those requirements. Each separate requirment should be uniquely numbered.]}

\subsection{External interfaces}

\subsubsection{User interfaces}
\subsubsection{Hardware interfaces}
\subsubsection{Software interfaces}
\subsubsection{Communications interfaces}

\subsection{System features}
\subsection{Performance requirements}
\subsection{Design constraints}
\subsection{Software system attributes}
\subsection{Other requirements}

\appendix

\section{Acronyms and Abbreviations}
\begin{acronym}
%
% Make one entry per line, even if they are long lines that wrap and look
% ugly. This makes it simple to sort the list using emacs' sort-lines
% command. 3/27/2000 jhrg
%
% $Id$
\acro{AS}{Aggregation Server}
\acro{BNF} {Backus-Naur Form}
\acro{CE}{Constraint Expression}
\acro{CGI}{Common Gateway Interface}
\acro{COARDS}{Cooperative Ocean/Atmosphere Research Data Service}
\acro{CSV}{Comma Separated Values}
\acro{DAP}{Data Access Protocol}
\acro{DAS}{Dataset Attribute Structure}
\acro{DDS}{Dataset Descriptor Structure}
\acro{DODS}{Distributed Oceanographic Data System}, See the DODS home page: \texttt{http://\-unidata.ucar.edu\-/packages\-/dods/}
\acro{DataDDS}{Data Dataset Descriptor Structure}
\acro{FGDC}{Federal Geographic Data Community}
\acro{HTML}{Hypertext Markup Language}
\acro{HTTP}{HyperText Transfer Protocol}
\acro{MIME}{Multimedia Internet Mail Extensions}
\acro{SRS}{Software Requirements Specification}, See IEEE 830--1998
\acro{URI}{Uniform Resource Identifiers}
\acro{URL}{Uniform Resource Locator}
\acro{W3C}{The World Wide Web Consortium}, See http://www.w3c.org/
\acro{WWW}{The World Wide Web}
\acro{XDR}{External Data Representation}
\acro{XML}{Extensible Markup Language}
%%% Local Variables: 
%%% mode: latex
%%% TeX-master: t
%%% End: 

\end{acronym}

\section{Change log}

\begin{verbatim}
$Log: dods-server-srs.tex,v $
Revision 1.3  2000/07/21 15:21:42  jimg
*** empty log message ***

Revision 1.2  2000/07/20 21:35:58  jimg
Minor changes

Revision 1.1  2000/07/17 05:54:59  jimg
Added

\end{verbatim}

\printgloss{dods-glossary}

\raggedright

\bibliography{dods}

\end{document}
