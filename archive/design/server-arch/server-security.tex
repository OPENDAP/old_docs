

% Server security
% 6/29/2000 jhrg
%
% $Id$

\documentclass{article}
\usepackage{epsfig}
\usepackage{changebar}
\usepackage{acronym}
\usepackage{gloss}

% \psfigurepath{server-figs}

% Note: to get the gloosary to work, run bibtex in the registry.gls.aux file,
% then latex registry, then bibtex *.gls, then latex. Also, make sure to set
% your BST and BIBINPUTS environment variables so that the BST and BIB files
% will be found.
\makegloss

% Change paragraph typesetting; eliminate indenting and add more space between
% paragraphs. 2/15/2000 jhrg
\setlength{\parindent}{0em}     % Amount of indentation
\addtolength{\parskip}{1ex}     % Vertical separation

\begin{document}

\title{DODS Server Security SRS}
\author{James Gallagher\thanks{The University of Rhode Island,
    jgallagher@gso.uri.edu}}
\date{\today \\ $Revision$ }

\bibliographystyle{plain}

\maketitle
\tableofcontents

\section{Introduction}

This is the \ac{SRS} for client-side caching in \acs{DODS}. Client-side
caching provides performance improvements for datasets users access
frequently.

When a \textbf{DODS client}\onlygloss{dods-client} program accesses a
dataset, the documents it retrieves are stored in the client-side cache. The
cache is an \acs{HTTP}~1.1~\cite{w3c:http} compliant cache. It can store, in
principle, any document accessible through an HTTP server. DODS client
programs typically access three types of documents: the \acs{DAS}, \acs{DDS}
and \acs{DataDDS}. These correspond to the three main types of objects DODS
servers make available. Since these three objects are all encapsulated in
\acs{MIME}~\cite{rfc:mime} documents\footnote{Actually, an HTTP response in
  \emph{not} a valid MIME document, but once the first line is read the rest
  is valid MIME}, they can be cached by any HTTP~1.1 compliant caching
system.
  
The cache will be built using the \textbf{libwww}\onlygloss{libwww} software
package provided by the \ac{W3C}. This will free us from having to design and
implement the caching system ourselves. What we will have to do is decide
which features of the caching subsystem we want and how to configure libwww
to provide our clients with those features.

This document conforms to the IEEE~830-1998 \ac{SRS} recommended practice.
Since the recommended practice covers a wide range of possible projects, some
of the information in it is not appropriate for this part of \acs{DODS}.
Where that is the case, or I think it is the case, I have marked the section
N/A.

\textbf{Bold face} type is used to indicate a word or phrase that may be
found in the glossary.

\emph{Emphasized} text contained in square brackets ([]) is used to indicate
an editorial comment about the information that should be provided in a part
of the \ac{SRS}.

\subsection{Purpose}
This \ac{SRS} contains the requirements for the client-side caching. Its
audience is the entire \textbf{DODS core group}\onlygloss{dods-core-group}
and interested people outside that group.

\subsection{Scope}
The software to be developed is the client-side cache for \acs{DODS}.

\subsection{Overview}

The remainder of this document is organized as follows:
\begin{enumerate}
\item Section~\ref{sec:overall} provides background for the specific
requirements and relates those requirements to the rest of DODS.
\item Section~\ref{sec:specific} lists the specific requirements for the
  Cache.
\item Following Section~\ref{sec:specific} are a list of acronyms and
  abbreviations, a change log, a glossary and references.
\end{enumerate}

\section{Overall Description}
\label{sec:overall}

\emph{[This section of the SRS should describe the general factors that
  affect the product and its requirements. This section should not state
  specific requirements. Instead, it provides a background for those
  requirements, which should be defined in Section 3.]}

\subsection{Product perspective}

The most important features of the client-side cache are that it be fast and
that it return correct responses. Users expect network-aware software to be
smart enough to reuse previously accessed information when that is possible.
In addition, the cache \emph{must} return correct responses because to do
otherwise will undermine user's faith in DODS. This is particularly true
because DODS provides (at the present) no `control panel' that enables users
to turn off caching. Users can disable caching using a configuration file,
but such files are complicated and likely to be ignored. Thus the DODS
client-side cache must be fast but if it returns invalid responses, users are
likely to feel frustrated.

DODS clients receive information from \textbf{DODS
  servers}\onlygloss{dods-server} which are enhanced \textbf{HTTP
  servers}\onlygloss{http-server}.  Since a DODS server uses an HTTP server
for all the network operations, DODS servers, in theory, don't need to
support caching.  DODS clients currently support a client-side cache and the
HTTP implementation handles the caching operations. In its current
implementation, this cache assumes that any document is valid for 24 hours
after it was received. This is the simplest form of caching and is prone to
failure if, for example, a document changes more frequently than once per
day.

The DODS servers must be involved in any caching scheme that bases
expiration times or validation checks on the actual information being cached.

Caching as described in the HTTP~1.1 specification provides several features
which DODS should use. The HTTP~1.1 cache enables servers to set an
expiration date and time for each document. This is sent in the HTTP response
header. In addition, HTTP~1.1 provides for conditional requests where the
server can return either a complete document or a header only with a
directive to fetch the document from the client's cache\footnote{Of course,
  the client must only send a conditional request if it has a potential hit
  in the cached already.}. The latter is part of the HTTP~1.1 cache
validation system.

To make use of more advanced features of the HTTP~1.1 caching system some
changes will be required of the DODS servers. These changes are minor
relative to the benefits of client-side caching; the bulk of the caching
software we inherit from the \acs{W3C}'s libwww package.

The following sections describe how the software operates within various
constraints.

\subsubsection{System interfaces}
The principle system interface is the set of MIME headers in the request sent
to, and the response sent from, the DODS server . The DODS server will look
at these headers and be responsible for setting parameters in responses which
are used by the caching system. These include the date of the response, the
expiration time of response, etc. The DODS server interface is described
briefly in the DODS Programmer's Guide~\cite{DODS:prog-guide}.

\subsubsection{User interfaces}
\label{sec:ui}
\paragraph{The client-side configuration file}
The cache is configured using the \texttt{.dodsrc} file. Each user should
have this file in their home directory. DODS clients automatically should
create the configuration file with default values if it does not exist. The
file should be an ASCII file and may contain parameters that are not used by
the cache (i.e., it might be used by other parts of DODS). See
Figure~\ref{fig:config-file}.

\begin{figure}
\begin{center}
\begin{verbatim}
USE_CACHE=1
MAX_CACHE_SIZE=20
MAX_CACHED_OBJ=5
IGNORE_EXPIRES=0
CACHE_ROOT=/home/jimg/.dods_cache/
\end{verbatim}
\end{center}
\caption{A sample client-side cache configuration file.}
\label{fig:config-file}
\end{figure}

\paragraph{The server-side configuration file}
\label{sec:server-config}
The server-side configuration file must be unique for each DODS server at a
site where there are several different servers. This file should contain
information about expiration times for various responses. The file might
contain information that is not important for the client-side cache.

The server-side configuration files should ASCII files.

To ensure that the files are unique for each server, the files should be
named after both DODS and the particular server they configure. For a server
with the dispatch script \texttt{nph-<xxx>}, the server-side configuration
file should be named \texttt{dodsxxx.rc}. Note that these file are \emph{not
  invisible} in contrast to the client-side configuration files.

\subsubsection{Hardware interfaces}
N/A

\subsubsection{Software interfaces}
\paragraph{libwww, version 5.2.x}
libwww is the reference implementation of HTTP and is freely available from
the \ac{W3C}. The libwww cache interface provides a way to configure libwww's
cache. The client can change the cache's configuration based on the server
version and client configuration information using the libwww cache
interface. The libwww cache interface is described in the libwww
documentation.

\subsubsection{Communications interfaces}
\paragraph{HTTP~1.1} DODS clients and servers use HTTP for all their network
I/O.

\subsubsection{Memory}
Users should be able to place a limit on the maximum size of the cache since
the cache. The default maximum size of the cache should be 20MB.

\subsubsection{Operations}
N/A

\subsubsection{Site adaptation requirements}
Users must have both the client-side configuration file and the cache
directory tree. When a client first starts, it must check for these and
create them if they do not exist. In the case where they are created, the
configuration file should be initialized to the default parameters. These
are:
\begin{itemize}
\item use the cache: true
\item cache max size: 20 MB
\item item max size: 5 MB
\item ignore expires: false (meaning honor expires headers)
\item cache root: \$HOME/.dods\_cache (the root of the cache directory tree)
\end{itemize}

Several users may share a single cache directory tree but cached responses
are lock by individual users while they are being used. This is to ensure
consistency during (possible) updates due to validation.

Servers will also need configuration files, see
Section~\ref{sec:server-config}.

\subsection{Product functions}
Functions for the cache:
\begin{enumerate}
\item Store copies of responses in the cache.
\item Compare cached responses against expiration times.
\item Use remote servers to validate and/or update cached responses.
\item Retrieve responses from the cache.
\end{enumerate}

\subsection{User characteristics}
N/A

\subsection{Constraints}
\emph{[This section lists any other items that may limit the developer's
  options.]}

\subsubsection{Regulatory policies}
N/A

\subsubsection{Hardware limitations}
N/A

\subsubsection{Interfaces to other applications}
N/A

\subsubsection{Parallel operations}
N/A 

\subsubsection{Audit functions}
N/A 

\subsubsection{Control functions}
N/A 

\subsubsection{Higher-order language requirements}
N/A 

\subsubsection{Signal handshake protocols}
N/A 

\subsubsection{Reliability requirements}
The client-side cache should never return invalid responses when a properly
operating HTTP~1.1 compliant cache would have returned a correct response.

\subsubsection{Criticality of the application}
N/A 

\subsubsection{Safety and security considerations}
N/A

\subsection{Assumptions and dependencies}
\label{sec:assumptions}
The system designers can assume that the following are true:
\begin{enumerate}
\item Clients are linked with libwww 5.2.x or greater.
\item DODS servers less than version 3.2 provide neither validation services
  nor expiration times and the cache must use a default expiration time.
\end{enumerate}

\section{Specific Requirements}
\label{sec:specific}
\emph{[This section of the \ac{SRS} lists all of the software requirements to
  a level of detail sufficient to enable designers to design a system to
  satisfy those requirements, and testers to test that the systems satisfies
  those requirements.]}

\subsection{External interfaces}
\subsubsection{User interfaces}
The only user interface for the client-side cache is through configuration
files\footnote{Maybe we should add a button to the progress indicator that
  clears the cache?}. See Section~\ref{sec:ui}
and Figure~\ref{fig:config-file}. 

\subsubsection{Hardware interfaces}
N/A

\subsubsection{Software interfaces}
There are two software interfaces: MIME and \acs{CGI}~1.1\cite{w3c:cgi}. 

The MIME standard describes how information about both requests and responses
are passed to and from HTTP and DODS servers. The caching software is uses
the following MIME headers:
\begin{itemize}
\item Server
\item Date
\item Expires
\item Cache-Control
\item If-Modified-Since 
\end{itemize}

libwww is an implementation of HTTP~1.1. It is used by the client-side parts
of the DODS \ac{DAP}.

The CGI~1.1 standard describes how CGI programs are passed information from
the HTTP request (i.e., how the headers are passed to the CGI program).

\subsubsection{Communications interfaces}
The HTTP~1.1 standard describes how client programs interact with HTTP
servers. DODS uses only a subset of HTTP. We use only the GET method and not
POST, PUT, etc.

\subsection{System features}
In this part of the \ac{SRS} `3.1 servers' means any DODS server at or before
version 3.1.x and `3.2 servers' means any DODS servers at or greater than
version 3.2.0.

The requirements in this section are ranked 1--5, where 1 is the highest
importance and 5 of lowest importance.

\subsubsection{3.1 servers}
\begin{enumerate}
\item Because 3.1 servers provide no support for expiration times or
  validation, the cache must use a client-specified default expiration time.
  $(1)$
\item The default expiration time should be configurable on a per-client
  basis using the client-side configuration file. $(2)$
\item Different default expiration times may be listed in the default
  configuration file for various URLs using UNIX file globbing patterns. $(5)$
\end{enumerate}

\subsubsection{3.2 servers}
\paragraph{Expiration}
\begin{enumerate}
\item 3.2 servers will send an expiration header with all responses. $(3)$
\item For any response which should not be cached, the expiration time will
  be the current time plus the response will contain a \texttt{Cache-Control:
    no-cache} header. $(3)$
\item The expiration times for various responses from a specific server must
  be configurable. The configuration should enable a person who installs or
  maintains a server to set specific expiration times for groups of
  responses, individual responses and a default. $(3)$
\item The expiration times should be specified using the server-side
  configuration file. They should use regular expressions or UNIX file
  globbing to describe groups of responses. $(5)$
\item Expiration header values should be set using a conservative algorithm
  so that a client's cache never (or hardly ever?) will treat a cache entry
  as valid when it is in fact bad. $(3)$
\end{enumerate}

\paragraph{Validation}
\begin{enumerate}
\item 3.2 servers must support conditional requests which use the
  \texttt{If-Modified-Since} header. ``A \texttt{GET} method with an
  \texttt{If-Modified-Since} header and no \texttt{Range} header requests
  that the identified entity be transferred only if it has been modified
  since the date given by the \texttt{If-Modified-Since}
  header''\cite{w3c:http}. $(3)$
\item Servers must support \texttt{Last-Modified} headers since these are
  requied for conditional requests. (3)
\item A more advanced server could set the value of the
  \texttt{Last-Modified} header based on the constraint since one variable in
  a dataset might have been modified more recently than others. If a client
  is only interested in an older variable, it should be able to use a cached
  value if one exists, even if some other variables have changed more
  recently.\footnote{Thanks to Benno Blumenthal for bringing
    \texttt{Last-Modified} to my attention.} (5)
\item Clients which communicate with 3.2 servers may use conditional requests
  for any entity which they have previously cached, even if that cache entry
  has gone stale. $(3)$
\item 3.2 servers which serve data stored in files should use file
  modification times to answer a conditional \texttt{GET}. $(3)$
\item 3.2 servers which serve data from stores other than files (e.g., RDBs)
  should use the latest modification time for the requested data. These
  servers should do the best they can, always erring on the side of not
  validating an old entry. $(4)$
\item Validation for time series data should take into account that older
  data may be reprocessed so actual modification times should be used, not
  the times associated with data collection. $(4)$
\end{enumerate}

\paragraph{Cached Data}
Data responses can be cached just as all other responses. However, because
one data request might be subsumed by another, the HTTP~1.1 exact-match
algorithm for detecting cache hits will result in unnecessary network
accesses. 
\begin{enumerate}
\item When a cache-aware client makes a data request, it should determine if
  a previously received response can be used to satisfy that request. $(4)$
\item To reduce the complexity of the cache, the client should only check for
  older, cached, data responses that completely subsume the current request.
  $(4)$
\item When a subsuming response is found, the client must read that response
  and use it to fulfill the request. $(4)$
\end{enumerate}

\subsection{Performance requirements}
\subsection{Design constraints}
\subsection{Software system attributes}
\subsection{Other requirements}

\appendix

\section{Acronyms and Abbreviations}
\begin{acronym}
%
% Make one entry per line, even if they are long lines that wrap and look
% ugly. This makes it simple to sort the list using emacs' sort-lines
% command. 3/27/2000 jhrg
%
% $Id$
\acro{AS}{Aggregation Server}
\acro{BNF} {Backus-Naur Form}
\acro{CE}{Constraint Expression}
\acro{CGI}{Common Gateway Interface}
\acro{COARDS}{Cooperative Ocean/Atmosphere Research Data Service}
\acro{CSV}{Comma Separated Values}
\acro{DAP}{Data Access Protocol}
\acro{DAS}{Dataset Attribute Structure}
\acro{DDS}{Dataset Descriptor Structure}
\acro{DODS}{Distributed Oceanographic Data System}, See the DODS home page: \texttt{http://\-unidata.ucar.edu\-/packages\-/dods/}
\acro{DataDDS}{Data Dataset Descriptor Structure}
\acro{FGDC}{Federal Geographic Data Community}
\acro{HTML}{Hypertext Markup Language}
\acro{HTTP}{HyperText Transfer Protocol}
\acro{MIME}{Multimedia Internet Mail Extensions}
\acro{SRS}{Software Requirements Specification}, See IEEE 830--1998
\acro{URI}{Uniform Resource Identifiers}
\acro{URL}{Uniform Resource Locator}
\acro{W3C}{The World Wide Web Consortium}, See http://www.w3c.org/
\acro{WWW}{The World Wide Web}
\acro{XDR}{External Data Representation}
\acro{XML}{Extensible Markup Language}
%%% Local Variables: 
%%% mode: latex
%%% TeX-master: t
%%% End: 

\end{acronym}

\section{Change log}

\begin{verbatim}
$Log: server-security.tex,v $
Revision 1.1  2000/10/30 18:11:25  jimg
*** empty log message ***

Revision 1.8  2000/05/12 19:30:55  jimg
*** empty log message ***

Revision 1.7  2000/03/31 22:26:27  jimg
Added information about Last-Modified headers from Benno Blumenthal.

Revision 1.6  2000/03/27 21:30:56  jimg
Changed title. !!!

Revision 1.5  2000/03/27 17:27:08  jimg
Fixed various language goofs. Cleared up some of the material in sections 1
and 2.

Revision 1.4  2000/03/25 03:35:48  jimg
Changes, references, etc.

Revision 1.3  2000/03/22 23:56:37  jimg
Complete, but not proofread, version.

Revision 1.2  2000/03/22 05:49:41  jimg
Nearly complete...

Revision 1.1  2000/03/22 00:09:11  jimg
First version.

\end{verbatim}

\printgloss{dods-glossary}

\raggedright

\bibliography{dods}

\end{document}