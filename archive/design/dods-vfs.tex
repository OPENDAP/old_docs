
% $Id$
%
% Written by Glenn Flierl in early 1994. jhrg

% Hi James - you said you were studying TeX - this is the header which
% I generally include (usually with \input ms.def) with my files.
% regards, grf
%
% PS - if you're running TeX, I recommend my version of xdvi, which
% displays the final document. It has a ``ReTeX'' button which allows
% you to edit in an emacs window, save, and by ReTeXing, see the
% revisions in the display window.  regards, grf
%

\magnification=\magstep1
\raggedbottom
\def\title#1{\bigskip\goodbreak\centerline{\bf #1}\medskip}
\def\section#1{\medskip\goodbreak\noindent{\bf #1}\medskip}
\def\subsection#1{\medskip\goodbreak\noindent{\sl #1}\medskip}
\def\references{\narrower\parindent=-20pt}
\def\jour#1#2{{\sl #1}, {\bf #2}}
\def\ni{\noindent}

\title{Virtual File Systems/ Databases}

Files systems are a particular form of data system, with some
strengths and some weaknesses. Since we've begun talking seriously
about a virtual files system/ virtual datasystem, we should assess
carefully what it gains us and how to get around its disadvantages.

\section{File Systems}

A simple file system (e.g. MSDOS) allows organizing information in a
tree structure with the leaves (files) being opaque objects. This is
often a good way to structure a data set; e.g., if we consider a
GLOBEC cruise, it might be organized

{\obeylines
cruise1
~~ctd
~~~~ctd001
~~~~ctd002
~~~~ctd003
~~~~...
~~nettows
~~~~mocness01
~~~~mocness02
~~~~modness03
~~~~...
cruise2
...
}

\ni Navigating through this structure is pretty intuitive and users would
find a particular piece of information quickly. But complex queries -
find the ctd's and associated nettow's at 40N, 70W -- would be difficult
indeed (that's why a file system is not a database). With NFS, the
different directories could be located on different machines, but must
be explicitly cross--mounted.

\subsection{Virtual File System}

As I understand Neumann's work, a virtual file system allows the users
to define their own directory organization for the files. So another
user might see the same files as in a structure


{\obeylines
ctd
~~cruise1
~~~~ctd001
~~~~ctd002
~~~~ctd003
~~~~...
~~cruise2
~~...
nettows
~~cruise1
~~~~mocness01
~~~~mocness02
~~~~modness03
~~~~...
~~cruise2
~~...
}

\ni or

{\obeylines
ctd
~~ctd001
~~ctd002
~~ctd003
~~...
nettows
~~mocness01
~~mocness02
~~modness03
~~...
...
}

\ni In UNIX, one could conceive of doing this by building a new
directory structure with the leaves being links to the real files.
Prospero, in addition to making this kind of restructuring relatively
painless, adds a couple of features: (1) simpler linking to other
machines with no need for mounting the remote filesystem, (2)
association of attributes with directories (and perhaps files?) which
can also be used for automatic grouping. For example, if the files in
the above example had the PI's name as one of their attributes, one
could then construct a view of the data based on PI as the topmost
level in the tree. I'm not sure, but is seems as though this feature
could be of considerable use -- perhaps with some extension.

\subsection{What does a virtual file system gain us?}

\item{1} A file system gives a familiar structure to a data system,
one which everyone will be comfortable navigating around. Adding extra
information (attributes, descriptors, ... up to hypertext) will make
it easier.
\item{2} Reading data from a file, either with {\tt more} or a program
is a common well--understood operation. This is certainly an important
aspect. But allowing this to occur smoothly will become a crucial
element differentiating DODS from previous systems.
\item{3} Allowing the users to make their own views and structure for
the data as their research evolves should be of help for all.

\section{Towards a Data System}

What elements are missing from a virtual system? Our requirements
include 
\item{1} Locating data --- the virtual file system can be a great aid
in this, though it does not fully satisfy the requirements. We need
also a search mechanism that would build a virtual directory of data
sets satifying some specified conditions.
\item{2} Selecting data --- although the virtual file system allows
selecting at the level of a file, we clearly need to able to get
either less or more than that: in some cases, we may want only part of
what is contained in a file, and in other cases we want information
collected from many files.
\item{3} Retrieving data --- here the file system is not nearly as
functional as we require. A file is basically opaque - the user cannot
have any {\sl a priori} knowledge about its structure. In contrast we
would ideally like a MATLAB user, for example, to be able to behave as
though the entire system were an extended set of MATLAB files tailored
by the specific selections/projections made.

\ni How can we integrate these elements smoothly with a virtual file
system? Let's consider each element, though not necessarily in order.

\subsection{Retrieving data}

I think an extension of the Mosaic URL notion probably would do the
job here: a resource is specified by a prefix indicating the protocol.
Thus a dataset could be read by a netcdf program by calling it

{\tt
netcdf:stuff
}

\ni whereas it could be typed directly using

{\tt
dods:stuff
}

\ni or put into Mosaic with

{\tt
html:stuff
}

\ni Note: this usage is somewhat different from Mosaic's. They
interpret the protocol: within Mosaic and cause different kinds of
communication to occur, based on some set they've chosen to implement.
The translation to htmp is done within the client. What I'm proposing
is that this translation occur within the ``translator'' module, {\tt
html} or {\tt netcdf}. In the first case, it may understand various
protocols such as gopher, WWW, and the anonymous ftp protocol. The
client now only has to read html format; if it runs into something
else, it could span a suitable viewer.

The translators (and the telephone operators) are responsible for
negotiating the protocols between the server and the translator to
find the most efficient transmission method. Figure 1 shows the idea.

\subsection{Locating data/ Subselecting data}

In our data model, we had the idea of variable names and values for
those variables. The fundamental database operations worked to select/
project/ merge based on those names. At first sight, the directory
tree structure doesn't seem to fit within that model. But with a bit
of associated information (attached as are the Prospero attributes)
one could map back and forth between a directory structure and a
tree--organized database.

{\obeylines
cruise=1
~~instrument=ctd
~~~~station=001
~~~~station=002
~~~~station=003
~~~~...
~~instrument=nettows
~~~~tow=01
~~~~tow=02
~~~~tow=03
~~~~...
cruise=2
...
}

\ni This correspondence suggests that we can think of specifying a
filename by its path can be considered a particular form of database
operation --- $v1=x1~\&~v2=x2~\&~v3=x3...$

But we also need to express more complex selections. While the URL
syntax allows postfixing a selection (signalled by a '?'), this is not
a very robust in that you may not be able to group multiple
selections. My own preference is for a function notation

{\tt
stuff(cruise=1\&station>=2\&station<=5)
}

\section{Virtual Database}

I think we should aim for a system which melds the virtual file system
with a data system, giving us the advantages of familiar structuring
and retrieval with the capability for more complex selections. A
syntax such as

{\tt
trans:virtual\_path(arguments)
}

\ni can be used. For example consider how a request

{\tt
more ascii://plankton.whoi.edu/globec/cruise1/ctd(station=5|station=7)
}

\ni might be resolved. The ``open'' call from {\tt more} to the DODS
library/ file system causes the ``ascii'' translator to be loaded and
connects to the operator on ``plankton.'' That walks through the
directory structure until it finds the appropriate ``file'' and loads
its server and passes the arguments to it. The server negotiates with
the translator to find the preferred interchange protocol and then
begins transmitting data in response to ``read'' requests.

\end

