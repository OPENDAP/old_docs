%
% This file contains information on URLs. It describes how the different
% parts of DODS use URLs as well as the structure DODS assumes those URLs to
% have. 
%
% $Id$
%

\documentstyle[code,12pt,html,psfig,margins]{article}

%\psfigurepath{url-figs}

\begin{document}

\title{DODS---Uniform Resource Locators}
\author{}
\date{\today}

\maketitle

\begin{abstract}

  This document describes the syntax used by the Distributed Oceanographic
  Data System (DODS) to reference objects on the Internet. The syntax is
  based on the Uniform Resource Locator syntax used by the World Wide Web
  system designed at \CERN\@. In addition to the syntax description, an
  overview of the use of resource locators by the main components of the
  system is given.

\end{abstract}

\newpage

\section{Uses of URLs by DODS}

The Data Delivery Mechanism consists of libraries with which users re-link
their programs as well as stand alone utilities built at least partially from
those libraries. URLs are used verbatim by the data delivery components of
DODS\@. The URL is passed to the server or translator using the API {\em
  open\/} call (e.g., {\tt ncopen()}). The client library stub is responsible
for removing the envelope of the URL and sending either the path or embedded
URL to the correct host/server or host/translator.

When a user wants to access data via a DODS server they must give to a client
(i.e., a program linked with one of the DODS reimplemented API libraries) A
URL which references that data. Users may choose to supply a constraint along
with that URL effectively limiting the parts of the data set that the client
can see. For example, suppose a data set exists which contains several
arrays:

\begin{code}{cb}
    Dataset {
        Int32 u[time_a = 16][lat = 17][lon = 21];
        Int32 v[time_a = 16][lat = 17][lon = 21];
        Float64 lat[lat = 17];
        Float64 lon[lon = 21];
        Float64 time[time = 16];
    } fnoc1;
\end{code}

\begin{code}{cb}
    Dataset {
        Int32 u[time_a = 1][lat = 17][lon = 21];
    } fnoc1;
\end{code}

The user can refer to the entire dataset using a URL without a constraint
expression;
\htmladdnormallink{http://dods.gso.uri.edu/cgi-bin/nc/data/fnoc1.nc}
{http://dods.gso.uri.edu/cgi-bin/nc/data/fnoc1.nc.dds}. If the user were to
specify this, then the client-library will receive from the server a DDS like
the one in Figure~\ref{url:fig:dds}. However, if the user knows that they
only want to work with a small part of the data set they can supply a
constraint expression along with the URL\@. For example, suppose that the
user only wants the latitude and longitude values for $time_a = 2$, and
furthermore that they are only interested in the array $u$. They could supply
the user program with the following URL:
\htmladdnormallink{http://dods.gso.uri.edu/cgi-bin/nc/data/fnoc1.nc?u[2:2][1:17][1:21]}
{http://dods.gso.uri.edu/cgi-bin/nc/data/fnoc1.nc?u[2:2][1:17][1:21].dds} The
user program will receive DDS in Figure~\ref{url:fig:dds2}. For program which
are designed to read the entire dataset without user interaction this is a
powerful additional feature.

\end{document}
