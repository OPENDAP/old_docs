
% Another take on the registry SRS. This time follow IEEE 830-1998.
% 1/11/2000 jhrg
%
% $Id$

\documentclass{article}
%\usepackage{html}
\usepackage{psfig}
\usepackage{changebar}
\usepackage{acronym}
\usepackage{gloss}
%%
% These are html links which are used often enough in writing about DODS to
% merit an input file.
% jhrg. 4/17/94
%
% File rationalized and updated while writing the DODS User
% Guide. Also includes other useful abbreviations.
% tomfool 3/15/96
%
% $Id$
%
% HTML references to DODS documents
%
% For most of the DODS documentation there are three html references defined:
% 1) The upper case \newcommand produces the title of the paper, an
%    html link in the online documentation and a footnote in the paper
%    version.  
% 2) A capitalized version produces a capitalized description of the
%    paper and a link in the online version (but no footnote in the
%    paper version). 
% 3) A lower case version which produces a lower case description of
%    the paper and a html link, but no footnote.

% External references to documents:
%
% In order for external references to work (via latex2html) the labels used
% throughout the documents must be unique. The text labels are all prefixed
% with a document identifier (e.g., ddd) with a colon separater. The
% \externalrefs below (way below...) includes the perl files html.sty uses to
% make the cross refs.

\newcommand{\DODS}{\htmladdnormallink{Distributed Oceanographic Data System}
  {http://www.unidata.ucar.edu/packages/dods/}}

\newcommand{\Dods}{\htmladdnormallink{DODS}{http://www.unidata.ucar.edu/packages/dods/}}

\newcommand{\wrkshp}{\htmladdnormallink{Report on the First Workshop for the
    Distributed Oceanographic Data System, Proposed System Architectures}
  {http://www.unidata.ucar.edu/packages/dods/archive/reports/workshop1/section3.13.html}}

% not DODS URLs, but useful all the same...

\newcommand{\www}{\htmladdnormallink{World Wide Web} 
  {http://www.w3.org/hypertext/WWW/TheProject.html}}
\newcommand{\url}{\htmladdnormallink{Uniform Resource Locator}
  {http://www.w3.org/hypertext/WWW/Addressing/URL/url-spec.html}}

\newcommand{\uri}{\htmladdnormallinkfoot{Uniform Resource Identifiers}
  {http://www.w3.org/hypertext/WWW/Addressing/URL/uri-spec.html}}

\newcommand{\urn}{\htmladdnormallinkfoot{Uniform Resource Name}
  {http://www.acl.lanl.gov/URI/archive/uri-archive.messages/1143.html}}

\newcommand{\HTTPD}{\htmladdnormallinkfoot{HTTPD}
  {http://hoohoo.ncsa.uiuc.edu/docs/Overview.html}}

\newcommand{\HTTP}{\htmladdnormallinkfoot{HyperText Transfer Protocol}
  {http://www.w3.org/hypertext/WWW/Protocols/HTTP/HTTP2.html}}

%\newcommand{\HTML}{\htmladdnormallinkfoot{HyperText Markup Language}
%  {http://www.w3.org/hypertext/WWW/MarkUp/MarkUp.html}}

% CERN used to be `Conseil Europeen pour la Recherche Nucleaire'
\newcommand{\CERN}
  {\htmladdnormallink{European Laboratory for Particle Physics}
  {http://www.cern.ch/}}

\newcommand{\NCSA}
  {\htmladdnormallink{National Center for Supercomputing Applications}
  {http://www.ncsa.uiuc.edu/}}

\newcommand{\WWWC}{\htmladdnormallink{World Widw Web Consortium}
  {http://www.w3.org/}}

\newcommand{\CGI}{\htmladdnormallinkfoot{Common Gateway Interface}
  {http://hoohoo.ncsa.uiuc.edu/cgi/overview.html}}

\newcommand{\MIME}{\htmladdnormallink{Multipurpose Internet Mail Extensions}
  {http://www.cis.ohio-state.edu/htbin/rfc/rfc1590.html}}

\newcommand{\netcdf}{\htmladdnormallink{NetCDF}
  {http://www.unidata.ucar.edu/packages/netcdf/guide.txn_toc.html}}

\newcommand{\JGOFS}{\htmladdnormallink{Joint Geophysical Ocean Flux Study}
  {http://www1.whoi.edu/jgofs.html}}

\newcommand{\jgofs}{\htmladdnormallink{JGOFS}
  {http://www1.whoi.edu/jgofs.html}}

\newcommand{\hdf}{\htmladdnormallink{HDF}
  {http://www.ncsa.uiuc.edu/SDG/Software/HDF/HDFIntro.html}}

\newcommand{\Cpp}{{\rm {\small C}\raise.5ex\hbox{\footnotesize ++}}}

% Commands

\newcommand{\pslink}[1]{\small
\begin{quote}
  A \htmladdnormallink{postscript}{#1} version of this document is
  available.  You may also use \htmladdnormallink{anonymous
  ftp}{ftp://ftp.unidata.ucar.edu/pub/dods/ps-docs/} to access postscript files
  of all of the DODS documentation.
\end{quote}
\normalsize
}

\newcommand{\declaration}[1]{\small
{\tt {#1}}
\normalsize
}

% All the links from here down are for the white papers. I'm not sure that any
% of these are still valid given the reorganization of the web site. 5/12/98
% jhrg.

\newcommand{\DDA}{\htmladdnormallinkfoot{DODS---Data Delivery Architecture}
  {http://www.unidata.ucar.edu/packages/dods/archive/design/data-delivery-arch/data-delivery-arch.html}}
\newcommand{\Dda}{\htmladdnormallink{Data Delivery Architecture}
  {http://www.unidata.ucar.edu/packages/dods/archive/design/data-delivery-arch/data-delivery-arch.html}}
\newcommand{\dda}{\htmladdnormallink{data delivery architecture}
  {http://www.unidata.ucar.edu/packages/dods/archive/design/data-delivery-arch/data-delivery-arch.html}}

\newcommand{\DDD}{\htmladdnormallinkfoot{DODS---Data Delivery Design}
  {http://www.unidata.ucar.edu/packages/dods/archive/design/data-delivery-design/data-delivery-design.html}}
\newcommand{\Ddd}{\htmladdnormallink{Data Delivery Design}
  {http://www.unidata.ucar.edu/packages/dods/archive/design/data-delivery-design/data-delivery-design.html}}
\newcommand{\ddd}{\htmladdnormallink{data delivery design}
  {http://www.unidata.ucar.edu/packages/dods/archive/design/data-delivery-design/data-delivery-design.html}}

\newcommand{\URL}{\htmladdnormallinkfoot{DODS---Uniform Resource Locator}
  {http://www.unidata.ucar.edu/packages/dods/archive/design/urls/urls.html}}
\newcommand{\Url}{\htmladdnormallink{Uniform Resource Locator}
  {http://www.unidata.ucar.edu/packages/dods/archive/design/urls/urls.html}}
\newcommand{\dodsurl}{\htmladdnormallink{uniform resource locator}
  {http://www.unidata.ucar.edu/packages/dods/archive/design/urls/urls.html}}

\newcommand{\DAP}{\htmladdnormallinkfoot{DODS---Data Access Protocol}
  {http://www.unidata.ucar.edu/packages/dods/archive/design/api/api.html}}
\newcommand{\Dap}{\htmladdnormallink{Data Access Protocol}
  {http://www.unidata.ucar.edu/packages/dods/archive/design/api/api.html}}
\newcommand{\dap}{\htmladdnormallink{data access protocol}
  {http://www.unidata.ucar.edu/packages/dods/archive/design/api/api.html}}

\newcommand{\SOFT}
        {\htmladdnormallinkfoot{DODS---Software Development Environment}
  {http://www.unidata.ucar.edu/packages/dods/archive/managment/software/software.html}}
\newcommand{\Soft}{\htmladdnormallink{Software Development Environment}
  {http://www.unidata.ucar.edu/packages/dods/archive/managment/software/software.html}}
\newcommand{\soft}{\htmladdnormallinkfoot{software development environment}
  {http://www.unidata.ucar.edu/packages/dods/archive/managment/software/software.html}}

% I used to call the DAP the API, then for a few days I called it the
% DTP. These def's were used then. Rather than find everywhere they are used
% (not that hard, but...) I just hacked the macros. NB: the source file for
% the DAP document is still called `api.tex'

\newcommand{\API}{\htmladdnormallinkfoot{DODS---Data Access Protocol}
  {http://www.unidata.ucar.edu/packages/dods/archive/design/api/api.html}}
\newcommand{\api}{\htmladdnormallink{data access protocol}
  {http://www.unidata.ucar.edu/packages/dods/archive/design/api/api.html}}

\newcommand{\DTP}{\htmladdnormallinkfoot{DODS---Data Access Protocol}
  {http://www.unidata.ucar.edu/packages/dods/archive/design/api/api.html}}
\newcommand{\dtp}{\htmladdnormallink{data access protocol}
  {http://www.unidata.ucar.edu/packages/dods/archive/design/api/api.html}}

\newcommand{\TOOLKIT}{\htmladdnormallinkfoot{DODS---Client and Server Toolkit}
  {http://www.unidata.ucar.edu/packages/dods/archive/implementation/toolkits/toolkits.html}}
\newcommand{\Toolkit}{\htmladdnormallink{Client and Server Toolkit}
  {http://www.unidata.ucar.edu/packages/dods/archive/implementation/toolkits/toolkits.html}}
\newcommand{\toolkit}{\htmladdnormallink{client and server toolkit}
  {http://www.unidata.ucar.edu/packages/dods/archive/implementation/toolkits/toolkits.html}}

\newcommand{\TKR}{\htmladdnormallinkfoot{DODS---DAP Toolkit Reference}
  {http://www.unidata.ucar.edu/packages/dods/archive/implementation/toolkits/toolkits.html}}
\newcommand{\Tkr}{\htmladdnormallink{DAP Toolkit Reference}
  {http://www.unidata.ucar.edu/packages/dods/archive/implementation/toolkits/toolkits.html}}
\newcommand{\tkr}{\htmladdnormallink{DAP toolkit reference}
  {http://www.unidata.ucar.edu/packages/dods/archive/implementation/toolkits/toolkits.html}}

% I know that these are wrong. The papers have been moved to the gso web
% site, but I'm not sure exactly where. 5/12/98 jhrg

\newcommand{\DM}{\htmladdnormallinkfoot{DODS---Data Model}
  {http://lake.mit.edu/dods-dir/dodsdm2.html}}
\newcommand{\Dm}{\htmladdnormallink{Data Model}
  {http://lake.mit.edu/dods-dir/dodsdm2.html}}
\newcommand{\dm}{\htmladdnormallink{data model}
  {http://lake.mit.edu/dods-dir/dodsdm2.html}}

\newcommand{\DD}{\htmladdnormallinkfoot{DODS---Data Delivery}
  {http://lake.mit.edu/dods-dir/dods-dd.html}}

% external refs for DODS documents

\externallabels{http://www.unidata.ucar.edu/packages/dods/design/api}
  {/home/www/packages/dods/archive/design/api/labels.pl}
\externallabels{http://www.unidata.ucar.edu/packages/dods/design/data-delivery-arch}
  {/home/www/packages/dods/archive/design/data-delivery-arch/labels.pl}
\externallabels{http://www.unidata.ucar.edu/packages/dods/design/data-delivery-design}
  {/home/www/packages/dods/archive/design/data-delivery-design/labels.pl}
\externallabels{http://www.unidata.ucar.edu/packages/dods/implementation/toolkits}
  {/home/www/packages/dods/archive/implementation/toolkits/labels.pl}

%\renewcommand{\externalref}[2]{#2}

%%% Local Variables: 
%%% mode: latex
%%% TeX-master: t
%%% End: 

\psfigurepath{registry-figs}

% Note: to get the gloosary to work, run bibtex in the registry.gls.aux file,
% then latex registry, then bibtex *.gls, then latex. Also, make sure to set
% your BST and BIBINPUTS environment variables so that the BST and BIB files
% will be found.
\makegloss

% Change paragraph typesetting; eliminate indenting and add more space between
% paragraphs. 2/15/2000 jhrg
%\setlength{\parindent}{0em}     % Amount of indentation
%\addtolength{\parskip}{1ex}     % Vertical separation

\begin{document}

\title{DODS Dataset Registry}
\author{James Gallagher\thanks{The University of Rhode Island,
    jgallagher@gso.uri.edu}}
\date{\today \\ $Revision$ }

\bibliographystyle{plain}

\maketitle
\tableofcontents

\section{Introduction}

This is the \ac{SRS} for the dataset Registry. The Registry provides a way
for data providers to tell the DODS project about their dataset(s). This
inforamtion may be used by the project in a number of ways; building lists of
datasets for users to search~\cite{gallagher:list-srs}, sending information
about DODS datasets to other projects, etc.

\textbf{Bold face} type is used to indicate a word or phrase that may be
found in the glossary.

\emph{Emphasized} text contained in square brackets ([]) is used to indicate
an editorial comment about the information that should be provided in this
part of the \ac{SRS}.

% This contains the requirements for the DODS Dataset registry. This document
% and its companions---the DODS File server \ac{SRS} and the DODS Dataset List
% \ac{SRS}---describe pieces of a general purpose interface that should
% work for all DODS datasets. This interface should provide access to any DODS
% data, as either ASCII or binary, from computers running UNIX, Mac or MS
% operating systems. 

\subsection{Purpose}
This \ac{SRS} contains the requirements for the Dataset Registry. Its
audience is the entire \textbf{DODS core group}
\gloss[nocite]{dods-core-group}.

\subsection{Scope}
The software to be produced is the Dataset Registry.

The Registry will provide a way for dataset URLs to be recorded at one site.
Each time a DODS server is installed or a new dataset is added to an existing
server, the \textbf{data provider} \gloss[nocite]{data-prov} will be able
to register that dataset using the Registry.

\subsection{Overview}

The Registry requirements are discussed in this document.

This document conforms to the IEEE 830-1998 \ac{SRS} recommended practice.
Since that document covers a wide range of possible projects, some of the
information in it is not appropriate for this part of DODS. Where that is the
case, or I think it is the case, I have marked the section N/A.

\section{Overall Description}

\subsection{Product perspective}

Part of the philosophy of \acs{DODS} is that data providers should have as
much freedom when installing a \acs{DODS} server as do people who install
{\tt ftp} or {\tt http} servers.  There is no requirement that
\textbf{\acs{DODS} servers} \gloss[nocite]{dods-server} be registered or
authorized in any way.  However, this system-wide requirement does not mean
that additional capabilities cannot be made available by a registration
subsystem, it only means that users cannot be forced to use such a system.

The Registry is a web-based software tool. It will hold URLs to DODS datasets
and provide access to that information so that the List may be automatically
built. The List software will use output from the Registry. In addition to
direct input from a user, the Registry may interact with DODS servers (for
example, to get status information or to verify the reference).  The intent
of the Registry is to provide a semi-automated way to maintain the List and
to provide data providers with a way to add their DODS servers to the List.

Figure~\ref{fig:prod-perspective} shows how the Registry relates to other
parts \acs{DODS}.

\begin{figure}
\begin{center}
\psfig{file=prod-perspective.ps,width=5in}
\end{center}
\caption{Relation of the Registry software to other relevant parts
  of DODS.}
\label{fig:prod-perspective}
\end{figure}

The following sections describe how this software operates within various
constraints.

\subsubsection{System interfaces}
For the Registry, the system interfaces are:
\begin{itemize}
\item An HTML/form for input of information.
\item An undetermined interface for passing information to the List system.
\end{itemize}

\subsubsection{User interfaces}
The user interface for the Registry will be a \textbf{web browser}.
\gloss[nocite]{web-browser}

\subsubsection{Hardware interfaces}
N/A

\subsubsection{Communications interfaces}
N/A

\subsubsection{Memory}
N/A

\subsubsection{Operations}
The Registry will need weekly (manual) maintenance to ensure that information
added by data providers is correct. 

\subsubsection{Site adaptation requirements}
N/A

\subsection{Product functions}
The  Registry functions provide a way for users to add \acs{DODS}
datasets to a central registry and to access that central registry.

Functions for the Registry:
\begin{enumerate}
\item Post a dataset entry: Add a dataset reference to the registry. In
  addition, this function will provide an way to bind other information to
  the dataset reference. Anyone should be able to post an entry, but the
  entry should not actually be added to registry until some form of QC is
  performed by the Registry maintainer(s).

\item Remove a dataset entry: Remove a particular reference to a dataset. This
  operation should be restricted to the maintainer(s) of the registry.
\end{enumerate}

\subsection{User characteristics}
Users of the List will be all types of \acs{DODS} users including very
technical users, science users and possibly K--12 users. The List may not be
the most appropriate dataset location tool for the latter group and another
more inclusive tool might need to be developed for this group of users.

Users of the Registry will be \acs{DODS} server installers and generally they
will be technical or semi-technical users.

\subsection{Constraints}
This section lists any other items that may limit the developer's options.

\subsubsection{Regulatory policies}
N/A

\subsubsection{Hardware limitations}
The List and Registry systems must run on UNIX.

\subsubsection{Interfaces to other applications}
\begin{enumerate}
\item HTTP 1.1
\item \acs{CGI} 1.1
\item Apache {\tt httpd} 1.3.x
\item Netscape Navigator 3.x and higher.
\item MS Internet Explorer 5.x and higher.
\end{enumerate}

\subsubsection{Parallel operations}
N/A 

\subsubsection{Audit functions}
N/A 

\subsubsection{Control functions}
N/A 

\subsubsection{Higher-order language requirements}
N/A 

\subsubsection{Signal handshake protocols}
N/A 

\subsubsection{Reliability requirements}
New entries to the Registry should not be automatically included until they
have been validated.

\subsubsection{Criticality of the application}
N/A 

\subsubsection{Safety and security considerations}
\label{sec:security}
Any use of \ac{CGI} should be robust to attacks (i.e., hacking). They should
be resistant to shell-escape attacks.

Functions that are not accessible to the public should be password protected.

\subsection{Assumptions and dependencies}
The System designers can assume that the following are true:
\begin{itemize}
\item The software will run on one of the UNIX platforms we support and that
  they will be able to choose which platform if there are tradeoffs between
  them.
\item It will be OK to use \ac{CGI}---there will be no security limits placed
  on CGI software assuming it meets the basic security requirements (see
  section~\ref{sec:security}).
\item All users will have access to the Internet and a web browser.
\end{itemize}

\section{Specific Requirements}

\subsection{External interfaces}
\subsubsection{User interfaces}
\begin{enumerate}
\item Post an entry: Used to queue an entry for addition to the Registry.
  This interface must provide the user with a way to supply the following
  information:
  \begin{enumerate}
  \item URL: The reference to the dataset to be registered. If this dataset
    has a file/catalog server, this URL should reference that server. This
    is a required item.
  \item Title: A short---phrase or sentence, no more---description of the
    dataset. This is a required item.
  \item Update: A boolean flag; true indicates that this entry is intended
    to change an entry that was previously posted and included in the
    registry. Required.
  \item Science contact: Name and phone, email and/or mail contact
    information. To be used for questions about content. This is an
    optional item.
  \item Technical contact: Name and phone, email and/or mail contact
    information. To be used for questions about content. This is an
    optional item.
  \item Description: Text description of the dataset. Can be any text.
    Optional.
  \end{enumerate}
  When this information is requested of a user, the required items should be
  listed/shown first, then the optional items. Whether an item is required
  or optional should be explicitly stated for each item.

  When users prepare and entry for posting, they should be shown what will be
  posted and asked whether to continue or cancel.

  This interface will be usable by all users.

\item QC an entry: Look at an entry that has been posted to the Registry.
  If the entry is OK, add it to the registry. An entry is OK if:
  \begin{enumerate}
  \item The URL references a DODS dataset.
  \item The Title is spelled correctly.
  \item Update: if state of this item matches the contents of the registry
  \item All other fields: They are spelled correctly.
  \end{enumerate}
  If any of the fields are not OK, then use the technical and/or science
  contact information to talk to someone and get the information corrected. 

  This interface will have limited access; only the registry maintainer(s)
  should be able to access it.

\item Remove a dataset entry: This interface must provide a way for the
  Registry maintainer(s) to remove an entry. The entry to be removed will
  be specified by either its title or URL. The interface must show the
  selected entry and offer the option to remove or cancel. 

  This interface will be accessible to the registry maintainer(s) only.
\end{enumerate}

\subsubsection{Hardware interfaces}
N/A
\subsubsection{Software interfaces}
N/A
\subsubsection{Communications interfaces}
N/A

\subsection{Functions}
The User Interface requirements are sufficient to describe the Registry.


\appendix

\section{Acronyms and Abbreviations}
\begin{acronym}
\acro{SRS}{Software Requirements Specification}, See IEEE 830--1998
\acro{DODS}{Distributed Oceanographic Data System}, See the DODS home page:
{\tt http://unidata.ucar.edu/packages/dods/}
\acro{CGI}{Common Gateway Interface}, Version 1.1
\end{acronym}

\section{Change log}

\begin{verbatim}
$Log: registry.tex,v $
Revision 1.11  2000/05/12 19:30:55  jimg
*** empty log message ***

Revision 1.10  2000/02/15 04:03:24  jimg
Removed information about the List; see list.tex for that SRS. Folded in
Peter's first set of changes.

Revision 1.9  2000/01/21 21:16:27  jimg
Fixed up the List User Interface section.

Revision 1.8  2000/01/21 06:56:32  jimg
Added glossary and User interface requirements. Maybe they are all the
requirements these sub-systems need?

Revision 1.7  2000/01/21 00:51:24  jimg
Moved in place of old version

Revision 1.1  2000/01/18 01:39:19  jimg
Initial version
\end{verbatim}

\bibliography{dods}

\printgloss{dods-glossary}

\end{document}