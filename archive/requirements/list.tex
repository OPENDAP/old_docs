
% Another take on the registry SRS. This time follow IEEE 830-1998.
% 1/11/2000 jhrg
%
% $Id$

\documentclass{article}
%\usepackage{html}
\usepackage{psfig}
\usepackage{changebar}
\usepackage{acronym}
\usepackage{gloss}
%%
% These are html links which are used often enough in writing about DODS to
% merit an input file.
% jhrg. 4/17/94
%
% File rationalized and updated while writing the DODS User
% Guide. Also includes other useful abbreviations.
% tomfool 3/15/96
%
% $Id$
%
% HTML references to DODS documents
%
% For most of the DODS documentation there are three html references defined:
% 1) The upper case \newcommand produces the title of the paper, an
%    html link in the online documentation and a footnote in the paper
%    version.  
% 2) A capitalized version produces a capitalized description of the
%    paper and a link in the online version (but no footnote in the
%    paper version). 
% 3) A lower case version which produces a lower case description of
%    the paper and a html link, but no footnote.

% External references to documents:
%
% In order for external references to work (via latex2html) the labels used
% throughout the documents must be unique. The text labels are all prefixed
% with a document identifier (e.g., ddd) with a colon separater. The
% \externalrefs below (way below...) includes the perl files html.sty uses to
% make the cross refs.

\newcommand{\DODS}{\htmladdnormallink{Distributed Oceanographic Data System}
  {http://www.unidata.ucar.edu/packages/dods/}}

\newcommand{\Dods}{\htmladdnormallink{DODS}{http://www.unidata.ucar.edu/packages/dods/}}

\newcommand{\wrkshp}{\htmladdnormallink{Report on the First Workshop for the
    Distributed Oceanographic Data System, Proposed System Architectures}
  {http://www.unidata.ucar.edu/packages/dods/archive/reports/workshop1/section3.13.html}}

% not DODS URLs, but useful all the same...

\newcommand{\www}{\htmladdnormallink{World Wide Web} 
  {http://www.w3.org/hypertext/WWW/TheProject.html}}
\newcommand{\url}{\htmladdnormallink{Uniform Resource Locator}
  {http://www.w3.org/hypertext/WWW/Addressing/URL/url-spec.html}}

\newcommand{\uri}{\htmladdnormallinkfoot{Uniform Resource Identifiers}
  {http://www.w3.org/hypertext/WWW/Addressing/URL/uri-spec.html}}

\newcommand{\urn}{\htmladdnormallinkfoot{Uniform Resource Name}
  {http://www.acl.lanl.gov/URI/archive/uri-archive.messages/1143.html}}

\newcommand{\HTTPD}{\htmladdnormallinkfoot{HTTPD}
  {http://hoohoo.ncsa.uiuc.edu/docs/Overview.html}}

\newcommand{\HTTP}{\htmladdnormallinkfoot{HyperText Transfer Protocol}
  {http://www.w3.org/hypertext/WWW/Protocols/HTTP/HTTP2.html}}

%\newcommand{\HTML}{\htmladdnormallinkfoot{HyperText Markup Language}
%  {http://www.w3.org/hypertext/WWW/MarkUp/MarkUp.html}}

% CERN used to be `Conseil Europeen pour la Recherche Nucleaire'
\newcommand{\CERN}
  {\htmladdnormallink{European Laboratory for Particle Physics}
  {http://www.cern.ch/}}

\newcommand{\NCSA}
  {\htmladdnormallink{National Center for Supercomputing Applications}
  {http://www.ncsa.uiuc.edu/}}

\newcommand{\WWWC}{\htmladdnormallink{World Widw Web Consortium}
  {http://www.w3.org/}}

\newcommand{\CGI}{\htmladdnormallinkfoot{Common Gateway Interface}
  {http://hoohoo.ncsa.uiuc.edu/cgi/overview.html}}

\newcommand{\MIME}{\htmladdnormallink{Multipurpose Internet Mail Extensions}
  {http://www.cis.ohio-state.edu/htbin/rfc/rfc1590.html}}

\newcommand{\netcdf}{\htmladdnormallink{NetCDF}
  {http://www.unidata.ucar.edu/packages/netcdf/guide.txn_toc.html}}

\newcommand{\JGOFS}{\htmladdnormallink{Joint Geophysical Ocean Flux Study}
  {http://www1.whoi.edu/jgofs.html}}

\newcommand{\jgofs}{\htmladdnormallink{JGOFS}
  {http://www1.whoi.edu/jgofs.html}}

\newcommand{\hdf}{\htmladdnormallink{HDF}
  {http://www.ncsa.uiuc.edu/SDG/Software/HDF/HDFIntro.html}}

\newcommand{\Cpp}{{\rm {\small C}\raise.5ex\hbox{\footnotesize ++}}}

% Commands

\newcommand{\pslink}[1]{\small
\begin{quote}
  A \htmladdnormallink{postscript}{#1} version of this document is
  available.  You may also use \htmladdnormallink{anonymous
  ftp}{ftp://ftp.unidata.ucar.edu/pub/dods/ps-docs/} to access postscript files
  of all of the DODS documentation.
\end{quote}
\normalsize
}

\newcommand{\declaration}[1]{\small
{\tt {#1}}
\normalsize
}

% All the links from here down are for the white papers. I'm not sure that any
% of these are still valid given the reorganization of the web site. 5/12/98
% jhrg.

\newcommand{\DDA}{\htmladdnormallinkfoot{DODS---Data Delivery Architecture}
  {http://www.unidata.ucar.edu/packages/dods/archive/design/data-delivery-arch/data-delivery-arch.html}}
\newcommand{\Dda}{\htmladdnormallink{Data Delivery Architecture}
  {http://www.unidata.ucar.edu/packages/dods/archive/design/data-delivery-arch/data-delivery-arch.html}}
\newcommand{\dda}{\htmladdnormallink{data delivery architecture}
  {http://www.unidata.ucar.edu/packages/dods/archive/design/data-delivery-arch/data-delivery-arch.html}}

\newcommand{\DDD}{\htmladdnormallinkfoot{DODS---Data Delivery Design}
  {http://www.unidata.ucar.edu/packages/dods/archive/design/data-delivery-design/data-delivery-design.html}}
\newcommand{\Ddd}{\htmladdnormallink{Data Delivery Design}
  {http://www.unidata.ucar.edu/packages/dods/archive/design/data-delivery-design/data-delivery-design.html}}
\newcommand{\ddd}{\htmladdnormallink{data delivery design}
  {http://www.unidata.ucar.edu/packages/dods/archive/design/data-delivery-design/data-delivery-design.html}}

\newcommand{\URL}{\htmladdnormallinkfoot{DODS---Uniform Resource Locator}
  {http://www.unidata.ucar.edu/packages/dods/archive/design/urls/urls.html}}
\newcommand{\Url}{\htmladdnormallink{Uniform Resource Locator}
  {http://www.unidata.ucar.edu/packages/dods/archive/design/urls/urls.html}}
\newcommand{\dodsurl}{\htmladdnormallink{uniform resource locator}
  {http://www.unidata.ucar.edu/packages/dods/archive/design/urls/urls.html}}

\newcommand{\DAP}{\htmladdnormallinkfoot{DODS---Data Access Protocol}
  {http://www.unidata.ucar.edu/packages/dods/archive/design/api/api.html}}
\newcommand{\Dap}{\htmladdnormallink{Data Access Protocol}
  {http://www.unidata.ucar.edu/packages/dods/archive/design/api/api.html}}
\newcommand{\dap}{\htmladdnormallink{data access protocol}
  {http://www.unidata.ucar.edu/packages/dods/archive/design/api/api.html}}

\newcommand{\SOFT}
        {\htmladdnormallinkfoot{DODS---Software Development Environment}
  {http://www.unidata.ucar.edu/packages/dods/archive/managment/software/software.html}}
\newcommand{\Soft}{\htmladdnormallink{Software Development Environment}
  {http://www.unidata.ucar.edu/packages/dods/archive/managment/software/software.html}}
\newcommand{\soft}{\htmladdnormallinkfoot{software development environment}
  {http://www.unidata.ucar.edu/packages/dods/archive/managment/software/software.html}}

% I used to call the DAP the API, then for a few days I called it the
% DTP. These def's were used then. Rather than find everywhere they are used
% (not that hard, but...) I just hacked the macros. NB: the source file for
% the DAP document is still called `api.tex'

\newcommand{\API}{\htmladdnormallinkfoot{DODS---Data Access Protocol}
  {http://www.unidata.ucar.edu/packages/dods/archive/design/api/api.html}}
\newcommand{\api}{\htmladdnormallink{data access protocol}
  {http://www.unidata.ucar.edu/packages/dods/archive/design/api/api.html}}

\newcommand{\DTP}{\htmladdnormallinkfoot{DODS---Data Access Protocol}
  {http://www.unidata.ucar.edu/packages/dods/archive/design/api/api.html}}
\newcommand{\dtp}{\htmladdnormallink{data access protocol}
  {http://www.unidata.ucar.edu/packages/dods/archive/design/api/api.html}}

\newcommand{\TOOLKIT}{\htmladdnormallinkfoot{DODS---Client and Server Toolkit}
  {http://www.unidata.ucar.edu/packages/dods/archive/implementation/toolkits/toolkits.html}}
\newcommand{\Toolkit}{\htmladdnormallink{Client and Server Toolkit}
  {http://www.unidata.ucar.edu/packages/dods/archive/implementation/toolkits/toolkits.html}}
\newcommand{\toolkit}{\htmladdnormallink{client and server toolkit}
  {http://www.unidata.ucar.edu/packages/dods/archive/implementation/toolkits/toolkits.html}}

\newcommand{\TKR}{\htmladdnormallinkfoot{DODS---DAP Toolkit Reference}
  {http://www.unidata.ucar.edu/packages/dods/archive/implementation/toolkits/toolkits.html}}
\newcommand{\Tkr}{\htmladdnormallink{DAP Toolkit Reference}
  {http://www.unidata.ucar.edu/packages/dods/archive/implementation/toolkits/toolkits.html}}
\newcommand{\tkr}{\htmladdnormallink{DAP toolkit reference}
  {http://www.unidata.ucar.edu/packages/dods/archive/implementation/toolkits/toolkits.html}}

% I know that these are wrong. The papers have been moved to the gso web
% site, but I'm not sure exactly where. 5/12/98 jhrg

\newcommand{\DM}{\htmladdnormallinkfoot{DODS---Data Model}
  {http://lake.mit.edu/dods-dir/dodsdm2.html}}
\newcommand{\Dm}{\htmladdnormallink{Data Model}
  {http://lake.mit.edu/dods-dir/dodsdm2.html}}
\newcommand{\dm}{\htmladdnormallink{data model}
  {http://lake.mit.edu/dods-dir/dodsdm2.html}}

\newcommand{\DD}{\htmladdnormallinkfoot{DODS---Data Delivery}
  {http://lake.mit.edu/dods-dir/dods-dd.html}}

% external refs for DODS documents

\externallabels{http://www.unidata.ucar.edu/packages/dods/design/api}
  {/home/www/packages/dods/archive/design/api/labels.pl}
\externallabels{http://www.unidata.ucar.edu/packages/dods/design/data-delivery-arch}
  {/home/www/packages/dods/archive/design/data-delivery-arch/labels.pl}
\externallabels{http://www.unidata.ucar.edu/packages/dods/design/data-delivery-design}
  {/home/www/packages/dods/archive/design/data-delivery-design/labels.pl}
\externallabels{http://www.unidata.ucar.edu/packages/dods/implementation/toolkits}
  {/home/www/packages/dods/archive/implementation/toolkits/labels.pl}

%\renewcommand{\externalref}[2]{#2}

%%% Local Variables: 
%%% mode: latex
%%% TeX-master: t
%%% End: 

\psfigurepath{registry-figs}

% Note: to get the gloosary to work, run bibtex in the registry.gls.aux file,
% then latex registry, then bibtex *.gls, then latex. Also, make sure to set
% your BST and BIBINPUTS environment variables so that the BST and BIB files
% will be found.
\makegloss

% Change paragraph typesetting; eliminate indenting and add more space between
% paragraphs. 2/15/2000 jhrg
%\setlength{\parindent}{0em}     % Amount of indentation
%\addtolength{\parskip}{1ex}     % Vertical separation

\begin{document}

\title{DODS Dataset List SRS}
\author{James Gallagher\thanks{The University of Rhode Island,
    jgallagher@gso.uri.edu}}
\date{\today \\ $Revision$ }

\bibliographystyle{plain}

\maketitle
\tableofcontents

\section{Introduction}

This is the \ac{SRS} for the DODS List. The List provides a way for users to
find and access DODS datasets.

This document is the first \ac{SRS} written for DODS using the IEEE 830-1998
\ac{SRS} standard as a guide. Since the standard covers a wide range of
possible projects, some of the information in it is not appropriate for this
part of DODS. Where that is the case, or I think it is the case, I have
marked the section N/A.

\textbf{Bold face} type is used to indicate a word or phrase that may be
found in the glossary.

\emph{Emphasized} text contained in square brackets ([]) is used to indicate
an editorial comment about the information that should be provided in this
part of the \ac{SRS}.

% This contains the requirements for the DODS Dataset registry. This document
% and its companions---the DODS File server \ac{SRS} and the DODS Dataset List
% \ac{SRS}---describe pieces of a general purpose interface that should
% work for all DODS datasets. This interface should provide access to any DODS
% data, as either ASCII or binary, from computers running UNIX, Mac or MS
% operating systems. 

\subsection{Purpose}
This \ac{SRS} contains the requirements for the Dataset List. Its audience is
the entire \textbf{DODS core group}\gloss[nocite]{dods-core-group}.

\subsection{Scope}
The software to be produced is: the Dataset List.

The List will contain an entry for each \textbf{registered dataset}
\gloss[nocite]{reg-dataset}. The List will ultimately be built using 
information from the Registry~\cite{gallagher:registry-srs}. However, the
development of the List should be decoupled from the development of the
Registry so initial versions of the List may be built some other way (e.g.,
by modifying an existing list of datasets).

The List will not be responsible for collecting information about datasets
nor will it be responsible for performing quality control on the information
about datasets. Those are requirements of the Registry.

The goal of the List is to build on existing parts of DODS and to provide a
more comprehensive way to access data. Using only a \textbf{web browser}
\gloss[nocite]{web-browser} it should be possible to find servers for
specific datasets and access data from those servers. Specifically, a user
should be able to get to any data served by DODS in only a few mouse clicks
using nothing more than a web browser.

\subsection{Overview}

The remainder of this document is organized as follows:
\begin{enumerate}
\item Section~\ref{sec:overall} provides background for the specific
requirements and relates those requirements to the rest of DODS.
\item Section~\ref{sec:specific} lists the specific requirements for the List.
\item Following Section~\ref{sec:specific} are a list of acronyms and
  abbreviations, a change log, a glossary and references.
\end{enumerate}

Because the List is essentially a web document, there are no functional
requirements, only interface requirements. The functions behind the List is
contained in other parts of DODS (e.g., the \texttt{html} and \texttt{info}
services of the DODS servers).

\section{Overall Description}
\label{sec:overall}

\emph{[This section of the SRS should describe the general factors that
  affect the product and its requirements. This section should not state
  specific requirements. Instead, it provides a background for those
  requirements, which should be defiend in Section 3.]}

\subsection{Product perspective}

Part of the philosophy of \acs{DODS} is that \textbf{data providers}
\gloss[nocite]{data-prov} should have as much freedom when installing a
\acs{DODS} server as do people who install {\tt ftp} or {\tt http} servers.
There is no requirement that \textbf{\acs{DODS} servers}
\gloss[nocite]{dods-server} be registered or authorized in any way.  However,
this system-wide requirement does not mean that additional capabilities
cannot be made available by a dataset list subsystem, it only means that data
providers cannot be forced to include their datasets in such a system.

The List is designed to augment \acs{DODS}' current client base by building
on the existing Web capabilities of the servers. As of version 3.1,
\acs{DODS} servers can provide an HTML form\cite{gallagher:HTMLform} which
can be used to request either ASCII or binary data objects. However, there is
currently no organized way to find a \textbf{\acs{DODS} dataset}
\gloss[nocite]{dods-dataset}.  The List is intended to fill that void.

Because the List will work with web browsers, it should provide a simple way
for users to access the Web form interface of the \acs{DODS} servers. Doing
so will link dataset location and data access. That linkage is a fundamental
goal of the software described by this \ac{SRS}.

The List will ultimately be generated by software using information from the
Registry. This software will be able to assume that the Registry contains
only valid information. Thus, neither the List nor the software that
ultimately will generate the List needs to perform QC operations on that
inforamtion. Such operations are requriements of the Registry system.

Figure~\ref{fig:prod-perspective} shows how the List relates
to other parts of \acs{DODS}.

\begin{figure}
\begin{center}
\psfig{file=prod-perspective.ps,width=5in}
\end{center}
\caption{The relation of the List software to other relevant parts
  of DODS.}
\label{fig:prod-perspective}
\end{figure}

The following sections describe how the software operates within various
constraints.

\subsubsection{System interfaces}
The system interface is a web server.

\subsubsection{User interfaces}
The user interface for the List will be a web browser.

\subsubsection{Hardware interfaces}
N/A

\subsubsection{Communications interfaces}
N/A

\subsubsection{Memory}
N/A

\subsubsection{Operations}
The List will be checked for errors at least once every four days.

\subsubsection{Site adaptation requirements}
N/A

\subsection{Product functions}
Functions for the List:
\begin{enumerate}
\item Access the List.
\item Access information from a dataset (using the Info service).
\item Access the HTML form for a dataset (using the HTML service).
\end{enumerate}

\begin{quote}
Design Note:

The List should not actually access data from a dataset. When a user views
the List and chooses a dataset, it is nominally the HTML form service of that
dataset's server that provides access to the data (although the actual
implementation might be different). The List itself \emph{will probably}
directly access the HTML form and Info services.
\end{quote}

\subsection{User characteristics}
Users of the List will be all types of \acs{DODS} users including very
technical users, science users and possibly K--12 users. The List may not be
the most appropriate dataset location tool for the latter group and another
more inclusive tool might need to be developed for this group of users.

\subsection{Constraints}
\emph{[This section lists any other items that may limit the developer's
  options.]}

\subsubsection{Regulatory policies}
N/A

\subsubsection{Hardware limitations}
The List system must run on UNIX.

\subsubsection{Interfaces to other applications}
\begin{enumerate}
\item DODS servers, version 3.1 and greater.
\item Standalone HTML form genreation web services.\footnote{These will be
    needed until the transition to the 3.1$+$ servers is complete.}
\item HTTP 1.1
\item \acs{CGI} 1.1
\item Apache {\tt httpd} 1.3.x
\item Netscape Navigator 3.x and higher.
\item MS Internet Explorer 5.x and higher.
\end{enumerate}

\subsubsection{Parallel operations}
N/A 

\subsubsection{Audit functions}
N/A 

\subsubsection{Control functions}
N/A 

\subsubsection{Higher-order language requirements}
N/A 

\subsubsection{Signal handshake protocols}
N/A 

\subsubsection{Reliability requirements}
The List must not contain invalid entries for more than four days.

\subsubsection{Criticality of the application}
N/A 

\subsubsection{Safety and security considerations}
N/A

\subsection{Assumptions and dependencies}
\label{sec:assumptions}
The System designers can assume that the following are true:
\begin{enumerate}
\item The software will run on one of the UNIX platforms we support and that
  they will be able to choose which platform if there are tradeoffs between
  them.
\item All users will have access to the Internet and a web browser.
\item All QC of information about datasets will be performed elsewhere.
\item A separate system will build the List.
\end{enumerate}

\section{Specific Requirements}
\label{sec:specific}
\emph{[This section of the \ac{SRS} all of the software requirements to a
  level of detail sufficient to enable designers to design a system to
  satisfy those requirements, and testers to test that the systems satisfies
  those requirements.]}

\subsection{External interfaces}
\subsubsection{User interfaces}

This interface must provide, in a web browser, a page of HTML which lists
every DODS dataset. Once the Registry has been built, \emph{every DODS
  dataset} will mean, every dataset in the Registry. Until the Registry has
been built it will mean every dataset accessible with the Matlab
GUI~\cite{sgouros:MLGUI} or otherwise known to the group to be
\textbf{publicly accessible} \gloss[nocite]{publicly-accessible}.

Each dataset in the HTML page will be represented by a link to the dataset's
HTML form interface.\footnote{For a single-file dataset, the HTML form
  generated by the 3.1 servers will be shown. For a multi-file dataset, the
  HTML form will need to be fixed up.} Specifically, the information to be
provided for each dataset is:
  \begin{description}
  \item [Title] The title of this dataset. A sentence or phrase. This will
    start the entry for the dataset and will appear on a new line on the HTML
    page. The text will reference\footnote{A \textbf{reference} in this
      context is a hyperlink in HTML.} the HTML
    form~\cite{gallagher:HTMLform} interface of this dataset.

    \begin{quote}
      Note: 

      This feature of a DODS server may be accessed using the \texttt{.html}
      URL suffix. However, this feature of the DODS servers has just been
      introduced and, as of 2/15/2000, has only been released as beta
      software. Furthermore the HTML form interface may not work well for
      multi-file dataset. These problems will have to be addressed by the
      design of the HTML form service of the DODS servers
      (see~\cite{gallagher:HTMLform}). 
    \end{quote}

  \item [More Information... (button)] This HTML button displays the text
    returned by the datasets \emph{info}\footnote{This feature of a DODS
      server is accessed using the \texttt{.info} URL suffix.} service. This
    information will be displayed in a separate window (but not a separate
    browser window\ldots e.g., it could be displayed using a JavaScript
    window). 
  \end{description}
%   \item [Science contact] Email for questions/feedback about the content of
%     the dataset. 

%     Optional.

%   \item [Technical contact] Email for questions/feedback about the DODS server
%     or other computer issues. 

%     Optional.

%   \item [Description] A paragraph describing the contents of the dataset.

%     Optional.

%   \item [Date] The date when this dataset was added to the List or its entry
%     in the List was last changed (not when the dataset itself was added or
%     last changed).

%     Optional. 

  The Title will be used to link to the web-based user interfaces for the
  dataset. The URL may be used by \textbf{web crawlers}
  \gloss[nocite]{web-crawler}.

  \begin{quote}
    I'm pushing the Science contact, Technical contact, Description and Date
    information into the ancillary DAS and/or the suplementatl HTML that
    users can bind with datasets using the info service. This means that the
    FGCD-CS\ldots [I forget the full abbreviation] information used/displayed
    by the List will now be in the DAS along with whatever other stuff we put
    there. Also, it means that the List is now decoupled from the metadata
    part of the project. If we decide there needs to be more (or less)
    metadata the requirements for the List do not change (and presumably
    neither does the design).

    On the flip side, this means that the List will not have as uniform an
    information content as if \emph{it} contained the contact, etc.,
    information. 
  \end{quote}

\subsubsection{Hardware interfaces}
N/A
\subsubsection{Software interfaces}
N/A
\subsubsection{Communications interfaces}
N/A

\subsection{Functions}
The User Interface requirements are sufficient to describe the List.

\appendix

\section{Acronyms and Abbreviations}
\begin{acronym}
\acro{SRS}{Software Requirements Specification}, See IEEE 830--1998
\acro{DODS}{Distributed Oceanographic Data System}, See the DODS home page:
{\tt http://unidata.ucar.edu/packages/dods/}
\acro{CGI}{Common Gateway Interface}, Version 1.1
\end{acronym}

\section{Change log}

\begin{verbatim}
$Log: list.tex,v $
Revision 1.4  2000/07/20 21:35:58  jimg
Minor changes

Revision 1.3  2000/02/16 00:36:48  jimg
More of Peter's changes, plus the split is now complete.

Revision 1.2  2000/02/15 06:00:41  jimg
First crack at Peter's changes.

Revision 1.1  2000/02/15 00:13:33  jimg
Created List-only SRS from the combined List and Registry SRS.

Revision 1.9  2000/01/21 21:16:27  jimg
Fixed up the List User Interface section.

Revision 1.8  2000/01/21 06:56:32  jimg
Added glossary and User interface requirements. Maybe they are all the
requirements these sub-systems need?

Revision 1.7  2000/01/21 00:51:24  jimg
Moved in place of old version

Revision 1.1  2000/01/18 01:39:19  jimg
Initial version
\end{verbatim}

\printgloss{dods-glossary}

\raggedright

\bibliography{dods}

\end{document}