
% A template for an IEEE 830-1998 SRS.
% 1/11/2000 jhrg
%
% $Id$

\documentclass{article}

\usepackage{changebar}
\usepackage{acronym}
\usepackage{gloss}
\usepackage{xspace}
\usepackage{epsfig}

% Use these on place of epsfig for regular PS figures.
% \usepackage{psfig}
% \psfigurepath{my-figs}

\newcommand{\na}{{\sc N/A}\xspace}
\newcommand{\Cpp}{\rm {\small C}\raise.5ex\hbox{\footnotesize ++}\xspace}
\newcommand{\dap}{\rm {\small DAP}\raise.5ex\hbox{\footnotesize ++}\xspace}
\newcommand{\maewesturl}{http://maewest.gso.uri.edu/\-cgi-bin/\-nph-dsp/\-htn\_sst\_decloud/\-1992/\-i92098065016.htn\_d.Z\xspace}

% Note: to get the gloosary to work, run bibtex in the *.gls.aux file,
% then latex registry, then bibtex *.gls, then latex. Also, make sure to set
% your BST and BIBINPUTS environment variables so that the BST and BIB files
% will be found.
\makegloss

% Change paragraph typesetting; eliminate indenting and add more space between
% paragraphs. 2/15/2000 jhrg
\setlength{\parindent}{0em}     % Amount of indentation
\addtolength{\parskip}{1ex}     % Vertical separation

\begin{document}

\title{SRS Template}
\author{Who\thanks{Where, email@cpu.org}}
\date{\today \\ $Revision$ }

\bibliographystyle{plain}

\maketitle
\tableofcontents

%%%%%%%%%%%%%%%%%%%%%%%%%%%%%%%  Introdunction  %%%%%%%%%%%%%%%%%%%%%%%%%%%%

\section{Introduction}
\emph{[What is this document about, in very general terms.]}

This document conforms to the IEEE~830-1998 \ac{SRS} recommended practice.
Since the recommended practice covers a wide range of possible projects, some
of the information in it is not appropriate for this part of \acs{DODS}.
Where that is the case, or I think it is the case, I have marked the section
N/A.

\textbf{Bold face} type is used to indicate a word or phrase that may be
found in the glossary.

\emph{Emphasized} text contained in square brackets ([]) is used to indicate
an editorial comment about the information that should be provided in a part
of the \ac{SRS}.

\subsection{Purpose}
\emph{[Who is this document for?]}

\subsection{Scope}
\emph{[What part of the project does this SRS cover.]}

\subsection{Overview}

The remainder of this document is organized as follows:
\begin{enumerate}
\item Section~\ref{sec:overall} provides background for the specific
requirements and relates those requirements to the rest of DODS.
\item Section~\ref{sec:specific} lists the specific requirements for the
  Cache.
\item Following Section~\ref{sec:specific} are a list of acronyms and
  abbreviations, a change log, a glossary and references.
\end{enumerate}

%%%%%%%%%%%%%%%%%%%%%%%%%  Overall Description  %%%%%%%%%%%%%%%%%%%%%%%%%%%%%

\section{Overall Description}
\label{sec:overall}

\emph{[This section of the SRS should describe the general factors that
  affect the product and its requirements. This section should not state
  specific requirements. Instead, it provides a background for those
  requirements, which should be defined in Section 3.]}

\subsection{Product perspective}

\subsubsection{System interfaces}
\subsubsection{User interfaces}
\subsubsection{Hardware interfaces}
\subsubsection{Software interfaces}
\subsubsection{Communications interfaces}
\subsubsection{Memory}
\subsubsection{Operations}
\subsubsection{Site adaptation requirements}

\subsection{Product functions}

\subsection{User characteristics}

\subsection{Constraints}
\emph{[This section lists any other items that may limit the developer's
  options.]}

\subsubsection{Regulatory policies}
\subsubsection{Hardware limitations}
\subsubsection{Interfaces to other applications}
\subsubsection{Parallel operations}
\subsubsection{Audit functions}
\subsubsection{Control functions}
\subsubsection{Higher-order language requirements}
\subsubsection{Signal handshake protocols}
\subsubsection{Reliability requirements}
\subsubsection{Criticality of the application}
\subsubsection{Safety and security considerations}

\subsection{Assumptions and dependencies}

%%%%%%%%%%%%%%%%%%%%%%%%%%%  Specific Requirements %%%%%%%%%%%%%%%%%%%%%%%%%

\section{Specific Requirements}
\label{sec:specific}
\emph{[This section of the \ac{SRS} lists all of the software requirements to
  a level of detail sufficient to enable designers to design a system to
  satisfy those requirements, and testers to test that the systems satisfies
  those requirements. Each separate requirment should be uniquely numbered.]}

\subsection{External interfaces}

\subsubsection{User interfaces}
\subsubsection{Hardware interfaces}
\subsubsection{Software interfaces}
\subsubsection{Communications interfaces}

\subsection{System features}
\subsection{Performance requirements}
\subsection{Design constraints}
\subsection{Software system attributes}
\subsection{Other requirements}

\appendix

\section{Acronyms and Abbreviations}
\begin{acronym}
%
% Make one entry per line, even if they are long lines that wrap and look
% ugly. This makes it simple to sort the list using emacs' sort-lines
% command. 3/27/2000 jhrg
%
% $Id$
\acro{AS}{Aggregation Server}
\acro{BNF} {Backus-Naur Form}
\acro{CE}{Constraint Expression}
\acro{CGI}{Common Gateway Interface}
\acro{COARDS}{Cooperative Ocean/Atmosphere Research Data Service}
\acro{CSV}{Comma Separated Values}
\acro{DAP}{Data Access Protocol}
\acro{DAS}{Dataset Attribute Structure}
\acro{DDS}{Dataset Descriptor Structure}
\acro{DODS}{Distributed Oceanographic Data System}, See the DODS home page: \texttt{http://\-unidata.ucar.edu\-/packages\-/dods/}
\acro{DataDDS}{Data Dataset Descriptor Structure}
\acro{FGDC}{Federal Geographic Data Community}
\acro{HTML}{Hypertext Markup Language}
\acro{HTTP}{HyperText Transfer Protocol}
\acro{MIME}{Multimedia Internet Mail Extensions}
\acro{SRS}{Software Requirements Specification}, See IEEE 830--1998
\acro{URI}{Uniform Resource Identifiers}
\acro{URL}{Uniform Resource Locator}
\acro{W3C}{The World Wide Web Consortium}, See http://www.w3c.org/
\acro{WWW}{The World Wide Web}
\acro{XDR}{External Data Representation}
\acro{XML}{Extensible Markup Language}
%%% Local Variables: 
%%% mode: latex
%%% TeX-master: t
%%% End: 

\end{acronym}

\section{Change log}

\begin{verbatim}
$Log: SRS-template.tex,v $
Revision 1.4  2000/07/20 21:35:58  jimg
Minor changes

Revision 1.3  2000/07/17 05:46:19  jimg
Added more boiler plate based on the caching SRS.

\end{verbatim}

\printgloss{dods-glossary}

\raggedright

\bibliography{dods}

\end{document}
