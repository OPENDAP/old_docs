\renewcommand{\chaptertitle}{Southeast Regional Workshop}
\chapter{\texorhtml{}{Appendix C }\chaptertitle}

%% $Id$

\begin{center}
November 17th - 19th, 2000\\
NOAA - Coastal Services Center\\
Charleston, SC\\
\end{center}

\section{Introduction and Workshop Objectives}

Anne Ball, SE regional DODS coordinator, convened the workshop at 9:00
am.  The workshop facilitator was Steve Bliven.  The objectives of the
workshop were stated as follows:

\begin{itemize}
  \item To familiarize workshop attendees with the DODS server and client.
  \item Determine whether DODS meets the needs of organizations in the southeast.
  \item Develop a list of issues/suggestions to ensure the success of DODS.
  \item Determine how to best implement DODS in the southeast.
\end{itemize}

All the workshop participants introduced themselves and provided
information on what organizations they represented.  See the list
later in this appendix for a listing of workshop attendees.

\section{DODS}

\subsubsection{Background}

Peter Cornillon of the University of Rhode Island (URI) provided a
brief history of DODS.  The conception of DODS was started in the
early `90s and is based on two different client-server systems; one
that looked at satellite data and another that looked at in-situ data.
A meeting in 1992 discussed the development of building an
``umbrella'' client-server system that incorporates both of the above
client-server systems into one.  The first workshop had forty
attendees and was held at URI to discuss a distributed system.
Although it became apparent at this workshop that the system being
designed was based on sufficiently general concepts that it need not
be constrained to oceanography, it was also clear that without a
well-defined focus, progress could well be hampered.  As a result, the
system was called the Distributed Ocean Data System, and was to be
UNIX based.  The effort was initially funded by NSA and NOAA.  More
recently, the National Ocean Partnership Program (NOPP) awarded the
DODS effort a grant to develop a Virtual Ocean Data Hub system.

The first year of the proposed NOPP effort was to obtain input from
the community.  Five regions were formed: Northeast, Southeast, Gulf
Coast, West Coast, and Great Lakes (which was dropped and replaced by
one focusing on GIS) to hold workshops.  Each was led by a regional
coordinator.  The regional coordinators have control over their
workshop, and they also act as a focal point for the expenditures in
the out years.  Each of the regional coordinators have an interest in
oceanography but are from differing backgrounds so as to bring as much
diverse input as possible to the overall effort.  Based on the five
reports generated by each of the regional workshops, a final report
will be generated which will be used at a national workshop.  After
the national workshop is held an overall report discussing
recommendations for the DODS project will be developed.  A technical
meeting will then be held to discuss the technical findings and
recommendations of the national report.

\subsubsection{Overview and Demonstration of  DODS}

Before going into a demonstration of DODS, a document describing DODS
was passed out to each of the participants.  Listed
below are some the highlights of the DODS system as it was presented
to the workshop participants.

Peter emphasized that DODS is an architectural framework to move data
across the network.

The underlying philosophy of DODS  is twofold:
\begin{itemize}
  \item Anyone willing to share their data should be able to do so via DODS.
  \item The user should be able to use the application package with
which she or he is the most familiar to examine or analyze the data of
interest.
\end{itemize}

Peter explained the five steps involved to analyze data:
\begin{enumerate}
\item LOCATE
\item SUBSET
\item ACQUIRE
\item INGEST
\item ANALYZE
\end{enumerate}

DODS addresses the middle three steps, i.e., SUBSETTING, ACQUIRING,
and INGESTING.  DODS doesn't focus on LOCATION or ANALYSIS.  What DODS
is \emph{not} and the underlying philosophy of DODS were covered.

Peter went on to explain that DODS is built from the \emph{bottom up},
with high functionality at the data acquisition level and that
functionality decreasing as you move up to the inventory level and the
directory level.

Syntactic and semantic metadata were described and how they relate to
the data, inventory, and directory levels.  Semantic metadata can be
sub-setted into \emph{use} and \emph{search} categories.  \emph{Use}
metadata can be further subdivided into \emph{translational} use
metadata and \emph{descriptive} use metadata.  DODS provides a rigid
structure for syntactic metadata.  Although not mandatory, DODS
recommends including at a minimum translational use metadata.

Four levels of interoperability at the data level (for data accessible
over the network) were defined:

\begin{description}
\item[Level 0] - no syntactic or semantic metadata - FTP.
\item[Level 1] - rigid syntactic, no semantic metadata - DODS
\item[Level 2] - rigid syntactic, human readable semantic use
  metadata - A subset of DODS data sets.
\item[Level 3] - rigid syntactic, consistent semantic use metadata;
  i.e., machine-readable - A subset of the DODS Level 2 data sets.
\end{description}

Before going into the first demonstration on the use of DODS, three
DODS data objects were defined:

\begin{enumerate}
\item The data descriptor structure (DDS), this is the syntactic
  metadata for a data set.
\item The data attribute structure (DAS), this is the semantic
  metadata for a data set.
\item Data - the actual data.
\end{enumerate}

Additionally, DODS servers support several other services, including:
\begin{itemize}
\item .ASCII - an ASCII representation of the data.
\item .info - a more readable version of the DDS and DAS combined.
\item .html form - a web based form that will help to build a DODS URL.
\end{itemize}

A demonstration was given on how to access a WOCE/TOPEX data set,
which was copied from CD-ROM to a server machine at URI.  In order to
access data, Peter requested data from a DODS server via URLs.  The
URL was then passed to a DODS client, via http.  At this point the
DODS client enters the data into an application from the DODS server
via httpd.  Any data sub-setting is done at the server.  It was
emphasized that \emph{writing DODS URLS can be difficult!}

Three classes of interfaces were described based on the difficulty
involved in building a URL.  These are in order of increasing
ease-of-use:

\begin{description}
\item[Command Line] No help is given creating a URL.
\item[General Purpose URL Builder] A menu-based interface based on
  data dimensions and names.
\item[Graphical User Interface] A URL builder based on geophysical
  parameter input.
\end{description}

Another demonstration was given showing the functionality of the DODS
Dataset Access Form to access the TOPEX dataset.  More buttons will be
added to this form in the future.

The DODS data model was explained along with the data model
constraints.  The following operations are permitted when requesting
data:

\begin{itemize}
\item Projection.
\item Relational operations on list and sequence elements.
\item Rectangular and decimated subsets of arrays and grids.
\end{itemize}

\subsubsection{Issues raised}

After the first live demo on the usage of DODS, the floor was opened
up for discussion. Some of the questions and comments were as follows:

\begin{enumerate}
\item Regarding a user who has data and wants to make them available via
  DODS (how is this done)?  

\emph{Answer}:
  \begin{itemize}
  \item Determine which server is most appropriate for your data.
  \item Go to the DODS home page and download the binary version of
    the appropriate server. (Better documentation is being written for
    these steps.)
  \item You will need to install the server on your own computer, and
    tell the server where the data are located.
\end{itemize}

Peter stated that it is up to each region to determine how they would
like to handle server installation.  Regional support could be used to
support server installation.  In the proposal, Peter thought that each
region would either:

\begin{itemize}
  \item Hire an individual for a given region to help install servers.
  \item OR let sub-contracts to people to install their own servers.
\end{itemize}

He stated that it is really up to the regions on how they want to do
it and that the region perhaps should identify some high priority
sites for server installation.

\item Regarding DODS support for server installation; what happens if
  someone runs into a problem installing a server?  

\emph{Answer}:
  \begin{itemize}
  \item Requests for support should be addressed to the DODS support
    site.  This will allow the DODS project to track problems.
  \end{itemize}

\item If you have other systems in place and want to install a DODS
  server, will this create problems related to the other servers?
  
\emph{Answer}:
  \begin{itemize}
  \item Putting a DODS server on your system does not impact existing servers.
  \end{itemize}

\item Can curvilinear data be handled?  

\emph{Answer}:
\begin{itemize}
\item Yes.  To the best of our knowledge, DODS has been able to move
  every data type that has been encountered.  However, this does not
  mean that appropriate semantic metadata exist to effectively use
  that data.
\end{itemize}


\item Can DODS be extended? 

\emph{Answer}:
  \begin{itemize}
  \item The DODS project will consider extending the DODS data model.
  \end{itemize}

\item The efficiency of the data model is a concern for complex data
sets.  It was thought that if the DODS data model was tweaked a little
bit, these problems might be resolved.  This is a translational use
metadata issue.

\item  A question was raised as to how DODS deals with data compression.

\emph{Answer}:
\begin{itemize}
  \item DODS uses standard compression techniques.
\end{itemize}

\item Why is DODS not a data locator?


\emph{Answer}:
\begin{itemize}
  \item The core of the DODS project is how you move data around.
  \item They are working in the direction of the data locator issue.
  \item Have a contract to develop a web-crawler based on the DODS Dir function.
\end{itemize}

\item Peter sees three issues facing the group (to this point in the workshop):
\begin{itemize}
  \item Moving the data around in a system-efficient way?
  \item Adding additional ``bells and whistles'' to DODS.
  \item What data do you want served?
\end{itemize}

\item Comments regarding rigidity of metadata standards.
\begin{itemize}
  \item Imposing standards as a group.
  \item Nothing precluding users from making own decisions on metadata.
\end{itemize}
\end{enumerate}

Peter gave two more demonstrations:

\begin{enumerate}
\item  Live Access Server to PMEL Data
  Steps involved:
  \begin{itemize}
  \item LAS generates HTML on a form that the user fills out to select
    data.
  \item The user selection is returned to the LAS.
  \item LAS parses the returned request.
  \item Pass request on to Ferret (a NetCDF program).
  \item Ferret then requests a subset from the server site and moved
    over the network.
  \item Data is then returned to browser.
  \item If a server is down, some data won't show up.
  \item Select COADS data (sea surface temperature).
  \item Live Access Server returns a .gif image.
\end{itemize}

(Peter noted that there is metadata available at the LAS site.  This
allows LAS users to show location, time, and so on of data in the date
selection window.)

\item Demo using Matlab.
\begin{itemize}
  \item Peter showed various functionalities using DODS and the Live Access Server.
\end{itemize}
\end{enumerate}

\subsection{Discussion on server related issues}

\subsubsection{Metadata}

Issues concerning metadata were numerous and varied.  Peter re-stated
that only syntactic metadata is \emph{required} to run DODS and that
the server will provide that metadata.  The user is \emph{not} required to
provide semantic metadata.  He indicated that it's OK if metadata is
bound to the data being served.  A few additional points were stated
regarding DODS and metadata:

\begin{itemize}
\item One does not have to reformat or restructure semantic metadata
  bound to a file.
\item Semantic metadata that are not bound to the file may be included
  in the DAS.
\item DODS will take advantage of what it can in regard to metadata
  but can't guarantee that it will take advantage of the structure.
\item Some sites have extensive metadata on the web.
\end{itemize}

\subsubsection{Other metadata issues/comments/suggestions}

For a file to be useful, the translational use metadata encoded in the
DAS should, at a minimum, provide for the translation of names to a
common form, and specify units, and missing value flags.

\begin{enumerate}
\item The SE regional workgroup recommends that at minimum,
  translational use metadata should be available via the DAS.
\item The DODS core should have the ability to associate FGDC
  compliant metadata that may be generated and stored by a third party
  site, and associate that with a DODS dataset. (Third party metadata
  cannot be handled yet and will require a modification to the DODS
  core.)
\item Several participants expressed an interest in quality flags and
  felt that these should be available in the DAS.
\item Peter stated that he would try to work out issues with FGDC
  metadata.  He said that he would bring up the issue of third party
  metadata generators even though this could potentially discourage
  people from serving their data.  The DODS team will also consider
  handling other metadata that are contained in web pages, either by
  incorporating the web page into the DAS or including a URL pointer
  to the web page in the DAS.
\item DODS will not force people to provide metadata.  The SE
  workgroup can require metadata of its participants, but cannot
  prohibit people from serving their data.
\item Work closely with data providers to come up with more useful
  metadata.
\item Data providers should provide a URL link to their metadata.
\end{enumerate}

\subsubsection{Liability}

The issue of having a disclaimer with served data was discussed.  In
many situations in the southeast, disclaimers are necessary/required
to satisfy the South Carolina General Council.  One opinion was that
disclaimers should be located on the client end.  Also, the southeast
region would/should not be allowed to put up a DODS server for data
that has disclaimers unless they (the region) could be insured that
they see the disclaimer first.

\subsubsection{Data Formats and Storage}

Data formats in the southeast region include ASCII files, ACCESS,
EXCEL, Oracle, SEQUEL, INFORMIX, SYBASE, and others.  Peter suggested
that the region identify how many users in the region use which
packages and from that list, look for solutions to accommodate those
users.  He pointed out that DODS needs to make it trivial to install
servers for simple datasets.

\subsubsection{Suggestions/comments concerning servers}

\begin{enumerate}
\item The southeast region should come up with a matrix listing
  available DODS servers that are out there and the various data types
  that need to be served.
\item In what format are the different data types stored in (INFORMIX,
  Oracle, SYBASE, or just ASCII files)?
\item Some level of data management on the server side will be needed.
\item Money will have to be spent on managing data.
\item DODS servers cannot handle Microsoft EXCEL and ACCESS files yet.
  Peter will investigate this.
\item Some workshop participants indicated that they would require
  tech support to install servers, others said that they could handle
  it themselves.  Still others indicated that they need a certain
  level of commitment from principal investigators before server
  installation.
\end{enumerate}

\subsubsection{Discussion on client related issues}

Based on the demonstration of the DODS system, workshop participants
indicated that they would like to see a DODS web browser client.
Additionally, it would be desirable if the browser could be ``beefed
up'' to make it more of a point-and-click interface.

A consensus of the group with regard to application software used
found that participants used SAS while others used EXCEL, ACCESS, and
ArcINFO products.  Peter indicated that the DODS project has
subcontracted to NGDC and ESRI to build a DODS GIS client as well as
to provide access to GIS data in other DODS clients.  A timeline of
eighteen months was given to complete the task.

Two questions were posed related to the client:
\begin{enumerate}
\item Can metadata stored in the Matlab GUI be updated?  

\emph{Answer}: Yes,
  this may be done over the web.
\item On what platforms are clients available?  

\emph{Answer}: Command line
  clients are available on all platforms that support the DODS core.
\end{enumerate}

\section{NCDDC Web Site}

At this point, John Ellis provided a demonstration of the National
Coastal Data Development Center (NCDDC) web site being built at
Stennis Space Center.  In brief, the NCDDC system is:
\begin{itemize}
  \item A virtual data hub, which supports data discovery, based on FGDC
metadata records.
  \item A data delivery process, which directs a user request to the
appropriate server capability.
  \item At the moment, DODS is a server option from the pull-down menu
giving a URL listing of available DODS servers.
\end{itemize}

NCDDC currently has two web sites, one operational and one developmental.

\section{Regional Data Sets}

The workshop attendees were asked the following questions regarding data sets that they held in 
their respective agencies:

\begin{enumerate}
\item How many data sets do they have?
\item What are the spatial/temporal variables of each data set?
\item Volume in bytes?
\item Format of data?
\item Machine/operating system?
\item Metadata availability?
\item Percent archived at a national archive?
\end{enumerate}

\subsection{Chris Friel - Florida Marine Research Institute}
 
Number of data sets:
\begin{itemize}
  \item 175 GIS (a mix of raster and vector).
  \item 75 flat/relational.
\end{itemize}

Spatial/temporal variables of data set:
\begin{itemize}
  \item Mainly Florida (1970-2000).
  \item Biological, some base map data.
  \item Biophysical monitoring.
\end{itemize}

Volume in bytes:
\begin{itemize}
  \item Several hundred gigabytes.
\end{itemize}

Format of data:
\begin{itemize}
  \item Arc/INFO, ArcView (shapefiles), SAS, Access, Excel.
\end{itemize}

Machine/operating system:
\begin{itemize}
  \item NT or UNIX SGI machines.
\end{itemize}

Metadata availability:
\begin{itemize}
  \item 40\% have FGDC metadata.
\end{itemize}

Percent archived at a national archive:
\begin{itemize}
  \item A small percentage in archive.
\end{itemize}


\subsection{Jim Nelson - UNC Marine Sciences}

Number of data sets:
\begin{itemize}
  \item Individual PI or project holdings.
  \item Coastal Observing Survey data (8 platforms, 2 operational now, 4 next year).
\end{itemize}

Spatial/temporal variables of data set:
\begin{itemize}
  \item GA shelf/S. Atlantic Bight.
  \item Hydrological, ADP currents, biological, chemical.
\end{itemize}

Volume in bytes:
\begin{itemize}
  \item     4 gigabytes/year now.
\end{itemize}

Format of data:
\begin{itemize}
  \item ASCII processed data in Matlab.
\end{itemize}

Machine/operating system:
\begin{itemize}
  \item SUN or Linux.
  \item UNIX, Windows, MAC.
\end{itemize}

Metadata availability:
\begin{itemize}
  \item Varies currently, needs work internally.
\end{itemize}

Percent archived at a national archive:
\begin{itemize}
  \item Small amount some provided to NWS.
\end{itemize}

\subsection{Reyna Sabina - NOAA/Atlantic Oceanographic Marine Laboratory (AOML)}

Number of data sets:
\begin{itemize}
  \item One.
\end{itemize}

Spatial/temporal variables of data set:
\begin{itemize}
  \item XBT in Atlantic.
  \item Drifter buoys (global).
  \item Hydrographic.
  \item Transportation.
  \item Hurricane tracking.
\end{itemize}

Volume in bytes:
\begin{itemize}
  \item Approximately 10 gigabytes.
\end{itemize}

Format of data:
\begin{itemize}
  \item Flat files (from PI's), DBMS.
\end{itemize}

Machine/operating system:
\begin{itemize}
  \item SUN, UNIX
  \item DBMS
\end{itemize}

Metadata availability:
\begin{itemize}
  \item Some.
\end{itemize}

Percent archived at a national archive:
\begin{itemize}
  \item     Most.
\end{itemize}

\subsection{Tim Snoots - SC Department of Natural Resources}

Number of data sets:
\begin{itemize}
  \item PI or long-term monitoring.
  \item 10 data sets.
\end{itemize}

Spatial/temporal variables of data set:
\begin{itemize}
  \item SC estuarine, some extends to shelf.
  \item Hydrographic, water quality, water chemistry, biological, sediment data.
\end{itemize}

Volume in bytes:
\begin{itemize}
  \item Less than 10 gigabytes.
\end{itemize}

Format of data:
\begin{itemize}
  \item Microsoft Access, ASCII, Excel, SAS.
\end{itemize}

Machine/operating system:
\begin{itemize}
  \item Servers, Windows NT.
  \item Clients, Windows 95, 98, NT, Macintosh.
\end{itemize}

Metadata availability:
\begin{itemize}
  \item Less than 10\%, in ASCII format.
\end{itemize}

Percent archived at a national archive:
\begin{itemize}
  \item Not clear on how much.
\end{itemize}

\subsection{John Ellis - NOAA/National Coastal Data Development Center (NCDDC)}

Number of data sets:
\begin{itemize}
  \item No holdings.
\end{itemize}

Spatial/temporal variables of data set:
\begin{itemize}
  \item Great Lakes, coast of ME, Gulf of Mexico, Guam, Puerto Rico, FL.
  \item Meteorological/oceanographic, bathymetry, water structure,
currents, water level, tides, surface winds, waves.
\end{itemize}

Volume in bytes:
\begin{itemize}
  \item Unknown.
\end{itemize}

Format of data:
\begin{itemize}
  \item Varies.
\end{itemize}

Machine/operating system:
\begin{itemize}
  \item Varies.
\end{itemize}

Metadata availability:
\begin{itemize}
  \item 180 metadata records/sets FGDC compliant.
\end{itemize}

Percent archived at a national archive:
\begin{itemize}
  \item Unknown.
\end{itemize}


\subsection{Jim Frysinger - College of Charleston}

Number of data sets:
\begin{itemize}
  \item 10 raw, QC 1.
\end{itemize}

Spatial/temporal variables of data set:
\begin{itemize}
  \item Sea surface.
  \item 6 months of sampling data (meteorological data).
\end{itemize}

Volume in bytes:
\begin{itemize}
  \item 3 megabytes.
\end{itemize}

Format of data:
\begin{itemize}
  \item ASCII, Excel.
\end{itemize}

Machine/operating system:
\begin{itemize}
  \item UNIX, Linux.
\end{itemize}

Metadata availability:
\begin{itemize}
  \item Minimal.
\end{itemize}

Percent archived at a national archive:
\begin{itemize}
  \item 0 percent.
\end{itemize}

\subsection{Beth Judge - SC Sea Grant}

Number of data sets:
\begin{itemize}
  \item Individual PI.
\end{itemize}

Spatial/temporal variables of data set:
\begin{itemize}
  \item Geological, currents, mapping (vector and raster), side-scan sonar.
\end{itemize}

Volume in bytes:
\begin{itemize}
  \item Unknown.
\end{itemize}

Format of data:
\begin{itemize}
  \item Arc/INFO, Excel, ASCII, and others.
\end{itemize}

Machine/operating system:
\begin{itemize}
  \item Windows NT, UNIX, varied.
\end{itemize}

Metadata availability:
\begin{itemize}
  \item Minimal.
\end{itemize}

Percent archived at a national archive:
\begin{itemize}
  \item Little.
\end{itemize}

\subsection{Andrew Meredith - NOAA/CSC Coastal Remote Sensing}

Number of data sets:
\begin{itemize}
  \item 3 databases.
\end{itemize}

Spatial/temporal variables of data set:
\begin{itemize}
  \item Water quality, biological.
  \item Coastal change analysis.
  \item National in nature.
  \item Unknown temporal.
\end{itemize}

Volume in bytes:
\begin{itemize}
  \item Less than 10 gigabytes.
\end{itemize}

Format of data:
\begin{itemize}
  \item Flat files, IMAGINE.
\end{itemize}

Machine/operating system:
\begin{itemize}
  \item NT, UNIX
\end{itemize}

Metadata availability:
\begin{itemize}
  \item 100 percent.
\end{itemize}

Percent archived at a national archive:
\begin{itemize}
  \item All.
\end{itemize}

\subsection{Alan Lewitus - Baruch Institute and SC DNR}

Number of data sets:
\begin{itemize}
  \item 7 personal data sets.
\end{itemize}

Spatial/temporal variables of data set:
\begin{itemize}
  \item SC
  \item Biochemical, hydrography, water column, estuarine.
\end{itemize}

Volume in bytes:
\begin{itemize}
  \item Approximately 10 megabytes.
\end{itemize}

Format of data:
\begin{itemize}
  \item Excel.
\end{itemize}

Machine/operating system:
\begin{itemize}
  \item Windows 98.
\end{itemize}

Metadata availability:
\begin{itemize}
  \item None.
\end{itemize}

Percent archived at a national archive:
\begin{itemize}
  \item None.
\end{itemize}

\section{Web Browser}

At this point in the workshop, a lengthy discussion took place
regarding the functionality of a web browser interface that many
thought would be usefuul to preview data accessible via DODS.  Many
ideas were put on the table as to what that functionality should be.
In summary, participants felt those functions should include:

\begin{itemize}

\item LOCATION/SUBSETTING 
\begin{itemize}
\item Graphics (rubber-banding).
\item Coordinates (typing in ranges or geographic regions).
\item Consistency (Boolean operations) for searching.
\item Space/time/variable (keyword) processing and acquisition.
\item Will be designed the way NCDDC system works (similar to FGDC clearinghouse).
\item Select by sub-setting.
\end{itemize}

\item MANIPULATION
\begin{itemize}
\item May want to see mean value.
\item Count by values (e.g., percent clear skies).
\item These characteristics would be nice, but not now. (Can be done
  by Java Applet).
\end{itemize}

\item DISPLAY
\begin{itemize}
\item Properties of data.
\item Graphics.
\item Relationship between variables.
\item Disclaimer information (constraints to using data).
\item Volume of data.
\end{itemize}

\item DOWNLOAD DATA
\begin{itemize}
\item Fully described data.
\item The data itself (in different formats).
\begin{itemize}
  \item ASCII
  \item Arc/INFO
  \item Matlab
  \item IDL
  \item NetCDF
  \item ADF
\end{itemize}
\end{itemize}

\item AUTOMATION
\end{itemize}

\section{How should DODS be implemented in the southeast?}

Anne explained that she was a firm believer in standards and that the
group now has an opportunity to use DODS as a standard to move data
around.  She encouraged all the participants to download servers from
Peter's web site and try them out and after that, think about where we
should go from here.  The goal is to establish a relationship in the
end.

Steve listed the major issues that he saw as being important:
\begin{enumerate}
\item What are you going to do about metadata, more or less, for DODS?
\item Links to other programs or groups.
\item Include outreach to others not present at the meeting.
\item System installation issues.
\item The need to develop a web browser interface on the client side of DODS.
\end{enumerate}

Based on these issues, the floor was opened up for suggestions/comments:
\begin{itemize}
  \item Installation nodes, should they be centralized or distributed?
  \item Time, money (resource availability).
  \item Time frame, sustaining after initial funding.
  \item Using DODS needs to be worked into everyday activities.
  \item Implementation and maintenance issues still need resolving.
  \item Security issues, PI's and system people need to be made more comfortable.
  \item The ``O'' in DODS was disconcerting for people not interested in oceanography.
  \item The data model may not be adequate for other fields.
  \item Data access protocol needs to be generalized.
  \item The more groups using the DODS data access protocol, and the
more tools that each community can share with each other, the better.
\end{itemize}

\section{CONCLUSIONS}

\subsubsection{Major Metadata Issues}

Listed below are the MAJOR metadata issues that the workshop participants saw as being the 
most important:
\begin{itemize}
\item Define/encourage some ``minimal'' set of semantic metadata.
\item Provide regional training/education.
\begin{itemize}
  \item Develop a white paper/web site.
  \item Done at an institutional level.
  \item Have a central training site.
\end{itemize}
\item Evaluate metadata requirements for ``locate'' function.
\end{itemize}

\subsubsection{Links To Other Groups}
\begin{itemize}
\item Link to FGDC. 
\begin{itemize}
\item Partnerships with others:
\begin{itemize}
\item GCMD (Global Change Master Directory)
\item FGDC
\item NCDDC
\item NAML (participate at national level)
\item Coastwatch
\item NSF/LABNET/``bubba net''
\item PMEL/NOAA Server.
\item Others...
\end{itemize}
\end{itemize}
\item FGDC clearinghouse link to DODS.
\item DODS core must go to FGDC clearinghouse metadata.
\item Other repositories of data should be identified and encouraged to use DODS.
\item Investigate links to regional education efforts.
\item Encourage use of DODS via outreach.
\item Help connect (training/facilitation).
\end{itemize}

\subsubsection{Inclusion of ``Those Not Here''}
\begin{itemize}
\item Timing of efforts:
\begin{itemize}
  \item Server/client availability.
  \item Pilot project/prototype.
\end{itemize}
\item Mechanisms to involve groups:
\begin{itemize}
  \item Define client needs of ``customers.''
  \item Link/leverage to current efforts.
\end{itemize}
\item Develop client?
\begin{itemize}
  \item Web client (NCDCC).
  \item Define web client needs (focus group).
  \item Fund development?
\end{itemize}
\end{itemize}

\subsubsection{System Installation}
\begin{itemize}
\item Security documentation:
\begin{itemize}
  \item Get definition/explanation.
  \item Disseminate to systems manager.
\end{itemize}
\item Centralized node vs. distributed node:
\begin{itemize}
  \item Can't happen at CSC.
  \item Who could/would be central node and/or repository?
\end{itemize}
\item Centralized ``help desk'' (exists at national level now):
\begin{itemize}
  \item Tutorial.
  \item Training-centralized.
  \item On-site training.
\end{itemize}
\end{itemize}

\subsubsection{Client Interface}
\begin{itemize}
\item Command line at core (national issue):
\begin{itemize}
  \item URL builder at user end.
  \item Define command line priority issues.
\end{itemize}
\item Define URL builder priority needs (comma delimited).
\item Decide during/after pilot project.
\item Create web client focus group. (HIGHEST PRIORITY).
\end{itemize}

\subsubsection{Resource Availability}
\begin{enumerate}
\item DODS/NOPP.
\item Partners.
\item Personal resources:
\begin{itemize}
  \item Money to build metadata (site-specific or centralized).
  \item Money to buy server computer (site specific).
  \item Money for systems administration and assistance (regionalized).
  \item Money for server installation and assistance (regionalized).
\end{itemize}
\end{enumerate}

\subsubsection{Implementation Plan}
\begin{itemize}
\item Feedback loop to DODS/users (what went smoothly, what were the ``gotchas'').
\item E-mail list that ties into status report:
\begin{itemize}
  \item It should have an archive function.
  \item Use DODS node to host list.
\end{itemize}
\item Pilot project.
\item Regional web page.
\item Unidata has a tracking mechanism.
\end{itemize}

\subsubsection{Program Sustainability}
\begin{itemize}
\item Incorporation into day-to-day activities.
\item Maintain partnerships:
\begin{itemize}
  \item Which are the most useful or valuable.
\end{itemize}
\item Status/update reports.
\end{itemize}


\section{Workshop Summary}

\subsubsection{General}

\begin{itemize}
\item DODS must work interactively with the FGDC Clearinghouse architecture (Z39.50) and 
metadata standard.
\item Needs to have a stronger front end for locating data.
\item Need servers for relational databases, Excel, Access, SAS, S+.
\item Needs the ability to show a disclaimer BEFORE a user accesses data.
\item Needs interface for both sophisticated and unsophisticated users, local \& global customer.
\item Auto generation of DODS URLs.
\end{itemize}

\subsubsection{Support}

\begin{itemize}
\item Servers must be easy to install.
\item Provide a help desk for answering questions on installing
  servers and developing clients.
\item High quality, easy to understand documentation.
\end{itemize}

\subsubsection{Security issues}

\begin{itemize}
\item Document known httpd security issues and make them available to DODS server 
implementers.
\item Ensure DODS software is secure.
\item Provide a means of restricting access to select data through
  password protection or IP exclusion and provide documentation on how
  to implement this. (This is already available in DODS release 3.2.)
\end{itemize}

Although DODS is middle-ware and does not attempt to provide data
location and analysis functions, workshop attendees felt that DODS was
not terribly useful to them without these.  In order to provide these
functions better, DODS should partner with other organizations.
Potential partners include:

\begin{itemize}
\item Federal Geographic Data Committee (FGDC).
\item Global Change Master Directory (GCMD).
\item National Association of Marine Laboratories (NAML) LabNet project.
\item NOAA National Coastal Data Development Center (NCDDC).
\end{itemize}

\subsubsection{Suggestions for years 2 and 3}


\begin{itemize}
\item Support for installing server.
\item Web browser and direct download for clients.
\item Create a mailing list for SE participants.        
\item The DODS proposal to the National Ocean Partnership Program,
  roughly 60K would be available to the southeast for support.  A
  suggestion for using these funds would be to hire one full-time
  programmer to perform the following tasks:
\begin{itemize}
\item Build a web site with information pertinent to DODS users in the
  southeast.
\item Provide help desk support.
\item Software development (support for web client and DODS URL builder in 
coordination with DODS development team).
\item Coordinate with FGDC.
\item Explore metadata to DODS relationships.
\end{itemize}
\end{itemize}



\section{Additional Information}

The use of persistent cookies is not allowed on federal government
systems.  The Department of Commerce's implementation of this policy
can be found at: 

\Url{http://www.doc.gov/webresources/CookiesPolicy.html} 

It should be noted that a number of agencies, including NOAA, have
reacted to this policy by not allowing the use of ANY cookies.

Several workshop attendees reported that they must include disclaimers
\emph{BEFORE} users can access their data.  A sample disclaimer follows:

\begin{quote}
DISCLAIMER: THE DATA AND ASSOCIATED DATA FILES FOUND USING THIS
SOFTWARE ARE PROVIDED ``AS IS,'' WITHOUT WARRANTY TO THEIR PERFORMANCE,
MERCHANTABLE STATE, OR FITNESS FOR ANY PARTICULAR PURPOSE. THE ENTIRE
RISK ASSOCIATED WITH THE RESULTS AND PERFORMANCES OF THIS SOFTWARE IS
ASSUMED BY THE USER.
\end{quote}

The executive order for the National Spatial Data Infrastructure which
is carried out by the FGDC is available at the following URL:

\Url{http://www.fgdc.gov/publications/documents/geninfo/execord.html}



\section{Attendees}

\begin{center}
\begin{tabular}{lp{1.5in}ll} \\
\textbf{Name} & \textbf{Organization} & \textbf{Phone} & \textbf{Email} \\
Anne Ball & NOAA/CSC - SE Regional Coordinator & (843) 740-1229 & Anne.Ball@noaa.gov \\
Steve Bliven & Facilitator & (508) 997-3826 & bliven@massed.net \\
Stephen Brueske & NOAA NWS & (843) 744-1732 & Stephen.Brueske@noaa.gov \\
Peter Cornillon & URI/NOPP/DODS & (401) 874-6283 &pcornillon@gso.uri.edu \\
John Ellis & NOAA/ NCDDC & (228) 688-4090 & john.ellis@nrlssc.navy.mil \\
Madeline Fletcher & Baruch Institute & (803) 777-5288 & fletcher@sc.edu \\
Chris Friel & Florida Marine Research Institute (FMRI) & (727) 896-8626 & Chris.Friel@fwc.state.fl.us \\
Jim Fritz & Rapporteur - TPMC & (781) 545-1346 & jfritz@tpmc.com \\
James Frysinger & College of Charleston & (843) 225-0805 & frysingerj@cofc.edu \\
Beth Judge & SC Sea Grant & (843) 727-2078 & ekjudge@clemson.edu \\
Alan Lewitus & Baruch Institute \& SC DNR & (843) 762-5415 & lewitus@belle.baruch.sc.edu \\
Andrew Meredith & NOAA/CSC Coastal Remote Sensing & (843) 740-1291 & Andrew.Meredith@noaa.gov \\
Jim Nelson & UNC Marine Sciences & (912) 598-2473 & nelson@skio.peachnet.edu \\
Dwayne Porter & Baruch Institute & (803) 777-4615 & porter@sc.edu \\
Reyna Sabina & NOAA/AOML & (305) 361-4324 & Reyna.Sabina@noaa.gov \\
George Shirey & NOAA/CSC - Metadata & (843) 740-1205 & George.Shirey@noaa.gov \\
Tim Snoots & SC Department of Natural Resources (DNR) & (843) 762-5651 & snootst@mrd.dnr.state.sc.us \\
John Ulmer & NOAA/CSC - Systems & (843) 740-1228 & John.Ulmer@noaa.gov \\
\end{tabular}
\end{center}


                


%%% Local Variables: 
%%% mode: latex
%%% TeX-master: t
%%% End: 
