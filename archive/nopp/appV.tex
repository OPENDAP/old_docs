\renewcommand{\chaptertitle}{Technical Interchange Meeting with %
Environmental Systems Research Institute}
\chapter{\texorhtml{}{Appendix E }\chaptertitle}
\label{app,esri}

%% $Id$

\begin{center}
November 7 \& 8, 2000\\
Environmental Systems Research Institute\\
New York Street\\
Redlands CA
\end{center}


\section{Meeting Summary}


\subsubsection{To facilitate data exchange technology between GIS users and scientific users} 

DODS has been a supporting technology to scientists for several years. It has provided an 
extensible and reliable architecture to share oceanographic and atmospheric data sets of spatial 
and temporal types, using Internet connectivity, and desktop applications common to research 
scientists. DODS users recognize that a significant volume of relevant spatial data exists among 
the GIS community. Understanding how common GIS store, document, read, and distribute data 
is essential to expanding DODS functionality and providing research scientist effective and 
timely tools. DODS users also recognize that non-scientists among the GIS community could 
greatly benefit from, access to the existing and future DODS technology, data and understanding 
of ocean scientists' needs.


\subsection{Main Points}

 \subsubsection{To identify and understand the commonalities in the technical design and development of 
ESRI products and DODS software/infrastructure}

The DODS community and ESRI have both designed and developed information architecture for 
distributing and accessing data from multiple hosts. Technical overviews and demonstrations of 
both approaches were given. Both systems share many similarities in their overall goal of 
distributed data publishing and access. However, these systems differ considerably in their 
technical implementation. Discussion also centered on supported data types and how to handle 
the advanced types that DODS partners need using ESRI tools. The common use of XML by 
both architectures may become a very important overlap as both designs mature. Developing 
tools to enhance data transfer across between both architectures 

\subsubsection{To assist ESRI in understanding ocean scientist user requirements}

ESRI has a long term business interest in continuing to expand its marine sector. Through 
periodic meetings of this type, ESRI hopes to better support their marine/scientist clients with 
expanded data models, distribution systems, and desktop tools. ESRI is also interested in 
providing meeting-planning support to their marine/scientist clients during the annual ESRI user 
conference. DODS partners gave detailed presentations and demonstrations of existing DODS 
projects, their requirements, and the main technical hurdles that DODS developers encounter. 
ESRI developers were especially interested in how ocean scientists used and modeled volume 
data types.



\subsubsection{To assist DODS designers and developers in understanding current GIS technology}

Senior software developers from ESRI's ArcIMS, SDE and Metadata groups provided detailed 
technical overviews and answered questions from DODS developers. Demonstrations of all of 
ESRI's major tools were provided. New ESRI software that is still in development was also 
discussed. Topic areas that frequently came up included storage of grid and volume data types, 
ESRI client needs when conducting data discovery, and the mechanics of extending ArcIMS 
with a DODS ``connector''. DODS developers specially note that more technical documentation 
is needed to describe the basic file-metadata requirements that an ESRI client needs to access a 
DODS data stream in order to provide those metadata from DODS Servers.


\subsubsection{To lay the groundwork for year 2 and 3 of the project}

This meeting was the first for the beginning of a three-year effort. Specific technical tasks were 
selected and discussed for development in year one. There was general discussion on how this 
meeting corresponds to other regional NOPP/DODS meetings and the national NOPP/DODS 
meeting. The outcome of these meetings as well as the results of year one development will 
likely direct year two and three plans. Project activity in year two and three will be increased.

In year one, DODS partners will build a DODS-ArcIMS connector that will provide DODS 
clients access to GIS data from an ArcIMS configuration. This will provide the background for 
DODS developers to build a mechanism that ESRI clients can use to access the DODS 
architecture in the future. DODS partners will begin working with the ArcIMS SDK to develop a 
connectivity mechanism for GIS clients to DODS servers.


\subsection{General Observations}

ESRI decided not to support the Z39.50 protocol for metadata. Metadata will be stored in an 
XML format within an RDBMS/SDE and will be accessible through full text searches submitted 
using ArcCatalog and published with ArcIMS.

The Geo-statistical Analyst has great potential for scientific use. When it can point to a data 
source located on a RDBMS/SDE its utility will be even more significant.


ESRI has a custom utility for loading grids into an RDBMS/SDE, however there are some 
combinations of RDBMS/SDE and ArcIMS that currently do not work together.

ArcIMS has some basic data access problems with heterogeneous network environments. These 
may be solved if the industry problems with the UFS are resolved. 

ESRI does not have a data type for volume or time-series data, however they are interested in 
building a capability to work with these data types.

XML is a fundamental part to data communication for ESRI web applications and may be part of 
the next major change in the design of DODS.

ArcIMS is extensible using standard languages such as C++ and JAVA. JAVA Beans will be 
provided that will be able to wrap COM based extensions.

Informix and DB2 have a more object oriented design to their spatial component compared to 
SDE on those RDBM's, and has therefore a thinner layer.





\subsection{November 7}


Introduction by Richard Lawrence

Meeting goals
\begin{itemize}
\item   Forum for DODS
\item   Discuss and form vision
\item   Technical overview
\item   Discuss commonalities
\end{itemize}

Meeting introductions
\begin{center}
\begin{tabular}{llp{2.5in}} \\ 
Richard Lawrence &      ESRI &          Professional Services \\
Jeanne Rebstock &       ESRI &          Environmental Solutions Manager \\
Simon Evans &           ESRI &          Professional Services, Marine Industry \\
Neil Millett &          ESRI &          Professional Services \\
Art Hadad &             ESRI &          Development Manager (ArcIMS) \\
Steve Copp &            ESRI &          Software Development (SDE) \\
Gene Vaatveit &         ESRI &          Software Development (Metadata) \\
Kim Burns &             ESRI &          Geography Network representative \\
Ted Habermann &         NOAA &          Principle Investigator  \\
Dan Holloway &          URI &           Data site population, user interface and server \\
Dale Kiefer &           USC & \\
Nathan Potter &         ORST &          DODS developer, JAVA, SQL Server \\
Frank Obrien & & \\
Chris Finch &           JPL &           Data provider, HDF Server and PC port of DODS \\
Len McWilliams &        Informix &      Geospatial solutions to Informix and Internet applications \\
Mark Ohrenschall &      NOAA &          Programming Freeform and DODS servers, SDE  applications and simple data type support. \\
Steve Hankin &          NOAA &        PMEL Ocean scientist requirements, early in DODS, original LAS  \\
John Cartwright &       NOAA &          ARCIMS and JAVA servlett \\
Frank O'Brien &         USC &           wants to extract data, serve to DODS, has proposal to NRL \\
Daniel Martin &         CSC/TPMC &      Meeting minutes \\
\end{tabular}
\end{center}

Jeane Rebstock comments
\begin{itemize}
\item Provides high level support to the ocean industries.
\item ESRI wants to grow ocean industry business.
\item ESRI wants to understand user needs.
\item Mentioned that NRL is interested in DODS/NOPP.
\end{itemize}


DODS Overview and Ocean Science User Requirements with Dan 
Holloway 

(for complete presentation see \Url{http://www.unidata.ucar.edu/packages/dods/} )
 
\begin{itemize}
\item NOPP Virtual Ocean Data Hub Project
\item 50 participants
\item 4 regional coordinators:
Texas A\&M
NOAA CSC
State of Maine
Oregon State University
\end{itemize}

\begin{itemize}
\item DODS is an architecture to subset, acquire and ingest data.
\item To enhance sharing.
\item To use client tools to facilitate - Directory, Inventory and Data components.
\item High functionality for data access.
\item Primary DODS development is on data and not on directory support.
\item DODS constraints:
\begin{itemize}
\item No reformatting of data.
\item Function with minimum metadata.
\item Easy to interface with existing systems.
\item Must contain essential metadata to function.
\end{itemize}
\end{itemize}

\begin{itemize}
\item Explanation of a typical URL - can be difficult to write, this may be a barrier.
\item DODS has several interfaces (command line, URL builder and GUI).
\item Describe a matrix of interfaces and client applications and their status and pros/cons.
\item 1 Terabyte of data, 300 data sets, 15 sites globally.
\item IDL, Excel and Web browser are the primary clients.
\item Now porting to JAVA, the JDBC based SQL server is now available for downloading.
\item Specialized servers have been written for different data sets.
\item DOS Data Model and constraints.
\item DODS Data Objects (DDS, DAS, Data).
\item Brief open discussion to reintroduce DODS at a high level.
\end{itemize}










RDBMS Talk with Len McWilliams

\begin{itemize}
\item Brief discussion on the benefits of using a relational database.
\item Informix has incorporated SDE functionality and especially projection capabilities into 
the core Informix spatial components.
\item DataBlade is customizable and written in C.
\item New extensions can be written.
\item TEDS project is involved with grid data.
\item There are geodetic extensions.
\item ESRI and Informix have a joint effort underway.
\item The blobs that contain spatial information in the RDBMS are in a ESRI format.
\end{itemize}
        OGC Comments

\begin{itemize}
\item All functions are OGC compliant.
\item OGC has a standard request format.
\item ESRI has an OGC listener/connector that monitors a specific port.
\item The listener converts OGC to an AXL request to forward.
\item OGC specifications and ESRI/Informix collaboration.
\item ESRI connections use XML/HTTP to web applications and Socket type to Databases.
\item OGC Web Mapping Testbed 2 is still working and in specification mode.
\item David Beddoe was the ESRI contact for the technical subcommittee.
\item Conducting testing of WMTB2 with ArcIMS.
\item FGDC funded a broad partnership between NGDC and NOAA's Climate Diagnostic 
Center to test WMTB2. NOPP is involved in testing those standards through NGDC. 
\item WMTB2 is important to NOPP.
\end{itemize}

        NOPP Meeting

\begin{itemize}
\item NOPP user meetings are crucial to understand user needs.
\item These meeting minutes will be incorporated into the regional meeting summary.
\item Are existing NOPP/DODS tools available to current users in need?
\item Overview of the 4 regional meetings underway.
\item NRL is considering a catalog engine using MEL in the development of DEI system.
\item Steve Hankin on board of national meeting, this will occur around March-May, no 
location yet.
\item ESRI is interested in attending and is on the invite list.
\item The regional meetings are primarily for data providers.
\item General discussion on the fate of FGDC if all federal executive orders were rescinded.
\end{itemize}





User Requirements and Live Access Server with Steve Hankin

(for full text see \Url{http://shark.pmel.noaa.gov/\~{}hankin/ESRI/})

\begin{itemize}
\item TMAP, Thermal Model Analysis Program.
\item Lots of software development.
\item Modelers rarely collaborate/connect with others and generate large volumes of data.
\item Focus on current efforts is to build a collaborative system.
\item Bulk of new/needed data access methods are for research purposes.
\item In situ, Satellite, Models, Moorings, Cruises, Drifters, Coastal.
\item Data management is not fully developed for drifters.
\item Plans are to have 3000 active in 6-7 years between surface and 2000m.
\item Other data types include multibeam sonar and ROV based video, samples and still 
imagery.
\item Jeanne mentioned that the spottiness of oceanographic data is a major problem, and that a 
new ESRI tool called the Geo-statistical Analyst Extension may be of interest.
\item Steve - 3D renditions/models of data, especially over time is essential to chemical 
oceanography applications.
\item Present ESRI strategy for grid data seems to be image based.
\item An ESRI staff person will talk later about using grid data with SDE, this is important.
\item Some researchers prefer grid data to limit the volume issue.
\item Level 2 data is along track data stream.
\item Level 3 data is gridded, calibrated and cleaned data.
\item Do users need access to original, full raw data\item Or is a level 3 scientific output optimal?
\item The types of users vary, there are instrument developers, modelers, educators.
\item Many prefer the grid product because of convenient size.
\end{itemize}

        Open Discussion

\begin{itemize}
\item There are many data management issues in visualizing time series data.
\item The TYROS satellites have vertical sounders that provide a rich data stream.
\item Radiosome observations are a rich data source with a long history.
\item Atmospheric and oceanographic research are closely tied.
\item Len - Navy has a similar project - TEDS in Monterey, CA and is related the MEL (master 
environmental library).
\begin{itemize}
\item Data collection.
\item Query
\item Integration
\item Access
\end{itemize}
\item The Environmental Scenario Generator Project is using mySQL with the NCEP re-
analysis grids and taking over much of the TEDS work.
\item TEDS/ESG are investigating Informix as a spatial solution.
\item Should we be investigating expanded partnerships?
\item The multi dimension aspect is crucial.
\item ESRI's goal to merge the computer gaming graphics advances with data driven needs.
\item Many of the animated graphics are not data driven.
\item Mark Abbot at Oregon State is a contact for DODS.
\item Dawn Wright at Oregon is using ArcIMS with Tiffany Vance at PME.
\item ESRI has worked with users collecting multibeam sonar data.
\item Discussion on how we work in true 3D, many move data to IDL.
\item What about commercial efforts?
\item SAIC (Science International Corporation) is a commercial NOPP partner and are also 
involved in the Environmental Scenario Generator Project.
\item Gulf of Mexico has large data sets in the Deepstar Program, maybe propriety but are 
working with DODS as a data provider.
\item Oil and gas industry has heavy data applications in the ROV business/annual inspection.
\end{itemize}


Live Access Server Demonstration

\begin{itemize}
\item A web GUI for data sets in formats supported by DODS.
\item Java map based with server side graphics.
\item Multiple back-end DODS applications doing the work.
\item Easy to set up.
\item Streamed data as text or bitmap image capability.
\item Several different views including X,Y,Z and time were rendered to a browser
        pan/zoom functions.
\item There are a variety of output types such as comma delimited, NetCDF.
\item DODS has a requirement to provide data to GIS users.
\item DODS doesn't want to require the user to have a license.
\item Wants to support typical GIS layer types.
\item Ended up choosing the (SVF) Single Variable Format - ArcView grid.
\item Unfortunately, there is no metadata associated with this format.
\end{itemize}

        Open Discussion

\begin{itemize}
\item Lack of float type support in the ESRI data model is a real issue with scientific grid data.
\item Float types require a pallette object for display.
\item Perhaps a modified WRLD file could help give ArcView access to NetCDF and to other 
science formats.
\item Can an extension to ArcView be written for NetCDF?
\item How do we get NetCDF data into 3D Analyst/Spatial Analyst?
\item Maybe the world file could direct a skip operation to the right location in NetCDF.
\item The large audience of ArcView users and tools warrants a work around for NetCDF.
\item HDF allows redirection/linking of header file to a data file, like the WRLD file config.
\item Maybe this could be wrapped in XML.
\item NCSA folks recast their data from float to integer.
\item People do not want intermediate file formats.
\item Maybe a quick solution could be built even if it is not the end all in design.
\item There are a lot of float data archives out there.
\item New ESRI products will have float capability for grids.
\end{itemize}

        Return to LAS Demonstration

\begin{itemize}
\item No temporary files are written.
\item Will performance become an issue?
\item It is designed for the extraction of modest slices of data.
\item DODS compresses the data stream.
\item DODS is an open distributed system for moving data, mirroring can be used.
\item Load balancing can be configured.
\item Choking of data requests can be delt with incrementally, it is not an issue now.
\item Use standard HTML over HTTP, so firewalls are not a problem.
\item JSP is used to render as a servlett.
\item Using JSP the performance hit is on startup/compile of the servlett.
\item After servlett startup it is in process server.
\item Connection to the core is with a Bean.
\item A NY site has it own data format and customized server, the data can be accessed using 
DODS.
\item A database of base URLs are stored on the LAS.
\item How does an application connect to DODS?
\item The request and response process is described in detail.
\end{itemize}

        Open Discussion

\begin{itemize}
\item The architecture is similar to the ESRI model and to a data blade.
\item IDL and MatLab do not have client libraries.
\item There is a JAVA API for NetCDF.
\item A remote RDBMS could also be connected.
\item Can we identify a group of commonalities?
\item AXL to DAD.
\item The binary data stream may be hard to work with.
\item Maybe integration is best done as two separate components.
\item Need to divide the problem into the binary blobs and the metadata.
\item In regards to metadata, if the AXL has the functionality for DODS communications we 
can avoid building a connector.
\item DODS does not need metadata, what metadata does ESRI clients need?
\item The GML and AXL is published.
\item AXL contain reference to the data stream.
\item Connections are made to ArcIMS with HTTP and by Sockets to RDBMS.
\item Perhaps XML can be used with DODS, is it adequate for client needs?
\item XML is used with and without data in ESRI products.
\end{itemize}




        Continuation of LAS Demonstration

\begin{itemize}
\item Multiple ways to access data from servers.
\item NCEP Re-Analysis, CDC as 2.5-degree square by 6 hour samples.
\item This data set is in the terabyte range.
\item Project level data management is impractical.
\end{itemize}

        Open Discussion

\begin{itemize}
\item CDC is the partner for the wester water assessment, now has SDE for coverages.
\item UNEP is building their version of a distributed server configuration with ArcIMS.
\item The UNEP site likely to be very large, and needs DODS-like data and resources.
\item UNEP is typically poorly funded, however Ted Turner is a major supporter of this.
\item Hands on demo of LAS with Pacific cruise data, slices, graphics etc.
\item DODS has a self-describing data design.
\item Prototype demo for the NOPP system described.
\item Sister server concept allows all to share in a collaborative setting.
\item Display can show model and observation data in the same session.
\item Non grid data example showing cruise based carbon measurements.
\item Great care was taken to render the data as simply as possible without corruption.
\end{itemize}


General Discussion
        
        Topics or question summary from the morning session

\begin{itemize}
\item What protocol is used with ArcIMS, HTTP or SOCKETS?
\item Are OGC Web Mapping Testbed Standards a spec yet?
\item Does ArcIMS employ an OGC listener?
\item To request a Geo-statistical Analyst demonstration for grid generation.
\item What grid types does ArcIMS and ArcMap support and how do we avoid conversion?
\item Can we point a WRLD file at a blob?
\item Discuss the DODS application server diagram with Art Hadad.
\item AXL specification should be given to all participants.
\end{itemize}

        What are the priority questions for the DODS - ESRI integration?
        
\begin{itemize}
\item How do we provide DODS data access to GIS users?
\item How do we provide GIS data access to DODS users?
\item Identify common issues tomorrow.
\item What is minimum metadata requirement for sharing data between DODS and ESRI 
users?
\item No minimum metadata requirements for DODS users.
\item Metadata can be encapsulated in many different ways.
\item DODS has error handling that covers basic QA/QC.
\item What is the basic metadata for an ESRI client?
\item Is the Marine XML standard something that should be investigated?
\item Marine XML has been under construction for several years.
\item Do we build a DODS connector or server application?
\item Do we need the ArcIMS SDK to build a DODS connector?
\item ESRI has a ArcIMS SDK that is in Beta and can be made available to Beta members.
\item Maybe a DODS connector would be a good first step.
\item See ArcIMS architecture design.
\item Discussion of the request and response process and role of a DODS connector.
\item ArcIMS SDK will not be released as a product in ArcIMS 3.1.
\item Does ArcIMS provide direct SQL access to the RDBMS?
\item Which side should we focus on, the connector or a spatial server?
\item Brief talk about the JIVA, Joint Intelligence Visualization Application.
\item JIVA began as a NIMA project with MOIMS, users wanted a vector map.
\item JIVA was basically a catalog of footprints and status maps.
\item JIVA was built with SDE for coverages.
\item With Informix envelopes or extents are stored within the RDBMS.
\item Brief talk on the USGS Geoengine.
\item The differences between the Oracle and Informix database model for spatial capabilities.
\end{itemize}


ArcIMS Demonstration with Hugh Kegan

\begin{itemize}
\item Map services can be built on the fly.
\item Geography Network overview.
\begin{itemize}
\item Maps, data, and geo-services.
\item This is a catalog of catalogs.
\end{itemize}
\item A map service can hit multiple data sources.
\item Overview of a customized JAVA site.
\item Menu expansion can be programmed.
\item Extraction of data can be implemented.
\item Netscape 6 should work, it has not been tested.
\end{itemize}


ArcMAP Demonstration

\begin{itemize}
\item Showed linking to web-based data sources at multiple sites.
\item Feature service data can be saved in ArcMAP.
\item Re-projection can be handled on the fly through XML.
\end{itemize}





ArcCatalog Demostration

\begin{itemize}
\item Goal is to be OGC compliant.
\item Allows query and selection on metadata.
\item Metadata is stored in either a geodatabase or in file, if shapefiles exist.
\item Searches through metadata on multiple database connections can be done.
\end{itemize}


Map on Demand Demonstration

\begin{itemize}
\item ArcIMS forwards a coodinate to ArcMAP.
\item ArcMap renders and image and forwards to ArcIMS.
\item General discussion on VPF to SDE and grid input to SDE.
\item Grids can be loaded into certain versions of SDE with a special ESRI provided tool.
\end{itemize}

ArcScene Demonstration

\begin{itemize}
\item To replace the 3D Analyst extension.
\item To be released as an extension to ArcView, ArcInfo desktop in first quarter 2001.
\item ArcObjects can be used to extend ArcScene.
\item Flybys, visibility domes, and vertical digitizing where shown.
\item Wireframe and draped surface renders.
\item DXF can be read as an input type.
\end{itemize}


Spatial Analyst Demonstration

\begin{itemize}
\item Does not render volumes.
\item Examples shown include cost surface, visibility, and mobility.
\item Multiple surfaces (criteria) were used to determine least effort path.
\item Paths can be restricted to a vector network, such as a road.
\end{itemize}


ArcPad Demonstration

\begin{itemize}
\item Pocket PC and WAP phone updates to an SDE/RDBMS in real-time.
\end{itemize}

SDE and GRID Discussion with Steve Copp 

\begin{itemize}
\item SDE will support floating point grids on 3.1 release.
\item The term grid to ESRI means specifically the ESRI GRID format.
\item The goal in Arc 8.x is to develop grid functionality regardless of grid/image format.
\item Any raster format supported by ESRI can go into SDE.
\item Aux files are used to store statistics on the grid data, if the native format cannot handle it.
\item See ESRI object model diagram for raster data objects.
\item The RDO (raster data object) uses some of the ERDAS library.
\item This mechanism allows extension by building a DLL and registering it to access non-
supported image formats.
\item The aux file can contain an attribute table, LUT, palettes, histograms projection 
information etc.
\item The aux file is automatically built and maintained.
\item ArcView and ArcInfo share the same grid compatibility in 8.1.
\item RDBMS can be used to manage the raster data.
\item Very large raster data sets work well in SDE/RDBMS.
\item Pyramids are generated and tiles are typically 1/4 of the image size.
\item Usually only a small blob is requested and can usually be passed on quickly.
\item The goal is to reduce trips to database and reduce number of tiles that are returned.
\item Pyramids are made only with Nearest Neighbor.
\item The underlying table construction is the same across RDBMS.
\item Nearest Neighbor, Bilinear and Bicubic available in Arc 8.x.
\item LZ77 (PNG,ZIP) is used to compress image data.
\item There is a JAVA and C API to SDE.
\item All pyramids must be completely in or outside of the RDBMS.
\item There is no data type for true volumes or time dependant data.
\item ESRI would like to investigate multi-dimensional data types.
\item ESRI would like to adopt a voxel/volumetric capability.
\item For display and analysis performance ESRI envisions using true discrete frame 
techniques as a starting point.
\item Each frame could be made into a pyramid.
\item Discussion on the possibly of making a cube/blob as a tile.
\item Informix Data Blade allows the indexed extraction from a blob.
\item Discussion on performance and blob/cube requests and sub-setting.
\item The TEDS program is trying to use tiles.
\item ESRI true 3D development is just beginning.
\item ArcIMS has specific file formats that are supported, same as ArcView 3 and MO.
\item WRLD file data is handled in AXL.
\item ArcIMS is needed unless access is available across TCP - LAN.
\item We can't expect a DODS provider to run ArcIMS.
\item DODS delivers data right to the local application.
\item Users can run functions on the server side.
\item DODS doesn't yet have a multi-user access problem.
\item ArcIMS 3.1 supports re-projection of images.
\item All pyramids are kept in a single table.
\item Tile size and levels are tunable, but probably will not yield a huge increase in 
performance.
\item ESRI has plans to make a HDF reader.
\item COM tools for data loading and display is on Windows only.
\item Unix clients are for access only.
\item Flat binary and LZ77 are the blob formats currently supported.
\item Compression is tracked through a visible flag.
\item Compression really helps on machines with a fast CPU and slow disks.
\item SID is lossy and must be decompressed to enter a DB.
\item This combination makes little sense and is not used.
\item JPEG 2000 maybe supported in SDE 8.2.
\item ArcObjects gives full access to data in memory.
\end{itemize}


\subsection{November 8}

Why Use a Relational Database Management System
 
\begin{itemize}
\item Not all data need reside in the database, external data can be linked.
\item An RDBMS is not always the optimum solution.
\item Index capabilities are a strength.
\item Rtree index method is scalable and self-balancing.
\item Integration of many advanced data types.
\item Relationships can be modeled.
\item Performance can be enhanced through parallel architectures in scalable configurations.
\item Ease of management for large data sets.
\item Projection of the data can be done on the fly.
\item Access to SQL queries.
\item Backup, recovery, rollback, replication, transactions, multi-user, and other routine 
RDBMS functionality.
\item The Informix and IBM/DB2 approach is more object based then Oracle.
\end{itemize}

        General Discussion

\begin{itemize}
\item How does a client make a request, and conduct discovery of a data source?
\item Answer - SQL, JDBC, ODBC, ArcSDE.
\item In the GEODE project (USGS) a GUI interface was developed with pre-compiled SQL 
queries.
\item Queries can also be constructed on the fly through an application.
\item In ArcIMS queries are made via HTTP.
\item In the JAVA world the discovery mechanism is missing.
\item The science data types vary and effect what is available on request.
\item A published schema may be hard for a client application to process.
\item GEODE had a server resource index like the ArcIMS directory list.
\item DODS general overview.
\end{itemize}

        What are the interoperable issues to provide access to GIS clients.

\begin{itemize}
\item See S.Hankin web page for full text: 

\Url{http://shark.pmel.noaa.gov/\~{}hankin/OCMIP/DODS/DODS.htm}.
\item Steve Hankin describes the mechanism to enable a DODS client using a diagram.
\item A server is needed for each format of data provided.
\item Format and location of data is abstracted.
\item Scientist must write the read method if they are not using a supported format such as 
NetCDF, HDF or FreeForm.
\item Writing a server, using FreeForm maybe a barrier for some scientist to publish data.
\item Freeform data description method can help decipher many formats.
\item Most scientific data is generic.
\item Need to reduce the barrier to adoption of DODS, scientist, have little time to write 
servers.
\item Scientists do not want to reformat data.
\item A registry of servers is available.
\item A web crawler is also under construction.
\item Goal is....GIS client and science data.
\item Goal is....GIS data and science client.
\end{itemize}


ArcIMS Discussion with Art Hadad

\begin{itemize}
\item We need a bi-directional mechanism for the integration of scientific and GIS data.
\item We need to provide GIS users access to the Terra bytes of scientific data.
\item MODIS is a new and huge data stream of hyperspectral data in the 10-100m spatial 
resolution, destined for GIS clients.
\item MODIS data is in the EOSHDF.
\item LIDAR is another major data source for measurement of elevation, vegetation and tree 
counts.
\item How do we get a DODS server into the ArcIMS drop down list, how much metadata is 
needed for a GIS connection?
\item General introduction to the ArcIMS architecture:
\begin{itemize}
\item Presentation
\item Business Logic
\item Storage
\end{itemize}
\end{itemize}


ArcIMS Architecture Diagram

\begin{itemize}
\item Any client that can read XML can access data from ArcIMS.
\item Discussed ArcIMS clients, HTML, JAVA, AEJ, ArcView, ArcMAP etc.
\item The web server has several connectors, ASP, WMS 1.0, Cold Fusion etc.
\item In 3.1 there is a set of JAVA Server Beans forming the applink used to write your own 
connector.
\item AXL is sent to app server.
\item The POST mechanism is the preferred mechanism.
\item The appserver forwards AXL to the spatial server and off to storage to retrieve data.
\item The manager is built with JAVA applets.
\item The spatial servers are written in C++,  this is the only part left to port to Unix.
\item General discussion on major ArcIMs components
\item ArcIMS is being ported to Linux, SGI, AIX
\item Spatial servers do the major work.
\item More physical servers can be added and managed through virtual servers.
\item AXL is the glue that binds the configuration to the request and response.
\item The spatial server is a container.
\item 5 extensions out of the box include, image, feature, query, geocode, and extract.
\item Custom services can be extended.
\item Multiple workspaces exist for SDE, raster....
\end{itemize}

Spatial Server Diagram

 


\begin{itemize}
\item Extensions to ArcIMS are very open.
\item With an Image Service and simple clients, the request will return a URL of the location 
of an image.
\item With an Image Service and smarter clients, the response includes the XML encoded 
binary data (GIF, JPEG, PNG).
\item With an Image Service a get feature request will respond with text encoded within an 
inflated XML file.
\item An Image Service is server side rendering.
\item With a Feature Service the response is an inflated AXL file with embedded proprietary 
binary stream.
\item Clients render in feature streamed configurations.
\item The servers render in Image Server configurations.
\item What do we need to know to build a DODS connector.
\item It appears that a DODS connector must handle communications in both directions, from 
DODS URL to AXL and from an AXL stream output from the app server to a DODS 
stream.
\end{itemize}

        
Here are the options

\begin{itemize}
\item Extend the connectors.
\item Add a new data source.
\item Modify on the client. 
\end{itemize}




Central Cloud Diagram

 


\begin{itemize}
\item ESRI uses the Xercese tool to parse XML into standard usable objects.
\item Thick client construction is challenging within DODS design.
\item XML is the basic solution to a variety of changing problems that ESRI faces.
\item What is needed to make a new data source - unpublished technology the ArcIMS SDK?
\item The ArcIMS SDK will be a product no sooner then 6 months but maybe later.
\item ArcIMS SDK will not have admin components for security reasons.
\item ArcIMS spatial server is C++.
\item ESRI has a prototype that is essentially a JAVA proxy server extension to the spatial 
servers.
\item ESRI also has a prototype that allows the registration of a COM object.
\item Design work for connectors will remain stable, not so on the server side - possibly.
\item Getting a DOD client to talk with a ArcIMS server is recommended as a first, easiest 
step.
\item The technology of the ArcIMs SDK is in flux and would effect development within the 
business logic section to ArcIMS.
\item ESRI recommends development of a connector for DODS client to GIS data.
\item A metadata server extension will be available on ArcIMS after ArcIMS 3.1.
\item JDBC and ODBC connections to the server will be available in the future.
\item Distributed GIS is the goal.
\item The NOPP project has a 3-year plan. Higher funding levels are slated for year 2 and 3.
\item A DODS connector.
\item General agreement that in year 1 we will develop a DODS connector.
\item This will also allow developers to better understand ArcIMS development and allow the 
ArcIMS SDK to stabilize.
\item From the web server back the configuration works on SOCKETS.
\item Discussion on the emap link.
\item Discussion on the appserver link functionality for communication.
\item A set of JAVA Beans will be provided for wrapping AXL to read/write Map, Layer, 
Coordinate System, and other operations
\item This set of BEANS handles the SOCKET communications.
\item James Gallagher is the principal developer behind DODS.
\item HTTP and XML has been in the plans for the past year.
\item TomCat will be fully supported in 3.1, now in Beta 2.
\item The application server link is a wrapper class (JAVA Bean) to simplify development.
\item More comments about COM implementation.
\end{itemize}

        First Year Goal

\begin{itemize}
\item DODS team evaluates the appserver link.
\item Writing a DODS connector.
\item We can probably create the connector with the non-ESRI funds and later use the ESRI 
funds to address the appserver work.
\item Showing the new functionality of the connector will be important.
\item Connectors do not communicate among one another.
\item To be registered with the Geography Network, a map service or ARCIMS is not required.
\item Can the GN handle legacy systems.
\item Some general discussion.
\item ESRI clarifies that the JAVA BEANS do not decode the feature stream class (at this 
point), only ESRI clients can decode the feature-streamed class.
\item The ArcIMS SDK is available now only to Beta users.
\item Query Server is necessary to get back and XML open stream.
\item Beans will be in the SDK to parse feature streams.
\item Discussion on the performance burdens of a query server, especially between the spatial 
server and the connector.
\item Maybe a Bean could deliver a binary out to the DODS client and skip the connector.
\item Discussion on the pros and cons of a feature service.
\item Using scale dependency can help resolve performance problems, maybe an Image Server 
at a broad scale and Feature Server at a fine scale.
\item ESRI recommends server side processing with HTTP in general.
\item A query service can request against either an Image or feature service.
\item Layer contents can be dynamic.
\item Projections can be done on the fly, including images.
\item How do we map to a DODS client?
\item DODS requests are basically an SQL-like form.
\item Discussion on how does the DODS client receives data to render as polygons or lines?
\item The regional groups are to define some of the limitations of the DODS data model.
\item NFS systems can be a problem, a mixed file system (windows/linux) can be a problem 
for multiple ArcIMS Spatial Servers, this is especially problematic with dynamic map 
services.
\item Sun and Microsoft may be able to resolve their differences on the UFS.
\item The GN uses 20 Map Servers, 3 web servers, and 1 Oracle/SDE with maximum virtual 
servers, over a million maps per day.
\item All services and data are hosted at ESRI.
\end{itemize}


NOAA NNDC Server and SDE with Ted Habermann

\begin{itemize}
\item Overview of NOAA Server and SDE.
\item SQL functionality and potential.
\item A database is built to store:
\begin{itemize}
\item Data Info.
\item Data Source.
\item Display Look.
\item Search Look.
\item Query Look.
\end{itemize}
\item A browser interface is used to manage the catalog that is housed in a RDBMS MySQL.
\item Nathan wraps the JDBC request into a DODS URL.
\item PERL is used within the data manager tools.
\item Demonstration of the SDE license manager monitor page.
\item Most of the data management difficulties are social and not technical.
\item Not a lot of blob data, except satellite sources, in scientific data.
\item Informix allows SQL access of RDBMS.
\item Demonstration of the SPIDR2 database.
\item Sara Graves is authoring the Earth Science XML schema, maybe solution for 
commonality.
\item Metadata will be crucial here also.
\item Search mechanisms are still unclear at this point.
\item It is easy to forward and DODS URL to a SQL.
\item Once data is in the NNDC it is DODS/ArcIMS accessible.
\end{itemize}


Geo-statistical Analyst Demonstration

\begin{itemize}
\item Variety of grid options including CoKrigging.
\item Will be released in the 1st quarter.
\item Chi sq. will come later, in a follow on release.
\item The core utility is the visualization of the analysis and integration with spatial data 
sources.
\item Processing done in existing map units.
\item Examples included semi-variagram, error mapping, and clipping visual display to a 
polygon.
\item Approximately 100,000 data set processed in 20 minutes.
\item General discussion on the grid resolution output.
\item DODS staff saw great opportunity for scientific use.
\item Data sources include personal geodatabases. SDE sources in later releases.
\item Open discussion on multiple topics.
\item Multiple data types, contours, and grids can be incorporated.
\item New shape files/geo-databases can be output.
\item Ted discussed ArcIMS and SDE license changes.
\item The database approach is expanding.
\item The SDE component to Informix enhanced performances through load balancing.
\item SDE is similar to a driver.
\end{itemize}


Geography Network Presentation

\begin{itemize}
\item Includes publishers - government and non-profits who do not charge for data.
\item Partners - commercial units typically selling a service or data.
\item A collaborative to publish and share data and services, especially live data.
\item A type of marketplace.
\item Search capabilities of catalogs.
\item The all publisher list represents the ArcIMS sites.
\item Contains 4 geoservices, such as a geocoder.
\item 187 clearinghouses
\item Also contains ``solutions'' which are interactive ArcIMS sites.
\item Description of the registration process.
\item Some discussion about whether a non-ArcIMS site can provide a data stream.
\item ESRI will adopt the new international metadata standard.
\item ESRI developing a new tool to extract necessary metadata from a registrant site.
\item Exposure is a benefit to providers.
\item It is unclear if a non-ESRI map service could actively participate delivering data.
\item The GN has a fixed user interface.
\item ESRI is looking for 24X7 participants with data sources or services of national interest.
\item This is an ArcIMS site currently using a single SDE/RDBMS sources hosted by ESRI.
\item There is no specific QA/QC element or review of the sources or services.
\item Map services include Web Map Server (WMS) and AXL designed sites.
\item ESRI clients cannot access WMS services.
\item Guided tour of several GN pages.
\item There are redundant system components.
\item The GN hub submits an AXL request to the remote provider.
\item Password protection is an option.
\item Gateway sites like Texas exist, private GN's are a possibility also.
\item Custom provider applications can be hosted.
\item ESRI recommends an ArcIMS or WMS as the best mechanism to participate in the GN.
\item NOAA ``Lights at Night'' maybe a good map service to publish.
\item Published services access TIFF, JPEG, MrSID and standard ArcIMS readable formats.
\item MrSID performance isn't optimum.
\item ESRI has resampled and made pyramid data sets with scale dependencies to increase 
performance.
\end{itemize}


Metadata Discussion with Gene Vaatveit

\begin{itemize}
\item ESRI does not currently support a search using the Z39.50 protocol.
\item Metadata searches are to query local or ArcIMS sources.
\item ArcIMS metadata server should be released in 3.2 and runs with XML.
\item This sits on top of SDE/RDBMS.
\item ESRI has discussed development with BlueAngle, doesn't look likely.
\item Currently metadata is either stored on disk or in an RDBMS.
\item ArcIMS will contain a metadata extension that queries an SDE/RDBMS configuration.
\item ESRI has developed its own database design for metadata storage.
\item The ESRI design is more text-based, somewhat like Yahoo or generic document system.
\item Metadata can be stored separate from the data, and is not mandatory.
\item Searches can be full text or field based.
\item There is no enforcement of mandatory tags.
\item ESRI has made a prototype that works with Blue Angle.
\item Z39.50 sites work poorly in firewall configurations, they do not cross on port 80.
\item BlueAngle would be very expensive to re-package within the standard ESRI catalog.
\item There has been talk about wrapping a Z39.50 request in XML.
\item The OGC catalog service is a wrapper on Z39.50.
\item Isite is the server search component.
\item XML has some distinctive advantages, flexibility, open.
\item Full text searches of a RDBMS can be problematic, ESRI does not yet utilize the specific 
tools that each RDBMS vendor has to enhance this function.
\item ESRI is seeking a common solution not tied to a particular RDBMS.
\item Perhaps a new release of SDE will have more advanced text search capabilities.
\item Public classes exist for read/write of shapefiles.
\item Shapefile weakness is that it is defined with dbf and field names are restricted to 8 
characters.
\item How do we make the scientific data available to the GIS users?
\item Mark Ohrenschall is writing a translator for shapefiles.
\end{itemize}


        
Open Discussion and Recap of Issues

\begin{itemize}
\item A WMS will not show as a ArcIMS service or layer to an ESRI client.
\item The Geo-statistical Analyst will not read data from SDE in release 1.
\item The Geo-statistical Analyst looks very interesting for research purposes.
\item First step is to build a DODS connector, probably with existing resources.
\item 15K is available for consulting.
\item March/April is the national meeting.
\item Lets get the beta 3.1 ArcIMS to start investigating.
\item General discussion on experienced DODS participants.
\item Peter Cornillon is the primary scientist behind this.
\item James Gallagher is the chief architect of DODS.
\item OSU is the source of the newer SQL activities.
\item The MATLAB and IDL component are mostly out of the Northeast.
\item NOPP and DODS relationship discussion.
\item Ted is leading the policy/coordinating effort.
\item Scientific data to GIS clients is Peter's priority.
\item DODS reflects a broad selection of the ocean community.
\item Lets get the AXL specifications out to the mailing list.
\item Do we have any design specifications for the DODS connector?
\item The development path design will be ad hoc.
\item We'd like to work on the integration of the DODS SQL server and SDE.
\item Informix may be the preferred platform for working with the scientific spatial data.
\item How do we get grids into the Geo-statistical Layer?
\item There is no volumetric data type within the ESRI domain.
\item The new 3D Analyst Extension is a 2.5 D application.
\item How can we continue dialog with the marine user representation with ESRI?
\item Simon and Richard are the lead ESRI contacts.
\item ESRI would like to develop formal user requirement profiles for marine users.
\item ESRI will have a marine track at the user conference.
\item Ted will be submitting a NOPP paper to the ESRI user meeting.
\item Simon may be able to help coordinate a 10-15 person meeting that overlaps with the user 
conference.
\item DODS and NOPP interests may be broader then usual ESRI centric to be hosted within 
the user conference.
\item The group would probably be smaller then this workgroup.
\item Will need the ArcIMS SDK to work with the feature streaming capability.
\item How do we transition from the connector that supports DODS clients to GIS data?
\item It is still unclear how the connector can or can't support development of GIS client 
connection to DODS data sources/servers.
\item Continue talks with Informix.
\item Document user profile access to ESRI dev.
\item User conference may be a good opportunity to show a new capability of DODS.
\item Coordinate IMS SDK.
\item Organize a list of data.
\item If ESRI supported WMS, DODS could publish to GN.
\item There is a capability of publishing WMT from DODS, the (WMT-DODS gateway) with 
grid data only.
\item This is a prototype on WMT1.0.
\item A WMT2 is under construction but unavailable.
\item We should take a look at the technical specification of the Monterey grid work (TEDS).
\end{itemize}

\section{Attendees}

\subsubsection{ESRI}

\begin{center}
\begin{tabular}{lll} \\ 
Richard Lawrence &      rlawrence@esri.com &            909-793-2853 x 1700 \\
Jeanne Rebstock &       jrebstock@esri.com &            909-793-2853 x 1160 \\
Simon Evans &           sevans@esri.com &               909-793-2853 x 2380 \\
Neil Millett &          nmillet@esri.com &              909-793-2853 x 2709 \\
Art Hadad &             ahadad@esri.com &               909-793-2853 x 1055 \\
Steve Copp &            scoop@esri.com &                909-793-2853 x 1480 \\
Gene Vaatveit &         gvaatveit@esri.com &            909-793-2853 x 2335 \\
Kim Burns &             kburns@esri.com &               909-793-2853 x 2712 \\
\end{tabular}
\end{center}

\subsubsection{DODS}


\begin{center}
\begin{tabular}{lll} \\ 
Ted Habermann &         Ted.Habermann@noaa.gov &        303-497-6472 \\
Dan Holloway &  d.holloway@gso.uri.edu &        401-874-6831 \\
Steve Hankin &          hankin@pmel.noaa.gov &  206-526-6080 \\
John Cartwright &       jcc@ngdc.noaa.gov &             303-497-6284 \\
Nathan Potter &         ndp@oce.orst.edu        &       541-737-2293 \\
Mark Ohrenschall &      mao@ngdc.noaa.gov &             303-497-6507 \\
Dale Kiefer &           kiefer@usc.edu  &       213-740-5814 \\
Frank Obrien &          fjobrien@home.com       &       714-730-6858 \\
Chris Finch &           chris.finch@jpl.nas.gov & 818-354-2390 \\
\end{tabular}
\end{center}


\subsubsection{Other}
                                                                                
\begin{center}
\begin{tabular}{lll} \\ 
Len McWilliams &        len.mcwilliams@informix.com &   703-847-3359 \\
Daniel Martin &                 dmartin@tpmc.com        &       781-544-3803 \\
\end{tabular}
\end{center}

\section{Select HTTP References}

\begin{description}
\item[NOPP]     \Url{http://core.cast.msstate.edu/NOPPpg1.html}
\item[DODS Home]        \Url{http://www.unidata.ucar.edu/packages/dods/}
\item[LAS presentation] \Url{http://shark.pmel.noaa.gov/\~{}hankin/ESRI/}
\item[NOPP Proposal]    \Url{http://www.unidata.ucar.edu/packages/dods/archive/proposals/nopp-html/nopp.html}
\item[ESRI User Conference]     \Url{http://www.esri.com/events/uc/index.html}
\item[Open GIS Consortium]      \Url{http://www.opengis.org/}
\item[NVDS]     \Url{http://nndc.noaa.gov/?home.shtml}
\item[NNDC]     \Url{http://nndc.noaa.gov/?http://ols.nndc.noaa.gov/plolstore/plsql/olstore.main?look=1}
\item[PMEL]     \Url{http://www.pmel.noaa.gov/}
\item[SDE License status]       \xlink{http://dev.ngdc.nndc.noaa.gov/cgi-bin/wt/nndcp/\-ShowDatasets?\-query=\&dataset=400029\-\&search\_look=1\-\&group\_id=14\-\&display\_look=1}{http://dev.ngdc.nndc.noaa.gov/cgi-bin/wt/nndcp/ShowDatasets?query=\&dataset=400029\&search_look=1\&group_id=14\&display_look=1}
\item[Newport Surf Report]      \Url{http://nwprtsrf.oce.orst.edu:8080/nwprtsrf/surf.html}
\item[NNDC Developer]   \xlink{http://dev.ngdc.nndc.noaa.gov/cgi-bin/wt/nndcp/Devel\_Site\_Map}{http://dev.ngdc.nndc.noaa.gov/cgi-bin/wt/nndcp/Devel_Site_Map}
\end{description}


\section{Table of Acronyms}

\begin{description}
\item[API]              Application Programming Interface
\item[ArcIMS]           Arc Internet Map Server
\item[AXL]              Arc eXtensible Markup Language
\item[CDC]              Climate Data Center
\item[COM]              Common Object Model
\item[DODS]             Distributed Oceanographic Data System
\item[ESRI]             Environmental System Research Institute
\item[FGDC]             Federal Geographic Data Committee
\item[GML]              Geography Markup Language
\item[GN]               Geography Network
\item[GUI]              Graphic User Interface
\item[HDF]              Hierarchical Data Format
\item[IDL]              Interactive Data Language
\item[JDBC]             JAVA Database Connection
\item[JSP]              JAVA Server Pages
\item[LAS]              Live Access Server
\item[MEL]              Master Environmental Library
\item[MrSID]            Multi Resolution Seamless Image Database
\item[NCEP]             National Center for Environmental Prediction
\item[NNDC]             NOAA National Data Centers
\item[NOPP]             National Oceanographic Partnership Program
\item[ODBC]             Open Database Connectivity
\item[OGC]              Open GIS Consortium
\item[PMEL]             Pacific Marine Environmental Laboratories
\item[SDE]              Spatial Database Engine
\item[SDK]              Software Development Kit
\item[SVF]              Single Variable Format
\item[TEDS]             Tactical Environmental Decision Support System
\item[UNEP]             Unite Nations Environmental Program
\item[UFS]              Universal File System
\item[VPF]              Vector Product Format
\item[WAP]              Wireless Access Protocol
\item[WMS]              Web Map Server(OGC interface specification)
\item[WMT]              Web Mapping Testbed
\item[XML]              eXtensible Markup Language
\end{description}


%%% Local Variables: 
%%% mode: latex
%%% TeX-master: t
%%% End: 

