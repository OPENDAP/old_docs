\renewcommand{\chaptertitle}{Northeast Regional Workshop}
\chapter{\texorhtml{}{Appendix D }\chaptertitle}

%% $Id$

\begin{center}
January 8th - 10th, 2001\\
URI - W. Alton Jones Campus\\
West Greenwich, RI\\
\end{center}

\section{Introduction and Workshop Objectives}

The meeting convened with Linda Mercer, NE regional DODS coordinator,
asking the audience to introduce themselves (see \sectionref{IV,attendees}).  An
overhead was displayed showing the following questions that the
workshop attendees were to address:

\begin{enumerate}
\item Is there an area(s) of common interest in the Northeast?
\item What datasets will be served via the system?
\item Is the DODS data model adequate for the datasets to be served in the NE?
\item What are the important interface issues for users in the region?
\item What types of semantic metadata will be required and what
  semantic standards will be used?
\item Is a central regional node required to provide coordination
  including services such as user support, data location, etc.?
\end{enumerate}

\section{Overview and Demonstration of DODS}

Dan Holloway, URI DODS technical lead, went to the DODS demonstration
website and proceeded to show a series of web pages describing DODS.
A complete overview of the DODS data model can be found at:
http://po.gso.uri.edu/~dan/dods-regional-
workshops/dods-regional-workshops.html.

Dan explained the five steps involved to analyze data:
\begin{enumerate}
\item LOCATE
\item SUBSET
\item ACQUIRE
\item INGEST
\item ANALYZE
\end{enumerate}

The DODS portion involves the middle three steps, i.e., SUBSETTING, ACQUIRING, 
and INGESTING.  DODS does not focus on LOCATION or ANALYSIS.  What DODS 
is not, and the underlying philosophy of DODS, were covered.

Dan went on to explain that DODS is built from the bottom up, i.e., by
putting high functionality at the data acquisition level, and that
functionality decreases from the bottom up (the data level being at
the bottom, the inventory level in the middle and the directory level
at the top).

The underlying philosophy of DODS was reiterated and is twofold:
\begin{itemize}
\item Anyone willing to share their data should be able to do so via DODS.
\item The user should be able to use the application package with
  which she or he is the most familiar, to examine or analyze the data
  of interest.
\end{itemize}

Constraints of the DODS system include making it easy for the
scientist to serve data AND to make it easy to access data.

Syntactic and semantic metadata were described and how they relate to
the data, inventory, and directory levels.  Semantic metadata can be
sub-setted into use and search categories.  Use metadata can be
further subdivided into translational use metadata and descriptive use
metadata. The focus of DODS is on translational use metadata.

Four levels of interoperability at the data level (for data accessible
over the network) were explained as follows:
\begin{description}
\item[Level 0]  no syntactic or semantic metadata - FTP.
\item[Level 1]  rigid syntactic, no semantic metadata - DODS.
\item[Level 2] rigid syntactic, human readable semantic metadata - A
  subset of DODS datasets.
\item[Level 3] rigid syntactic, consistent semantic metadata; i.e.,
  machine-readable - A subset of the DODS Level 2 datasets.
\end{description}

Three DODS data objects were defined:
\begin{enumerate}
\item The data descriptor structure (DDS) - the syntactic metadata for a dataset.
\item The data attribute structure (DAS) - the semantic metadata for a dataset.
\item Data - the actual data in binary structure.
\end{enumerate}

Additionally, DODS servers support several other services: 
\begin{itemize}
\item .ASCII (an ASCII representation of the data).
\item .info (a more readable version of the .dds and .das combined).
\item .html form (a web-based form that will help to build a DODS URL).
\end{itemize}

It was emphasized that writing DODS URLS can be difficult!

Three classes of interfaces were described based on the difficulty
involved in building a URL.  These are in order of increasing
ease-of-use:

\begin{description}
\item[Command Line] No help is given creating a URL.
\item[General Purpose URL Builder] A menu-based interface based on
  data dimensions and names.
\item[Graphical User Interface] A URL builder based on geophysical
  parameter input.
\end{description}

The DODS data model was explained along with the data model
constraints.  The following operations are permitted when requesting
data:
\begin{itemize}
\item Projection.
\item Relational operations on list and sequence elements.
\item Rectangular and decimated subsets of arrays and grids.
\end{itemize}

And groupings of these data types:
\begin{itemize}
\item Array
\item Structure
\item List
\item Sequence 
\item Grid
\end{itemize}

Sequences and Grids are two separate constructs.

Data types served in the NE: 
\begin{itemize}
\item Hydrography
\item Moorings
\item Bathymetry
\item Model output
\item Satellite products
\end{itemize}

Global:
\begin{itemize}
\item Surface fluxes
\item Water column climatologies
\item Satellite products
\item COADS climatologies
\item Near real-time numerical model weather and ocean predictions
\end{itemize}

Dan demonstrated the methods of accessing data served by DODS.

An example of the Live Access Server (LAS) was given.
Steps involved:
\begin{itemize}
\item Browser requests GIF image from PMEL site.
\item PMEL Live Access Server passes request to Ferret.
\item Ferret requests subset of monthly COADS SST data from local disk.
\item Ferret requests subset of monthly NCEP Marine data from CDC/NOAA in Boulder.
\item Ferret re-grids COADS to NCEP grid, differences, generates a GIF
  image and returns the GIF image to Live Access Server.
\item Live Access Server returns GIF image to the browser.
\end{itemize}

Another example used the DODS Matlab Graphical User Interface to
select, request, adn download data directly into the Matlab workspace.

Dan gave an example using PMEL data and another utilizing Matlab.

\section{Workshop Participant Presentations}

A good portion of the second day of the workshop was devoted to
workshop participants giving presentations related to their
organization and how they tie into the DODS concept. To set a context
for the DODS discussion, David Mountain (NMFS, Woods Hole) gave an
overview of a NMFS proposal for ``Ocean Observing Systems in the Gulf
of Maine - A proposal to Integrate Existing Systems,'' presented to
directors from marine organizations/agencies in the Northeast region,
October 30, 2000, sponsored by the Regional Association for Research
in the Gulf of Maine (RARGOM). The purpose of the meeting was to
determine interest in developing an integrated system, comprised of
monitoring data from existing programs that will generate analytic
products in order to meet a broad spectrum of needs across the Gulf of
Maine region.

\subsection{Where Are we Now?}

Many Operational Observation Programs:
\begin{itemize}
\item National Weather Service
\item National Marine Fisheries Service
\item National Ocean Service
\item Dept of Fisheries and Oceans (Canada)
\item Division of Marine Fisheries (MA)
\end{itemize}

Where are we now?
\begin{itemize}
\item Many datasets available via the web.
\item Little coordination in the collection or analysis of the datasets.
\end{itemize}

Why Now?
\begin{itemize}
\item Improved technology.
\item Increased understanding of the Gulf Ecosystem.
\item From major research EOCHAB, GLOBEC, etc.
\item The future is now.
\end{itemize}

The Future is Now.
\begin{itemize}
\item An Integrated Ocean Observing System: A Strategy for
  Implementing the First Steps of a US Plan.
\item A report (R.Fresch, chair).
\end{itemize}

Recommendations:
\begin{enumerate}
\item Use and enhance existing observational systems.
\item Integrate into regional demo programs.
\end{enumerate}

System Components:
\begin{enumerate}
\item Measurement.
\item Data storage.
\item Data communication.
\item Analysis.
\item Integration.
\end{enumerate}

DODS can be the critical data communication component for a
distributed, integrated observing system in the northeast.


\subsection{Linda Mercer - Maine Department of Marine Fisheries}

The Maine Department of Marine Resources has numerous datasets, many
of which include physical parameters such as temperature and salinity,
that it would be willing to share via DODS.

Examples include:

\subsubsection{Environmental Monitoring Program }

Year started: 1989

Data collection site: BBH Station

Collection frequency: hourly

\begin{itemize}
\item Barometric Pressure (mb) (in)
\item Precipitation (in)
\item Relative Humidity (%)
\item Salinity (ppt)    
\item Sea Surface Temperature (c,f) 1905
\item Sea Bottom Temperature (c,f) 
\item Solar Radiation (langley/min)
\item Air Temperature (c,f)
\item Tide Height (ft)
\item Wind Direction (deg) 
\item Wind Speed (mph)
\end{itemize}

\subsubsection{Aquaculture Program}

Year started: 1988

Data collection site: finfish aquaculture sites

Collection frequency: baseline at time of application then yearly 

\begin{itemize}
\item Current speed (m/sec)
\item Dissolved Oxygen Depth Profile (%, mg/l)
\item Salinity Depth Profile (ppt)
\item Sea Temperature Depth Profile (c)
\item pH Depth Profile  (pH/m)
\end{itemize}

\subsubsection{Stock Enhancement Program}

Year started: 1985

Data collection site: coastal Maine Rivers

Collection frequency: depends on field schedule of various projects

\begin{itemize}
\item Air Temperature (c,f)
\item Dissolved Oxygen (Surface/Mid/Bottom) (mg/l)
\item Salinity (Surface/Mid/Bottom) (ppt)
\item Temperature (Surface/Mid/Bottom) (c,f)
\item Water Depth (m)
\item Secchi Disk (cm)
\item Weather Description (qualitative)
\end{itemize}

\subsubsection{Water Quality Program}

Year started: 1989

Data collection site: 1800 rain \& sample sites along coastal Maine

Collection frequency: 6/yr (min), March- November

\begin{itemize}
\item Salinity (ppt)
\item Water Temperature (c,f)
\item Tide Status (high/med/low)
\item Weather Description (qualitative)
\item Precipitation (in)
\item Dissolved Oxygen (mg/l)
\end{itemize}

\subsubsection{Survey - CDT}

Year started: 1989

Data collection site: various vessel surveys along Maine coast

Collection frequency: survey schedule

\begin{itemize}
\item Water Temperature Depth Profile (c/m)
\item Salinity (conductivity) Depth Profile (ppt/m)     
\end{itemize}


\subsection{Linda Mercer - Gulf of Maine Ocean Observing System}

The Gulf of Maine Ocean Observing System (GoMOOS) is a NOPP-funded
project that is developing a system of buoys and CODAR to collect
oceanographic and meteorological data in the Gulf of Maine and make
these data available on continuous real-time basis.  GoMOOS is a pilot
component of the Northeast Ocean Observing System (NEOOS) that is, in
turn, planned as one of approximately six regional observing systems
envisioned to constitute a national observing system.  

GoMOOS is a consortium of scientists, policymakers, and industry, and
the observing system is designed, not as a research project, but as a
utility that will build, deploy, operate, transmit/process/archive
data, and maintain the infrastructure required to do this.  The data
and information produced will allow those who depend on the Gulf of
Maine for their livelihood and well- being, and those whose business
is marine research, to undertake their pursuits and enhance the
understanding of the Gulf more efficiently and profitably than ever
before.

The initial phase of GoMOOS will consist of 13 buoys placed
strategically around the Gulf of Maine based on input from a variety
of user groups, and a CODAR system with four sites that will provide
coverage out to 200 km.

\subsubsection{GoMOOS Data Products Summary}

\begin{itemize}

\item Wave products:
\begin{itemize}
\item Height
\item Trends
\item Nowcasts
\item Forecasts
\end{itemize}

\item Meteorological.observations:
\begin{itemize}
\item Wind (hourly)
\item Wind (4 times daily)
\item Fog
\end{itemize}


\item Currents:
\begin{itemize}
\item Surface currents maps
\item Water column currents
\item Large scale circulation features
\item Interannual variation
\end{itemize}


\item Temperature Patterns:
\begin{itemize}
\item Sea surface temperature
\item SST composites
\item Archives
\item Water column temperature
\end{itemize}

\item Salinity

\item Dissolved Oxygen

\item Primary Productivity

\item Turbidity

\item Whale Sounds
\end{itemize}

Direct transfer of the data to scientists from GoMOOS can be
effectively handled via DODS. A web-based graphical User Interface
will be used to provide ``user-friendly'' data and data products that
address the information requirements of Gulf user groups.


\subsection{Nicholas Wolff -  Bigelow Laboratory for Ocean Sciences}

He mentioned that he is here to get more familiar with what DODS is
and to share that with researchers at Bigelow.  Nick works with Lew
Incze who previously expressed an interest in DODS for serving several
of his larval lobster time series with associated hydrographic data.


\subsection{David Mountain - National Marine Fisheries Service}

NMFS datasets for DODS

\begin{tabular}[t]{ll}
Now:&   Historic current meter records (NetCDF). \\
Future: & Hydrographic  (~1500 CTD casts/yr) \\
&       Zooplankton     (~ 700 longo tows/yr) \\
&       CPR             (monthly CPR transects) \\
&       Fish            (monthly trawls/yr) \\
\end{tabular}

\subsubsection{Problems/Needs}

Serving data - data in ORACLE

Retrieving data - plug-ins for ORACLE, S-Plus, Java?

David pointed out that the intention of NMFS is to serve all their
data via DODS.  They currently have some Matlab and IDL users.

\subsection{Bruce Tripp - Woods Hole Oceanographic Institute}

Data to share:
Data sources at WHOI would be potential candidates for DODS.

ECOHAB data:
\begin{itemize}
\item PI's sharing data and a website.
\item Buoy data.
\end{itemize}

Multiple sources for Gulf of Maine data.

Historic aspect of data needs to be explored.

Contaminated sediment database from the Regional Marine Research
Program in Gulf of Maine.

\begin{itemize}
\item Had multiple sources of data.
\item Difficulty in critiquing the data to determine quality.
\end{itemize}

There is a new ocean observatory, the Martha's Vineyard Coastal Ocean
Observatory headed by Jim Edson.  He has created a multi-observatory
website (draft version at this time).  However, the website probably
won't be used to house data.  There is collaboration between five
coastal observatories on the East Coast.  They could potentially house
a DODS server and share data.  Sediment transport and meteorology is
the primary focus of the observatory groups.


\subsection{Chuck Denham - USGS, Woods Hole}

Data to share:
\begin{itemize}
\item Current meter data.
\item Bathymetric data.
\item Data is fairly distributed but have to know where to get it.
\item Geographical range, Woods Hole (Gulf of ME, NY and NJ).
\item EPIC formatted data for ADCP.  Data descriptors are needed for instruments.
\end{itemize}

\subsection{Steve Hale - EPA Monitoring Assessment Program, EMAP}

MAIA - Estuary Report

Why Share Estuarine Data?
\begin{itemize}
\item Broad-scale and long-term ecological processes (e.g.,
  eutrophication, global change).
\item No single group can collect all the data. 
\end{itemize}

Impediments to data sharing.

Technology:
\begin{itemize}
\item Data formats
\item Hardware
\end{itemize}

Sociological.

How to exchange data:
\begin{itemize}
\item Directories/catalogs/clearinghouses
\item Standards (Z39.50)
\item Publishing on the web
\end{itemize}

How to integrate:

\begin{itemize}

\item Standards

\item Metadata
\begin{itemize}
\item Information on methods
\item Information on data quality
\end{itemize}

\item Databases:
\begin{itemize}
\item Core DBMS
\item Database constellation
\item Centralized metadata, distributed data
\item GIS
\end{itemize}
\end{itemize}

Coastal Zone 2000 Monitoring in the Northeast (map on display).

Showed metadata standard overhead.

EMAP Coastal zone works off an ORACLE database.

Bibliography on-line.

\subsection{Don Byrne - NJ Division of Fish, Game and Wildlife}

See \sectionref{IV,NJ}

New Jersey has a trawl survey and is willing to share data from this
survey that dates from 1988 to the present using DODS.  New Jersey
data are not currently available on a server.


\subsection{Wendell Brown - Univ. of Mass at Dartmouth}

Built an archive at the University of New Hampshire, with metadata.
Physical oceanographic data.

Advanced Fisheries Management Information System (AFMIS)
Professors Brian Rothchild and Allan Robinson were major contributors.

Employs simple ecosystem models.

Components:
\begin{enumerate}
\item Sensors and platforms.
\item Data and knowledge.
\item Models with data assimilation.
\item Reports distributed to users.
\end{enumerate}

Focus is on fisheries but can be used in other resource management functions.

Showed AFMIS functional elements.

Real-time forecast demonstration of concept (predicting where fish would be).

List of forecast products:
\begin{itemize}
\item Discussion of model initialization
\item Discussion of forecast products
\item Synoptic forecast images
  \begin{itemize}
  \item Georges Bank plan view
  \item Georges Bank vertical resources
  \end{itemize}
\item Tidal Velocity Forecast images
\end{itemize}

\section{More DODS}

Paul Hemenway of URI, with the use of a flip chart, provided a
graphical representation of how DODS works.  He stated that it takes
some effort to put available datasets up on DODS.  DODS servers are
specific to the data format you are serving (NetCDF, HDF, binary,
ascii).  The DODS client is specific to the analysis package/language
you are using (Matlab, IDL, EXCEL, etc.).

Dan Holloway explained that DODS has two ASCII servers: FreeForm
(handles binary very well and ASCII fairly well) and JGOFS (uses tree
method object).  He discussed how JGOFS data is accessed through DODS.
Dan pointed out that data location is an issue, and quality of data is
also important.  Bandwidth of data was also discussed.  Finally, data
need not be replicated (when considering setting up a separate
server).


\subsection{DODS Workshop Questions}

Listed below are some of the more important regional questions and
answers pertaining to DODS that were presented to the workshop
participants:

What would it take to be a DODS server (basic information)?

An informational listing of items necessary to set up a server was
thought to be very important. Maintenance issues should be among the
topics covered. The DODS technical staff at URI could provide the
informational listing and any support required.

What datasets will be served in the NE?

Most of the datasets discussed at the workshop were from the northern
part of the Northeast region, i.e. Gulf of Maine and Georges Bank.
More participation is needed from the mid-Atlantic regions.

Datasets in the Northeast:
\begin{itemize}
\item Fisheries databases
\item GOMOOS and other observing systems
\item ECOHAB
\item AFMIS
\item USGS - WHOI and MWRA
\item NOS tidal gauge
\item NOAA status and trends
\item Coastal 2000
\item REDIMS contaminants
\item Fish data
\item Shellfish
\item Temperature data
\item Climatology
\item Shoreline data collected by USGS (with focus on monitoring
  before and after storms)
\item Sediment transport data
\end{itemize}

Any modeling activities (data assimilation possibilities) by Lynch, Huije, GLOBEC.

Talking to the modeling community will be important!

Who is serving DODS datasets?  

\begin{itemize}
\item PMEL
\item UCAR
\item URI
\item USGS
\item NGDC
\item Goddard
\item Lamar
\item BIO
\item Mass Bay Modeling
\end{itemize}

Number of DODS client sites (unknown); can download DODS clients and servers.

Is the DODS data model adequate for the datasets to be served in the NE?

It was emphasized that data residing in DBMS format be accessible via DODS.

The need for a data catalog(s) and the development of a template for
the catalog was stressed:
\begin{itemize}
\item    Web-based.
\item Would be compiled into a regional, centralized site.
\item Searchable by keyword.
\item Geographic.
\item Should be able to serve up .gif and .jpeg images.
\item Show links to informational pages describing the datasets.
\item Would include pointers to URL's for the actual data.
\end{itemize}

What are important interface issues for users in the region?

Being able to view generated maps and graphics would be useful.
Presently, DODS does not support data in ACCESS format.

What types of semantic metadata will be required and what semantic
metadata standards will be used?

COARDS is the minimum standard so far.  Data providers are encouraged
to submit a maximum amount but DODS only requires a minimum amount of
syntactic metadata and no semantic metadata to serve datasets.

Is a central regional node required to provide coordination including
services such as user support, data location, etc.?

Yes, and the regional node should do the following:
\begin{itemize}
\item Provide user support until DODS users get up and running.
  Someone within each region, probably a person from an existing
  organization, should be chosen.
\item Setting up a DODS regional work group will be helpful.
\item Other groups that were not in attendance at the regional
  workshop should be included (CT, NY, DE).
\item Perhaps take DODS on the road to demonstrate its capabilities to
  others not at the regional workshop.
\item An outreach component is needed.
\item Get data online and document difficulty in using DODS.
\item Have a contractor do a demo for other states and regional players.
\end{itemize}

\section{Regional Pilot Project and Demonstration}

From the presentations and group discussion it appears that DODS has
focused (successfully) on large datasets and sophisticated users in
the region.  The group felt that many good data produced by resource
management agencies and research projects are yet to be found in small
data sets housed on a variety of platforms and in a myriad of
formats.  Wider access to these data is desirable but it is assumed
that the task of making them DODS accessible is a large task.  What
actually would it cost an organization or agency in new software,
hardware, and skills --- infrastructure as well as actual dollar
expenses --- to bring a data base on line in DODS?  A collaborative
demonstration project between the NMFS and state fisheries agencies of
ME, MA and NJ was discussed as one means to address this question and
to guide others in the process. Simultaneously, this demo would guide
the NOPP project in the selection of new DODS-supplied data tools that would be
broadly useful to the community.

The four agencies would have to assemble all fisheries data
(documenting QA/QC, formats, software used, etc.)  This task could be
large. DODS would support the task to bring all state fish data on
line (perhaps initially via a single server at NMFS) It is expected
that partial QA/QC and different sampling methods might make inter-
comparison difficult.  Discussion and resolution of these issues would
help to create a valuable regional dataset from the current collection
of disconnected fragments.  A full-time person, knowledgeable in
fisheries and fieldwork, perhaps based at URI, would be needed to
interact with state fishery staff to assist with in-state data
conversion/formatting as an initial step.

Other groups of smaller datasets are held by marine labs and
individual research projects that might also be assembled in a
parallel DODS demonstration to assess datasets collected for different
purposes and how to move them into the DODS system.

The demonstration pilot project should address the following:
(Ex: using fisheries data from NMFS, ME, MA, or NJ)
\begin{enumerate}
\item Creating one or more DODS servers for NMFS, ME, NH, and NJ.
  \begin{itemize}
  \item Develop a GUI interface that has higher resolution for NE
    (similar to Matlab).
  \item It may need to be modified for non-Matlab users.
  \end{itemize}
\item Define what being a DODS client is for each agency (ex: EXCEL).
\item Develop a web-based graphical demonstration using data from
  different sources.
\item Provide an HTML template listing regional datasets (a simple
  catalog).  The template should be able to get data from multiple
  remote sources and bring back to a web-site for viewing.
\item The demonstration project should include password protection.
\end{enumerate}

Other issues concerning the demonstration project:
\begin{enumerate}
\item Who will create the demo was discussed with no resolution yet.
\item A technical person and data person would be needed to do the demo.
\item Have the demo up and running before the national meeting.
\item The group discussed the other possibilities for demonstration
  projects such as using contaminant data created by the USGS and MWRA
  or the northeast observing systems including the new Martha's
  Vineyard Coastal Observatory.
\end{enumerate}

\section{Final Comments/Issues}

\begin{enumerate}
\item Decisions must be made on how much metadata needs to be
  attached to datasets.  Currently, regional datasets may not have any
  metadata associated with them.  Minimal semantic metadata standards
  need to be developed.
\item The willingness of other players in the region to serve up data
  via DODS needs to be determined.
\item Natural resource agencies may not have adequate data management
  resources to implement DODS.  There will need to be considerable
  ``hand holding'' provided to less technically sophisticated users.
  There will likely be different issues for different user groups.
\item More client server development is needed.
\item The need for a PowerPoint quick demonstration of DODS.
\item Model outputs would be a useful data type served by DODS.
\item There was a desire to have biological data (fish data) served by DODS.
\end{enumerate}

\section{Attendees}
\label{IV,attendees}

NOPP/DODS NE Regional Workshop
W. Alton Jones Campus, URI
January 8 - 10, 2001

\begin{center}
  \begin{tabular}[t]{lp{2.0in}l} \\ 
\textbf{Name} & \textbf{Organization} & \textbf{Email} \\
Wendell Brown &                 Umass Dartmouth & wbrown@umass.edu \\
Frank Bub &                     School for Marine Science \& Technology - SMST of UMass Dartmouth & fbub@umassd.edu \\
Don Byrne &                     NJ Division of Fish \& Wildlife& njfgwbyrne@plexi.com \\
Chuck Denham &                  USGS Woods Hole &       cdenham@usgs.gov \\
John Evans &                    Univ. of Maine          &       jevans@umeoce.maine.edu \\
John Fracassi &                 Rutgers University      &       johnf@arctic.rutgers.edu \\
Jim Fritz &                     TPMC                    &       jfritz@tpmc.com \\
Steve Hale &                    EPA - Narragansett Lab &         hale.steve@epa.gov \\
Paul Hemenway &         Univ. of R.I.           &       phemenway@gso.uri.edu \\
Dan Holloway &                  Univ. of R.I./DODS &            d.holloway@gso.uri.edu \\
Linda Mercer &                  Maine Dept. of Marine Resources & linda.mercer@state.me.us \\
David Mountain &                NMFS Woods Hole, MA & david.mountain@noaa.gov \\
David Remsen &                  Marine Biological Lab   &       dremsen@mbl.edu \\
Bruce Tripp &                   Woods Hole Oceanographic Institute & btripp@whoi.edu \\
Nicholas Wolff &                        Bigelow Lab             &       nwolff@bigelow.org \\
  \end{tabular}
\end{center}

\section{Ocean Monitoring Programs Conducted by
The New Jersey Department of Environmental Protection}
\label{IV,NJ}

\begin{center}
  \begin{tabular}[t]{|p{.75in}|p{0.5in}|p{1.0in}|p{0.5in}|p{0.5in}|p{0.5in}|p{0.75in}|} \hline
\textbf{Agency} & \textbf{Program} & \textbf{Variables} & \textbf{Samp.\-/Yr.} & \textbf{Svys\-/Yr.} & \textbf{From} & \textbf{Offshore Limit} \\ \hline
Bureau of Marine Fisheries & 
Ground\-fish      & 
Fish \& invert. no., wt., length; temp.,sal., DO (TSO), sea state, weather &
\~{}200 &
5 (\~{}bi-monthly)&
1988 &  
90 ft. isobath \\ \hline
Bureau of Shellfisheries &
Surf Clam &
Surf clam vol., no., length; (TSO) &
\~{}350 &
1 (summer) & 
1988 & 
3 n.m. \\ \hline
Bureau of Marine Water Monitoring &
Bacteria &           
Conc. of fecal coliform bacteria &
\~{}200 &
5-8 &
1976 &
3 n.m. (state waters) \\ \hline
Bureau of Marine Water Monitoring &
Nutrients &
Conc. of total suspended solids, NH3, NO3, NO2, PO4, total N, TSO &
\~{}140 &
4 (quarterly) &
1989 &
3 n.m. (state waters) \\ \hline
Bureau of Marine Water Monitoring &
Phyto\-plankton &
Phyto. Species, cell conc., TSO &
5 &
1, summer &
1985 &
3 n.m. (state waters) \\ \hline
  \end{tabular}
\end{center}



%%% Local Variables: 
%%% mode: latex
%%% TeX-master: t
%%% End: 
