\documentclass{report}
\usepackage{hyperlatex}
\W\usepackage{sequential}
\usepackage{fancyhdr}
\usepackage{longtable}

%% $Id$

\renewcommand{\rmdefault}{ptm}
\W\newcommand{\vspace}[1]{}
\newcommand{\pause}{\vspace{10pt}}
\T\setlength{\parindent}{0pt}
\T\setlength{\parskip}{\medskipamount}
\T\setcounter{tocdepth}{1}

\W\htmlcss{http://www.unidata.ucar.edu/packages/dods/css/quickstyle.css}
\W\htmldirectory{synth}
\W\htmltitle{DODS NOPP}

\T\setlength{\textheight}{610pt}

\T\pagestyle{fancy}
%\T\setlength\headheight{25pt}
\T\fancyfoot[CE,CO]{}
\T\fancyhead[RO,RE]{Page \thepage}
\T\fancyhead[LO,LE]{\textbf{DODS/NOPP Regional Workshop Synthesis Report}}

\newcommand{\tblhd}[1]{\textbf{#1}}
\newcommand{\synthURL}{http://www.unidata.\-ucar.edu/\-packages/\-dods/\-reports/\-nopp-synth-html/\-synthesis.html}
\newcommand{\synthPS}{../synthesis.ps.gz}
\newcommand{\synthPDF}{../synthesis.pdf}
\W\renewcommand{\HlxIcons}[1]{http://www.unidata.ucar.edu/packages/dods/icons/}

\begin{ifhtml}
\newcommand{\toppanel}{\EmptyP{\HlxSeqPrevUrl\HlxUpUrl\HlxSeqNextUrl}{%
  \xml*{table width="100%" cellpadding=0 cellspacing=2}\xml{tr}
  \xml*{td bgcolor="##99ccff"}%
  \EmptyP{\HlxSeqNextUrl}
  {\xlink{\HlxCallEval{\HlxImage}{\HlxPure{\HlxSeqNextTitle}}{next.gif}}%
    {\HlxSeqNextUrl}}%
  {\htmlimage[alt=""]{\HlxIcons{}blank.gif}}%
  \xml*{/td}%
  \xml*{td bgcolor="##99ccff"}%
  \EmptyP{\HlxUpUrl}
  {\xlink{\HlxCallEval{\HlxImage}{\HlxPure{\HlxUpTitle}}{up.gif}}%
    {\HlxUpUrl}}%
  {\htmlimage[alt=""]{\HlxIcons{}blank.gif}}%
  \xml*{/td}%
  \xml*{td bgcolor="##99ccff"}%
  \EmptyP{\HlxSeqPrevUrl}
  {\xlink{\HlxCallEval{\HlxImage}{\HlxPure{\HlxSeqPrevTitle}}{previous.gif}}%
    {\HlxSeqPrevUrl}}%
  {\htmlimage[alt=""]{\HlxIcons{}blank.gif}}%
  \xml*{/td}%
  \xml*{td align="center" bgcolor="##99ccff" width="100%"}%
    \textbf{\chaptertitle}\xml{br}\HlxThisTitle%
  \xml*{/td}%
  \HlxPanelFields
  \xml*{/tr}%
  \xml*{/table}}}
\end{ifhtml}
\newcommand{\chaptertitle}{NOPP Regional Workshop Synthesis Report}

\begin{document}

\title{\textbf{Regional Workshop Synthesis Report}\\
\vspace{15pt}National Ocean Partnership Program,\\
Virtual Ocean Data Hub (VODHub)\\
\vspace{15pt}Distributed Oceanographic Data System (DODS)} 
%\author{Prepared for \\\vspace{10pt}P.I. Dr. Peter Cornillon \\
%Prepared by:\\ Technology Planning \& Management Corporation}

\maketitle

\texorhtml{}{\begin{center}Printable versions of this document are available. 
  \xlink{PostScript}{\synthPS} or \xlink{PDF}{\synthPDF}.\end{center}}

\tableofcontents

\chapter{Report Objective}

To provide a synthesis of the NOPP/DODS regional workshop results.
The contents of this report include:

\begin{enumerate}
\item A brief background on the NOPP/DODS project, its goals and
  objectives, and information about the regional workshops.
\item A response summary of the five basic issues presented at the
  Gulf Coast, Southeast, Northeast, and West Coast NOPP/DODS regional
  workshops.
\item A summary of the NOPP/DODS-Environmental Systems Research
  Institute (ESRI) workshop.
\item Other ideas, suggestions, and recommendations made by the
  regional workshops. 
\texorhtml{}{\item Appendices containing the final reports from each of the
  regional NOPP/DODS workshops.}
\end{enumerate}

\texorhtml{The online version of this report includes appendices
  containing the final reports from each of the regional NOPP/DODS
  workshops.  (See \texttt{\synthURL}.)}{}

\chapter{DODS Background}

The DODS project was begun in the early 1990's.  An outgrowth of the
development of two earlier oceanographic data systems, XBrowse and
JGOFS, DODS sought to take advantage of the internet to improve access
to oceanographic data.  At a meeting in 1992, participants discussed
the development of a new data distribution system that would
incorporate the best features of both of these earlier models, as well
as additional features suggested by their shortcomings.  This first
workshop had forty attendees and was held at the University of Rhode
Island.  The vision that emerged from that and subsequent meetings was
of a distributed system that would encourage individual scientists to
be their own data providers, alongside the traditional centralized
data archives, such as the federal data centers.

Since that time, the DODS system has developed into a sophisticated
data transport mechanism, as well as several specialized program
interfaces that allow unmediated access to scientific data from a
variety of user applications, ranging from simple web browsers (e.g.
Netscape, Internet Explorer) to advanced data analysis packages (e.g.
Matlab, IDL, Ferret).  Data made available through DODS ranges from
large collections of simple satellite data files to complex relational
databases.

In 2000, the National Ocean Partnership Program (NOPP) awarded DODS a
grant to develop a Virtual Ocean Data Hub system, to create a
structure through which ocean data can easily be made available to the
larger ocean sciences community.

Plans for the first year center around an effort to develop a
cooperative approach to creating the Hub system.  Four regional groups
were formed: Northeast, Southeast, Gulf Coast, and West Coast where
workshops have been held to discuss the DODS concept.  A fifth
workshop examined the overlapping development of ESRI's Geographic
Information System (GIS) technology and DODS.  The workshops were led
by regional coordinators who ran each of the workshops and who have
control on how money allocated to the regions will be spent in the out
years (second and third years).  Each of the regional coordinators
have an interest in oceanography but are from differing backgrounds,
with the idea of bringing as many diverse contributions as possible to
the overall effort.

A final synthesis report (this paper) is based on the reports
generated by each of the regional workshops.  This report is to be
provided as background to participants at a national workshop, to
be held in Washington, DC, April 25-27, 2001.  After the national
workshop, an overall report discussing recommendations for the DODS
project will be written.  A technical meeting will then be held to
address technical issues related to the implementation of
recommendations of the national report.

\chapter{DODS/NOPP Program Objectives}

The objective of the DODS/NOPP project is to plan and implement a
Virtual Ocean Hub (VODHub) system that will provide users with the
ability to easily access certain data types measured at specified
locations and times, regardless of data source, and without special
efforts or insights on the part of the user about the data source(s).

Specific research objectives identified in the BAA (Broad Agency
Announcement) are to:

\begin{enumerate}
\item Identify individuals and organizations who will take the lead to
  foster development of community-based data conventions for the
  available variety of data types;
\item Partner with local institutions (public and private) to improve
  access to coastal and regional data using these same community-based
  data conventions;
\item Enhance connections to existing national and international
  archives of ocean data as well as the program offices of major ocean
  programs (e.g., WOCE, JGOFS);
\item Partner with international groups to foster a world-wide ``ocean
  data dictionary'' to further commonality of access for all sources
  of ocean data; and
\item Work with national and international user standards groups
  (e.g., International Hydrographic Organization and their Electronic
  Navigation Charts) to foster access to ocean data.
\end{enumerate}

\chapter{The DODS Regional Workshops}

Five regional workshops were held between October 2000 and January
2001.  Attendance at the workshops ranged in size from 15 to 40
participants.

\begin{center}
  \begin{tabular}[t]{|l|l|l|} \hline
    \tblhd{Workshop} & \tblhd{Location} & \tblhd{Workshop Coordinator}
    \\ \hline
    Gulf Coast (GC) & Stennis, MS & Worth Nowlin \\ \hline
    ESRI (ESRI) & Redlands, CA & Ted Habermann \\ \hline
    Southeast (SE) & Charleston, SC & Anne Ball \\ \hline
    Northeast (NE) & Greenwich, RI & Linda Mercer \\ \hline
    West Coast (WC) & Corvallis, OR & Mark Abbott \\ \hline
  \end{tabular}
\end{center}

Although the workshop agendas varied, each included:

\begin{enumerate}
\item An overview and demonstration of DODS presented by University of
  Rhode Island personnel; 
\item A presentation by workshop attendees of information on agency program
  activity and datasets that could potentially be served via DODS in
  the VODHub; 
\item A summary of the data interests and data holdings of the
  workshop attendees, presented by each attendee; 
\item A discussion pertaining to DODS functionality as it relates to the
  VODHub; and 
\item A discussion addressing regional data needs that might
  be addressed by the VODHub.
\end{enumerate}


\texorhtml{The appendices to this report, containing an account of the
  proceedings of each workshop are available in the online version.
  (See \texttt{\synthURL}.)}{
The appendices to this report contain an account of the proceedings of
each workshop.}

\chapter{DODS Basic Issues}

The following issues were presented to the regional workshop
participants for consideration:

\begin{enumerate}
\item Is the DODS data model adequate for datasets to be served in the
  region, and if not, what additions are required?  
\item What are the important interface issues for users in the region?
  This question addresses issues related to the basic functionality of
  the system ranging from data discovery to data use.  
\item What types of semantic metadata will be required and what
  semantic metadata standard(s) will be used?  The focus should be on
  search and use metadata. 
\item What datasets will be served via the system as part of this
  effort?  Seed the system?  What assistance will be needed?  
\item Is a central regional node required to provide coordination for
  the region including such services as user support and data
  location, as well as other currently unforeseen services?
\end{enumerate}

\section{Response to Issue \#1}

\emph{Is the DODS data model adequate for datasets to be served in the
region, and if not, what additions are required?}\pause

When asked whether the DODS data model was adequate, participants at
the five workshops saw no major flaws in the DODS data model.  Some
workshop participants suggested the DODS data model be modified to
handle maps, graphs, images stored in GIF or JPEG format, and
animation data types.

Some secondary data model issues also arose.  For example, it is easy
to use the DODS data model to transport satellite swath data, but most
users want these data in more typical projections that require further
processing.  The requirement of further processing could be considered
a data model issue.  

% It was also noted that model outputs generated by
% scientists could be served via DODS.  DODS us well suited to serve
% model output.  Although this is not a problem for the data model, it
% might be a problem to a catalog system that depended on space and time
% coordinates.  Neither of these are, technically, data model issues,
% but they might be considered to occupy a nearby niche.

\section{Response to Issue \#2}

\emph{What are the important interface issues for users in the region?  This
question addresses issues related to the basic functionality of the
system ranging from data discovery to data use.}\pause

Data location was consistently identified as important (one group said
``essential'') for optimal use of the VODHub.  Some of the regions
suggested that a regional web- based data catalog (with a template) be
developed.  Furthermore, the catalog should be searchable by keyword
and geography.  Some argued that the catalog should list any available
dataset, whether or not it is accessible by DODS.

Two other interface issues emerged from the workshops:

\begin{description}
\item[Previewing capability]  One region proposed the development of a
  web browser interface that would allow a user to preview data before
  retrieving data.  Functionality should include location/sub-setting,
  manipulation, display, and data download.  (This is, in fact,
  similar to the Live Access Server, whose development is underway.
  An operational LAS can be found at
  \xlink{http://ferret.wrc.noaa.gov/las}{\texttt{http://ferret.wrc.noaa.gov/las}}.)
  
\item[Data request] Most of the workshops agreed that writing URLs is
  difficult.  One workshop suggested that a graphical, web browser,
  interface be developed to assist in the building of URLs.  Another
  workshop suggested a URL builder designed for sophisticated and
  unsophisticated users both locally and globally.

\end{description}

\section{Response to Issue \#3}

\emph{What types of semantic metadata will be required and what semantic
metadata standard(s) will be used?  The focus should be on search and
use metadata.}\pause

``Search'' metadata is the information needed to select
a dataset to use (e.g. location and time of measurements), while
``use'' metadata is information needed to use a dataset (e.g.  missing
value flags, or units).  The two sets overlap, but they are distinct
notions.

All of the regions agreed that semantic metadata was vital to the DODS
effort.  Most of the workshop focus was on use metadata, with
relatively little attention paid to search metadata.  Workshop
attendees emphasized the importance of being able to find the data you
need, but the metadata requirements proposed had more to do with
making the data you've found easy to use.

There was not a great deal of discussion about how important it is
that use metadata be machine-readable---that is, supplied in a
consistent format a computer can parse.  

%However, there appeared to be
%wide agreement that the creation of effective catalog strategies was
%essential to a working system.  This, in turn, implies a commitment to
%machine-accessible discovery metadata somehow tied to the datasets.
%One basis for this conclusion was the mention of FGDC and GCMD at all
%of the workshops, as a means for collecting and disseminating discovery
%metadata.

Several lengthy discussions dealt with the varieties of use metadata
that make datasets valuable, though there was little talk about
interoperability with respect to this variety of metadata.  One
possible implication of this is that machine-level interoperability
was not seen as crucial by workshop participants.  If someone is going
to use data from two different datasets, perhaps they will inevitably
do so at the user level rather than at the machine level, if only to
maintain a degree of control over the process.

The importance of associating existing data set descriptions---in the
FGDC clearinghouse or the GCMD---to data sets was dealt with in at
least one of the workshops.  This led to a discussion of third party
metadata description in general and the value of rendering these
accessible via the data access protocol, even when the descriptions
and the data do not reside on the same server.

Topping the workshop lists of essential metadata types were:

\begin{description}
\item[Names and descriptions] What is meant by the abbreviation of the
  variable name?
\item[Units and scaling factors] What units are the data delivered in?
\item[Missing values] How are missing values flagged in the dataset?
\end{description}

Several participants commented that a useful metadata collection would
have a place for data quality indicators.  Suggestions for these took
many different forms, including simple flags to indicate accuracy and
precision (Gulf Coast workshop), to descriptions of quality control
procedures (West Coast), to quality ``flags'' (Southeast).  One
suggestion (from the Southeast workshop) was that if a web-based
description exists for a dataset, the URL for this description should
be in the dataset's data attribute structure (DAS).

Finally, there were several comments about metadata people wished they
could have for datasets they would like to use.  These suggestions
included:

\begin{itemize}
\item Methods;
\item Platforms;
\item Instrument type and details, such as calibration data;
\item Data originator (P.I.);
\item Funding source;
\item Palettes for display of imagery, such as maps and sections;
\item Publications based on this data; 
\item Comparisons with other data; and 
\item Processing algorithms.
\end{itemize}

The Southeast and Gulf Coast workshops were explicit in urging that
the DODS project accommodate the Z39.50 metadata standard, and do so
by working with the FGDC Clearinghouse.  They suggested that DODS
should be able to use metadata supplied by a third party, such as
FGDC, and should be able to do so automatically.  The Gulf Coast
workshop further recommended that data originators should make use of
commercial software packages to assist them in preparing metadata in
FGDC format.

One suggestion (from the Southeast workshop) was that support should
be provided to those interested in proper descriptions of their
datasets.This support could take the form of:

\begin{enumerate}
\item A white paper that would be made available via either the
national or a regional DODS website, and 
\item Training sessions offered either at a central site or at
individual institutions.
\end{enumerate}




\section{Response to Issue \#4}

\emph{What datasets will be served via the system as part of this effort?
Seed the system?  What assistance will be needed?}\pause

Many of the regional agencies and participating scientists have
datasets stored in a variety of commercial DBMS software packages.
Workshop participants recommended that DODS servers be developed to
handle Microsoft EXCEL and ACCESS files, SAS, S+, and ArcINFO GIS
formatted data.

The workshop attendees had varying levels of expertise when it came to
installing servers.  It was noted that money would come from the
project to support server installation.  In the proposal, each region
would either:

\begin{itemize}
\item Hire an individual to use the money to help install servers, or
\item Let a sub-contract for people to install their own servers.
\end{itemize}

Ultimately, it is up to each region to decide how they want to proceed
and perhaps identify a few high priority sites for server
installation.

It was noted that the DODS project bears a great deal of
responsibility for the success of this phase.  in particular, the
workshops pointed out that: 

\begin{itemize}
\item Servers must be easy to install;
\item A regional help desk should be provided for answering questions
  on installing servers and developing clients; and
\item High quality, easy to understand documentation should be made
  available.
\end{itemize}

Some of the workshops suggested that funds would be well spent to
provide even more intensive help to people who want to install DODS
servers.

Each workshop spent some time detailing the datasets attendees knew
about that could be served by DODS.  These lists are available in the
respective appendices.

It was noted in the Northeast workshop that there are many good
datasets developed by various government resource management agencies
(who were not represented at the workshop).  These data are presumed
to be small datasets housed on a variety of platforms and stored with
many different kinds of software.  Making these datasets DODS
accessible will be a large task.

\section{Response to Issue \#5}

\emph{Is a central regional node required to provide coordination for the
region including such services as user support, data location, etc.?}\pause

The requirements for and function of a regional ``node'' varied from
workshop to workshop, but the following themes did emerge:

\begin{itemize}
\item Providing user support such as ``help desks,'' and recording
  problems users have experienced using DODS and installing servers;
\item Developing web-based data catalogs containing regional datasets,
  model output, and products participants are willing to serve;
\item Creating pilot projects to demonstrate the use of DODS for
  current regional players and for other organizations and researchers
  not in attendance at the workshops;
\item Establishing ad hoc working groups and/or possible local nodes
  to tackle regional cooperative pilot studies, evaluating available
  datasets (including satellite data) and serving real-time data;
\item Developing functionality to address dataset certification, data
  archaeology and rescue, long-term archiving;
\item Defining data rules to make computer models and output
  accessible (e.g., ENSO vs. non-ENSO model runs); and
\item Creating a regional website and/or listserv for highlighting
  issues of regional importance and to facilitate greater
  communication among regional players.
\end{itemize}


\section{Other ideas, suggestions, and recommendations}

Several other issues were identified by the regional workshops.  These
are outlined in the sections that follow.

\subsection{Archiving Issues}

Two workshops expressed the need for a committment to an archive for
data and product sets considered to be of long-term value.  Many of
these datasets are compiled by individual researchers and on their
retirement or death may no longer be accessible.

\subsection{Security Issues}

Several security issues were identified.

\begin{description}
\item[Data Security] Several attendees in the Southeast and Gulf Coast
  workshops expressed the desire to control access to data served by
  DODS.  It was pointed out that being able to restrict access to
  particular datasets is an important feature.  
  
  A few participants in the Southeast workshop discussed the need to
  be able to display a disclaimer to users of DODS data.
  This is apparently a requirement of some prospective data
  providers. 
  
\item[System Security] It is vital for the DODS project to be able to
  assure data providers that there are no security ``leaks'' in the
  DODS software.  It was suggested that Perl software isn't as secure
  as compiled software (C++ was specifically mentioned).  It was also
  suggested that the DODS documentation or web site contain a section
  describing known httpd security issues.
  
\item[Data Integrity] Some users expressed concern that data served by
  DODS could be modified by users.  DODS is a read-only protocol,
  however, so this is not possible.

\end{description}

\subsection{Inclusion of others}

Two of the workshops especially sought to look for other groups that
would be interested in making use of DODS and the VODHub.  In the
Northeast, the various government resource management agencies were
identified, and in the Gulf Coast workshop, attendees mentioned that
industry scientists (e.g. oil and gas exploration) could potentially
be interested.  The Gulf Coast workshop also suggested inviting
representatives from Canada, Mexico, and Cuba to use regional data
centers and potentially contribute datasets.

\subsection{Suggestions for Years 2 and 3}

Funding in the out years was essentially left up to each region to
decide where money was to be spent.  Some suggestions included:

\begin{itemize}
\item Web browser and direct download for clients;
\item Establish mailing lists for workshop participants; 
\item Create a web client focus group; 
\item Develop regional pilot projects;
\item Coordinate efforts with FGDC;
\item Explore metadata to DODS relationships; and
\item Hire outside consultants and/or programmers to:
  \begin{itemize}
  \item Build websites with information pertinent to DODS users;
  \item Perform demonstrations on the use of DODS;
  \item Provide help desk support; and
  \item Develop software for web client(s) and DODS URL builders.
  \end{itemize}
\end{itemize}


\chapter{Summary of the DODS-ESRI Workshop}

The purpose of the DODS-ESRI workshop was different than the regional
workshops.  This meeting was to facilitate the creation of data
exchange technology that would allow data to be shared between
Geographic Information System (GIS) systems and scientific users.  The
meeting's purpose was to:

\begin{itemize}
\item Examine the overlapping development between ESRI and DODS;
\item Help ESRI understand ocean scientist user requirements;
\item Assist DODS designers and developers to understand current GIS
  technology; and
\item Lay the groundwork for year two and three of the NOPP/DODS project.
\end{itemize}

The summary report for this workshop is included in
appendix~\ref{app,esri} of this report. 

Some of the priority questions for the DODS-ESRI integration include:

\begin{itemize}
\item How to provide DODS data access to GIS users?
\item How to provide GIS access to DODS users?
\end{itemize}

Next steps identified:

\begin{itemize}
\item Build a DODS-ArcIMS connector.  This will provide DODS clients
  access to GIS data.  It may be the starting point for building
  client software for ESRI users to access DODS data.
\item Begin working with the ArcIMS Software Development Kit (SDK).
  Although the technology is still in development, it has the
  functionality to provide GIS users access to DODS servers.
\item Schedule an additional meeting at the ESRI National user
  conference.
\end{itemize}

Unresolved Issues:
\begin{itemize}
\item Data discovery mechanism is not clear.
\item Understanding and accessing GIS metadata requirements for a DODS
  client.
\end{itemize}

\begin{comment}
\texorhtml{}{
\chapter{Appendices}

The appendices to this report are linked below:

\begin{tabular}[t]{ll}
Appendix I: &  \xlink{Gulf Coast}{appI.txt} \\
Appendix II: &  \xlink{ESRI}{appII.txt} \\
Appendix III: &  \xlink{Southeast}{appIII.txt} \\
Appendix IV: &  \xlink{Northeast}{appIV.txt} \\
Appendix V: &  \xlink{West Coast}{appV.txt} \\
\end{tabular}
}
\end{comment}

\T\newcommand{\HlxEval}[1]{}
\newcommand{\sectionref}[1]{section~\ref{#1}}
\newcommand{\tableref}[1]{Table~\ref{#1}}
\HlxEval{
(put 'appendix       'hyperlatex 'hyperlatex-ts-format-appendix)

(defun hyperlatex-ts-format-appendix ()
  (progn
    (hyperlatex-setcounter "chapter" 0)
    (hyperlatex-define-macro "thechapter" 0
      (concat "\\Alph{chapter}") "")))
}
\newcommand{\Url}[1]{\xlink{#1}{#1}}
\W\renewcommand{\paragraph}[1]{\textbf{#1}  }
\texorhtml{\newenvironment{tablelist}{\begin{list}{}{%
      \setlength{\topsep}{0mm}%
      \setlength{\itemsep}{0mm}%
      \setlength{\leftmargin}{0mm}%
      \setlength{\rightmargin}{0mm}%
      \setlength{\itemindent}{0mm}%
      \setlength{\parskip}{0mm}%
      \setlength{\partopsep}{0mm}}}{\end{list}}
}{\newenvironment{tablelist}{\begin{itemize}}{\end{itemize}}}

\appendix

\renewcommand{\chaptertitle}{Gulf of Mexico Regional Workshop}
\chapter{\texorhtml{}{Appendix A }\chaptertitle}

% $Id$

\begin{center}31 October - 2 November 2000\\
Stennis, MS
\end{center}

\section{Introductions and Background}

The workshop was called to order at 0910 on 31 October 2000, by W.
Nowlin. Provisional agendas and participant lists were distributed.
The agenda is given in \sectionref{I,agenda}. Each attendee gave a
self-introduction. Attendees and affiliations are given in
\sectionref{I,attendees}.

Nowlin gave an overview of the Virtual Ocean Data System Hub project
funded by the National Oceanographic Partnership Program (NOPP). This
presentation, given in \sectionref{I,presentation}, included:

\begin{enumerate}
\item The overall goal of the project and specific research objectives
  identified in the NOPP Broad Area Announcement to which the project
  was a response;  
\item The organizational structure of the project;
\item The phases of the program;
\item Issues to be considered at regional workshops;
\item The objective, approach, timing, and participants for the
  national workshop; and 
\item An outline of the implementation expected to emerge and guide
  the out year proposal.  
\end{enumerate}

\section{The Distributed Oceanographic Data System}

D. Holloway (University of Rhode Island) gave a presentation
describing the Distributed Oceanographic Data System (DODS), including
real time demonstrations of DODS capabilities using the Internet. This
presentation is summarized in \sectionref{I,DODS}. Included were
discussions of

\begin{itemize}
\item  What is DODS?
\item  Interoperability and metadata
\item         What is required to serve data via DODS-enabled servers?
\item         What is required to locate and access data?
\item         DODS and web browsers
\item         DODS-aware applications (MATLAB, Excel, IDL,..)
\item         What types of data can/cannot be served?
\item         Future of DODS applications/servers
\end{itemize}

This session continued through the morning of the first day with many
questions and comments.

Several general points of agreement emerged from this session. DODS is
principally a bottom up data system allowing the user easy data access
from within his application provided the data are on a DODS server and
the user has knowledge of the URL for the data. DODS does not provide
a data catalog; that will be a necessary part of the regional and
national data systems. Not all users have need for DODS. As examples,
modelers do not normally work in a DODS-enabled application, and many
analysts prefer to examine the data and metadata carefully prior to
its use in an application. However, DODS can be a very useful element
of the required regional and national data models. DODS service of
data is to be encouraged, but should not be mandatory.

Another key point of agreement is the necessity for data originators
to freely share data if they are to be part of an observing system,
regional, national or international. The tendency of oceanographers to
treat data as proprietary should cease.

A third point on which there is general agreement is the need for
adequate metadata to accompany the data sets. It is particularly
important to have adequate use metadata, and to maintain quality flags
as are available. This point was revisited later in the workshop.

\section{Presentations Regarding Regional Data and Products}

Representatives of each participating regional institution were offered time to make a 
presentation regarding data holdings, ongoing and planned observational programs, and models. 
It was suggested that each presenter include indications of data sets that they would be willing to 
share, whether they could be served via a DODS server, and data sets they would like to receive. 
Presentations continued during the afternoon of day one and the morning of the second day. 
Considerable discussion ensued as part of the presentations, which are
summarized in \sectionref{I,regions}.

Presentations were made by:
\begin{itemize}
\item John Blaha (NAVOCEANO) speaking on the Northern Gulf Littoral
  Initiative;
\item John Lever (NAVOCEANO) speaking on the willingness of NAVOCEANO
  to provide 24 hr by 7 d server capability for U.S. ocean data;
\item J. J. O'Brien (FSU, COAPS) speaking on SEVEER, SeaWinds from QuickScat, and the 
NCOM under development for the Gulf of Mexico;
\item Nan Walker (LSU Coastal Studies Institute) speaking on real time and archived data from 
the Earth Scan Laboratory and CSI, BAYWATCH, and WAVCIS;
\item Frank Muller-Karger and Doug Myhre (USF) speaking on fields from SeaWifs and AVHRR 
satellite data now being served by USF
\item George Ioup (U. New Orleans) speaking on data sets from Lake Ponchartrain, Barataria 
Basin, and other littoral waters that could be made available
\item Jim Corbin (MSU Engineering Research Center) speaking on the Distributed Marine 
Environmental Forecasting System;
\item Worth Nowlin (TAMU) speaking on data holdings and ongoing projects of Texas A\&M 
University;
\item Norman Guinasso (TAMU, but representing the TX General Land Office) speaking on the 
Texas Automated Buoy Project;
\item Ruben Solis (Texas Water Development Board) speaking on hydrological and environmental 
data being monitored in Texas and on models of Texas estuaries run by
his agency
\item Patrick Michaud (TAMU Corpus Christi) speaking on the TCOON and on present and 
planned CODAR observations along the coastal Gulf;
\item Mark Luther (USF) speaking on the Coastal Ocean Monitoring and Prediction System on the 
Florida shelf, including the Tampa PORTS, and developments of the USF Center for 
Ocean Technology Development;
\item Robert Molinari (NOAA/AOML) speaking on archived ocean station, ship-of-opportunity, 
and drifter data that could be made available, the real time and delay mode data from 
the NOAA GOOS center, and the Florida Bay project;
\item Tony Amos (Univ. of Texas Marine Science Center) speaking on the many long-term, 
multidisciplinary data sets available from the Institute and of his real time data and 
tidal predictions;
\item Susan Starke (NOAA Coastal Data Development Center) speaking on plans of the center to 
facilitate access to ``coastal'' data, from the EEZ to 300-km inland. She gave a brief 
overview of Data Exchange Interfaces, a middleware under development for Navy and 
intended for use by the Center; and 
\item Richard Campanella (Tulane Univ.) speaking on the Long-term Ecosystem Assessment 
Group
\end{itemize}


\tableref{I,table1} \texorhtml{on \pageref{I,table1}}{} indicates the
various data sets offered by the presenters for sharing. The
descriptions are necessarily brief (perhaps schematic in some cases).
These data sets will be more fully described on a web site to be
maintained by TAMU for reference of all interested users of regional
data sets from the Gulf of Mexico.  Tentative commitments were made
toward the end of the workshop to which of these data sets would be
served initially, by what time and whether it seemed likely that they
would be offered via a DODS server.  That information also is
indicated in \tableref{I,table1} and will be available on the TAMU web
site.



\section{General Issues for DODS Regional Workshops}

The workshop was asked to consider five general issues. That was done.

\begin{enumerate}
\item Is the DODS data model adequate for data sets to be served in the region? What additions are 
required?
\item What are important interface issues for regional users? From data discovery to use?

Key to any successful data and product exchange system is the willingness of data originators to 
openly share data in a timely manner. Participants in the distributed data system for the Gulf of 
Mexico region must be willing to do so.

It was agreed that DODS is useful as one basis for data sharing in a distributed mode. Of course, 
it is very useful for users wishing to operate within DODS-enabled application software. 
However, it is not adequate, or intended, for all regional needs:
\begin{itemize}
  \item   Data location is essential (catalogs are needed);
  \item   Many clients use applications not supported by DODS;
  \item   Large archives now serving data need not necessarily change to DODS mode;
  \item   Distribution of data and products in real time to users (especially public) may be best done 
via other methods.
\end{itemize}

Encouragement of new data servers to use the DODS mode is very desirable and will enhance 
data utilization by many users. It was agreed that Gulf of Mexico regional data providers/users 
would serve data via DODS servers as feasible. For some applications this will not be the most 
expeditious or logical method. Open exchange of data and products via the Internet is the desired 
outcome. 

Preparation and maintenance of a catalog of information now on DODS servers, as well as on 
other servers of ocean data, in the region is one essential initial step. (See issue number 5 below.)

A concern expressed by workshop participants was the need for a continuing archive for data 
and product sets considered being of long-term value. Many such sets are compiled by individual 
researchers and on their retirement or death may no longer be accessible. NAVOCEANO is 
providing a server for non-classified ocean data. It will be available for national as well regional 
use. In cases where data sets are used frequently, and especially if they are large, they can reside 
at NAVOCEANO. 

The need to preserve real-time data streams as time series was stressed. Real-time data streams 
should be further quality controlled in delayed mode and aggregated to produce as complete 
series of quality data as possible. This is already the situation for many data sets (e.g., drifter, 
ship-of-opportunity, and Argo data).

DODS offers password protection. However, it was agreed that data confidentiality will be up to 
the data provider who serves the data. 

\item What types of semantic metadata will be required? (Focus of search and use metadata.) What 
standards will be used?

The workshop considered what metadata of this type will be required. It was agreed that at a 
minimum the following information must be included:
\begin{itemize}
\item Where do the data reside?
\item 4-D location information,
\item Definitions of parameters,
\item Units (including standards where appropriate),
\item Accuracy and precision (as available), and
\item Flags from originators and secondary reviewers.
\end{itemize}

Very desirable additional information that should be included includes:
\begin{itemize}
\item Methods,
\item Platform,
\item Instrument type, model, band,
\item Calibration data, and
\item Reference to algorithms used.
\end{itemize}

Other useful metadata that is recommended includes:
\begin{itemize}
\item Supporting descriptions,
\item Data originator and contact information,
\item Source of funding for data collection, and
\item Reports in which data are included.
\end{itemize}

In discussing desirable formats for metadata, consideration was given to the desirability of 
having some degree of uniformity and the fact that data served by federal servers must use a 
standard format for metadata. It was agreed to adopt for the region FGDC. The amount of 
metadata required by FGDC is actually less than agreed necessary by the workshop. Moreover, 
there appear to be software packages to assist the data originator in preparing metadata in FGDC 
format.

\item What data sets will be served initially as part of this effort in order to seed the system? What 
assistance is needed?

The representatives present were queried as to what data sets they would agree to serve initially, 
by when would the initial data sets be served, whether they would enable a DODS server, and 
whether they would require assistance to do so.

Shown in \sectionref{I,table1} are the institutes holding Gulf of
Mexico data and product sets they are willing to share. As indicated
some already are served via web sites. Those marked with the priority
``initial'' will be served first, by the dates shown and by DODS
server if so indicated.

This represents a significant commitment to the establishment of a Gulf of Mexico regional data 
system, the first step toward a model-based Gulf observing system.

\item Is a regional node needed for coordination, including data location, reference for user support, 
etc.?

A number of activities were recognized as necessary first steps in the implementation of a Gulf 
of Mexico regional ocean data system as part of a national system.

\begin{itemize}
\item    COAPS at FSU agreed to set up and maintain a list serve to enable ease of communication 
among regional participants. 

\item   NAVOCEANO will procure and maintain a server for distribution of non-classified ocean 
data, model output, and products via the Internet.

\item   TAMU will set up and maintain a web site indicating data, model output and products that 
participants are prepared to serve. It will include links to web sites where data are now 
being served. It will include a catalog of regional data on DODS-enabled servers.

\item  The formation of several ad hoc groups was agreed to.

  \begin{enumerate}
  \item A working group on bathymetry and coastlines with the
    objective of specifying the best currently available data sets for
    bathymetry and coastline locations for the Gulf of Mexico.
    Membership will include representatives from COAPS, TAMU (Wm.
    Bryant's group), NGDC, and NAVOCEANO. (Bryant to chair?)
    
  \item A working group to consider and recommend the best approach to
    serving all real time data now being collected for the Gulf by
    non-federal institutions. This method should allow a user access
    to the intersection of the data sets. If possible it should be
    accessible through a DODS server. Membership will include
    representatives of TCOON, TABS, COMPS, WAVCIS, Florida Bay,
    NAVOCEANO, BAYWATCH, and URI (Dan Holloway).  (Michaud to chair?)

    
  \item An ad hoc group to consider research needed to evaluate
    regional satellite and model- derived data fields produced by more
    than one organization, with a view to identifying characteristics
    and recommending specific fields for use. Such fields include SST,
    color, wind, and sea surface height. Membership was not
    determined, but O'Brien might be willing to chair.

  \end{enumerate}
\end{itemize}


\end{enumerate}


\section{Other General Considerations}

It was hoped that industry participants could be entrained into the system. As an example of the 
potential information that might be shared, Nowlin reviewed data sets and model output held by 
the oil and gas producers joint industry projects EJIP and CASE. This includes current meter 
time series, CODAR data, synoptic oceanographic survey data, rather fine resolution circulation 
model output, and atmospheric storm data and analyses.

It was agreed that Nowlin would send to the workshop participants a copy of the project budget 
as submitted to NOPP. Those budgets for years after the initial planning year were purposely 
vague. The reason for distribution is to allow regional participants a say in determining priorities 
for items to be included in out-year budget requests.

To increase basic data coverage in the Gulf, it was considered desirable that provisions be made 
for increasing ship-of-opportunity data from both commercial and research vessels regularly 
traversing this area. It was suggested that such vessels might be equipped with XBT launchers, 
ADCPs, and Improved Meteorological (IMET) packages.

It was generally agreed desirable to include representatives of Cuban and Mexican institutions in 
this Gulf of Mexico regional data center.


\begin{longtable}{|p{0.75in}|p{2.75in}|p{0.5in}|p{0.5in}|p{0.5in}|}
  \caption{Institutes holding Gulf of Mexico data sets they are
    willing to share. Some are now available via web site as
      indicated. Those marked as initial priority will be served
      first, by dates shown and by DODS server if so
      indicated.\label{I,table1}}
\\ \hline
\textbf{Institution} & \textbf{Data or Program} & \textbf{Priority} &
      \textbf{Date of Service} &    \textbf{DODS?} \\ \hline
\endfirsthead
\caption{Institutes holding Gulf of Mexico data}
\\ \hline
\textbf{Institution} & \textbf{Data or Program} & \textbf{Priority} &
      \textbf{Date of Service} &    \textbf{DODS?} \\ \hline
\endhead
\hline
\endfoot
NOAA/AOML &    
 \begin{tablelist}
 \item  Florida Bay project
 \item Data from NOAA GOOS Center, available now by web site
 \item Archived drifter and station  data, available on request
 \end{tablelist}
&
initial  &   3/01 & yes, willing, depends on \$ \\ \hline

FSU &
\begin{tablelist}
\item SeaWinds from QuickScat:       
               operational and research 
               products available via web 
               site;
             \item          6-hr winds at 0.5 degree 
               resolution from July 1999
             \item    NCOM for northern Gulf
             \end{tablelist}
&
initial
&
     02/01
&     yes
\\ \hline

LSU  & WAVCIS & & &                                            perhaps \\
&             Earth Scan Lab archived and 
               real time data:  & & & \\
&               AVHRR images for LATEX
                 period &        initial   &   03/01  &   yes \\
&               CSI archived data  & &  ?  & yes \\
&               BAYWATCH program  & & ? &  yes \\
&               LATEX B data   & &  ? &  yes \\ \hline

MSU 
&
\begin{tablelist}
\item Offer computational power   
\item DODS server
\item Investigate link to MEL for service
\end{tablelist}
&
   initial  & &  yes \\ \hline

NDBC & Buoy data, available now via web site & initial & & yes, if
support available \\ \hline

NAVO &  New server for U.S. data sets  & initial  & procure & yes \\
   &         \textbf{Selected archived data sets:} & & & \\
   &               Gridded global DBDBV bathymetry & initial   &   03/01 &    yes \\
   &       Gridded global T-S data  base  & initial   &   03/01 &    yes \\
&             \textbf{Selected operational products: } & & & \\
&               MODIS fields for Gulf & initial   &   03/01 &    yes \\
&               COAMP for Amer. Med. & initial   &   03/01 &    yes \\
&               Global SST  &                 initial  &      ?    &   ? \\
&             Selected 3-dimensional POM     
               output   & initial  &     03/01 &    yes \\ 
&             Selected data from Northern
               Gulf of Mexico Littoral 
               Initiative  & initial  &     03/01 &    yes \\ \hline
TAMU  &       LATEX A data  & initial  &     03/01 &    yes \\ 
&              Historical daily river 
               discharge from major U.S. 
               rivers & initial & 03/01 & yes \\
&             Analyzed wind fields for 
               LATEX period & second & & \\
&             LATEX C data  & second & & \\
&             NEGOM data & & & \\
&             Historical archives of MBT, 
               XBT, ocean station, drifter, 
               moored current meter, and 
               ADCP data  & second & & \\
&             MAMES I \& III; CHEMO I \& II, 
               available now on request 
               via FTP site & & & \\
&             NOAA Status and Trends, 
               available now on request 
               via FTP site  & & & \\
&             EPA EMAP data, available now 
               via web site  & & & \\
&             National Estuarine Programs 
               data from Galveston and 
               Corpus Christi Bays  & & & \\ \hline

TAMU Corpus Christi &  TCOON data and products,
      available now via web site & & & perhaps \\
&             CODAR observations   &          initial &     02-03/01 &
      yes \\ \hline

Tulane & & & & \\ \hline

Texas General 
Land Office    
&
Texas Automated Buoy System  data, real-time available 
                now via TAMU web site

& initial  &     03/01 &    yes \\ \hline

Texas Water Development Board
&
  TX estuarine hydro. surveys,   
    1987-97, available now via 
          web site 
& initial &     early 01  & yes \\ 
&             Sonde estuarine water quality 
               data, available  now via web 
               site 
& initial &     early 01  & yes \\ 
&             TX coastal hydrology records, 
               available now via web site & & & \\
&             Model output for TX estuaries & & & \\ \hline
University of Colorado  &  Daily fields of sea surface   
    height anomaly & initial & & yes \\
&             Model average sea surface height for GoM & initial & &
yes \\ \hline
               

University   
of New       
Orleans &     
Lake Pontchartrain project  &   initial  &     ?    & perhaps \\             
&Barataria Basin project   &     initial  &    05/01 &    perhaps \\
&Coastal Data, Northern GoM &    available     &     &    perhaps \\
&Acoustics Data, Northern GoM  & initial  &    08/01    & perhaps \\ \hline


University   
of South       
Florida        
&
Tampa PORTS data, available    in real time now via web site & & & \\
&             COMPS:          
               archived data             &  initial    &  03/01 &   yes \\
&             ECOHAB & & & \\
&             Archived satellite data 
               (AVHRR, Sea WIFS):      
               Reduced resolution AVHRR
                 since 1993            
&       initial   &   03/01  &   yes (password for SeaWIFS) \\
&             Current satellite data 
               streams (Sea WIFS, CZS, 
               AVHRR, MODIS) & & & \\ \hline

USM &          TBD & & & \\ \hline

University of Texas Marine Science Institute &
Archived ocean station data & & & \\
&     Long records of tidal height,  
         surface T \& S & initial   &   03/01  &   yes \\
&      Tidal predictions, available
      via web site &   &                03/01  &   yes \\
&     Records of effects of storms
               on sea level, available via 
               web site  & & &  no  \\
&             Marine mammal, turtle, and bird 
               stranding data & & & \\ \hline

             \end{longtable}


\section{Agenda}
\label{I,agenda}

\begin{center}
  Gulf of Mexico Regional Workshop on an Integrated Data System for Oceanography
31 October - 2 November 2000
Naval Oceanographic Office
Stennis, Mississippi
\end{center}

\subsection{31 October 2000}

\subsubsection{Opening and welcoming remarks}

\subsubsection{Objectives of Gulf Component of an Integrated Ocean Data System}

\begin{itemize}
\item    Enhance data set discovery, sharing, and access through the use of web-based DODS servers 
and DODS-enabled applications
\item     Increase the number of Gulf of Mexico data sets available in a readily usable format
                Historical
                Real time
\item     Identify suite of data types of special interest to Gulf workers
\item     Foster collaborations on Gulf of Mexico region specific issues
\end{itemize}

\subsubsection{DODS as a capability}

   Briefing and demonstration of DODS server capabilities by University of Rhode Island 
personnel
\begin{itemize}
\item What is DODS?
\item                 Interoperability and metadata
\item                 What is required to serve data?
\item                 What is required to locate and access data?
\item                 DODS and web browsers
\item                 DODS-aware applications (MATLAB, Excel, IDL,..)
\item                 What types of data can/cannot be served?
\item                 Future of DODS applications/servers
\end{itemize}

\subsubsection{Gulf Regional Activities---Ongoing and Planned Activities}

Presentations by participants representing institutions.
\begin{itemize}
\item     Naval Oceanographic Office
  \begin{itemize}
  \item Northern Gulf Littoral Initiative
  \item Web-based server initiative
  \end{itemize}
\item     Florida State University
  \begin{itemize}
  \item Special Quikscatt products for the Gulf
  \item High-resolution Gulf model using NCOM
  \end{itemize}
\item     Louisiana State University
  \begin{itemize}
  \item Earth Scan Laboratory remote sensing capabilities
  \item WAVCIS- Wave-current surge information system
  \item BAYWATCH: The Vermilion-Cote Blanche Bay Physical Measurements Program
  \end{itemize}
\item     Mississippi State University
  \begin{itemize}
  \item Distributed Marine-Environment Forecast System
  \end{itemize}
\item     Texas A\&M University
  \begin{itemize}
  \item MMS-sponsored activities and data archive
  \item Modeling activities
  \end{itemize}
\item     Texas General Land Office
  \begin{itemize}
  \item Texas Automated Buoy System
  \end{itemize}
\item     Texas Water Development Board
  \begin{itemize}
  \item Bays and Estuaries Program - data collection and dissemination
  \item Oil spill and other modeling activities
  \end{itemize}
\item     University of South Florida
  \begin{itemize}
  \item Remote sensing capabilities
  \item Tampa PORTS
  \item Coastal Ocean Measurement and Prediction System
  \end{itemize}
\item     AOML
\item     University of South Florida
\item     University of Alabama
\item     Eddy Joint Industry Project
\item     University of Texas, Marine Science Institute
\item     TAMU Corpus Christi
\item     Tulane
\item     University of Southern Mississippi
\item     University of New Orleans
\item     Minerals Management Service
\item     National Coastal Data Development Center
\end{itemize}

\subsection{1 November 2000}

\subsubsection{Gulf Regional Activities (Continued)}

\subsubsection{Regional Partnerships}
\begin{itemize}
\item     Motivation for becoming a partner
\item     NOPP-DODS support for regional partners
\item     Summary of datasets to be shared (which are of interest; which do you have to share?)
\item     General versus region-specific datasets
\item     Selection of data types of special regional interest (perhaps, fresh water distributions or 
measures of hypoxia) and what metadata are needed
\item     Interests in cooperative studies
\end{itemize}

\subsubsection{DODS Questions}
Having considered data sets that are available and might be served, this is an opportunity to 
return to specifics of DODS. Issues such as the following may need further consideration:

\begin{itemize}
\item Central/distributed servers (resources)
\item         Data quality control
\item         Metadata
\item         Recognition for data contributions
\item         Security
\item         Sub sampling and bandwidth for large datasets
\item         Catalog and data discovery
\item         Continued data availability
\end{itemize}

\subsection{2 November 2000}

Decisions regarding future activities

\begin{itemize}
\item     Assign DODS server installations. Some support is available via the NOPP project.
\item     Assign data sets to be served as practical demonstration of regional interest and participation
\item     Regional organization needed for future development of an integrated regional-national international data system for oceanography
\item     Needs for regional activities (e.g., web-based data information center giving pointers to data sets, or assistance with server installations)
\item     Select representatives to national workshop
\end{itemize}

\subsubsection{Open Discussion and Wrap up}

\subsubsection{Adjourn Meeting}

\section{Meeting Attendees and Affiliations}
\label{I,attendees}

\begin{center}
\begin{tabular}[t]{ll}
Tony Amos &          University of Texas Marine Science Institute \\
Landry Bernard &     NAVOCEANO \\
John Blaha &         NAVOCEANO \\
Jim Bonner &         TAMU-CC/TEES \\
Jim Braud &          NAVOCEANO \\
Richard Campanella & Tulane/Xavier Center for Bioenvironmental
Research \\
Jim Corbin &         MSU ERC/CCS \\
Steve Foster &       MSU ERC/IDSL \\
Jim Fritz &          TPMC \\
Mike Garcia &        SAIC/NDBC \\
Norman Guinasso &    GERG/Texas A\&M University \\
Martha Head &        NAVOCEANO \\
Dan Holloway &       University of Rhode Island \\
Matthew Howard &     Texas A\&M University \\
Stephan Howden &     University of Southern Mississippi \\
George Ioup &        University of New Orleans, Stennis \\
Peter Lessing &      NDBC \\
John Lever &         NAVOCEANO \\
Alexis Lugo-Fernandez & Minerals Management Service \\
Mark Luther &        University of South Florida \\
Melanie Magee &      Gulf of Mexico Program \\
Robert ``Buzz'' Martin & Texas General Land Office \\
Eugene Meier &       Gulf of Mexico Program \\
Patrick Michaud &    TAMU-CC/CBI \\
Bob Molinari &       AOML/NOAA \\
Steven Morey &       COAPS/Florida State University \\
Frank Muller-Karger & University of South Florida \\
Doug Myhre &         University of Soutn Florida \\
Worth Nowlin &       Texas A\&M University - NAVOCEANO \\
Jim O'Brien &        COAPS/Florida State University \\
George Rey &         LEAG/CBR \\
Reyna Sabina &       AOML/NOAA \\
Mitch Shank &        NAVOCEANO \\
Ruben Solis &        Texas Water Dev. Board \\
Susan Starke &       NCDDC/NOAA \\
Vembu Subramanian &  University of South Florida \\
Molly Sullivan &     Tulane University \\
Jack Tamul &         NAVOCEANO \\
William Teague &     NRL \\
Nan Walker &         Louisiana State University \\
Patti Walker &       DATASTAR/NDBC \\
\end{tabular}
\end{center}

\section{Introduction to NOPP-sponsored National Data Hub project by Worth 
Nowlin}
\label{I,presentation}

\subsection{Project Goal and Objectives}

Plan and implement a ``Virtual Ocean Data Hub'' (VODHub) system that will provide users with 
the ability to ``easily access certain data types in specified locations/times, regardless of data 
source, and without special efforts or insights on the part of the user about the data source(s)''.

Specific research objectives identified in the BAA are:

\begin{enumerate}
\item Identify individuals/organizations that will take the lead to foster development of 
community-based conventions for specific data types.
\item Partner with local institutions (public and private) to improve access to coastal and regional 
data via community-based conventions.
\item Enhance connections to existing national and international archives of ocean data as well as 
the program offices of major ocean programs (e.g., WOCE, JGOFS) via developed 
community based conventions.
\item Partner with international groups to foster a worldwide ``ocean data dictionary'' to further 
commonality of access for all sources of ocean data.
\item Work with national and international user standards groups (e.g., International Hydrographic 
Organization and their Electronic Navigation Charts) to foster access to ocean data via a 
growing number of user interfaces.
\end{enumerate}

\subsection{Phases}

\subsubsection{Year 1: The Planning Phase}

\begin{itemize}
\item     Regional Workshops (5)

\item     Synthesis of regional workshop results

\item     National Workshop

\item     Prepare Final Recommendations
\end{itemize}

\subsubsection{Years 2 \& 3: Implementation}

\subsection{Regional Workshops}

Asked to consider the following issues:

\begin{enumerate}
\item Is DODS data model adequate for datasets to be served in region?
  What additions are required?
  
\item What are important interface issues for regional users? From
  data discovery to use.
  
\item What types of semantic metadata will be required? What standards
  will be used? Focus on search and use metadata.
  
\item What datasets will be served initially as part of this effort?
  Seed the system? What assistance is needed?
  
\item Is a regional node needed for coordination? Data location,
  reference for user support, etc.
\end{enumerate}

\subsection{National Workshop}

\subsubsection{Background}
        Synthesize results of regional workshops. Study by Executive Committee.

\subsubsection{Objective}
        Develop the implementation plan for years 2 and beyond.

\subsubsection{Approach}
        Address same issues as at regional workshops, with synthesis as background.

\subsubsection{Timing}
        Nine months after project began

\subsubsection{Participants}
        Project management
        Regional coordinators
        Representative from federal agencies with holdings, e.g., NAVO, NODC
        Representative from national organizations of marine institutions, e.g., NAML, Sea Grant 
\subsubsection{Program}
        Representative from international organizations, e.g., Australian BMRC
        Representative from industry with significant data holdings
        Experts with data systems/networks
        Representative from other NOPP projects developing model/assimilation capabilities

\subsubsection{Review}
        Workshop recommendations to be reviewed by community


\subsubsection{Implementation Plan}

Three principal areas will be:

\begin{description}
\item[Population]
Population refers to addition of data to system. Focus on in situ data sets --- particularly regional.

\item[System Core]

Areas of expansion considered necessary are:
\begin{itemize}
\item      Development of GIS clients
\item     Work with virtual data sets
\item     Web interfaces
\item     Improve Data Access Protocol
\end{itemize}

\item[System Maintenance]
Including system documentation and user support
\end{description}

\section{Presentation by Dan Holloway - University of Rhode Island}
\label{I,DODS}

(\Url{http://po.gso.uri.edu/\~{}dan/dods-regional-workshops/dods-regional-workshops.html})

The URL listed above is a link to the presentation given by the DODS group at the Gulf of 
Mexico NOPP regional workshop.

Following is a summary of a number of important points discussed during the presentation.

\subsection{Introduction to DODS}

DODS is an open source software project designed to allow users to easily access and move 
data over the network.

The DODS project has two underlying principles it adheres to:

\begin{itemize}
\item Anyone willing to share data should be able to do so via DODS; e.g., the scientist, a state 
agency, a private company, or a federal data center.
\item Users should be able to use the application package with which they are most familiar to 
examine or analyze the data of interest.
\end{itemize}

How these principles are incorporated into the design are as follows:

\begin{itemize}
\item Data providers must not be required to store their data in any special format.
\item The data system must require a minimum set of metadata from the data provider.
\item The DODS core software must be easy to interface to existing applications so those scientists 
can use the packages they are most familiar with.
\item The data system must provide all the metadata that is required to effectively use the data in 
the client application.
\end{itemize}

The requirement to provide the minimal set of metadata required to use the data directly 
competes against the requirement to not burden the data provider with meeting specific metadata 
requirements.

The DODS project has taken a bottom-up approach toward solving the distributed data access 
problem. Rather than focus on the directory level, or data location aspect of the distributed data 
access problem, DODS has focused on the data-level interoperability. As part of that effort, the 
DODS project has delimited the metadata requirements for distributed data access into `use' and 
`search' metadata. To achieve data-level interoperability a definition of `use' metadata was 
formulated by segregating the data-level metadata requirements into syntactic and semantic `use' 
metadata. The DODS core software provides a strict syntactic 'use' metadata representation in 
the data transmission component of the software. This is required in order for software 
components on the client-side to be able to decipher the encoded binary data stream. The DODS 
core software supports but does not require additional semantic `use' metadata from the data 
provider. However, as the level of metadata increases, both `use' metadata and `search' metadata, 
the degree of interoperability that can be attained with remote datasets increases. It is important 
to note that all of the more advanced client applications using DODS to access remote data have 
specific semantic `use' metadata requirements, but none of these go so far as to require FGDC 
CSDGM metadata fields.

During the presentation two DODS client applications were demonstrated.

\subsubsection{NOAA's Live Access Server}

The Live Access Server (LAS) is a web-server application that provides an interface to data 
stored at NOAA's Pacific Marine Environmental Lab in Seattle, and a number of DODS served 
datasets located at NOAA's Climate Diagnostic Center in Boulder, and the International 
Research Institute at Columbia's Lamont-Doherty Earth Observatory.

This web-server uses a DODS-enabled client application, Ferret, to retrieve data from these 
remote sites and process it into one of the available representations defined in the interface. 
Ferret is a notable example of a scientific application whose functionality has been extended by 
enabling it to access remote datasets via DODS.

The strength of this interface is that it provides a standardized interface to a relatively large 
number of gridded datasets. Additionally, it can be easily customized for use at other sites.

The weakness of this approach is that it predefines the range of analysis possible on the remote 
datasets. The interface does permit the data to be easily downloaded, but the data must then be 
ingested into the scientist's application for further study.

\subsubsection{DODS Matlab GUI}

The Matlab GUI was developed as a testbed application to better understand the problems 
associated with building applications based on distributed data access. The primary result of this 
effort has been to better understand the semantic `use' metadata requirements for data-level 
interoperability. The Matlab GUI has limited, though strict semantic `use' metadata requirements.

The Matlab GUI provides an interface to two sets of oceanographic data, global datasets 
providing Sea Surface Temperature, Winds, etc., and local in-situ datasets for the Gulf of Maine 
GLOBEC project. These two sets present different problems to building applications using 
DODS.

The strength of this interface is that it provides direct access to the data in the scientist's analysis 
application. Once the GUI has selected and retrieved the remote data, the data is now located in 
the scientist's workspace for further analysis, or saving to local files; no additional steps are 
required.

The weakness of this interface is that it can require a high-level of effort to add new datasets to 
the interface. However, this constraint has led to better understanding of the various aspects of 
`use' level metadata requirements.

\subsection{DODS current development activities}

\begin{itemize}
\item  The DODS core software has been ported to Java.
\item A JDBC server has been built and is currently in beta test at OSU and MBARI.
\item A GrADS server has been built using the Java port by the COLA group. This server can 
provide access to GRIB data via DODS. Additionally, this server can be used to process data 
remotely, returning the results of those operations as DODS datasets.
\item There is a native Windows port of the DODS core, as well as a Cygwin port of the software 
for PCs.
\item There is a WMT-DODS gateway prototype in development at NASA's Goddard Space Flight 
Center.
\item The DODS Matlab GUI is being rewritten to facilitate adding new and n-dimensional 
datasets, and to allow the user to easily customize the interface for their use. 
\end{itemize}

\subsection{DODS future development activities}

\begin{itemize}
\item The project is actively pursuing migrating the core's data access protocol (DAP), to use XML 
as an encoding scheme.
\item DODS is working with NGDC and ESRI to make DODS data accessible to GIS clients, and 
GIS data accessible to DODS clients.
\item The project is investigating designs to support tertiary metadata servers, as well as client-side 
metadata support, to permit interoperability for datasets with limited associated metadata.
\item A web-crawler will be designed and implemented as part of the NOPP project to assist data 
location.
\end{itemize}

\subsection{Comments on DODS}

The DODS project has taken a bottom-up approach in proposing solutions to the distributed data 
access problem. At its current stage of development the project is focused on data-level 
interoperability in that problem domain. Goals of the project have been to permit scientists to 
easily serve, and access data over the Internet. To accomplish this the software must not require 
any reformatting of data by providers, and permit direct access to data within the scientist's 
existing analysis environment. To reach these goals the project has selected several client 
applications to demonstrate the concept, Matlab, idl, and NetCDF. The project also identified a 
number of commonly used data formats, as well as packages which support access to PI help 
datasets. Arguably, pre-built servers for every data format used in the science community are not 
currently provided by the DODS project, but as has been demonstrated by various groups writing 
DODS servers can be a realistic alternative.

At its current stage of development DODS does provide direct access to data within a number of 
scientific analysis environments. Unfortunately, given the lack of standardized interfaces for 
searching for data within the community at large, or within specific data repositories, 
formulating DODS URLs to access datasets can be difficult. One goal of an integrated data 
system should be to work toward standardized mechanisms for accessing distributed data, which 
includes both ease of location and use of those data.

\section{Regional Presentations}
\label{I,regions}

\subsection{Presentation by John Blaha -- Naval Oceanographic Office}

(Used overheads)
 
\subsubsection{Northern Gulf Littoral Initiative (NGLI)}

Purpose:
\begin{itemize}
\item Develop reliable, multidisciplinary models of the Mississippi Sound and adjoining rivers, 
bays, and shelf waters through the operation of a sustainable forecasting system.

\item Has worked with numerous groups in the past.

\item Develop skill toward shore from sea.

\item First year was a modeling effort, some RS and in-situ observations, tied into Navy Hub.

\item Slide showing map of region.

\item Bathymetry is the Navy's business.

\item Slide of intrusions map.

\item Nested palm 200 to 500 meter mesh.

\item Slide of model generation, banding effects of shelf area.

\item Slide of basin model.

\item Slide In-situ domain (measurement infrastructure).

\item Slide of sediment transport. 

\item Slides of drifter data.

\item Slides of ADCP locations.

\item Slide of Altimeter Calibration.

\item Slide of Local Geoid - SST data.

\end{itemize}

Question: On what DODS node will data appear?

\begin{itemize}
\item Outside of firewall on MEL Server.
\end{itemize}

\subsection{Presentation by John Lever - NAVOCEANO}

(Presentation on Power Point)

Possible venues of DODS

\begin{description}
\item[Slide] Objective Functional Architecture.

\item[Slide] NAVOceano Data Distribution.
\begin{itemize}
\item Web page showing available products.
\item Data warehouse.
\end{itemize}

\item[Slide] Web Server flow chart.
\begin{itemize}
\item Access to data warehouse.
\item NGLI link.
\end{itemize}

\item[Slide] Potential DWH Architecture WEB Proxy.

\item[Slide] Dissemination Architecture.
\begin{itemize}
\item NAVO LAN.
\item Firewall DMZ.
\item Internet
\end{itemize}

\item[Slide] Master Environmental Library  (MEL).
\begin{itemize}
\item MEL v3.0.
\end{itemize}


\item[Slide] MEL Expanded Flow Chart.

\item[Slide] MEL data discovery and delivery.

\item[Slide] DWH Mass Storage.
\begin{itemize}
\item Purpose is for service production, processing and selected archiving of NAVOCEANO data sets.
\end{itemize}

\item[Slide] FY01 Mass Storage Acquisition.
\begin{itemize}
\item       High-Volume Near-line Storage.
\item       On-line Storage.
\item Network Connectivity Upgrades to Data Servers.
\item         Network Attached Storage.
\end{itemize}

\item[Slide] High-Volume Near-line Storage.

\item[Slide] Online Storage.

Many other partners involved.

Metadata is FGDC compliant.  Metadata not necessarily tagged with data can be downloaded if 
desired.
\end{description}

\subsection{Presentation by Jim O'Brien -  COAPS, Florida State University}

(Presentation on PowerPoint)

Modeling the Gulf of Mexico with Satellite Winds

Several slides of example animations:
\begin{itemize}
\item     Showing Near-real-time Winds and Surface Pressures.
\item     \Url{http://www.coaps.fsu.edu/\~{}zierden/qscat} AND \Url{http://www.coaps.fsu.edu/\~{}zierden/qscat/gulf.shtml}
\item     Swath data are used, not gridded data or interpolated fields.
\end{itemize}

SeaWinds on QuckSCAT Satellite.

SEE  \Url{http://www.coaps.fsu.edu/cgi-bin/qscat/animations.cgi?request=listr\&region=smex}

Sea Winds Daily Coverage.

Examples of Sea Winds Overpass.

Two products available:
\begin{itemize}
\item     Research. SEE \Url{http://www.coaps.fsu.edu/scatterometry/Qscat/gridded.shtml}
\item     Operational. SEE \Url{http://manati.wwb.noaa.gov/quikscat/}
\end{itemize}

Research Quality Gridded Winds:
\begin{itemize}
\item     Global, six-hourly, gridded winds.
\item     Currently 1x1 degree.
\end{itemize}

Gulf of Mexico Modeling Goals with the NCOM:
\begin{itemize}
\item     Get a better understanding of meso-scale dynamics.
\item     Examine two-way ocean interaction between the continental shelf and the Gulf of Mexico basin.
\item     Model oceanic response to energetic episodic forcing in upper ocean wave stratification.
\item     Apply improved modeled physics to ocean nutrient distribution.
\item     Improve ocean prediction capabilities in Gulf using SeaWinds scatterometer data and 
TOPEX/Poseidon altimeter data.
\end{itemize}

Have a 6-min. model, currently running with a 3-min. model, eventually will have local model 
going down to 1 km. Eventually all of Gulf done in 1-km model.

Will deliver half by half degree, every 6 hours gridded winds for the Gulf of Mexico for this 
project using Seawinds satellite.
 
Lots of animation on web site.
Web site:     \Url{http://www.coaps.fsu.edu}

Goal is to have very good model runs.

\subsection{Presentation by Nan Walker - LSU}

(Used overheads)

Main data sources of the Coastal Studies Institute:

\begin{itemize}
\item Real time data.
\begin{itemize}
\item Earth Scan Lab satellite imagery.
\item WAVCIS wave-current data.
\item BAYWATCH physical measurements program.
\end{itemize}

\item Historic data archives.
\begin{itemize}
\item Estuarine time series.
\item LATEX inner shelf.
\item Physical measurement in hypoxia region.
\item Satellite imagery.
\end{itemize}
\end{itemize}

Earth Scan Laboratory of Coastal Studies Institute was started in 1988.

NOAA AVHRR:
\begin{itemize}
\item Detects suspended sediments, temp.
\item Spatial resolution is 1 square km.
\end{itemize}

Orbview:
\begin{itemize}
\item Used for detecting chlorophyll-a suspended, and other suspended sediments.
\end{itemize}

Projects: GOES-8GVAR
\begin{itemize}
\item Temp @ 4x4 square km. and water vapor @ 8x8 square km.
\item Has visible channel @ 1km.
\end{itemize}

Slide - Composite image.

Project involvement:
\begin{itemize}
\item EPA EMPACT project.
\item MODIS Terra and Aqua (will be getting X-Band soon).
\item RADAR- ERS-2, Radarsat (future).
\item IRS-P4 Ocean Color (future).
\end{itemize}

WAVCIS Project (headed by Greg Stone):
\begin{itemize}
\item Focus is on LA shelf in Mississippi Sound.
\item Real time, high resolution of waves and currents.
\item Wave-current surge information system for coastal LA.
\item Good for comparing wave data.
\item No conductivity or surface temp. data.
\item Using cellular/satellite communication.
\item Using acoustic Doppler tech.
\end{itemize}

Benefits associated with WAVCIS program:
\begin{itemize}
\item Directional waves measurement.
\item Current velocity.
\item Water level storm surge.
\item Wind speed and direction.
\item Using data with other groups.
\end{itemize}

Has a website where you can pick parameters and do analyses.

Fundamental research.

BAYWATCH Program (Funded by US Army Corps), Measures:
\begin{itemize}
\item Currents.
\item Water level.
\item Turbidity.
\item Salinity/Conductivity.
\item Wind speed/direction, air pressure/temperature.
\end{itemize}

Showed a GOES-8 ocean temperature animation loop of circulation in the Gulf of Mexico.

LSU is open to possibility of serving data via DODS.

\subsection{Presentation by Frank Muller-Karger and Doug Myhre - University of South Florida}

(Used PowerPoint file)

Remote Sensing Capabilities at USF

USF Datasets:
\begin{itemize}
\item Satellite.
  \begin{itemize}
  \item Remote sensing lab.
  \end{itemize}

\item Field.
  \begin{itemize}
  \item PORTS
  \item COMPS
  \item ECOHAB
  \item (+ numerous field programs)
  \end{itemize}

\item Satellite Sensors:
  \begin{itemize}
  \item Current AVHRR.
    \begin{itemize}
    \item NOAA-12
    \item NOAA-14
    \item NOAA-15
    \item NOAA-16
    \end{itemize}
  \item SeaWIFS
  \item TERRA/MODIS
  \end{itemize}

\item Historic (since '93):
  \begin{itemize}
  \item AVHRR
  \item SeaWIFS
  \item CZCS (1978-1986)
  \end{itemize}

\item Coverage:
  \begin{itemize}
  \item Land (southeastern US, Central America, northern South America, and Caribbean Islands).
  \item Ocean (Gulf of Mexico, Caribbean Sea, eastern tropical Pacific).
  \end{itemize}

\item Primary Data streams:
  \begin{itemize}
  \item Terascan system (L-Band)
    \begin{itemize}
    \item Orbview -II/SeaWIFS
    \item POES/AVHRR
    \end{itemize}

  \item Apogee Solutions (X-Band)
    \begin{itemize}
    \item Terra/MODIS
    \end{itemize}
  \end{itemize}


\item Volume of Data Collected:
  \begin{itemize}
  \item CZCS  \~{}20 GB
  \item AVHRR  690MB/day
    (3 satellites, 12 passes/day)
  \item SeaWIFS 150MB/day
    (1 pass/day)
  \item MODIS
    (4 passes/day)
  \end{itemize}

\item Current Products:
  See website, \Url{http://paria.marine.usf.edu}


\item Secondary data streams, X-Band:
  \begin{itemize}
  \item Long term
    \begin{itemize}
    \item ENVISAT (MERIS)
    \item ADEOS-II (GLI)
    \item NPOESS
    \end{itemize}
  \item (FMK is member of ENVISAT, ADEOS-II Science teams)
  \end{itemize}

\item Data Archived:
  \begin{itemize}
  \item AVHRR, all raw data and products
  \item SeaWIFS, all raw data and products
  \item MODIS, products
  \end{itemize}


\item Data Archives:
  \begin{itemize}
  \item Data Stored on 12'' Worm Optical Disk.
    AVHRR raw data.
  \item CD-ROM 600 disk jukebox (300gb Capacity)
    SeaWIFS Raw and LO data.
  \end{itemize}

\item Other data archives:
  \begin{itemize}
  \item DVD-R
  \item DLT Tape (future).
  \item MODIS Raw Data.
  \end{itemize}


\item Samples shown on overheads


\item USF requirements:
  \begin{itemize}
  \item Coastal shelf salinity/hydrography.
  \item Coastal shelf winds.
  \item Coastal shelf currents.
  \item Offshore currents for shelf-break force studies.
  \item River discharge and nutrient concentration.
  \item Yucatan Strait transport.
  \item Strait transport.
  \end{itemize}


\item Possible Cooperative Efforts:
  \begin{itemize}
  \item Integration of FL COMPS/TEXAS and other in-situ observing systems with real-time remote 
    sensing systems.
  \item Integration of above with regional modeling efforts.
  \end{itemize}


\item USF Plans:
  \begin{itemize}
  \item NOWCAST circulation model of West Florida Shelf.
  \item Integrated current, wind field, and real time satellite data product (dynamic vector overlays).
  \item Surface flux estimates.
  \item Collect data from future sensors.
  \end{itemize}

\item Worries:
  \begin{itemize}
  \item Nice to have DODS, but where does this take us in the long-term with data that each of us 
    depend on and that are not available via centralized legacy systems?
  \item DODS effort should focus on a few, select datasets considered critical for long-term archival.
  \end{itemize}
\end{itemize}


\subsection{Presentation by George Ioup - University of New Orleans}

(Spoke from notes)

Some surveys are site-specific and not very regular.  Data for some models are sparse.

\begin{enumerate}
\item UNO involved with Lake Pontchartrain Basin Project.
 
Making data public is the goal, on website at a minimum.

Some data include:
\begin{itemize}
\item Water turbidity.
\item Conductivity.
\item Temperature.
\item Chlorophyll A.
\item Orleans Parish rain chemistry.
\item Circulation models.
\item Fecal coliform bacteria.
\item Sediment chemistry.
\item Lake bathymetry.
\end{itemize}

Data can potentially go on DODS.  Lake Pontchartrain Atlas to be published.

Contacts:  
\begin{itemize}
\item Professor Alex McCorquodale, Civil and Environmental Engineering
\item Professors Mike Porrier and Bob Cashner, Biological Sciences
\item Professor Shea Penland, Geology and Geophysics
\end{itemize}

\item Barataria Basin Project

UNO, LSU, LSU-Ag, LUMCON, Tulane, Dillard

Environmental measurements in Barataria Basin related to carbon cycling in a coastal estuary.

Making data public is the goal, on website at a minimum.
Data can potentially go on DODS.

Contact:  Professor Ken Holladay, Mathematics

\item Louisiana and Northern Gulf Coast Coastal Measurements

Many measurements made with USGS, Corps of Engineers, National Marine Fisheries, 
Louisiana Department of Natural Resources, and other sponsorships.  Data already available on 
website.  Data can potentially go on DODS. 

Contacts:  Professors Shea Penland and Denise Reed, Geology and Geophysics
        
\item Northern Gulf of Mexico Acoustics Data

USM, NRL-Stennis, UNO

Making data public is goal, on website at a minimum.  Data can potentially go on DODS.  

\begin{itemize}
\item  Ambient noise measurements.
\item Marine mammal acoustic measurements.
\item Acoustic propagation, including propagation through fronts and eddys.
\end{itemize}

Contacts: \begin{itemize}

\item  Professors George and Juliette Ioup, Physics
\item Professor Grayson Rayborn, Physics and Astronomy, USM
\end{itemize}
\end{enumerate}

\subsection{Presentation by Jim Corbin - MSU ERC/SSC/IDSL}

\subsubsection{Distributed Marine Environment Forecast System - DMEFS}

The proposed DMEFS will be a research testbed for demonstrating the
integration of various technologies and components prior to DoD
operational use, and an ``open'' framework in which to operate
validated climate, weather, and ocean (CWO) models The focus is:


\begin{itemize}
\item To create an open framework for a distributed system for describing and predicting the 
marine environment that will accelerate the evolution of timely and accurate forecasting. 
\item To adapt distributed (scalable) computational technology into oceanic and meteorological 
predictions, especially on the regional and tactical scales
\item To shorten the model development time by expanding the
collaboration among the model 
developers, the software engineering community and the operational end-users. 
\item To provide a system with the look and feel of an operational entity in terms of infrastructure 
but with advanced computers, software technology and none of the constraints of an 
operational environment. To provide unique cross-cutting capabilities (i.e., software 
integration technologies and support resources) to DoD and yet let DoD simultaneously 
leverage existing internal expertise and investments, including legacy components (e.g., 
ocean models, atmospheric models, tools, etc.). 
\item To provide a framework that is extensible and designed for rapid prototyping, validation, and 
deployment of new models and tools and be operational over evolving heterogeneous 
platforms distributed over wide areas with web-based access of forecast-derived information.
\item Middle tier servers that form a distributed, shared, persistent, collaborative environment for 
model development, validation, coupling, deployment and operational use. Implemented as a 
multi-tier, object oriented system. The middle tier components act as proxies of services 
rendered by the back-end resources.
\end{itemize}

The Web Portal for DMEFS provides a seamless web access to remote resources through secure 
kerberized CORBA channels hiding complexity of the high performance, heterogeneous back-
end systems:
\begin{itemize}
\item Primary operational user is Naval Oceanographic Office
\item Primary research user is the Naval research Lab Stennis
\item Develop and/or run met/ocean models
\item Validate METOC models
\item Create METOC support products from model runs and other data
\end{itemize}

Provides a graphical problem-solving environment to do problem solving from workstation, 
desktop, or laptop. A core set of services will be provided by the DMEFS architecture to support 
the development of meta computing applications including:
\begin{itemize}
\item Resource Management, Discovery, and Access Control
\item Security and Access Control
\item Transaction Services
\item Communication Services
\item Event and Notification Services
\item METOC Digital Library and Data Services
\end{itemize}

Web Browser based front end that can:
\begin{itemize}
\item Couple models.
\item Develop models.
\item Visualize results.
\item Access external data. 
\item Deploy models.
\item Set schedules.
\item Provide operational execution and access operational data.
\end{itemize}

\subsection{Presentation by Worth Nowlin - Texas A\&M}

MMS-sponsored activities from which data are or will be available include:

\begin{itemize}
\item LATEX
\item Northeast Gulf of Mexico Chemical and Hydrographic Study
\item CHEMO I and II
\item MAMES I and III
\item Deep Gulf of Mexico Benthic Biology
\end{itemize}

Also available are archives of:
        Ocean station, XBT, MBT, and AXBT data, current measurements (moored, drifters and 
shipboard ADCP)

Ancillary data archives include:
\begin{itemize}
\item Daily river discharge from U.S. rivers
\item Surface observations via GTS
\item Analyzed surface wind fields.
\end{itemize}

Additional data sets include:
\begin{itemize}
\item EPA EMAP data focused on contaminants
\item NOAA Status and Trends data for 14 years
\item National Estuarine Product data from Galveston and Corpus
Christi Bays
\end{itemize}

TAMU agrees to set up a DODS server and initially serve data from LATEX A and daily river 
discharge files.

\subsection{Presentation by Norman Guinasso - representing the Texas General Land Office}


\subsubsection{Texas Automated Buoy System (TABS)}

\begin{itemize}
\item Initiated as an Interagency Research Contract with Texas A\&M University (TAMU) in 1994.
\item Work is carried out at GERG and Department of Oceanography, TAMU
\item The mission is prediction of oil spill movement along Texas
coast.
\item First buoys were installed in 1995.
\item Currently operate seven buoys with single point current meters and ocean surface 
temperature sensors.
\item Data sampled at 30 minute intervals and served in near real real-time on the Internet  
\item Fulfills operational need of Texas General Land office.
\end{itemize}

Operational Elements of TABS with web pages:
\begin{itemize}
\item Operation of buoys and data presentation at \Url{http://www.gerg.tamu.edu/tglo}.
\item Continuous assembly of meteorological data for modeling purposes
  at 

\Url{http://seawater.tamu.edu/tglo/}
\item Forecasts of ocean currents using POM and Spectral numerical
models of ocean currents at 
\Url{http://seawater.tamu.edu/tglo/}
\end{itemize}

How TABS data and predictions are used for tactical oil spill response 

\Url{http://resolute.gerg.tamu.edu/\~{}norman/TABS-DODS.htm}  (Slide 1)

TABS data was effectively used to reduce the cost of the response to Buffalo Barge 292 Oil Spill 
in 1996. Knowledge of ocean currents prevented a massive mobilization of resources along the 
northern Texas Coast off Sabine and allowed cleanup efforts to be directed further down the 
coast. 
\Url{http://resolute.gerg.tamu.edu/\~{}norman/TABS-DODS.htm}  (Slide 2)

Operate two kinds of buoys, anchored in place by heavy chain.
\begin{itemize}
\item TABS I , TABS II 
\Url{http://resolute.gerg.tamu.edu/\~{}norman/TABS-DODS.htm}  (Slide 6)

\item TABS II with ADCP and meteorological package

\Url{http://resolute.gerg.tamu.edu/\~{}norman/TABS-DODS.htm}  (Slide 7)
\end{itemize}

\subsubsection{TABS buoy data communications.}

TABS I or TABS II
\begin{itemize}
\item Motorola Integrated by cell phone. Cell phone service provided
over most of central GOM 
shelf by Petrocom 
\Url{http://resolute.gerg.tamu.edu/\~{}norman/TABS-DODS.htm}  (Slide 8)
\end{itemize}

TABS II 
\begin{itemize}
\item    Westinghouse 1000 Satellite Cell Phone at 4800 baud
\item Buoys make digital data calls using NorcomNetworks X.25 network
  connecting to computers at Texas A\&M University
\end{itemize}

TABS data

Available on web page as graphics and downloadable ASCII files. Historical data available as 
Web CGI enquiries on TABS database.  Historical enquiries produce graphics or downloadable 
files.

Cost of Data Transmission using Petrocom Cell Phone Network
\$1.09 per minute

\begin{center}
\begin{tabular}{|p{.75in}|p{.6in}|p{.6in}|p{.6in}|p{.6in}|p{.6in}|}  \hline
& \textbf{Bytes per reading} & \textbf{Reading Interval} & \textbf{Bytes per day} & \textbf{Bytes per month} & \textbf{Data cost per month} \\ \hline
Single point current meter &   149  &  30 minutes &  7,152  & 214,560 & \$196 \\ \hline
\end{tabular}
\end{center}


Cost of Data Transmission using NORCOM Satellite Network
\$185 per month for first 500,000 bytes,  \$0.38 per 1000 additional bytes

\begin{center}
\begin{tabular}{|p{.75in}|p{.6in}|p{.6in}|p{.6in}|p{.6in}|p{.6in}|}  \hline
& \textbf{Bytes per reading} & \textbf{Reading Interval} & \textbf{Bytes per day} & \textbf{Bytes per month} & \textbf{Data cost per month} \\ \hline
Single point current meter &   149  &  30 minutes &  7,152  & 214,560 & \$185 \\ \hline
ADCP &                        1100  &   Hourly    & 26,400  & 792,000 & \$296 \\ \hline
\end{tabular}
\end{center}


\subsubsection{TABS Future Developments}

Winter, 2000-2001

\begin{itemize}
\item Install TABS II buoy with ADCP at Flower Garden Banks National
  Marine Sanctuary (FGBNMS)

\item Install 2 TABS II buoys with ADCPs in Mississippi Sound as part
  of Northern Gulf Littoral Initiative (NGLI)

\item Install meteorological package on three TABS II buoys

\item Install SeaBird Micro-Seacat TS sensor on  FGBNMS buoy
\end{itemize}

Spring-Summer 2001

\begin{itemize}
\item Replace electromagnetic single point current meters with
  Aanderaa 2D Doppler Current meters Install and test Falmouth
  Scientific NXIC conductivity sensor on TABS buoy
\end{itemize}

2001-2002

\begin{itemize}
\item Install Bottom mounted packages that communicate through surface
  buoys, to measure nutrients, dissolved oxygen, light, and other
  parameters in near real time at two to four TABS sites along Texas
  Coast.
\end{itemize}

\subsection{Presentation by Ruben Solis - Texas Water Development Board}

Estuarine Hydrographic Surveys
 
(\xlink{http://hyper20.twdb.state.tx.us/data/bays\_estuaries/surveypage.html}{http://hyper20.twdb.state.tx.us/data/bays_estuaries/surveypage.html})

\begin{itemize}
\item Synoptic measurements collected in Texas bays from 1987 to 1997
\item Data includes physical (water level, velocity, flow) and quality (salinity, pH, DO, temperature) 
measurements
\item Data available through clickable map, in text and graphical format
\item Metadata included with data files
\end{itemize}

Coastal Hydrology (\xlink{http://hyper20.twdb.state.tx.us/data/bays\_estuaries/hydrologypage.html}{http://hyper20.twdb.state.tx.us/data/bays_estuaries/hydrologypage.html}): 
\begin{itemize}
\item Historical inflows (1940-1998, and being updated) for Texas bays
\item Monthly flows available in text format from web site
\item Rudimentary metadata available with data files
\end{itemize}

Sonde Data (\xlink{http://hyper20.twdb.state.tx.us/data/bays\_estuaries/sondpage.html}{http://hyper20.twdb.state.tx.us/data/bays_estuaries/sondpage.html}): 
\begin{itemize}
\item Continuous sonde measurements of salinity, pH, DO, temperature, and at some sites turbidity, 
water elevation
\item Two sondes/estuary typically deployed
\item Program initiated in 1986
\item Text data and graphs of data available on web site
\end{itemize}

Hydrodynamic and Oil Spill Modeling
 
(\xlink{http://hyper20.twdb.state.tx.us/data/bays\_estuaries/bhydpage.html}{http://hyper20.twdb.state.tx.us/data/bays_estuaries/bhydpage.html})
\begin{itemize}
\item Hydrodynamic models for Corpus Christi and Galveston Bays conducted daily
\item Animated output displaying currents and water elevations available on web
\item Short-term (1-day) forecasts displayed
\end{itemize}

\subsection{Presentation by Patrick Michaud - Conrad Blucher Institute for Surveying and Science; 
Texas A\&M University Corpus Christi}

\begin{itemize}
\item Texas Coastal Ocean Observation Network:
\begin{itemize}
\item Map of flags collecting data.
\item Zoomed on Corpus Christi.
\item Showed individual tidal stations.
\item Put in for water circulation and property boundaries. 
\item Collect mean tidal datum.
\end{itemize}

\item Entirely Web based.

\item Web Page of Port Aransas:
\begin{itemize}
\item List of station info and latest observations.
\item Water level.
\item Air temp.
\item Wind speed.
\item Etc.
\end{itemize}

\item Historical data are available.

\item Quality control done everyday.
\begin{itemize}
\item Must have continuous data stream.
\end{itemize}

\item Web page of graphs explained. 

\item Data query page, can get station they are interested in:
\begin{itemize}
\item Perform sophisticated data retrieval.
\item Can make adjustments (feet vs. meters).
\item View for mean sea level.
\item Compare two stations.
\end{itemize}


\item Mainly developed for internal use.

\item Can provide in raw data format.

\item Metadata sent back to requestor (cited numerous examples of using metadata). 

\item System does numerous checks and is completely automated.

\item Produces tidal datum.

\item Benchmark leveling, dates, equipment, etc.

\item Moving vector map demo.

\item Others have asked for tidal datum to be produced by their shop.

\item Tide gauge point source data is their forte.

\item Have potential to integrate into DODS.
\begin{itemize}
\item Need to produce capabilities to share data but would require significant investment.
\end{itemize}
\end{itemize}


\subsection{Presentation by Mark Luther - University of South Florida}

(Used overheads)

Coastal Ocean Monitoring and Prediction System (COMPS)

Gulf of Mexico Ocean Observing Systems - website

COMPS web site  \Url{http://comps.marine.usf.edu}:
\begin{itemize}
\item Map of real time observing sites on west coast of FL.
\item Owned and operated by USF or other agencies.
\item Four buoys btw. 20 - 50 isobaths, several coastal stations with water level, met, 
temp/salinity.
\end{itemize}

Can click on station to:
\begin{itemize}
\item Get photos.
\item Show metadata on station.
\item Show numerous parameter measurements by variable.
\end{itemize}

Have data for last 24 hrs in graphical format.
\begin{itemize}
\item Line of sight radio and/or GOES satellite telemetry.
\item Have a GOES downlink, it's free but slow.
\end{itemize}

Developed a custom data logger for ADCP, T/S, and met data, with GOES
telemetry and spread-spectrum radio.

Overheads of buoys:
\begin{itemize}
\item Have full suite of instruments - ADCP (downward-looking), wind
  speed/direction, air temp., humidity, barometic pressure,
  precipitation, incoming radiation (SW, LW), MicroCat T/S sensors.
\item Held by heavy chain and RR wheels.
\end{itemize}

Two types:
\begin{itemize}
\item Low-cost - telemeters met only via single GOES ID
\item Cadillac - telemeters ADCP and T/S data and met data on Dual GOES ID's.
\end{itemize}

Graphs from individual buoys during a storm.
\begin{itemize}
\item Various measurements from a meter below surface.
\end{itemize}

Began in 1997:
\begin{itemize}
\item Data archived in ASCII flat files.
\item Developing searchable database system
\end{itemize}

Tampa Bay PORTS, a sub-system of COMPS (\Url{http://ompl.marine.usf.edu/PORTS}.

Suite of Models based on Princeton model.
\begin{itemize}
\item Overhead showing models.
\item Oil spill trajectory modeling is also done.
\end{itemize}

Data is delivered from PORTS to harbor pilots via a Vessel Traffic Information System 
(\Url{http://www.rossdsc.com/ais.htm})

Overhead of Tropical Storm Josephine (Oct. 1996), color showing non-tidal component - 
simulated storm surge from numerical model of West Florida Shelf.
\begin{itemize}
\item Extensive flooding in Tampa Bay region from modest storm
\item See \Url{http://ocg6.marine.usf.edu/WeisbergSite/StormSurgePlot.ppt} for detailed storm surge 
study
\end{itemize}

Have some COMP sites in Cuban waters.
Developing coastal ARGO drifters/profilers

Text from flyer:

\subsubsection{Coastal Ocean Monitoring and Prediction System}

Florida is the United States' fourth most populous state, with 80\% of
the population living in a coastal county.  Several recent storms have
brought large, unpredicted flooding to Florida's west coast.  The
coastal sea level response to tropical and extra-tropical storms
results from wind forcing over the entire continental shelf.  Much of
the local response may be due to storm winds quite distant from the
local area of concern, a case in point being tropical storm Josephine,
a modest storm that nevertheless caused extensive flooding in the
Tampa Bay area.

The University of South Florida has implemented a real-time Coastal
Ocean Monitoring and Prediction System (COMPS) for West Florida. COMPS
provides additional data needed for a variety of management issues,
including more accurate predictions of coastal flooding by storm
surge, safety and efficiency of marine navigation, search and rescue
efforts, and fisheries management, as well as supporting basic
research programs.  COMP consists of an array of instrumentation both
along the coast and offshore, combined with numerical circulation
models, and builds upon existing in-situ measurements and modeling
programs funded by various state and federal agencies.  In addition,
COMPS links to the USF Remote Sensing Laboratory, which collects
real-time satellite imagery via its HRPT and X-Band receivers.  This
observing system fulfills all of the requirements of the Coastal
Module of the Global Ocean Observing System (CGOOS).  Data and model
products are disseminated in real-time to federal, state, and local
management officials, as well as the general public, via the Internet
(URL \Url{http://comps.marine.usf.edu}).  COMPS is designed to support
a variety of operational and research efforts, including storm surge
prediction, environmental protection, coastal erosion and sediment
transport, red tide research (ECOHAB - Ecology of Harmful Algal
Blooms), and hyperspectral satellite remote sensing of coastal ocean
dynamics (HYCODE).  A precedent for this system already exists in the
form of the Tampa Bay PORTS (Physical Oceanographic Real- Time System)
- itself a first for monitoring estuaries.

The majority of COMPS stations are fully operational, with additional
stations planned for the near future. An array of offshore buoys
measures current, temperature, salinity, and meteorological
parameters, with satellite telemetry of the data to shore. Additional
buoys have been deployed off Sarasota as part of the ECOHAB and HYCODE
efforts.  A network of coastal towers that are instrumented with water
level, temperature, salinity, meteorological, and bio- optical sensor
augments buoy observations. Many of these sites are operated in
collaboration with the United States Coast Guard, the Citrus County
Office of Emergency Management, and the Pasco County Office of
Disaster Preparedness. Additional instrumentation will be installed at
Boca Grande, Cedar Key, Keaton Beach, and near the mouth of the St.
Marks River, enhancing existing stations operated by partner agencies.

A numerical circulation model, based on the Princeton Ocean Model, has
been developed for the entire West Florida Shelf, with an offshore
boundary stretching from the Mississippi Delta to the Florida Keys.
This model has been successful in simulating past storm surge events
and will be coupled to the COMPS real-time data stream to be run in a
nowcast/forecast mode. Sea surface temperature and ocean color data
from the West Florida shelf routinely collected by our Remote Sensing
Laboratory can be combined with in situ data and model output to
provide a comprehensive analysis of oceanic conditions.

The COMPS data archival and distribution system will collate data
streams from the USF- operated sites with those from sites operated by
other agencies into a seamless web-based interface.  We have multiple
satellite downlinks (both DRGS and DOMSAT) for receiving GOES data
telemetry from remote sites.  We are collaborating with the NOAA
National Ocean Data Center, the NOAA Coastal Services Center, and the
National Ocean Service to develop a comprehensive data base management
system for the acquisition, archival, quality assurance, and
distribution of these data.

Collaborating Agencies: Florida Dept. of Environmental Protection, Florida Marine Research 
Institute, Florida Institute of Oceanography, Citrus County, Pasco County, United States 
Geological Survey, National Oceanic and Atmospheric Administration, United States Coast 
Guard, Office of Naval Research, Minerals Management Service, U.S. Environmental Protection 
Agency.

For more information contact:
\begin{description}
\item[Prof. Mark E. Luther]
Tel: 727-553-1528               
luther@marine.usf.edu           

\item[Prof. Robert H. Weisberg]
Tel: 727-553-1568
weisberg@marine.usf.edu
\end{description}

\subsection{Presentation by Bob Molinari - NOAA/AOML}

Global Ocean Observing System

NCEP
\begin{itemize}
\item Both real time and delayed mode data.
\end{itemize}

What's available:

Drifter Data:
\begin{itemize}
\item      Lots available in Gulf of Mexico.
\item Recent in nature.
\item 1996, '97
\item 1998 had a large deployment.
\item 1999 had good coverage in Caribbean.
\item Jan to May 2000.
\end{itemize}

Overhead of 1968 data.
Overhead of 1970 data.

How do they get data out?
\begin{itemize}
\item      Takes a month for delayed drifter data.
\item Can also get raw data.
\item Both historical and real-time.
\end{itemize}

Would like to put other data on-line:
\begin{itemize}
\item Some data in Yucatan. 
\item Has hurricane slice data.
\item Florida Bay data.
\end{itemize}

Data available from other sources as well.

Question as to whether AOML should be a DODS site.

The data to be served by AOML comes under the purview of the NOAA Global Ocean 
Observing System Center established at the laboratory. The objectives of the Center are to 
provide to NOAA and to other users the data needed to initialize weather and climate forecast 
models (in real-time) and to increase our understanding of the coupled climate system (in 
delayed-mode).  The GOOS Center manages:

\begin{itemize}

\item     The Global Drifter Center which provides surface current, SST and some meteorological 
data. These data are presently available at the AOML web page and we would be willing to 
serve these data as part of DODS.

\item    The US VOS XBT network which provides temperature profiles. Both historical and recent 
data are available at the AOML web page and we would be willing to serve these data as part 
of DODS.

\end{itemize}

We can also provide historical CTD data collected in the Gulf of
Mexico by AOML. These data are presently not available on the Web.
Finally we can serve the data resulting from NOAA's Florida Bay
project.  Both real- and delayed mode-data can be provided to DODS.

\subsection{Presentation by Tony Amos - University of Texas - Marine Science Institute (UTMSI)}

Using slides, Amos showed the setting of the UTMSI and some of the
projects that might provide data for a DODS site.

\begin{itemize}
  
\item UTMSI is located on the north end of Mustang Island, Texas, one
  of a series of barrier islands bordering the Gulf of Mexico on the
  ocean side and Corpus Christi Bay on the landward side.  UTMSI is a
  graduate research unit of the University of Texas' College of
  Natural Sciences with a faculty of 13 and 25 graduate students.  In
  addition, two six-week upper level summer courses are given each
  year.  The research interest is primarily in bays and estuaries, and
  mostly biological, with research projects in mariculture, harmful
  algal blooms, macro-algae and phytoplankton, fish endrochronology,
  early life history of fish, predator/prey relationships, benthic
  ecology, and physical oceanography.  The physical setting is ideal
  with the large range of marine environments nearby, ranging from the
  Gulf to the hyper-saline Laguna Madre.  There are several data sets
  that could be posted on a DODS server.
\item The R/V Longhorn is UTMSI's main research vessel.  This 105-ft R/V has made several 
hundred short cruises into the Gulf since 1972.  From the late 1970's onward, CTD stations 
have been made on most of these multi-purpose cruises.  The data would need standardizing 
as they have been made with evolving STD/CTD instruments.  For the past decade, Sea-Bird 
911/Plus CTDs have been used.  Also, meteorological and surface oceanographic data are 
routinely measured throughout every Longhorn cruise.
\item Current meter data from the continental shelf area in the Gulf, and from passes and bays are 
available
\item Continuous meteorological, tide and current data have been collected at the Pier Laboratory 
in the Aransas Pass Ship Channel (main entrance to Corpus Christi Bay from the Gulf)
\item Amos would make a twenty-two-year time series of surf zone sea temperature and salinity 
available.
\item Real time data is available on the UTMSI webs site (\Url{http://www.utmsi.utexas.edu}).  Click on 
Weather and Tides (demonstrated) to get current and previous day's sea, tides and weather 
data in graphic or spreadsheet form.
\item Overheads showing important recent storms: Hurricanes Allen, Gilbert and last year's Brett, 
Tropical Storm Josephine and Francis.  Data depicting real and predicted tides.  Could make 
tidal predictions for different Texas locations available for DODS.
\item Other data sets at UTMSI may be available.  These include time-series of dissolved oxygen 
in the Laguna, benthic data, and beach erosion information.
\end{itemize}

Amos also posed the question; would it be feasible to post Gulf of Mexico bathymetric and 
shoreline data on a DODS site.

\subsection{Presentation by Susan Starke - NOAA/National Coastal Data Development Center}

(PowerPoint Demo)

NCDDC Phase 0 Goals:
\begin{itemize}
\item Develop mission statement staffing plan.
\item Initiate Phase I planning process.
\item Implement initial proof-of -concept.
\item Establish facility at Stennis Space Center \& hold NCDDC dedication.
\end{itemize}

Mission Function:
\begin{itemize}
\item Provide archive and access for the long-term coastal data record.
\item Includes data cataloging/data mining, data access, data QC/Integration, archiving, and new 
product development.
\end{itemize}

Phase I Planning:
\begin{itemize}
\item Concept of operations (CON-OPS) by 10/00.
documents ``nuts and bolts'' of phase 0 operations.
\item Phase 0 to Phase 1 Transition Plan by 10/00.
\item Phase I requirements documentation by 12/00.
\item Includes feedback from phase 0 operations.
\item How to validate requirements.
\end{itemize}

\subsection{Presentation by Richard Campanella - Tulane-Xavier Center for Bioenvironmental 
Research - Long-term Ecosystem Assessment Group (LEAG)}


The Long-term Ecosystem Assessment Group (LEAG) was organized in 1999 by a consortium 
of academic/research institutions and NAVOCEANO. The goals of LEAG include:

\begin{itemize}

\item Establish a cooperative effort between the government (Naval Oceanographic Office), 
academia (CBR), and private industry (COTS Technology)
\item Use the Mississippi River and areas of the Gulf of Mexico as a natural laboratory to 
conduct research to evaluate the extent of ecological and economic impacts of nutrient 
over-enrichment in this region
\item Create a biotechnology corridor between Louisiana and
Mississippi for codevelopment of 
biosensors, Autonomous Underwater Vehicles (AUV's), models for major river and gulf 
systems, and deep sea monitoring and communication technologies
\end{itemize}

The Center for Bioenvironmental Research constitutes the lead academic entity of the Long 
Term Estuary Assessment Group (LEAG), a collaboration of government, academic and private 
organizations that is conducting research on river-ocean interactions and coastal oceanographic 
processes. LEAG's goal is to develop effective management strategies to deal with 
environmental issues related to rivers and their interactions with coastal margin and ocean 
ecosystems.  One issue of interest is the coastal eutrophication caused by excessive introduction 
of nutrients into an aquatic ecosystem, leading to increased algal production and increased 
availability of particulate organic carbon. The effects of eutrophication, hypoxic dead zones, 
harmful algal blooms, ad changing fisheries populations can have global economic and 
biological implications.

The partnership established with LEAG will utilize some of the best minds in the fields of 
biology, geology, engineering, chemistry and oceanography. It will enable research discoveries 
and new technological developments that would not be possible if the participating organizations 
were not working in unison. Through its Mississippi River Interdisciplinary Research (MiRIR) 
Program, the CBR is conducting scientific and cultural research and education programs on the 
river. The Naval Oceanographic Office is an accomplished oceanographic operations facility that 
is uniquely qualified to support data management, modeling and AUV development activities. 
COTS Technology specializes in the acoustics, robotics and optics of marine research, and will 
provide technical, financial and operational background for the development of AUV's for data 
collection.  Through the Office of Naval Research (ONR), LEAG is fostering what is anticipated 
to be the first living biosensor that will be deployed on a autonomous underwater vehicle.

Additional partners that have joined LEAG include the Louisiana Universities Marine 
Consortium (LUMCON); the University of Southern Mississippi; Woods Hole Oceanographic 
Institute (WHOI) in Massachusetts; and the U.S. Army Corps of Engineers' Waterways 
Experimentation Station (WES), Vicksburg, MS.

By utilizing the Mississippi River and the Gulf of Mexico as a natural laboratory, the LEAG 
program will improve the capacity of the United States to monitor the risk of exposure to 
defense related toxicants in other systems throughout the world.

LEAG is interested in seeing what it can offer to the DODS, and what it can utilize from the 
DODS.

%%% Local Variables: 
%%% mode: latex
%%% TeX-master: t
%%% End: 

\renewcommand{\chaptertitle}{West Coast Regional Workshop}
\chapter{\texorhtml{}{Appendix B }\chaptertitle}

%% $Id$
 
\begin{center}
January 17th and 18th, 2001\\
Oregon State University\\
Corvallis, OR
\end{center}

\section{Executive Summary}

The science presentations on research in the California Current
highlighted a few key themes.  These include:
\begin{itemize}
\item Processes vary on a veriety of time and space space scales
  (e.g., local vs. remote forcing).
\item Mesoscale variability is an important component of the
  California Current.
\item Eastern boundary currents are productive biologically, and
  coupling with physical forcing is a critical issue.  In a sense,
  these issues are almost dogma, but they do lead to some common
  requirements for any data system:
\item Preservation and stewardship of data $\rightarrow$ time series are
  critical.
\item Integration across variety of data types $\rightarrow$
  biology/physics/chemistry.
\item Access by non-experts $\rightarrow$ managers as well as researchers who
  are not specialists with every data set.
\end{itemize}

Discussions during the workshop led to strong support for the idea of
a West Coast regional node, in addition to the local nodes.  The
regional node would provide services that would be beyond the
capabilities of a local node or would be done more efficiently at a
regional site.  Possible functions include:
\begin{itemize}
\item Data set certification - mechanisms to provide some degree of
  assurance of data quality for decades.
\item Data archaeology and rescue - many datasets are lacking homes.
\item Long-term archiving - when projects end, investigators retire, etc.
\item Advanced visualization services - display tools such as those
  developed at NOAA/PMEL as part of the Live Access Server
  (http://ferret.wrc.pmel.noaa.gov/las).
\item Define computer scenarios and output that might be made
  accessible (e.g., ENSO vs. non-ENSO model runs).
\end{itemize}

Participants at the workshop discussed possible local nodes that might form a 
cooperative, pilot study for the West coast.  The local nodes include:

\begin{itemize}
\item Scripps Institution of Oceanography
\item University of California, Santa Barbara
\item Pacific Fisheries Environmental Group
\item Monterey Bay Aquarium Research Institute
\item Naval Postgraduate School
\item University of California, Santa Cruz
\item San Francisco State University
\item Oregon State University
\item Pacific Marine Environmental Laboratory
\item University of Washington
\end{itemize}

In parallel, there are projects, some of which span multiple
institutions: California Cooperative Oceanic Fisheries Investigations
(CalCOFI), SIO Data Zoo, Mineral Management Service (MMS), Northeast
Pacific GLOBEC, Pacific Northwest Coastal Ecosystem Regional Study
(PNCERS), Partnership for Interdisciplinary Studies of Coastal Oceans
(PISCO), National Oceanographic Partnership Program (NOPP, which
includes multiple programs), and Coastal Oceanography Program (CoOP,
including COAST and WEST).  The point was not to list all possible
projects and all possible institutions, but to develop a
``reasonable'' list for a possible pilot study.

Datasets (again, not meant to be comprehensive) include: CTD,
moorings, drifters, ADCP, CODAR, SeaSoar/Scanfish, surface
meteorology, satellite imagery, chlorophyll, zooplankton biomass, and
nutrients.

Participants discussed the minimum set of semantic metadata that would
be necessary to use ``minimally'' processed data.  That is, we do not
expect people to serve raw volts from a sensor.  These are:

\begin{itemize}
\item Time and location - this will require an ancillary file that
  describes the time/space axes (e.g., latitude/longitude convention,
  time convention, etc.).
\item Descriptive variable name - including something that would
  indicate the sensor type.
\item Units
\item Missing data value - e.g., -9999, `NaN', etc.
\item Scaling, including slope/offset.
\end{itemize}

This information would allow a researcher to read in a data file and
know enough about it to conduct useful analyses of it.  It was noted
that the COARDS convention as part of the NetCDF standard would help
with naming, etc.  

See
\Url{http://www.unidata.ucar.edu/packages/udunits/index.html} for more
details.  

Also see
\xlink{http://ferret.wrc.noaa.gov/noaa\_coop/coop\_cdf\_profile.html}{http://ferret.wrc.noaa.gov/noaa_coop/coop_cdf_profile.html}.

A minimum set of metadata for data set discovery (that is, to help
users find datasets of interest) could be built up from the NASA
Global Change Master Directory (GCMD) which has an extensive set of
catalog information, standard terms, etc.  The GCMD is also willing to
help users describe their data holdings (see
\Url{http://gcmd.gsfc.nasa.gov/}).

For data presentation, it is often useful to provide palettes for
display of imagery, maps, sections, etc.  This would fall under the
category of presentation metadata.

Lastly, participants discussed information that would help ``certify''
datasets and increase user confidence.  This metadata would include:

\begin{itemize}
\item Publications based on the data set.
\item Calibration files.
\item Quality control procedures and description.
\item Comparisons with other data.
\item Error fields, caveats.
\item Processing algorithms.
\end{itemize}

These four ``layers'' (``use'' or semantic metadata, discovery metadata,
presentation metadata, and certification metadata) are all optional,
but the more layers a data provider delivers, the more useful the
data.  At a minimum, we expect that any data set in the pilot study
would provide the use/semantic metadata.  Other layers of metadata
could then be added on this base.

Participants recommended that endangered datasets critical for
California Current research be identified and a rescue plan developed
and implemented.  A plan for a West Coast DODS would build up a set of
local nodes and establish a regional, loosely coordinating node.

\section{Introductions and NOPP Objective}

The meeting convened with Mark Abbott, West Coast DODS regional
workshop coordinator, welcoming the workshop participants and then
asking everyone to introduce themselves (\sectionref{II,attendees}
lists the workshop participants).  He noted that there is an abundance
of data available to scientists (historic, etc.) and that the NOPP
project would facilitate greater data sharing.  Mark stated that he
wants to focus on science as it pertains to DODS in this workshop.

He presented the NOPP Objective: ``to easily access certain data types
in specified locations/times, regardless of data sources, and without
special efforts or insights on the part of the user about the data
sources.''

\subsubsection{VODHub (DODS) Workshop Issues}

During the course of the workshop, the following questions pertaining
to DODS would be addressed:
\begin{itemize}
\item Is the DODS data model adequate?  If not, what additions are
  required?
\item What are the user interface issues?  This should include basic
  functionality ranging from data discovery to data use.
\item What types of metadata are required?  The focus should be on
  search and use metadata.
\item What datasets will be served via DODS?  These may include fixed
  sets, region- specific sets, and missing sets.
\item Is a central regional node needed for coordination?
\end{itemize}

\subsubsection{West Coast Workshop Issues}

In addition to the VODHub issues above, the following specific issues
pertaining to the West Coast were to be discussed:
\begin{itemize}
\item Enhance dataset discovery, sharing, and access through the use
  of web-based servers and applications, not just DODS.
\item Increase the number of West Coast datasets available in readily
  usable format (include both historical and real-time).
\item Identify a suite of data types of special interest to West Coast
  researchers.
\item Foster collaboration of West Coast region-specific issues.
\end{itemize}

\subsubsection{Workshop Agenda and Guiding Principles}

The workshop agenda was as follows:

\begin{enumerate}
\item Presentations on DODS and Live Access Server (LAS) capabilities.
\item Presentations on research activities on the West Coast and
  available datasets.
\item Evaluation of data servers.
\item Future activities and demonstrations.
\end{enumerate}

Some of the guiding principles should include:

\begin{itemize}
\item Simplicity and endurance.
\item Gain user experience by starting early, starting small end-to-end systems.
\item Multiple sources of data and services.
\item Science involvement is essential.
\item Learn from experience.
\end{itemize}

\section{Overview of the DODS System}

Paul Hemenway, of the DODS technical staff at URI, summarized the DODS
concept with a graphical example.  He pointed out that DODS is a data
transfer mechanism that allows users access to data at different sites
and does it independently of data format.  Paul drew an example of a
DODS client accessing data over the web from a DODS server.

The floor was then turned over to James Gallagher, one of the
developers of DODS, who covered the underlying aspects of DODS and how
they relate to the workshop.  James covered the following issues:

\begin{itemize}
\item What types of metadata are required?
\item Level of architecture and design constraints of DODS.
\item The DODS protocol: capabilities and deficiencies.
\item Is the data model sufficient?   
\item DODS is an end-to-end system.
\end{itemize}

James began the overview of DODS by accessing the DODS demonstration
website
(\Url{http://po.gso.uri.edu/~dan/dods-regional-workshops/dods-regional-workshops.html}).
By going through a series of web pages, the DODS model concept was
conveyed to the workshop.  Some of the more important web pages are
highlighted below:

\paragraph{WEB PAGE}  What is DODS? 
\begin{itemize}
\item An architectural framework to allow a user to easily access data
  over the network in a consistent fashion.
\item DODS is an extension of the web.
\end{itemize}

\paragraph{WEB PAGE}  DODS can subset, acquire, and ingest data.  
\begin{itemize}
\item DODS does not locate or analyze data for a user.
\item DODS can move model and non-model data.  
\end{itemize}

As an aside, James pointed out that there are several levels of DODS
clients ranging from Matlab, Ferret, GRADS, and IDL to writing your
own client using languages such as C++ or Java.  It is also possible
to write applets to create clients.

\paragraph{WEB PAGE}  What DODS is not:
\begin{itemize}
\item An analysis package.
\item A data location service.
\item A web browser.
\item A tool to make value judgements with regard to data in the
  system. (Data quality is the responsibility of the provider AND of
  the user). But it is very easy to provide a listing of
  ``acceptable'' DODS datasets.
\end{itemize}

\paragraph{WEB PAGE}  Data level systems
\begin{itemize}
\item Directory level (Global Change Master Directory (GCMD) is a good example).
\item Inventory level (satellite data).
\item Data level.
\end{itemize}

DODS has been built from bottom up.  Other earth science systems are
built from the top down.  DODS focus is to build data transfer
functionality first and worry about inventory and directory
capabilities later.

\paragraph{WEB PAGE}  Underlying Philosophy of DODS
\begin{itemize}
\item Anyone willing to share their data should be able to so via DODS.
\item The user should be able to use the application package with
  which she or he is the most familiar to examine or analyze the data
  of interest.
\end{itemize}

\paragraph{WEB PAGE}  Metadata and Interoperability
\begin{description}
\item[Syntactic metadata] - information about the data type and
  structures at the computer level often referred to as the data model
  (the base level of information you need, in order to have something
  inside a computer).  
\item[Semantic metadata] - information about the content of the data
  (information about variables, etc.).
\end{description}

\paragraph{WEB PAGE}  Metadata relationships between levels - syntactic metadata
\begin{itemize}
\item The directory and inventory levels are generally subsets of that
  at the data level.
\end{itemize}

\paragraph{WEB PAGE} Metadata relations between levels - Semantic Metadata
At the data level (information needed to use the data):
\begin{itemize}
\item Parameter names (e.g. temperature).
\item Units (e.g. degrees centigrade).
\item Missing value parameter (e.g. -9999).
\end{itemize}

\paragraph{WEB PAGE}  At the directory level (information needed to locate datasets):
\begin{itemize}
\item Parameter ranges.
\item Parameter names.
\item Campaign.
\end{itemize}

\begin{description}
\item[Question:]  Where does DODS stand with federal metadata standards?
\item[Answer:] Some are currently served with metadata.  The group
  needs to decide on what metadata standard to use (FGDC, EPIC, etc.).
\end{description}

\paragraph{WEB PAGE}  Semantic metadata at the data level.
\begin{itemize}
\item Use metadata.
\item Directory level.
\end{itemize}

\paragraph{WEB PAGE}  Use Metadata.
\begin{description}
\item[Translational use metadata] - Metadata to translate dataset data
  objects to those with which the user is more familiar (e.g.,
  variable names, scaling of data values).  
\item[Descriptive use metadata] - Metadata that describes operations
  performed to obtain the delivered data.  
\end{description}

DODS focus is on translational use metadata.

\paragraph{WEB PAGE}  Levels of interoperability at the data level:
\begin{description}
\item[Level 0] - no syntactic or semantic metadata - FTP.
\item[Level 1] - rigid syntactic, no semantic metadata - DODS
\item[Level 2] - rigid syntactic, human readable semantic use
  metadata - A subset of DODS data sets.
\item[Level 3] - rigid syntactic, consistent semantic use metadata;
  i.e., machine-readable - A subset of the DODS Level 2 data sets.
\end{description}

\paragraph{WEB PAGE}  DODS supports three data objects:
\begin{enumerate}
\item Data descriptor structure - DDS  (syntactic metadata).
\item Data attribute structure  - DAS (semantic metadata).
\item Data (the actual data in a binary structure).
\end{enumerate}

In addition, DODS servers support several other services:
\begin{itemize}
\item .ascii, an ascii representation of the data.
\item .info, a more readable version of the .dds and .dat combined.
\item .html form, a web-based form that will help to build a DODS URL.
\end{itemize}  

James showed an example requesting WOCE TOPEX data from a DODS server.
The URL is passed to httpd via a DODS client.  The httpd server
receives the URL, passes it to the DODS server software, which reads
the data, and packages it to be returned to the client.  (Several
comments were made to the syntax that James was writing.)  It was
emphasized that \emph{writing DODS URLs can be difficult!}

Examples of the browser form and DODS dataset access form were
presented.  There were many comments on the content of the data
retrieved, such as latitude and longitude, as they relate to metadata.

\paragraph{WEB PAGE}  The DODS data model consists of the following data types:
\begin{itemize}
\item Byte
\item Integer
\item Short integer
\item Float
\item String
\item URL
\end{itemize}

And groupings of these data types:
\begin{itemize}
\item Array
\item Structure
\item List
\item Sequence (important for relational databases)
\item Grid (any array with axes of information that mean something)
\end{itemize}

A user can mix and match the above data types to represent any
possible data structure.  It was noted that satellite swath data is
difficult to represent in DODS.  Various array types and structures
(i.e., unstructured grid models) that workshop attendees worked with
were discussed.  James noted that handling complex arrays would likely
be a future enhancement to the DODS model.

\begin{description}
\item[Question:] For time varying, 3-D array datasets with many
  variables, how do users slice though the data to obtain time slice
  or level slice information?
\item[Answer:]  
\begin{enumerate}
\item For large datasets, where data are stored in HDF as ``chunked''
  data, it is possible to reach in and pull small sections out.  Model
  data is usually large and is typically stored in compressed
  granules.
\item Usually in multi parameter datasets, each parameter is stored in
  separate files, which means each has its own separate URL.
\end{enumerate}
\end{description}

Metadata about space and time is actually data in the dataset.
Latitude and longitude are easier to handle than time.

\paragraph{WEB PAGE}  Additional DODS core attributes include:
\begin{itemize}
\item De-referencing of DODS URLs in the constraint expression.
\item Server side functions.
\end{itemize}

James noted that space and time variables need to be handled in a
special way.

The File Server is a DODS accessible inventory of the files in a
multi-file dataset.

\paragraph{WEB PAGE}  The Aggregation Server was still in beta development (available in two 
months).

\paragraph{WEB PAGE}  DODS - Dir
\begin{itemize}
\item Uses a browser to run through a collection of files.
\item When the URL ends in a ``/'', this indicates that it is a directory.
\end{itemize}

\paragraph{WEB PAGE}  Client and server status graphic (see the DODS website for details).
Also, an analysis server for GRADS and SQL are available.

\subsection{Examples and Demos}

\subsubsection{First Demo}

James went to the Global Change Master Directory (GCMD) website.  He used the 
GCMD to find several DODS datasets.  Using the metadata provided, a DODS URL was 
selected.  The following procedures were used:

\begin{enumerate}
\item The URL was pasted into Matlab.
\item Data were obtained from a remote site, and imported into Matlab
  using the LOADDODS command.
\item Sea surface temperature was plotted in a GUI window.
\item A sub-sample was performed to get every 16th value.
\end{enumerate}

\begin{description}
\item[Question:]  Is the image data type served by DODS?
\item[Answer:] Animation and image formats are not served by DODS.
  James thought it should be added in the future.  Some users want
  this functionality.
\end{description}

\subsubsection{Second Demo}

James used Matlab and Level 3 metadata to create an image of sea
surface temperature (an example of Level 3 interoperability).  The
interface constructed the constraint based on geographic and time data
and, as a result, variables were created.  Other functionality was
demonstrated.

\begin{description}
\item[Question:] How does DODS differ from JGOFS (as it pertains to
  the NE GLOBEC program)?
\item[Answer:] Data management systems and data delivery systems are
  different.
  
\item[Question:] Are there security issues (e.g., breaking in or
  corrupting datasets)?
\item[Answer:] There are different levels.
\begin{enumerate}
\item Modifying your data files (but DODS is read-only so it is
  difficult to corrupt your data).
\item PERL is insecure on the web.
\item C++ software is more secure.
\item Security support exists but is not great right now.
\end{enumerate}
\end{description}

\subsection{Demonstration of the Live Access Server (LAS)}

Jon Callahan - PMEL

Several examples of the LAS were given by accessing data using a web
browser and using Ferret:

\begin{enumerate}
\item Displayed SODA data.
\item Pacific Fisheries Environmental Lab (did not connect).
\item Carbon Modeling Consortium (CMC) website.
\item Displayed CO2 emissions data for Mauna Loa. 
\item Displayed ship tracking.
\end{enumerate}

Comments:

\begin{enumerate}
\item LAS provides useful visualizations of data.
\item The data provider needs to know how to present data.  
\item The niche for LAS is data browsing.
\item Ferret is hard to use.
\item Prefers NetCDF format.
\item Jon could possibly put West Coast datasets up on the LAS, given
  funding.
\end{enumerate}

\begin{description}
\item[Question:]  Are there other interfaces?
\item[Answer:]  No other interfaces other than the web browser.
\end{description}

\section{Science Presentations and Related Regional Datasets}

 (see \sectionref{II,urls}
for important websites)

\subsection{Jane Huyer - OSU}

 (used overheads)

Dataset listing:
\begin{enumerate}
\item CTD sampling (3 and 5 times per year).
\item Satellite track.
\item GLOBEC LTOP in NCCS
\item CTD, biochemistry, Drifters, mooring, ADCP, mocness, vertical
  nets, towed acoustics.
\end{enumerate}


The goal is to look at long-term climate change.  Interested in long term and 
vulnerability.

\paragraph{OVERHEAD} - Example of five seasonal means.

\paragraph{OVERHEAD} - El Ni\~no September temperature graph.

\paragraph{OVERHEAD} - Derived data set.

Would like to make current meter data readily accessible.


\subsection{Jerry Wanetick - SIO}

(web presentation, www.ccs.ucsd.edu/zoo)

Center for Coastal Studies (CCS) Data Zoo Contents (see \sectionref{II,zoo})

Note:  Table of Non-Zoo CCS data is not available at the website. (but
see table~\ref{II,table1} \texorhtml{on \pageref{II,table1}}{}.

At the website: 
\begin{itemize}
\item Showed directory of data.
\item All data is formatted into ASCII.
\item Data could be served using a FreeForm server.
\item All time series data are in ASCII.
\end{itemize}

Community information node to get information about regional data.
\begin{itemize}
\item Would list performance of server, etc.
\item Data may need to be certified for making it usable.
\item Some people may not want to do this.
\end{itemize}

\paragraph{WEB PAGE} Plotted drifter data. (Jerry noted that SIO is putting together a catalog 
showing how each have been released).

\paragraph{WEB PAGE}       Showed drifter trajectory image.

\paragraph{WEB PAGE}       Displayed meteorological data.

\paragraph{WEB PAGE}        Showed current and temperature observation data.

\paragraph{WEB PAGE}       Displayed de-tided data.

\paragraph{WEB PAGE}       Displayed current meter data.

Comments:
\begin{itemize}
\item Possibly have a server that gives a user data in various formats
  (de-tided, raw, avg., etc.) and then creates a view of the dataset.
\item Regional node function (certification, community kinds of
  things)?
\end{itemize}

\paragraph{WEB PAGE}       ADCP data displayed from website.

\paragraph{WEB PAGE}       Data broken up by depth.

CCS Data, Zoo Data Characteristics:
\begin{itemize}
\item Heavy on velocity and temperature time series.
\item Hourly averages mostly. 
\item Missing data:
\begin{itemize}
\item Shipboard ADCP from CODE, SMILE, etc.
\item Incomplete CTD data from SMILE.
\item Big holes in SuperCODE (only partial CTD).
\end{itemize}
\item 4-minute averages available for SBCSMB \& NCCCS.
\item High frequency ADCP \& pressure data from Iwaves.
\item SMILE 7.5-minute data, possibly still at WHOI (may need to rescue soon).
\end{itemize}

CCFS Data, Zoo Science Issues:
\begin{itemize}
\item Hourly average data preclude internal wave studies.
\item Description of seasonal cycles of temperature and velocity for
  the West Coast shelf.
\item Response to wind forcing on the West Coast.
\begin{itemize}
  \item Analysis of remote forcing (coastal trapped waves) for the
  entire West Coast.  
\item Study the relative roles of local and
  remote wind forcing along the West Coast.
\end{itemize}
\item Abundant data available to estimate tidal constituents for
  currents all along the West Coast.  
\begin{itemize}
\item A high- resolution tide
  model including data assimilation of both velocities and sea level.
\item De-tiding satellite altimetry and shipboard ADCP coastal
  data.  
\item Internal tide climate (barotropic tide model -
  observed currents) following Foreman et al. (Dec 15, 2000 JGR).
\end{itemize}
\end{itemize}

\subsection{Hal Batchelder - OSU/COAS}

(used overheads)

The focus of his work is on GLOBEC data.

\paragraph{OVERHEAD} - Showed a listing of nineteen datasets, the area
covered, time period, and data source.

\paragraph{OVERHEAD} - Graphic of time series of three fish species.

\paragraph{OVERHEAD} - Graphic of larval ship surveys.

\paragraph{OVERHEAD} - Graphic of geographic distribution.

\paragraph{OVERHEAD} - Graphic of 3-5 time visitation.

\paragraph{OVERHEAD} - Graphic of long term sampling.

\paragraph{OVERHEAD} - Graphic showing Gulf of Alaska GLOBEC monitoring stations.

Comments:
\begin{itemize}
\item Lots of surface information, long-track, etc. have been collected.
\item Lots of other miscellaneous datasets have been collected.
\item Hard models to distribute.
\item There is a need to make models available to other researchers.
\item Lots of these models are going on in research groups, but how to
  share this data is key (near-line storage vs. on-line storage).
\end{itemize}

\subsection{Corrine James - OSU}

Oregon State's Satellite Archive for the NE Specific US GLOBEC Program (website).

Comments:

\begin{itemize}
\item Has a large archive of satellite data.
\item Long- range plan is to serve up AVHRR high resolution data.
\item Pathfinder data is available.
\item Will add SeaWiFS data in the future.
\item Data resides in a sequel server database.
\item Data is gridded to Level 3.
\item 1 kilometer resolution.
\end{itemize}

\subsection{Mark Abbott - OSU}

Remote Sensing Ocean Optics

(presentation from website,
\Url{http://nugget.oce.orst.edu/ORSOO/oregon/drifters})

Mark noted that data files are available via ftp through a web interface.

\paragraph{WEB PAGE}       WOCE style drifter (follows current, not wind).

There have been two sets of deployments:

\paragraph{WEB PAGE}       Displayed plots of drifters.

\paragraph{WEB PAGE}       Displayed data files (3000 points in a file).

\paragraph{WEB PAGE}       Satellite Dish Installation. (Installed an x-band receiving station.)

\paragraph{WEB PAGE}       Showed coverage of OSU satellite dish.

Comments by Mark:
\begin{itemize}
\item The web is good for quick requests for data.
\item People are moving to near real-time on West Coast.
\item Data provider issues need to be addressed.
\item A lot of data from the southern oceans.
\begin{itemize}
\item Moorings.
\item Older California drifter data.
\item Some profile data.
\end{itemize}
\end{itemize}

\subsection{Brian Schlining - MBARI}

 (used overheads).

The primary questions the staff at MBARI seek to answer are:
\begin{enumerate}
\item What are the mean and fluctuation components of phytoplankton primary production, 
biomass, and species composition on time scales ranging from days to years?
\item What are the physical, chemical, and biological processes responsible for the mean 
and fluctuating components?
\item What controls primary production and phytoplankton growth rates?
\item What is the role of meso and microzooplankton in coastal upwelling systems?
\item What is the fate of primary production?
\item What are the biological consequences of El Ni\~no?
\end{enumerate}


Scientific questions:
\begin{enumerate}
\item What are the physical links between the Tropical Pacific and the
  California Current, and what are their characteristic time and space
  scales?
\item What alternate physical processes affect the California Current
  at these characteristic scales?
\item What are the relative roles of mesoscale versus basin-scale
  dynamics in forcing ecosystem variability?
\item By which processes do physical forcing regulate ecosystem
  dynamics at seasonal, interannual and decadal time-scales?
\item By which processes do physical forcing regulate ecosystem
  structure and bio-diversity?
\end{enumerate}


\paragraph{OVERHEAD} - Overall display of El Ni\~no.

\paragraph{OVERHEAD} - Display of El Ni\~no bloom.

Brian noted that MBARI is putting together an El Ni\~no notebook.

\paragraph{OVERHEAD} - Showing buoys, moorings, and drifter data.

\paragraph{OVERHEAD} - Showing master plan.

Brian next went to the MBARI website to show what datasets are available:
\begin{itemize}
\item OASIS Mooring data
\item Wind data
\item CTD data
\item Barometer data
\end{itemize}

He mentioned that all the datasets would be available through DODS in the next few 
weeks.


\subsection{Toby Garfield - San Francisco State University}

(used overheads)

Data types available:
\begin{itemize}
\item Mooring data (at Scripps)
\item Meteorology
\item CODAR
\item Ship-based (CTD)
\item Bottle/Net
\item Underway data:
\begin{itemize}
\item ADCP.
\item Scanfish.
\item Floats.
\item Drifters.
\item Modeling.
\item Pioneer Seamount for broadband acoustic data.
\item CalCOOS (potential dataset).
\item CIRSI.
\end{itemize}
\end{itemize}

\section{Server Issues}

\begin{description}
\item[Question:]  What does it take to build a DODS server?

As an example, CTD data (from Jerry's Data Zoo) was used.  The CTD data has the 
following attributes:
\begin{itemize}
\item Currently they are flat ASCII files.
\item Have a header file.
\item Have other information in an ancillary file.
\end{itemize}
\end{description}

James Gallagher suggested to use the FreeForm server for CTD (columnar
data).  However, he noted that NetCDF is easier to configure, as
opposed to FreeForm which needs a special configuration script to read
columnar data.  A lot of discussion and questions arose about handling
this kind of data.

Some of the questions included:
\begin{description}
\item[Question:]  Does NetCDF require EPIC convention metadata?
\item[Answer:]  NO.
  
\item[Question:] What happens to your data files if there is sensor
  dropout, when you are using FreeForm?
\item[Answer:] The URI staff has had some experience with this but not
  sure what would happen.
  
\item[Question:] How long would it take to get a FreeForm server up
  and running for Jerry's data?
\item[Answer:] If the FreeForm server exists on the machine with data,
  the format file and additional files could be formatted in one hour.
  
\item[Question:] How do you tell DODS how your data is laid out?
\item[Answer:] It depends on which server you want to use and how your
  data is laid out.  The servers don't inherently understand directory
  structure.  JGOFS can traverse hierarchical collections of files.
  James is not sure if FreeForm can cross multiple data files.
  
\item[Question:] What is the next step to turn on a DODS server?
\item[Answer:] Put the executable in a directory along with other
  files in certain directories.
  
  Installation of the servers is handled automatically.  James noted
  that the FreeForm server is the most complicated server to set up.
  
\item[Question:] Can you get metadata from FreeForm?
\item[Answer:] You get no semantic metadata.  It must be written up in
  a structured text object.  The metadata goes into a third file.
  
\item[Question:] Is the format for FreeForm in DODS documentation
  presented with examples?
\item[Answer:] Yes, and there are examples.
\end{description}

James compared and contrasted JGOFS and NetCDF.  He noted that JGOFS
can be tailored quite easily to handle your data.

Supported DODS Servers (user software products):
\begin{itemize}
\item NetCDF
\item HDF 4 (HDF.EOS)
\item DSP
\item Matlab
\item FreeForm
\item JGOFS
\item RDBS (anything using JDBC)
\end{itemize}

Servers not supported:
HAO, GRIA, GrADS, WMT (Gateway)

There is a user guide for the FreeForm server.

\begin{description}
\item[Question:]  Are there unique issues concerning satellite data?
\item[Answer:] Anything below level 3 would be difficult.  Swath data
  is difficult.
  
\item[Question:] What about .jpg and .gif images?
\item[Answer:] These file types are easily viewed in a browser.
\end{description}

Comment:
Large format files, like towed acoustic data, need to be put into NetCDF format.

\begin{description}
\item[Question:] What are the maintenance issues and efforts involved
  once someone has established a DODS server?
\item[Answer:] URI tries to make sure configuration files never need
  to be changed.  Unidata provides user support.
\end{description}

\subsubsection{Is the DODS data model adequate?}

There were no major deficiencies noted.  Acoustic data and animation
capabilities were issues that would be tackled at a later date.

\section{The California Current System}

Throughout the science presentations, a few key themes emerged.

California Current System research issues:
\begin{itemize}
\item Processes vary on long time and large space scales (e.g., local
  vs. remote forcing).
\item Mesoscale variability is an important component.
\item Eastern boundary currents are productive biologically, and
  coupling with physical forcing is a critical issue.
\end{itemize}

In a sense, these issues are almost dogma, but they do lead to some common 
requirements for any data system:
\begin{itemize}
\item Preservation and stewardship of data $\rightarrow$ time series are critical.
\item Integration across a variety of data types $\rightarrow$ bio/phys/chem.
\item Access by non-experts $\rightarrow$ managers as well as researchers who
  are not specialists with every data set.
\end{itemize}

\section{Organization Issues}

\subsection{Regional Node}

Group discussion led to strong support for the idea of a West Coast
regional node, in addition to the local nodes (discussed below).  The
regional node would provide services that would be beyond the
capabilities of a local node or would be done more efficiently at a
regional site.  Possible functions include:
\begin{itemize}
\item Data set certification - mechanisms to provide some degree of
  assurance of data quality for decades.
\item Data archaeology and rescue - many datasets are lacking homes.
\item Long-term archiving - when projects end, investigators retire, etc.
\item Advanced visualization services - display tools such as those
  developed at NOAA/PMEL as part of the Live Access Server.
  (\Url{http://ferret.wrc.pmel.noaa.gov/las})
\item Define computer scenarios and output that might be made
  accessible (e.g., ENSO vs.  non-ENSO model runs).
\end{itemize}

\subsection{Datasets and Nodes}

The workshop participants briefly discussed possible local nodes that might form a 
cooperative pilot study for the West Coast.

\begin{itemize}
\item SIO
\item UCSB
\item PFEG
\item MBARI
\item NPS
\item UCSC
\item SFSU
\item OSU
\item PMEL
\item UW
\end{itemize}

In parallel, there are projects, some of which span multiple
institutions: CalCOFI, SIO Data Zoo, MMS, GLOBEC, PNCERS, PISCO, NOPP
(multiple programs), CoOP (COAST and WEST).  The point was not to list
all possible projects and all possible institutions, but to develop a
``reasonable'' list for a possible pilot study.

Datasets (again, not meant to be comprehensive) include: CTD,
moorings, drifters, ADCP, CODAR, SeaSoar/Scanfish, surface
meteorology, satellite imagery, chlorophyll, zooplankton biomass, and
nutrients.

\subsection{Metadata and Data Services}

Participants discussed the minimum set of semantic metadata (in the
DODS vernacular) that would be necessary to use ``minimally''
processed data. That is, we do not expect people to serve raw volts
from a sensor. These are:

\begin{itemize}
\item Time and location - this will require an ancillary file that
  describes the time/space axes (e.g., lat/lon convention, time
  convention, etc.).
\item Descriptive variable name, including something that would
  indicate the sensor type.
\item Units.
\item Missing data value - e.g., -9999, NaN, etc.
\item Scaling, including slope/offset.
\end{itemize}

This information would allow a researcher to read in a data file and
conduct useful science.  It was noted that the COARDS convention as
part of the NetCDF standard would help with naming, etc.  See
\Url{http://www.unidata.ucar.edu/packages/udunits/index.html} for more
details.  Also see
\xlink{http://ferret.wrc.noaa.gov/noaa\_coop/coop\_cdf\_profile.html}{http://ferret.wrc.noaa.gov/noaa_coop/coop_cdf_profile.html}.

A minimum set of metadata for dataset discovery (that is, to help
users find datasets of interest) could be built up from the NASA
Global Change Master Directory (GCMD) which has an extensive set of
catalog information, standard terms, etc.  The GCMD is also willing to
help users describe their data holdings (see
\Url{http://gcmd.gsfc.nasa.gov/}).

For data presentation, it is often useful to provide palettes for
display of imagery, maps, sections, etc.  This would fall under the
category of presentation metadata.

Lastly, we discussed information that would help ``certify'' datasets
and increase user confidence.  This metadata would include:
\begin{itemize}
\item Publications based on the data set.
\item Calibration files.
\item Quality control procedures and description.
\item Comparisons with other data.
\item Error fields, caveats.
\item Processing algorithms.
\end{itemize}

These four ``layers'' (``use'' or semantic metadata, discovery metadata,
presentation metadata, and certification metadata) are all optional,
but the more layers a data provider delivers, the more useful the
data.  At a minimum, we expect that any dataset in the pilot study
would provide the use/semantic metadata.  Other layers of metadata
could then be added on this base.


\section{Attendees}
\label{II,attendees}

\begin{center}
West Coast DODS Regional Workshop
January 17th and 18th, 2001
Attendee List
\end{center}

\begin{center}
  \begin{tabular}[t]{llll} \\
\textbf{Name} & \textbf{Organization} & \textbf{Phone} & \textbf{Email} \\
Mark Abbott &             COAS/OSU               & (541)737-4045         &  mark@oce.orst.edu \\
Hal Batchelder &          COAS/OSU               & (541)737-4500         &  hbatchelder@oce.orst.edu \\
Eric Beals &              OSU                    & (541)737-4548         &  beals@oce.orst.edu \\
Jon Callahan &            NOAA/PMEL              & (206)526-6801         &  callahan@pmel.noaa.gov \\
Jane Fleischbein &                OSU                    & (541)737-5698         &  flei@oce.orst.edu \\
Jim Fritz &                       TPMC                   & (781)545-1346         &  jfritz@tpmc.com \\
James Gallagher &         URI                    & (541)757-7992         &  jgallagher@gso.uri.edu \\
Toby Garfield &           SFSU-RTC               & (415)338-3713         &  garfield@sfsu.edu \\
Paul Hemenway &           URI                    & (401)874-6677         &  phemenway@gso.uri.edu \\
Jane Huyer &              OSU                    & (541)737-2108         &  ahuyer@oce.orst.edu \\
Corinne James &           OSU                    & (541)737-2270         &  corinne@oce.orst.edu \\
Mike Korso &              COAS/OSU               & (541)737-3079         &  kosro@oce.orst.edu \\
Nathan Potter &           OSU                    & (541)737-2293         &  ndp@oce.orst.edu \\
Brian Schlining &         MBARI                  & (831)775-1855         &  brian@mbari.org \\
Ted Strub &               OSU                    & (541)737-3015         &  tstrub@oce.orst.edu \\
Jerry Wanetick &          SIO                    & (858)534-7999         &  jwanetick@ucsd.edu \\
  \end{tabular}
\end{center}


\section{Select HTTP References}
\label{II,urls}

\begin{description}
\item[NASA Global Change Master Directory]
\Url{http://gcmd.gsfc.noaa.gov/cgi-bin/}

\item[NOAA PMEL LAS Examples]
\Url{http://ferret.wrc.noaa.gov/nopp/main.html}

\item[SIO Data Zoo]
\Url{http://www.ccs.ucsd.edu/zoo}

\item[GLOBEC Satellite Data]
\Url{http://coho.oce.orst.edu}

\item[GLOBEC Drifter Data]
\Url{http://nugget.oce.orst.edu/ORSOO/oregon/drifters}

\item[DODS]
\Url{http://unidata.ucar.edu/packages/dods}

\item[DODS Documentation]
\Url{http://top.gso.uri.edu/}

\item[DODS Workshop Presentations]
\Url{http://po.gso.uri.edu/~dan/dods-regional-workshops/dods-regional-workshops.html}

\item[MBARI]
\Url{http://www.mbari.org}

\item[NOAA PMEL LAS]
\Url{http://www.coho.oce.orst.edu}

\item[NOAA PMEL LAS]
\Url{http://www.ferret.noaa.gov/Ferret/LAS}

\item[DODS HOME]
\Url{http://unidata.ucar.edu/packages/dods/  (home page)}

\item[GLOBEC]
\Url{http://globec.oce.orst.edu}

\item[GCMD]
\Url{http://www.gcmd.gsfc.nasa.gov}
\end{description}

\section{CCS Data Zoo Contents}
\label{II,zoo}

\begin{longtable}{|p{0.75in}|p{0.75in}|p{1.0in}|p{1.25in}|p{0.75in}|}
  \caption{CCS Data Zoo.\label{II,table1}}
\\ \hline
\textbf{Project} & \textbf{Sponsor/ Contractor} & \textbf{Region} &
      \textbf{Data Types} &    \textbf{Dates} \\ \hline
\endfirsthead
\caption{CCS Data Zoo (continued).}
\\ \hline
\textbf{Project} & \textbf{Sponsor/ Contractor} & \textbf{Region} &
      \textbf{Data Types} &    \textbf{Dates} \\ \hline
\endhead
\hline
\endfoot
CAMP &
MMS/SAIC &
Pt. Conception &
Currents &
April 1992 - July 1994 \\ \hline
CCCCS (Central California Coastal Circulation Study) &
MMS/Ray\-theon &
Pt. Conception to 
San Francisco &
\begin{tablelist}
\item Moored Currents
\item CTD
\item Drifters
\item Sea Level (NOS)
\item Met
\item IR Sat 
\item XBT (unavailable) 
\end{tablelist} &
Feb 1984 - July 1985 \\ \hline
CODE (Coastal Ocean Dynamics Experiment) &
NFS/NCAR, NOAA, USGS, WHOI, SIO, OSU, UNH &
Pt. Reyes to Pt. Arena &
\begin{tablelist}
\item Moored Currents
\item CTD
\item Drifters
\item Met
\item XBT (unavailable)
\end{tablelist} &
April 1981 - August 1982 \\ \hline
CTZ (Coastal Transition Zone) &
ONR/OSU, NPS &
N. California &
ADCP &
1987 \\ \hline
Del Mar Temperatures &
&
Del Mar, CA &
Temperatures 
(Thermistor Chains) &
1978 \\ \hline
SBCSMB (Santa Barbara Channel, Santa Maria Basin Study) &
MMS/SIO &
Santa Barbara 
Channel, Santa 
Maria Basin &
\begin{tablelist}
\item Moored Currents
\item CTD (Survey, Moored)
\item Drifters
\item Met
\item XBT
\item ADCP (Survey, Moored)
\item Bottom Pressure
\item NDBC Met \& ADCP
\item AVHRR
\end{tablelist} &
April 1992 - Present \\ \hline
NCCCS (Northern California Coastal Circulation Study) &
MMS, EG\&G, and SIO &
San Francisco - Oregon Border &
\begin{tablelist}
\item Moored Currents
\item CTD (Survey, Moored)
\item Drifters
\item Met
\item XBT
\item Bottom Pressure
\end{tablelist} &
1986 - 1989 \\ \hline
OPUS (Organization of Persistent Upwelling Structures) &
NSF/OSU &
Pt. Conception - Pt. 
Arguello &
\begin{tablelist}
\item Moored Currents 
\item CTD
\item Drifters
\item Met
\item XBT
\item Bottom Pressure
\item Sea Level
\end{tablelist} &
Pilot: March - April 1981
Main: April - July 1983 \\ \hline
SBC (Santa Barbara Channel Study) &
MMS/SAIC, Dyanlysis of Princeton &
Santa Barbara 
Channel &
\begin{tablelist}
\item Moored Currents
\item CTD (Survey, Moored)
\item Drifters
\item Met
\item Bottom Pressure
\end{tablelist} &
Jan 1984 - Jan 1985 \\ \hline
SMILE (Shelf Mixed Layer Experiment) &
NSF/WHOI &
Pt. Reyes - Pt. Arena &
\begin{tablelist}
\item Moored Currents
\item CTD (Survey, Moored)
\item ADCP (Survey)
\item Met
\item Thermistor Chains
\item Sea Level
\end{tablelist} &
Nov 1988 - May 1989 \\ \hline
\end{longtable}

\begin{longtable}{|p{0.75in}|p{0.75in}|p{1.0in}|p{1.25in}|p{0.75in}|}
  \caption{non-Zoo CCS Data.\label{II,table2}}
\\ \hline
\textbf{Project} & \textbf{Sponsor/ Contractor} & \textbf{Region} &
      \textbf{Data Types} &    \textbf{Dates} \\ \hline
\endfirsthead
\caption{non-Zoo CCS Data (continued).}
\\ \hline
\textbf{Project} & \textbf{Sponsor/ Contractor} & \textbf{Region} &
      \textbf{Data Types} &    \textbf{Dates} \\ \hline
\endhead
\hline
\endfoot
CDIP (Coastal Data Information System) &
USACOE-CDBW/SIO &
West Coast, Hawaii, East Coast (couple of stations) &
Wave direction and 
energy spectra &
1977 - Present \\ \hline
Iwaves &
ONR/SIO &
San Diego County &
Bottom-mounted 
upward-looking ADCP 
\& Temperature &
1998-1999 \\ \hline
Swash2000, BSRip, Undertoad &
ONR/SIO &
La Jolla &
Numerous nearshore 
wave, current, run-up, 
sonic altimeter and bed 
stress measurements &
1980 - present \\ \hline
NSTS (Nearshore Sediment Transport Study) &
USACOE &
Torrey Pines, SD Co,
Santa Barbara &
Numerous nearshore 
wave, current, run-up, 
dye injection and bed 
stress measurements &
1977, 1980 \\ \hline
CCSTWS (Coast of California Storm and Tidal Wave Study) &
USACOE &
Dana Point - 
Mexican Border &
Sub-areal and 
bathymetric survey 
transects of the beach 
and nearshore region &
1980 - 1982 \\ \hline
\end{longtable}



%%% Local Variables: 
%%% mode: latex
%%% TeX-master: t
%%% End: 

\renewcommand{\chaptertitle}{Southeast Regional Workshop}
\chapter{\texorhtml{}{Appendix C }\chaptertitle}

%% $Id$

\begin{center}
November 17th - 19th, 2000\\
NOAA - Coastal Services Center\\
Charleston, SC\\
\end{center}

\section{Introduction and Workshop Objectives}

Anne Ball, SE regional DODS coordinator, convened the workshop at 9:00
am.  The workshop facilitator was Steve Bliven.  The objectives of the
workshop were stated as follows:

\begin{itemize}
  \item To familiarize workshop attendees with the DODS server and client.
  \item Determine whether DODS meets the needs of organizations in the southeast.
  \item Develop a list of issues/suggestions to ensure the success of DODS.
  \item Determine how to best implement DODS in the southeast.
\end{itemize}

All the workshop participants introduced themselves and provided
information on what organizations they represented.  See the list
later in this appendix for a listing of workshop attendees.

\section{DODS}

\subsubsection{Background}

Peter Cornillon of the University of Rhode Island (URI) provided a
brief history of DODS.  The conception of DODS was started in the
early `90s and is based on two different client-server systems; one
that looked at satellite data and another that looked at in-situ data.
A meeting in 1992 discussed the development of building an
``umbrella'' client-server system that incorporates both of the above
client-server systems into one.  The first workshop had forty
attendees and was held at URI to discuss a distributed system.
Although it became apparent at this workshop that the system being
designed was based on sufficiently general concepts that it need not
be constrained to oceanography, it was also clear that without a
well-defined focus, progress could well be hampered.  As a result, the
system was called the Distributed Ocean Data System, and was to be
UNIX based.  The effort was initially funded by NSA and NOAA.  More
recently, the National Ocean Partnership Program (NOPP) awarded the
DODS effort a grant to develop a Virtual Ocean Data Hub system.

The first year of the proposed NOPP effort was to obtain input from
the community.  Five regions were formed: Northeast, Southeast, Gulf
Coast, West Coast, and Great Lakes (which was dropped and replaced by
one focusing on GIS) to hold workshops.  Each was led by a regional
coordinator.  The regional coordinators have control over their
workshop, and they also act as a focal point for the expenditures in
the out years.  Each of the regional coordinators have an interest in
oceanography but are from differing backgrounds so as to bring as much
diverse input as possible to the overall effort.  Based on the five
reports generated by each of the regional workshops, a final report
will be generated which will be used at a national workshop.  After
the national workshop is held an overall report discussing
recommendations for the DODS project will be developed.  A technical
meeting will then be held to discuss the technical findings and
recommendations of the national report.

\subsubsection{Overview and Demonstration of  DODS}

Before going into a demonstration of DODS, a document describing DODS
was passed out to each of the participants.  Listed
below are some the highlights of the DODS system as it was presented
to the workshop participants.

Peter emphasized that DODS is an architectural framework to move data
across the network.

The underlying philosophy of DODS  is twofold:
\begin{itemize}
  \item Anyone willing to share their data should be able to do so via DODS.
  \item The user should be able to use the application package with
which she or he is the most familiar to examine or analyze the data of
interest.
\end{itemize}

Peter explained the five steps involved to analyze data:
\begin{enumerate}
\item LOCATE
\item SUBSET
\item ACQUIRE
\item INGEST
\item ANALYZE
\end{enumerate}

DODS addresses the middle three steps, i.e., SUBSETTING, ACQUIRING,
and INGESTING.  DODS doesn't focus on LOCATION or ANALYSIS.  What DODS
is \emph{not} and the underlying philosophy of DODS were covered.

Peter went on to explain that DODS is built from the \emph{bottom up},
with high functionality at the data acquisition level and that
functionality decreasing as you move up to the inventory level and the
directory level.

Syntactic and semantic metadata were described and how they relate to
the data, inventory, and directory levels.  Semantic metadata can be
sub-setted into \emph{use} and \emph{search} categories.  \emph{Use}
metadata can be further subdivided into \emph{translational} use
metadata and \emph{descriptive} use metadata.  DODS provides a rigid
structure for syntactic metadata.  Although not mandatory, DODS
recommends including at a minimum translational use metadata.

Four levels of interoperability at the data level (for data accessible
over the network) were defined:

\begin{description}
\item[Level 0] - no syntactic or semantic metadata - FTP.
\item[Level 1] - rigid syntactic, no semantic metadata - DODS
\item[Level 2] - rigid syntactic, human readable semantic use
  metadata - A subset of DODS data sets.
\item[Level 3] - rigid syntactic, consistent semantic use metadata;
  i.e., machine-readable - A subset of the DODS Level 2 data sets.
\end{description}

Before going into the first demonstration on the use of DODS, three
DODS data objects were defined:

\begin{enumerate}
\item The data descriptor structure (DDS), this is the syntactic
  metadata for a data set.
\item The data attribute structure (DAS), this is the semantic
  metadata for a data set.
\item Data - the actual data.
\end{enumerate}

Additionally, DODS servers support several other services, including:
\begin{itemize}
\item .ASCII - an ASCII representation of the data.
\item .info - a more readable version of the DDS and DAS combined.
\item .html form - a web based form that will help to build a DODS URL.
\end{itemize}

A demonstration was given on how to access a WOCE/TOPEX data set,
which was copied from CD-ROM to a server machine at URI.  In order to
access data, Peter requested data from a DODS server via URLs.  The
URL was then passed to a DODS client, via http.  At this point the
DODS client enters the data into an application from the DODS server
via httpd.  Any data sub-setting is done at the server.  It was
emphasized that \emph{writing DODS URLS can be difficult!}

Three classes of interfaces were described based on the difficulty
involved in building a URL.  These are in order of increasing
ease-of-use:

\begin{description}
\item[Command Line] No help is given creating a URL.
\item[General Purpose URL Builder] A menu-based interface based on
  data dimensions and names.
\item[Graphical User Interface] A URL builder based on geophysical
  parameter input.
\end{description}

Another demonstration was given showing the functionality of the DODS
Dataset Access Form to access the TOPEX dataset.  More buttons will be
added to this form in the future.

The DODS data model was explained along with the data model
constraints.  The following operations are permitted when requesting
data:

\begin{itemize}
\item Projection.
\item Relational operations on list and sequence elements.
\item Rectangular and decimated subsets of arrays and grids.
\end{itemize}

\subsubsection{Issues raised}

After the first live demo on the usage of DODS, the floor was opened
up for discussion. Some of the questions and comments were as follows:

\begin{enumerate}
\item Regarding a user who has data and wants to make them available via
  DODS (how is this done)?  

\emph{Answer}:
  \begin{itemize}
  \item Determine which server is most appropriate for your data.
  \item Go to the DODS home page and download the binary version of
    the appropriate server. (Better documentation is being written for
    these steps.)
  \item You will need to install the server on your own computer, and
    tell the server where the data are located.
\end{itemize}

Peter stated that it is up to each region to determine how they would
like to handle server installation.  Regional support could be used to
support server installation.  In the proposal, Peter thought that each
region would either:

\begin{itemize}
  \item Hire an individual for a given region to help install servers.
  \item OR let sub-contracts to people to install their own servers.
\end{itemize}

He stated that it is really up to the regions on how they want to do
it and that the region perhaps should identify some high priority
sites for server installation.

\item Regarding DODS support for server installation; what happens if
  someone runs into a problem installing a server?  

\emph{Answer}:
  \begin{itemize}
  \item Requests for support should be addressed to the DODS support
    site.  This will allow the DODS project to track problems.
  \end{itemize}

\item If you have other systems in place and want to install a DODS
  server, will this create problems related to the other servers?
  
\emph{Answer}:
  \begin{itemize}
  \item Putting a DODS server on your system does not impact existing servers.
  \end{itemize}

\item Can curvilinear data be handled?  

\emph{Answer}:
\begin{itemize}
\item Yes.  To the best of our knowledge, DODS has been able to move
  every data type that has been encountered.  However, this does not
  mean that appropriate semantic metadata exist to effectively use
  that data.
\end{itemize}


\item Can DODS be extended? 

\emph{Answer}:
  \begin{itemize}
  \item The DODS project will consider extending the DODS data model.
  \end{itemize}

\item The efficiency of the data model is a concern for complex data
sets.  It was thought that if the DODS data model was tweaked a little
bit, these problems might be resolved.  This is a translational use
metadata issue.

\item  A question was raised as to how DODS deals with data compression.

\emph{Answer}:
\begin{itemize}
  \item DODS uses standard compression techniques.
\end{itemize}

\item Why is DODS not a data locator?


\emph{Answer}:
\begin{itemize}
  \item The core of the DODS project is how you move data around.
  \item They are working in the direction of the data locator issue.
  \item Have a contract to develop a web-crawler based on the DODS Dir function.
\end{itemize}

\item Peter sees three issues facing the group (to this point in the workshop):
\begin{itemize}
  \item Moving the data around in a system-efficient way?
  \item Adding additional ``bells and whistles'' to DODS.
  \item What data do you want served?
\end{itemize}

\item Comments regarding rigidity of metadata standards.
\begin{itemize}
  \item Imposing standards as a group.
  \item Nothing precluding users from making own decisions on metadata.
\end{itemize}
\end{enumerate}

Peter gave two more demonstrations:

\begin{enumerate}
\item  Live Access Server to PMEL Data
  Steps involved:
  \begin{itemize}
  \item LAS generates HTML on a form that the user fills out to select
    data.
  \item The user selection is returned to the LAS.
  \item LAS parses the returned request.
  \item Pass request on to Ferret (a NetCDF program).
  \item Ferret then requests a subset from the server site and moved
    over the network.
  \item Data is then returned to browser.
  \item If a server is down, some data won't show up.
  \item Select COADS data (sea surface temperature).
  \item Live Access Server returns a .gif image.
\end{itemize}

(Peter noted that there is metadata available at the LAS site.  This
allows LAS users to show location, time, and so on of data in the date
selection window.)

\item Demo using Matlab.
\begin{itemize}
  \item Peter showed various functionalities using DODS and the Live Access Server.
\end{itemize}
\end{enumerate}

\subsection{Discussion on server related issues}

\subsubsection{Metadata}

Issues concerning metadata were numerous and varied.  Peter re-stated
that only syntactic metadata is \emph{required} to run DODS and that
the server will provide that metadata.  The user is \emph{not} required to
provide semantic metadata.  He indicated that it's OK if metadata is
bound to the data being served.  A few additional points were stated
regarding DODS and metadata:

\begin{itemize}
\item One does not have to reformat or restructure semantic metadata
  bound to a file.
\item Semantic metadata that are not bound to the file may be included
  in the DAS.
\item DODS will take advantage of what it can in regard to metadata
  but can't guarantee that it will take advantage of the structure.
\item Some sites have extensive metadata on the web.
\end{itemize}

\subsubsection{Other metadata issues/comments/suggestions}

For a file to be useful, the translational use metadata encoded in the
DAS should, at a minimum, provide for the translation of names to a
common form, and specify units, and missing value flags.

\begin{enumerate}
\item The SE regional workgroup recommends that at minimum,
  translational use metadata should be available via the DAS.
\item The DODS core should have the ability to associate FGDC
  compliant metadata that may be generated and stored by a third party
  site, and associate that with a DODS dataset. (Third party metadata
  cannot be handled yet and will require a modification to the DODS
  core.)
\item Several participants expressed an interest in quality flags and
  felt that these should be available in the DAS.
\item Peter stated that he would try to work out issues with FGDC
  metadata.  He said that he would bring up the issue of third party
  metadata generators even though this could potentially discourage
  people from serving their data.  The DODS team will also consider
  handling other metadata that are contained in web pages, either by
  incorporating the web page into the DAS or including a URL pointer
  to the web page in the DAS.
\item DODS will not force people to provide metadata.  The SE
  workgroup can require metadata of its participants, but cannot
  prohibit people from serving their data.
\item Work closely with data providers to come up with more useful
  metadata.
\item Data providers should provide a URL link to their metadata.
\end{enumerate}

\subsubsection{Liability}

The issue of having a disclaimer with served data was discussed.  In
many situations in the southeast, disclaimers are necessary/required
to satisfy the South Carolina General Council.  One opinion was that
disclaimers should be located on the client end.  Also, the southeast
region would/should not be allowed to put up a DODS server for data
that has disclaimers unless they (the region) could be insured that
they see the disclaimer first.

\subsubsection{Data Formats and Storage}

Data formats in the southeast region include ASCII files, ACCESS,
EXCEL, Oracle, SEQUEL, INFORMIX, SYBASE, and others.  Peter suggested
that the region identify how many users in the region use which
packages and from that list, look for solutions to accommodate those
users.  He pointed out that DODS needs to make it trivial to install
servers for simple datasets.

\subsubsection{Suggestions/comments concerning servers}

\begin{enumerate}
\item The southeast region should come up with a matrix listing
  available DODS servers that are out there and the various data types
  that need to be served.
\item In what format are the different data types stored in (INFORMIX,
  Oracle, SYBASE, or just ASCII files)?
\item Some level of data management on the server side will be needed.
\item Money will have to be spent on managing data.
\item DODS servers cannot handle Microsoft EXCEL and ACCESS files yet.
  Peter will investigate this.
\item Some workshop participants indicated that they would require
  tech support to install servers, others said that they could handle
  it themselves.  Still others indicated that they need a certain
  level of commitment from principal investigators before server
  installation.
\end{enumerate}

\subsubsection{Discussion on client related issues}

Based on the demonstration of the DODS system, workshop participants
indicated that they would like to see a DODS web browser client.
Additionally, it would be desirable if the browser could be ``beefed
up'' to make it more of a point-and-click interface.

A consensus of the group with regard to application software used
found that participants used SAS while others used EXCEL, ACCESS, and
ArcINFO products.  Peter indicated that the DODS project has
subcontracted to NGDC and ESRI to build a DODS GIS client as well as
to provide access to GIS data in other DODS clients.  A timeline of
eighteen months was given to complete the task.

Two questions were posed related to the client:
\begin{enumerate}
\item Can metadata stored in the Matlab GUI be updated?  

\emph{Answer}: Yes,
  this may be done over the web.
\item On what platforms are clients available?  

\emph{Answer}: Command line
  clients are available on all platforms that support the DODS core.
\end{enumerate}

\section{NCDDC Web Site}

At this point, John Ellis provided a demonstration of the National
Coastal Data Development Center (NCDDC) web site being built at
Stennis Space Center.  In brief, the NCDDC system is:
\begin{itemize}
  \item A virtual data hub, which supports data discovery, based on FGDC
metadata records.
  \item A data delivery process, which directs a user request to the
appropriate server capability.
  \item At the moment, DODS is a server option from the pull-down menu
giving a URL listing of available DODS servers.
\end{itemize}

NCDDC currently has two web sites, one operational and one developmental.

\section{Regional Data Sets}

The workshop attendees were asked the following questions regarding data sets that they held in 
their respective agencies:

\begin{enumerate}
\item How many data sets do they have?
\item What are the spatial/temporal variables of each data set?
\item Volume in bytes?
\item Format of data?
\item Machine/operating system?
\item Metadata availability?
\item Percent archived at a national archive?
\end{enumerate}

\subsection{Chris Friel - Florida Marine Research Institute}
 
Number of data sets:
\begin{itemize}
  \item 175 GIS (a mix of raster and vector).
  \item 75 flat/relational.
\end{itemize}

Spatial/temporal variables of data set:
\begin{itemize}
  \item Mainly Florida (1970-2000).
  \item Biological, some base map data.
  \item Biophysical monitoring.
\end{itemize}

Volume in bytes:
\begin{itemize}
  \item Several hundred gigabytes.
\end{itemize}

Format of data:
\begin{itemize}
  \item Arc/INFO, ArcView (shapefiles), SAS, Access, Excel.
\end{itemize}

Machine/operating system:
\begin{itemize}
  \item NT or UNIX SGI machines.
\end{itemize}

Metadata availability:
\begin{itemize}
  \item 40\% have FGDC metadata.
\end{itemize}

Percent archived at a national archive:
\begin{itemize}
  \item A small percentage in archive.
\end{itemize}


\subsection{Jim Nelson - UNC Marine Sciences}

Number of data sets:
\begin{itemize}
  \item Individual PI or project holdings.
  \item Coastal Observing Survey data (8 platforms, 2 operational now, 4 next year).
\end{itemize}

Spatial/temporal variables of data set:
\begin{itemize}
  \item GA shelf/S. Atlantic Bight.
  \item Hydrological, ADP currents, biological, chemical.
\end{itemize}

Volume in bytes:
\begin{itemize}
  \item     4 gigabytes/year now.
\end{itemize}

Format of data:
\begin{itemize}
  \item ASCII processed data in Matlab.
\end{itemize}

Machine/operating system:
\begin{itemize}
  \item SUN or Linux.
  \item UNIX, Windows, MAC.
\end{itemize}

Metadata availability:
\begin{itemize}
  \item Varies currently, needs work internally.
\end{itemize}

Percent archived at a national archive:
\begin{itemize}
  \item Small amount some provided to NWS.
\end{itemize}

\subsection{Reyna Sabina - NOAA/Atlantic Oceanographic Marine Laboratory (AOML)}

Number of data sets:
\begin{itemize}
  \item One.
\end{itemize}

Spatial/temporal variables of data set:
\begin{itemize}
  \item XBT in Atlantic.
  \item Drifter buoys (global).
  \item Hydrographic.
  \item Transportation.
  \item Hurricane tracking.
\end{itemize}

Volume in bytes:
\begin{itemize}
  \item Approximately 10 gigabytes.
\end{itemize}

Format of data:
\begin{itemize}
  \item Flat files (from PI's), DBMS.
\end{itemize}

Machine/operating system:
\begin{itemize}
  \item SUN, UNIX
  \item DBMS
\end{itemize}

Metadata availability:
\begin{itemize}
  \item Some.
\end{itemize}

Percent archived at a national archive:
\begin{itemize}
  \item     Most.
\end{itemize}

\subsection{Tim Snoots - SC Department of Natural Resources}

Number of data sets:
\begin{itemize}
  \item PI or long-term monitoring.
  \item 10 data sets.
\end{itemize}

Spatial/temporal variables of data set:
\begin{itemize}
  \item SC estuarine, some extends to shelf.
  \item Hydrographic, water quality, water chemistry, biological, sediment data.
\end{itemize}

Volume in bytes:
\begin{itemize}
  \item Less than 10 gigabytes.
\end{itemize}

Format of data:
\begin{itemize}
  \item Microsoft Access, ASCII, Excel, SAS.
\end{itemize}

Machine/operating system:
\begin{itemize}
  \item Servers, Windows NT.
  \item Clients, Windows 95, 98, NT, Macintosh.
\end{itemize}

Metadata availability:
\begin{itemize}
  \item Less than 10\%, in ASCII format.
\end{itemize}

Percent archived at a national archive:
\begin{itemize}
  \item Not clear on how much.
\end{itemize}

\subsection{John Ellis - NOAA/National Coastal Data Development Center (NCDDC)}

Number of data sets:
\begin{itemize}
  \item No holdings.
\end{itemize}

Spatial/temporal variables of data set:
\begin{itemize}
  \item Great Lakes, coast of ME, Gulf of Mexico, Guam, Puerto Rico, FL.
  \item Meteorological/oceanographic, bathymetry, water structure,
currents, water level, tides, surface winds, waves.
\end{itemize}

Volume in bytes:
\begin{itemize}
  \item Unknown.
\end{itemize}

Format of data:
\begin{itemize}
  \item Varies.
\end{itemize}

Machine/operating system:
\begin{itemize}
  \item Varies.
\end{itemize}

Metadata availability:
\begin{itemize}
  \item 180 metadata records/sets FGDC compliant.
\end{itemize}

Percent archived at a national archive:
\begin{itemize}
  \item Unknown.
\end{itemize}


\subsection{Jim Frysinger - College of Charleston}

Number of data sets:
\begin{itemize}
  \item 10 raw, QC 1.
\end{itemize}

Spatial/temporal variables of data set:
\begin{itemize}
  \item Sea surface.
  \item 6 months of sampling data (meteorological data).
\end{itemize}

Volume in bytes:
\begin{itemize}
  \item 3 megabytes.
\end{itemize}

Format of data:
\begin{itemize}
  \item ASCII, Excel.
\end{itemize}

Machine/operating system:
\begin{itemize}
  \item UNIX, Linux.
\end{itemize}

Metadata availability:
\begin{itemize}
  \item Minimal.
\end{itemize}

Percent archived at a national archive:
\begin{itemize}
  \item 0 percent.
\end{itemize}

\subsection{Beth Judge - SC Sea Grant}

Number of data sets:
\begin{itemize}
  \item Individual PI.
\end{itemize}

Spatial/temporal variables of data set:
\begin{itemize}
  \item Geological, currents, mapping (vector and raster), side-scan sonar.
\end{itemize}

Volume in bytes:
\begin{itemize}
  \item Unknown.
\end{itemize}

Format of data:
\begin{itemize}
  \item Arc/INFO, Excel, ASCII, and others.
\end{itemize}

Machine/operating system:
\begin{itemize}
  \item Windows NT, UNIX, varied.
\end{itemize}

Metadata availability:
\begin{itemize}
  \item Minimal.
\end{itemize}

Percent archived at a national archive:
\begin{itemize}
  \item Little.
\end{itemize}

\subsection{Andrew Meredith - NOAA/CSC Coastal Remote Sensing}

Number of data sets:
\begin{itemize}
  \item 3 databases.
\end{itemize}

Spatial/temporal variables of data set:
\begin{itemize}
  \item Water quality, biological.
  \item Coastal change analysis.
  \item National in nature.
  \item Unknown temporal.
\end{itemize}

Volume in bytes:
\begin{itemize}
  \item Less than 10 gigabytes.
\end{itemize}

Format of data:
\begin{itemize}
  \item Flat files, IMAGINE.
\end{itemize}

Machine/operating system:
\begin{itemize}
  \item NT, UNIX
\end{itemize}

Metadata availability:
\begin{itemize}
  \item 100 percent.
\end{itemize}

Percent archived at a national archive:
\begin{itemize}
  \item All.
\end{itemize}

\subsection{Alan Lewitus - Baruch Institute and SC DNR}

Number of data sets:
\begin{itemize}
  \item 7 personal data sets.
\end{itemize}

Spatial/temporal variables of data set:
\begin{itemize}
  \item SC
  \item Biochemical, hydrography, water column, estuarine.
\end{itemize}

Volume in bytes:
\begin{itemize}
  \item Approximately 10 megabytes.
\end{itemize}

Format of data:
\begin{itemize}
  \item Excel.
\end{itemize}

Machine/operating system:
\begin{itemize}
  \item Windows 98.
\end{itemize}

Metadata availability:
\begin{itemize}
  \item None.
\end{itemize}

Percent archived at a national archive:
\begin{itemize}
  \item None.
\end{itemize}

\section{Web Browser}

At this point in the workshop, a lengthy discussion took place
regarding the functionality of a web browser interface that many
thought would be usefuul to preview data accessible via DODS.  Many
ideas were put on the table as to what that functionality should be.
In summary, participants felt those functions should include:

\begin{itemize}

\item LOCATION/SUBSETTING 
\begin{itemize}
\item Graphics (rubber-banding).
\item Coordinates (typing in ranges or geographic regions).
\item Consistency (Boolean operations) for searching.
\item Space/time/variable (keyword) processing and acquisition.
\item Will be designed the way NCDDC system works (similar to FGDC clearinghouse).
\item Select by sub-setting.
\end{itemize}

\item MANIPULATION
\begin{itemize}
\item May want to see mean value.
\item Count by values (e.g., percent clear skies).
\item These characteristics would be nice, but not now. (Can be done
  by Java Applet).
\end{itemize}

\item DISPLAY
\begin{itemize}
\item Properties of data.
\item Graphics.
\item Relationship between variables.
\item Disclaimer information (constraints to using data).
\item Volume of data.
\end{itemize}

\item DOWNLOAD DATA
\begin{itemize}
\item Fully described data.
\item The data itself (in different formats).
\begin{itemize}
  \item ASCII
  \item Arc/INFO
  \item Matlab
  \item IDL
  \item NetCDF
  \item ADF
\end{itemize}
\end{itemize}

\item AUTOMATION
\end{itemize}

\section{How should DODS be implemented in the southeast?}

Anne explained that she was a firm believer in standards and that the
group now has an opportunity to use DODS as a standard to move data
around.  She encouraged all the participants to download servers from
Peter's web site and try them out and after that, think about where we
should go from here.  The goal is to establish a relationship in the
end.

Steve listed the major issues that he saw as being important:
\begin{enumerate}
\item What are you going to do about metadata, more or less, for DODS?
\item Links to other programs or groups.
\item Include outreach to others not present at the meeting.
\item System installation issues.
\item The need to develop a web browser interface on the client side of DODS.
\end{enumerate}

Based on these issues, the floor was opened up for suggestions/comments:
\begin{itemize}
  \item Installation nodes, should they be centralized or distributed?
  \item Time, money (resource availability).
  \item Time frame, sustaining after initial funding.
  \item Using DODS needs to be worked into everyday activities.
  \item Implementation and maintenance issues still need resolving.
  \item Security issues, PI's and system people need to be made more comfortable.
  \item The ``O'' in DODS was disconcerting for people not interested in oceanography.
  \item The data model may not be adequate for other fields.
  \item Data access protocol needs to be generalized.
  \item The more groups using the DODS data access protocol, and the
more tools that each community can share with each other, the better.
\end{itemize}

\section{CONCLUSIONS}

\subsubsection{Major Metadata Issues}

Listed below are the MAJOR metadata issues that the workshop participants saw as being the 
most important:
\begin{itemize}
\item Define/encourage some ``minimal'' set of semantic metadata.
\item Provide regional training/education.
\begin{itemize}
  \item Develop a white paper/web site.
  \item Done at an institutional level.
  \item Have a central training site.
\end{itemize}
\item Evaluate metadata requirements for ``locate'' function.
\end{itemize}

\subsubsection{Links To Other Groups}
\begin{itemize}
\item Link to FGDC. 
\begin{itemize}
\item Partnerships with others:
\begin{itemize}
\item GCMD (Global Change Master Directory)
\item FGDC
\item NCDDC
\item NAML (participate at national level)
\item Coastwatch
\item NSF/LABNET/``bubba net''
\item PMEL/NOAA Server.
\item Others...
\end{itemize}
\end{itemize}
\item FGDC clearinghouse link to DODS.
\item DODS core must go to FGDC clearinghouse metadata.
\item Other repositories of data should be identified and encouraged to use DODS.
\item Investigate links to regional education efforts.
\item Encourage use of DODS via outreach.
\item Help connect (training/facilitation).
\end{itemize}

\subsubsection{Inclusion of ``Those Not Here''}
\begin{itemize}
\item Timing of efforts:
\begin{itemize}
  \item Server/client availability.
  \item Pilot project/prototype.
\end{itemize}
\item Mechanisms to involve groups:
\begin{itemize}
  \item Define client needs of ``customers.''
  \item Link/leverage to current efforts.
\end{itemize}
\item Develop client?
\begin{itemize}
  \item Web client (NCDCC).
  \item Define web client needs (focus group).
  \item Fund development?
\end{itemize}
\end{itemize}

\subsubsection{System Installation}
\begin{itemize}
\item Security documentation:
\begin{itemize}
  \item Get definition/explanation.
  \item Disseminate to systems manager.
\end{itemize}
\item Centralized node vs. distributed node:
\begin{itemize}
  \item Can't happen at CSC.
  \item Who could/would be central node and/or repository?
\end{itemize}
\item Centralized ``help desk'' (exists at national level now):
\begin{itemize}
  \item Tutorial.
  \item Training-centralized.
  \item On-site training.
\end{itemize}
\end{itemize}

\subsubsection{Client Interface}
\begin{itemize}
\item Command line at core (national issue):
\begin{itemize}
  \item URL builder at user end.
  \item Define command line priority issues.
\end{itemize}
\item Define URL builder priority needs (comma delimited).
\item Decide during/after pilot project.
\item Create web client focus group. (HIGHEST PRIORITY).
\end{itemize}

\subsubsection{Resource Availability}
\begin{enumerate}
\item DODS/NOPP.
\item Partners.
\item Personal resources:
\begin{itemize}
  \item Money to build metadata (site-specific or centralized).
  \item Money to buy server computer (site specific).
  \item Money for systems administration and assistance (regionalized).
  \item Money for server installation and assistance (regionalized).
\end{itemize}
\end{enumerate}

\subsubsection{Implementation Plan}
\begin{itemize}
\item Feedback loop to DODS/users (what went smoothly, what were the ``gotchas'').
\item E-mail list that ties into status report:
\begin{itemize}
  \item It should have an archive function.
  \item Use DODS node to host list.
\end{itemize}
\item Pilot project.
\item Regional web page.
\item Unidata has a tracking mechanism.
\end{itemize}

\subsubsection{Program Sustainability}
\begin{itemize}
\item Incorporation into day-to-day activities.
\item Maintain partnerships:
\begin{itemize}
  \item Which are the most useful or valuable.
\end{itemize}
\item Status/update reports.
\end{itemize}


\section{Workshop Summary}

\subsubsection{General}

\begin{itemize}
\item DODS must work interactively with the FGDC Clearinghouse architecture (Z39.50) and 
metadata standard.
\item Needs to have a stronger front end for locating data.
\item Need servers for relational databases, Excel, Access, SAS, S+.
\item Needs the ability to show a disclaimer BEFORE a user accesses data.
\item Needs interface for both sophisticated and unsophisticated users, local \& global customer.
\item Auto generation of DODS URLs.
\end{itemize}

\subsubsection{Support}

\begin{itemize}
\item Servers must be easy to install.
\item Provide a help desk for answering questions on installing
  servers and developing clients.
\item High quality, easy to understand documentation.
\end{itemize}

\subsubsection{Security issues}

\begin{itemize}
\item Document known httpd security issues and make them available to DODS server 
implementers.
\item Ensure DODS software is secure.
\item Provide a means of restricting access to select data through
  password protection or IP exclusion and provide documentation on how
  to implement this. (This is already available in DODS release 3.2.)
\end{itemize}

Although DODS is middle-ware and does not attempt to provide data
location and analysis functions, workshop attendees felt that DODS was
not terribly useful to them without these.  In order to provide these
functions better, DODS should partner with other organizations.
Potential partners include:

\begin{itemize}
\item Federal Geographic Data Committee (FGDC).
\item Global Change Master Directory (GCMD).
\item National Association of Marine Laboratories (NAML) LabNet project.
\item NOAA National Coastal Data Development Center (NCDDC).
\end{itemize}

\subsubsection{Suggestions for years 2 and 3}


\begin{itemize}
\item Support for installing server.
\item Web browser and direct download for clients.
\item Create a mailing list for SE participants.        
\item The DODS proposal to the National Ocean Partnership Program,
  roughly 60K would be available to the southeast for support.  A
  suggestion for using these funds would be to hire one full-time
  programmer to perform the following tasks:
\begin{itemize}
\item Build a web site with information pertinent to DODS users in the
  southeast.
\item Provide help desk support.
\item Software development (support for web client and DODS URL builder in 
coordination with DODS development team).
\item Coordinate with FGDC.
\item Explore metadata to DODS relationships.
\end{itemize}
\end{itemize}



\section{Additional Information}

The use of persistent cookies is not allowed on federal government
systems.  The Department of Commerce's implementation of this policy
can be found at: 

\Url{http://www.doc.gov/webresources/CookiesPolicy.html} 

It should be noted that a number of agencies, including NOAA, have
reacted to this policy by not allowing the use of ANY cookies.

Several workshop attendees reported that they must include disclaimers
\emph{BEFORE} users can access their data.  A sample disclaimer follows:

\begin{quote}
DISCLAIMER: THE DATA AND ASSOCIATED DATA FILES FOUND USING THIS
SOFTWARE ARE PROVIDED ``AS IS,'' WITHOUT WARRANTY TO THEIR PERFORMANCE,
MERCHANTABLE STATE, OR FITNESS FOR ANY PARTICULAR PURPOSE. THE ENTIRE
RISK ASSOCIATED WITH THE RESULTS AND PERFORMANCES OF THIS SOFTWARE IS
ASSUMED BY THE USER.
\end{quote}

The executive order for the National Spatial Data Infrastructure which
is carried out by the FGDC is available at the following URL:

\Url{http://www.fgdc.gov/publications/documents/geninfo/execord.html}



\section{Attendees}

\begin{center}
\begin{tabular}{lp{1.5in}ll} \\
\textbf{Name} & \textbf{Organization} & \textbf{Phone} & \textbf{Email} \\
Anne Ball & NOAA/CSC - SE Regional Coordinator & (843) 740-1229 & Anne.Ball@noaa.gov \\
Steve Bliven & Facilitator & (508) 997-3826 & bliven@massed.net \\
Stephen Brueske & NOAA NWS & (843) 744-1732 & Stephen.Brueske@noaa.gov \\
Peter Cornillon & URI/NOPP/DODS & (401) 874-6283 &pcornillon@gso.uri.edu \\
John Ellis & NOAA/ NCDDC & (228) 688-4090 & john.ellis@nrlssc.navy.mil \\
Madeline Fletcher & Baruch Institute & (803) 777-5288 & fletcher@sc.edu \\
Chris Friel & Florida Marine Research Institute (FMRI) & (727) 896-8626 & Chris.Friel@fwc.state.fl.us \\
Jim Fritz & Rapporteur - TPMC & (781) 545-1346 & jfritz@tpmc.com \\
James Frysinger & College of Charleston & (843) 225-0805 & frysingerj@cofc.edu \\
Beth Judge & SC Sea Grant & (843) 727-2078 & ekjudge@clemson.edu \\
Alan Lewitus & Baruch Institute \& SC DNR & (843) 762-5415 & lewitus@belle.baruch.sc.edu \\
Andrew Meredith & NOAA/CSC Coastal Remote Sensing & (843) 740-1291 & Andrew.Meredith@noaa.gov \\
Jim Nelson & UNC Marine Sciences & (912) 598-2473 & nelson@skio.peachnet.edu \\
Dwayne Porter & Baruch Institute & (803) 777-4615 & porter@sc.edu \\
Reyna Sabina & NOAA/AOML & (305) 361-4324 & Reyna.Sabina@noaa.gov \\
George Shirey & NOAA/CSC - Metadata & (843) 740-1205 & George.Shirey@noaa.gov \\
Tim Snoots & SC Department of Natural Resources (DNR) & (843) 762-5651 & snootst@mrd.dnr.state.sc.us \\
John Ulmer & NOAA/CSC - Systems & (843) 740-1228 & John.Ulmer@noaa.gov \\
\end{tabular}
\end{center}


                


%%% Local Variables: 
%%% mode: latex
%%% TeX-master: t
%%% End: 

\renewcommand{\chaptertitle}{Northeast Regional Workshop}
\chapter{\texorhtml{}{Appendix D }\chaptertitle}

%% $Id$

\begin{center}
January 8th - 10th, 2001\\
URI - W. Alton Jones Campus\\
West Greenwich, RI\\
\end{center}

\section{Introduction and Workshop Objectives}

The meeting convened with Linda Mercer, NE regional DODS coordinator,
asking the audience to introduce themselves (see \sectionref{IV,attendees}).  An
overhead was displayed showing the following questions that the
workshop attendees were to address:

\begin{enumerate}
\item Is there an area(s) of common interest in the Northeast?
\item What datasets will be served via the system?
\item Is the DODS data model adequate for the datasets to be served in the NE?
\item What are the important interface issues for users in the region?
\item What types of semantic metadata will be required and what
  semantic standards will be used?
\item Is a central regional node required to provide coordination
  including services such as user support, data location, etc.?
\end{enumerate}

\section{Overview and Demonstration of DODS}

Dan Holloway, URI DODS technical lead, went to the DODS demonstration
website and proceeded to show a series of web pages describing DODS.
A complete overview of the DODS data model can be found at:
http://po.gso.uri.edu/~dan/dods-regional-
workshops/dods-regional-workshops.html.

Dan explained the five steps involved to analyze data:
\begin{enumerate}
\item LOCATE
\item SUBSET
\item ACQUIRE
\item INGEST
\item ANALYZE
\end{enumerate}

The DODS portion involves the middle three steps, i.e., SUBSETTING, ACQUIRING, 
and INGESTING.  DODS does not focus on LOCATION or ANALYSIS.  What DODS 
is not, and the underlying philosophy of DODS, were covered.

Dan went on to explain that DODS is built from the bottom up, i.e., by
putting high functionality at the data acquisition level, and that
functionality decreases from the bottom up (the data level being at
the bottom, the inventory level in the middle and the directory level
at the top).

The underlying philosophy of DODS was reiterated and is twofold:
\begin{itemize}
\item Anyone willing to share their data should be able to do so via DODS.
\item The user should be able to use the application package with
  which she or he is the most familiar, to examine or analyze the data
  of interest.
\end{itemize}

Constraints of the DODS system include making it easy for the
scientist to serve data AND to make it easy to access data.

Syntactic and semantic metadata were described and how they relate to
the data, inventory, and directory levels.  Semantic metadata can be
sub-setted into use and search categories.  Use metadata can be
further subdivided into translational use metadata and descriptive use
metadata. The focus of DODS is on translational use metadata.

Four levels of interoperability at the data level (for data accessible
over the network) were explained as follows:
\begin{description}
\item[Level 0]  no syntactic or semantic metadata - FTP.
\item[Level 1]  rigid syntactic, no semantic metadata - DODS.
\item[Level 2] rigid syntactic, human readable semantic metadata - A
  subset of DODS datasets.
\item[Level 3] rigid syntactic, consistent semantic metadata; i.e.,
  machine-readable - A subset of the DODS Level 2 datasets.
\end{description}

Three DODS data objects were defined:
\begin{enumerate}
\item The data descriptor structure (DDS) - the syntactic metadata for a dataset.
\item The data attribute structure (DAS) - the semantic metadata for a dataset.
\item Data - the actual data in binary structure.
\end{enumerate}

Additionally, DODS servers support several other services: 
\begin{itemize}
\item .ASCII (an ASCII representation of the data).
\item .info (a more readable version of the .dds and .das combined).
\item .html form (a web-based form that will help to build a DODS URL).
\end{itemize}

It was emphasized that writing DODS URLS can be difficult!

Three classes of interfaces were described based on the difficulty
involved in building a URL.  These are in order of increasing
ease-of-use:

\begin{description}
\item[Command Line] No help is given creating a URL.
\item[General Purpose URL Builder] A menu-based interface based on
  data dimensions and names.
\item[Graphical User Interface] A URL builder based on geophysical
  parameter input.
\end{description}

The DODS data model was explained along with the data model
constraints.  The following operations are permitted when requesting
data:
\begin{itemize}
\item Projection.
\item Relational operations on list and sequence elements.
\item Rectangular and decimated subsets of arrays and grids.
\end{itemize}

And groupings of these data types:
\begin{itemize}
\item Array
\item Structure
\item List
\item Sequence 
\item Grid
\end{itemize}

Sequences and Grids are two separate constructs.

Data types served in the NE: 
\begin{itemize}
\item Hydrography
\item Moorings
\item Bathymetry
\item Model output
\item Satellite products
\end{itemize}

Global:
\begin{itemize}
\item Surface fluxes
\item Water column climatologies
\item Satellite products
\item COADS climatologies
\item Near real-time numerical model weather and ocean predictions
\end{itemize}

Dan demonstrated the methods of accessing data served by DODS.

An example of the Live Access Server (LAS) was given.
Steps involved:
\begin{itemize}
\item Browser requests GIF image from PMEL site.
\item PMEL Live Access Server passes request to Ferret.
\item Ferret requests subset of monthly COADS SST data from local disk.
\item Ferret requests subset of monthly NCEP Marine data from CDC/NOAA in Boulder.
\item Ferret re-grids COADS to NCEP grid, differences, generates a GIF
  image and returns the GIF image to Live Access Server.
\item Live Access Server returns GIF image to the browser.
\end{itemize}

Another example used the DODS Matlab Graphical User Interface to
select, request, adn download data directly into the Matlab workspace.

Dan gave an example using PMEL data and another utilizing Matlab.

\section{Workshop Participant Presentations}

A good portion of the second day of the workshop was devoted to
workshop participants giving presentations related to their
organization and how they tie into the DODS concept. To set a context
for the DODS discussion, David Mountain (NMFS, Woods Hole) gave an
overview of a NMFS proposal for ``Ocean Observing Systems in the Gulf
of Maine - A proposal to Integrate Existing Systems,'' presented to
directors from marine organizations/agencies in the Northeast region,
October 30, 2000, sponsored by the Regional Association for Research
in the Gulf of Maine (RARGOM). The purpose of the meeting was to
determine interest in developing an integrated system, comprised of
monitoring data from existing programs that will generate analytic
products in order to meet a broad spectrum of needs across the Gulf of
Maine region.

\subsection{Where Are we Now?}

Many Operational Observation Programs:
\begin{itemize}
\item National Weather Service
\item National Marine Fisheries Service
\item National Ocean Service
\item Dept of Fisheries and Oceans (Canada)
\item Division of Marine Fisheries (MA)
\end{itemize}

Where are we now?
\begin{itemize}
\item Many datasets available via the web.
\item Little coordination in the collection or analysis of the datasets.
\end{itemize}

Why Now?
\begin{itemize}
\item Improved technology.
\item Increased understanding of the Gulf Ecosystem.
\item From major research EOCHAB, GLOBEC, etc.
\item The future is now.
\end{itemize}

The Future is Now.
\begin{itemize}
\item An Integrated Ocean Observing System: A Strategy for
  Implementing the First Steps of a US Plan.
\item A report (R.Fresch, chair).
\end{itemize}

Recommendations:
\begin{enumerate}
\item Use and enhance existing observational systems.
\item Integrate into regional demo programs.
\end{enumerate}

System Components:
\begin{enumerate}
\item Measurement.
\item Data storage.
\item Data communication.
\item Analysis.
\item Integration.
\end{enumerate}

DODS can be the critical data communication component for a
distributed, integrated observing system in the northeast.


\subsection{Linda Mercer - Maine Department of Marine Fisheries}

The Maine Department of Marine Resources has numerous datasets, many
of which include physical parameters such as temperature and salinity,
that it would be willing to share via DODS.

Examples include:

\subsubsection{Environmental Monitoring Program }

Year started: 1989

Data collection site: BBH Station

Collection frequency: hourly

\begin{itemize}
\item Barometric Pressure (mb) (in)
\item Precipitation (in)
\item Relative Humidity (%)
\item Salinity (ppt)    
\item Sea Surface Temperature (c,f) 1905
\item Sea Bottom Temperature (c,f) 
\item Solar Radiation (langley/min)
\item Air Temperature (c,f)
\item Tide Height (ft)
\item Wind Direction (deg) 
\item Wind Speed (mph)
\end{itemize}

\subsubsection{Aquaculture Program}

Year started: 1988

Data collection site: finfish aquaculture sites

Collection frequency: baseline at time of application then yearly 

\begin{itemize}
\item Current speed (m/sec)
\item Dissolved Oxygen Depth Profile (%, mg/l)
\item Salinity Depth Profile (ppt)
\item Sea Temperature Depth Profile (c)
\item pH Depth Profile  (pH/m)
\end{itemize}

\subsubsection{Stock Enhancement Program}

Year started: 1985

Data collection site: coastal Maine Rivers

Collection frequency: depends on field schedule of various projects

\begin{itemize}
\item Air Temperature (c,f)
\item Dissolved Oxygen (Surface/Mid/Bottom) (mg/l)
\item Salinity (Surface/Mid/Bottom) (ppt)
\item Temperature (Surface/Mid/Bottom) (c,f)
\item Water Depth (m)
\item Secchi Disk (cm)
\item Weather Description (qualitative)
\end{itemize}

\subsubsection{Water Quality Program}

Year started: 1989

Data collection site: 1800 rain \& sample sites along coastal Maine

Collection frequency: 6/yr (min), March- November

\begin{itemize}
\item Salinity (ppt)
\item Water Temperature (c,f)
\item Tide Status (high/med/low)
\item Weather Description (qualitative)
\item Precipitation (in)
\item Dissolved Oxygen (mg/l)
\end{itemize}

\subsubsection{Survey - CDT}

Year started: 1989

Data collection site: various vessel surveys along Maine coast

Collection frequency: survey schedule

\begin{itemize}
\item Water Temperature Depth Profile (c/m)
\item Salinity (conductivity) Depth Profile (ppt/m)     
\end{itemize}


\subsection{Linda Mercer - Gulf of Maine Ocean Observing System}

The Gulf of Maine Ocean Observing System (GoMOOS) is a NOPP-funded
project that is developing a system of buoys and CODAR to collect
oceanographic and meteorological data in the Gulf of Maine and make
these data available on continuous real-time basis.  GoMOOS is a pilot
component of the Northeast Ocean Observing System (NEOOS) that is, in
turn, planned as one of approximately six regional observing systems
envisioned to constitute a national observing system.  

GoMOOS is a consortium of scientists, policymakers, and industry, and
the observing system is designed, not as a research project, but as a
utility that will build, deploy, operate, transmit/process/archive
data, and maintain the infrastructure required to do this.  The data
and information produced will allow those who depend on the Gulf of
Maine for their livelihood and well- being, and those whose business
is marine research, to undertake their pursuits and enhance the
understanding of the Gulf more efficiently and profitably than ever
before.

The initial phase of GoMOOS will consist of 13 buoys placed
strategically around the Gulf of Maine based on input from a variety
of user groups, and a CODAR system with four sites that will provide
coverage out to 200 km.

\subsubsection{GoMOOS Data Products Summary}

\begin{itemize}

\item Wave products:
\begin{itemize}
\item Height
\item Trends
\item Nowcasts
\item Forecasts
\end{itemize}

\item Meteorological.observations:
\begin{itemize}
\item Wind (hourly)
\item Wind (4 times daily)
\item Fog
\end{itemize}


\item Currents:
\begin{itemize}
\item Surface currents maps
\item Water column currents
\item Large scale circulation features
\item Interannual variation
\end{itemize}


\item Temperature Patterns:
\begin{itemize}
\item Sea surface temperature
\item SST composites
\item Archives
\item Water column temperature
\end{itemize}

\item Salinity

\item Dissolved Oxygen

\item Primary Productivity

\item Turbidity

\item Whale Sounds
\end{itemize}

Direct transfer of the data to scientists from GoMOOS can be
effectively handled via DODS. A web-based graphical User Interface
will be used to provide ``user-friendly'' data and data products that
address the information requirements of Gulf user groups.


\subsection{Nicholas Wolff -  Bigelow Laboratory for Ocean Sciences}

He mentioned that he is here to get more familiar with what DODS is
and to share that with researchers at Bigelow.  Nick works with Lew
Incze who previously expressed an interest in DODS for serving several
of his larval lobster time series with associated hydrographic data.


\subsection{David Mountain - National Marine Fisheries Service}

NMFS datasets for DODS

\begin{tabular}[t]{ll}
Now:&   Historic current meter records (NetCDF). \\
Future: & Hydrographic  (~1500 CTD casts/yr) \\
&       Zooplankton     (~ 700 longo tows/yr) \\
&       CPR             (monthly CPR transects) \\
&       Fish            (monthly trawls/yr) \\
\end{tabular}

\subsubsection{Problems/Needs}

Serving data - data in ORACLE

Retrieving data - plug-ins for ORACLE, S-Plus, Java?

David pointed out that the intention of NMFS is to serve all their
data via DODS.  They currently have some Matlab and IDL users.

\subsection{Bruce Tripp - Woods Hole Oceanographic Institute}

Data to share:
Data sources at WHOI would be potential candidates for DODS.

ECOHAB data:
\begin{itemize}
\item PI's sharing data and a website.
\item Buoy data.
\end{itemize}

Multiple sources for Gulf of Maine data.

Historic aspect of data needs to be explored.

Contaminated sediment database from the Regional Marine Research
Program in Gulf of Maine.

\begin{itemize}
\item Had multiple sources of data.
\item Difficulty in critiquing the data to determine quality.
\end{itemize}

There is a new ocean observatory, the Martha's Vineyard Coastal Ocean
Observatory headed by Jim Edson.  He has created a multi-observatory
website (draft version at this time).  However, the website probably
won't be used to house data.  There is collaboration between five
coastal observatories on the East Coast.  They could potentially house
a DODS server and share data.  Sediment transport and meteorology is
the primary focus of the observatory groups.


\subsection{Chuck Denham - USGS, Woods Hole}

Data to share:
\begin{itemize}
\item Current meter data.
\item Bathymetric data.
\item Data is fairly distributed but have to know where to get it.
\item Geographical range, Woods Hole (Gulf of ME, NY and NJ).
\item EPIC formatted data for ADCP.  Data descriptors are needed for instruments.
\end{itemize}

\subsection{Steve Hale - EPA Monitoring Assessment Program, EMAP}

MAIA - Estuary Report

Why Share Estuarine Data?
\begin{itemize}
\item Broad-scale and long-term ecological processes (e.g.,
  eutrophication, global change).
\item No single group can collect all the data. 
\end{itemize}

Impediments to data sharing.

Technology:
\begin{itemize}
\item Data formats
\item Hardware
\end{itemize}

Sociological.

How to exchange data:
\begin{itemize}
\item Directories/catalogs/clearinghouses
\item Standards (Z39.50)
\item Publishing on the web
\end{itemize}

How to integrate:

\begin{itemize}

\item Standards

\item Metadata
\begin{itemize}
\item Information on methods
\item Information on data quality
\end{itemize}

\item Databases:
\begin{itemize}
\item Core DBMS
\item Database constellation
\item Centralized metadata, distributed data
\item GIS
\end{itemize}
\end{itemize}

Coastal Zone 2000 Monitoring in the Northeast (map on display).

Showed metadata standard overhead.

EMAP Coastal zone works off an ORACLE database.

Bibliography on-line.

\subsection{Don Byrne - NJ Division of Fish, Game and Wildlife}

See \sectionref{IV,NJ}

New Jersey has a trawl survey and is willing to share data from this
survey that dates from 1988 to the present using DODS.  New Jersey
data are not currently available on a server.


\subsection{Wendell Brown - Univ. of Mass at Dartmouth}

Built an archive at the University of New Hampshire, with metadata.
Physical oceanographic data.

Advanced Fisheries Management Information System (AFMIS)
Professors Brian Rothchild and Allan Robinson were major contributors.

Employs simple ecosystem models.

Components:
\begin{enumerate}
\item Sensors and platforms.
\item Data and knowledge.
\item Models with data assimilation.
\item Reports distributed to users.
\end{enumerate}

Focus is on fisheries but can be used in other resource management functions.

Showed AFMIS functional elements.

Real-time forecast demonstration of concept (predicting where fish would be).

List of forecast products:
\begin{itemize}
\item Discussion of model initialization
\item Discussion of forecast products
\item Synoptic forecast images
  \begin{itemize}
  \item Georges Bank plan view
  \item Georges Bank vertical resources
  \end{itemize}
\item Tidal Velocity Forecast images
\end{itemize}

\section{More DODS}

Paul Hemenway of URI, with the use of a flip chart, provided a
graphical representation of how DODS works.  He stated that it takes
some effort to put available datasets up on DODS.  DODS servers are
specific to the data format you are serving (NetCDF, HDF, binary,
ascii).  The DODS client is specific to the analysis package/language
you are using (Matlab, IDL, EXCEL, etc.).

Dan Holloway explained that DODS has two ASCII servers: FreeForm
(handles binary very well and ASCII fairly well) and JGOFS (uses tree
method object).  He discussed how JGOFS data is accessed through DODS.
Dan pointed out that data location is an issue, and quality of data is
also important.  Bandwidth of data was also discussed.  Finally, data
need not be replicated (when considering setting up a separate
server).


\subsection{DODS Workshop Questions}

Listed below are some of the more important regional questions and
answers pertaining to DODS that were presented to the workshop
participants:

What would it take to be a DODS server (basic information)?

An informational listing of items necessary to set up a server was
thought to be very important. Maintenance issues should be among the
topics covered. The DODS technical staff at URI could provide the
informational listing and any support required.

What datasets will be served in the NE?

Most of the datasets discussed at the workshop were from the northern
part of the Northeast region, i.e. Gulf of Maine and Georges Bank.
More participation is needed from the mid-Atlantic regions.

Datasets in the Northeast:
\begin{itemize}
\item Fisheries databases
\item GOMOOS and other observing systems
\item ECOHAB
\item AFMIS
\item USGS - WHOI and MWRA
\item NOS tidal gauge
\item NOAA status and trends
\item Coastal 2000
\item REDIMS contaminants
\item Fish data
\item Shellfish
\item Temperature data
\item Climatology
\item Shoreline data collected by USGS (with focus on monitoring
  before and after storms)
\item Sediment transport data
\end{itemize}

Any modeling activities (data assimilation possibilities) by Lynch, Huije, GLOBEC.

Talking to the modeling community will be important!

Who is serving DODS datasets?  

\begin{itemize}
\item PMEL
\item UCAR
\item URI
\item USGS
\item NGDC
\item Goddard
\item Lamar
\item BIO
\item Mass Bay Modeling
\end{itemize}

Number of DODS client sites (unknown); can download DODS clients and servers.

Is the DODS data model adequate for the datasets to be served in the NE?

It was emphasized that data residing in DBMS format be accessible via DODS.

The need for a data catalog(s) and the development of a template for
the catalog was stressed:
\begin{itemize}
\item    Web-based.
\item Would be compiled into a regional, centralized site.
\item Searchable by keyword.
\item Geographic.
\item Should be able to serve up .gif and .jpeg images.
\item Show links to informational pages describing the datasets.
\item Would include pointers to URL's for the actual data.
\end{itemize}

What are important interface issues for users in the region?

Being able to view generated maps and graphics would be useful.
Presently, DODS does not support data in ACCESS format.

What types of semantic metadata will be required and what semantic
metadata standards will be used?

COARDS is the minimum standard so far.  Data providers are encouraged
to submit a maximum amount but DODS only requires a minimum amount of
syntactic metadata and no semantic metadata to serve datasets.

Is a central regional node required to provide coordination including
services such as user support, data location, etc.?

Yes, and the regional node should do the following:
\begin{itemize}
\item Provide user support until DODS users get up and running.
  Someone within each region, probably a person from an existing
  organization, should be chosen.
\item Setting up a DODS regional work group will be helpful.
\item Other groups that were not in attendance at the regional
  workshop should be included (CT, NY, DE).
\item Perhaps take DODS on the road to demonstrate its capabilities to
  others not at the regional workshop.
\item An outreach component is needed.
\item Get data online and document difficulty in using DODS.
\item Have a contractor do a demo for other states and regional players.
\end{itemize}

\section{Regional Pilot Project and Demonstration}

From the presentations and group discussion it appears that DODS has
focused (successfully) on large datasets and sophisticated users in
the region.  The group felt that many good data produced by resource
management agencies and research projects are yet to be found in small
data sets housed on a variety of platforms and in a myriad of
formats.  Wider access to these data is desirable but it is assumed
that the task of making them DODS accessible is a large task.  What
actually would it cost an organization or agency in new software,
hardware, and skills --- infrastructure as well as actual dollar
expenses --- to bring a data base on line in DODS?  A collaborative
demonstration project between the NMFS and state fisheries agencies of
ME, MA and NJ was discussed as one means to address this question and
to guide others in the process. Simultaneously, this demo would guide
the NOPP project in the selection of new DODS-supplied data tools that would be
broadly useful to the community.

The four agencies would have to assemble all fisheries data
(documenting QA/QC, formats, software used, etc.)  This task could be
large. DODS would support the task to bring all state fish data on
line (perhaps initially via a single server at NMFS) It is expected
that partial QA/QC and different sampling methods might make inter-
comparison difficult.  Discussion and resolution of these issues would
help to create a valuable regional dataset from the current collection
of disconnected fragments.  A full-time person, knowledgeable in
fisheries and fieldwork, perhaps based at URI, would be needed to
interact with state fishery staff to assist with in-state data
conversion/formatting as an initial step.

Other groups of smaller datasets are held by marine labs and
individual research projects that might also be assembled in a
parallel DODS demonstration to assess datasets collected for different
purposes and how to move them into the DODS system.

The demonstration pilot project should address the following:
(Ex: using fisheries data from NMFS, ME, MA, or NJ)
\begin{enumerate}
\item Creating one or more DODS servers for NMFS, ME, NH, and NJ.
  \begin{itemize}
  \item Develop a GUI interface that has higher resolution for NE
    (similar to Matlab).
  \item It may need to be modified for non-Matlab users.
  \end{itemize}
\item Define what being a DODS client is for each agency (ex: EXCEL).
\item Develop a web-based graphical demonstration using data from
  different sources.
\item Provide an HTML template listing regional datasets (a simple
  catalog).  The template should be able to get data from multiple
  remote sources and bring back to a web-site for viewing.
\item The demonstration project should include password protection.
\end{enumerate}

Other issues concerning the demonstration project:
\begin{enumerate}
\item Who will create the demo was discussed with no resolution yet.
\item A technical person and data person would be needed to do the demo.
\item Have the demo up and running before the national meeting.
\item The group discussed the other possibilities for demonstration
  projects such as using contaminant data created by the USGS and MWRA
  or the northeast observing systems including the new Martha's
  Vineyard Coastal Observatory.
\end{enumerate}

\section{Final Comments/Issues}

\begin{enumerate}
\item Decisions must be made on how much metadata needs to be
  attached to datasets.  Currently, regional datasets may not have any
  metadata associated with them.  Minimal semantic metadata standards
  need to be developed.
\item The willingness of other players in the region to serve up data
  via DODS needs to be determined.
\item Natural resource agencies may not have adequate data management
  resources to implement DODS.  There will need to be considerable
  ``hand holding'' provided to less technically sophisticated users.
  There will likely be different issues for different user groups.
\item More client server development is needed.
\item The need for a PowerPoint quick demonstration of DODS.
\item Model outputs would be a useful data type served by DODS.
\item There was a desire to have biological data (fish data) served by DODS.
\end{enumerate}

\section{Attendees}
\label{IV,attendees}

NOPP/DODS NE Regional Workshop
W. Alton Jones Campus, URI
January 8 - 10, 2001

\begin{center}
  \begin{tabular}[t]{lp{2.0in}l} \\ 
\textbf{Name} & \textbf{Organization} & \textbf{Email} \\
Wendell Brown &                 Umass Dartmouth & wbrown@umass.edu \\
Frank Bub &                     School for Marine Science \& Technology - SMST of UMass Dartmouth & fbub@umassd.edu \\
Don Byrne &                     NJ Division of Fish \& Wildlife& njfgwbyrne@plexi.com \\
Chuck Denham &                  USGS Woods Hole &       cdenham@usgs.gov \\
John Evans &                    Univ. of Maine          &       jevans@umeoce.maine.edu \\
John Fracassi &                 Rutgers University      &       johnf@arctic.rutgers.edu \\
Jim Fritz &                     TPMC                    &       jfritz@tpmc.com \\
Steve Hale &                    EPA - Narragansett Lab &         hale.steve@epa.gov \\
Paul Hemenway &         Univ. of R.I.           &       phemenway@gso.uri.edu \\
Dan Holloway &                  Univ. of R.I./DODS &            d.holloway@gso.uri.edu \\
Linda Mercer &                  Maine Dept. of Marine Resources & linda.mercer@state.me.us \\
David Mountain &                NMFS Woods Hole, MA & david.mountain@noaa.gov \\
David Remsen &                  Marine Biological Lab   &       dremsen@mbl.edu \\
Bruce Tripp &                   Woods Hole Oceanographic Institute & btripp@whoi.edu \\
Nicholas Wolff &                        Bigelow Lab             &       nwolff@bigelow.org \\
  \end{tabular}
\end{center}

\section{Ocean Monitoring Programs Conducted by
The New Jersey Department of Environmental Protection}
\label{IV,NJ}

\begin{center}
  \begin{tabular}[t]{|p{.75in}|p{0.5in}|p{1.0in}|p{0.5in}|p{0.5in}|p{0.5in}|p{0.75in}|} \hline
\textbf{Agency} & \textbf{Program} & \textbf{Variables} & \textbf{Samp.\-/Yr.} & \textbf{Svys\-/Yr.} & \textbf{From} & \textbf{Offshore Limit} \\ \hline
Bureau of Marine Fisheries & 
Ground\-fish      & 
Fish \& invert. no., wt., length; temp.,sal., DO (TSO), sea state, weather &
\~{}200 &
5 (\~{}bi-monthly)&
1988 &  
90 ft. isobath \\ \hline
Bureau of Shellfisheries &
Surf Clam &
Surf clam vol., no., length; (TSO) &
\~{}350 &
1 (summer) & 
1988 & 
3 n.m. \\ \hline
Bureau of Marine Water Monitoring &
Bacteria &           
Conc. of fecal coliform bacteria &
\~{}200 &
5-8 &
1976 &
3 n.m. (state waters) \\ \hline
Bureau of Marine Water Monitoring &
Nutrients &
Conc. of total suspended solids, NH3, NO3, NO2, PO4, total N, TSO &
\~{}140 &
4 (quarterly) &
1989 &
3 n.m. (state waters) \\ \hline
Bureau of Marine Water Monitoring &
Phyto\-plankton &
Phyto. Species, cell conc., TSO &
5 &
1, summer &
1985 &
3 n.m. (state waters) \\ \hline
  \end{tabular}
\end{center}



%%% Local Variables: 
%%% mode: latex
%%% TeX-master: t
%%% End: 

\renewcommand{\chaptertitle}{Technical Interchange Meeting with %
Environmental Systems Research Institute}
\chapter{\texorhtml{}{Appendix E }\chaptertitle}
\label{app,esri}

%% $Id$

\begin{center}
November 7 \& 8, 2000\\
Environmental Systems Research Institute\\
New York Street\\
Redlands CA
\end{center}


\section{Meeting Summary}


\subsubsection{To facilitate data exchange technology between GIS users and scientific users} 

DODS has been a supporting technology to scientists for several years. It has provided an 
extensible and reliable architecture to share oceanographic and atmospheric data sets of spatial 
and temporal types, using Internet connectivity, and desktop applications common to research 
scientists. DODS users recognize that a significant volume of relevant spatial data exists among 
the GIS community. Understanding how common GIS store, document, read, and distribute data 
is essential to expanding DODS functionality and providing research scientist effective and 
timely tools. DODS users also recognize that non-scientists among the GIS community could 
greatly benefit from, access to the existing and future DODS technology, data and understanding 
of ocean scientists' needs.


\subsection{Main Points}

 \subsubsection{To identify and understand the commonalities in the technical design and development of 
ESRI products and DODS software/infrastructure}

The DODS community and ESRI have both designed and developed information architecture for 
distributing and accessing data from multiple hosts. Technical overviews and demonstrations of 
both approaches were given. Both systems share many similarities in their overall goal of 
distributed data publishing and access. However, these systems differ considerably in their 
technical implementation. Discussion also centered on supported data types and how to handle 
the advanced types that DODS partners need using ESRI tools. The common use of XML by 
both architectures may become a very important overlap as both designs mature. Developing 
tools to enhance data transfer across between both architectures 

\subsubsection{To assist ESRI in understanding ocean scientist user requirements}

ESRI has a long term business interest in continuing to expand its marine sector. Through 
periodic meetings of this type, ESRI hopes to better support their marine/scientist clients with 
expanded data models, distribution systems, and desktop tools. ESRI is also interested in 
providing meeting-planning support to their marine/scientist clients during the annual ESRI user 
conference. DODS partners gave detailed presentations and demonstrations of existing DODS 
projects, their requirements, and the main technical hurdles that DODS developers encounter. 
ESRI developers were especially interested in how ocean scientists used and modeled volume 
data types.



\subsubsection{To assist DODS designers and developers in understanding current GIS technology}

Senior software developers from ESRI's ArcIMS, SDE and Metadata groups provided detailed 
technical overviews and answered questions from DODS developers. Demonstrations of all of 
ESRI's major tools were provided. New ESRI software that is still in development was also 
discussed. Topic areas that frequently came up included storage of grid and volume data types, 
ESRI client needs when conducting data discovery, and the mechanics of extending ArcIMS 
with a DODS ``connector''. DODS developers specially note that more technical documentation 
is needed to describe the basic file-metadata requirements that an ESRI client needs to access a 
DODS data stream in order to provide those metadata from DODS Servers.


\subsubsection{To lay the groundwork for year 2 and 3 of the project}

This meeting was the first for the beginning of a three-year effort. Specific technical tasks were 
selected and discussed for development in year one. There was general discussion on how this 
meeting corresponds to other regional NOPP/DODS meetings and the national NOPP/DODS 
meeting. The outcome of these meetings as well as the results of year one development will 
likely direct year two and three plans. Project activity in year two and three will be increased.

In year one, DODS partners will build a DODS-ArcIMS connector that will provide DODS 
clients access to GIS data from an ArcIMS configuration. This will provide the background for 
DODS developers to build a mechanism that ESRI clients can use to access the DODS 
architecture in the future. DODS partners will begin working with the ArcIMS SDK to develop a 
connectivity mechanism for GIS clients to DODS servers.


\subsection{General Observations}

ESRI decided not to support the Z39.50 protocol for metadata. Metadata will be stored in an 
XML format within an RDBMS/SDE and will be accessible through full text searches submitted 
using ArcCatalog and published with ArcIMS.

The Geo-statistical Analyst has great potential for scientific use. When it can point to a data 
source located on a RDBMS/SDE its utility will be even more significant.


ESRI has a custom utility for loading grids into an RDBMS/SDE, however there are some 
combinations of RDBMS/SDE and ArcIMS that currently do not work together.

ArcIMS has some basic data access problems with heterogeneous network environments. These 
may be solved if the industry problems with the UFS are resolved. 

ESRI does not have a data type for volume or time-series data, however they are interested in 
building a capability to work with these data types.

XML is a fundamental part to data communication for ESRI web applications and may be part of 
the next major change in the design of DODS.

ArcIMS is extensible using standard languages such as C++ and JAVA. JAVA Beans will be 
provided that will be able to wrap COM based extensions.

Informix and DB2 have a more object oriented design to their spatial component compared to 
SDE on those RDBM's, and has therefore a thinner layer.





\subsection{November 7}


Introduction by Richard Lawrence

Meeting goals
\begin{itemize}
\item   Forum for DODS
\item   Discuss and form vision
\item   Technical overview
\item   Discuss commonalities
\end{itemize}

Meeting introductions
\begin{center}
\begin{tabular}{llp{2.5in}} \\ 
Richard Lawrence &      ESRI &          Professional Services \\
Jeanne Rebstock &       ESRI &          Environmental Solutions Manager \\
Simon Evans &           ESRI &          Professional Services, Marine Industry \\
Neil Millett &          ESRI &          Professional Services \\
Art Hadad &             ESRI &          Development Manager (ArcIMS) \\
Steve Copp &            ESRI &          Software Development (SDE) \\
Gene Vaatveit &         ESRI &          Software Development (Metadata) \\
Kim Burns &             ESRI &          Geography Network representative \\
Ted Habermann &         NOAA &          Principle Investigator  \\
Dan Holloway &          URI &           Data site population, user interface and server \\
Dale Kiefer &           USC & \\
Nathan Potter &         ORST &          DODS developer, JAVA, SQL Server \\
Frank Obrien & & \\
Chris Finch &           JPL &           Data provider, HDF Server and PC port of DODS \\
Len McWilliams &        Informix &      Geospatial solutions to Informix and Internet applications \\
Mark Ohrenschall &      NOAA &          Programming Freeform and DODS servers, SDE  applications and simple data type support. \\
Steve Hankin &          NOAA &        PMEL Ocean scientist requirements, early in DODS, original LAS  \\
John Cartwright &       NOAA &          ARCIMS and JAVA servlett \\
Frank O'Brien &         USC &           wants to extract data, serve to DODS, has proposal to NRL \\
Daniel Martin &         CSC/TPMC &      Meeting minutes \\
\end{tabular}
\end{center}

Jeane Rebstock comments
\begin{itemize}
\item Provides high level support to the ocean industries.
\item ESRI wants to grow ocean industry business.
\item ESRI wants to understand user needs.
\item Mentioned that NRL is interested in DODS/NOPP.
\end{itemize}


DODS Overview and Ocean Science User Requirements with Dan 
Holloway 

(for complete presentation see \Url{http://www.unidata.ucar.edu/packages/dods/} )
 
\begin{itemize}
\item NOPP Virtual Ocean Data Hub Project
\item 50 participants
\item 4 regional coordinators:
Texas A\&M
NOAA CSC
State of Maine
Oregon State University
\end{itemize}

\begin{itemize}
\item DODS is an architecture to subset, acquire and ingest data.
\item To enhance sharing.
\item To use client tools to facilitate - Directory, Inventory and Data components.
\item High functionality for data access.
\item Primary DODS development is on data and not on directory support.
\item DODS constraints:
\begin{itemize}
\item No reformatting of data.
\item Function with minimum metadata.
\item Easy to interface with existing systems.
\item Must contain essential metadata to function.
\end{itemize}
\end{itemize}

\begin{itemize}
\item Explanation of a typical URL - can be difficult to write, this may be a barrier.
\item DODS has several interfaces (command line, URL builder and GUI).
\item Describe a matrix of interfaces and client applications and their status and pros/cons.
\item 1 Terabyte of data, 300 data sets, 15 sites globally.
\item IDL, Excel and Web browser are the primary clients.
\item Now porting to JAVA, the JDBC based SQL server is now available for downloading.
\item Specialized servers have been written for different data sets.
\item DOS Data Model and constraints.
\item DODS Data Objects (DDS, DAS, Data).
\item Brief open discussion to reintroduce DODS at a high level.
\end{itemize}










RDBMS Talk with Len McWilliams

\begin{itemize}
\item Brief discussion on the benefits of using a relational database.
\item Informix has incorporated SDE functionality and especially projection capabilities into 
the core Informix spatial components.
\item DataBlade is customizable and written in C.
\item New extensions can be written.
\item TEDS project is involved with grid data.
\item There are geodetic extensions.
\item ESRI and Informix have a joint effort underway.
\item The blobs that contain spatial information in the RDBMS are in a ESRI format.
\end{itemize}
        OGC Comments

\begin{itemize}
\item All functions are OGC compliant.
\item OGC has a standard request format.
\item ESRI has an OGC listener/connector that monitors a specific port.
\item The listener converts OGC to an AXL request to forward.
\item OGC specifications and ESRI/Informix collaboration.
\item ESRI connections use XML/HTTP to web applications and Socket type to Databases.
\item OGC Web Mapping Testbed 2 is still working and in specification mode.
\item David Beddoe was the ESRI contact for the technical subcommittee.
\item Conducting testing of WMTB2 with ArcIMS.
\item FGDC funded a broad partnership between NGDC and NOAA's Climate Diagnostic 
Center to test WMTB2. NOPP is involved in testing those standards through NGDC. 
\item WMTB2 is important to NOPP.
\end{itemize}

        NOPP Meeting

\begin{itemize}
\item NOPP user meetings are crucial to understand user needs.
\item These meeting minutes will be incorporated into the regional meeting summary.
\item Are existing NOPP/DODS tools available to current users in need?
\item Overview of the 4 regional meetings underway.
\item NRL is considering a catalog engine using MEL in the development of DEI system.
\item Steve Hankin on board of national meeting, this will occur around March-May, no 
location yet.
\item ESRI is interested in attending and is on the invite list.
\item The regional meetings are primarily for data providers.
\item General discussion on the fate of FGDC if all federal executive orders were rescinded.
\end{itemize}





User Requirements and Live Access Server with Steve Hankin

(for full text see \Url{http://shark.pmel.noaa.gov/\~{}hankin/ESRI/})

\begin{itemize}
\item TMAP, Thermal Model Analysis Program.
\item Lots of software development.
\item Modelers rarely collaborate/connect with others and generate large volumes of data.
\item Focus on current efforts is to build a collaborative system.
\item Bulk of new/needed data access methods are for research purposes.
\item In situ, Satellite, Models, Moorings, Cruises, Drifters, Coastal.
\item Data management is not fully developed for drifters.
\item Plans are to have 3000 active in 6-7 years between surface and 2000m.
\item Other data types include multibeam sonar and ROV based video, samples and still 
imagery.
\item Jeanne mentioned that the spottiness of oceanographic data is a major problem, and that a 
new ESRI tool called the Geo-statistical Analyst Extension may be of interest.
\item Steve - 3D renditions/models of data, especially over time is essential to chemical 
oceanography applications.
\item Present ESRI strategy for grid data seems to be image based.
\item An ESRI staff person will talk later about using grid data with SDE, this is important.
\item Some researchers prefer grid data to limit the volume issue.
\item Level 2 data is along track data stream.
\item Level 3 data is gridded, calibrated and cleaned data.
\item Do users need access to original, full raw data\item Or is a level 3 scientific output optimal?
\item The types of users vary, there are instrument developers, modelers, educators.
\item Many prefer the grid product because of convenient size.
\end{itemize}

        Open Discussion

\begin{itemize}
\item There are many data management issues in visualizing time series data.
\item The TYROS satellites have vertical sounders that provide a rich data stream.
\item Radiosome observations are a rich data source with a long history.
\item Atmospheric and oceanographic research are closely tied.
\item Len - Navy has a similar project - TEDS in Monterey, CA and is related the MEL (master 
environmental library).
\begin{itemize}
\item Data collection.
\item Query
\item Integration
\item Access
\end{itemize}
\item The Environmental Scenario Generator Project is using mySQL with the NCEP re-
analysis grids and taking over much of the TEDS work.
\item TEDS/ESG are investigating Informix as a spatial solution.
\item Should we be investigating expanded partnerships?
\item The multi dimension aspect is crucial.
\item ESRI's goal to merge the computer gaming graphics advances with data driven needs.
\item Many of the animated graphics are not data driven.
\item Mark Abbot at Oregon State is a contact for DODS.
\item Dawn Wright at Oregon is using ArcIMS with Tiffany Vance at PME.
\item ESRI has worked with users collecting multibeam sonar data.
\item Discussion on how we work in true 3D, many move data to IDL.
\item What about commercial efforts?
\item SAIC (Science International Corporation) is a commercial NOPP partner and are also 
involved in the Environmental Scenario Generator Project.
\item Gulf of Mexico has large data sets in the Deepstar Program, maybe propriety but are 
working with DODS as a data provider.
\item Oil and gas industry has heavy data applications in the ROV business/annual inspection.
\end{itemize}


Live Access Server Demonstration

\begin{itemize}
\item A web GUI for data sets in formats supported by DODS.
\item Java map based with server side graphics.
\item Multiple back-end DODS applications doing the work.
\item Easy to set up.
\item Streamed data as text or bitmap image capability.
\item Several different views including X,Y,Z and time were rendered to a browser
        pan/zoom functions.
\item There are a variety of output types such as comma delimited, NetCDF.
\item DODS has a requirement to provide data to GIS users.
\item DODS doesn't want to require the user to have a license.
\item Wants to support typical GIS layer types.
\item Ended up choosing the (SVF) Single Variable Format - ArcView grid.
\item Unfortunately, there is no metadata associated with this format.
\end{itemize}

        Open Discussion

\begin{itemize}
\item Lack of float type support in the ESRI data model is a real issue with scientific grid data.
\item Float types require a pallette object for display.
\item Perhaps a modified WRLD file could help give ArcView access to NetCDF and to other 
science formats.
\item Can an extension to ArcView be written for NetCDF?
\item How do we get NetCDF data into 3D Analyst/Spatial Analyst?
\item Maybe the world file could direct a skip operation to the right location in NetCDF.
\item The large audience of ArcView users and tools warrants a work around for NetCDF.
\item HDF allows redirection/linking of header file to a data file, like the WRLD file config.
\item Maybe this could be wrapped in XML.
\item NCSA folks recast their data from float to integer.
\item People do not want intermediate file formats.
\item Maybe a quick solution could be built even if it is not the end all in design.
\item There are a lot of float data archives out there.
\item New ESRI products will have float capability for grids.
\end{itemize}

        Return to LAS Demonstration

\begin{itemize}
\item No temporary files are written.
\item Will performance become an issue?
\item It is designed for the extraction of modest slices of data.
\item DODS compresses the data stream.
\item DODS is an open distributed system for moving data, mirroring can be used.
\item Load balancing can be configured.
\item Choking of data requests can be delt with incrementally, it is not an issue now.
\item Use standard HTML over HTTP, so firewalls are not a problem.
\item JSP is used to render as a servlett.
\item Using JSP the performance hit is on startup/compile of the servlett.
\item After servlett startup it is in process server.
\item Connection to the core is with a Bean.
\item A NY site has it own data format and customized server, the data can be accessed using 
DODS.
\item A database of base URLs are stored on the LAS.
\item How does an application connect to DODS?
\item The request and response process is described in detail.
\end{itemize}

        Open Discussion

\begin{itemize}
\item The architecture is similar to the ESRI model and to a data blade.
\item IDL and MatLab do not have client libraries.
\item There is a JAVA API for NetCDF.
\item A remote RDBMS could also be connected.
\item Can we identify a group of commonalities?
\item AXL to DAD.
\item The binary data stream may be hard to work with.
\item Maybe integration is best done as two separate components.
\item Need to divide the problem into the binary blobs and the metadata.
\item In regards to metadata, if the AXL has the functionality for DODS communications we 
can avoid building a connector.
\item DODS does not need metadata, what metadata does ESRI clients need?
\item The GML and AXL is published.
\item AXL contain reference to the data stream.
\item Connections are made to ArcIMS with HTTP and by Sockets to RDBMS.
\item Perhaps XML can be used with DODS, is it adequate for client needs?
\item XML is used with and without data in ESRI products.
\end{itemize}




        Continuation of LAS Demonstration

\begin{itemize}
\item Multiple ways to access data from servers.
\item NCEP Re-Analysis, CDC as 2.5-degree square by 6 hour samples.
\item This data set is in the terabyte range.
\item Project level data management is impractical.
\end{itemize}

        Open Discussion

\begin{itemize}
\item CDC is the partner for the wester water assessment, now has SDE for coverages.
\item UNEP is building their version of a distributed server configuration with ArcIMS.
\item The UNEP site likely to be very large, and needs DODS-like data and resources.
\item UNEP is typically poorly funded, however Ted Turner is a major supporter of this.
\item Hands on demo of LAS with Pacific cruise data, slices, graphics etc.
\item DODS has a self-describing data design.
\item Prototype demo for the NOPP system described.
\item Sister server concept allows all to share in a collaborative setting.
\item Display can show model and observation data in the same session.
\item Non grid data example showing cruise based carbon measurements.
\item Great care was taken to render the data as simply as possible without corruption.
\end{itemize}


General Discussion
        
        Topics or question summary from the morning session

\begin{itemize}
\item What protocol is used with ArcIMS, HTTP or SOCKETS?
\item Are OGC Web Mapping Testbed Standards a spec yet?
\item Does ArcIMS employ an OGC listener?
\item To request a Geo-statistical Analyst demonstration for grid generation.
\item What grid types does ArcIMS and ArcMap support and how do we avoid conversion?
\item Can we point a WRLD file at a blob?
\item Discuss the DODS application server diagram with Art Hadad.
\item AXL specification should be given to all participants.
\end{itemize}

        What are the priority questions for the DODS - ESRI integration?
        
\begin{itemize}
\item How do we provide DODS data access to GIS users?
\item How do we provide GIS data access to DODS users?
\item Identify common issues tomorrow.
\item What is minimum metadata requirement for sharing data between DODS and ESRI 
users?
\item No minimum metadata requirements for DODS users.
\item Metadata can be encapsulated in many different ways.
\item DODS has error handling that covers basic QA/QC.
\item What is the basic metadata for an ESRI client?
\item Is the Marine XML standard something that should be investigated?
\item Marine XML has been under construction for several years.
\item Do we build a DODS connector or server application?
\item Do we need the ArcIMS SDK to build a DODS connector?
\item ESRI has a ArcIMS SDK that is in Beta and can be made available to Beta members.
\item Maybe a DODS connector would be a good first step.
\item See ArcIMS architecture design.
\item Discussion of the request and response process and role of a DODS connector.
\item ArcIMS SDK will not be released as a product in ArcIMS 3.1.
\item Does ArcIMS provide direct SQL access to the RDBMS?
\item Which side should we focus on, the connector or a spatial server?
\item Brief talk about the JIVA, Joint Intelligence Visualization Application.
\item JIVA began as a NIMA project with MOIMS, users wanted a vector map.
\item JIVA was basically a catalog of footprints and status maps.
\item JIVA was built with SDE for coverages.
\item With Informix envelopes or extents are stored within the RDBMS.
\item Brief talk on the USGS Geoengine.
\item The differences between the Oracle and Informix database model for spatial capabilities.
\end{itemize}


ArcIMS Demonstration with Hugh Kegan

\begin{itemize}
\item Map services can be built on the fly.
\item Geography Network overview.
\begin{itemize}
\item Maps, data, and geo-services.
\item This is a catalog of catalogs.
\end{itemize}
\item A map service can hit multiple data sources.
\item Overview of a customized JAVA site.
\item Menu expansion can be programmed.
\item Extraction of data can be implemented.
\item Netscape 6 should work, it has not been tested.
\end{itemize}


ArcMAP Demonstration

\begin{itemize}
\item Showed linking to web-based data sources at multiple sites.
\item Feature service data can be saved in ArcMAP.
\item Re-projection can be handled on the fly through XML.
\end{itemize}





ArcCatalog Demostration

\begin{itemize}
\item Goal is to be OGC compliant.
\item Allows query and selection on metadata.
\item Metadata is stored in either a geodatabase or in file, if shapefiles exist.
\item Searches through metadata on multiple database connections can be done.
\end{itemize}


Map on Demand Demonstration

\begin{itemize}
\item ArcIMS forwards a coodinate to ArcMAP.
\item ArcMap renders and image and forwards to ArcIMS.
\item General discussion on VPF to SDE and grid input to SDE.
\item Grids can be loaded into certain versions of SDE with a special ESRI provided tool.
\end{itemize}

ArcScene Demonstration

\begin{itemize}
\item To replace the 3D Analyst extension.
\item To be released as an extension to ArcView, ArcInfo desktop in first quarter 2001.
\item ArcObjects can be used to extend ArcScene.
\item Flybys, visibility domes, and vertical digitizing where shown.
\item Wireframe and draped surface renders.
\item DXF can be read as an input type.
\end{itemize}


Spatial Analyst Demonstration

\begin{itemize}
\item Does not render volumes.
\item Examples shown include cost surface, visibility, and mobility.
\item Multiple surfaces (criteria) were used to determine least effort path.
\item Paths can be restricted to a vector network, such as a road.
\end{itemize}


ArcPad Demonstration

\begin{itemize}
\item Pocket PC and WAP phone updates to an SDE/RDBMS in real-time.
\end{itemize}

SDE and GRID Discussion with Steve Copp 

\begin{itemize}
\item SDE will support floating point grids on 3.1 release.
\item The term grid to ESRI means specifically the ESRI GRID format.
\item The goal in Arc 8.x is to develop grid functionality regardless of grid/image format.
\item Any raster format supported by ESRI can go into SDE.
\item Aux files are used to store statistics on the grid data, if the native format cannot handle it.
\item See ESRI object model diagram for raster data objects.
\item The RDO (raster data object) uses some of the ERDAS library.
\item This mechanism allows extension by building a DLL and registering it to access non-
supported image formats.
\item The aux file can contain an attribute table, LUT, palettes, histograms projection 
information etc.
\item The aux file is automatically built and maintained.
\item ArcView and ArcInfo share the same grid compatibility in 8.1.
\item RDBMS can be used to manage the raster data.
\item Very large raster data sets work well in SDE/RDBMS.
\item Pyramids are generated and tiles are typically 1/4 of the image size.
\item Usually only a small blob is requested and can usually be passed on quickly.
\item The goal is to reduce trips to database and reduce number of tiles that are returned.
\item Pyramids are made only with Nearest Neighbor.
\item The underlying table construction is the same across RDBMS.
\item Nearest Neighbor, Bilinear and Bicubic available in Arc 8.x.
\item LZ77 (PNG,ZIP) is used to compress image data.
\item There is a JAVA and C API to SDE.
\item All pyramids must be completely in or outside of the RDBMS.
\item There is no data type for true volumes or time dependant data.
\item ESRI would like to investigate multi-dimensional data types.
\item ESRI would like to adopt a voxel/volumetric capability.
\item For display and analysis performance ESRI envisions using true discrete frame 
techniques as a starting point.
\item Each frame could be made into a pyramid.
\item Discussion on the possibly of making a cube/blob as a tile.
\item Informix Data Blade allows the indexed extraction from a blob.
\item Discussion on performance and blob/cube requests and sub-setting.
\item The TEDS program is trying to use tiles.
\item ESRI true 3D development is just beginning.
\item ArcIMS has specific file formats that are supported, same as ArcView 3 and MO.
\item WRLD file data is handled in AXL.
\item ArcIMS is needed unless access is available across TCP - LAN.
\item We can't expect a DODS provider to run ArcIMS.
\item DODS delivers data right to the local application.
\item Users can run functions on the server side.
\item DODS doesn't yet have a multi-user access problem.
\item ArcIMS 3.1 supports re-projection of images.
\item All pyramids are kept in a single table.
\item Tile size and levels are tunable, but probably will not yield a huge increase in 
performance.
\item ESRI has plans to make a HDF reader.
\item COM tools for data loading and display is on Windows only.
\item Unix clients are for access only.
\item Flat binary and LZ77 are the blob formats currently supported.
\item Compression is tracked through a visible flag.
\item Compression really helps on machines with a fast CPU and slow disks.
\item SID is lossy and must be decompressed to enter a DB.
\item This combination makes little sense and is not used.
\item JPEG 2000 maybe supported in SDE 8.2.
\item ArcObjects gives full access to data in memory.
\end{itemize}


\subsection{November 8}

Why Use a Relational Database Management System
 
\begin{itemize}
\item Not all data need reside in the database, external data can be linked.
\item An RDBMS is not always the optimum solution.
\item Index capabilities are a strength.
\item Rtree index method is scalable and self-balancing.
\item Integration of many advanced data types.
\item Relationships can be modeled.
\item Performance can be enhanced through parallel architectures in scalable configurations.
\item Ease of management for large data sets.
\item Projection of the data can be done on the fly.
\item Access to SQL queries.
\item Backup, recovery, rollback, replication, transactions, multi-user, and other routine 
RDBMS functionality.
\item The Informix and IBM/DB2 approach is more object based then Oracle.
\end{itemize}

        General Discussion

\begin{itemize}
\item How does a client make a request, and conduct discovery of a data source?
\item Answer - SQL, JDBC, ODBC, ArcSDE.
\item In the GEODE project (USGS) a GUI interface was developed with pre-compiled SQL 
queries.
\item Queries can also be constructed on the fly through an application.
\item In ArcIMS queries are made via HTTP.
\item In the JAVA world the discovery mechanism is missing.
\item The science data types vary and effect what is available on request.
\item A published schema may be hard for a client application to process.
\item GEODE had a server resource index like the ArcIMS directory list.
\item DODS general overview.
\end{itemize}

        What are the interoperable issues to provide access to GIS clients.

\begin{itemize}
\item See S.Hankin web page for full text: 

\Url{http://shark.pmel.noaa.gov/\~{}hankin/OCMIP/DODS/DODS.htm}.
\item Steve Hankin describes the mechanism to enable a DODS client using a diagram.
\item A server is needed for each format of data provided.
\item Format and location of data is abstracted.
\item Scientist must write the read method if they are not using a supported format such as 
NetCDF, HDF or FreeForm.
\item Writing a server, using FreeForm maybe a barrier for some scientist to publish data.
\item Freeform data description method can help decipher many formats.
\item Most scientific data is generic.
\item Need to reduce the barrier to adoption of DODS, scientist, have little time to write 
servers.
\item Scientists do not want to reformat data.
\item A registry of servers is available.
\item A web crawler is also under construction.
\item Goal is....GIS client and science data.
\item Goal is....GIS data and science client.
\end{itemize}


ArcIMS Discussion with Art Hadad

\begin{itemize}
\item We need a bi-directional mechanism for the integration of scientific and GIS data.
\item We need to provide GIS users access to the Terra bytes of scientific data.
\item MODIS is a new and huge data stream of hyperspectral data in the 10-100m spatial 
resolution, destined for GIS clients.
\item MODIS data is in the EOSHDF.
\item LIDAR is another major data source for measurement of elevation, vegetation and tree 
counts.
\item How do we get a DODS server into the ArcIMS drop down list, how much metadata is 
needed for a GIS connection?
\item General introduction to the ArcIMS architecture:
\begin{itemize}
\item Presentation
\item Business Logic
\item Storage
\end{itemize}
\end{itemize}


ArcIMS Architecture Diagram

\begin{itemize}
\item Any client that can read XML can access data from ArcIMS.
\item Discussed ArcIMS clients, HTML, JAVA, AEJ, ArcView, ArcMAP etc.
\item The web server has several connectors, ASP, WMS 1.0, Cold Fusion etc.
\item In 3.1 there is a set of JAVA Server Beans forming the applink used to write your own 
connector.
\item AXL is sent to app server.
\item The POST mechanism is the preferred mechanism.
\item The appserver forwards AXL to the spatial server and off to storage to retrieve data.
\item The manager is built with JAVA applets.
\item The spatial servers are written in C++,  this is the only part left to port to Unix.
\item General discussion on major ArcIMs components
\item ArcIMS is being ported to Linux, SGI, AIX
\item Spatial servers do the major work.
\item More physical servers can be added and managed through virtual servers.
\item AXL is the glue that binds the configuration to the request and response.
\item The spatial server is a container.
\item 5 extensions out of the box include, image, feature, query, geocode, and extract.
\item Custom services can be extended.
\item Multiple workspaces exist for SDE, raster....
\end{itemize}

Spatial Server Diagram

 


\begin{itemize}
\item Extensions to ArcIMS are very open.
\item With an Image Service and simple clients, the request will return a URL of the location 
of an image.
\item With an Image Service and smarter clients, the response includes the XML encoded 
binary data (GIF, JPEG, PNG).
\item With an Image Service a get feature request will respond with text encoded within an 
inflated XML file.
\item An Image Service is server side rendering.
\item With a Feature Service the response is an inflated AXL file with embedded proprietary 
binary stream.
\item Clients render in feature streamed configurations.
\item The servers render in Image Server configurations.
\item What do we need to know to build a DODS connector.
\item It appears that a DODS connector must handle communications in both directions, from 
DODS URL to AXL and from an AXL stream output from the app server to a DODS 
stream.
\end{itemize}

        
Here are the options

\begin{itemize}
\item Extend the connectors.
\item Add a new data source.
\item Modify on the client. 
\end{itemize}




Central Cloud Diagram

 


\begin{itemize}
\item ESRI uses the Xercese tool to parse XML into standard usable objects.
\item Thick client construction is challenging within DODS design.
\item XML is the basic solution to a variety of changing problems that ESRI faces.
\item What is needed to make a new data source - unpublished technology the ArcIMS SDK?
\item The ArcIMS SDK will be a product no sooner then 6 months but maybe later.
\item ArcIMS SDK will not have admin components for security reasons.
\item ArcIMS spatial server is C++.
\item ESRI has a prototype that is essentially a JAVA proxy server extension to the spatial 
servers.
\item ESRI also has a prototype that allows the registration of a COM object.
\item Design work for connectors will remain stable, not so on the server side - possibly.
\item Getting a DOD client to talk with a ArcIMS server is recommended as a first, easiest 
step.
\item The technology of the ArcIMs SDK is in flux and would effect development within the 
business logic section to ArcIMS.
\item ESRI recommends development of a connector for DODS client to GIS data.
\item A metadata server extension will be available on ArcIMS after ArcIMS 3.1.
\item JDBC and ODBC connections to the server will be available in the future.
\item Distributed GIS is the goal.
\item The NOPP project has a 3-year plan. Higher funding levels are slated for year 2 and 3.
\item A DODS connector.
\item General agreement that in year 1 we will develop a DODS connector.
\item This will also allow developers to better understand ArcIMS development and allow the 
ArcIMS SDK to stabilize.
\item From the web server back the configuration works on SOCKETS.
\item Discussion on the emap link.
\item Discussion on the appserver link functionality for communication.
\item A set of JAVA Beans will be provided for wrapping AXL to read/write Map, Layer, 
Coordinate System, and other operations
\item This set of BEANS handles the SOCKET communications.
\item James Gallagher is the principal developer behind DODS.
\item HTTP and XML has been in the plans for the past year.
\item TomCat will be fully supported in 3.1, now in Beta 2.
\item The application server link is a wrapper class (JAVA Bean) to simplify development.
\item More comments about COM implementation.
\end{itemize}

        First Year Goal

\begin{itemize}
\item DODS team evaluates the appserver link.
\item Writing a DODS connector.
\item We can probably create the connector with the non-ESRI funds and later use the ESRI 
funds to address the appserver work.
\item Showing the new functionality of the connector will be important.
\item Connectors do not communicate among one another.
\item To be registered with the Geography Network, a map service or ARCIMS is not required.
\item Can the GN handle legacy systems.
\item Some general discussion.
\item ESRI clarifies that the JAVA BEANS do not decode the feature stream class (at this 
point), only ESRI clients can decode the feature-streamed class.
\item The ArcIMS SDK is available now only to Beta users.
\item Query Server is necessary to get back and XML open stream.
\item Beans will be in the SDK to parse feature streams.
\item Discussion on the performance burdens of a query server, especially between the spatial 
server and the connector.
\item Maybe a Bean could deliver a binary out to the DODS client and skip the connector.
\item Discussion on the pros and cons of a feature service.
\item Using scale dependency can help resolve performance problems, maybe an Image Server 
at a broad scale and Feature Server at a fine scale.
\item ESRI recommends server side processing with HTTP in general.
\item A query service can request against either an Image or feature service.
\item Layer contents can be dynamic.
\item Projections can be done on the fly, including images.
\item How do we map to a DODS client?
\item DODS requests are basically an SQL-like form.
\item Discussion on how does the DODS client receives data to render as polygons or lines?
\item The regional groups are to define some of the limitations of the DODS data model.
\item NFS systems can be a problem, a mixed file system (windows/linux) can be a problem 
for multiple ArcIMS Spatial Servers, this is especially problematic with dynamic map 
services.
\item Sun and Microsoft may be able to resolve their differences on the UFS.
\item The GN uses 20 Map Servers, 3 web servers, and 1 Oracle/SDE with maximum virtual 
servers, over a million maps per day.
\item All services and data are hosted at ESRI.
\end{itemize}


NOAA NNDC Server and SDE with Ted Habermann

\begin{itemize}
\item Overview of NOAA Server and SDE.
\item SQL functionality and potential.
\item A database is built to store:
\begin{itemize}
\item Data Info.
\item Data Source.
\item Display Look.
\item Search Look.
\item Query Look.
\end{itemize}
\item A browser interface is used to manage the catalog that is housed in a RDBMS MySQL.
\item Nathan wraps the JDBC request into a DODS URL.
\item PERL is used within the data manager tools.
\item Demonstration of the SDE license manager monitor page.
\item Most of the data management difficulties are social and not technical.
\item Not a lot of blob data, except satellite sources, in scientific data.
\item Informix allows SQL access of RDBMS.
\item Demonstration of the SPIDR2 database.
\item Sara Graves is authoring the Earth Science XML schema, maybe solution for 
commonality.
\item Metadata will be crucial here also.
\item Search mechanisms are still unclear at this point.
\item It is easy to forward and DODS URL to a SQL.
\item Once data is in the NNDC it is DODS/ArcIMS accessible.
\end{itemize}


Geo-statistical Analyst Demonstration

\begin{itemize}
\item Variety of grid options including CoKrigging.
\item Will be released in the 1st quarter.
\item Chi sq. will come later, in a follow on release.
\item The core utility is the visualization of the analysis and integration with spatial data 
sources.
\item Processing done in existing map units.
\item Examples included semi-variagram, error mapping, and clipping visual display to a 
polygon.
\item Approximately 100,000 data set processed in 20 minutes.
\item General discussion on the grid resolution output.
\item DODS staff saw great opportunity for scientific use.
\item Data sources include personal geodatabases. SDE sources in later releases.
\item Open discussion on multiple topics.
\item Multiple data types, contours, and grids can be incorporated.
\item New shape files/geo-databases can be output.
\item Ted discussed ArcIMS and SDE license changes.
\item The database approach is expanding.
\item The SDE component to Informix enhanced performances through load balancing.
\item SDE is similar to a driver.
\end{itemize}


Geography Network Presentation

\begin{itemize}
\item Includes publishers - government and non-profits who do not charge for data.
\item Partners - commercial units typically selling a service or data.
\item A collaborative to publish and share data and services, especially live data.
\item A type of marketplace.
\item Search capabilities of catalogs.
\item The all publisher list represents the ArcIMS sites.
\item Contains 4 geoservices, such as a geocoder.
\item 187 clearinghouses
\item Also contains ``solutions'' which are interactive ArcIMS sites.
\item Description of the registration process.
\item Some discussion about whether a non-ArcIMS site can provide a data stream.
\item ESRI will adopt the new international metadata standard.
\item ESRI developing a new tool to extract necessary metadata from a registrant site.
\item Exposure is a benefit to providers.
\item It is unclear if a non-ESRI map service could actively participate delivering data.
\item The GN has a fixed user interface.
\item ESRI is looking for 24X7 participants with data sources or services of national interest.
\item This is an ArcIMS site currently using a single SDE/RDBMS sources hosted by ESRI.
\item There is no specific QA/QC element or review of the sources or services.
\item Map services include Web Map Server (WMS) and AXL designed sites.
\item ESRI clients cannot access WMS services.
\item Guided tour of several GN pages.
\item There are redundant system components.
\item The GN hub submits an AXL request to the remote provider.
\item Password protection is an option.
\item Gateway sites like Texas exist, private GN's are a possibility also.
\item Custom provider applications can be hosted.
\item ESRI recommends an ArcIMS or WMS as the best mechanism to participate in the GN.
\item NOAA ``Lights at Night'' maybe a good map service to publish.
\item Published services access TIFF, JPEG, MrSID and standard ArcIMS readable formats.
\item MrSID performance isn't optimum.
\item ESRI has resampled and made pyramid data sets with scale dependencies to increase 
performance.
\end{itemize}


Metadata Discussion with Gene Vaatveit

\begin{itemize}
\item ESRI does not currently support a search using the Z39.50 protocol.
\item Metadata searches are to query local or ArcIMS sources.
\item ArcIMS metadata server should be released in 3.2 and runs with XML.
\item This sits on top of SDE/RDBMS.
\item ESRI has discussed development with BlueAngle, doesn't look likely.
\item Currently metadata is either stored on disk or in an RDBMS.
\item ArcIMS will contain a metadata extension that queries an SDE/RDBMS configuration.
\item ESRI has developed its own database design for metadata storage.
\item The ESRI design is more text-based, somewhat like Yahoo or generic document system.
\item Metadata can be stored separate from the data, and is not mandatory.
\item Searches can be full text or field based.
\item There is no enforcement of mandatory tags.
\item ESRI has made a prototype that works with Blue Angle.
\item Z39.50 sites work poorly in firewall configurations, they do not cross on port 80.
\item BlueAngle would be very expensive to re-package within the standard ESRI catalog.
\item There has been talk about wrapping a Z39.50 request in XML.
\item The OGC catalog service is a wrapper on Z39.50.
\item Isite is the server search component.
\item XML has some distinctive advantages, flexibility, open.
\item Full text searches of a RDBMS can be problematic, ESRI does not yet utilize the specific 
tools that each RDBMS vendor has to enhance this function.
\item ESRI is seeking a common solution not tied to a particular RDBMS.
\item Perhaps a new release of SDE will have more advanced text search capabilities.
\item Public classes exist for read/write of shapefiles.
\item Shapefile weakness is that it is defined with dbf and field names are restricted to 8 
characters.
\item How do we make the scientific data available to the GIS users?
\item Mark Ohrenschall is writing a translator for shapefiles.
\end{itemize}


        
Open Discussion and Recap of Issues

\begin{itemize}
\item A WMS will not show as a ArcIMS service or layer to an ESRI client.
\item The Geo-statistical Analyst will not read data from SDE in release 1.
\item The Geo-statistical Analyst looks very interesting for research purposes.
\item First step is to build a DODS connector, probably with existing resources.
\item 15K is available for consulting.
\item March/April is the national meeting.
\item Lets get the beta 3.1 ArcIMS to start investigating.
\item General discussion on experienced DODS participants.
\item Peter Cornillon is the primary scientist behind this.
\item James Gallagher is the chief architect of DODS.
\item OSU is the source of the newer SQL activities.
\item The MATLAB and IDL component are mostly out of the Northeast.
\item NOPP and DODS relationship discussion.
\item Ted is leading the policy/coordinating effort.
\item Scientific data to GIS clients is Peter's priority.
\item DODS reflects a broad selection of the ocean community.
\item Lets get the AXL specifications out to the mailing list.
\item Do we have any design specifications for the DODS connector?
\item The development path design will be ad hoc.
\item We'd like to work on the integration of the DODS SQL server and SDE.
\item Informix may be the preferred platform for working with the scientific spatial data.
\item How do we get grids into the Geo-statistical Layer?
\item There is no volumetric data type within the ESRI domain.
\item The new 3D Analyst Extension is a 2.5 D application.
\item How can we continue dialog with the marine user representation with ESRI?
\item Simon and Richard are the lead ESRI contacts.
\item ESRI would like to develop formal user requirement profiles for marine users.
\item ESRI will have a marine track at the user conference.
\item Ted will be submitting a NOPP paper to the ESRI user meeting.
\item Simon may be able to help coordinate a 10-15 person meeting that overlaps with the user 
conference.
\item DODS and NOPP interests may be broader then usual ESRI centric to be hosted within 
the user conference.
\item The group would probably be smaller then this workgroup.
\item Will need the ArcIMS SDK to work with the feature streaming capability.
\item How do we transition from the connector that supports DODS clients to GIS data?
\item It is still unclear how the connector can or can't support development of GIS client 
connection to DODS data sources/servers.
\item Continue talks with Informix.
\item Document user profile access to ESRI dev.
\item User conference may be a good opportunity to show a new capability of DODS.
\item Coordinate IMS SDK.
\item Organize a list of data.
\item If ESRI supported WMS, DODS could publish to GN.
\item There is a capability of publishing WMT from DODS, the (WMT-DODS gateway) with 
grid data only.
\item This is a prototype on WMT1.0.
\item A WMT2 is under construction but unavailable.
\item We should take a look at the technical specification of the Monterey grid work (TEDS).
\end{itemize}

\section{Attendees}

\subsubsection{ESRI}

\begin{center}
\begin{tabular}{lll} \\ 
Richard Lawrence &      rlawrence@esri.com &            909-793-2853 x 1700 \\
Jeanne Rebstock &       jrebstock@esri.com &            909-793-2853 x 1160 \\
Simon Evans &           sevans@esri.com &               909-793-2853 x 2380 \\
Neil Millett &          nmillet@esri.com &              909-793-2853 x 2709 \\
Art Hadad &             ahadad@esri.com &               909-793-2853 x 1055 \\
Steve Copp &            scoop@esri.com &                909-793-2853 x 1480 \\
Gene Vaatveit &         gvaatveit@esri.com &            909-793-2853 x 2335 \\
Kim Burns &             kburns@esri.com &               909-793-2853 x 2712 \\
\end{tabular}
\end{center}

\subsubsection{DODS}


\begin{center}
\begin{tabular}{lll} \\ 
Ted Habermann &         Ted.Habermann@noaa.gov &        303-497-6472 \\
Dan Holloway &  d.holloway@gso.uri.edu &        401-874-6831 \\
Steve Hankin &          hankin@pmel.noaa.gov &  206-526-6080 \\
John Cartwright &       jcc@ngdc.noaa.gov &             303-497-6284 \\
Nathan Potter &         ndp@oce.orst.edu        &       541-737-2293 \\
Mark Ohrenschall &      mao@ngdc.noaa.gov &             303-497-6507 \\
Dale Kiefer &           kiefer@usc.edu  &       213-740-5814 \\
Frank Obrien &          fjobrien@home.com       &       714-730-6858 \\
Chris Finch &           chris.finch@jpl.nas.gov & 818-354-2390 \\
\end{tabular}
\end{center}


\subsubsection{Other}
                                                                                
\begin{center}
\begin{tabular}{lll} \\ 
Len McWilliams &        len.mcwilliams@informix.com &   703-847-3359 \\
Daniel Martin &                 dmartin@tpmc.com        &       781-544-3803 \\
\end{tabular}
\end{center}

\section{Select HTTP References}

\begin{description}
\item[NOPP]     \Url{http://core.cast.msstate.edu/NOPPpg1.html}
\item[DODS Home]        \Url{http://www.unidata.ucar.edu/packages/dods/}
\item[LAS presentation] \Url{http://shark.pmel.noaa.gov/\~{}hankin/ESRI/}
\item[NOPP Proposal]    \Url{http://www.unidata.ucar.edu/packages/dods/archive/proposals/nopp-html/nopp.html}
\item[ESRI User Conference]     \Url{http://www.esri.com/events/uc/index.html}
\item[Open GIS Consortium]      \Url{http://www.opengis.org/}
\item[NVDS]     \Url{http://nndc.noaa.gov/?home.shtml}
\item[NNDC]     \Url{http://nndc.noaa.gov/?http://ols.nndc.noaa.gov/plolstore/plsql/olstore.main?look=1}
\item[PMEL]     \Url{http://www.pmel.noaa.gov/}
\item[SDE License status]       \xlink{http://dev.ngdc.nndc.noaa.gov/cgi-bin/wt/nndcp/\-ShowDatasets?\-query=\&dataset=400029\-\&search\_look=1\-\&group\_id=14\-\&display\_look=1}{http://dev.ngdc.nndc.noaa.gov/cgi-bin/wt/nndcp/ShowDatasets?query=\&dataset=400029\&search_look=1\&group_id=14\&display_look=1}
\item[Newport Surf Report]      \Url{http://nwprtsrf.oce.orst.edu:8080/nwprtsrf/surf.html}
\item[NNDC Developer]   \xlink{http://dev.ngdc.nndc.noaa.gov/cgi-bin/wt/nndcp/Devel\_Site\_Map}{http://dev.ngdc.nndc.noaa.gov/cgi-bin/wt/nndcp/Devel_Site_Map}
\end{description}


\section{Table of Acronyms}

\begin{description}
\item[API]              Application Programming Interface
\item[ArcIMS]           Arc Internet Map Server
\item[AXL]              Arc eXtensible Markup Language
\item[CDC]              Climate Data Center
\item[COM]              Common Object Model
\item[DODS]             Distributed Oceanographic Data System
\item[ESRI]             Environmental System Research Institute
\item[FGDC]             Federal Geographic Data Committee
\item[GML]              Geography Markup Language
\item[GN]               Geography Network
\item[GUI]              Graphic User Interface
\item[HDF]              Hierarchical Data Format
\item[IDL]              Interactive Data Language
\item[JDBC]             JAVA Database Connection
\item[JSP]              JAVA Server Pages
\item[LAS]              Live Access Server
\item[MEL]              Master Environmental Library
\item[MrSID]            Multi Resolution Seamless Image Database
\item[NCEP]             National Center for Environmental Prediction
\item[NNDC]             NOAA National Data Centers
\item[NOPP]             National Oceanographic Partnership Program
\item[ODBC]             Open Database Connectivity
\item[OGC]              Open GIS Consortium
\item[PMEL]             Pacific Marine Environmental Laboratories
\item[SDE]              Spatial Database Engine
\item[SDK]              Software Development Kit
\item[SVF]              Single Variable Format
\item[TEDS]             Tactical Environmental Decision Support System
\item[UNEP]             Unite Nations Environmental Program
\item[UFS]              Universal File System
\item[VPF]              Vector Product Format
\item[WAP]              Wireless Access Protocol
\item[WMS]              Web Map Server(OGC interface specification)
\item[WMT]              Web Mapping Testbed
\item[XML]              eXtensible Markup Language
\end{description}


%%% Local Variables: 
%%% mode: latex
%%% TeX-master: t
%%% End: 



\end{document}











%%% Local Variables: 
%%% mode: latex
%%% TeX-master: t
%%% End: 



