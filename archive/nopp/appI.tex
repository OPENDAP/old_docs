\renewcommand{\chaptertitle}{Gulf of Mexico Regional Workshop}
\chapter{\texorhtml{}{Appendix A }\chaptertitle}

% $Id$

\begin{center}31 October - 2 November 2000\\
Stennis, MS
\end{center}

\section{Introductions and Background}

The workshop was called to order at 0910 on 31 October 2000, by W.
Nowlin. Provisional agendas and participant lists were distributed.
The agenda is given in \sectionref{I,agenda}. Each attendee gave a
self-introduction. Attendees and affiliations are given in
\sectionref{I,attendees}.

Nowlin gave an overview of the Virtual Ocean Data System Hub project
funded by the National Oceanographic Partnership Program (NOPP). This
presentation, given in \sectionref{I,presentation}, included:

\begin{enumerate}
\item The overall goal of the project and specific research objectives
  identified in the NOPP Broad Area Announcement to which the project
  was a response;  
\item The organizational structure of the project;
\item The phases of the program;
\item Issues to be considered at regional workshops;
\item The objective, approach, timing, and participants for the
  national workshop; and 
\item An outline of the implementation expected to emerge and guide
  the out year proposal.  
\end{enumerate}

\section{The Distributed Oceanographic Data System}

D. Holloway (University of Rhode Island) gave a presentation
describing the Distributed Oceanographic Data System (DODS), including
real time demonstrations of DODS capabilities using the Internet. This
presentation is summarized in \sectionref{I,DODS}. Included were
discussions of

\begin{itemize}
\item  What is DODS?
\item  Interoperability and metadata
\item         What is required to serve data via DODS-enabled servers?
\item         What is required to locate and access data?
\item         DODS and web browsers
\item         DODS-aware applications (MATLAB, Excel, IDL,..)
\item         What types of data can/cannot be served?
\item         Future of DODS applications/servers
\end{itemize}

This session continued through the morning of the first day with many
questions and comments.

Several general points of agreement emerged from this session. DODS is
principally a bottom up data system allowing the user easy data access
from within his application provided the data are on a DODS server and
the user has knowledge of the URL for the data. DODS does not provide
a data catalog; that will be a necessary part of the regional and
national data systems. Not all users have need for DODS. As examples,
modelers do not normally work in a DODS-enabled application, and many
analysts prefer to examine the data and metadata carefully prior to
its use in an application. However, DODS can be a very useful element
of the required regional and national data models. DODS service of
data is to be encouraged, but should not be mandatory.

Another key point of agreement is the necessity for data originators
to freely share data if they are to be part of an observing system,
regional, national or international. The tendency of oceanographers to
treat data as proprietary should cease.

A third point on which there is general agreement is the need for
adequate metadata to accompany the data sets. It is particularly
important to have adequate use metadata, and to maintain quality flags
as are available. This point was revisited later in the workshop.

\section{Presentations Regarding Regional Data and Products}

Representatives of each participating regional institution were offered time to make a 
presentation regarding data holdings, ongoing and planned observational programs, and models. 
It was suggested that each presenter include indications of data sets that they would be willing to 
share, whether they could be served via a DODS server, and data sets they would like to receive. 
Presentations continued during the afternoon of day one and the morning of the second day. 
Considerable discussion ensued as part of the presentations, which are
summarized in \sectionref{I,regions}.

Presentations were made by:
\begin{itemize}
\item John Blaha (NAVOCEANO) speaking on the Northern Gulf Littoral
  Initiative;
\item John Lever (NAVOCEANO) speaking on the willingness of NAVOCEANO
  to provide 24 hr by 7 d server capability for U.S. ocean data;
\item J. J. O'Brien (FSU, COAPS) speaking on SEVEER, SeaWinds from QuickScat, and the 
NCOM under development for the Gulf of Mexico;
\item Nan Walker (LSU Coastal Studies Institute) speaking on real time and archived data from 
the Earth Scan Laboratory and CSI, BAYWATCH, and WAVCIS;
\item Frank Muller-Karger and Doug Myhre (USF) speaking on fields from SeaWifs and AVHRR 
satellite data now being served by USF
\item George Ioup (U. New Orleans) speaking on data sets from Lake Ponchartrain, Barataria 
Basin, and other littoral waters that could be made available
\item Jim Corbin (MSU Engineering Research Center) speaking on the Distributed Marine 
Environmental Forecasting System;
\item Worth Nowlin (TAMU) speaking on data holdings and ongoing projects of Texas A\&M 
University;
\item Norman Guinasso (TAMU, but representing the TX General Land Office) speaking on the 
Texas Automated Buoy Project;
\item Ruben Solis (Texas Water Development Board) speaking on hydrological and environmental 
data being monitored in Texas and on models of Texas estuaries run by
his agency
\item Patrick Michaud (TAMU Corpus Christi) speaking on the TCOON and on present and 
planned CODAR observations along the coastal Gulf;
\item Mark Luther (USF) speaking on the Coastal Ocean Monitoring and Prediction System on the 
Florida shelf, including the Tampa PORTS, and developments of the USF Center for 
Ocean Technology Development;
\item Robert Molinari (NOAA/AOML) speaking on archived ocean station, ship-of-opportunity, 
and drifter data that could be made available, the real time and delay mode data from 
the NOAA GOOS center, and the Florida Bay project;
\item Tony Amos (Univ. of Texas Marine Science Center) speaking on the many long-term, 
multidisciplinary data sets available from the Institute and of his real time data and 
tidal predictions;
\item Susan Starke (NOAA Coastal Data Development Center) speaking on plans of the center to 
facilitate access to ``coastal'' data, from the EEZ to 300-km inland. She gave a brief 
overview of Data Exchange Interfaces, a middleware under development for Navy and 
intended for use by the Center; and 
\item Richard Campanella (Tulane Univ.) speaking on the Long-term Ecosystem Assessment 
Group
\end{itemize}


\tableref{I,table1} \texorhtml{on \pageref{I,table1}}{} indicates the
various data sets offered by the presenters for sharing. The
descriptions are necessarily brief (perhaps schematic in some cases).
These data sets will be more fully described on a web site to be
maintained by TAMU for reference of all interested users of regional
data sets from the Gulf of Mexico.  Tentative commitments were made
toward the end of the workshop to which of these data sets would be
served initially, by what time and whether it seemed likely that they
would be offered via a DODS server.  That information also is
indicated in \tableref{I,table1} and will be available on the TAMU web
site.



\section{General Issues for DODS Regional Workshops}

The workshop was asked to consider five general issues. That was done.

\begin{enumerate}
\item Is the DODS data model adequate for data sets to be served in the region? What additions are 
required?
\item What are important interface issues for regional users? From data discovery to use?

Key to any successful data and product exchange system is the willingness of data originators to 
openly share data in a timely manner. Participants in the distributed data system for the Gulf of 
Mexico region must be willing to do so.

It was agreed that DODS is useful as one basis for data sharing in a distributed mode. Of course, 
it is very useful for users wishing to operate within DODS-enabled application software. 
However, it is not adequate, or intended, for all regional needs:
\begin{itemize}
  \item   Data location is essential (catalogs are needed);
  \item   Many clients use applications not supported by DODS;
  \item   Large archives now serving data need not necessarily change to DODS mode;
  \item   Distribution of data and products in real time to users (especially public) may be best done 
via other methods.
\end{itemize}

Encouragement of new data servers to use the DODS mode is very desirable and will enhance 
data utilization by many users. It was agreed that Gulf of Mexico regional data providers/users 
would serve data via DODS servers as feasible. For some applications this will not be the most 
expeditious or logical method. Open exchange of data and products via the Internet is the desired 
outcome. 

Preparation and maintenance of a catalog of information now on DODS servers, as well as on 
other servers of ocean data, in the region is one essential initial step. (See issue number 5 below.)

A concern expressed by workshop participants was the need for a continuing archive for data 
and product sets considered being of long-term value. Many such sets are compiled by individual 
researchers and on their retirement or death may no longer be accessible. NAVOCEANO is 
providing a server for non-classified ocean data. It will be available for national as well regional 
use. In cases where data sets are used frequently, and especially if they are large, they can reside 
at NAVOCEANO. 

The need to preserve real-time data streams as time series was stressed. Real-time data streams 
should be further quality controlled in delayed mode and aggregated to produce as complete 
series of quality data as possible. This is already the situation for many data sets (e.g., drifter, 
ship-of-opportunity, and Argo data).

DODS offers password protection. However, it was agreed that data confidentiality will be up to 
the data provider who serves the data. 

\item What types of semantic metadata will be required? (Focus of search and use metadata.) What 
standards will be used?

The workshop considered what metadata of this type will be required. It was agreed that at a 
minimum the following information must be included:
\begin{itemize}
\item Where do the data reside?
\item 4-D location information,
\item Definitions of parameters,
\item Units (including standards where appropriate),
\item Accuracy and precision (as available), and
\item Flags from originators and secondary reviewers.
\end{itemize}

Very desirable additional information that should be included includes:
\begin{itemize}
\item Methods,
\item Platform,
\item Instrument type, model, band,
\item Calibration data, and
\item Reference to algorithms used.
\end{itemize}

Other useful metadata that is recommended includes:
\begin{itemize}
\item Supporting descriptions,
\item Data originator and contact information,
\item Source of funding for data collection, and
\item Reports in which data are included.
\end{itemize}

In discussing desirable formats for metadata, consideration was given to the desirability of 
having some degree of uniformity and the fact that data served by federal servers must use a 
standard format for metadata. It was agreed to adopt for the region FGDC. The amount of 
metadata required by FGDC is actually less than agreed necessary by the workshop. Moreover, 
there appear to be software packages to assist the data originator in preparing metadata in FGDC 
format.

\item What data sets will be served initially as part of this effort in order to seed the system? What 
assistance is needed?

The representatives present were queried as to what data sets they would agree to serve initially, 
by when would the initial data sets be served, whether they would enable a DODS server, and 
whether they would require assistance to do so.

Shown in \sectionref{I,table1} are the institutes holding Gulf of
Mexico data and product sets they are willing to share. As indicated
some already are served via web sites. Those marked with the priority
``initial'' will be served first, by the dates shown and by DODS
server if so indicated.

This represents a significant commitment to the establishment of a Gulf of Mexico regional data 
system, the first step toward a model-based Gulf observing system.

\item Is a regional node needed for coordination, including data location, reference for user support, 
etc.?

A number of activities were recognized as necessary first steps in the implementation of a Gulf 
of Mexico regional ocean data system as part of a national system.

\begin{itemize}
\item    COAPS at FSU agreed to set up and maintain a list serve to enable ease of communication 
among regional participants. 

\item   NAVOCEANO will procure and maintain a server for distribution of non-classified ocean 
data, model output, and products via the Internet.

\item   TAMU will set up and maintain a web site indicating data, model output and products that 
participants are prepared to serve. It will include links to web sites where data are now 
being served. It will include a catalog of regional data on DODS-enabled servers.

\item  The formation of several ad hoc groups was agreed to.

  \begin{enumerate}
  \item A working group on bathymetry and coastlines with the
    objective of specifying the best currently available data sets for
    bathymetry and coastline locations for the Gulf of Mexico.
    Membership will include representatives from COAPS, TAMU (Wm.
    Bryant's group), NGDC, and NAVOCEANO. (Bryant to chair?)
    
  \item A working group to consider and recommend the best approach to
    serving all real time data now being collected for the Gulf by
    non-federal institutions. This method should allow a user access
    to the intersection of the data sets. If possible it should be
    accessible through a DODS server. Membership will include
    representatives of TCOON, TABS, COMPS, WAVCIS, Florida Bay,
    NAVOCEANO, BAYWATCH, and URI (Dan Holloway).  (Michaud to chair?)

    
  \item An ad hoc group to consider research needed to evaluate
    regional satellite and model- derived data fields produced by more
    than one organization, with a view to identifying characteristics
    and recommending specific fields for use. Such fields include SST,
    color, wind, and sea surface height. Membership was not
    determined, but O'Brien might be willing to chair.

  \end{enumerate}
\end{itemize}


\end{enumerate}


\section{Other General Considerations}

It was hoped that industry participants could be entrained into the system. As an example of the 
potential information that might be shared, Nowlin reviewed data sets and model output held by 
the oil and gas producers joint industry projects EJIP and CASE. This includes current meter 
time series, CODAR data, synoptic oceanographic survey data, rather fine resolution circulation 
model output, and atmospheric storm data and analyses.

It was agreed that Nowlin would send to the workshop participants a copy of the project budget 
as submitted to NOPP. Those budgets for years after the initial planning year were purposely 
vague. The reason for distribution is to allow regional participants a say in determining priorities 
for items to be included in out-year budget requests.

To increase basic data coverage in the Gulf, it was considered desirable that provisions be made 
for increasing ship-of-opportunity data from both commercial and research vessels regularly 
traversing this area. It was suggested that such vessels might be equipped with XBT launchers, 
ADCPs, and Improved Meteorological (IMET) packages.

It was generally agreed desirable to include representatives of Cuban and Mexican institutions in 
this Gulf of Mexico regional data center.


\begin{longtable}{|p{0.75in}|p{2.75in}|p{0.5in}|p{0.5in}|p{0.5in}|}
  \caption{Institutes holding Gulf of Mexico data sets they are
    willing to share. Some are now available via web site as
      indicated. Those marked as initial priority will be served
      first, by dates shown and by DODS server if so
      indicated.\label{I,table1}}
\\ \hline
\textbf{Institution} & \textbf{Data or Program} & \textbf{Priority} &
      \textbf{Date of Service} &    \textbf{DODS?} \\ \hline
\endfirsthead
\caption{Institutes holding Gulf of Mexico data}
\\ \hline
\textbf{Institution} & \textbf{Data or Program} & \textbf{Priority} &
      \textbf{Date of Service} &    \textbf{DODS?} \\ \hline
\endhead
\hline
\endfoot
NOAA/AOML &    
 \begin{tablelist}
 \item  Florida Bay project
 \item Data from NOAA GOOS Center, available now by web site
 \item Archived drifter and station  data, available on request
 \end{tablelist}
&
initial  &   3/01 & yes, willing, depends on \$ \\ \hline

FSU &
\begin{tablelist}
\item SeaWinds from QuickScat:       
               operational and research 
               products available via web 
               site;
             \item          6-hr winds at 0.5 degree 
               resolution from July 1999
             \item    NCOM for northern Gulf
             \end{tablelist}
&
initial
&
     02/01
&     yes
\\ \hline

LSU  & WAVCIS & & &                                            perhaps \\
&             Earth Scan Lab archived and 
               real time data:  & & & \\
&               AVHRR images for LATEX
                 period &        initial   &   03/01  &   yes \\
&               CSI archived data  & &  ?  & yes \\
&               BAYWATCH program  & & ? &  yes \\
&               LATEX B data   & &  ? &  yes \\ \hline

MSU 
&
\begin{tablelist}
\item Offer computational power   
\item DODS server
\item Investigate link to MEL for service
\end{tablelist}
&
   initial  & &  yes \\ \hline

NDBC & Buoy data, available now via web site & initial & & yes, if
support available \\ \hline

NAVO &  New server for U.S. data sets  & initial  & procure & yes \\
   &         \textbf{Selected archived data sets:} & & & \\
   &               Gridded global DBDBV bathymetry & initial   &   03/01 &    yes \\
   &       Gridded global T-S data  base  & initial   &   03/01 &    yes \\
&             \textbf{Selected operational products: } & & & \\
&               MODIS fields for Gulf & initial   &   03/01 &    yes \\
&               COAMP for Amer. Med. & initial   &   03/01 &    yes \\
&               Global SST  &                 initial  &      ?    &   ? \\
&             Selected 3-dimensional POM     
               output   & initial  &     03/01 &    yes \\ 
&             Selected data from Northern
               Gulf of Mexico Littoral 
               Initiative  & initial  &     03/01 &    yes \\ \hline
TAMU  &       LATEX A data  & initial  &     03/01 &    yes \\ 
&              Historical daily river 
               discharge from major U.S. 
               rivers & initial & 03/01 & yes \\
&             Analyzed wind fields for 
               LATEX period & second & & \\
&             LATEX C data  & second & & \\
&             NEGOM data & & & \\
&             Historical archives of MBT, 
               XBT, ocean station, drifter, 
               moored current meter, and 
               ADCP data  & second & & \\
&             MAMES I \& III; CHEMO I \& II, 
               available now on request 
               via FTP site & & & \\
&             NOAA Status and Trends, 
               available now on request 
               via FTP site  & & & \\
&             EPA EMAP data, available now 
               via web site  & & & \\
&             National Estuarine Programs 
               data from Galveston and 
               Corpus Christi Bays  & & & \\ \hline

TAMU Corpus Christi &  TCOON data and products,
      available now via web site & & & perhaps \\
&             CODAR observations   &          initial &     02-03/01 &
      yes \\ \hline

Tulane & & & & \\ \hline

Texas General 
Land Office    
&
Texas Automated Buoy System  data, real-time available 
                now via TAMU web site

& initial  &     03/01 &    yes \\ \hline

Texas Water Development Board
&
  TX estuarine hydro. surveys,   
    1987-97, available now via 
          web site 
& initial &     early 01  & yes \\ 
&             Sonde estuarine water quality 
               data, available  now via web 
               site 
& initial &     early 01  & yes \\ 
&             TX coastal hydrology records, 
               available now via web site & & & \\
&             Model output for TX estuaries & & & \\ \hline
University of Colorado  &  Daily fields of sea surface   
    height anomaly & initial & & yes \\
&             Model average sea surface height for GoM & initial & &
yes \\ \hline
               

University   
of New       
Orleans &     
Lake Pontchartrain project  &   initial  &     ?    & perhaps \\             
&Barataria Basin project   &     initial  &    05/01 &    perhaps \\
&Coastal Data, Northern GoM &    available     &     &    perhaps \\
&Acoustics Data, Northern GoM  & initial  &    08/01    & perhaps \\ \hline


University   
of South       
Florida        
&
Tampa PORTS data, available    in real time now via web site & & & \\
&             COMPS:          
               archived data             &  initial    &  03/01 &   yes \\
&             ECOHAB & & & \\
&             Archived satellite data 
               (AVHRR, Sea WIFS):      
               Reduced resolution AVHRR
                 since 1993            
&       initial   &   03/01  &   yes (password for SeaWIFS) \\
&             Current satellite data 
               streams (Sea WIFS, CZS, 
               AVHRR, MODIS) & & & \\ \hline

USM &          TBD & & & \\ \hline

University of Texas Marine Science Institute &
Archived ocean station data & & & \\
&     Long records of tidal height,  
         surface T \& S & initial   &   03/01  &   yes \\
&      Tidal predictions, available
      via web site &   &                03/01  &   yes \\
&     Records of effects of storms
               on sea level, available via 
               web site  & & &  no  \\
&             Marine mammal, turtle, and bird 
               stranding data & & & \\ \hline

             \end{longtable}


\section{Agenda}
\label{I,agenda}

\begin{center}
  Gulf of Mexico Regional Workshop on an Integrated Data System for Oceanography
31 October - 2 November 2000
Naval Oceanographic Office
Stennis, Mississippi
\end{center}

\subsection{31 October 2000}

\subsubsection{Opening and welcoming remarks}

\subsubsection{Objectives of Gulf Component of an Integrated Ocean Data System}

\begin{itemize}
\item    Enhance data set discovery, sharing, and access through the use of web-based DODS servers 
and DODS-enabled applications
\item     Increase the number of Gulf of Mexico data sets available in a readily usable format
                Historical
                Real time
\item     Identify suite of data types of special interest to Gulf workers
\item     Foster collaborations on Gulf of Mexico region specific issues
\end{itemize}

\subsubsection{DODS as a capability}

   Briefing and demonstration of DODS server capabilities by University of Rhode Island 
personnel
\begin{itemize}
\item What is DODS?
\item                 Interoperability and metadata
\item                 What is required to serve data?
\item                 What is required to locate and access data?
\item                 DODS and web browsers
\item                 DODS-aware applications (MATLAB, Excel, IDL,..)
\item                 What types of data can/cannot be served?
\item                 Future of DODS applications/servers
\end{itemize}

\subsubsection{Gulf Regional Activities---Ongoing and Planned Activities}

Presentations by participants representing institutions.
\begin{itemize}
\item     Naval Oceanographic Office
  \begin{itemize}
  \item Northern Gulf Littoral Initiative
  \item Web-based server initiative
  \end{itemize}
\item     Florida State University
  \begin{itemize}
  \item Special Quikscatt products for the Gulf
  \item High-resolution Gulf model using NCOM
  \end{itemize}
\item     Louisiana State University
  \begin{itemize}
  \item Earth Scan Laboratory remote sensing capabilities
  \item WAVCIS- Wave-current surge information system
  \item BAYWATCH: The Vermilion-Cote Blanche Bay Physical Measurements Program
  \end{itemize}
\item     Mississippi State University
  \begin{itemize}
  \item Distributed Marine-Environment Forecast System
  \end{itemize}
\item     Texas A\&M University
  \begin{itemize}
  \item MMS-sponsored activities and data archive
  \item Modeling activities
  \end{itemize}
\item     Texas General Land Office
  \begin{itemize}
  \item Texas Automated Buoy System
  \end{itemize}
\item     Texas Water Development Board
  \begin{itemize}
  \item Bays and Estuaries Program - data collection and dissemination
  \item Oil spill and other modeling activities
  \end{itemize}
\item     University of South Florida
  \begin{itemize}
  \item Remote sensing capabilities
  \item Tampa PORTS
  \item Coastal Ocean Measurement and Prediction System
  \end{itemize}
\item     AOML
\item     University of South Florida
\item     University of Alabama
\item     Eddy Joint Industry Project
\item     University of Texas, Marine Science Institute
\item     TAMU Corpus Christi
\item     Tulane
\item     University of Southern Mississippi
\item     University of New Orleans
\item     Minerals Management Service
\item     National Coastal Data Development Center
\end{itemize}

\subsection{1 November 2000}

\subsubsection{Gulf Regional Activities (Continued)}

\subsubsection{Regional Partnerships}
\begin{itemize}
\item     Motivation for becoming a partner
\item     NOPP-DODS support for regional partners
\item     Summary of datasets to be shared (which are of interest; which do you have to share?)
\item     General versus region-specific datasets
\item     Selection of data types of special regional interest (perhaps, fresh water distributions or 
measures of hypoxia) and what metadata are needed
\item     Interests in cooperative studies
\end{itemize}

\subsubsection{DODS Questions}
Having considered data sets that are available and might be served, this is an opportunity to 
return to specifics of DODS. Issues such as the following may need further consideration:

\begin{itemize}
\item Central/distributed servers (resources)
\item         Data quality control
\item         Metadata
\item         Recognition for data contributions
\item         Security
\item         Sub sampling and bandwidth for large datasets
\item         Catalog and data discovery
\item         Continued data availability
\end{itemize}

\subsection{2 November 2000}

Decisions regarding future activities

\begin{itemize}
\item     Assign DODS server installations. Some support is available via the NOPP project.
\item     Assign data sets to be served as practical demonstration of regional interest and participation
\item     Regional organization needed for future development of an integrated regional-national international data system for oceanography
\item     Needs for regional activities (e.g., web-based data information center giving pointers to data sets, or assistance with server installations)
\item     Select representatives to national workshop
\end{itemize}

\subsubsection{Open Discussion and Wrap up}

\subsubsection{Adjourn Meeting}

\section{Meeting Attendees and Affiliations}
\label{I,attendees}

\begin{center}
\begin{tabular}[t]{ll}
Tony Amos &          University of Texas Marine Science Institute \\
Landry Bernard &     NAVOCEANO \\
John Blaha &         NAVOCEANO \\
Jim Bonner &         TAMU-CC/TEES \\
Jim Braud &          NAVOCEANO \\
Richard Campanella & Tulane/Xavier Center for Bioenvironmental
Research \\
Jim Corbin &         MSU ERC/CCS \\
Steve Foster &       MSU ERC/IDSL \\
Jim Fritz &          TPMC \\
Mike Garcia &        SAIC/NDBC \\
Norman Guinasso &    GERG/Texas A\&M University \\
Martha Head &        NAVOCEANO \\
Dan Holloway &       University of Rhode Island \\
Matthew Howard &     Texas A\&M University \\
Stephan Howden &     University of Southern Mississippi \\
George Ioup &        University of New Orleans, Stennis \\
Peter Lessing &      NDBC \\
John Lever &         NAVOCEANO \\
Alexis Lugo-Fernandez & Minerals Management Service \\
Mark Luther &        University of South Florida \\
Melanie Magee &      Gulf of Mexico Program \\
Robert ``Buzz'' Martin & Texas General Land Office \\
Eugene Meier &       Gulf of Mexico Program \\
Patrick Michaud &    TAMU-CC/CBI \\
Bob Molinari &       AOML/NOAA \\
Steven Morey &       COAPS/Florida State University \\
Frank Muller-Karger & University of South Florida \\
Doug Myhre &         University of Soutn Florida \\
Worth Nowlin &       Texas A\&M University - NAVOCEANO \\
Jim O'Brien &        COAPS/Florida State University \\
George Rey &         LEAG/CBR \\
Reyna Sabina &       AOML/NOAA \\
Mitch Shank &        NAVOCEANO \\
Ruben Solis &        Texas Water Dev. Board \\
Susan Starke &       NCDDC/NOAA \\
Vembu Subramanian &  University of South Florida \\
Molly Sullivan &     Tulane University \\
Jack Tamul &         NAVOCEANO \\
William Teague &     NRL \\
Nan Walker &         Louisiana State University \\
Patti Walker &       DATASTAR/NDBC \\
\end{tabular}
\end{center}

\section{Introduction to NOPP-sponsored National Data Hub project by Worth 
Nowlin}
\label{I,presentation}

\subsection{Project Goal and Objectives}

Plan and implement a ``Virtual Ocean Data Hub'' (VODHub) system that will provide users with 
the ability to ``easily access certain data types in specified locations/times, regardless of data 
source, and without special efforts or insights on the part of the user about the data source(s)''.

Specific research objectives identified in the BAA are:

\begin{enumerate}
\item Identify individuals/organizations that will take the lead to foster development of 
community-based conventions for specific data types.
\item Partner with local institutions (public and private) to improve access to coastal and regional 
data via community-based conventions.
\item Enhance connections to existing national and international archives of ocean data as well as 
the program offices of major ocean programs (e.g., WOCE, JGOFS) via developed 
community based conventions.
\item Partner with international groups to foster a worldwide ``ocean data dictionary'' to further 
commonality of access for all sources of ocean data.
\item Work with national and international user standards groups (e.g., International Hydrographic 
Organization and their Electronic Navigation Charts) to foster access to ocean data via a 
growing number of user interfaces.
\end{enumerate}

\subsection{Phases}

\subsubsection{Year 1: The Planning Phase}

\begin{itemize}
\item     Regional Workshops (5)

\item     Synthesis of regional workshop results

\item     National Workshop

\item     Prepare Final Recommendations
\end{itemize}

\subsubsection{Years 2 \& 3: Implementation}

\subsection{Regional Workshops}

Asked to consider the following issues:

\begin{enumerate}
\item Is DODS data model adequate for datasets to be served in region?
  What additions are required?
  
\item What are important interface issues for regional users? From
  data discovery to use.
  
\item What types of semantic metadata will be required? What standards
  will be used? Focus on search and use metadata.
  
\item What datasets will be served initially as part of this effort?
  Seed the system? What assistance is needed?
  
\item Is a regional node needed for coordination? Data location,
  reference for user support, etc.
\end{enumerate}

\subsection{National Workshop}

\subsubsection{Background}
        Synthesize results of regional workshops. Study by Executive Committee.

\subsubsection{Objective}
        Develop the implementation plan for years 2 and beyond.

\subsubsection{Approach}
        Address same issues as at regional workshops, with synthesis as background.

\subsubsection{Timing}
        Nine months after project began

\subsubsection{Participants}
        Project management
        Regional coordinators
        Representative from federal agencies with holdings, e.g., NAVO, NODC
        Representative from national organizations of marine institutions, e.g., NAML, Sea Grant 
\subsubsection{Program}
        Representative from international organizations, e.g., Australian BMRC
        Representative from industry with significant data holdings
        Experts with data systems/networks
        Representative from other NOPP projects developing model/assimilation capabilities

\subsubsection{Review}
        Workshop recommendations to be reviewed by community


\subsubsection{Implementation Plan}

Three principal areas will be:

\begin{description}
\item[Population]
Population refers to addition of data to system. Focus on in situ data sets --- particularly regional.

\item[System Core]

Areas of expansion considered necessary are:
\begin{itemize}
\item      Development of GIS clients
\item     Work with virtual data sets
\item     Web interfaces
\item     Improve Data Access Protocol
\end{itemize}

\item[System Maintenance]
Including system documentation and user support
\end{description}

\section{Presentation by Dan Holloway - University of Rhode Island}
\label{I,DODS}

(\Url{http://po.gso.uri.edu/\~{}dan/dods-regional-workshops/dods-regional-workshops.html})

The URL listed above is a link to the presentation given by the DODS group at the Gulf of 
Mexico NOPP regional workshop.

Following is a summary of a number of important points discussed during the presentation.

\subsection{Introduction to DODS}

DODS is an open source software project designed to allow users to easily access and move 
data over the network.

The DODS project has two underlying principles it adheres to:

\begin{itemize}
\item Anyone willing to share data should be able to do so via DODS; e.g., the scientist, a state 
agency, a private company, or a federal data center.
\item Users should be able to use the application package with which they are most familiar to 
examine or analyze the data of interest.
\end{itemize}

How these principles are incorporated into the design are as follows:

\begin{itemize}
\item Data providers must not be required to store their data in any special format.
\item The data system must require a minimum set of metadata from the data provider.
\item The DODS core software must be easy to interface to existing applications so those scientists 
can use the packages they are most familiar with.
\item The data system must provide all the metadata that is required to effectively use the data in 
the client application.
\end{itemize}

The requirement to provide the minimal set of metadata required to use the data directly 
competes against the requirement to not burden the data provider with meeting specific metadata 
requirements.

The DODS project has taken a bottom-up approach toward solving the distributed data access 
problem. Rather than focus on the directory level, or data location aspect of the distributed data 
access problem, DODS has focused on the data-level interoperability. As part of that effort, the 
DODS project has delimited the metadata requirements for distributed data access into `use' and 
`search' metadata. To achieve data-level interoperability a definition of `use' metadata was 
formulated by segregating the data-level metadata requirements into syntactic and semantic `use' 
metadata. The DODS core software provides a strict syntactic 'use' metadata representation in 
the data transmission component of the software. This is required in order for software 
components on the client-side to be able to decipher the encoded binary data stream. The DODS 
core software supports but does not require additional semantic `use' metadata from the data 
provider. However, as the level of metadata increases, both `use' metadata and `search' metadata, 
the degree of interoperability that can be attained with remote datasets increases. It is important 
to note that all of the more advanced client applications using DODS to access remote data have 
specific semantic `use' metadata requirements, but none of these go so far as to require FGDC 
CSDGM metadata fields.

During the presentation two DODS client applications were demonstrated.

\subsubsection{NOAA's Live Access Server}

The Live Access Server (LAS) is a web-server application that provides an interface to data 
stored at NOAA's Pacific Marine Environmental Lab in Seattle, and a number of DODS served 
datasets located at NOAA's Climate Diagnostic Center in Boulder, and the International 
Research Institute at Columbia's Lamont-Doherty Earth Observatory.

This web-server uses a DODS-enabled client application, Ferret, to retrieve data from these 
remote sites and process it into one of the available representations defined in the interface. 
Ferret is a notable example of a scientific application whose functionality has been extended by 
enabling it to access remote datasets via DODS.

The strength of this interface is that it provides a standardized interface to a relatively large 
number of gridded datasets. Additionally, it can be easily customized for use at other sites.

The weakness of this approach is that it predefines the range of analysis possible on the remote 
datasets. The interface does permit the data to be easily downloaded, but the data must then be 
ingested into the scientist's application for further study.

\subsubsection{DODS Matlab GUI}

The Matlab GUI was developed as a testbed application to better understand the problems 
associated with building applications based on distributed data access. The primary result of this 
effort has been to better understand the semantic `use' metadata requirements for data-level 
interoperability. The Matlab GUI has limited, though strict semantic `use' metadata requirements.

The Matlab GUI provides an interface to two sets of oceanographic data, global datasets 
providing Sea Surface Temperature, Winds, etc., and local in-situ datasets for the Gulf of Maine 
GLOBEC project. These two sets present different problems to building applications using 
DODS.

The strength of this interface is that it provides direct access to the data in the scientist's analysis 
application. Once the GUI has selected and retrieved the remote data, the data is now located in 
the scientist's workspace for further analysis, or saving to local files; no additional steps are 
required.

The weakness of this interface is that it can require a high-level of effort to add new datasets to 
the interface. However, this constraint has led to better understanding of the various aspects of 
`use' level metadata requirements.

\subsection{DODS current development activities}

\begin{itemize}
\item  The DODS core software has been ported to Java.
\item A JDBC server has been built and is currently in beta test at OSU and MBARI.
\item A GrADS server has been built using the Java port by the COLA group. This server can 
provide access to GRIB data via DODS. Additionally, this server can be used to process data 
remotely, returning the results of those operations as DODS datasets.
\item There is a native Windows port of the DODS core, as well as a Cygwin port of the software 
for PCs.
\item There is a WMT-DODS gateway prototype in development at NASA's Goddard Space Flight 
Center.
\item The DODS Matlab GUI is being rewritten to facilitate adding new and n-dimensional 
datasets, and to allow the user to easily customize the interface for their use. 
\end{itemize}

\subsection{DODS future development activities}

\begin{itemize}
\item The project is actively pursuing migrating the core's data access protocol (DAP), to use XML 
as an encoding scheme.
\item DODS is working with NGDC and ESRI to make DODS data accessible to GIS clients, and 
GIS data accessible to DODS clients.
\item The project is investigating designs to support tertiary metadata servers, as well as client-side 
metadata support, to permit interoperability for datasets with limited associated metadata.
\item A web-crawler will be designed and implemented as part of the NOPP project to assist data 
location.
\end{itemize}

\subsection{Comments on DODS}

The DODS project has taken a bottom-up approach in proposing solutions to the distributed data 
access problem. At its current stage of development the project is focused on data-level 
interoperability in that problem domain. Goals of the project have been to permit scientists to 
easily serve, and access data over the Internet. To accomplish this the software must not require 
any reformatting of data by providers, and permit direct access to data within the scientist's 
existing analysis environment. To reach these goals the project has selected several client 
applications to demonstrate the concept, Matlab, idl, and NetCDF. The project also identified a 
number of commonly used data formats, as well as packages which support access to PI help 
datasets. Arguably, pre-built servers for every data format used in the science community are not 
currently provided by the DODS project, but as has been demonstrated by various groups writing 
DODS servers can be a realistic alternative.

At its current stage of development DODS does provide direct access to data within a number of 
scientific analysis environments. Unfortunately, given the lack of standardized interfaces for 
searching for data within the community at large, or within specific data repositories, 
formulating DODS URLs to access datasets can be difficult. One goal of an integrated data 
system should be to work toward standardized mechanisms for accessing distributed data, which 
includes both ease of location and use of those data.

\section{Regional Presentations}
\label{I,regions}

\subsection{Presentation by John Blaha -- Naval Oceanographic Office}

(Used overheads)
 
\subsubsection{Northern Gulf Littoral Initiative (NGLI)}

Purpose:
\begin{itemize}
\item Develop reliable, multidisciplinary models of the Mississippi Sound and adjoining rivers, 
bays, and shelf waters through the operation of a sustainable forecasting system.

\item Has worked with numerous groups in the past.

\item Develop skill toward shore from sea.

\item First year was a modeling effort, some RS and in-situ observations, tied into Navy Hub.

\item Slide showing map of region.

\item Bathymetry is the Navy's business.

\item Slide of intrusions map.

\item Nested palm 200 to 500 meter mesh.

\item Slide of model generation, banding effects of shelf area.

\item Slide of basin model.

\item Slide In-situ domain (measurement infrastructure).

\item Slide of sediment transport. 

\item Slides of drifter data.

\item Slides of ADCP locations.

\item Slide of Altimeter Calibration.

\item Slide of Local Geoid - SST data.

\end{itemize}

Question: On what DODS node will data appear?

\begin{itemize}
\item Outside of firewall on MEL Server.
\end{itemize}

\subsection{Presentation by John Lever - NAVOCEANO}

(Presentation on Power Point)

Possible venues of DODS

\begin{description}
\item[Slide] Objective Functional Architecture.

\item[Slide] NAVOceano Data Distribution.
\begin{itemize}
\item Web page showing available products.
\item Data warehouse.
\end{itemize}

\item[Slide] Web Server flow chart.
\begin{itemize}
\item Access to data warehouse.
\item NGLI link.
\end{itemize}

\item[Slide] Potential DWH Architecture WEB Proxy.

\item[Slide] Dissemination Architecture.
\begin{itemize}
\item NAVO LAN.
\item Firewall DMZ.
\item Internet
\end{itemize}

\item[Slide] Master Environmental Library  (MEL).
\begin{itemize}
\item MEL v3.0.
\end{itemize}


\item[Slide] MEL Expanded Flow Chart.

\item[Slide] MEL data discovery and delivery.

\item[Slide] DWH Mass Storage.
\begin{itemize}
\item Purpose is for service production, processing and selected archiving of NAVOCEANO data sets.
\end{itemize}

\item[Slide] FY01 Mass Storage Acquisition.
\begin{itemize}
\item       High-Volume Near-line Storage.
\item       On-line Storage.
\item Network Connectivity Upgrades to Data Servers.
\item         Network Attached Storage.
\end{itemize}

\item[Slide] High-Volume Near-line Storage.

\item[Slide] Online Storage.

Many other partners involved.

Metadata is FGDC compliant.  Metadata not necessarily tagged with data can be downloaded if 
desired.
\end{description}

\subsection{Presentation by Jim O'Brien -  COAPS, Florida State University}

(Presentation on PowerPoint)

Modeling the Gulf of Mexico with Satellite Winds

Several slides of example animations:
\begin{itemize}
\item     Showing Near-real-time Winds and Surface Pressures.
\item     \Url{http://www.coaps.fsu.edu/\~{}zierden/qscat} AND \Url{http://www.coaps.fsu.edu/\~{}zierden/qscat/gulf.shtml}
\item     Swath data are used, not gridded data or interpolated fields.
\end{itemize}

SeaWinds on QuckSCAT Satellite.

SEE  \Url{http://www.coaps.fsu.edu/cgi-bin/qscat/animations.cgi?request=listr\&region=smex}

Sea Winds Daily Coverage.

Examples of Sea Winds Overpass.

Two products available:
\begin{itemize}
\item     Research. SEE \Url{http://www.coaps.fsu.edu/scatterometry/Qscat/gridded.shtml}
\item     Operational. SEE \Url{http://manati.wwb.noaa.gov/quikscat/}
\end{itemize}

Research Quality Gridded Winds:
\begin{itemize}
\item     Global, six-hourly, gridded winds.
\item     Currently 1x1 degree.
\end{itemize}

Gulf of Mexico Modeling Goals with the NCOM:
\begin{itemize}
\item     Get a better understanding of meso-scale dynamics.
\item     Examine two-way ocean interaction between the continental shelf and the Gulf of Mexico basin.
\item     Model oceanic response to energetic episodic forcing in upper ocean wave stratification.
\item     Apply improved modeled physics to ocean nutrient distribution.
\item     Improve ocean prediction capabilities in Gulf using SeaWinds scatterometer data and 
TOPEX/Poseidon altimeter data.
\end{itemize}

Have a 6-min. model, currently running with a 3-min. model, eventually will have local model 
going down to 1 km. Eventually all of Gulf done in 1-km model.

Will deliver half by half degree, every 6 hours gridded winds for the Gulf of Mexico for this 
project using Seawinds satellite.
 
Lots of animation on web site.
Web site:     \Url{http://www.coaps.fsu.edu}

Goal is to have very good model runs.

\subsection{Presentation by Nan Walker - LSU}

(Used overheads)

Main data sources of the Coastal Studies Institute:

\begin{itemize}
\item Real time data.
\begin{itemize}
\item Earth Scan Lab satellite imagery.
\item WAVCIS wave-current data.
\item BAYWATCH physical measurements program.
\end{itemize}

\item Historic data archives.
\begin{itemize}
\item Estuarine time series.
\item LATEX inner shelf.
\item Physical measurement in hypoxia region.
\item Satellite imagery.
\end{itemize}
\end{itemize}

Earth Scan Laboratory of Coastal Studies Institute was started in 1988.

NOAA AVHRR:
\begin{itemize}
\item Detects suspended sediments, temp.
\item Spatial resolution is 1 square km.
\end{itemize}

Orbview:
\begin{itemize}
\item Used for detecting chlorophyll-a suspended, and other suspended sediments.
\end{itemize}

Projects: GOES-8GVAR
\begin{itemize}
\item Temp @ 4x4 square km. and water vapor @ 8x8 square km.
\item Has visible channel @ 1km.
\end{itemize}

Slide - Composite image.

Project involvement:
\begin{itemize}
\item EPA EMPACT project.
\item MODIS Terra and Aqua (will be getting X-Band soon).
\item RADAR- ERS-2, Radarsat (future).
\item IRS-P4 Ocean Color (future).
\end{itemize}

WAVCIS Project (headed by Greg Stone):
\begin{itemize}
\item Focus is on LA shelf in Mississippi Sound.
\item Real time, high resolution of waves and currents.
\item Wave-current surge information system for coastal LA.
\item Good for comparing wave data.
\item No conductivity or surface temp. data.
\item Using cellular/satellite communication.
\item Using acoustic Doppler tech.
\end{itemize}

Benefits associated with WAVCIS program:
\begin{itemize}
\item Directional waves measurement.
\item Current velocity.
\item Water level storm surge.
\item Wind speed and direction.
\item Using data with other groups.
\end{itemize}

Has a website where you can pick parameters and do analyses.

Fundamental research.

BAYWATCH Program (Funded by US Army Corps), Measures:
\begin{itemize}
\item Currents.
\item Water level.
\item Turbidity.
\item Salinity/Conductivity.
\item Wind speed/direction, air pressure/temperature.
\end{itemize}

Showed a GOES-8 ocean temperature animation loop of circulation in the Gulf of Mexico.

LSU is open to possibility of serving data via DODS.

\subsection{Presentation by Frank Muller-Karger and Doug Myhre - University of South Florida}

(Used PowerPoint file)

Remote Sensing Capabilities at USF

USF Datasets:
\begin{itemize}
\item Satellite.
  \begin{itemize}
  \item Remote sensing lab.
  \end{itemize}

\item Field.
  \begin{itemize}
  \item PORTS
  \item COMPS
  \item ECOHAB
  \item (+ numerous field programs)
  \end{itemize}

\item Satellite Sensors:
  \begin{itemize}
  \item Current AVHRR.
    \begin{itemize}
    \item NOAA-12
    \item NOAA-14
    \item NOAA-15
    \item NOAA-16
    \end{itemize}
  \item SeaWIFS
  \item TERRA/MODIS
  \end{itemize}

\item Historic (since '93):
  \begin{itemize}
  \item AVHRR
  \item SeaWIFS
  \item CZCS (1978-1986)
  \end{itemize}

\item Coverage:
  \begin{itemize}
  \item Land (southeastern US, Central America, northern South America, and Caribbean Islands).
  \item Ocean (Gulf of Mexico, Caribbean Sea, eastern tropical Pacific).
  \end{itemize}

\item Primary Data streams:
  \begin{itemize}
  \item Terascan system (L-Band)
    \begin{itemize}
    \item Orbview -II/SeaWIFS
    \item POES/AVHRR
    \end{itemize}

  \item Apogee Solutions (X-Band)
    \begin{itemize}
    \item Terra/MODIS
    \end{itemize}
  \end{itemize}


\item Volume of Data Collected:
  \begin{itemize}
  \item CZCS  \~{}20 GB
  \item AVHRR  690MB/day
    (3 satellites, 12 passes/day)
  \item SeaWIFS 150MB/day
    (1 pass/day)
  \item MODIS
    (4 passes/day)
  \end{itemize}

\item Current Products:
  See website, \Url{http://paria.marine.usf.edu}


\item Secondary data streams, X-Band:
  \begin{itemize}
  \item Long term
    \begin{itemize}
    \item ENVISAT (MERIS)
    \item ADEOS-II (GLI)
    \item NPOESS
    \end{itemize}
  \item (FMK is member of ENVISAT, ADEOS-II Science teams)
  \end{itemize}

\item Data Archived:
  \begin{itemize}
  \item AVHRR, all raw data and products
  \item SeaWIFS, all raw data and products
  \item MODIS, products
  \end{itemize}


\item Data Archives:
  \begin{itemize}
  \item Data Stored on 12'' Worm Optical Disk.
    AVHRR raw data.
  \item CD-ROM 600 disk jukebox (300gb Capacity)
    SeaWIFS Raw and LO data.
  \end{itemize}

\item Other data archives:
  \begin{itemize}
  \item DVD-R
  \item DLT Tape (future).
  \item MODIS Raw Data.
  \end{itemize}


\item Samples shown on overheads


\item USF requirements:
  \begin{itemize}
  \item Coastal shelf salinity/hydrography.
  \item Coastal shelf winds.
  \item Coastal shelf currents.
  \item Offshore currents for shelf-break force studies.
  \item River discharge and nutrient concentration.
  \item Yucatan Strait transport.
  \item Strait transport.
  \end{itemize}


\item Possible Cooperative Efforts:
  \begin{itemize}
  \item Integration of FL COMPS/TEXAS and other in-situ observing systems with real-time remote 
    sensing systems.
  \item Integration of above with regional modeling efforts.
  \end{itemize}


\item USF Plans:
  \begin{itemize}
  \item NOWCAST circulation model of West Florida Shelf.
  \item Integrated current, wind field, and real time satellite data product (dynamic vector overlays).
  \item Surface flux estimates.
  \item Collect data from future sensors.
  \end{itemize}

\item Worries:
  \begin{itemize}
  \item Nice to have DODS, but where does this take us in the long-term with data that each of us 
    depend on and that are not available via centralized legacy systems?
  \item DODS effort should focus on a few, select datasets considered critical for long-term archival.
  \end{itemize}
\end{itemize}


\subsection{Presentation by George Ioup - University of New Orleans}

(Spoke from notes)

Some surveys are site-specific and not very regular.  Data for some models are sparse.

\begin{enumerate}
\item UNO involved with Lake Pontchartrain Basin Project.
 
Making data public is the goal, on website at a minimum.

Some data include:
\begin{itemize}
\item Water turbidity.
\item Conductivity.
\item Temperature.
\item Chlorophyll A.
\item Orleans Parish rain chemistry.
\item Circulation models.
\item Fecal coliform bacteria.
\item Sediment chemistry.
\item Lake bathymetry.
\end{itemize}

Data can potentially go on DODS.  Lake Pontchartrain Atlas to be published.

Contacts:  
\begin{itemize}
\item Professor Alex McCorquodale, Civil and Environmental Engineering
\item Professors Mike Porrier and Bob Cashner, Biological Sciences
\item Professor Shea Penland, Geology and Geophysics
\end{itemize}

\item Barataria Basin Project

UNO, LSU, LSU-Ag, LUMCON, Tulane, Dillard

Environmental measurements in Barataria Basin related to carbon cycling in a coastal estuary.

Making data public is the goal, on website at a minimum.
Data can potentially go on DODS.

Contact:  Professor Ken Holladay, Mathematics

\item Louisiana and Northern Gulf Coast Coastal Measurements

Many measurements made with USGS, Corps of Engineers, National Marine Fisheries, 
Louisiana Department of Natural Resources, and other sponsorships.  Data already available on 
website.  Data can potentially go on DODS. 

Contacts:  Professors Shea Penland and Denise Reed, Geology and Geophysics
        
\item Northern Gulf of Mexico Acoustics Data

USM, NRL-Stennis, UNO

Making data public is goal, on website at a minimum.  Data can potentially go on DODS.  

\begin{itemize}
\item  Ambient noise measurements.
\item Marine mammal acoustic measurements.
\item Acoustic propagation, including propagation through fronts and eddys.
\end{itemize}

Contacts: \begin{itemize}

\item  Professors George and Juliette Ioup, Physics
\item Professor Grayson Rayborn, Physics and Astronomy, USM
\end{itemize}
\end{enumerate}

\subsection{Presentation by Jim Corbin - MSU ERC/SSC/IDSL}

\subsubsection{Distributed Marine Environment Forecast System - DMEFS}

The proposed DMEFS will be a research testbed for demonstrating the
integration of various technologies and components prior to DoD
operational use, and an ``open'' framework in which to operate
validated climate, weather, and ocean (CWO) models The focus is:


\begin{itemize}
\item To create an open framework for a distributed system for describing and predicting the 
marine environment that will accelerate the evolution of timely and accurate forecasting. 
\item To adapt distributed (scalable) computational technology into oceanic and meteorological 
predictions, especially on the regional and tactical scales
\item To shorten the model development time by expanding the
collaboration among the model 
developers, the software engineering community and the operational end-users. 
\item To provide a system with the look and feel of an operational entity in terms of infrastructure 
but with advanced computers, software technology and none of the constraints of an 
operational environment. To provide unique cross-cutting capabilities (i.e., software 
integration technologies and support resources) to DoD and yet let DoD simultaneously 
leverage existing internal expertise and investments, including legacy components (e.g., 
ocean models, atmospheric models, tools, etc.). 
\item To provide a framework that is extensible and designed for rapid prototyping, validation, and 
deployment of new models and tools and be operational over evolving heterogeneous 
platforms distributed over wide areas with web-based access of forecast-derived information.
\item Middle tier servers that form a distributed, shared, persistent, collaborative environment for 
model development, validation, coupling, deployment and operational use. Implemented as a 
multi-tier, object oriented system. The middle tier components act as proxies of services 
rendered by the back-end resources.
\end{itemize}

The Web Portal for DMEFS provides a seamless web access to remote resources through secure 
kerberized CORBA channels hiding complexity of the high performance, heterogeneous back-
end systems:
\begin{itemize}
\item Primary operational user is Naval Oceanographic Office
\item Primary research user is the Naval research Lab Stennis
\item Develop and/or run met/ocean models
\item Validate METOC models
\item Create METOC support products from model runs and other data
\end{itemize}

Provides a graphical problem-solving environment to do problem solving from workstation, 
desktop, or laptop. A core set of services will be provided by the DMEFS architecture to support 
the development of meta computing applications including:
\begin{itemize}
\item Resource Management, Discovery, and Access Control
\item Security and Access Control
\item Transaction Services
\item Communication Services
\item Event and Notification Services
\item METOC Digital Library and Data Services
\end{itemize}

Web Browser based front end that can:
\begin{itemize}
\item Couple models.
\item Develop models.
\item Visualize results.
\item Access external data. 
\item Deploy models.
\item Set schedules.
\item Provide operational execution and access operational data.
\end{itemize}

\subsection{Presentation by Worth Nowlin - Texas A\&M}

MMS-sponsored activities from which data are or will be available include:

\begin{itemize}
\item LATEX
\item Northeast Gulf of Mexico Chemical and Hydrographic Study
\item CHEMO I and II
\item MAMES I and III
\item Deep Gulf of Mexico Benthic Biology
\end{itemize}

Also available are archives of:
        Ocean station, XBT, MBT, and AXBT data, current measurements (moored, drifters and 
shipboard ADCP)

Ancillary data archives include:
\begin{itemize}
\item Daily river discharge from U.S. rivers
\item Surface observations via GTS
\item Analyzed surface wind fields.
\end{itemize}

Additional data sets include:
\begin{itemize}
\item EPA EMAP data focused on contaminants
\item NOAA Status and Trends data for 14 years
\item National Estuarine Product data from Galveston and Corpus
Christi Bays
\end{itemize}

TAMU agrees to set up a DODS server and initially serve data from LATEX A and daily river 
discharge files.

\subsection{Presentation by Norman Guinasso - representing the Texas General Land Office}


\subsubsection{Texas Automated Buoy System (TABS)}

\begin{itemize}
\item Initiated as an Interagency Research Contract with Texas A\&M University (TAMU) in 1994.
\item Work is carried out at GERG and Department of Oceanography, TAMU
\item The mission is prediction of oil spill movement along Texas
coast.
\item First buoys were installed in 1995.
\item Currently operate seven buoys with single point current meters and ocean surface 
temperature sensors.
\item Data sampled at 30 minute intervals and served in near real real-time on the Internet  
\item Fulfills operational need of Texas General Land office.
\end{itemize}

Operational Elements of TABS with web pages:
\begin{itemize}
\item Operation of buoys and data presentation at \Url{http://www.gerg.tamu.edu/tglo}.
\item Continuous assembly of meteorological data for modeling purposes
  at 

\Url{http://seawater.tamu.edu/tglo/}
\item Forecasts of ocean currents using POM and Spectral numerical
models of ocean currents at 
\Url{http://seawater.tamu.edu/tglo/}
\end{itemize}

How TABS data and predictions are used for tactical oil spill response 

\Url{http://resolute.gerg.tamu.edu/\~{}norman/TABS-DODS.htm}  (Slide 1)

TABS data was effectively used to reduce the cost of the response to Buffalo Barge 292 Oil Spill 
in 1996. Knowledge of ocean currents prevented a massive mobilization of resources along the 
northern Texas Coast off Sabine and allowed cleanup efforts to be directed further down the 
coast. 
\Url{http://resolute.gerg.tamu.edu/\~{}norman/TABS-DODS.htm}  (Slide 2)

Operate two kinds of buoys, anchored in place by heavy chain.
\begin{itemize}
\item TABS I , TABS II 
\Url{http://resolute.gerg.tamu.edu/\~{}norman/TABS-DODS.htm}  (Slide 6)

\item TABS II with ADCP and meteorological package

\Url{http://resolute.gerg.tamu.edu/\~{}norman/TABS-DODS.htm}  (Slide 7)
\end{itemize}

\subsubsection{TABS buoy data communications.}

TABS I or TABS II
\begin{itemize}
\item Motorola Integrated by cell phone. Cell phone service provided
over most of central GOM 
shelf by Petrocom 
\Url{http://resolute.gerg.tamu.edu/\~{}norman/TABS-DODS.htm}  (Slide 8)
\end{itemize}

TABS II 
\begin{itemize}
\item    Westinghouse 1000 Satellite Cell Phone at 4800 baud
\item Buoys make digital data calls using NorcomNetworks X.25 network
  connecting to computers at Texas A\&M University
\end{itemize}

TABS data

Available on web page as graphics and downloadable ASCII files. Historical data available as 
Web CGI enquiries on TABS database.  Historical enquiries produce graphics or downloadable 
files.

Cost of Data Transmission using Petrocom Cell Phone Network
\$1.09 per minute

\begin{center}
\begin{tabular}{|p{.75in}|p{.6in}|p{.6in}|p{.6in}|p{.6in}|p{.6in}|}  \hline
& \textbf{Bytes per reading} & \textbf{Reading Interval} & \textbf{Bytes per day} & \textbf{Bytes per month} & \textbf{Data cost per month} \\ \hline
Single point current meter &   149  &  30 minutes &  7,152  & 214,560 & \$196 \\ \hline
\end{tabular}
\end{center}


Cost of Data Transmission using NORCOM Satellite Network
\$185 per month for first 500,000 bytes,  \$0.38 per 1000 additional bytes

\begin{center}
\begin{tabular}{|p{.75in}|p{.6in}|p{.6in}|p{.6in}|p{.6in}|p{.6in}|}  \hline
& \textbf{Bytes per reading} & \textbf{Reading Interval} & \textbf{Bytes per day} & \textbf{Bytes per month} & \textbf{Data cost per month} \\ \hline
Single point current meter &   149  &  30 minutes &  7,152  & 214,560 & \$185 \\ \hline
ADCP &                        1100  &   Hourly    & 26,400  & 792,000 & \$296 \\ \hline
\end{tabular}
\end{center}


\subsubsection{TABS Future Developments}

Winter, 2000-2001

\begin{itemize}
\item Install TABS II buoy with ADCP at Flower Garden Banks National
  Marine Sanctuary (FGBNMS)

\item Install 2 TABS II buoys with ADCPs in Mississippi Sound as part
  of Northern Gulf Littoral Initiative (NGLI)

\item Install meteorological package on three TABS II buoys

\item Install SeaBird Micro-Seacat TS sensor on  FGBNMS buoy
\end{itemize}

Spring-Summer 2001

\begin{itemize}
\item Replace electromagnetic single point current meters with
  Aanderaa 2D Doppler Current meters Install and test Falmouth
  Scientific NXIC conductivity sensor on TABS buoy
\end{itemize}

2001-2002

\begin{itemize}
\item Install Bottom mounted packages that communicate through surface
  buoys, to measure nutrients, dissolved oxygen, light, and other
  parameters in near real time at two to four TABS sites along Texas
  Coast.
\end{itemize}

\subsection{Presentation by Ruben Solis - Texas Water Development Board}

Estuarine Hydrographic Surveys
 
(\xlink{http://hyper20.twdb.state.tx.us/data/bays\_estuaries/surveypage.html}{http://hyper20.twdb.state.tx.us/data/bays_estuaries/surveypage.html})

\begin{itemize}
\item Synoptic measurements collected in Texas bays from 1987 to 1997
\item Data includes physical (water level, velocity, flow) and quality (salinity, pH, DO, temperature) 
measurements
\item Data available through clickable map, in text and graphical format
\item Metadata included with data files
\end{itemize}

Coastal Hydrology (\xlink{http://hyper20.twdb.state.tx.us/data/bays\_estuaries/hydrologypage.html}{http://hyper20.twdb.state.tx.us/data/bays_estuaries/hydrologypage.html}): 
\begin{itemize}
\item Historical inflows (1940-1998, and being updated) for Texas bays
\item Monthly flows available in text format from web site
\item Rudimentary metadata available with data files
\end{itemize}

Sonde Data (\xlink{http://hyper20.twdb.state.tx.us/data/bays\_estuaries/sondpage.html}{http://hyper20.twdb.state.tx.us/data/bays_estuaries/sondpage.html}): 
\begin{itemize}
\item Continuous sonde measurements of salinity, pH, DO, temperature, and at some sites turbidity, 
water elevation
\item Two sondes/estuary typically deployed
\item Program initiated in 1986
\item Text data and graphs of data available on web site
\end{itemize}

Hydrodynamic and Oil Spill Modeling
 
(\xlink{http://hyper20.twdb.state.tx.us/data/bays\_estuaries/bhydpage.html}{http://hyper20.twdb.state.tx.us/data/bays_estuaries/bhydpage.html})
\begin{itemize}
\item Hydrodynamic models for Corpus Christi and Galveston Bays conducted daily
\item Animated output displaying currents and water elevations available on web
\item Short-term (1-day) forecasts displayed
\end{itemize}

\subsection{Presentation by Patrick Michaud - Conrad Blucher Institute for Surveying and Science; 
Texas A\&M University Corpus Christi}

\begin{itemize}
\item Texas Coastal Ocean Observation Network:
\begin{itemize}
\item Map of flags collecting data.
\item Zoomed on Corpus Christi.
\item Showed individual tidal stations.
\item Put in for water circulation and property boundaries. 
\item Collect mean tidal datum.
\end{itemize}

\item Entirely Web based.

\item Web Page of Port Aransas:
\begin{itemize}
\item List of station info and latest observations.
\item Water level.
\item Air temp.
\item Wind speed.
\item Etc.
\end{itemize}

\item Historical data are available.

\item Quality control done everyday.
\begin{itemize}
\item Must have continuous data stream.
\end{itemize}

\item Web page of graphs explained. 

\item Data query page, can get station they are interested in:
\begin{itemize}
\item Perform sophisticated data retrieval.
\item Can make adjustments (feet vs. meters).
\item View for mean sea level.
\item Compare two stations.
\end{itemize}


\item Mainly developed for internal use.

\item Can provide in raw data format.

\item Metadata sent back to requestor (cited numerous examples of using metadata). 

\item System does numerous checks and is completely automated.

\item Produces tidal datum.

\item Benchmark leveling, dates, equipment, etc.

\item Moving vector map demo.

\item Others have asked for tidal datum to be produced by their shop.

\item Tide gauge point source data is their forte.

\item Have potential to integrate into DODS.
\begin{itemize}
\item Need to produce capabilities to share data but would require significant investment.
\end{itemize}
\end{itemize}


\subsection{Presentation by Mark Luther - University of South Florida}

(Used overheads)

Coastal Ocean Monitoring and Prediction System (COMPS)

Gulf of Mexico Ocean Observing Systems - website

COMPS web site  \Url{http://comps.marine.usf.edu}:
\begin{itemize}
\item Map of real time observing sites on west coast of FL.
\item Owned and operated by USF or other agencies.
\item Four buoys btw. 20 - 50 isobaths, several coastal stations with water level, met, 
temp/salinity.
\end{itemize}

Can click on station to:
\begin{itemize}
\item Get photos.
\item Show metadata on station.
\item Show numerous parameter measurements by variable.
\end{itemize}

Have data for last 24 hrs in graphical format.
\begin{itemize}
\item Line of sight radio and/or GOES satellite telemetry.
\item Have a GOES downlink, it's free but slow.
\end{itemize}

Developed a custom data logger for ADCP, T/S, and met data, with GOES
telemetry and spread-spectrum radio.

Overheads of buoys:
\begin{itemize}
\item Have full suite of instruments - ADCP (downward-looking), wind
  speed/direction, air temp., humidity, barometic pressure,
  precipitation, incoming radiation (SW, LW), MicroCat T/S sensors.
\item Held by heavy chain and RR wheels.
\end{itemize}

Two types:
\begin{itemize}
\item Low-cost - telemeters met only via single GOES ID
\item Cadillac - telemeters ADCP and T/S data and met data on Dual GOES ID's.
\end{itemize}

Graphs from individual buoys during a storm.
\begin{itemize}
\item Various measurements from a meter below surface.
\end{itemize}

Began in 1997:
\begin{itemize}
\item Data archived in ASCII flat files.
\item Developing searchable database system
\end{itemize}

Tampa Bay PORTS, a sub-system of COMPS (\Url{http://ompl.marine.usf.edu/PORTS}.

Suite of Models based on Princeton model.
\begin{itemize}
\item Overhead showing models.
\item Oil spill trajectory modeling is also done.
\end{itemize}

Data is delivered from PORTS to harbor pilots via a Vessel Traffic Information System 
(\Url{http://www.rossdsc.com/ais.htm})

Overhead of Tropical Storm Josephine (Oct. 1996), color showing non-tidal component - 
simulated storm surge from numerical model of West Florida Shelf.
\begin{itemize}
\item Extensive flooding in Tampa Bay region from modest storm
\item See \Url{http://ocg6.marine.usf.edu/WeisbergSite/StormSurgePlot.ppt} for detailed storm surge 
study
\end{itemize}

Have some COMP sites in Cuban waters.
Developing coastal ARGO drifters/profilers

Text from flyer:

\subsubsection{Coastal Ocean Monitoring and Prediction System}

Florida is the United States' fourth most populous state, with 80\% of
the population living in a coastal county.  Several recent storms have
brought large, unpredicted flooding to Florida's west coast.  The
coastal sea level response to tropical and extra-tropical storms
results from wind forcing over the entire continental shelf.  Much of
the local response may be due to storm winds quite distant from the
local area of concern, a case in point being tropical storm Josephine,
a modest storm that nevertheless caused extensive flooding in the
Tampa Bay area.

The University of South Florida has implemented a real-time Coastal
Ocean Monitoring and Prediction System (COMPS) for West Florida. COMPS
provides additional data needed for a variety of management issues,
including more accurate predictions of coastal flooding by storm
surge, safety and efficiency of marine navigation, search and rescue
efforts, and fisheries management, as well as supporting basic
research programs.  COMP consists of an array of instrumentation both
along the coast and offshore, combined with numerical circulation
models, and builds upon existing in-situ measurements and modeling
programs funded by various state and federal agencies.  In addition,
COMPS links to the USF Remote Sensing Laboratory, which collects
real-time satellite imagery via its HRPT and X-Band receivers.  This
observing system fulfills all of the requirements of the Coastal
Module of the Global Ocean Observing System (CGOOS).  Data and model
products are disseminated in real-time to federal, state, and local
management officials, as well as the general public, via the Internet
(URL \Url{http://comps.marine.usf.edu}).  COMPS is designed to support
a variety of operational and research efforts, including storm surge
prediction, environmental protection, coastal erosion and sediment
transport, red tide research (ECOHAB - Ecology of Harmful Algal
Blooms), and hyperspectral satellite remote sensing of coastal ocean
dynamics (HYCODE).  A precedent for this system already exists in the
form of the Tampa Bay PORTS (Physical Oceanographic Real- Time System)
- itself a first for monitoring estuaries.

The majority of COMPS stations are fully operational, with additional
stations planned for the near future. An array of offshore buoys
measures current, temperature, salinity, and meteorological
parameters, with satellite telemetry of the data to shore. Additional
buoys have been deployed off Sarasota as part of the ECOHAB and HYCODE
efforts.  A network of coastal towers that are instrumented with water
level, temperature, salinity, meteorological, and bio- optical sensor
augments buoy observations. Many of these sites are operated in
collaboration with the United States Coast Guard, the Citrus County
Office of Emergency Management, and the Pasco County Office of
Disaster Preparedness. Additional instrumentation will be installed at
Boca Grande, Cedar Key, Keaton Beach, and near the mouth of the St.
Marks River, enhancing existing stations operated by partner agencies.

A numerical circulation model, based on the Princeton Ocean Model, has
been developed for the entire West Florida Shelf, with an offshore
boundary stretching from the Mississippi Delta to the Florida Keys.
This model has been successful in simulating past storm surge events
and will be coupled to the COMPS real-time data stream to be run in a
nowcast/forecast mode. Sea surface temperature and ocean color data
from the West Florida shelf routinely collected by our Remote Sensing
Laboratory can be combined with in situ data and model output to
provide a comprehensive analysis of oceanic conditions.

The COMPS data archival and distribution system will collate data
streams from the USF- operated sites with those from sites operated by
other agencies into a seamless web-based interface.  We have multiple
satellite downlinks (both DRGS and DOMSAT) for receiving GOES data
telemetry from remote sites.  We are collaborating with the NOAA
National Ocean Data Center, the NOAA Coastal Services Center, and the
National Ocean Service to develop a comprehensive data base management
system for the acquisition, archival, quality assurance, and
distribution of these data.

Collaborating Agencies: Florida Dept. of Environmental Protection, Florida Marine Research 
Institute, Florida Institute of Oceanography, Citrus County, Pasco County, United States 
Geological Survey, National Oceanic and Atmospheric Administration, United States Coast 
Guard, Office of Naval Research, Minerals Management Service, U.S. Environmental Protection 
Agency.

For more information contact:
\begin{description}
\item[Prof. Mark E. Luther]
Tel: 727-553-1528               
luther@marine.usf.edu           

\item[Prof. Robert H. Weisberg]
Tel: 727-553-1568
weisberg@marine.usf.edu
\end{description}

\subsection{Presentation by Bob Molinari - NOAA/AOML}

Global Ocean Observing System

NCEP
\begin{itemize}
\item Both real time and delayed mode data.
\end{itemize}

What's available:

Drifter Data:
\begin{itemize}
\item      Lots available in Gulf of Mexico.
\item Recent in nature.
\item 1996, '97
\item 1998 had a large deployment.
\item 1999 had good coverage in Caribbean.
\item Jan to May 2000.
\end{itemize}

Overhead of 1968 data.
Overhead of 1970 data.

How do they get data out?
\begin{itemize}
\item      Takes a month for delayed drifter data.
\item Can also get raw data.
\item Both historical and real-time.
\end{itemize}

Would like to put other data on-line:
\begin{itemize}
\item Some data in Yucatan. 
\item Has hurricane slice data.
\item Florida Bay data.
\end{itemize}

Data available from other sources as well.

Question as to whether AOML should be a DODS site.

The data to be served by AOML comes under the purview of the NOAA Global Ocean 
Observing System Center established at the laboratory. The objectives of the Center are to 
provide to NOAA and to other users the data needed to initialize weather and climate forecast 
models (in real-time) and to increase our understanding of the coupled climate system (in 
delayed-mode).  The GOOS Center manages:

\begin{itemize}

\item     The Global Drifter Center which provides surface current, SST and some meteorological 
data. These data are presently available at the AOML web page and we would be willing to 
serve these data as part of DODS.

\item    The US VOS XBT network which provides temperature profiles. Both historical and recent 
data are available at the AOML web page and we would be willing to serve these data as part 
of DODS.

\end{itemize}

We can also provide historical CTD data collected in the Gulf of
Mexico by AOML. These data are presently not available on the Web.
Finally we can serve the data resulting from NOAA's Florida Bay
project.  Both real- and delayed mode-data can be provided to DODS.

\subsection{Presentation by Tony Amos - University of Texas - Marine Science Institute (UTMSI)}

Using slides, Amos showed the setting of the UTMSI and some of the
projects that might provide data for a DODS site.

\begin{itemize}
  
\item UTMSI is located on the north end of Mustang Island, Texas, one
  of a series of barrier islands bordering the Gulf of Mexico on the
  ocean side and Corpus Christi Bay on the landward side.  UTMSI is a
  graduate research unit of the University of Texas' College of
  Natural Sciences with a faculty of 13 and 25 graduate students.  In
  addition, two six-week upper level summer courses are given each
  year.  The research interest is primarily in bays and estuaries, and
  mostly biological, with research projects in mariculture, harmful
  algal blooms, macro-algae and phytoplankton, fish endrochronology,
  early life history of fish, predator/prey relationships, benthic
  ecology, and physical oceanography.  The physical setting is ideal
  with the large range of marine environments nearby, ranging from the
  Gulf to the hyper-saline Laguna Madre.  There are several data sets
  that could be posted on a DODS server.
\item The R/V Longhorn is UTMSI's main research vessel.  This 105-ft R/V has made several 
hundred short cruises into the Gulf since 1972.  From the late 1970's onward, CTD stations 
have been made on most of these multi-purpose cruises.  The data would need standardizing 
as they have been made with evolving STD/CTD instruments.  For the past decade, Sea-Bird 
911/Plus CTDs have been used.  Also, meteorological and surface oceanographic data are 
routinely measured throughout every Longhorn cruise.
\item Current meter data from the continental shelf area in the Gulf, and from passes and bays are 
available
\item Continuous meteorological, tide and current data have been collected at the Pier Laboratory 
in the Aransas Pass Ship Channel (main entrance to Corpus Christi Bay from the Gulf)
\item Amos would make a twenty-two-year time series of surf zone sea temperature and salinity 
available.
\item Real time data is available on the UTMSI webs site (\Url{http://www.utmsi.utexas.edu}).  Click on 
Weather and Tides (demonstrated) to get current and previous day's sea, tides and weather 
data in graphic or spreadsheet form.
\item Overheads showing important recent storms: Hurricanes Allen, Gilbert and last year's Brett, 
Tropical Storm Josephine and Francis.  Data depicting real and predicted tides.  Could make 
tidal predictions for different Texas locations available for DODS.
\item Other data sets at UTMSI may be available.  These include time-series of dissolved oxygen 
in the Laguna, benthic data, and beach erosion information.
\end{itemize}

Amos also posed the question; would it be feasible to post Gulf of Mexico bathymetric and 
shoreline data on a DODS site.

\subsection{Presentation by Susan Starke - NOAA/National Coastal Data Development Center}

(PowerPoint Demo)

NCDDC Phase 0 Goals:
\begin{itemize}
\item Develop mission statement staffing plan.
\item Initiate Phase I planning process.
\item Implement initial proof-of -concept.
\item Establish facility at Stennis Space Center \& hold NCDDC dedication.
\end{itemize}

Mission Function:
\begin{itemize}
\item Provide archive and access for the long-term coastal data record.
\item Includes data cataloging/data mining, data access, data QC/Integration, archiving, and new 
product development.
\end{itemize}

Phase I Planning:
\begin{itemize}
\item Concept of operations (CON-OPS) by 10/00.
documents ``nuts and bolts'' of phase 0 operations.
\item Phase 0 to Phase 1 Transition Plan by 10/00.
\item Phase I requirements documentation by 12/00.
\item Includes feedback from phase 0 operations.
\item How to validate requirements.
\end{itemize}

\subsection{Presentation by Richard Campanella - Tulane-Xavier Center for Bioenvironmental 
Research - Long-term Ecosystem Assessment Group (LEAG)}


The Long-term Ecosystem Assessment Group (LEAG) was organized in 1999 by a consortium 
of academic/research institutions and NAVOCEANO. The goals of LEAG include:

\begin{itemize}

\item Establish a cooperative effort between the government (Naval Oceanographic Office), 
academia (CBR), and private industry (COTS Technology)
\item Use the Mississippi River and areas of the Gulf of Mexico as a natural laboratory to 
conduct research to evaluate the extent of ecological and economic impacts of nutrient 
over-enrichment in this region
\item Create a biotechnology corridor between Louisiana and
Mississippi for codevelopment of 
biosensors, Autonomous Underwater Vehicles (AUV's), models for major river and gulf 
systems, and deep sea monitoring and communication technologies
\end{itemize}

The Center for Bioenvironmental Research constitutes the lead academic entity of the Long 
Term Estuary Assessment Group (LEAG), a collaboration of government, academic and private 
organizations that is conducting research on river-ocean interactions and coastal oceanographic 
processes. LEAG's goal is to develop effective management strategies to deal with 
environmental issues related to rivers and their interactions with coastal margin and ocean 
ecosystems.  One issue of interest is the coastal eutrophication caused by excessive introduction 
of nutrients into an aquatic ecosystem, leading to increased algal production and increased 
availability of particulate organic carbon. The effects of eutrophication, hypoxic dead zones, 
harmful algal blooms, ad changing fisheries populations can have global economic and 
biological implications.

The partnership established with LEAG will utilize some of the best minds in the fields of 
biology, geology, engineering, chemistry and oceanography. It will enable research discoveries 
and new technological developments that would not be possible if the participating organizations 
were not working in unison. Through its Mississippi River Interdisciplinary Research (MiRIR) 
Program, the CBR is conducting scientific and cultural research and education programs on the 
river. The Naval Oceanographic Office is an accomplished oceanographic operations facility that 
is uniquely qualified to support data management, modeling and AUV development activities. 
COTS Technology specializes in the acoustics, robotics and optics of marine research, and will 
provide technical, financial and operational background for the development of AUV's for data 
collection.  Through the Office of Naval Research (ONR), LEAG is fostering what is anticipated 
to be the first living biosensor that will be deployed on a autonomous underwater vehicle.

Additional partners that have joined LEAG include the Louisiana Universities Marine 
Consortium (LUMCON); the University of Southern Mississippi; Woods Hole Oceanographic 
Institute (WHOI) in Massachusetts; and the U.S. Army Corps of Engineers' Waterways 
Experimentation Station (WES), Vicksburg, MS.

By utilizing the Mississippi River and the Gulf of Mexico as a natural laboratory, the LEAG 
program will improve the capacity of the United States to monitor the risk of exposure to 
defense related toxicants in other systems throughout the world.

LEAG is interested in seeing what it can offer to the DODS, and what it can utilize from the 
DODS.

%%% Local Variables: 
%%% mode: latex
%%% TeX-master: t
%%% End: 
