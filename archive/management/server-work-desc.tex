
% Description of the work to be done for the four new (9/24/98) DODS servers.
%
% $Id$

\documentclass[12pt]{article}
\usepackage{html}

%
% These are html links which are used often enough in writing about DODS to
% merit an input file.
% jhrg. 4/17/94
%
% File rationalized and updated while writing the DODS User
% Guide. Also includes other useful abbreviations.
% tomfool 3/15/96
%
% $Id$
%
% HTML references to DODS documents
%
% For most of the DODS documentation there are three html references defined:
% 1) The upper case \newcommand produces the title of the paper, an
%    html link in the online documentation and a footnote in the paper
%    version.  
% 2) A capitalized version produces a capitalized description of the
%    paper and a link in the online version (but no footnote in the
%    paper version). 
% 3) A lower case version which produces a lower case description of
%    the paper and a html link, but no footnote.

% External references to documents:
%
% In order for external references to work (via latex2html) the labels used
% throughout the documents must be unique. The text labels are all prefixed
% with a document identifier (e.g., ddd) with a colon separater. The
% \externalrefs below (way below...) includes the perl files html.sty uses to
% make the cross refs.

\newcommand{\DODS}{\htmladdnormallink{Distributed Oceanographic Data System}
  {http://www.unidata.ucar.edu/packages/dods/}}

\newcommand{\Dods}{\htmladdnormallink{DODS}{http://www.unidata.ucar.edu/packages/dods/}}

\newcommand{\wrkshp}{\htmladdnormallink{Report on the First Workshop for the
    Distributed Oceanographic Data System, Proposed System Architectures}
  {http://www.unidata.ucar.edu/packages/dods/archive/reports/workshop1/section3.13.html}}

% not DODS URLs, but useful all the same...

\newcommand{\www}{\htmladdnormallink{World Wide Web} 
  {http://www.w3.org/hypertext/WWW/TheProject.html}}
\newcommand{\url}{\htmladdnormallink{Uniform Resource Locator}
  {http://www.w3.org/hypertext/WWW/Addressing/URL/url-spec.html}}

\newcommand{\uri}{\htmladdnormallinkfoot{Uniform Resource Identifiers}
  {http://www.w3.org/hypertext/WWW/Addressing/URL/uri-spec.html}}

\newcommand{\urn}{\htmladdnormallinkfoot{Uniform Resource Name}
  {http://www.acl.lanl.gov/URI/archive/uri-archive.messages/1143.html}}

\newcommand{\HTTPD}{\htmladdnormallinkfoot{HTTPD}
  {http://hoohoo.ncsa.uiuc.edu/docs/Overview.html}}

\newcommand{\HTTP}{\htmladdnormallinkfoot{HyperText Transfer Protocol}
  {http://www.w3.org/hypertext/WWW/Protocols/HTTP/HTTP2.html}}

%\newcommand{\HTML}{\htmladdnormallinkfoot{HyperText Markup Language}
%  {http://www.w3.org/hypertext/WWW/MarkUp/MarkUp.html}}

% CERN used to be `Conseil Europeen pour la Recherche Nucleaire'
\newcommand{\CERN}
  {\htmladdnormallink{European Laboratory for Particle Physics}
  {http://www.cern.ch/}}

\newcommand{\NCSA}
  {\htmladdnormallink{National Center for Supercomputing Applications}
  {http://www.ncsa.uiuc.edu/}}

\newcommand{\WWWC}{\htmladdnormallink{World Widw Web Consortium}
  {http://www.w3.org/}}

\newcommand{\CGI}{\htmladdnormallinkfoot{Common Gateway Interface}
  {http://hoohoo.ncsa.uiuc.edu/cgi/overview.html}}

\newcommand{\MIME}{\htmladdnormallink{Multipurpose Internet Mail Extensions}
  {http://www.cis.ohio-state.edu/htbin/rfc/rfc1590.html}}

\newcommand{\netcdf}{\htmladdnormallink{NetCDF}
  {http://www.unidata.ucar.edu/packages/netcdf/guide.txn_toc.html}}

\newcommand{\JGOFS}{\htmladdnormallink{Joint Geophysical Ocean Flux Study}
  {http://www1.whoi.edu/jgofs.html}}

\newcommand{\jgofs}{\htmladdnormallink{JGOFS}
  {http://www1.whoi.edu/jgofs.html}}

\newcommand{\hdf}{\htmladdnormallink{HDF}
  {http://www.ncsa.uiuc.edu/SDG/Software/HDF/HDFIntro.html}}

\newcommand{\Cpp}{{\rm {\small C}\raise.5ex\hbox{\footnotesize ++}}}

% Commands

\newcommand{\pslink}[1]{\small
\begin{quote}
  A \htmladdnormallink{postscript}{#1} version of this document is
  available.  You may also use \htmladdnormallink{anonymous
  ftp}{ftp://ftp.unidata.ucar.edu/pub/dods/ps-docs/} to access postscript files
  of all of the DODS documentation.
\end{quote}
\normalsize
}

\newcommand{\declaration}[1]{\small
{\tt {#1}}
\normalsize
}

% All the links from here down are for the white papers. I'm not sure that any
% of these are still valid given the reorganization of the web site. 5/12/98
% jhrg.

\newcommand{\DDA}{\htmladdnormallinkfoot{DODS---Data Delivery Architecture}
  {http://www.unidata.ucar.edu/packages/dods/archive/design/data-delivery-arch/data-delivery-arch.html}}
\newcommand{\Dda}{\htmladdnormallink{Data Delivery Architecture}
  {http://www.unidata.ucar.edu/packages/dods/archive/design/data-delivery-arch/data-delivery-arch.html}}
\newcommand{\dda}{\htmladdnormallink{data delivery architecture}
  {http://www.unidata.ucar.edu/packages/dods/archive/design/data-delivery-arch/data-delivery-arch.html}}

\newcommand{\DDD}{\htmladdnormallinkfoot{DODS---Data Delivery Design}
  {http://www.unidata.ucar.edu/packages/dods/archive/design/data-delivery-design/data-delivery-design.html}}
\newcommand{\Ddd}{\htmladdnormallink{Data Delivery Design}
  {http://www.unidata.ucar.edu/packages/dods/archive/design/data-delivery-design/data-delivery-design.html}}
\newcommand{\ddd}{\htmladdnormallink{data delivery design}
  {http://www.unidata.ucar.edu/packages/dods/archive/design/data-delivery-design/data-delivery-design.html}}

\newcommand{\URL}{\htmladdnormallinkfoot{DODS---Uniform Resource Locator}
  {http://www.unidata.ucar.edu/packages/dods/archive/design/urls/urls.html}}
\newcommand{\Url}{\htmladdnormallink{Uniform Resource Locator}
  {http://www.unidata.ucar.edu/packages/dods/archive/design/urls/urls.html}}
\newcommand{\dodsurl}{\htmladdnormallink{uniform resource locator}
  {http://www.unidata.ucar.edu/packages/dods/archive/design/urls/urls.html}}

\newcommand{\DAP}{\htmladdnormallinkfoot{DODS---Data Access Protocol}
  {http://www.unidata.ucar.edu/packages/dods/archive/design/api/api.html}}
\newcommand{\Dap}{\htmladdnormallink{Data Access Protocol}
  {http://www.unidata.ucar.edu/packages/dods/archive/design/api/api.html}}
\newcommand{\dap}{\htmladdnormallink{data access protocol}
  {http://www.unidata.ucar.edu/packages/dods/archive/design/api/api.html}}

\newcommand{\SOFT}
        {\htmladdnormallinkfoot{DODS---Software Development Environment}
  {http://www.unidata.ucar.edu/packages/dods/archive/managment/software/software.html}}
\newcommand{\Soft}{\htmladdnormallink{Software Development Environment}
  {http://www.unidata.ucar.edu/packages/dods/archive/managment/software/software.html}}
\newcommand{\soft}{\htmladdnormallinkfoot{software development environment}
  {http://www.unidata.ucar.edu/packages/dods/archive/managment/software/software.html}}

% I used to call the DAP the API, then for a few days I called it the
% DTP. These def's were used then. Rather than find everywhere they are used
% (not that hard, but...) I just hacked the macros. NB: the source file for
% the DAP document is still called `api.tex'

\newcommand{\API}{\htmladdnormallinkfoot{DODS---Data Access Protocol}
  {http://www.unidata.ucar.edu/packages/dods/archive/design/api/api.html}}
\newcommand{\api}{\htmladdnormallink{data access protocol}
  {http://www.unidata.ucar.edu/packages/dods/archive/design/api/api.html}}

\newcommand{\DTP}{\htmladdnormallinkfoot{DODS---Data Access Protocol}
  {http://www.unidata.ucar.edu/packages/dods/archive/design/api/api.html}}
\newcommand{\dtp}{\htmladdnormallink{data access protocol}
  {http://www.unidata.ucar.edu/packages/dods/archive/design/api/api.html}}

\newcommand{\TOOLKIT}{\htmladdnormallinkfoot{DODS---Client and Server Toolkit}
  {http://www.unidata.ucar.edu/packages/dods/archive/implementation/toolkits/toolkits.html}}
\newcommand{\Toolkit}{\htmladdnormallink{Client and Server Toolkit}
  {http://www.unidata.ucar.edu/packages/dods/archive/implementation/toolkits/toolkits.html}}
\newcommand{\toolkit}{\htmladdnormallink{client and server toolkit}
  {http://www.unidata.ucar.edu/packages/dods/archive/implementation/toolkits/toolkits.html}}

\newcommand{\TKR}{\htmladdnormallinkfoot{DODS---DAP Toolkit Reference}
  {http://www.unidata.ucar.edu/packages/dods/archive/implementation/toolkits/toolkits.html}}
\newcommand{\Tkr}{\htmladdnormallink{DAP Toolkit Reference}
  {http://www.unidata.ucar.edu/packages/dods/archive/implementation/toolkits/toolkits.html}}
\newcommand{\tkr}{\htmladdnormallink{DAP toolkit reference}
  {http://www.unidata.ucar.edu/packages/dods/archive/implementation/toolkits/toolkits.html}}

% I know that these are wrong. The papers have been moved to the gso web
% site, but I'm not sure exactly where. 5/12/98 jhrg

\newcommand{\DM}{\htmladdnormallinkfoot{DODS---Data Model}
  {http://lake.mit.edu/dods-dir/dodsdm2.html}}
\newcommand{\Dm}{\htmladdnormallink{Data Model}
  {http://lake.mit.edu/dods-dir/dodsdm2.html}}
\newcommand{\dm}{\htmladdnormallink{data model}
  {http://lake.mit.edu/dods-dir/dodsdm2.html}}

\newcommand{\DD}{\htmladdnormallinkfoot{DODS---Data Delivery}
  {http://lake.mit.edu/dods-dir/dods-dd.html}}

% external refs for DODS documents

\externallabels{http://www.unidata.ucar.edu/packages/dods/design/api}
  {/home/www/packages/dods/archive/design/api/labels.pl}
\externallabels{http://www.unidata.ucar.edu/packages/dods/design/data-delivery-arch}
  {/home/www/packages/dods/archive/design/data-delivery-arch/labels.pl}
\externallabels{http://www.unidata.ucar.edu/packages/dods/design/data-delivery-design}
  {/home/www/packages/dods/archive/design/data-delivery-design/labels.pl}
\externallabels{http://www.unidata.ucar.edu/packages/dods/implementation/toolkits}
  {/home/www/packages/dods/archive/implementation/toolkits/labels.pl}

%\renewcommand{\externalref}[2]{#2}

%%% Local Variables: 
%%% mode: latex
%%% TeX-master: t
%%% End: 


\begin{document}

\title{Work~Description: Additional~Data~Servers~for~DODS}
\author{James Gallagher\\
Richard Chinman}
\date{\today}

\maketitle

\begin{htmlonly}
\pslink{http://dcz.cvo.oneworld.com/server-work-desc.ps}
\end{htmlonly}

\tableofcontents

\section{Introduction}

The DODS Project has undertaken, since 1995, the task of building a
distributed data system using a two-tier client-server architecture.  The
project now contains servers for six (6) different data formats.  We need to
add generic servers for data stored using the GRIB and BUFR file formats to
the suite of existing servers. In addition, we need specialized servers for a
GIF-based storage format used at IFREMER and a relational database management
system used at Oregon State University.

While the four servers needed are described here, each is to be considered a
separate project. Individuals or organizations proposing to build one need
not build the others, or be prohibited from building others.

\section{Background Information}

Before considering this work, interested parties should become familiar
with DODS, its programming libraries and tools and the specifics of the
target file format or data system with which they propose to work. 

Information about DODS is available both on line and on paper. All of the
DODS Guides and white papers may be found on the \htmladdnormallinkfoot{DODS
  Web page.} {http://unidata.ucar.edu/packages/dods/} Copies of the User's
and Programmer's Guides are available through the \htmladdnormallink{DODS
  Project Management Office}{mailto:chinman@ucar.edu} at UCAR, Boulder,
CO.\footnote{Richard Chinman, chinman@ucar.edu, 303.497.8696}

\begin{enumerate}
  
\item Information on GRIB can be found \htmladdnormallinkfoot{on the Web,}
  {http://wesley.wwb.noaa.gov/reading\_grib.html} through the World
  Meteorological Organization or through the DODS project management office.

\item Information on BUFR is available \htmladdnormallinkfoot{on the
    Web}{http://www.scd.ucar.edu/dss/pub/reanalysis/data\_usr.html} or through
  the DODS project management office.
  
\item Information about the IFREMER gif-based format is available through the
  DODS project management office.

\item Information about the Oregon State University relational database is
  available through the DODS project management office.
  
\end{enumerate}

\section{Work to be Done}

Work on each server should be completed in four parts unless it can be
shown that a more sensible plan exists:

\begin{enumerate}
  
\item Submit a design for critical review. If it would help to also have a
  preliminary design review, that could be arranged.

\item Submit the software.

\item Submit an automated test suite for the server. The server should pass
  all of the tests except in unusual circumstances.

\item Submit documentation for the server. This should cover any
  idiosyncrasies of the server. 

\end{enumerate}

The payment schedule for this work is to be proposed by the interested
parties. 

\begin{quote}
  
  It may be possible to pay consultants be extending existing contracts
  that DODS has. Those working on these servers would technically be
  employees/sub-contractors of/for those companies. Contact the
  \htmladdnormallink{DODS Program Manager}{mailto:chinman@ucar.edu} for more
  information.

\end{quote}

\section{Requirements for the Four Servers}

The four servers share some requirements because they all must function
within DODS. However, the GRIB and BUFR servers have more restrictive
requirements because they are to be part of the general DODS distribution.
They will be installed by many different data providers. The IFREMER and OSU
servers are intended only for use by their respective institutions and the
requirements are consequently less specific; those servers' design will be
determined more by the target data system than by DODS' architecture.

During the evaluation of the proposals or statements of work, additional
requirements may be identified. If that is the case, any extra burden those
place on the design will be taken into account. Either the time or money
alloted to the work will be increased or some other accommodation will be
made.\footnote{In other words, we know how these things go. We don't want to
  leave out something important, but we also don't want to put anybody
  through the ringer either.}

The requirements common to all four servers are:

\begin{enumerate}
  
\item The servers must adhere to the DODS Data Access Protocol (DAP). It is
  OK if support for other protocols is also built into the servers, but
  maintenance of that portion of the software must be described in detail in
  the server's supporting documentation.
  
\item The servers must provide access to data using objects and data types
  defined by the DODS DAP. The software package {\tt
    DODS-dap-<version>.tar.gz}, available from the DODS Web site, contains an
  implementation of these objects and data types. The DODS Programmer's Guide
  describes how to write servers using this implementation of the DAP.

\item The servers must implement the DODS constraint expression processing
  scheme using the core library as supplied or provide both faithful
  emulation and a convincing argument for not using that implementation. The
  intent here is simplify maintenance and future development by consolidating
  the expression evaluation software.

\end{enumerate}

\subsection{The GRIB and BUFR Servers}

The requirements specific to the GRIB and BUFR servers are:

\begin{enumerate}
  
\item The GRIB and BUFR servers must run on UNIX. We must be able to build
  binaries for the current set of supported platforms using the current
  development tools. As of \today\ the supported platforms are: 

  \begin{enumerate}

  \item SUN Sparc (Solaris 2.x)
  \item Dec Alpha (Digital UNIX)
  \item SGI Mips (IRIX 5.3, 6.x)
  \item Intel 586/686 (Linux 2.x)
  \item HP (HP/UX)

  \end{enumerate}
  
  The current development environment uses the FSF's gcc/g++ compiler and STL
  implementation. In addition, implementors are encouraged to support other
  platforms and/or tools particularly Windows 95/98, Windows NT and different
  vendor's compilers.  Currently, DODS provides support for neither non-UNIX
  operating systems nor other compilers. However, recent changes to DODS make
  it simpler to build DODS with various tools and ostensibly other operating
  systems. For more information about this subject, contact the
  \htmladdnormallink{DODS technical
    lead}{mailto:jgallagher@gso.uri.edu}.\footnote{James Gallagher,
    jgallagher@gso.uri.edu, 541.752.7499}
  
\item The GRIB and BUFR servers must conform to the general design of the
  existing DODS servers so as to minimize time needed to maintain the server
  software once it is completed. To satisfy this requirement the software
  does not need to follow the exact architecture of existing servers, but it
  should use similar mechanisms and deviate from the existing design pattern
  only if the new design is demonstrably superior.
  
\item The GRIB and BUFR servers must have essentially the same
  installation overhead for users as do the current servers. That is, users
  should be able to install the server by copying programs into a web
  server's CGI binaries directory. Any further installation steps are
  discouraged unless the GRIB format makes it absolutely necessary. For
  example, the FreeForm server requires that one or more format description
  files be written before the server will work. There is no way around that
  since those files are dependent on the data and we cannot control that!

\end{enumerate}

\subsection{The IFREMER and OSU Servers}

The requirements specific to the IFREMER and OSU servers are:

\begin{enumerate}

\item The servers must run on platforms which pose the lowest overall
  maintenance burden for the target institutions.
  
\item Source code for the servers must be supplied as well as executable
  images.
  
\item Considerable latitude is available when implementing these servers, but
  a significant deviation from the design of other DODS servers must be
  explained in the server's documentation.

\end{enumerate}

\section{Proposing Work}

Potential Server developers should expect to submit a proposal or statement
of work describing the development effort, accompanied by a budget, to the
DODS project management office at UCAR. In addition, descriptions of past
work (e.g., a resum\'e) should be submitted prior to,
or with, the proposal or statement. The project office will consult with
the DODS project technical lead to evaluate the proposals.

\end{document}
