
% Description of the work to be done for the second version of the loaddods
% command for IDL
%
% $Id$

\documentclass[12pt]{article}
\usepackage{epsfig,html}

%
% These are html links which are used often enough in writing about DODS to
% merit an input file.
% jhrg. 4/17/94
%
% File rationalized and updated while writing the DODS User
% Guide. Also includes other useful abbreviations.
% tomfool 3/15/96
%
% $Id$
%
% HTML references to DODS documents
%
% For most of the DODS documentation there are three html references defined:
% 1) The upper case \newcommand produces the title of the paper, an
%    html link in the online documentation and a footnote in the paper
%    version.  
% 2) A capitalized version produces a capitalized description of the
%    paper and a link in the online version (but no footnote in the
%    paper version). 
% 3) A lower case version which produces a lower case description of
%    the paper and a html link, but no footnote.

% External references to documents:
%
% In order for external references to work (via latex2html) the labels used
% throughout the documents must be unique. The text labels are all prefixed
% with a document identifier (e.g., ddd) with a colon separater. The
% \externalrefs below (way below...) includes the perl files html.sty uses to
% make the cross refs.

\newcommand{\DODS}{\htmladdnormallink{Distributed Oceanographic Data System}
  {http://www.unidata.ucar.edu/packages/dods/}}

\newcommand{\Dods}{\htmladdnormallink{DODS}{http://www.unidata.ucar.edu/packages/dods/}}

\newcommand{\wrkshp}{\htmladdnormallink{Report on the First Workshop for the
    Distributed Oceanographic Data System, Proposed System Architectures}
  {http://www.unidata.ucar.edu/packages/dods/archive/reports/workshop1/section3.13.html}}

% not DODS URLs, but useful all the same...

\newcommand{\www}{\htmladdnormallink{World Wide Web} 
  {http://www.w3.org/hypertext/WWW/TheProject.html}}
\newcommand{\url}{\htmladdnormallink{Uniform Resource Locator}
  {http://www.w3.org/hypertext/WWW/Addressing/URL/url-spec.html}}

\newcommand{\uri}{\htmladdnormallinkfoot{Uniform Resource Identifiers}
  {http://www.w3.org/hypertext/WWW/Addressing/URL/uri-spec.html}}

\newcommand{\urn}{\htmladdnormallinkfoot{Uniform Resource Name}
  {http://www.acl.lanl.gov/URI/archive/uri-archive.messages/1143.html}}

\newcommand{\HTTPD}{\htmladdnormallinkfoot{HTTPD}
  {http://hoohoo.ncsa.uiuc.edu/docs/Overview.html}}

\newcommand{\HTTP}{\htmladdnormallinkfoot{HyperText Transfer Protocol}
  {http://www.w3.org/hypertext/WWW/Protocols/HTTP/HTTP2.html}}

%\newcommand{\HTML}{\htmladdnormallinkfoot{HyperText Markup Language}
%  {http://www.w3.org/hypertext/WWW/MarkUp/MarkUp.html}}

% CERN used to be `Conseil Europeen pour la Recherche Nucleaire'
\newcommand{\CERN}
  {\htmladdnormallink{European Laboratory for Particle Physics}
  {http://www.cern.ch/}}

\newcommand{\NCSA}
  {\htmladdnormallink{National Center for Supercomputing Applications}
  {http://www.ncsa.uiuc.edu/}}

\newcommand{\WWWC}{\htmladdnormallink{World Widw Web Consortium}
  {http://www.w3.org/}}

\newcommand{\CGI}{\htmladdnormallinkfoot{Common Gateway Interface}
  {http://hoohoo.ncsa.uiuc.edu/cgi/overview.html}}

\newcommand{\MIME}{\htmladdnormallink{Multipurpose Internet Mail Extensions}
  {http://www.cis.ohio-state.edu/htbin/rfc/rfc1590.html}}

\newcommand{\netcdf}{\htmladdnormallink{NetCDF}
  {http://www.unidata.ucar.edu/packages/netcdf/guide.txn_toc.html}}

\newcommand{\JGOFS}{\htmladdnormallink{Joint Geophysical Ocean Flux Study}
  {http://www1.whoi.edu/jgofs.html}}

\newcommand{\jgofs}{\htmladdnormallink{JGOFS}
  {http://www1.whoi.edu/jgofs.html}}

\newcommand{\hdf}{\htmladdnormallink{HDF}
  {http://www.ncsa.uiuc.edu/SDG/Software/HDF/HDFIntro.html}}

\newcommand{\Cpp}{{\rm {\small C}\raise.5ex\hbox{\footnotesize ++}}}

% Commands

\newcommand{\pslink}[1]{\small
\begin{quote}
  A \htmladdnormallink{postscript}{#1} version of this document is
  available.  You may also use \htmladdnormallink{anonymous
  ftp}{ftp://ftp.unidata.ucar.edu/pub/dods/ps-docs/} to access postscript files
  of all of the DODS documentation.
\end{quote}
\normalsize
}

\newcommand{\declaration}[1]{\small
{\tt {#1}}
\normalsize
}

% All the links from here down are for the white papers. I'm not sure that any
% of these are still valid given the reorganization of the web site. 5/12/98
% jhrg.

\newcommand{\DDA}{\htmladdnormallinkfoot{DODS---Data Delivery Architecture}
  {http://www.unidata.ucar.edu/packages/dods/archive/design/data-delivery-arch/data-delivery-arch.html}}
\newcommand{\Dda}{\htmladdnormallink{Data Delivery Architecture}
  {http://www.unidata.ucar.edu/packages/dods/archive/design/data-delivery-arch/data-delivery-arch.html}}
\newcommand{\dda}{\htmladdnormallink{data delivery architecture}
  {http://www.unidata.ucar.edu/packages/dods/archive/design/data-delivery-arch/data-delivery-arch.html}}

\newcommand{\DDD}{\htmladdnormallinkfoot{DODS---Data Delivery Design}
  {http://www.unidata.ucar.edu/packages/dods/archive/design/data-delivery-design/data-delivery-design.html}}
\newcommand{\Ddd}{\htmladdnormallink{Data Delivery Design}
  {http://www.unidata.ucar.edu/packages/dods/archive/design/data-delivery-design/data-delivery-design.html}}
\newcommand{\ddd}{\htmladdnormallink{data delivery design}
  {http://www.unidata.ucar.edu/packages/dods/archive/design/data-delivery-design/data-delivery-design.html}}

\newcommand{\URL}{\htmladdnormallinkfoot{DODS---Uniform Resource Locator}
  {http://www.unidata.ucar.edu/packages/dods/archive/design/urls/urls.html}}
\newcommand{\Url}{\htmladdnormallink{Uniform Resource Locator}
  {http://www.unidata.ucar.edu/packages/dods/archive/design/urls/urls.html}}
\newcommand{\dodsurl}{\htmladdnormallink{uniform resource locator}
  {http://www.unidata.ucar.edu/packages/dods/archive/design/urls/urls.html}}

\newcommand{\DAP}{\htmladdnormallinkfoot{DODS---Data Access Protocol}
  {http://www.unidata.ucar.edu/packages/dods/archive/design/api/api.html}}
\newcommand{\Dap}{\htmladdnormallink{Data Access Protocol}
  {http://www.unidata.ucar.edu/packages/dods/archive/design/api/api.html}}
\newcommand{\dap}{\htmladdnormallink{data access protocol}
  {http://www.unidata.ucar.edu/packages/dods/archive/design/api/api.html}}

\newcommand{\SOFT}
        {\htmladdnormallinkfoot{DODS---Software Development Environment}
  {http://www.unidata.ucar.edu/packages/dods/archive/managment/software/software.html}}
\newcommand{\Soft}{\htmladdnormallink{Software Development Environment}
  {http://www.unidata.ucar.edu/packages/dods/archive/managment/software/software.html}}
\newcommand{\soft}{\htmladdnormallinkfoot{software development environment}
  {http://www.unidata.ucar.edu/packages/dods/archive/managment/software/software.html}}

% I used to call the DAP the API, then for a few days I called it the
% DTP. These def's were used then. Rather than find everywhere they are used
% (not that hard, but...) I just hacked the macros. NB: the source file for
% the DAP document is still called `api.tex'

\newcommand{\API}{\htmladdnormallinkfoot{DODS---Data Access Protocol}
  {http://www.unidata.ucar.edu/packages/dods/archive/design/api/api.html}}
\newcommand{\api}{\htmladdnormallink{data access protocol}
  {http://www.unidata.ucar.edu/packages/dods/archive/design/api/api.html}}

\newcommand{\DTP}{\htmladdnormallinkfoot{DODS---Data Access Protocol}
  {http://www.unidata.ucar.edu/packages/dods/archive/design/api/api.html}}
\newcommand{\dtp}{\htmladdnormallink{data access protocol}
  {http://www.unidata.ucar.edu/packages/dods/archive/design/api/api.html}}

\newcommand{\TOOLKIT}{\htmladdnormallinkfoot{DODS---Client and Server Toolkit}
  {http://www.unidata.ucar.edu/packages/dods/archive/implementation/toolkits/toolkits.html}}
\newcommand{\Toolkit}{\htmladdnormallink{Client and Server Toolkit}
  {http://www.unidata.ucar.edu/packages/dods/archive/implementation/toolkits/toolkits.html}}
\newcommand{\toolkit}{\htmladdnormallink{client and server toolkit}
  {http://www.unidata.ucar.edu/packages/dods/archive/implementation/toolkits/toolkits.html}}

\newcommand{\TKR}{\htmladdnormallinkfoot{DODS---DAP Toolkit Reference}
  {http://www.unidata.ucar.edu/packages/dods/archive/implementation/toolkits/toolkits.html}}
\newcommand{\Tkr}{\htmladdnormallink{DAP Toolkit Reference}
  {http://www.unidata.ucar.edu/packages/dods/archive/implementation/toolkits/toolkits.html}}
\newcommand{\tkr}{\htmladdnormallink{DAP toolkit reference}
  {http://www.unidata.ucar.edu/packages/dods/archive/implementation/toolkits/toolkits.html}}

% I know that these are wrong. The papers have been moved to the gso web
% site, but I'm not sure exactly where. 5/12/98 jhrg

\newcommand{\DM}{\htmladdnormallinkfoot{DODS---Data Model}
  {http://lake.mit.edu/dods-dir/dodsdm2.html}}
\newcommand{\Dm}{\htmladdnormallink{Data Model}
  {http://lake.mit.edu/dods-dir/dodsdm2.html}}
\newcommand{\dm}{\htmladdnormallink{data model}
  {http://lake.mit.edu/dods-dir/dodsdm2.html}}

\newcommand{\DD}{\htmladdnormallinkfoot{DODS---Data Delivery}
  {http://lake.mit.edu/dods-dir/dods-dd.html}}

% external refs for DODS documents

\externallabels{http://www.unidata.ucar.edu/packages/dods/design/api}
  {/home/www/packages/dods/archive/design/api/labels.pl}
\externallabels{http://www.unidata.ucar.edu/packages/dods/design/data-delivery-arch}
  {/home/www/packages/dods/archive/design/data-delivery-arch/labels.pl}
\externallabels{http://www.unidata.ucar.edu/packages/dods/design/data-delivery-design}
  {/home/www/packages/dods/archive/design/data-delivery-design/labels.pl}
\externallabels{http://www.unidata.ucar.edu/packages/dods/implementation/toolkits}
  {/home/www/packages/dods/archive/implementation/toolkits/labels.pl}

%\renewcommand{\externalref}[2]{#2}

%%% Local Variables: 
%%% mode: latex
%%% TeX-master: t
%%% End: 


\begin{document}

\title{Work~Description: Access~to~DODS~Servers~for~IDL}
\author{James Gallagher\\
Richard Chinman}
\date{\today}

\maketitle

\begin{htmlonly}
\pslink{http://dcz.cvo.oneworld.com/idl-loaddods-work.ps}
\end{htmlonly}

\section{Introduction}

The DODS project needs to build a tool which will enable the IDL data
analysis tool developed by Research Systems, Inc. to read data from DODS
servers. We have a preliminary tool built by colleagues at JPL, but this needs
to be enhanced in several ways and may serve only as an example of desired
functionality. 

In this document we provide some background information on the existing
IDL-DODS software, describe the work to be done and list, explicitly,
important requirements for this software.

\section{Background}

The DODS software which enables users of IDL to access data served by DODS
was written by colleagues at JPL. This software was based on a similar
program developed for users of Matlab. The software as written does not take
advantage of many features of IDL. In addition, the developers at JPL used
IDL 4 and it is our understanding that IDL 5 provides many enhancements which
will simplify this software and make it more flexible.

In order to understand how the IDL-DODS software got to its current state, we
need to provide some explanation of the development of our command line
interface to Matlab. This software was developed under Matlab 4 which
presented us with two problems: Matlab 4 could not link external commands
written in \Cpp\ and it could not deal with arrays larger than two
dimensions.  DODS, however, provided low-level access to the servers using
only \Cpp \footnote{That has since been changed, we now have a companion Java
  class library which can be used to access the servers. I'm not sure if that
  matters here, however. 12/22/98 jhrg} and has N-dimensional arrays. To
bridge this gap the software was split into two pieces, one that communicated
with the DODS servers (written in \Cpp\, using our class library) and one
external command written in C. The external command communicated to the \Cpp\ 
DODS client using standard UNIX IPC. Further the external command was
restricted so that it would work only with scalars, vectors and matrices.

Because our colleagues at JPL used this software to build the IDL interface to
DODS, that software has the same essential architecture even though IDL poses
none of the above limits. The organization of the current IDL-DODS interface
is shown in Appendix~\ref{current-arch}.

The design of the current IDL-DODS interface can certainly be changed. The
current design is an artifact of modifying existing software to get something
running quickly. We want to be sure that readers of this document understand
that the existing software does not represent a design we think is best
suited to the task.

\section{Statement of Work}

There are five deliverables defined by this document:

\begin{enumerate}

\item Submit a preliminary design for review.

\item Submit a design for approval and an implementation schedule including
  critical milestones if warranted.

\item Deliver the software described in the design along with appropriate
  documentation. 

\item Provide a suite of tests for the software and work with Unidata/UCAR
  personnel to test on the following platforms: 

  \begin{itemize}
    \item Sun/Solaris
    \item SGI/IRIX
    \item Dec-Alpha/OSF
    \item Intel/Linux
  \end{itemize}
  
\item Present a demonstration of the deliverable to Unidata/UCAR and the
  DODS project manager, Richard Chinman.

\end{enumerate}

\subsection{What will be provided}

\begin{enumerate}
   \item The current version of the IDL-DODS command line tool code. 
   \item Any relevant documentation for the IDL-DODS command line tool.
   \item Test programs.
   \item A primary and secondary project liaison (technical) and a project
     monitor.
   \item \Cpp\ Library to process requests from DODS servers. This includes
     software to parse incoming data.
   \item Documentation for the DODS \Cpp\ class library.
\end{enumerate}

\section{Requirements}

The most important requirements are listed here. There are other features,
currently present in the IDL-DODS interface, that are desirable but are less
important. 

\begin{enumerate}

\item N-dimensional arrays: The new version of the IDL-DODS interface should
  intern arrays of any dimensionality.

\item The new version of the IDL-DODS interface should make use of IDL's
  data types wherever possible (e.g., Structures).

\item Single program architecture: Unless there is good reason to do
  otherwise, the two program architecture should be abandoned in favor of a
  single external command (e.g., dynamically loaded module) written in \Cpp.
  
\item Creation of global variables: The IDL-DODS interface should create
  global variables, structures, etc. when appropriate. The current version of
  the interface cannot do this.

\item Command options of the current software: The command options of the
  current software should be supported where appropriate.

\item Help option: There should be a help option which provides a brief
  explanation of the supported command options.

\end{enumerate}

\section{Ownership}

The DODS project will have sole ownership of IDL-DODS interface. This
includes, but is not limited to, the right to use, sell or giveaway the
software. In this section when we say `the software' we mean the source code,
binary copies and documentation. We reserve the right to attach our copyright
to the software---we are amenable to acknowledging authorship by a third
party, but the software must be copyrighted as part of the DODS project based
at the University of Rhode Island. We forbid any use of the software not
explicitly described in our copyright statement unless permission from us
(the DODS project) is obtained in writing.

\section{Support and Maintenance}

The DODS project will be responsible for long-term maintenance and support of
the IDL-DODS interface. However, for a period of 60 days after delivery we
should be able to ask the authors of the IDL-DODS interface to fix problems
which we and the authors agree come under the purview of this description of
work. 

\appendix

\section{Current Architecture}
\label{current-arch}

The current architecture of the IDL-DODS interface is shown in
Figure~\ref{data-flow}. The current external command sends a DODS URL to the
DODS client. The DODS client dereferences that URL, reads data from the
server and sends those data back to the external command using a simple
protocol the two programs share.

\begin{figure}
\epsfig{file=idl-loaddods-data-flow.eps,width=6in}
\caption{Data flow of the current IDL-DODS interface.}
\label{data-flow}
\end{figure}

\end{document}