%
% This file describes the software development environment used to build DODS
% here at URI. 
%
% $Id$
%

\documentstyle[12pt,html,psfig,margins]{article}

%
% These are html links which are used often enough in writing about DODS to
% merit an input file.
% jhrg. 4/17/94
%
% File rationalized and updated while writing the DODS User
% Guide. Also includes other useful abbreviations.
% tomfool 3/15/96
%
% $Id$
%
% HTML references to DODS documents
%
% For most of the DODS documentation there are three html references defined:
% 1) The upper case \newcommand produces the title of the paper, an
%    html link in the online documentation and a footnote in the paper
%    version.  
% 2) A capitalized version produces a capitalized description of the
%    paper and a link in the online version (but no footnote in the
%    paper version). 
% 3) A lower case version which produces a lower case description of
%    the paper and a html link, but no footnote.

% External references to documents:
%
% In order for external references to work (via latex2html) the labels used
% throughout the documents must be unique. The text labels are all prefixed
% with a document identifier (e.g., ddd) with a colon separater. The
% \externalrefs below (way below...) includes the perl files html.sty uses to
% make the cross refs.

\newcommand{\DODS}{\htmladdnormallink{Distributed Oceanographic Data System}
  {http://www.unidata.ucar.edu/packages/dods/}}

\newcommand{\Dods}{\htmladdnormallink{DODS}{http://www.unidata.ucar.edu/packages/dods/}}

\newcommand{\wrkshp}{\htmladdnormallink{Report on the First Workshop for the
    Distributed Oceanographic Data System, Proposed System Architectures}
  {http://www.unidata.ucar.edu/packages/dods/archive/reports/workshop1/section3.13.html}}

% not DODS URLs, but useful all the same...

\newcommand{\www}{\htmladdnormallink{World Wide Web} 
  {http://www.w3.org/hypertext/WWW/TheProject.html}}
\newcommand{\url}{\htmladdnormallink{Uniform Resource Locator}
  {http://www.w3.org/hypertext/WWW/Addressing/URL/url-spec.html}}

\newcommand{\uri}{\htmladdnormallinkfoot{Uniform Resource Identifiers}
  {http://www.w3.org/hypertext/WWW/Addressing/URL/uri-spec.html}}

\newcommand{\urn}{\htmladdnormallinkfoot{Uniform Resource Name}
  {http://www.acl.lanl.gov/URI/archive/uri-archive.messages/1143.html}}

\newcommand{\HTTPD}{\htmladdnormallinkfoot{HTTPD}
  {http://hoohoo.ncsa.uiuc.edu/docs/Overview.html}}

\newcommand{\HTTP}{\htmladdnormallinkfoot{HyperText Transfer Protocol}
  {http://www.w3.org/hypertext/WWW/Protocols/HTTP/HTTP2.html}}

%\newcommand{\HTML}{\htmladdnormallinkfoot{HyperText Markup Language}
%  {http://www.w3.org/hypertext/WWW/MarkUp/MarkUp.html}}

% CERN used to be `Conseil Europeen pour la Recherche Nucleaire'
\newcommand{\CERN}
  {\htmladdnormallink{European Laboratory for Particle Physics}
  {http://www.cern.ch/}}

\newcommand{\NCSA}
  {\htmladdnormallink{National Center for Supercomputing Applications}
  {http://www.ncsa.uiuc.edu/}}

\newcommand{\WWWC}{\htmladdnormallink{World Widw Web Consortium}
  {http://www.w3.org/}}

\newcommand{\CGI}{\htmladdnormallinkfoot{Common Gateway Interface}
  {http://hoohoo.ncsa.uiuc.edu/cgi/overview.html}}

\newcommand{\MIME}{\htmladdnormallink{Multipurpose Internet Mail Extensions}
  {http://www.cis.ohio-state.edu/htbin/rfc/rfc1590.html}}

\newcommand{\netcdf}{\htmladdnormallink{NetCDF}
  {http://www.unidata.ucar.edu/packages/netcdf/guide.txn_toc.html}}

\newcommand{\JGOFS}{\htmladdnormallink{Joint Geophysical Ocean Flux Study}
  {http://www1.whoi.edu/jgofs.html}}

\newcommand{\jgofs}{\htmladdnormallink{JGOFS}
  {http://www1.whoi.edu/jgofs.html}}

\newcommand{\hdf}{\htmladdnormallink{HDF}
  {http://www.ncsa.uiuc.edu/SDG/Software/HDF/HDFIntro.html}}

\newcommand{\Cpp}{{\rm {\small C}\raise.5ex\hbox{\footnotesize ++}}}

% Commands

\newcommand{\pslink}[1]{\small
\begin{quote}
  A \htmladdnormallink{postscript}{#1} version of this document is
  available.  You may also use \htmladdnormallink{anonymous
  ftp}{ftp://ftp.unidata.ucar.edu/pub/dods/ps-docs/} to access postscript files
  of all of the DODS documentation.
\end{quote}
\normalsize
}

\newcommand{\declaration}[1]{\small
{\tt {#1}}
\normalsize
}

% All the links from here down are for the white papers. I'm not sure that any
% of these are still valid given the reorganization of the web site. 5/12/98
% jhrg.

\newcommand{\DDA}{\htmladdnormallinkfoot{DODS---Data Delivery Architecture}
  {http://www.unidata.ucar.edu/packages/dods/archive/design/data-delivery-arch/data-delivery-arch.html}}
\newcommand{\Dda}{\htmladdnormallink{Data Delivery Architecture}
  {http://www.unidata.ucar.edu/packages/dods/archive/design/data-delivery-arch/data-delivery-arch.html}}
\newcommand{\dda}{\htmladdnormallink{data delivery architecture}
  {http://www.unidata.ucar.edu/packages/dods/archive/design/data-delivery-arch/data-delivery-arch.html}}

\newcommand{\DDD}{\htmladdnormallinkfoot{DODS---Data Delivery Design}
  {http://www.unidata.ucar.edu/packages/dods/archive/design/data-delivery-design/data-delivery-design.html}}
\newcommand{\Ddd}{\htmladdnormallink{Data Delivery Design}
  {http://www.unidata.ucar.edu/packages/dods/archive/design/data-delivery-design/data-delivery-design.html}}
\newcommand{\ddd}{\htmladdnormallink{data delivery design}
  {http://www.unidata.ucar.edu/packages/dods/archive/design/data-delivery-design/data-delivery-design.html}}

\newcommand{\URL}{\htmladdnormallinkfoot{DODS---Uniform Resource Locator}
  {http://www.unidata.ucar.edu/packages/dods/archive/design/urls/urls.html}}
\newcommand{\Url}{\htmladdnormallink{Uniform Resource Locator}
  {http://www.unidata.ucar.edu/packages/dods/archive/design/urls/urls.html}}
\newcommand{\dodsurl}{\htmladdnormallink{uniform resource locator}
  {http://www.unidata.ucar.edu/packages/dods/archive/design/urls/urls.html}}

\newcommand{\DAP}{\htmladdnormallinkfoot{DODS---Data Access Protocol}
  {http://www.unidata.ucar.edu/packages/dods/archive/design/api/api.html}}
\newcommand{\Dap}{\htmladdnormallink{Data Access Protocol}
  {http://www.unidata.ucar.edu/packages/dods/archive/design/api/api.html}}
\newcommand{\dap}{\htmladdnormallink{data access protocol}
  {http://www.unidata.ucar.edu/packages/dods/archive/design/api/api.html}}

\newcommand{\SOFT}
        {\htmladdnormallinkfoot{DODS---Software Development Environment}
  {http://www.unidata.ucar.edu/packages/dods/archive/managment/software/software.html}}
\newcommand{\Soft}{\htmladdnormallink{Software Development Environment}
  {http://www.unidata.ucar.edu/packages/dods/archive/managment/software/software.html}}
\newcommand{\soft}{\htmladdnormallinkfoot{software development environment}
  {http://www.unidata.ucar.edu/packages/dods/archive/managment/software/software.html}}

% I used to call the DAP the API, then for a few days I called it the
% DTP. These def's were used then. Rather than find everywhere they are used
% (not that hard, but...) I just hacked the macros. NB: the source file for
% the DAP document is still called `api.tex'

\newcommand{\API}{\htmladdnormallinkfoot{DODS---Data Access Protocol}
  {http://www.unidata.ucar.edu/packages/dods/archive/design/api/api.html}}
\newcommand{\api}{\htmladdnormallink{data access protocol}
  {http://www.unidata.ucar.edu/packages/dods/archive/design/api/api.html}}

\newcommand{\DTP}{\htmladdnormallinkfoot{DODS---Data Access Protocol}
  {http://www.unidata.ucar.edu/packages/dods/archive/design/api/api.html}}
\newcommand{\dtp}{\htmladdnormallink{data access protocol}
  {http://www.unidata.ucar.edu/packages/dods/archive/design/api/api.html}}

\newcommand{\TOOLKIT}{\htmladdnormallinkfoot{DODS---Client and Server Toolkit}
  {http://www.unidata.ucar.edu/packages/dods/archive/implementation/toolkits/toolkits.html}}
\newcommand{\Toolkit}{\htmladdnormallink{Client and Server Toolkit}
  {http://www.unidata.ucar.edu/packages/dods/archive/implementation/toolkits/toolkits.html}}
\newcommand{\toolkit}{\htmladdnormallink{client and server toolkit}
  {http://www.unidata.ucar.edu/packages/dods/archive/implementation/toolkits/toolkits.html}}

\newcommand{\TKR}{\htmladdnormallinkfoot{DODS---DAP Toolkit Reference}
  {http://www.unidata.ucar.edu/packages/dods/archive/implementation/toolkits/toolkits.html}}
\newcommand{\Tkr}{\htmladdnormallink{DAP Toolkit Reference}
  {http://www.unidata.ucar.edu/packages/dods/archive/implementation/toolkits/toolkits.html}}
\newcommand{\tkr}{\htmladdnormallink{DAP toolkit reference}
  {http://www.unidata.ucar.edu/packages/dods/archive/implementation/toolkits/toolkits.html}}

% I know that these are wrong. The papers have been moved to the gso web
% site, but I'm not sure exactly where. 5/12/98 jhrg

\newcommand{\DM}{\htmladdnormallinkfoot{DODS---Data Model}
  {http://lake.mit.edu/dods-dir/dodsdm2.html}}
\newcommand{\Dm}{\htmladdnormallink{Data Model}
  {http://lake.mit.edu/dods-dir/dodsdm2.html}}
\newcommand{\dm}{\htmladdnormallink{data model}
  {http://lake.mit.edu/dods-dir/dodsdm2.html}}

\newcommand{\DD}{\htmladdnormallinkfoot{DODS---Data Delivery}
  {http://lake.mit.edu/dods-dir/dods-dd.html}}

% external refs for DODS documents

\externallabels{http://www.unidata.ucar.edu/packages/dods/design/api}
  {/home/www/packages/dods/archive/design/api/labels.pl}
\externallabels{http://www.unidata.ucar.edu/packages/dods/design/data-delivery-arch}
  {/home/www/packages/dods/archive/design/data-delivery-arch/labels.pl}
\externallabels{http://www.unidata.ucar.edu/packages/dods/design/data-delivery-design}
  {/home/www/packages/dods/archive/design/data-delivery-design/labels.pl}
\externallabels{http://www.unidata.ucar.edu/packages/dods/implementation/toolkits}
  {/home/www/packages/dods/archive/implementation/toolkits/labels.pl}

%\renewcommand{\externalref}[2]{#2}

%%% Local Variables: 
%%% mode: latex
%%% TeX-master: t
%%% End: 



% $Id$
% Names for various programs used to document, build, debug and test DODS

\newcommand{\httpd} {{\tt httpd}}
\newcommand{\ftp} {{\tt ftp}}
\newcommand{\ftpd} {{\tt ftpd}}
\newcommand{\xfig} {{\tt xfig}}
\newcommand{\xpaint} {{\tt xpaint}}
\newcommand{\cextdoc} {{\tt cextdoc}}
\newcommand{\cvs} {{\tt cvs}}
\newcommand{\gcc} {{\tt gcc}}
\newcommand{\bison} {{\tt bison}}
\newcommand{\flex} {{\tt flex}}
\newcommand{\cproto} {{\tt cproto}}
\newcommand{\cextract} {{\tt cextract}}
\newcommand{\gdb} {{\tt gdb}}
\newcommand{\dmalloc} {{\tt dmalloc}}
\newcommand{\kdsi} {{\tt kdsi}}
\newcommand{\halstead} {{\tt halstead}}
\newcommand{\mccabe} {{\tt mccabe}}
\newcommand{\calls} {{\tt calls}}
\newcommand{\gnats} {{\tt gnats}}
\newcommand{\etags} {{\tt etags}}
\newcommand{\autoconf} {{\tt autoconf}}
\newcommand{\dejagnu} {{\tt dejagnu}}
\newcommand{\perl} {{\tt perl}}
\newcommand{\guidebugger} {{\tt ddd}}
\newcommand{\make} {{\tt make}}
\newcommand{\Gpp} {{\tt g++}}
\newcommand{\libGpp} {{\tt libg++}}

\newcommand{\latextohtml}{ {\bf LaTeX}{2}{\tt{HTML}} }

\begin{document}

\title{DODS---Software Development Environment and Tools}
\author{}
\date{\today}

\maketitle

\begin{abstract}

  This document describes software tools used to build the Distributed
  Oceanographic Data System. The nature of this project, as described in the
  report of the First Workshop on the Distributed Oceanographic Data System,
  makes it very difficult to rely on commercial software to design and
  implement the system. We have assembled a collection of freely available
  tools which are being used to design and implement DODS\@. Organizations
  that want to contribute to the development of this system should consider
  using these tools so that their efforts can be focused on developing
  components for the system rather than working around differences between
  the development environments.

\end{abstract}


% Biolerplate text warning that anything can (and almost certainly will)
% change. 
%
% jhrg 3/25/94
%
% $Id$

\small

\centerline{{\bf Note}} 

\begin{quotation}

\begin{htmlonly}
  This is a working document made available to solicit comments. To receive
  e-mail notice of new or significantly changed documents, send e-mail
  to \htmladdnormallink{majordomo@unidata.ucar.edu}
  {mailto:majordomo@unidata.ucar.edu} with 'subscribe dods' in the
  body of the message.
\end{htmlonly}

\begin{latexonly}
  This is a working document made available to solicit comments. To receive
  e-mail notice of new or significantly changed documents, send e-mail
  to majordomo@unidata.ucar.edu with 'subscribe dods' in the body of
  the message. 

  Hypertext references in the hardcopy version of this document are
  represented using footnotes that contain a Universal Resource Locator (URL)
  to the referenced document.  To read the online documentation, open the URL
  `http://www.unidata.ucar.edu/' with a World Wide Web browser.
\end{latexonly}

\end{quotation}

\normalsize

%
% Biolerplate text expaining that this document is intended for programmers.
%
% jhrg 4/17/94
%
% $Id$

\small

\begin{quotation}
  This document is written for people interested in implementing software
  that will be part of, or work in parallel with, the Distributed
  Oceanographic Data System. It is also intended for people who will manage
  those developers. It assumes familiarity with the development of small to
  medium sized software systems.
\end{quotation}

\normalsize

\begin{htmlonly}
\pslink{file://dods.gso.uri.edu/pub/DODS/software.ps}
\end{htmlonly}

\newpage

\tableofcontents

\newpage

\section{Introduction}
\label{introduction}

This document describes the software development tools used by \DODS
(DODS) developers at the University of Rhode Island. Most of the tools used
are listed here, but some tools that are of general use outside software
development are not. The purpose of this document is to provide people
working with the DODS team at URI a picture of software development within
the group, and to streamline collaborative work with those groups.  This
document does not describe how to use these tools (e.g., how to use a \Cpp
compiler). 


The remainder of this document is broken into two sections: an overview of
the DODS development process and a listing of the tools along with very
brief explanatory notes about them. 
\begin{htmlonly}
The latter contains links to each package in either the DODS software tools
ftp site or, for those that are actively maintained, their home ftp site.
\end{htmlonly}

\section{Software Development Outlook}

The design, development, and testing of DODS is principally carried out by
four people with varying levels of help from many other people in the
oceanographic community. Because the goals of DODS, as described in the
``\wrkshp'' require a fairly complex system, it is important to streamline
as much of the implementation phase as possible. However, DODS is a research
effort, and it does not make sense to treat the development of DODS as if it
were a commercial venture.

Some important characteristics of the DODS effort:

\begin{itemize}

\item DODS is intended for use on UNIX workstations that are connected to the
  Internet. We assume that any graphical interface software will use the X
  Window System.

\item The DODS design and implementation will strive to reuse as much
  software as possible. In a similar vein, it will try to generate as much
  reusable software as is practical.

\item DODS will work towards a modular design. Components may be usable as
  stand alone programs or tools as well as an integrated system.

\item DODS will automate as many of the implementation and maintenance tasks
  as is practical within the scope of the project. This includes version
  control for source code and documentation, automated testing, problem
  reporting, etc.

\item All DODS source code must be freely distributable.

\item DODS will be developed in an open manner. Documentation of the design
  is made available as it is produced. Software components are made available
  for testing/use as they are built.

\end{itemize}

\section{Software Tools}

This section lists tools that we have assembled to build DODS\@. Some of the
tools listed here might not be used at all during the first year of the
effort, but they are included because when the whole group is taken together
they make an implied development plan for DODS\@. Virtually all the tools
listed here are freely available. All of the tools may be obtained using \ftp
from the DODS \ftp\ archive at 
\htmladdnormallink{ftp://dods.gso.uri.edu/pub/tools}
{ftp://dods.gso.uri.edu/pub/tools}.

\subsection{Design and Documentation}

The tools listed here are grouped according to function within the area of
software design and documentation. Some of these tools are used for other
things as well, and in those cases they may be listed elsewhere too. It is
very tempting to break away from our goal of using only freely available
tools in this area since many relatively inexpensive programs are available
for text and graphics work. However, it has been our experience that these
(free) tools afford us the best mix of short term usefulness and long term
reuse of our papers, figures, etc.

\subsubsection{Text}

Gnu Emacs is used for most text preparation. In addition to excellent support
for many programming related activities, Gnu Emacs is extensible using a
built in lisp dialect called {\em elisp}. If you are willing to forgo emacs'
extensibility, any text or word processing program will pose no compatibility
problems as long as it can produce ASCII text as output.

\subsubsection{Mark up languages}

\TeX\ and \LaTeX\ are used for almost all of our document preparation. All
our online documentation is in HTML\@. Glenn Flierl has a set of \TeX\ macros
that simplify using \TeX\ for short documents. He also has a set of elisp
functions which automate translation of documents written using those macros
into HTML\@.  For documents written in \LaTeX, we use \latextohtml to produce
the online versions.

\subsubsection{Network Access to Documentation}

World Wide Web browsers and \httpd\ are used to provide online access to HTML
documents. In addition, \ftp\ and \ftpd\ are used to provide access to
postscript files which may be downloaded and printed by a user.

\subsubsection{Figures}

Line drawings are done using \xfig. A number of specialized graph drawing
tools have been investigated, but although many support special operations
for creating structure charts, etc., none seemed easier or faster than \xfig.
A freely available bit and pixmap editor named \xpaint\ has also proved useful.

\subsubsection{Source Code Documentation}

Functions in DODS specific libraries need to be documented. It is best to
keep documentation for library functions together with the source code that
implements those functions. \cextdoc\ can be used to extract comments from
source files to automatically create documentation for the functions they
contain. Of course, those comments must be accurate for this to work.

\subsubsection{Version Control}
\label{cvs}

The Concurrent Version System (CVS) is used for version control of all
documentation, both text and figures\footnote{HTML documents are not under
CVS control if they are generated from controlled source documents.}.  CVS
manages a single repository for the documents which can be shared by several
machines.  People on any one of those machines can work with the
documentation and be sure that changes they make will be both documented and
will not be lost as others simultaneously work on the documents. CVS is used
for source code management as well.

\subsection{Program Development, Testing and Maintenance}

This section describes many tools that are being used or are planned to be
used during the program development, testing and maintenance phases of work
on DODS.

\subsubsection{Programming language compilers and interpreters}

There are many tools used by DODS that fall into the compiler or interpreter
category. However, the majority of source code used to build DODS is \Cpp
and ANSI C source code. This source code is written with the explicit
knowledge that it will be compiled with \gcc\ version 2.7.x or greater. In
addition to C and \Cpp, other languages used are the \flex, \bison, {\tt
awk}, \perl\ and the Bourne shell. Note that \flex\ and \bison\ are compiled
languages while {\tt awk}, \perl\ and {\tt sh} are interpreted.

While not compilers themselves, \cproto\ and \cextract\ are two tools that
can be used to automatically generate ANSI C function prototypes. These are
used to build prototype include files for modules which use a set of function
defined in another module or for library header files. Other developers do
not need to use these, but they should still generate ANSI prototypes for
functions used outside their own module.

\subsubsection{Debugging}

\gdb\ is used for interactive debugging of C programs. \guidebugger\ is a
free GUI interface to \gdb\ and {\tt dbx} which makes following data
structures simpler. The library \dmalloc\ is used to find problems with
dynamically allocated memory. Since \dmalloc\ requires modifications to
source code, and because it is very useful to retain such test code so that
it may be re-tested when changes are made, other DODS developers should use it
rather than something else. We chose \dmalloc\ over other tools because it
works on all our development platforms (Sun4, Alpha and Dec Mips).

\subsubsection{Testing, Analysis and Maintenance}

We use \dejagnu\ to build test suites for DODS\@. While \dejagnu\ is required
to run these, they are very useful. Every piece of software that is part of
DODS should have associated with it test. 

There are several tools that can be used to analyze source files for
programs. \kdsi\ is a source line counter which also returns simple statistics
on comments. \halstead\ and \mccabe\ return various measurements of functional
and modular complexity for C source files. Simple function call graphs may be
built using the \calls\ tool. In addition, the emacs source file tag program
\etags\ can be used to ease navigation through someone else's source files by a
maintainer that is unfamiliar with those source files.

All though we have not used a problem report database on other projects, the
distributed development plan for DODS indicates that such a problem tracking
tool might be of considerable benefit. \gnats\ is such a tool.

\subsubsection{Version Control}

As described in section~\ref{cvs}, we use \cvs\ for version control.

\subsubsection{Program builds}

DODS will use Gnu Make to direct the compilation and linkage of programs and
libraries. People (users or developers) that want to build DODS from the
source code must be willing to use Gnu Make since it supports features DODS
may use that are not part of many other make utilities. 

DODS may use Gnu's \autoconf\ system to port DODS from its development platform
to other intended platforms.

\section{Concise Listing of Tools}

These tables summarize the tools used to build DODS.

\begin{table}[hp]
\caption{Tools used for documentation and design}
\begin{center}
\begin{tabular}{|l|l|}
\hline
{\em Name} & {\em Use} \\ \hline

\htmladdnormallink{Gnu Emacs}{ftp://prep.ai.mit.edu/pub/gnu} 
& Text editing \\

\htmladdnormallink{\TeX}{ftp://ftp.cs.umb.edu:pub/tex/unixtex.ftp} 
& Typesetting \\

\htmladdnormallink{\LaTeX}{ftp://ftp.cs.umb.edu:pub/tex/unixtex.ftp} 
& Typesetting \\

\htmladdnormallink{\TeX\ $\rightarrow$ html macros}
{ftp://dods.gso.uri.edu/pub/tools/grf-tex-example.tex}
& Online HTML generation \\

\htmladdnormallink{\latextohtml}
{http://www-dsed.llnl.gov/files/programs/unix/latex2html/}
& Online HTML generation\\

\htmladdnormallink{NetScape}{http://home.netscape.com/}
& Access to documents; GUI construction \\

\htmladdnormallink{\httpd}{http://hoohoo.ncsa.uiuc.edu/}
& Provide online documents \\

\htmladdnormallink{\ftpd}{http://wuarchive.wustl.edu/packages/wuarchive-ftpd/}
& Provide postscript documents \\

\htmladdnormallink{\xfig}
{ftp://ftp.x.org/contrib/applications/drawing_tools/xfig/}
& Figure generation \\

\htmladdnormallink{\cextdoc}{ftp://dods.gso.uri.edu/pub/tools/}
& Source code documentation \\

\htmladdnormallink{\cvs}{ftp://prep.ai.mit.edu/pub/gnu/} 
& Version control \\ \hline
\end{tabular}
\end{center}
\end{table}

\begin{table}[hp]
\caption{Tools used for development, testing and maintenance}
\begin{center}
\begin{tabular}{|l|l|}
\hline
{\em Name} & {\em Use} \\ \hline
\htmladdnormallink{\gcc}{ftp://prep.ai.mit.edu/pub/gnu/} 
& ANSI C compiler \\

\htmladdnormallink{\Gpp}{ftp://prep.ai.mit.edu/pub/gnu/}
& C++ compiler \\

\htmladdnormallink{\libGpp}{ftp://prep.ai.mit.edu/pub/gnu/}
& General C++ class library \\

\htmladdnormallink{\flex}{ftp://prep.ai.mit.edu/pub/gnu/} 
& Lexical scanner generator \\

\htmladdnormallink{\bison}{ftp://prep.ai.mit.edu/pub/gnu/} 
& Parser generator \\

\htmladdnormallink{sed}{ftp://prep.ai.mit.edu/pub/gnu/}
& Pattern based text manipulator \\

\htmladdnormallink{awk}{ftp://prep.ai.mit.edu/pub/gnu/}
& Pattern based scripting language \\

\htmladdnormallink{\perl}{ftp://prep.ai.mit.edu/pub/gnu/}
& General scripting language \\

\htmladdnormallink{\cproto}{ftp://dods.gso.uri.edu/pub/tools/} 
& C function prototype extractor \\

\htmladdnormallink{\cextract}{ftp://dods.gso.uri.edu/pub/tools/} 
& C function prototype extractor \\

\htmladdnormallink{\gdb}{ftp://prep.ai.mit.edu/pub/gnu/} 
& Runtime debugger \\

\htmladdnormallink{\guidebugger}
{ftp://ftp.x.org/contrib/utilities/ddd-1.4a.README}
& GUI debugger interface \\

\htmladdnormallink{\dmalloc}{ftp://ftp.letters.com/src/dmalloc/} 
& Dynamic memory debugger \\

\htmladdnormallink{\calls}{ftp://dods.gso.uri.edu/pub/tools/} 
& Simple function call graphs \\

\htmladdnormallink{\kdsi}{ftp://dods.gso.uri.edu/pub/tools/} 
& C source line counting \\

\htmladdnormallink{\mccabe}{ftp://dods.gso.uri.edu/pub/tools/} 
& C source complexity statistics \\

\htmladdnormallink{\halstead}{ftp://dods.gso.uri.edu/pub/tools/} 
& C source complexity statistics \\

\htmladdnormallink{\gnats}{ftp://prep.ai.mit.edu/pub/gnu/} 
& Problem report database \\

\htmladdnormallink{\etags}{ftp://prep.ai.mit.edu/pub/gnu/} 
& Function tags \\ 

\htmladdnormallink{\cvs}{ftp://prep.ai.mit.edu/pub/gnu/} 
& Version control \\ 

\htmladdnormallink{\make}{ftp://prep.ai.mit.edu/pub/gnu/}
& Program builds \\

\htmladdnormallink{\autoconf}{ftp://prep.ai.mit.edu/pub/gnu/} 
& Source code porting \\

\htmladdnormallink{\dejagnu}{ftp://prep.ai.mit.edu/pub/gnu/}
& Test managment \\ \hline
\end{tabular}
\end{center}
\end{table}

\end{document}