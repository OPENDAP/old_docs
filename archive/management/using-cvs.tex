
% Notes on using cvs.
%
% $Id$

\documentclass{dods-paper} 

\rcsInfo $Id$

\T\setcounter{secnumdepth}{4}     % Put \paragraph in TOC
\T\setcounter{tocdepth}{4}
\T\setlength{\parindent}{0em}     % Amount of indentation
\T\addtolength{\parskip}{1ex}     % Vertical separation

\W\newcommand{\*}{*}

\title{Using CVS}
\author{James Gallagher\thanks{jgallagher@opendap.org}}
\date{\rcsInfoDate, Revision: \rcsInfoRevision}
\htmladdress{James Gallagher <jgallagher@gso.uri.edu>, 
  \rcsInfoDate, Revision: \rcsInfoRevision}
\htmldirectory{cvs-html}
\htmlname{cvs}
\figpath{.}

% Change paragraph typesetting; eliminate indenting and add more space between
% paragraphs.
\begin{document}

\maketitle

\T\tableofcontents

\section{Introduction}

This is a summary of CVS tricks I've found useful. Some of this information
is hard to find in the (few) books on CVS and some is specific to the OPeNDAP
effort. This is \emph{not} a replacement for reading
the CVS documentation. There's also some information on using CVS as part of
the release process in ``OPeNDAP Software Release Process.''

\section{Using CVS to get the OPeNDAP Software}

To start the ball rolling, check out the OPeNDAP software module. Unless you
are working on \lit{dodsdev.gso.uri.edu} you will be using CVS to access a
remote repository. We used to use the \lit{pserver} method of login, but
since June 2004 we've been using \lit{ssh}. Here's how to access the
repository:

\begin{itemize}
\item Set \lit{CVS\_RSH} to \lit{ssh}
\item Set \lit{CVSROOT} to
  \lit{:ext:<user>@dodsdev.gso.uri.edu:/usr/local/cvs-repository} where
  \lit{<user>} is your user name (or \lit{anoncvs} if you want
  anonymous/read-only access. Instead of setting CVSROOT, you can use the CVS
  option -d \lit{:ext:<user>@dodsdev.gso }\ldots Once you have checked out
  directories, you can run CVS commands within them without setting CVSROOT
  or using -d.
\end{itemize}

Make sure not to add a trailing \lit{/} to the repository name; some
versions of CVS gag if you do.

If you want to get the entire OPeNDAP source tree, use the command:.

\begin{vcode}{it}
cvs -d :ext:<user>@dodsdev.gso.uri.edu:/usr/local/cvs-repository co DODS
\end{vcode}

It should ask for your password. Unlike the \lit{pserver} access method,
using \lit{ssh} will prompt you for your password \emph{every} time you run a
CVS command that affects the remote repository. This can get old quickly;
see Section~\ref{sec:automatic-login}. 

This will check out the entire \lit{DODS} \emph{module} from CVS. If this is
what you want, you're all set. To build the software you may need autoconf,
GNU make, bison and flex in addition to a decent C++ compiler (we're building
with gcc 3.2.1 and VC++ 6.0).

In the past the DODS CVS module was much larger. To help people cope with
that we used to advise that they check out only the part of the module in
which they were interested. There's little need for this now, but in case
someone finds the information useful, here's how to do it:

\begin{quote}
From now on I'm going to leave off the -d option and repository string.
You'll need it unless you set CVSROOT or are working inside already checked
out directories.
\end{quote}

\begin{vcode}{it}
cvs co -l DODS
\end{vcode}

This will checkout the root directory without recurring through all the
subdirectories (the \lit{-l} option suppresses recurring on all the
subdirectories). You will have to select the subdirectories you want by hand.
Unfortunately, there is no way to tell it to checkout everything except
$<foo>$ (\lit{co} is an alias for \lit{checkout}).

Once you have checked out the root directory you no longer need to feed CVS
the \lit{-d <big long string>} thing; CVS remembers which repository to use.

Use:

\begin{vcode}{it}
cvs co DODS/etc

cvs co DODS/src/dap
cvs co DODS/src/nc-dods
\end{vcode}

to get only the source directories for the DAP library and netCDF server, for
example, or

\begin{vcode}{it}
cvs co DODS/src
\end{vcode}

to get all the source directories.

Because it is not possible to get CVS to create the subdirectories of DODS without
checking out the subdirectories contents (which would amount to the entire DODS
module - it is very large) it is hard to know which subdirectories you might want.
Here is a list of the subdirectories under DODS:

\begin{vcode}{it}
src     The source code.
etc     Destination for servers and also some support scripts.
bin     You must make this by hand.
lib     You must make this by hand.
include The include files installed by src/dap.
\end{vcode}

As with all software, there are dependencies that exist between these
directories. If you get the \lit{src} directory, you will also need \lit{etc}
and include before you can build any of the software. In addition, you must
make the \lit{bin} and \lit{lib} directories.

\section{Write Access and Automatic Logins}
\label{sec:automatic-login}

In the past write access to CVS used \lit{pserver}. We stopped supporting
that and switched to \lit{ssh} in June of 2004. Using \lit{ssh} is more
secure than \lit{pserver} since passwords are not sent over the network as
plain text. However, to get write access to the repository, you'll now need a
full-fledged account on \lit{dodsdev.gso.uri.edu}. Contact Mark Schneider
\lit{<mschneider@gso.uri.edu>} or James Gallagher
\lit{<jgallagher@opendap.org>} for a login.

Once you have a login, you will likely soon tire of re-typing your password.
Here's how to set up automatic \lit{ssh} login. This is particularly useful if
you're access CVS code from within an IDE. 

\emph{[Thanks to Rob Morris and Mark Schneider for the following. jhrg]}

\subsection{Automatic login from Unix}

Here's Unix instructions to \lit{ssh} to \lit{dodsdev.gso.uri.edu} without
requiring logging in (tested under Red Hat and Mandrake Linux, Solaris, OSF and
OS X).

On the host machine you want to login from:
\begin{enumerate}
\item Generate a \lit{ssh} key\\
\begin{vcode}{it}
\% ssh-keygen -t dsa
\end{vcode}

Just hit return as the pass phrase.

\item Copy the key generated to account on \lit{dodsdev}\\
\begin{vcode}{it}
\% scp .ssh/id_dsa.pub <user>@dodsdev.gso.uri.edu:.ssh/<unique filename>
\end{vcode}

Enter your dodsdev password when prompted.

\item Repeat step one and two for each host. Use a different unique filename
    for each iteration.

\item Use \lit{ssh} to login to \lit{dodsdev.gso.uri.edu}. For each filename
    that was copied in step two, use an editor to paste the keys in those
    files into your \lit{\$(HOME)/.ssh/authorized\_keys} file. One key per
    line.
\end{enumerate}

Your should now be able to \lit{ssh} to \lit{dodsdev.gso.uri.edu} without
entering a password (use the \lit{ssh} option -l \lit{<user>} when your user
name on a host differs from that on \lit{dodsdev}).

To get CVS assess to dodsdev without a password from series
of UNIX hosts:
\begin{enumerate}
\item Do the above steps to setup ssh for each unix host you
    desire.
\item Add the following variables to your environment for that
    host
    \begin{vcode}{it}
      export CVS\_RSH=ssh
      export CVSROOT=:ext:<user>@dodsdev.gso.uri.edu:/usr/local/cvs-repository
    \end{vcode}
\end{enumerate}

\subsection{Automatic login from Win32}

On the win32 host machine you want to login from:
\begin{enumerate}
\item Download \lit{putty.zip} from \lit{http://www.puttyssh.org/} and unzip
    it into \lit{"C:$\backslash$Program Files"} (or a directory of your choice).
\item Launch \lit{puttygen} and generate a ``1024 bit DSA key".
\item Paste the public key genared into your
    \lit{\$(HOME)/.ssh/authorized\_keys} 
    file on dodsdev.
\item Save the private key to disk (in your putty install directory
    probably) to some file such as \lit{dodsdev\_private.ppk} or a
    name of your choice.
\item Create a short cut to \lit{pagent} and modify that shortcuts'
    properties ``Target" field so that pagent takes a 1st arg
    of the file saved in step number four.
\item Put that shortcut in your startup folder and reboot
    (or just run it).  Your private key should have been
    loaded.
\item You can now configure and run \lit{puttySSH} in a rather
    straight-forward manner.
\end{enumerate}

Your should now be able to use \lit{ssh} to access \lit{dodsdev.gso.uri.edu}
without entering a password.

To get cvs assess to dodsdev from a win32 host:
\begin{enumerate}
\item Do the above steps to setup \lit{ssh} if necessary.
\item Download and install to \lit{tortoiseCVS} from
  \lit{http://tortoisecvs.sourceforge.net/}
\item Add the following variables to your environment for that
    host.
    \begin{vcode}{it}
    CVS_RSH      TortoisePlink.exe
    CVSROOT      :ext:<user>@dodsdev.gso.uri.edu:/usr/local/cvs-repository

    C:\> cd <some empty dir>
    C:\> cvs co DODS -r release-3-4
    \end{vcode}
    Or just right click in windows explorer and select ``cvs checkout"
    (very slick).
\end{enumerate}

To make it so that CVS doesn't require a password:
\begin{enumerate}
\item Create a short cut to \lit{pagent} and modify that shortcuts'
    properties ``Target" field so that pagent takes a 1st arg
    of the file saved in step number four of the ssh install instructions
    above.
\item Put that shortcut in your startup folder and reboot
    (or just run it).  Your private key should always load
    at login time (transparently).
\end{enumerate}

\section{Branching}

If you have checked out code a want to check changes in on a branch instead
of the main trunk, do the following:
\begin{enumerate}
\item Tag the branch point.
\item Create the branch.
\item Update your copy of the code so that it is associated with that branch.
\item Check in the code.
\end{enumerate}

\begin{quote}
If you want to make a branch off of a branch, these same steps will work.
However, it is best to stick to the trunk-and-branch model and not bifurcate
branches themselves since it can be very hard for other people to figure out
just where a particular revision fits into the CVS module.
\end{quote}

Perform the following steps in the directory that contains the code you want
to move to the new branch (which I'll call \lit{test-branch}):

\begin{vcode}{it}
cvs tag root-of-test-branch
cvs tag -b test-branch
\end{vcode}

The first command marks the place just before the branch point. This is
important if you later need to get the code as it was just before you
branched. The \lit{test-branch} tag \emph{does not} provide a way to do that.
In fact accessing that tag will get you the most recent code \emph{on the
branch}. So, you need to create a plain tag to be able to refer to the code
as it existed right before the branch was created.

The second command creates the branch. The \lit{-b} option makes this tag a
branch. The symbolic name \lit{test-branch} can be used to refer to the
branch. Without a symbolic name, you'll be hosed.\footnote{In general, never
  use the CVS revision numbers unless you have to. Symbolic names are the
  most practical way to deal with groups of files. However, CVS does support
  date-based access including dates like \lit{YESTERDAY}. This is a very
  useful feature!}

\begin{vcode}{it}
cvs update -r test-branch
\end{vcode}

This updates the tag of the code in the current working directory to
\lit{test-branch}. Since the tag is on a branch, now so is this code.

\begin{vcode}{it}
cvs ci -m "<some message>"
\end{vcode}

The code is now checked in on the branch.

\begin{quote}
Note that there's a test CVS module that can be checked out and played with.
You cannot do anything too terrible to this since the `code' is just a couple
of test files. Before doing anything that might cause the DODS repository to
get hosed, try it out there. The name of the test module is \lit{cvs-test}.
\end{quote}

\section{Merging}
\label{sec:merge}

Suppose you create a branch and make several releases from it. All the while,
other developers are adding changes to the trunk and changes/fixes to the
released code. As each successive release is made, the changes associated
with that release should be merged back into the trunk software. However, if
the entire branch is merged back onto the trunk some changes/fixes will be
lost --- they will be undone.\footnote{This is an odd quirk of CVS. It lets
you remove changes made on a segment of a branch. It's important to
\emph{not} do that by mistake since it's rarely what you want to do.} Here's
how to work around that.

\begin{verbatim}

                   --------------------------------
                  /       ^           ^           ^
                 /        3-1-RC1   3-1-0       3-1-1
                /
     3-1       /
       \      /
        \--> /
   -------------------------------------------------------
                  Trunk -->
\end{verbatim}

Figure: The CVS trunk and several tags. The 3-1 tag defines the code that
forms the basis for a release, 3-1-RC1 is the first release candidate for
version 3.1, the tag 3-1 is the first code sent to users and 3-1-1 is first
set of fixes for that code. (The tags in the figure don't include the word
\lit{release} because that made it too crowded.)

First, tag every release point. Second only merge revisions with tags --- this
will keep things reasonably sane. Referring to the figure above, once the
3-1-RC1 tag is created (but before any further work is done) the software
could be diagrammed as:


\begin{verbatim}
                   --------
                  /       ^
                 /        3-1-RC1
                /
     3-1       /
       \      /
        \--> /
   -------------------------------------------------------
                  Trunk -->
\end{verbatim}

The command to merge the `3-1-RC1 changes' back onto the trunk is:

\begin{vcode}{it}
cvs update -j 3-1
\end{vcode}

This command is run in a directory that contains the trunk software.

After a while more changes are made and the 3-1-0 tag is created. A diagram
of the software now looks like:


\begin{verbatim}
                   --------------------
                  /       ^           ^
                 /        3-1-RC1   3-1-0
                /
     3-1       /
       \      /
        \--> /
   -------------------------------------------------------
                  Trunk -->
\end{verbatim}

To merge the changes made between the 3-1-RC1 and 3-1-0 tags, use:

\begin{vcode}{it}
cvs update -j 3-1-RC1 -j 3-1
\end{vcode}

this command merges the changes between the two tags without removing the
changes between the start of the branch and the tag 3-1-RC1 (which have
already been merged). Note that the 3-1-0 tag is not used in this command. It
is, however, a good idea to make it before doing the merge to make sure that
the tag is created since subsequent merges will depend on it.

\section{Useful commands}

\subsection{Shell macros}

Here are some (Bourne) shell aliases I find useful:
\begin{description}
\item [\lit{cvss}] Show the status of any file that is not up to date in CVS.
This is useful when you want to know which files have been changed.

\begin{vcode}{it}
alias cvss='cvs status 2>&1 | grep "'"^File\|Examining"'" | grep -v Up-to-date'
\end{vcode}

\item [\lit{cvsl}] This does the same thing as cvss, but for the current
directory only.

\begin{vcode}{it}
alias cvsl='cvs status -l 2>&1 | grep "'"^File\|Examining"'" | grep -v Up-to-date'
\end{vcode}


\item [\lit{cvsu}] Show which files in the CWD are not in CVS. This is useful
when you're adding files and you want to make sure that something has not
been left out.

\begin{vcode}{it}
alias cvsu='cvs status -l 2>&1 | grep Unknown'
\end{vcode}

\item [\lit{cvsm}] Get the filename only of every file in the CWD that's
changed since the last check in. You can use this with an editor to great
advantage. For example:

\begin{vcode}{it}
     emacs `cvsm` 
\end{vcode}

Opens all the changed files. You can then use Emacs' CVS tools to look at the
diffs, etc.

\begin{vcode}{it}
alias cvsm='cvsl | cut -f 2 -d " "'
\end{vcode}

\item [\lit{cvsc}] Which files in the CWD had conflicts on during the last
merge operation? Note that this macro is also set up for use with an editor.

\begin{vcode}{it}
alias cvsc='cvsl | grep conflicts | cut -f 2 -d " "'
\end{vcode}

\end{description}

\subsection{CVS logs}

Looking at logs (the stuff you write into \lit{cvs commit}):

\begin{vcode}{it}
cvs log -N -l -d ">1999-05-19" | more
\end{vcode}

Lists all the logs changed on or after 5/19/1999 (use $<$ for only after).
The \lit{-l} option suppresses recursive behavior and \lit{-N} suppresses
listing the symbolic tag names (which can be a long list for each file). When
you're making a source release this can be an easy way to update the
\lit{ChangeLog} file.

\subsection{Adding files to a module}

Adding files:

\begin{vcode}{it}
cvs add <file>|<pattern>
\end{vcode}

Adds the file or files. You must run \lit{cvs commit} to actually add them to
the repository.

Binary file:

\begin{vcode}{it}
cvs add -kb <file>|<pattern>
\end{vcode}

The \lit{-kb} flag suppresses keyword expansion; in case binary files (data
files, GIF images, ...) contain a sequence of bytes that just happens to be
\lit{\$Id\$} or any of the other keywords recognized by CVS, using the
\lit{-kb} flag will keep those bytes from being changed! If you already added
a binary file you can use the \lit{cvs admin} command to turn of keyword
expansion:

\begin{vcode}{it}
cvs admin -kb <file>|<pattern>
\end{vcode}

\subsection{Checking out code}

Checking out everything before a certain date:

\begin{vcode}{it}
cvs checkout -D 1999-03-25 DODS/src/nc-dods
\end{vcode}

checks out all the sources in \lit{DODS/src/nc-dods} as they were \emph{on or
before} 25 March 1999.

Checking out tagged sources. See the CVS manual or info pages about tags. I
use tags for \emph{all} the releases, so you need to look at this:

\begin{vcode}{it}
cvs checkout -r release-2-10 DODS/src/nc-dods
\end{vcode}

checks out all the sources tagged \lit{release-2-10}. Note that CVS does not
allow dots (.) in tag names. Also note that if the tag is a branch point you
now have a branch of the source tree to work in. Changes you make will be
checked in on the branch, not the trunk (that's good). To move the changes
over to the trunk see Section~\ref{sec:merge}.

The checkout command does a complete checkout of a module or part of a
module. If you want new sources inside an existing project tree, use
\lit{update}. If the file or directory does not exist, use \lit{update -d}.
In fact, it is best to use \lit{update -d} always unless you \emph{know} that
no new files or directories have been added.\footnote{Also useful with
\lit{update} is the \lit{-P} option which will prune any empty directories.
CVS keeps the directory when files are removed because other revisions might
need that directory. The \lit{-P} option removes that clutter when it's not
necessary.} Here's examples of the two above operations using update instead
of checkout:

\begin{vcode}{it}
cvs update -d -D 1999-03-25 nc-dods

cvs update -d -r release-2-10 nc-dods
\end{vcode}

From within \lit{DODS/src/} will get the \lit{nc-dods} sources from on or
before 25 march 1999 or tagged \lit{release-2-10}. The difference is that
these commands are run from within an existing tree; the checkout is run
above the tree.

\subsection{Using the status command}

Looking at information about the revisions of a source file:

\begin{vcode}{it}
cvs status -v ChangeLog
\end{vcode}

Here's the output for that file:

\begin{vcode}{it}
[dcz:/usr/local/ferret-DODS/src/nc-dods-2.19.2] cvs status -v ChangeLog
===================================================================
File: ChangeLog         Status: Up-to-date

   Working revision:    1.3.4.2 Wed Apr 14 03:33:45 1999
   Repository revision: 1.3.4.2 /usr/local/cvs-repository/DODS/src/nc-dods/ChangeLog,v
   Sticky Tag:          ferret (branch: 1.3.4)
   Sticky Date:         (none)
   Sticky Options:      (none)

   Existing Tags:
	ferret                          (branch: 1.3.4)
	release-2-18                    (branch: 1.3.2)
	no-gnu                          (branch: 1.2.4)
	release-2-17                    (branch: 1.2.2)
\end{vcode}

You can see that there are four tagged revisions (the tags are ferret,
\lit{release-2-18}, \lit{no-gnu} and \lit{release-2-17}). The \lit{Sticky
Tag} indicates that I'm using the revision tagged \lit{ferret}. Note that it
is a tag on a branch and that I've made some changes. Here's the status of
the same file in the development directory (i.e., on the trunk):

\begin{vcode}{it}
[dcz:/usr/local/DODS/src/nc-dods] cvs status -v ChangeLog
===================================================================
File: ChangeLog         Status: Up-to-date

   Working revision:    1.3     Thu Jan 21 22:56:55 1999
   Repository revision: 1.3     /usr/local/cvs-repository/DODS/src/nc-dods/ChangeLog,v
   Sticky Tag:          (none)
   Sticky Date:         (none)
   Sticky Options:      (none)

   Existing Tags:
	ferret                          (branch: 1.3.4)
	release-2-18                    (branch: 1.3.2)
	no-gnu                          (branch: 1.2.4)
	release-2-17                    (branch: 1.2.2)
\end{vcode}

Note that the same tags are listed but that the Stick Tag is none and the
revision number of the working file is 1.3, not 1.3.4.2 as in the previous
example.

Aside: Don't confuse CVS' revision numbers with our version numbers.

\appendix

\section{ChangeLog}

\begin{verbatim}
$Log: using-cvs.tex,v $
Revision 1.8  2004/06/23 19:52:28  jimg
Added new CVS info for win32 from Rob.

Revision 1.7  2004/06/22 23:30:27  jimg
Fixed/replace Sectionref usage

Revision 1.6  2004/06/22 20:45:18  jimg
Updated with information about the new CVS access method (using ssh).

Revision 1.5  2004/01/09 20:01:02  jimg
Fixed some errors and made it less of a collection of old ASCII files and
more of a single paper.

Revision 1.4  2004/01/08 21:08:10  jimg
Formatted to the end and removed some material that was better covered in
"OPeNDAP Software Release Process."

Revision 1.3  2004/01/08 19:15:16  jimg
Started editing and formatting.

Revision 1.2  2003/06/19 00:36:16  jimg
Start at formatting...

Revision 1.1  2003/06/19 00:12:33  jimg
Initial version.
\end{verbatim}

\end{document}