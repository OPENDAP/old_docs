
% Intellectual Property Inventory for DAS, llc
%
% 8/23/2000 jhrg

\documentclass{article}
%\usepackage{html}
%\usepackage{psfig}
%\usepackage{changebar}
%\usepackage{acronym}
\usepackage{gloss}
\usepackage{xspace}

% Note: to get the gloosary to work, run bibtex in the registry.gls.aux file,
% then latex registry, then bibtex *.gls, then latex. Also, make sure to set
% your BST and BIBINPUTS environment variables so that the BST and BIB files
% will be found.
\makegloss

% Change paragraph typesetting; eliminate indenting and add more space between
% paragraphs. 2/15/2000 jhrg
\setlength{\parindent}{0em}     % Amount of indentation
\addtolength{\parskip}{1ex}     % Vertical separation

\newcommand{\dap}{\rm {\small DAP}\raise.5ex\hbox{\footnotesize ++}\xspace}

\begin{document}

\title{DAS Intellectual Property Inventory}
\author{James Gallagher\thanks{Data Access Software, {\sc L.L.C.}, 
    jgallagher@gso.uri.edu}}
\date{\today \\ $Revision$ }

\bibliographystyle{plain}

\maketitle
\tableofcontents

\section{Software by Creation Date}

This section lists each individual software component developed by DODS.
Unless otherwise noted, all the software is copyright URI/MIT. See
Appendix~\ref{URI-copyright}. For each piece of software the date given is
the date on which the software was completed. If substantial changes were
made to the software after that date, those changes and their dates are also
given. 

 \subsection{The DAP++ library}

\begin{itemize}
\item 1994/8 The transport modules
\item 1995/12 The constraint expression evaluator
\end{itemize}

\subsection{Servers}

\begin{itemize}
\item 1994/12 NetCDF. This server was upgraded from using the netCDF
    2.x API to the netCDF 3.x API on 1999/7.
\item 1995/1 JGOFS
\item 1996/9 DSP
\item 1996/9 HDF\footnote{California Institute of Technology holds the
    copyright to this software. See Appendix~\ref{HDF-copyright}.}
\item 1996/10 Matlab
\item 1998/4 FreeForm
\end{itemize}

\subsection{Tools}

\begin{itemize}
\item 1998/3 \texttt{asciival}
\item 1999/4 \texttt{www\_int}\footnote{Based on a design from 1997/7.}
\end{itemize}

\subsection{Clients}

\begin{itemize}
\item 1995/7 NCView X Window system client
\item 1996/10 Matlab command line client
\item 1997/6 Matlab graphical client
\item 1999/6 IDL command line client\footnote{This is the CVS entry date; the
    software was written under contract to URI by Research Systems, Inc.
    a few months earlier.}
\end{itemize}

\section{Features by date of introduction}

This section lists dates features were introduced into different software
components. 

N.B.: This is very slow going. I'll do this for the \dap code and we can all
decide if this needs to be done for the other software. 

\subsection{The DAP++ library}
\begin{table}[h]
\caption{Features in the \dap and the date they were added}
\label{tab:dap-features}
\begin{minipage}{\linewidth}
\begin{center}
\begin{tabular} {|rl|} \hline
1994/12 & Attribute vectors \\ \hline

1995/3 & Changes to the transmission and representation of strings \\
%1995/11 & Vector class introduced \\ 
\hline

1996/1 &  Multiple virtual connections \\
%1996/2 &  Parser argument class\\
1996/3 &   Constraint expressions \\
1996/5 &   Constraint expression functions\\
1996/6 &  First version of the progress indicator\\
1996/8 &   Int16, Float32 types\\ \hline

1997/5 &   Aliases to the attributes \\
1997/5 &   Lexical scoping to the attributes \\
%1997/8 &   DODSFilter class (Unfies some server operations)\\
%1997/9 &   DataDDS class\\
1997/9 &  Constraint expressions fully realized\footnote{The earlier
  implementation did not handle all possible operations as described in the
  design, only the more common ones.} \\
1997/12 &   Reference manual comments to header files\\ \hline

1998/2 &  Negotiated compression of data responses \\
1998/9 &  Shorthand variable names in CEs \\
1998/10 & Synthetic variables for CEs \\ \hline

1999/1 &  Grid Selection Expressions constraint function \\
1999/1 &   UInt16 datatype \\
1999/2 &   Accept-Types MIME header to HTTP request \\
1999/3 &   Int16, \ldots, types to the attributes \\
%1999/3 &  Constraint relational operators implemented as templates\\
%1999/5 &  InternalErr class\\
1999/12 &   Client-side caching \\ \hline

%2000/1 &  Exception handling\\
2000/3 &  \dap split into gui and non-gui versions \\
2000/4 &  New progress indicator \\
2000/6 &  PC Port \\ \hline
\end{tabular}
\end{center}
\end{minipage}
\end{table}

\clearpage
\appendix

\section{URI-MIT Copyright}
\label{URI-copyright}
\begin{verbatim}
(c) Copyright 1994-2000 by
The University of Rhode Island and The Massachusetts Institute of Technology

Portions of this software were developed by the Graduate School of
Oceanography (GSO) at the University of Rhode Island (URI) in collaboration
with The Massachusetts Institute of Technology (MIT).
 
Access and use of this software shall impose the following obligations and
understandings on the user. The user is granted the right, without any fee
or cost, to use, copy, modify, alter, enhance and distribute this software,
and any derivative works thereof, and its supporting documentation for any
purpose whatsoever, provided that this entire notice appears in all copies
of the software, derivative works and supporting documentation.  Further,
the user agrees to credit URI/MIT in any publications that result from the
use of this software or in any product that includes this software. The
names URI, MIT and/or GSO, however, may not be used in any advertising or
publicity to endorse or promote any products or commercial entity unless
specific written permission is obtained from URI/MIT. The user also
understands that URI/MIT is not obligated to provide the user with any
support, consulting, training or assistance of any kind with regard to the
use, operation and performance of this software nor to provide the user
with any updates, revisions, new versions or "bug fixes".

THIS SOFTWARE IS PROVIDED BY URI/MIT "AS IS" AND ANY EXPRESS OR IMPLIED
WARRANTIES, INCLUDING, BUT NOT LIMITED TO, THE IMPLIED WARRANTIES OF
MERCHANTABILITY AND FITNESS FOR A PARTICULAR PURPOSE ARE DISCLAIMED. IN NO
EVENT SHALL URI/MIT BE LIABLE FOR ANY SPECIAL, INDIRECT OR CONSEQUENTIAL
DAMAGES OR ANY DAMAGES WHATSOEVER RESULTING FROM LOSS OF USE, DATA OR
PROFITS, WHETHER IN AN ACTION OF CONTRACT, NEGLIGENCE OR OTHER TORTUOUS
ACTION, ARISING OUT OF OR IN CONNECTION WITH THE ACCESS, USE OR PERFORMANCE
OF THIS SOFTWARE.
\end{verbatim}

\section{Caltech Copyright}
\label{HDF-copyright}
\begin{verbatim}
Copyright (c) 1996, California Institute of Technology.  
ALL RIGHTS RESERVED.   U.S. Government Sponsorship acknowledged. 

THESE TERMS APPLY TO ALL USERS OF THIS SOFTWARE.  Access and use of
this software shall impose the following obligations and
understandings on the user.  The user is granted the right, without
any fee or cost, to use, copy, modify, enhance and distribute this
software, and any derivative works thereof, and its supporting
documentation for research purposes only, provided that this entire
notice appears in all copies of the software, derivative works and
supporting documentation.  The names Caltech or JPL however, may not 
be used in any advertising or publicity to endorse or promote any
products or commercial entity unless specific written permission is
obtained from Caltech. The user also understands that Caltech/JPL is
not obligated to provide the user with any support, consulting,
training or assistance of any kind with regard to the use, operation
and performance of this software nor to provide the user with any
updates, revisions, new versions or "bug fixes".

THIS SOFTWARE IS PROVIDED "AS IS" AND ANY EXPRESS OR IMPLIED
WARRANTIES, INCLUDING BUT NOT LIMITED TO, THE IMPLIED WARRANTIES OF
MERCHANTIBILITY AND FITNESS FOR A PARTICULAR PURPOSE ARE DISCLAIMED.
IN NO EVENT SHALL CALTECH BE LIABLE FOR ANY SPECIAL, INDIRECT OR
CONSEQUENTIAL DAMAGES OR ANY DAMAGES WHATSOEVER RESULTING FROM LOSS OF
USE, DATA OR PROFITS, WHETHER IN AN ACTION OF CONTRACT, NEGLIGENCE OR
OTHER TORTUOUS ACTION, ARISING OUT OF OR IN CONNECTION WITH THE
ACCESS, USE OF PERFORMANCE OF THIS SOFTWARE.
\end{verbatim}

\end{document}