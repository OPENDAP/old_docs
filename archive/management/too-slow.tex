% Record of an exchange between james and peter, 9/99

% $Id$

\documentclass[10pt]{article}
\usepackage{hyperlatex}
\T\usepackage{Vmargin}
\W\usepackage{sequential}

\begin{document}

\htmltitle{DODS Progress}
\htmladdress{tomss@ids.net}
\htmldirectory{too-slow}
\htmlname{ts}

\W\newcommand{\textsf}{\textit}
\T\setlength{\parindent}{0pt}
\T\setlength{\parskip}{\medskipamount}
\T\renewcommand{\rmdefault}{ptm}
\T\renewcommand{\sfdefault}{phv}
\T\renewcommand{\ttdefault}{cmtt}
\T\setmarginsrb{1in}{1in}{1in}{1.5in}{0pt}{0mm}{0pt}{0mm}

\title{DODS Progress}
\author{}
\date{as of \today}

\maketitle

\T\tableofcontents

\section{PROJECT STATUS (Where are we?)}

Since the end of the last developers' meeting:

\begin{itemize}
\item Minor Matlab GUI improvements, including about 80 or so datasets
  in the ML GUI (50? new since last meeting), one half of which are
  GLOBEC and about a third large area coverage;
\item DODS has been incorporated into Ferret and been released;
\item New numeric datatypes have been added to the core\footnote{This
    work is better described as fully supporting the netCDF/ferret
    users.};
\item Many bug fixes and minor feature advancements;
\item More datasets served, though not all accessible from Matlab GUI.
\end{itemize}

In progress:

\begin{itemize}
  \item Linked DODS to GrADS and are working toward its release;
  \item IDL command line client;
  \item IDL GUI;
  \item Port to Windows;
  \item WWW Interface (IFH);
  \item Improvements in the release process;
  \item The catalog server;
  \item Core implementation in Java;
  \item Translation of sequences to array;
  \item Definition of  DODS metadata standard.
\end{itemize}


\section{DIAGNOSIS (Why are we where we are?)}

While there are several reasons to feel pleased at the progress of the
DODS project, there are also several reasons for concern.  Even though
we are $1/2$ way through the CAN, done with the NOAA funding and $3/4$
of the way through our NSF funding we do not have a significant user
base.

Some smaller problems also continue to plague the project.  For
example, the release process is not worked out well, and the project
has been unable to assimilate appropriate new technology as it has
become available and popular (e.g. XML and CORBA).

\subsection{Too few clients}

Right now we have only one client that can access most DODS data and
that in a limited way.  This is the Matlab GUI, and it cannot access
n-dimensional data, and, as we've seen, adding datasets to it is a
tedious and labor-intensive process.  The situation is analogous to
building the Web so that Netscape has to install every web server
\emph{and} modify their browser for every server they add.

The other clients have difficulties as well.  Without implementation
of translation, the Ferret and GrADS clients can't see data served as
sequences.  The IFH works, but does so partly by limiting the scope of
its ambition.

The \emph{only} way we can get out from under this is to refocus the
client development effort to emphasize other clients besides the
Matlab GUI.

\W\textit{
\T\begin{center}
\T\fbox{\begin{minipage}{5in}
\textsf{\textbf{Note:} In order to make this into a white paper,
  expressing a single point of view, the points in this section need
  to be clarified and agreed upon.  For example, what does ``emphasize
  other clients'' mean?
  \begin{itemize}
  \item Create simple IDL and Matlab clients to emulate the IFH?
    (With which development resources?  That is, who does it, and what
    don't they do at the same time?)
  \item Making the Matlab GUI getxxx functions easier to use by
    themselves without the graphics overhead of the rest of the GUI?
  \item Intense focus on the Matlab GUI to achieve a stable state, and
    then suspend work on it while other clients are developed?
  \item Abandon the development of the Matlab GUI?
  \item Something else?
  \end{itemize}}
\T\end{minipage}}
\T\end{center}
\W}

\subsection{Not enough datasets}

There seem to be two reasons for this:

\begin{enumerate}
\item Servers aren't getting installed fast enough, and 
\item It takes too long to write the \texttt{getxxx} functions and
  archive \texttt{.m} files necessary for the Matlab GUI to use these
  datasets.
\end{enumerate}

The second problem may be the more important, since it has consumed by
far the most time.  Adding data servers is only half of the problem
since right now the project only supports one client, and it can't
deal with datasets without the getxxx function and archive .m files.
In other words, in the absence of clients that can deal with data
directly from a server, installing such a server is \emph{a completely
  useless gesture} unless one also writes the necessary Matlab
scripts.  

There are a number of major data suppliers who have expressed an
interest in serving their data via DODS. This is exactly what we want
and need. Although we would prefer that they do most of the adding, we
have to help. But we won't have the resources for several months. The
extraordinary list of potential suppliers is:

\begin{enumerate}
  \item IFREMER
  \item NODC
  \item GMU+
  \item Goddard DAAC
  \item U Hawaii
  \item JPL
    \begin{enumerate}
      \item Zlotnicki
      \item Liu
    \end{enumerate}
  \item Rutgers
  \item WHOI
  \item Townsend's group
  \item AGU
\end{enumerate}

Nonetheless, with 2.5 full time people working on the problem since
the last meeting, we have only been able to add 80+ datasets.  That
translates to about one dataset per week per person, or another way of
looking at it, it costs on the order of \$2k to add a dataset to DODS,
and this is with someone who is familiar with the system. Admittedly a
lot of the work put in by the project has been to fix up other
people's datasets, and to deal with problems introduced by datasets
with which we are not familiar or bugs in the server software.

\section{PRESCRIPTION (Where do we go from here?)}

\W\textit{
\T\begin{center}
\T\fbox{\begin{minipage}{5in}
\textsf{\textbf{Note:} Again, before this section can be completed,
  agreement must be reached on the earlier note.  The tasks to be
  completed depend on a decision there.}
\T\end{minipage}}
\T\end{center}
\W}
\subsection{Tasks}

Software development tasks:

\begin{itemize}
\item Complete translation

\item Complete SQL access

\item Complete PC port

\item Incorporate CS into IFH?

\end{itemize}

Data incorporation tasks:

?

\subsection{Organization}

Some of what the project needs is a clearer definition of people's
responsibilities.  Part of doing might be to separate data team from
software development.  There are, of course, overlaps (Nathan's major
task, for example, is software driven by the needs of a data
provider.)

Within software development, a further clear division between core and
UI development should be considered.

\end{document}











%\section{Where are we?}

%It seems that we are stalled; we are moving forward but very slowly,
%at least on the surface.  My impression is that since the end of the
%last Developer's Meeting all that we really have to show is:

%\begin{itemize}
%  \item Minor Matlab GUI improvements;
%  \item About 80 or so datasets in the ML GUI (50? new since last
%    meeting), one half of which are GLOBEC and about a third large
%    area coverage;
%  \item DODS has been incorporated into Ferret and been released and I
%    think that it is seeing users;
%  \item New numeric datatypes have been added to the
%    core\footnote{This work is better described as fully supporting
%      the netCDF/ferret users.};
%  \item Many bug fixes and minor feature advancements;
%  \item More datasets served, though not all accessible from Matlab GUI.
%\end{itemize}

%  These are the visible things that have been added and
%they are good! In progress:

%\begin{itemize}
%  \item Linked DODS to GrADS and are working toward its release;
%  \item IDL command line client;
%  \item IDL GUI;
%  \item WWW Interface (IFH);
%  \item The release process;
%  \item The catalog server.

%\end{itemize}

%All of these things sort of work, but they are not things that we are
%ready to flash up in front of the public with the exception of GrADS.
%In addition to these things numerous little problems have been
%addressed, some only partially: we have changed the way we handle
%releases, made progress on the catalog services, etc.
 
%In other words, we have made progress on a lot of things that are not
%very visible, but we have not made a lot of progress in the major
%ones. Even though we are $1/2$ way through the CAN, done with the NOAA
%funding and $3/4$ of the way through our NSF funding we do not have a
%significant user base.

%The list of partial progress items tells an interesting story.
%Of the six items, four are user interfaces; that is our project's
%real weakness (and the Achilles heel of all distributed systems).

%The catalog server relates to our attempts to make datasets easier to
%add to the system.  Actually they will become a little harder to add
%(but only if you want them to be compliant with the DODS data
%standard-in-progress), since the overhead involved in maintaining the
%catalog and the metadata is significant.  However, the hope is that
%the datasets in the system will be readily usable by more clients than
%is currently the case.  Basically, we're putting more of a burden on
%each server installation so that the clients (\emph{all} of them) can
%do more.


%The release process problems are noise in the sense that they say
%nothing special about DODS since all projects have these problems.
%That does not mean we can ignore them, just that we can do what
%everybody else does in this respect (which we are) and assume it
%will get better over time.  On the other hand, recent breakdowns in
%the release process have produced unreasonably long delays in making
%bug fixes available.

%On the bright side, DODS is receiving a \emph{lot} of attention; there
%seems to be general agreement that this is the way to go and that we
%are far along the path.\footnote{Part of this perception is due to the
%  fact that no one really knows how small our user base is.}

%\subsection{Datasets to add}

%I am particularly concerned because I don't see an
%end in sight to the software fixes/additions that will allow us to add
%datasets quickly.  The idea of making these datasets COARDS+ compliant
%is so that they can be added to the system quickly.  We already have
%ALL of the metadata needed, it's just a question of putting it in the
%right form.

%In the meantime we have a number of major suppliers
%who have expressed an interest in serving their data via DODS. This is
%exactly what we want and need. Although we would prefer that they do
%the adding, I think that we have to help. But we don't have the
%resources for several months. The potential suppliers are:

%\begin{enumerate}
%\item IFREMER
%\item NODC (they may be willing to install a server)
%\item GMU+ (ready to go I think)
%\item Goddard DAAC (need a little push but are there)
%\item U Hawaii (You've seen my e-mail on this one.)
%\item JPL
%  \begin{enumerate}
%  \item Zlotnicki (I think that all we need to do is follow 
%     up here)
%   \item Liu (Same as a; he'll go if we pursue him I think.)
%  \end{enumerate}
%\item Rutgers (They have installed servers, we just haven't 
%   coordinated with them.)
%\item WHOI (same as 6.1)
%\item Townsend's group (they will install servers when we
%   contact them.)
%\item AGU (ready to go I think)
%\end{enumerate}

%I am sure that there are more, but this gives an idea of 
%where we are. This is quite an amazing list. Think of where
%we would be if we could get servers at all of these sites.
%Most of these sites are ready to go, but they will not do 
%anything until we give them a push. 

%\section{Why?}

%Why do we have so few users with so many potential users? My take on
%this is that: (1) we do not have enough datasets accessible via DODS
%and (2) we do not have enough clients. Let's examine each of these:

%\subsection{Not enough datasets}

%Why don't we have enough datasets? After all, we have an
%extraordinary list of potential providers, we have had 
%2.5 full time people working on the problem since the last
%meeting and yet we have only been able to add 80+ datasets. 
%That translates to about one dataset per week per person, or 
%another way of looking at it, it costs on the order of \$2k
%to add a dataset to DODS, and this is with someone who is
%familiar with the system. Admittedly a lot of the work put
%in by Paul, Kwok-Lin and Ruth has been to fix up other peoples 
%datasets, learn to deal with problems introduced by datasets
%with which we are not familiar or problems associated with 
%bugs in the server software. 

%Another part of the problem has been the 
%time required to add the dataset to the GUI. In fact this
%has consumed by far the most time. I'll get back to this 
%below.


%\subsection{Not enough clients} 

%Right now we have only one client that can access most DODS data and
%that in a limited way.  This is the Matlab GUI, and it cannot access
%n-dimensional data, and, as we've seen, adding datasets to it is a
%tedious and labor-intensive process.  It really is a problem with the
%ML GUI and having that drive our project (which it does to a great
%extent). The situation is analogous to building the Web so that
%Netscape has to install every web server \emph{and} modify their
%browser for every server they add.

%The other clients have difficulties as well.  Without implementation
%of translation, the Ferret client can't see data serverd as sequences.
%The status of the GrADS and IDL clients are, at this writing,
%uncertain.

%The \emph{only} way we can get out from under this is to change the
%focus of the project away from the ML GUI by adding more clients.
%That's why it is important to emphasize the NetCDF library and the
%IFH.  We should expand and improve the IFH, as well as introduce
%simpler Matlab and IDL clients.


%\subsection{Several unfinished tasks}

%Another important problem is that we don't seem to finish the
%many tasks that we have in a timely fashion. We have a 
%large list of tasks partly done:

%\begin{itemize}
%\item Internal translation from sequence to array
%\item IDL client
%\item IFH
%\item IDL GUI
%\item addition of PC access
%\item definition of a DODS metadata standard
%\end{itemize}

%\section{What to do about it}

%James has suggested that we reallocate our resources. If I understand
%properly, he was/is thinking of making a more formal division between
%the population effort and software development. It seems that we
%already have that to a large extent though and we are still
%struggling. I have an alternative to suggest.

%First, let me lay out the goals. They are:

%\begin{enumerate}
%\item To work with the groups that I have listed above to add
%   DODS servers; and 
%\item To add more clients, and/or make our clients more 
%   functional.
%\end{enumerate}

%I think that we should freeze all software development that does not
%move us toward these two objectives. In particular, I think that we
%should back off of the catalog services until we have cleaned up some
%of the stuff that is hanging around.  I also think that we have to be
%more specific about who does what, especially with regard to those
%working on population.

%\textsf{\paragraph{James:} Installing servers should be separate
%  from adding datasets to the ML GUI. That's why we need to broaden
%  our client base. My idea behind stressing the metadata stuff
%  (catalog services) was to shorten the time required to add datasets
%  to the ML GUI. Once we're free from that as a constraint, we're free
%  in a lot of ways.
%}

%\subsection{Proposed Tasks}

%Here are the proposed highest priority tasks:

%\begin{enumerate}
  
%\item Reza to continue on sequence to array translation. We need a
%  firm completion date for this. Reza will also continue to work on
%  bug fixes for FreeForm and NetCDF.

%\item Nathan to continue the work that he is doing on SQL access, if
%  possible to move the completion date forward.
  
%\item Rob to finish the PC job; i.e., to move this to where we can use
%  it. Then he should move on to the IDL client and GUI problems.
%  Perhaps find someone else at JPL to add servers.\footnote{A problem
%    at JPL is that they have not integrated DODS into the
%    PO-DAAC. The servers seem to sit on Rob's machine. That doesn't
%    make sense. They should not sit on a developer's machine but
%    on one of the PO-DAAC servers. This concerns me for two reasons:
%    (1) every time Rob moves (sneezes) the servers go down, and more
%    importantly (2) it shows a lack of commitment on the PO-DAAC's
%    part.}
  
%\item Kwok-Lin to working on the regional data sets, focussing on the
%  necessary metadata. In the meantime she can continue work on the
%  long time series data sets.  She could also put together a data set
%  tester for data sets in the ML GUI. The idea is simple. Using the
%  GUI she begins by making a simple request of a small amount of data
%  from each data set. She then saves this session(s).  Then she writes
%  a Matlab script that simply issues each of the URL's and does a
%  compare on what comes back and what she got earlier.  This would
%  serve two purposes. First, it would alert us to problems with
%  specific data sets. Second, it would allow us to build a data base
%  of data set availability. Both would be VERY valuable.

  
%\item Ruth to continue with the data sets that she is currently adding
%  to the system.


%\item Deirdre come up with an outline of what the new GUI would
%look like. We have a meeting with James, Glenn, Deirdre, me
%and a Matlab consultant whom we hire for this 2 day meeting.
%The idea of the meeting is to hammer out the specs for this
%interface (which includes the geographic and nongeographic
%interface). These specs to be completed by 1 November. 


%\item Ethan to continue getting the current release out. Then he
%  should focus on the metrics. The status outlined above is based on
%  NO information re the real number of users.  We need this
%  information.  This is more important than the federation metrics.


%\item James continue to guide the technical aspects of the
%project, but also to complete the IFH and the Matlab
%n-dimensional client. Time frame? IFH ~ 2 weeks; Matlab
%client ~ 3 weeks? James also to write a GIF generator for
%DODS array data. (2 weeks?) I think that this could add
%a very large number of users to the system, basically
%everyone who uses a web browser to look at one of our
%SST images becomes a DODS users! The same is true of
%other data sets. We can easily put together simple
%specialized interfaces for some of our data sets such
%as our SST fields. James also to work with Ethan
%on making it easier for others to build DODS on their
%machines.


%\end{enumerate}

%\subsection{A New Proposal}

%To address some of the functional difficulties with using the Matlab
%GUI, I suggest: 

%\begin{enumerate}
  
%\item adding a FreeForm server at the GCMD that would provide a list
%  of data set titles and URLs in response to a query to the GCMD. The
%  query would be based on a free text search of all DODS data sets. If
%  a variable was specified and the variables of the data set are
%  unknown, then it is returned with an unknown flag;
  
%\item adding a non-geographic search capability to the Matlab GUI. The
%  idea here is that the user obtains a list of possible data sets from
%  the GCMD and here with a GUI that mimics the IFH builds a URL within
%  Matlab and requests data;
  
%\item n-d capability be added to the GUI.
%\end{enumerate}

%I am sure that there are problems with what I have suggested, but we
%have to start the discussion somewhere and I am convinced that we need
%the discussion. James has been emphasizing the addition of new tools
%to make things easier in the future, but my impression is that too
%much effort is going into that now. I know that I have crammed a lot
%of stuff here, but these are the things that I think need to be done
%for us to take off. We are already working in this general direction,
%but I don't see the unified picture. This is an attempt to put it
%together.

%The bottom line is that we need:

%\begin{enumerate}
%\item Easy access to a list of ALL DODS URLs;
%\item A way to access any of these datasets in Matlab (later
%in IDL);
%\item Ferret completely functional;
%\item A basic web GUI that will get at any DODS dataset;
%\item GrADS completely functional;
%\item Matlab to handle n-dimensional datasets;
%\item An IDL GUI; and
%\item MANY MORE DATASETS.
%\end{enumerate}

%and eventually a geographic Web GUI.







