
% Notes on the release process.
%

% Add a software roadmap. What's in the various directories, what
% it does. This should make it easy to find documentation files so that we can
% make sure to update them when code changes. Maybe what we really need is a
% list of documentation and a check list of stuff to go over when changes are
% made.

\documentclass{dods-paper} 

\rcsInfo $Id$

\T\setcounter{secnumdepth}{4}     % Put \paragraph in TOC
\T\setcounter{tocdepth}{4}

\W\newcommand{\*}{*}

\title{OPeNDAP Software Release Process}
\author{James Gallagher\thanks{jgallagher@opendap.org} 
  \and Dan Holloway\thanks{dholloway@opendap.org}
  \and Ethan Davis\thanks{edavis@ucar.edu}}
\date{\rcsInfoDate, Revision: \rcsInfoRevision}
\htmladdress{James Gallagher <jgallagher@gso.uri.edu>, 
  \rcsInfoDate, Revision: \rcsInfoRevision}
\htmldirectory{rel-html}
\htmlname{rel}
\figpath{.}

\begin{document}

\maketitle

%% \begin{abstract}
%%   This describes the release process for OPeNDAP software. It includes
%%   checklists for generating successive Release Candidates and for common CVS
%%   operations. It also contains an explanation of the meaning of version
%%   numbers.
%% \end{abstract}

% \T\pagebreak

\T\tableofcontents

\section{Introduction}

Deciding when to release software is hard. The choices depend on the
long-term goals of the group doing the work and which features they think
should be bundled together. However, this paper is not about choosing which
features go in which release. This paper describes a process that starts once
the features have been determined and are implemented in the software. Once
we decide to release, we will follow the process described here.

\section{Release Checklists}

This section contains some checklists that we have developed over time to
help make the process of releasing software more predictable. Collecting these
lists in one place should help co\"ordinate our group and also make it easier
to fix broken parts of, or add missing parts to, the process. 

All the OPeNDAP software is held in one CVS module. Even so, each discrete
piece of software (a server, the library of common code, {\it et c.} is in
its own directory. Each source directory has its own version number. With the
exception of the server dispatch software (which is in \texttt{DODS/etc}) all
of the software is held in directories under \texttt{DODS/src} and each of
those directories contains a separate deliverable.

In practice, we release the OPeNDAP software mostly {\it en masse}. However,
if a fix for a particular component really needs to be released quickly, we
can increment just that component's version number and release it alone. That
is, suppose the current release is version 2.1 and we add an important fix to
\texttt{asciival} which is at version 2.1.6. We can bump the version number of
\texttt{asciival} to 2.1.7 and put that in the 2.1 release. We don't need to
release all the other components.

When we decide to there are enough new features and/or fixes for a new
release, the version number of the release as a whole is increased.
Appendix~\ref{version-numbers} has information on version numbering.

\subsection{Making a new release}

Making a new software release can be roughly divided into two parts. In the
first part we start making Release Candidates (essentially beta releases). As
problems are found in these, we fix those problems and produce a successive
candidate. This iterative process continues until the current Release
Candidate has no major problems and can be made a formal release.

Here are the steps used to iterate over making release candidates toward
making a formal release:

\begin{enumerate}

\item Create a release branch in CVS
\item Start documentation efforts:
  \begin{enumerate}
  \item Gather information needed for documentation (web pages,
    README/INSTALL docs, and formal manuals):
    \begin{itemize}
    \item the new features and major bug fixes (this information
      should be part of the ChangeLog files).
    \item the system requirements.
    \item compatibility with previous versions.
    \item the list of the systems on which the code has
      been built (include compiler version information).
    \end{itemize}
    This list will be the rough draft of the release notes.
  \item Create web pages for this release.
  \item Create release notes.  (Take the list made above, and check it
    into the documentation DODS CVS tree at DODS-doc/notes.)  The
    release notes will eventually consist of a list of the important
    changes, followed by a list of explanations for each change.
    Here's an example from release 3.4:
    \begin{vcode}{ix}
      Code changes
      1.  Using libcurl instead of libwww.
      2.  URL dereferencing done with curl instead of Perl.
      ...

      CODE CHANGES
      1. The DODS core now links to libcurl instead of libwww for WWW
      functions.  The libcurl library is smaller than libwww, is being
      actively developed, and is MT-safe.

      2. The server dispatch scripts no longer use the Perl HTTP, MIME and
      LWP classes.  Instead, URL dereferencing is done with curl, a
      freestanding binary that is part of the libcurl build.
      ...
    \end{vcode}
  \item Start the process of adding that information to the Guides.
  \end{enumerate}
  It's important to get started on these tasks early because they are
  complicated and, unlike a build or a test suite, there's no automated way to
  determine when they are `ready.'
\item Iterate through the Release Candidates until no major bugs remain. See
  Section~\ref{release-candidate} for the steps involved in making individual
  release candidates.
\item Evaluate the software (See Section~\ref{full-release}) to determine if
  there are issues with the source not yet addressed by the process of
  release candidate preparation.
\item Mark the branch as a formal release
  (e.g., \texttt{cvs tag release-3-4-3FCS}). NB: I prefer to do this
  \emph{after} making the source files. If you use this ordering of
  the steps and find a problem, make sure to re-tag the branch (see
  \texttt{cvs tag -h} for information about moving (or forcing) a tag.
\item Make source and binary distribution files. See Section~\ref{source-files}
\item Place source and binary on \texttt{ftp} site
\item Update web pages for formal release

\end{enumerate}

\subsubsection{Making release candidates}
\label{release-candidate}

\begin{enumerate}

\item Prepare the code to be a Release Candidate (see
  Section~\ref{cvs-prepare}). 
\item Run nightly builds until the tests are clean.
\item Update all \texttt{version.h} files.
\item Tag CVS branch as a release candidate (e.g., \texttt{cvs tag
  release-3-4RC1}).
\item Make source tar files (see Section~\ref{source-files})
\item Put source distribution files on the \texttt{ftp} site
\item Check that web site is up-to-date (i.e., so that people can get the
  release candidate).
\item Announce release candidate source code.
\item Tag and merge the branch (see Section~\ref{cvs-tag-merge}).
\item \[Optional\] Make binary builds; release.
\item If there are major bugs, fix them and redo this list.
\item If there are no major bugs, we are done making release candidates

\end{enumerate}

Note that the Release Candidate process concerns the source code and binary
builds only. Once we've decided to make a formal release, There are other
things that must be done {\em before} it is ready to become a formal
release. See Section~\ref{full-release} for details on the steps
required before a formal release can be made.

\subsubsection{Before making a formal release}
\label{full-release}

Think about each of the following questions before moving on from release
candidates to a formal release:
\begin{itemize}
\item Do we have binaries for all the platforms we support?
\item Double check information needed for documentation:
  \begin{itemize}
  \item new features and major bug fixes.
  \item system requirements.
  \item compatibility with previous versions.
  \item the list of the systems on which the code has been built.
  \end{itemize}
\item Is the documentation sufficient? Check README and INSTALL files in the
  \texttt{src}, \texttt{etc} and \texttt{docs} directories and the NEWS file
  in the top-level directory. The text documentation files in \texttt{docs}
  are used when making binary distributions, the ones in the \texttt{src}
  directories are included with the source distributions and the ones in
  \texttt{etc} are included in both.
\item Do the formal manuals need to be updated? If so, this process should be
  started (however, in general the release should not wait on the manuals).
\item Is the web-site current?
\item Can we all answer questions about the installation and use of the
  software? Do we all understand the software's limitations?
\end{itemize}

\subsubsection{Making source and binary distribution files for a formal release}
\label{source-files}

\begin{enumerate}

\item Check out a clean copy of the release branch using CVS's
  \texttt{export} command (e.g., \texttt{cvs export -r release-3-4RC3 -d
    DODS-3.4 DODS}).
\item Remove any source/directories that we don't intend to include in this
  release. Since the trunk often contains more software than we are actually
  releasing, this is an important step. 
\item Verify that this software builds: \texttt{./configure},
  \texttt{make World}, \texttt{make tar}\\
  NB: It's often a good idea to see
  the build as it happens and to get a log. To do that, use \texttt{tee} as
  in: \texttt{make World 2>\&1 | tee World.log}. If you're using \texttt{csh}
  change \texttt{2>\&1 |} to \texttt{|\&}.
  \begin{itemize}
  \item Build binaries and run tests.
  \item Build binary distribution files (\texttt{make binary-tar})
  \item Remove the binaries: \texttt{make distclean}
  \item Add the version numbers to the directory names: \texttt{make
    update-version}
  \item Make the source code tar files: \texttt{make source-tar} (at the
    top-level) or \texttt{make tar} in a \texttt{DODS/src/{\em component}}
    directory.
  \item Test both the source and binary distributions!
    \begin{enumerate}
    \item Expand all the binary distribution files to make sure they
      hold software that works.
    \item Expand the source distribution files and verify they build.
    \end{enumerate}
  \end{itemize}

\end{enumerate}

\section{CVS Checklists}

Using CVS can be complicated. This section contains some short checklists for
common CVS operations.

\subsection{Prepare the code to be a Release Candidate}
\label{cvs-prepare}

\note{For the DAP software, some of these steps must be done in the
  \texttt{DODS\_ROOT} and \texttt{DODS\_ROOT/etc} directories in addition to
  \texttt{DODS\_ROOT/src/dap}. For the other software, follow these in just
  the \texttt{DODS/src/{\em component}} directory.}

\begin{enumerate}
\item Check that the \texttt{configure} script is up-to-date WRT
  \texttt{etc/aclocal.m4}. Use \texttt{make configure}.

\item Rebuild the dependencies using \texttt{make depend}.

\item Examine the directory for files that should have been removed or are
  leftover from development/testing and remove those by hand. In some cases
  files will need to be removed from CVS also (\texttt{cvs rm -f {\em
      files}}).

\item Run the tests. Do this starting from a clean set of sources
  (\texttt{make clean; make check}).

\item Update release number in \texttt{version.h}. Use the three part version
  numbers. Check this file in and add today's date as the log entry. This
  makes figuring out when a particular version was made simpler (although
  there are other ways). See Appendix~\ref{version-numbers} for information
  about version numbers.

\item Update the \texttt{ChangeLog}. Scan changed log files using
  \texttt{cvsdate ">{\em date of last version}"}.\footnote{\texttt{cvsdate} is
    short command that runs \texttt{cvs log} and parses the output. A copy
    should be in \texttt{DODS/etc}.}

\item Make sure that all files in the directory are checked into CVS. Use
  \texttt{cvs ci}. Use the \texttt{-l} flag with \texttt{cvs ci} in
  \texttt{DODS\_ROOT} to check in only that directory and avoid recurring
  through all the software.

\item Examine the INSTALL and README files. Are they correct? Anything new
  missing?

\item Look at the README/INSTALL files in DODS/etc {\em and} DODS/docs. Are
  they correct? The first set of README/INSTALL files are sent with both
  source and binaries, the second set are bundled into the binary
  distributions only.
\end{enumerate}

\subsection{Tag and merge the branch}
\label{cvs-tag-merge}

\begin{enumerate}
\item Go to the CVS directory that holds the release code and tag it with the
  same release number as is in the version.h file.

\item Merge back into the trunk. To merge the changes made in the branch
  software back to the trunk, do the following: 
  \begin{enumerate}
  \item Go to the CVS directory that holds the trunk software.
  \item Use \texttt{cvs update -j {\em release-branch-tag}} for the first
    merge
  \item Use \texttt{cvs update -j {\em previous-merge-tag} -j {\em
      release-branch-tag}} for subsequent merges\footnote{There's an argument you
    can pass to \texttt{cvs update} that will suppress expansion of
    \texttt{\$Id\$} and \texttt{\$Log\$} symbols. Using this will cut down on
    spurious conflicts (but you don't get those symbols expanded). You can
    resolve the conflicts by hand or use the \texttt{resolve\_conflicts.pl}
    script.}
  \end{enumerate}
  See Appendix~\ref{version-numbers} for an explanation of the relation
  between version numbers and CVS tags.

\item In the \lit{ChangeLog} file in the trunk software, record the cvs command
  used to do the merge and the arguments to \texttt{-j} that should be used
  for the next merge.

\item To test for conflicts use: 

  \begin{vcode}{ti}
    alias conflict='grep -l "<<<<<<" `find ! -type d ! -name *.cft`'
  \end{vcode}

  and run the results through the
  \texttt{resolve\_conflicts.pl} script. Some files get munged by this so
  examine the result.\footnote{\texttt{resolve\_conflicts.pl} keeps a copy
    of the original file in \texttt{{\em filename}.cft}.}
  
\item After merging each directory, add a note in ChangeLog and check in the
  code using \texttt{Merged with \texttt{release-<rev>}} as the log entry.

\item Save and check in the changes that result from the merge.

\item Try to compile the newly merged code. Resolve problems introduced by
  the merge.
\end{enumerate}

\T\appendix

\section{Version Numbers}
\label{version-numbers}

Version/revision numbers in DODS are formed from two or three numbers
separated by a dot. Here's how to interpret the version numbers:
\begin{quote}
  x.y.z
\end{quote}

where 
\begin{itemize}
\item x: This is the major version number. Two different major versions are
  almost certain to be incompatible. This is used to indicate a big
  change in the way the software works.

\item y: This is the minor version number. This indicates an important change
  in the software. Two different minor versions maybe incompatible in some
  ways but mostly work together.

  Each minor version has a corresponding tag in CVS {\em that is a branch}.
    That is, you can checkout/update using \texttt{-r release-x-y} and get
    the latest code for that \lit{major.minor} release number. Fixes for the
    release should get checked in on the appropriate branch.

\item z: This is the sub-minor version number. These are used to indicate
  that problems with a particular version of the software have been
  found and fixed. Different sub-minor versions should work together
  except that a problem fixed might cause slightly different behavior.
\end{itemize}

\section{Version Numbers and CVS Tags}

Each sub-minor version number corresponds to a real release of software.
Each has a tag that looks like \texttt{release-x-y-z}. However, this tag is
{\em not} a branch. This might seem confusing at first, but it makes
working with CVS simpler. Here's an example. Lets say we have made a
release of version 3.1.0 and some problems have been found. We can commit
changes to that release until we are happy with the fixes. At that time we
then tag {\em that point} on the 3-1 branch as 3-1-1 and make the source
code tar files for version 3.1.1. See Figure~\ref{fig:cvs-branching}.

\figureplace{CVS branching}{htbp}{fig:cvs-branching}{cvs-branch.eps,width=3in}%
	    {cvs-branch.gif}{}

Why make the non-branch tags? Because what we really need are place holders
for merging the changes back onto the trunk. See the Using CVS paper for a
discussion of CVS merging strategy. While we work on the 3.1 code, fixing
bugs and tweaking features, other completely new features are added to the
trunk code.

\section{Targets in the Makefiles}

Listed here are targets in the Makefiles that are used to help make releases.

\begin{description}
\item[\texttt{Makefile}] If the \lit{Makefile.in} has changed, rebuild the
  \lit{Makefile} either by running \lit{config.status} or \lit{configure}. In
  source code that has just been exported from CVS, you will need to run
  \lit{configure} to build the \lit{Makefile}.

\item[\texttt{depend}] (Re)build the dependencies. This target rebuilds the
  dependencies in the \lit{Makefile.in}. You should not have to use this
  target with code that's been exported from CVS; if you do, then it is a
  sign that something is wrong with the \lit{Makefile.in} that is in CVS.
  You'll need to go back to code that is in CVS, run make depend and then
  check in the resulting \lit{Makefile.in}.

\item[\texttt{check-version}] Check that the version number in the
  \lit{Makefile} matches the version number in the version.h file. If this
  returns an error, you need to go back to code in CVS, re-run
  \lit{configure} and then export the code.

\item[\texttt{update-version}] Read the current version from the
  \lit{version.h} file and append it to the CWD's name (i.e., \lit{jg-dods}
  becomes \lit{jg-dods-3.1.2}). Don't run this in the directories under CVS
  control (those that have the `CVS' subdirs in them). Only run this in an
  exported directory.

\item[\texttt{all}] Build all the main programs in the directory. 

\item[\texttt{check}] Run the tests for this this software. In some cases,
  you have to install the software (using the install target) before this
  does anything except return an obscure error. Bummer. This should have
  been run by the nightly build, so running it when making the source or
  binary tars is not completely necessary\ldots

\item[\texttt{install}] Copy the binaries from this directory into the
  \lit{bin}, \lit{lib}, \lit{etc} and \lit{include} directories.

\item[\texttt{binary-tar}] Build a tarball that contains the binaries built
  by `all' and installed by `install'. The binaries are tarred in the
  directories where they are installed, not where they are built.

\item[\texttt{clean}] Remove most generated files (i.e., executables, object
  files, core files, emacs backup files, ...).

\item[\texttt{distclean}] Remove all generated files except grammar files,
  the \lit{Makefile} and \lit{configure}.

\item[\texttt{tar}] Make a source code tarball. Run distclean before using
  this target to make the source tar balls. Note that in the top-level
  \lit{Makefile} this is called \lit{source-tar}.
\end{description}

If all goes well, the targets should be run in the following order after
running \lit{configure}: \texttt{check-version} (should return OK),
\texttt{update-version} (should set the directory name so that the source tar
balls have the correct names), \texttt{all} (builds the code),
\texttt{install} (installs it), \texttt{binary-tar} (creates the binary
distributions for this platform), \texttt{distclean} (gets rid of the same
before making the source tar balls), \texttt{tar} (makes the source code tar
balls).

\section{ChangeLog}

\begin{vcode}{it}
  $Log: release-process.tex,v $
  Revision 1.18  2004/01/09 20:32:51  jimg
  Fixed a latex-typo.

  Revision 1.17  2004/01/09 20:16:55  jimg
  Editing...

  Revision 1.16  2004/01/08 18:34:00  jimg
  Added info about NEWS files.

  Revision 1.15  2004/01/08 00:37:07  jimg
  Minor formatting changes.

  Revision 1.14  2004/01/06 23:01:47  jimg
  Added an item about testing the distribution files.

  Revision 1.13  2004/01/06 22:03:49  jimg
  I added some additional information to the check lists and tried to tie them
  together a bit more.

  Revision 1.12  2003/12/08 16:41:02  tom
  added information about release notes, fixed some typos.

  Revision 1.11  2003/12/04 22:42:08  jimg
  Fixed an error in the merge recipe!

  Revision 1.10  2003/09/24 22:28:18  edavis
  Added some details to the "Making release candidates" and the "Making
  source tar files" sections.

  Revision 1.9  2003/06/30 23:07:19  edavis
  Added examples of commands for several steps. Added some detail
  to the documentation requirements.

  Revision 1.8  2003/06/26 15:26:41  tom
  fixed \figureplace usage (added \figpath)

  Revision 1.7  2003/06/24 23:25:57  jimg
  Added figure; changed to \figureplace. Latex works but the figure is big,
  hyperlatex almost works. The web pages have a snazzy new look about them...

  Revision 1.6  2003/06/24 22:39:09  tom
  put it into dods-paper class. other little adjustments

  Revision 1.5  2003/06/24 21:56:05  jimg
  I added some hypelatex stuff, but I'm really pretty clueless about it all.

  Revision 1.4  2003/06/20 18:02:35  jimg
  Changes. I'm done with this for a while even though it could use more work...

  Revision 1.3  2003/06/19 01:47:46  jimg
  Added hyperlatex (but not tested) and a short abstract.

  Revision 1.2  2003/06/19 01:40:43  jimg
  First pass at formatting.

  Revision 1.1  2003/06/19 00:55:25  jimg
  Initial version

\end{vcode}

\end{document}
