
% This file contains the abstract for the TOS 4/95 DODS demonstration
% poster. 
%
% jhrg 4/10/95

% $Id$ 

\documentstyle[12pt,margins]{article}
\begin{document}

\begin{abstract}

The Distributed Oceanographic Data System (DODS) is being developed to
provide a high degree of direct connectivity between the science applications
used by research oceanographers and on-line science data.  The design of DODS
integrates the procedural interface of widely used Application Programmer
Interfaces (APIs) such as NetCDF, with the network interface of the World
Wide Web, WWW.  DODS design extends the functional capabilities of data
access APIs beyond local file systems to include access to remote data.
Remote data files, named using WWW universal reference locators, are accessed
by the same API procedure calls used to access local data files.  DODS' use
of WWW protocols expands the domain of WWW access to include scientific
processing and analysis applications and extends the functionality of WWW to
include procedure based access.  A formally defined DODS data access protocol
specifies the services provided by DODS/WWW servers and is used for
transferring information between DODS/API client and DODS/WWW servers.  The
data access protocol provides a system independent method for accessing
different distributed data resources and makes possible the dynamic
translation from one API to another within DODS.  From a researcher's
perspective data resources available through DODS are provided in the format
required by his or her science application API.  We will demonstrate the main
components of the DODS client-server architecture accessing on-line data from
DODS servers supporting two different APIs (NetCDF and JGOFS) referencing two
different data models (relational and gridded).

\end{abstract}

\end{document}
