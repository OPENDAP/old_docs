
% Paper for submission to TOS. This could also be the beginings of some user
% documentation or a `welcome' page in our HTML doc tree...
%
% $Id$


\documentstyle[12pt,psfig,margins,chicago]{article}

\psfigurepath{tos1-figs}
\topmargin -8ex
\footskip 4ex
\headheight 8ex
\footheight -4ex
\textwidth 6.25in
\textheight 8.25in
\oddsidemargin 0em
%\parskip 1ex
\def \dods{Distributed Oceanographic Data System }
\def \dodsns{Distributed Oceanographic Data System}
\def \1em{\hskip 1em}
\def \2em{\hskip 2em}
\def \3em{\hskip 3em}
\def \4em{\hskip 4em}
\def \dap{data access protocol }
\def \dapns{data access protocol}
%\pagestyle{empty}

\begin{document}
\bibliographystyle{chicago}
\Large
\centerline {\bf DODS: Providing direct access }
\centerline {\bf to distributed research data resources.}
\normalsize

\renewcommand{\baselinestretch}{1.5}

%\begin{htmlonly}
%\section{Introduction}
%\end{htmlonly}
%\begin{latexonly}
\large
\bigskip
\noindent
{\bf Introduction}
\medskip
\normalsize 
%\end{latexonly}
\renewcommand{\baselinestretch}{1.5}

\noindent
This paper describes the design of a distributed data system for
oceanographic data. The system, currently under development will enable
research oceanographers to interactively access and directly import
distributed, on-line science data using their own personal research analysis
and processing applications. The system is being developed jointly by
researchers and staff at the University of Rhode Island Graduate School of
Oceanography and the Massachusetts Institute of Technology.

The realization that such a system was needed first became evident in 1992 at
The Oceanography Society meeting held in Seattle.  At that meeting a
significant number of on-line data systems were demonstrated that illustrated
the potential of providing access to research data over the Internet.
However, no pair of the demonstrated systems could communicate easily.  It
was clear that while numerous useful oceanographic data systems were being
developed there had been no coordination to permit interoperability between
them.

In October 1993, the issue of developing an interoperable, distributed system
for accessing oceanographic research data was explored in detail at a
workshop organized with the help of The Oceanography Society and held at the
W. Alton Jones Campus of the University of Rhode Island.  The workshop was
made possible with funds from NASA and NOAA.  In order to obtain a diverse
and representative perspective of different groups' needs regarding the
application and management of oceanographic data, oceanographic researchers,
data system developers and data archive managers were invited.

The workshop had three primary objectives: 1) to develop a vision of a
distributed data system for oceanographic research data, 2) to generate a set
of requirements for such a system and 3) to identify system architectures
that would meet these requirements.  A report, titled ``Report on the First
Workshop for the Distributed Oceanographic Data System''
\cite{cornillon:dodsws1} of the workshop has been published and can be
accessed via the Internet\footnote{\tt http://lake.mit.edu/dods.html}.  At
the end of the workshop, the Distributed Oceanography Data System, DODS,
development team was formed.  The development team was tasked with the
responsibility of generating a detailed system design and building the core
components of DODS.

In this article I will summarizes the motivation for embarking on the
development of the \dods in the first place and provide an overview of the
design and implementation strategy.

%\begin{htmlonly}
%\section{Motivation}
%\end{htmlonly}
%\begin{latexonly}
\large
\bigskip
\noindent
{\bf Motivation}
\medskip
\normalsize 
%\end{latexonly}

\noindent The motivation for the development of the \dods was spurred by the
rapid increase in the number of data sets available on the Internet coupled
with the lack of coordination between the systems that were developed to
access and/or distribute these data.  Particularly within the oceanographic
community, researchers were making large data sets available on-line (e.g;
University of Hawaii's AVHRR ftp site, University of Colorado's SANDDUNES,
Scripps Institute of Oceanography SSABLE, University of Rhode Island
XBrowse).  Within the federal government, agencies were beginning to explore
development of distributed data systems within the context of Global Change
(e.g; NASA's Mission to Planet Earth, NOAA's Climate and Global Change) and
federal archives were looking to providing on-line access to archives.
However, it appeared from the outside that these federal efforts were being
undertaken with little or no actual development coordination.  As for the
systems being developed by individual researchers there had been no
development coordination between them and as a result they were incompatible.
Even though, as in the case of the systems sited above, many were serving the
same data sets (i.e.; NOAA satellite AVHRR data).

Another issue, of concern to oceanographic researchers, was the feeling that
the the data resources of individual researchers were being overlooked within
the scope of the federal data management programs.  This was considered a
serious oversight.  Why?  Because, researchers produce data as well as use
it.  Researchers tend to invest a considerable amount of effort `cleaning
up', calibrating and processing raw data for use in their research.  The time
and knowledge they put into massaging their data increases its scientific
value significantly, making it more valuable to the research community as a
whole.  The federal data systems being developed had not identified the need
to integrate access to individual PI's data sets within their systems.  Thus
high quality research data from researchers would not be readily available
through these systems.  Eventually, the raw version of a PI's data would end
up in one of the national data centers where it would be available to the
federal data systems, however, the valuable undocumented knowledge that the
PI initially invested in preparing the data for his or her research would be
lost to the system and the community.

%\begin{htmlonly}
%\section{DODS: An extension to existing interfaces}
%\end{htmlonly}
%\begin{latexonly}
\large
\bigskip
\noindent
{\bf DODS: An extension to existing interfaces}
\medskip
\normalsize 
%\end{latexonly}

\noindent The Distributed Oceanographic Data System (DODS) is not a single
stand alone data management or archive system.  It is not a system for
creating data base schema and inventories nor for managing archival
data sets.  Rather, DODS is a method for directly accessing data on the
Internet.  It is a {\em data access protocol} that provides both a common
functional interface to data systems with on-line data and a well
defined data model with which to represent the data on those systems.  It is
designed to be integrated {\em with\/} already existing user applications and
resource management systems, not to replace them.  DODS extends the
operational capabilities of existing systems in two important ways; first, it
transforms them to a distributed client-server system and second, it provides
a integrated view of their resources which is system independent.

In order to implement these extended operational capabilities DODS provides
both a defined data delivery model and software tools.  The data delivery
model provides the necessary infrastructure and semantics for application
clients and data servers to interoperate with one another.  The tools enable
data systems developers to create servers that generate a canonical
representation of their data resources. In many cases without modification of
the storage form of data sets.  In addition the tools, through the use of
customized translators, will allow data users to utilize their data analysis
and processing applications as interpreters of the canonical data
representation provided by the servers. DODS then is envisioned as a
cooperative network of autonomous systems, both large and small, which
interoperate via the DODS \dap and supported interfaces.

In order to better understand what the tangible objectives of DODS are, it is
useful to pose a research scenario that will help to illustrate the
functional concepts that are being implemented within DODS.

A physical oceanographer at the University of Rhode Island has put together a
program of research that involves the tracking of isopycnal floats within the
Gulf Stream and the Western North Atlantic.  The parameters that he deals
with are float position (determined from sound source telemetry), time,
temperature and pressure.  In addition he often utilizes CTD, xbt, current
meter and satellite data in his analysis.  The researcher utilizes a suite of
processing and analysis applications that he has customized specifically for
his research data and he uses these applications on an almost daily basis as
new float data comes in.

Currently he is involved in a program studying the dynamics of the North
Atlantic Current in the region east of the Canadian Maritimes.  Because he
had not worked in this area before he was interested in acquiring historical
data in order to gain some long term perspective of the major dynamic
processes occuring in the region and to help determine how best to organize
and execute a field study with his isopycnal floats.  WOCE CMDAC current
meter data from Oregon State University, GTSPP temperature and salinity data
at NODC and URI's satellite SST data are available on-line and are data sets
that might contain information of interest.  The researcher \underline{will}
utilize his own processing and analysis application for looking at any
historical data that is relevant.

The researcher is also interested in exchanging data with other researchers
who are collaborating in the North Atlantic Current Study.  These data sets
are the most relevant to his research and are only available through
proprietary arrangements with his colleagues.  Again the researcher will want
to be able to use his own software applications for conducting his analysis.

\medskip \centerline{\bf The DODS Implementation Scenario}

DODS client libraries are linked to the researcher processing and analysis
applications, in effect making the applications procedure calls, such as {\bf
  open, read, plot,} etc., surrogates for DODS supported operations.  DODS
server libraries are installed at the systems that provide on-line access to
data as well as at the systems of the research collaborators.  The researcher
at URI is able to interactively analyzes the historical data from the three
different on-line data systems by providing a WWW universal reference locator
with embedded search constraints as the argument to one of his application
procedure calls (i.e.; {\bf read}).  The underlying DODS applications
transparently submits a request from the client application to the
appropriate DODS data server.  The DODS data server takes advantage of the
data system's querying capabilities to locate the data of interest and
translates that data into a DODS specified canonical form.  The DODS server
then transmits it to the client.  The DODS client application translates the
canonical representation into the format that the initiating application
procedure (i.e.; {\bf read}) expects and then imports the data into the
application.  The research can then continue his processing and analysis
using the tools that he prefers for doing research.

The same is true for the data that he and his research collaborators wish to
exchange.  They access remote data over the Internet and have it imported
into their systems in the format that they support.  The data system access,
query, data translation and transfer are all transparent to the researcher.
Each researcher maintains data on his system in the format that he or she
utilizes locally without needing to be concerned about the format
requirements of other researchers.

\medskip \centerline{\bf Current Data Access Scenario} Prior to the
researcher utilizing his own applications to review the historical data there
are a number of preliminary steps that he must go through.  First he must
access each data system either through TELNET or WWW and utilize each
system's query interface to locate data of interest.  Next he must transfer
the data to his system, typically utilizing ftp to copy the data files.  In the case of the GTSPP data, he must also uncompress it.  If the
data is in a format that is not supported by his applications he must either,
reformat the data or recode, recompile and relink his application.  Finally, a GTSPP file contains
the global coverage of temperature and salinity data for a three month period
and is large (10,000 records, 10MB).  The researcher
will want to extract the data of interest and discard the rest of the file,
if only for space considerations since a year's worth of data would require
40MBytes of disk space.  Having gone through these steps the data can now be
imported into the researcher's applications for analysis.

In order to exchange data with collaborating researchers, the URI researcher
must go through many of the same steps that were required to get the
historical data.  He could make arrangements with his colleagues to ensure
that the data was provided in a form that his software could read but this
could lead to multiple copies of the same dataset in cases where more than
one format were required.  This only complicates the data management burden
on researchers who wish to provide their research data to other researchers.

It is important to reiterate that in the illustration of DODS implementation
above the URI researcher has not needed to modify his analysis application
nor to learn how to use a new program or system in order to access
distributed data resources from both large data archives or individual PIs.
The only procedure necessary to make the researcher's application DODS
compliant was to relink it with DODS client library software.  Another
important feature is that the very application the researcher turns to
everyday for conducting his research processing and analysis has been
transformed into an agent that support access to distributed data resources.
And finally, regardless of the form that the data is in at the remote server
source when it is imported to the client application it is translated into
the format that the calling procedure expects.

%\begin{htmlonly}
%\section{Current Status of DODS Development Effort}
%\end{htmlonly}
%\begin{latexonly}
\large
\bigskip
\noindent
{\bf Current Status of DODS Development Effort}
\medskip
\normalsize
%\end{latexonly}

\noindent The DODS development team has made significant progress toward the
goal of developing the system envisioned at the workshop.  As a result of the
developments team's evaluations, along with ideas and comments contributed
from the workshop attendees and other interested individuals, many of the
ideas put forward at the meeting have evolved during the formative stages of
the design process.  An open dialogue with individuals outside of the
development team was particularly helpful to the design of the data model.
The following specific issues have been addressed by the development team
since the workshop:{\renewcommand{\baselinestretch}{1.0} \footnote{For those
    interested in learning more about the detailed technical aspects of the
    \dods development effort, DODS developers maintain online documentation
    at {\tt http:/lake.mit.edu/dods.html}.}}

\begin{itemize}
 
   \item Specified a DODS system architecture based on the client-server
     model.  With the success of systems such as NCSA Mosaic, gopher and URI's
     XBrowse a client-server based approached was considered the most
     versatile and extensible.  

   \item Designed the DODS data delivery architecture.  The data delivery
     architecture specifies the configuration of client and server
     components, the operations performed by the system, and where those
     operations are performed.

   \item Designed a DODS data model which can support a wide range of earth
     science data types and structures.  The DODS data model is the central
     component of the DODS data access protocol.  It provides the means for
     translating between different data access mechanisms.

   \item Implemented a prototype of DODS using remote procedure calls (RPC)
     in order to determine feasibility of remote data access through existing
     interfaces such as netCDF.

   \item Identified HTTP as the network transfer protocol that will be used
     by DODS.  HTTP was selected for a number of reasons.  It is already a
     widely used protocol and is effectively managed by developers at NCSA,
     CERN and elsewhere.  Many researchers already have installed HTTP client
     and server software on their systems (e.g., NCSA Mosaic and NCSA HTTPD)
     and are familiar with its use. In addition the Common Gateway
     Interface (CGI) mechanism of HTTPD is flexible enough to allow
     sophisticated specialization of the server.

   \item Identified two candidate system applications for prototype
     development: netCDF and JGOFS. These two interfaces have different
     underlying data models and are representative of different types of
     data systems.  JGOFS supports a relational data model, whereas netCDF is
     designed for gridded data.  Implementing these two different
     application will present a broader overview of the technical challenges
     to extending DODS beyond these interfaces.
\end{itemize}

Currently, the JGOFS and netCDF client libraries and data servers are being
coded.  A DODS {\em toolkit} composed of core components that can be used to
build DODS client libraries and data servers for different systems is also
under development.  The toolkit will provide modules that manage the network
communications, implement the DODS data model, manage URLs, implement the
\dap operators, etc.  Prototyping of parts of the system will be taking
place in the Winter of 1995.

%\begin{htmlonly}
%\section{Summary}
%\end{htmlonly}
%\begin{latexonly}
\large
\bigskip
\noindent
{\bf Summary}
\medskip
\normalsize
%\end{latexonly}

\noindent
The \dods is being developed to help address the need within the
oceanographic community for researchers to easily access data resources over
the Internet.  DODS developers are creating a system that will allow
researchers to take advantage of the rapidly increasing availability of data
on-line by implementation of a data access protocol within a client-server
based infrastructure.  Two unique aspects of the DODS system design are that
it utilizes researchers' data analysis and processing applications for
accessing distributed data resources by transforming them into client agents
within DODS and it supports a system independent view of the data resources
available through its servers.  Rather than trying to replace or build a
better system, DODS is purposefully designed to be integrated with other
already existing data systems and user applications and to take advantage of
and extend those systems' capabilities.

\medskip
\large
\noindent{\bf Acknowledgements}\\
\normalsize
I would like to thank James Gallagher for his careful review of the paper and
critical input.

\bibliography{TOS-4-95}

\end{document}
