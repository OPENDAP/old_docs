
% The abstract for the DODS data delivery poster which was accepted by TOS
% for presentation at their 1995 biannual conference.
%
% $Id$

\documentstyle[12pt,margins]{article}

\begin{document}

\begin{abstract}

This poster describes the design of the HTTP-based data servers for
the Distributed Oceanographic Data System (DODS), and the clients that access
them. The data servers are built by assembling three programs which read
parameters passed from National Center for Supercomputing Applications's
HTTPD to the programs via the Common Gateway Interface (CGI) interface. In
the design, HTTPD is used as a network I/O management tool which simplifies
setup and maintenance of the data server. The programs which comprise a data
server return Multipurpose Internet Mail Extensions (MIME) documents as query
results. Two of the programs return information about the content of data set
(variable names, types, attributes) while the third program returns the
values of one of the variables in the data set.  This poster also describes
the DODS client libraries which use the information provided by data servers
to mimic the responses produced by various data-access Application Programmer
Interfaces (API) when they are used to read data files or data sets. For any
particular API, the DODS client library presents the same link-time interface
as the original library implementing that API. However, the DODS client
library replaces the local file I/O calls in the library with HTTP messages
to a DODS data server.  While the construction of the data servers is
typically straight forward, building the surrogate library for a particular
API can be very complicated. This complexity is the result of the need to
translate from the DODS Data Access Protocol (DAP) representation of
information to the representation used by the API. Although hard to
implement, this system has the advantage of providing for interoperability
between several different APIs supported by DODS. As new APIs are added the
set of interoperable APIs will further expand.

\end{abstract}

\end{document}