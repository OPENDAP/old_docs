
% A general technical description of DODS. Submitted to the Fourth WWW
% conference to be held on 12-14 Dec 1995 in Boston, MA.
%
% $Id$

\documentstyle[12pt,psfig,margins,harvard,html]{article}

\psfigurepath{WWW-conf-figs}

%
% These are html links which are used often enough in writing about DODS to
% merit an input file.
% jhrg. 4/17/94
%
% File rationalized and updated while writing the DODS User
% Guide. Also includes other useful abbreviations.
% tomfool 3/15/96
%
% $Id$
%
% HTML references to DODS documents
%
% For most of the DODS documentation there are three html references defined:
% 1) The upper case \newcommand produces the title of the paper, an
%    html link in the online documentation and a footnote in the paper
%    version.  
% 2) A capitalized version produces a capitalized description of the
%    paper and a link in the online version (but no footnote in the
%    paper version). 
% 3) A lower case version which produces a lower case description of
%    the paper and a html link, but no footnote.

% External references to documents:
%
% In order for external references to work (via latex2html) the labels used
% throughout the documents must be unique. The text labels are all prefixed
% with a document identifier (e.g., ddd) with a colon separater. The
% \externalrefs below (way below...) includes the perl files html.sty uses to
% make the cross refs.

\newcommand{\DODS}{\htmladdnormallink{Distributed Oceanographic Data System}
  {http://www.unidata.ucar.edu/packages/dods/}}

\newcommand{\Dods}{\htmladdnormallink{DODS}{http://www.unidata.ucar.edu/packages/dods/}}

\newcommand{\wrkshp}{\htmladdnormallink{Report on the First Workshop for the
    Distributed Oceanographic Data System, Proposed System Architectures}
  {http://www.unidata.ucar.edu/packages/dods/archive/reports/workshop1/section3.13.html}}

% not DODS URLs, but useful all the same...

\newcommand{\www}{\htmladdnormallink{World Wide Web} 
  {http://www.w3.org/hypertext/WWW/TheProject.html}}
\newcommand{\url}{\htmladdnormallink{Uniform Resource Locator}
  {http://www.w3.org/hypertext/WWW/Addressing/URL/url-spec.html}}

\newcommand{\uri}{\htmladdnormallinkfoot{Uniform Resource Identifiers}
  {http://www.w3.org/hypertext/WWW/Addressing/URL/uri-spec.html}}

\newcommand{\urn}{\htmladdnormallinkfoot{Uniform Resource Name}
  {http://www.acl.lanl.gov/URI/archive/uri-archive.messages/1143.html}}

\newcommand{\HTTPD}{\htmladdnormallinkfoot{HTTPD}
  {http://hoohoo.ncsa.uiuc.edu/docs/Overview.html}}

\newcommand{\HTTP}{\htmladdnormallinkfoot{HyperText Transfer Protocol}
  {http://www.w3.org/hypertext/WWW/Protocols/HTTP/HTTP2.html}}

%\newcommand{\HTML}{\htmladdnormallinkfoot{HyperText Markup Language}
%  {http://www.w3.org/hypertext/WWW/MarkUp/MarkUp.html}}

% CERN used to be `Conseil Europeen pour la Recherche Nucleaire'
\newcommand{\CERN}
  {\htmladdnormallink{European Laboratory for Particle Physics}
  {http://www.cern.ch/}}

\newcommand{\NCSA}
  {\htmladdnormallink{National Center for Supercomputing Applications}
  {http://www.ncsa.uiuc.edu/}}

\newcommand{\WWWC}{\htmladdnormallink{World Widw Web Consortium}
  {http://www.w3.org/}}

\newcommand{\CGI}{\htmladdnormallinkfoot{Common Gateway Interface}
  {http://hoohoo.ncsa.uiuc.edu/cgi/overview.html}}

\newcommand{\MIME}{\htmladdnormallink{Multipurpose Internet Mail Extensions}
  {http://www.cis.ohio-state.edu/htbin/rfc/rfc1590.html}}

\newcommand{\netcdf}{\htmladdnormallink{NetCDF}
  {http://www.unidata.ucar.edu/packages/netcdf/guide.txn_toc.html}}

\newcommand{\JGOFS}{\htmladdnormallink{Joint Geophysical Ocean Flux Study}
  {http://www1.whoi.edu/jgofs.html}}

\newcommand{\jgofs}{\htmladdnormallink{JGOFS}
  {http://www1.whoi.edu/jgofs.html}}

\newcommand{\hdf}{\htmladdnormallink{HDF}
  {http://www.ncsa.uiuc.edu/SDG/Software/HDF/HDFIntro.html}}

\newcommand{\Cpp}{{\rm {\small C}\raise.5ex\hbox{\footnotesize ++}}}

% Commands

\newcommand{\pslink}[1]{\small
\begin{quote}
  A \htmladdnormallink{postscript}{#1} version of this document is
  available.  You may also use \htmladdnormallink{anonymous
  ftp}{ftp://ftp.unidata.ucar.edu/pub/dods/ps-docs/} to access postscript files
  of all of the DODS documentation.
\end{quote}
\normalsize
}

\newcommand{\declaration}[1]{\small
{\tt {#1}}
\normalsize
}

% All the links from here down are for the white papers. I'm not sure that any
% of these are still valid given the reorganization of the web site. 5/12/98
% jhrg.

\newcommand{\DDA}{\htmladdnormallinkfoot{DODS---Data Delivery Architecture}
  {http://www.unidata.ucar.edu/packages/dods/archive/design/data-delivery-arch/data-delivery-arch.html}}
\newcommand{\Dda}{\htmladdnormallink{Data Delivery Architecture}
  {http://www.unidata.ucar.edu/packages/dods/archive/design/data-delivery-arch/data-delivery-arch.html}}
\newcommand{\dda}{\htmladdnormallink{data delivery architecture}
  {http://www.unidata.ucar.edu/packages/dods/archive/design/data-delivery-arch/data-delivery-arch.html}}

\newcommand{\DDD}{\htmladdnormallinkfoot{DODS---Data Delivery Design}
  {http://www.unidata.ucar.edu/packages/dods/archive/design/data-delivery-design/data-delivery-design.html}}
\newcommand{\Ddd}{\htmladdnormallink{Data Delivery Design}
  {http://www.unidata.ucar.edu/packages/dods/archive/design/data-delivery-design/data-delivery-design.html}}
\newcommand{\ddd}{\htmladdnormallink{data delivery design}
  {http://www.unidata.ucar.edu/packages/dods/archive/design/data-delivery-design/data-delivery-design.html}}

\newcommand{\URL}{\htmladdnormallinkfoot{DODS---Uniform Resource Locator}
  {http://www.unidata.ucar.edu/packages/dods/archive/design/urls/urls.html}}
\newcommand{\Url}{\htmladdnormallink{Uniform Resource Locator}
  {http://www.unidata.ucar.edu/packages/dods/archive/design/urls/urls.html}}
\newcommand{\dodsurl}{\htmladdnormallink{uniform resource locator}
  {http://www.unidata.ucar.edu/packages/dods/archive/design/urls/urls.html}}

\newcommand{\DAP}{\htmladdnormallinkfoot{DODS---Data Access Protocol}
  {http://www.unidata.ucar.edu/packages/dods/archive/design/api/api.html}}
\newcommand{\Dap}{\htmladdnormallink{Data Access Protocol}
  {http://www.unidata.ucar.edu/packages/dods/archive/design/api/api.html}}
\newcommand{\dap}{\htmladdnormallink{data access protocol}
  {http://www.unidata.ucar.edu/packages/dods/archive/design/api/api.html}}

\newcommand{\SOFT}
        {\htmladdnormallinkfoot{DODS---Software Development Environment}
  {http://www.unidata.ucar.edu/packages/dods/archive/managment/software/software.html}}
\newcommand{\Soft}{\htmladdnormallink{Software Development Environment}
  {http://www.unidata.ucar.edu/packages/dods/archive/managment/software/software.html}}
\newcommand{\soft}{\htmladdnormallinkfoot{software development environment}
  {http://www.unidata.ucar.edu/packages/dods/archive/managment/software/software.html}}

% I used to call the DAP the API, then for a few days I called it the
% DTP. These def's were used then. Rather than find everywhere they are used
% (not that hard, but...) I just hacked the macros. NB: the source file for
% the DAP document is still called `api.tex'

\newcommand{\API}{\htmladdnormallinkfoot{DODS---Data Access Protocol}
  {http://www.unidata.ucar.edu/packages/dods/archive/design/api/api.html}}
\newcommand{\api}{\htmladdnormallink{data access protocol}
  {http://www.unidata.ucar.edu/packages/dods/archive/design/api/api.html}}

\newcommand{\DTP}{\htmladdnormallinkfoot{DODS---Data Access Protocol}
  {http://www.unidata.ucar.edu/packages/dods/archive/design/api/api.html}}
\newcommand{\dtp}{\htmladdnormallink{data access protocol}
  {http://www.unidata.ucar.edu/packages/dods/archive/design/api/api.html}}

\newcommand{\TOOLKIT}{\htmladdnormallinkfoot{DODS---Client and Server Toolkit}
  {http://www.unidata.ucar.edu/packages/dods/archive/implementation/toolkits/toolkits.html}}
\newcommand{\Toolkit}{\htmladdnormallink{Client and Server Toolkit}
  {http://www.unidata.ucar.edu/packages/dods/archive/implementation/toolkits/toolkits.html}}
\newcommand{\toolkit}{\htmladdnormallink{client and server toolkit}
  {http://www.unidata.ucar.edu/packages/dods/archive/implementation/toolkits/toolkits.html}}

\newcommand{\TKR}{\htmladdnormallinkfoot{DODS---DAP Toolkit Reference}
  {http://www.unidata.ucar.edu/packages/dods/archive/implementation/toolkits/toolkits.html}}
\newcommand{\Tkr}{\htmladdnormallink{DAP Toolkit Reference}
  {http://www.unidata.ucar.edu/packages/dods/archive/implementation/toolkits/toolkits.html}}
\newcommand{\tkr}{\htmladdnormallink{DAP toolkit reference}
  {http://www.unidata.ucar.edu/packages/dods/archive/implementation/toolkits/toolkits.html}}

% I know that these are wrong. The papers have been moved to the gso web
% site, but I'm not sure exactly where. 5/12/98 jhrg

\newcommand{\DM}{\htmladdnormallinkfoot{DODS---Data Model}
  {http://lake.mit.edu/dods-dir/dodsdm2.html}}
\newcommand{\Dm}{\htmladdnormallink{Data Model}
  {http://lake.mit.edu/dods-dir/dodsdm2.html}}
\newcommand{\dm}{\htmladdnormallink{data model}
  {http://lake.mit.edu/dods-dir/dodsdm2.html}}

\newcommand{\DD}{\htmladdnormallinkfoot{DODS---Data Delivery}
  {http://lake.mit.edu/dods-dir/dods-dd.html}}

% external refs for DODS documents

\externallabels{http://www.unidata.ucar.edu/packages/dods/design/api}
  {/home/www/packages/dods/archive/design/api/labels.pl}
\externallabels{http://www.unidata.ucar.edu/packages/dods/design/data-delivery-arch}
  {/home/www/packages/dods/archive/design/data-delivery-arch/labels.pl}
\externallabels{http://www.unidata.ucar.edu/packages/dods/design/data-delivery-design}
  {/home/www/packages/dods/archive/design/data-delivery-design/labels.pl}
\externallabels{http://www.unidata.ucar.edu/packages/dods/implementation/toolkits}
  {/home/www/packages/dods/archive/implementation/toolkits/labels.pl}

%\renewcommand{\externalref}[2]{#2}

%%% Local Variables: 
%%% mode: latex
%%% TeX-master: t
%%% End: 


\begin{document}

\bibliographystyle{agsm}

\begin{latexonly}
\author{James Gallagher\thanks{The University of Rhode Island Graduate School
    of Oceanography, South Ferry Road, Narragansett, RI. 02882}\\ 
  \and George Milkowski$^{*}$} 
\end{latexonly}

\begin{htmlonly}
\author{James Gallagher\thanks{The University of Rhode Island Graduate School
    of Oceanography, South Ferry Road, Narragansett, RI. 02882\\
    jimg@dcz.gso.uri.edu}\\ 
  \and George Milkowski\thanks{The University of Rhode Island Graduate School
    of Oceanography, South Ferry Road, Narragansett, RI. 02882\\
    george@zeno.gso.uri.edu}}
\end{htmlonly}

\title{Data Transport Within The Distributed Oceanographic Data System}

\date{\today}

\maketitle

\begin{abstract}

The \DODS\ is a client-server system which enables existing data analysis
programs to be transformed from software which is limited to accessing local
data to clients which can access data from any of a large number of data
servers. This is done without requiring modification to the existing
software's source code by re-implementing the data access API libraries used
by those programs. Once a given API library is modified to read from a DODS
data server, any program which uses that library as its sole means of
accessing data can be re-linked with the new implementation and can function
as a client within DODS. Data servers which can be accessed by these clients
are built using filter programs and one or more httpd Common Gateway
Interface (CGI) programs. Data sets made available are thus accessible via
URLs. While the data servers' principal function is to provide the client
programs with values from the data sets, they also make available, as text,
information about the variables in the data set and their data types. This
information can be accessed by any WWW browser.

\end{abstract}

\clearpage

\tableofcontents

\clearpage

% This portion of the paper was originally part of a paper that appeared in
% RV Tech. The CVS ID for that doc was: 
% `rvtec.tex,v 1.1 1995/07/12 19:45:22 george Exp'

\section {Introduction}

It does not take long using the World Wide Web (WWW) to discover many hundreds
of scientific datasets that are currently available on-line to researchers.
For oceanographers the volume and diversity of data from national archives,
special program offices or other researchers that is of potential value to
their research is overwhelming.  However, while a large number of oceanographic
datasets are on-line, from the research oceanographer's point of view there
are significant impediments which often make acquiring and using these
on-line data hard\cite{muntz:data}.

The storage format of data is often specific to a particular system making it
difficult to view or combine several datasets even though, as Pursch points
out, combining data sets is often a key requirement of global-scale earth
science\cite{pursch:newtools}. Furthermore, most data archives have developed
their own data management systems with specialized interfaces for navigating
their data resources. Examples of such specialized systems include the Global
Land Information System (GLIS)\cite{USGS:glisurl} developed by the
U.S. Geological Survey (USGS) and the NOAA/NASA Oceans Pathfinder Data System
developed at the Jet Propulsion Laboratory
(JPL)\cite{JPL:oceansurl}. Virtually none of these data systems interoperate
with each other, making it necessary for a user to visit many systems and
`learn' multiple interfaces in order to acquire data.  Finally, even after
the data has been successfully transfered to the researcher's local system,
in order to {\em use} the data a researcher must convert that data into the
format that his or her data analysis application requires or alternatively
modify the analysis application\cite{pursch:newtools}.

Many national data centers and university laboratories are now providing
remote access to their scientific data holdings through the World Wide
Web. Users are able to select, display and transfer these data using WWW
browsers such as Netscape and NCSA Mosaic.  Examples of such systems are the
NOAA / PMEL - Thermal Modeling and Analysis Project\cite{PMEL:thermalurl} and
the University of Rhode Island Sea surface Temperature
Archive\cite{URI:ssturl}.  However, current generation HTML browsers
(Netscape 1.1, Mosaic 2.4) are limited and cumbersome when compared to other
data access and display client-server systems such as the Global Land
Information System (GLIS) developed by the USGS or URI's XBrowse **cite
Xbrowse**.  While WWW browsers are very useful for pedagogical purposes they
have limited capabilities in terms of data analysis or manipulation.  In the
era of Global Change researchers will need tools that provide both network
access and data analysis functionality
\cite{pursch:newtools}\cite{NSF:globalchange}.

Finally, while large data centers have a clearly defined data policy, and
often a mandate to make data accessible to members of the research
community\cite{NOAA:policy}, there is no infrastructure that enables
individual scientists to make data accessible to others in a simple
way. While many scientists in the earth-sciences community share data, and
cite shared data as one of their most important resources, doing so is often
cumbersome\cite{DODS:workshop1}. Systems like the World Wide Web make sharing
research results vastly simpler, but do little to reduce the difficulty of
sharing raw data.

To address these problems, researchers at the University of Rhode Island and
the Massachusetts Institute of Technology are creating a network tool that,
while taking advantage of WWW data resources, helps to resolve the issue of
multiple data formats and different data systems interfaces.  This network
tool, called the Distributed Oceanographic Data System (DODS), enables
oceanographers to interactively access distributed, on-line science data
using the one interface that a researcher is already familiar with; existing
data analysis application software (i.e., legacy systems) while at the same
time providing a set of tools which can be used to build new application
software specifically intended to work with distributed resources.  The
architecture and design of DODS makes it possible for a researcher to
open, read, subsample and import directly into his or her data analysis
applications scientific data resources using the WWW.  The researcher will not
need to know either what format is used to store the data or how the data is
actually accessed and served by the remote data system\cite{DODS:workshop1}.

\section {Extending Existing Data Access APIs}

The Distributed Oceanographic Data System is a specification for directly
accessing, representing and transferring science data on a network. It is a
{\em data access protocol\/} which includes both a common functional interface
to data systems and a data model for representing data on those systems.  It
is designed to be integrated with already existing user applications and
resource management systems, not to replace them.

DODS models data analysis programs as some body of user written code linked
with one or more API libraries. The API presents a specialized interface
which the user program uses to read data. It is straightforward to split the
user program at the program-library interface, and by adding suitable
interprocess communication layers, create a classical client and server which
can use peer-to-peer communications across a
network. Figure~\ref{fig:prog2client-server} shows how this can be done using
Sun's RPC technology\cite{sun:rpc}. However, any suitable network layer can
accomplish this goal\cite{stevens:unp}.

For the remainder of this paper a data access API that has been modified to
satisfy each of its functions by communicating with a matching data server,
as opposed to accessing a local file, will be referred to as a {\em client
library\/}. The matching data server will be referred to as a {\em data
server\/} or simply as a {\em server\/}.

Once a client library has been constructed for a given API, it is possible to
re-link many user programs written for the API with this new library. If the
API hides the storage format of the data being accessed and the program uses
the API correctly (i.e., without taking advantage of any undocumented
features present in the original version but not present in the new
client-library version), then the program will require no modification to
work with the new implementation of the API. A program thus re-linked can
read data from any machine so long as that machine has installed a matching
data server. Because the same API has been used by many programs,
re-implementing it so that it reads information over the network facilitates
the transformation of each program into a client capable of accessing data
provided by any suitable server on the network.

\begin{figure}
\centerline{\psfig{figure=prog2client-server.ps,width=4.5in}}
\caption{Implementation of Distributed Application---API Access Using
  Remote Procedure Calls.}
\label{fig:prog2client-server}
\end{figure}

\subsection {Data Delivery Design}

%
% This portion of the paper was originally part of a paper that appeared on
% the DODS home page. The CVS ID for that doc was: 
% `data-delivery-design.tex,v 1.12 1995/01/20 22:18:02 jimg Exp'
%

Several designs were considered for the data delivery mechanism of DODS.
They were socket-based peer-to-peer communications, RPC-based peer-to-peer
communications, virtual file systems and HTTP/CGI-based client/server
systems. The first three of these different designs are compared in: \wrkshp\ 
and ``\DD'', which presents our rationale behind prototyping the RPC-based
design for DODS. However, as a result of those prototypes and the development
of HTTP as a {\em de facto\/} data communications standard, we changed the
data delivery design to an HTTP/CGI-based system.

By using HTTP as a transport protocol, we are able to tap into a large base
of existing software which will likely evolve along with the Internet as a
whole. Because the development of large-scale distributed systems is
relatively new there are many problems which must still be addressed for
these systems to be robust. These problems include naming resources
independently of their physical location, choosing between two objects which
appear to be the same but which differ in terms of quality. These are general
problems which are hard to solve because they will be solved effectively only
when the Internet community reaches a consensus on which of the available
solutions are best. HTTP, because it is so widely accepted, provides a
reasonable base for such solutions. This view is supported by the recent
Internet Engineering Task Force (IETF)\cite{IETF:url} work on extending the
\HTTP\ and \HTML\ standards.

\subsection{Client/Server and Program/Library Interfaces}

The DODS client library and data server programs communicate information
using URLs and MIME\cite{rfc:mime} documents. Figure~\ref{fig:data-transport}
shows these two communication paths.

All information sent from the server programs to the client library is
enclosed in a MIME document. Two of the three programs return information
about the variables contained in the data set as {\tt text/plain} MIME
documents. These documents can then be parsed by software in the client
library. In addition, these text documents can be read by any software that
can process {\tt ASCII} text. Thus, the responses made by the server are
specifically suited to use by the DODS client libraries, they can also be
used by many more general programs. For example, it is possible the use a
general purpose World Wide Web browser to `read' these documents.

The third data server program returns binary data encoded using Sun
Microsystems External Data representation (XDR)\cite{sun:xdr} scheme. The
data is enclosed in a binary MIME document. This document can be read by
software that is part of the client library using the additional information
contained in the two {\tt ASCII} documents. This document can only be read by
software that can interpret the datatype information sent by the server in
the {\tt ASCII} documents. Because of this, it is not possible for other
general purpose WWW browsers to interpret this file (although most browsers
can read and save to disk any arbitrary data).

In order to provide link-time compatibility with the original API libraries,
the DODS client libraries must present {\em exactly\/} the same external
interface as the original libraries. However, these new libraries perform
very different operations on the data (although, for a API used to access a
self-describing data format the operations are analogous). One difference
between the two is that most data access APIs use file names to refer to data
sets. In the simplest case these file names are given on the command line by
the user and passed, without modification, to the API. The API uses the file
name to open a file and returns an identifier of some type to the user
program. Subsequent access to the data are made through this identifier.

In this simple scenario, it is possible to substitute a URL in place of a
file name (in part because both are stored in string-type variables). This
same user program can be invoked, on the command line, using a URL in place
of the file name. The program will, in almost all cases, pass the URL to the
API to open the data set. However, since the user program has been re-linked
with the DODS re-implementation of the API, the functions in the API will
correctly interpret the URL as a remote reference. Clearly, one requirement
that a user program must meet in order to be re-linked with DODS is that it
must not itself try to open or otherwise manipulate the `file name' which
will be passed to the API.

Rewriting existing APIs is hard to implement; it would be much simpler to
write our own data display and analysis program to read data from DODS data
servers. However, it is important that as much legacy software as possible be
able to read the data made accessible by DODS data servers. While this may
sound like a trivial requirement, it is wrong to assume that existing data
analysis software is simple or can be rewritten at little cost; the existing
software in the sciences is no less expensive to rewrite than in any other
field. Furthermore, some researchers tailor there research efforts to the
characteristics of particular data systems and see the costs of abandoning
those systems as very high\cite{DODS:workshop1}.

\begin{figure}
\centerline{\psfig{figure=data-transport.ps}}
\caption{MIME documents are used by the server programs to return information
  to the client processes.}
\label{fig:data-transport}
\end{figure}

\subsection{Data Servers}
\label{ddd:data-servers}

A DODS data server is a collection of three executable programs and/or CGIs
which provide access to data using HTTP. Each of the program/CGI units is
capable of satisfying one of the three requests which are defined in the
\Dap. The DAP defines two ways to access metadata which describes the
contents of the data set (one for use by the DODS surrogate libraries and one
for use by third-party software and users) and one way to access data. Data
access is accomplished by reading the variables which comprise the data set.
This access can be modified using a {\em constraint expression\/} so that
only portions of the variable are actually read from the data set.

The ability to evaluate constraint expressions is an essential characteristic
of a DODS data server. In many cases reading a single variable from a data
set results in data which is of little or no interest to the user. Often
users are interested in those values of a variable which meet some additional
criteria (e.g., they fall within a certain time range). For a complete
description of the data types supported by DODS and the constraint expression
operators, see ``\DAP''.

\subsection{Basic Requirements for the Data Servers}

The data servers must satisfy two requirements: they must provide access to
data via the DAP and they must use on-line data without requiring its
modification. 

Providing access to data using the DAP is necessary because that is how the
DODS architecture provides interoperability between different APIs. Because
the data servers translate accesses to a data set from the DAP into either
an API (e.g., \netcdf\ if the data set is stored using that API) or a special
format (e.g., GRIB), any (client) process that uses the DAP can access the
data. The underlying access mechanism is hidden from the client by the API. 

In the current design of DODS, meeting this requirement means that for
each API or format in which data is stored, a new DODS data server must be
built.

The other requirement which each server must satisfy is that data, however it
is stored, should not require modification to be served by a DODS data
server. This is important because many data sets are large and thus very
expensive to modify. It is a poor practice to force data providers to modify
their data to suit the needs of a system. Rather, DODS data servers must be
able to translate access via the DAP into the local storage mechanism
without changing that local storage mechanism.

This requirement limits DODS to those APIs and formats which are, to some
extent, self-describing. Because the DAP bases access on reading a named
variable, it must be possible, for each data set, to define the set of
variables and to `read' those variables from the data set. However some data
sets do not contain enough information to make remote access a
reality. Instead, additional information, not in the data set itself, is
needed. This information can be stored in ancillary data files which
accompany the data set. Note that these files are separate from the data
set, they are not added to the data set and do not require any modification
of the data set. 

% This portion of the paper was originally part of a paper that appeared on
% the DODS home page. The CVS ID for that doc was: 
% `api.tex,v 1.11 1994/12/05 17:26:42 jimg Exp'

\section{The Data Access Protocol}
\label{www:dap}

The Data Access Protocol (DAP) is used as an intermediate representation for
data which are nominally accessed using an established third-party
Application Programmer Interface (API) (e.g., netCDF). Because, in addition
to a method for data access, the DAP defines the content of ancillary
information about data sets, it provides the means to access data sets
through a single protocol using any of the DODS supported APIs. Because the
\dap\ serves as an intermediate representation for several different APIs, it
can be used to translate between any two of those APIs\cite{treinish:models}.
translation between various APIs was an important requirement for DODS
because it was felt that any system which could not perform such translation
would artificially limit the data accessible by any one client
program\cite{DODS:workshop1}.

The DODS DAP design contains three important parts: A data model which
describes data types that can be supported by the protocol and how they are
handled, the data set description and data set attribute structures which
describe the structure of data sets and the data they contain, and a small
set of messages that are used to access data. Each of these components are
described in the following subsections.

\subsection{The Data Model}

Data models provide a way to organize scientific data sets so that useful
relationships between individual datum are evident. Many data models have
been specifically designed to make using the data in a computer program
simpler\cite{rew:netcdf}\cite{NCSA:HDF}. Examples of computationally oriented
data models for scientific data are hierarchical, sequential, and gridded
data models\cite{treinish:models}.

Data models are abstract, however, and to be used by a computer program they
must first be implemented by a programmer. Often this implementation takes
the form of an API---a library of functions which can read and write data
using a data model or models as
guidance\cite{rew:netcdf}\cite{NCSA:HDF}. Thus every data access API can be
viewed as implementing some data model, or in some cases several data models.

Because DODS needs to support several very different data models, it is
important to design it around a core set of concepts that can be applied
equally well to each of those data models. If that can be done, then
translation between data represented in those different models may be
possible\cite{treinish:models}.

Currently DODS supports two very different data access APIs: netCDF and
JGOFS\cite{JGOFS:url}. The netCDF API is designed for access to gridded data,
but has some limited capabilities to access sequence data (although not with
all of its supported programming language interfaces). The JGOFS API provides
access to relational or sequence data.  Both APIs support access in several
programming languages (at least C and Fortran) and both provide extensive
support for limiting the amount of data retrieved.  For example a program
accessing a gridded data set using netCDF can extract a subsampled portion or
{\em hyperslab} of that data\cite{rew:netcdf}. Likewise, the JGOFS API
provides a powerful set of operators which can be used to specify which type
of sequence elements to extract (e.g., only those corresponding to data
captured between 1:00am and 2:00am) as well as masking certain parameters
from the returned elements so that only those parameters needed by the
program are returned.

The DODS DAP uses the concepts of variables and operators as the base for the
data model. Within the data model, a data set consists of one or more
variables where each variable is described formally by a number of
attributes.  Variables associate names with each component of a data set, and
those names are used to refer to the components of the data set. In addition
to their different attributes, it is possible to operate on individual
variables or named collections of variables. The principal operation is {\em
access}, although in a future version of DODS it will be possible to modify
this in a number ways.

\subsubsection{Base-Type Variables}
\label{base-type-variables}

Variables in the DODS DAP have two forms. They are either base types or type
constructors. Base type variables are similar to predefined variables in
procedural programming languages like C or Fortran (e.g., {\tt int} or {\tt
  integer*4}).  While these certainly have an internal structure, it is not
possible to access parts of that structure using the DAP\@. Instead the DAP
is used to transfer the values of those variables from the server to the
client and, once on the client side, access those values. These types of
variables correspond to the simplest types of variables used in both common
analysis software and data access APIs\cite{treinish:models}.

\subsubsection{Type Constructor Variables} 

Type constructor variables describe the grouping of one or more variables
within a data set. These classes are used to describe different types of
relations between the variables that comprise the data set. This information
can be useful to people who would like to understand more about the data set
than can be conveyed with implicit relations. It is also designed to be
useful to other programs/processes in the data access chain.  There are six
classes of type constructor variables defined by the DAP: lists, arrays,
structures, sequences, functions, and grids. The type constructor classes
besides structure provide information that is used in the translation of
subsetting operations (hyperslabbing or selections and projections in netCDF
or JGOFS parlance, respectively). They also provide a means to describe many
different data types since each of the constructor types can contain each
other (as well as instances of themselves). Thus, as is the case with
programming languages such as C, DODS provides for an infinite variety of
data types built using various combinations of the constructor and base types\cite{horowitz:cpaper}.

\subsection{The External Representation of Variables}
\label{api:external-rep}

Each of the base-type and type constructor variables has an external
representation defined by the \dap. This representation is used when an
object of the given type is transferred from one computer to another.
Defining a single external representation simplifies the translation of
variables from one computer to another when those computers use different
internal representations for those variable types.  The \dap\ uses Sun
Microsystems' XDR\cite{sun:xdr} protocol for the external representation of
all of the base type variables. This representation was chosen so that
values would be transparent across various machines without transforming them
first to {\tt ASCII}. For some types, {\tt ASCII} is an acceptable `network'
representation, but other types (e.g., floating point types) require
significantly more storage when represented as {\tt ASCII}. XDR was chosen
Because it defines binary representations for such types\cite{sun:xdr} and
because of its widespread availability.

\subsection{Dataset Descriptor Structure}
\label{api:dds}

In order to translate from the user program's API to the data set's API, the
translator process must have some knowledge about the types of the variables,
and their semantics, that comprise the data set. It must also know something
about the relations of those variables---even those relations which are only
implicit in the data set's own API\@. This knowledge about the data set's
structure is contained in a text description of the data set called the {\em
  Dataset Description Structure}.

The data set description structure (DDS) does not describe how the
information in the data set is physically stored, nor does it describe how
the data set's API is used to access that data. Those pieces of information
are contained in the data set's API and in the translating server,
respectively.  The server uses the DDS to describe the logical structure of a
particular data set---the DDS contains knowledge about the data set variables
and the interrelations of those variables.  In addition, the DDS can be used
to satisfy some of the DODS supported APIs data set description calls. For
example, netCDF has a function which returns the names of all the variables
in a netCDF data file. The DDS can be used to get that information.

The DDS is a textual description of the variables and their classes that
comprise the entire data set. The data set descriptor syntax is based on the
variable declaration/definition syntax of C\cite{ritchie:c}. A variable that
is a member of one of the base type classes is declared by by writing the
class name followed by the variable name.

An example DDS entry is shown in Figure~\ref{fig:dds}. Suppose that three
experimenters have each performed temperature measurements at different
locations and at different times. This information could be held in a data set
consisting of a sequence of the experimenter's name, the time and location of
each measurement and the list of measurements themselves, and indicates that
there is a relation between the experimenter, location, time and temperature
called temp\_measurement.

\begin{figure}
\begin{center}
\begin{verbatim}
              data set {
                  int catalog_number;
                  function {
                    independent:
                      string experimenter;
                      int time;
                      structure {
                          float latitude;
                          float longitude;
                      } location;
                    dependent:
                      sequence {
                          float depth;
                          float temperature;
                      } temperature;
                  } temp_measurement;
              } data;
\end{verbatim}
\end{center}
\caption{Example Dataset Descriptor Entry.}
\label{fig:dds}
\end{figure}

\subsection{Dataset Attribute Structure}
\label{api:das}

The Dataset Attribute Structure (DAS) is used to store attributes for
variables in the data set. We define an attribute as any piece of information
about a variable that the creator of the data set wants to bind with that
variable {\em excluding\/} the information contained in the DDS. This
definition is essentially the one used by both netCDF\cite{rew:netcdf} and
HDF\cite{NCSA:HDF}. The characteristics described by the DDS are always
defined for every variable; they are data type information about the
variable. Attributes, on the other hand, are intended to store extra
information about the data such as a paragraph describing how it was
collected or processed. In principle attributes are not processed by software
other than to be displayed. However, many systems rely on attributes to store
extra information that is necessary to perform certain manipulations on data.
In effect, attributes are used to store information that is used `by
convention' rather than `by design'. DODS can effectively support these
conventions by passing the attributes from data set to user program via the
DAS. Of course, DODS cannot enforce conventions in data sets where they were
not followed in the first place.

Every attribute of a variable is a triple: attribute name, type and
value. The attributes specified using the DAS are different from the
information contained in the DDS. Each attribute is completely distinct from
the name, type and value of its associated variable. The name of an attribute
is an identifier, following the normal rules for an identifier in a
programming language with the addition that the `/' character may be
used. The type of an attribute may be one of: Byte, Int32, Float64, String or
Url. An attribute may be scalar or vector.

When the \dap\ is used to read the attributes of a variable and that variable
contains other variables, only the attributes of the named variable are
returned. In other words, while the DDS is a hierarchical structure, the DAS
is {\em not\/}; it is similar to a flat-file database.

\section {Conclusion}

The \DODS\ (DODS) is a client-server system which provides scientific
researchers with a tool to access data from a wide variety of sources
including other scientists as well as national data centers. Unlike most other
distributed data systems, DODS uses existing analysis programs to access data
by providing software developers and users with new implementations of
existing data access APIs. These new implementations make use of remote data
servers to satisfy the APIs function calls. Thus the user program needs no
modification to read remote data and program authors, who may not be software
development specialists, do not need to learn a new data access paradigm to
use the remote data. By judiciously choosing which APIs to re-implement,
a large body of software developed outside of the DODS project, with which
users are already comfortable, is able to access remote data via DODS
servers.

Data servers for DODS are built using the WWW server httpd from NCSA. A data
server consists of three filter programs and a dispatch CGI. Each data set is
referred to via a URL which contains the name of the CGI and some identifying
keywords which vary from API to API. Two of the three programs which comprise
a data server return textual descriptions of the contents of the data set and
can be viewed by any WWW browser. However, the principal function of these
two filter programs is to provide information to the client library which it
will use to request and decode the information returned by the third filter
program---the values of discrete variables within the data set.

Each data set is accessed using an intermediate representation that is
independent of a particular machine representation or API. This enables the
client library which replaces API X to access a data server which provides
access to data stored on disk in files written using API Y given that a
correct DODS data server for API Y and a correct client library for API X
exist. Thus for the set of APIs which DODS chooses to address,
researchers are free to access data without concern for its native storage
format. 

Currently DODS supports two different data access APIs: netCDF and JGOFS. As
of \today\ and alpha release of DODS is available (both C and C++ source code
as well as pre-compiled binaries) from
ftp::/dods.gso.uri.edu/pub/dods. Additional documentation on DODS may be
found at http://dods.gso.uri.edu.

\bibliography{WWW-conf-12-95}

\end{document}
