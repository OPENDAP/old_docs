The OPeNDAP Technical Working Conference 2003 took place in Boulder,
Colorado on 19-21 March 2003.  The meeting was attended by approximately
40 people, some of whom attended intermittently due to the inclement
weather.  Slides from the presentations are presented on-line at:
\xlink{http://www.po.gso.uri.edu/tracking/meetings/tech03/}{http://www.po.gso.uri.edu/tracking/meetings/tech03/}
This report is available from that site as well.

The presentations were divided into 7 categories:
\begin{enumerate}
   \item System-wide Overviews-B.
   \item The New \ac{DAP} Specification-B.
   \item The \ac{AIS}-B.
   \item The Treatment of In-Situ Data-B.
   \item Data Discovery and Inventory Methods, Tools, and Issues-B.
   \item Connecting \ac{SIS} to \ac{GIS}-B.
   \item Experiences with Serving and Accessing \ac{OPeNDAP} Data.
\end{enumerate}

Breakout discussions for at least an hour each were held for the
five topics indicated with a "B".
 
The overviews concentrated on \ac{OPeNDAP} (Cornillon), the evolution
of \ac{OPeNDAP} into the field of high-performance computing (Fox),
and \ac{OPeNDAP}'s role in the developing Ocean Observing System
(Hankin).

That the specification of the \ac{DAP} needs revision has been evident
for some time.  Particularly clear has been that the specification
needs to be independent of implementation solutions, and independent
of the underlying transport mechanism.  To that end, relevant talks
and a breakout session were devoted to discussion the \ac{DAP} Specification,
and its revision.  The result was several good ideas about elements
that should be included or revised in the new specification, but the
overall reaction was very favorable toward the proposed modifications.

The \ac{DAP} was originally designed as a means of coding and transporting
data from a server to a client in such a manner that the content of the
original data is preserved, but that the storage format of the data
is transparent to the client.  However, no specific information about
the data is required other than the syntactic metadata needed to
``unpack'' the data at the client when they are received.  Users and
computer programs need to know more about the data than just the
semantic metadata in order to make intelligent use of those data.
The need for an ``Ancillary Information System'' to append and augment
the data and metadata from an original server has become incressingly
clear as users try to use the data they access through \ac{OPeNDAP}.
Two prototypes of \ac{AIS}-like systems were demonstrated at the
meeting.  Then, during the \ac{AIS} breakout session, various aspects
of the proposed system were discussed.

The problem of treating in-situ data in a tractable fashion that allows
fast access to a wide variety of queries was discussed extensively
in the in-situ breakout session.  Several suggestions were made which
will be pursued, including the design and implementation of an
aggregation server for in-situ data.

The problem of data discovery, once data are available through
\ac{OPeNDAP}, is of growing concern amoung users.  \ac{THREDDS}
is an emerging technology that addresses the issue of 
cataloging and traversing inventories for users seeking information
about datasets.  An implementation of \ac{THREDDS} on one system
was described.  Also, the problem of discovery of datasets and
their inventories, was demonstrated using the \ac{ODC}.  The breakout
session concentrated on the question of how providers can make
their datasets, served by \ac{OPeNDAP}, visible to the outside
world via google and other search engines.  One conclusion was
that ``Best Practices'' documents should be made available to aid
providers.

\ac{GIS} users form a vast community of communities that access
and analyze a myriad of data, but who, as of the present,
cannot access data served by \ac{OPeNDAP}.  A large effort
to create a viable, available \ac{OPeNDAP} to \ac{GIS}
connector has been unsuccessful so far.  Therefore a breakout
was held to address this issue.  The result was a few
initiatives among the participants to remedy the situation,
by various means.

Finally, the pure \ac{OPeNDAP} users presented several examples
of \ac{OPeNDAP} servers in use, tests of those servers, 
integration of \ac{OPeNDAP} into other systems, and the stages
of development of systems and projects that use or will use
\ac{OPeNDAP}.

The meeting was a good mix of presentations of the order of
20 minutes each, and breakout discussions of the order of one
or two hours each, which lead to a clear sense of participation
by all, and most with a sense of their individual work being 
significant contributions toward the progress and goals of the
whole \ac{OPeNDAP} endeavor.  The result was a feeling that the
project is vibrant and growing;  the current software is being
used extensively and successfully, and the discussions point
to major improvements that will enhance the use of \ac{OPeNDAP}
by high performance computer, modelers, users of in-situ data,
and users who need to make sure their data conform to uniform
standards of their own design and/or choosing.

% $Id$
%
% $Log: meetingsummary.tex,v $
% Revision 1.2  2003/05/05 16:40:45  tom
% edited, LaTeX-ized
%
% Revision 1.1  2003/04/29 14:09:43  paul
% Starting OPeNDAP Technical Working Conference 2003 Report
%
%

%%% Local Variables: 
%%% mode: latex
%%% TeX-master: t
%%% End: 
