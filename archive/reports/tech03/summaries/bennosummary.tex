Benno Blumenthal was not able to attend the meeting due to the
blizzard.  However, his presentation is available on the web.  A
mirror of the \ac{IRI} Data Library has been set up using
\ac{DODS}/\ac{OPeNDAP} and \ac{THREDDS} to transfer the data,
metadata, and collection structure between the two sites.  Standard
\ac{HTTP} software is used to cache the metadata, and custom code to
incrementally cache the data on the mirror site. Once the code is more
fully developed, a similar system should be able to cache from the
entire collection of \ac{THREDDS} servers.

The mirroring scheme has three key parts: a server, a client, and a cache. 

\begin{enumerate}
  \item Server - serves datasets using \ac{THREDDS} and \ac{DODS} 
  \item Client - views, manipulates, and re-serves a \ac{THREDDS}/\ac{DODS} 
        dataset collection 
  \item Cache - maintains a local copy of the information 
\end{enumerate}

Suggestions - Our experience leads to some suggestions for both the 
\ac{THREDDS} and \ac{DODS}/\ac{OPeNDAP} standards. 

\begin{enumerate}

  \item Proper HTTP Headers

    \begin{itemize} 
    \item XML files and DODS/OPeNDAP responses should have headers so
      that they are properly cachable. This includes at a minimum the
      Last-Modified header, but could also require
    \item \lit{Cache-Control:} public lines and \lit{Vary:} specifications
      depending on the behavior of the server. We, for example,
      include \lit{Vary:} Authorization lines on pages that are derived from
      password-protected datasets.
    \end{itemize}
    
  \item \lit{Content-Length} also helps increase reliability: caches
    will not cache responses that do not match their
    \lit{Content-Length} specification.

     \begin{itemize}
       
     \item Most http servers serve normal files with last-modified
       tags; those servers require cgi scripts to set those header
       lines if the pages are to be cachable.
       
     \item Ingrid does display the last-modified information, which
       can be helpful in checking a given collection or dataset.

     \end{itemize}

     There are, of course, other ways of looking at the HTTP headers
     to make sure that ones servers are delivering last-modified tags
     on a given WWW response.

  \item DODS Request Size Negotiation

     \begin{itemize}
       
     \item One way to get good transfer rates is to ask for larger
       pieces: this is a pure win for servers that stream, and even
       for servers that process in a single chunk, the ideal size
       could very well be larger than a single lat/lon slice. Short of
       the client trying a whole bunch of sizes and keeping track of
       the results, there is no good way to figure out the optimal
       size. And it is easy to end up with a server that has
       ill-defined behavior when the request is too large, the classic
       response being an error message inserted in the data stream.
       
     \item The classic C-behavior where the client asks for as much as
       it wants and the server returns as much as it can has a certain
       grace-and-style.  It is not entirely clear whether we can achieve the
       same.

        \item Given gigabit ethernet, can we really stick with a 2GB 
          limit on the size of a single request? 
      \end{itemize}

  \item global aliases
     \begin{itemize}
       
     \item THREDDS has an interesting ability to have multiple OPeNDAP
       servers for a particular dataset. This means that a server that
       is re-serving a dataset could make that particularly clear by
       also marking the dataset by including an attribute containing
       the URL of the original DODS server. There are a few cases
       where one might want to carry this farther:
       
     \item If one has picked out one variable from a much larger
       dataset (e.g. the best-estimate from a dataset which also
       includes number-of-observations, std-dev, smoothed, unsmoothed
       version), it would be nice if that relationship could be
       indicated as well. If only some of the metadata has changed, it
       would be nice if the client could figure out that the data
       itself does not need to be recopied.
     \end{itemize}
     
   \item Literature references in XML.  Frequently (one hopes) the
     dataset metadata includes literature references. There must be
     one or more XML standards for transmitting such information: it
     would be great if we could pick and support one.
     
   \item Visualization metadata.  Some visualization metadata should
     get transmitted with the dataset, particularly preferred
     colorscales. At the moment, we have a list of named colorscales
     and carry the name across, but we would prefer to be able to
     describe an arbitrary colorscale.  My preference would be to
     transmit this as a specialized DODS dataset, with the independent
     variable corresponding to the data values and the dependent
     variable(s) giving the color values. This would be one example of
     an attribute being a reference to (another) DODS
     dataset/variable.

  \item Short and long names for datasets

    \begin{itemize}
       
    \item Language-based clients can make good use of short as well as
      more complete descriptions of datasets. THREDDS should
      facilitate that.
      
    \item For example, the CDC dataset that I used as an earlier
      example is represented in Ingrid as


\begin{vcode}{bs} 
 THREDDS
  (Public Climate Data from the NOAA-CIRES Climate Diagnostics Center) @@
  .CPC\_.25x.25\_Daily\_US\_UNIFIED\_Precipitation
  (Monthly Accumulated Precipitation) @@
\end{vcode}

     \item and the dataset that I read via THREDDS from the Data
     Library is  

\begin{vcode}{bs} 
 THREDDS
  (IRI/LDEO Climate Data Library) @@
  .NOAA .NCEP .EMC .CMB .GLOBAL .Reyn\_SmithOIv2 .weekly .ssta
\end{vcode}
     
   \item While the long names are good for display, it is very useful
     to have short unique names that can be used to concisely refer to
     the datasets on a server or a server in a collection of servers.
     We all concoct these short names for internal use: THREDDS should
     let us share them with each other. Something as simple as having
     both \lit{name} and \lit{long\_name} in the standard with only
     \lit{name} required would suffice to allow data providers to
     share their short names. 


   \end{itemize}

 \item  Arrays of bytes vs. strings

   
   Some servers translate NetCDF arrays of byte data into DODS arrays
   of one character strings. Easier for client writers if you leave
   them as byte arrays or---better yet---translate them to multi-character
   strings.  If you do not translate to multi-character strings, the
   client has to figure out that it needs to translate to
   multi-character strings, and the client does not even know that the
   data came from NetCDF files in the first place, i.e. it has an even
   harder problem than the NetCDF-to-DODS server did.

\end{enumerate}

% $Id$
%
% $Log: bennosummary.tex,v $
% Revision 1.2  2003/05/05 16:40:45  tom
% edited, LaTeX-ized
%
% Revision 1.1  2003/04/29 14:09:01  paul
% Starting OPeNDAP Technical Working Conference 2003 Report
%
%

%%% Local Variables: 
%%% mode: latex
%%% TeX-master: t
%%% End: 
