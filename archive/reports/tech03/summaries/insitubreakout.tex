(Led by: Joe Wielgosz, who also took notes.)

Overall goal:   Find consensus on a convention for representing in-situ 
data in OPeNDAP, to facilitate development of interoperable clients and 
servers

\subsubsection{Attendees and individual goals:}

\paragraph{Project PI:}

\begin{description}
\item[Peter Cornillon / OPeNDAP]
        Address in-situ issues as a primary goal for OPeNDAP. Support mooring 
data collections as an initial step.
\end{description}

\paragraph{Software group representatives:}

\begin{description}
\item[Joe Wielgosz / COLA (facilitator)] Support BUFR weather
  observations in GrADS-DODS Server, and provide GrADS clients with
  access to in-situ data using OPeNDAP
\item[Steve Hankin / NOAA-PMEL] Provide NetCDF-based clients with
  access to in-situ data using OPeNDAP
\item[Dan Holloway / OPeNDAP] Continue development of JGOFS server
  (?).
\item[Don Denbo / NOAA-PMEL] Integrate OPeNDAP into EPIC client and
  server software
\end{description}

\paragraph{Data provider representatives:}
\begin{description}
\item[Kevin Gomes / MBARI]
        Provide intranet and public access to instrument readings from MOOS 
(MBARI Ocean Observing System)
\item[Gang Yuan, Yingshuo Shen / U Hawaii-IPRC(ADPRC)] Provide access
  to data on APDRC EPIC server using OPeNDAP
\item[David White / USC-Baruch Institute] Provide access to various
  in-situ data, stored in NetCDF, including moored buoys, using
  OPeNDAP.
\item[don't have name / U Alaska-GEM] Provide access to
  "heterogeneous datasets with homogeneous fields" (various data
  collections, stored in NetCDF) using OPeNDAP don't have name / NODC
  Provide access to NODC complete holdings using OPeNDAP
\end{description}


\subsubsection{Notes:}

Discussion was based around powerpoint slides, attached.

\paragraph{Data types:}

\begin{itemize}
\item Most data are collections of linear series (ok to represent as
  two-level Sequence)
\item Some data may have additional dimension(s), e.g. current meters
  (M levels x N times for each instrument) - need more Sequence
  levels.
\end{itemize}

\paragraph{Data structure:}

\begin{itemize}
\item Field chosen for indexing/sorting is not always
  instrument/profile ID - time is primary index, for example, for
  synchronized wave meters (M instruments for each synchronized time)
\item Searchable header fields are usually identified ahead of time to
  allow caching/extraction; however header fields in general are not
  consistent
\item Cornillon emphasized goal of aggregating data from multiple
  centers (with assumption of different structures) into a single
  searchable collection - for example, unified interface to all
  available mooring data for US coastal waters
\end{itemize}

\paragraph{Data model:}

General response was positive. Refinement-based usage model matches
experience of most attendees. Some potential issues were raised:

\begin{itemize}
\item Searches crosswise to indexing may be very very slow (i.e. one
  point from each profile) - based on JGOFS experience - so could be
  dangerous to provide data model which allows such searches. But is
  this a data model issue or a server implementation issue?
\item Many in-situ archives include disparate data (few consistent
  variables or header fields) - however, OPeNDAP data model does not
  allow variation in structure between profiles
\end{itemize}        

\paragraph{Clients/servers}

\begin{itemize}
\item No particular response to mention of C, FORTRAN, GIS as
  potential client interfaces.
\item NetCDF translation would provide access to in-situ data in
  Ferret, ncdump, other tools with no extra development.
\item Don Denbo proposes a simplified Java API for in-situ data
  "hybrid" architecture - NetCDF files plus separate
  store/cache/database for header info - is extremely common.  need a
  server that can aggregate NetCDF in-situ data, accessing separate
  header info store for efficient searching
\end{itemize}

\paragraph{Next steps:}

\begin{itemize}
\item Joe - develop slides into a more formal document to be discussed, 
revised and posted online for reference.
\item OPeNDAP - consider developing NetCDF in-situ aggregation capability
\end{itemize}
                                
%
% $Log: insitubreakout.tex,v $
% Revision 1.2  2003/05/05 16:40:45  tom
% edited, LaTeX-ized
%
% Revision 1.1  2003/04/29 14:09:01  paul
% Starting OPeNDAP Technical Working Conference 2003 Report
%
%

%%% Local Variables: 
%%% mode: plain-tex
%%% TeX-master: t
%%% TeX-master: t
%%% End: 
