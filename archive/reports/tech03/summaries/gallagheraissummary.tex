James Gallagher presented an overview, possible architectures,
and prototype of the \ac{AIS}.  The premise is that people who
provide data will not, in general, also provide consistent and 
complete use metadata for those data.  The \ac{AIS} provides
a means to append the required metadata to data in a way 
that is transparent to the user making an OPeNDAP request.  
It also allows the addition of aliases to attributes and variables,
and it does not alter the original dataset.  The architectures
include merging objects and \ac{AIS} resources on the client 
or on the server.  The current \ac{AIS} prototype only implements
adding attributes.  James gave two examples, and an evaluation
of the prototype.

Conclusions:
\begin{enumerate}
  \item The \ac{AIS} needs more use for a complete evaluation.
  \item The \ac{AIS} is cumbersome while developing \ac{AIS} resources.
  \item The \ac{AIS} server built using enveloping requires cumbersome URLs
  \item The attribute-only implementation is limited.
\end{enumerate}

% $Id$
%
% $Log: gallagheraissummary.tex,v $
% Revision 1.2  2003/05/05 16:40:45  tom
% edited, LaTeX-ized
%
% Revision 1.1  2003/04/29 14:09:01  paul
% Starting OPeNDAP Technical Working Conference 2003 Report
%
%

%%% Local Variables: 
%%% mode: latex
%%% TeX-master: t
%%% End: 
