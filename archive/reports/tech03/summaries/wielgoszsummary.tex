Joe Wielgosz reviewed the basics of the \ac{GDS}. He then
described new capabilities implemented in release 1.2 including

\begin{itemize}
  \item in-situ data support
  \item the ability to perform analysis using remote and local
        datasets as input
  \item features designed to enhance:

    \begin{itemize}
      \item reliability
      \item scalability
      \item security
      \item ease of use
    \end{itemize}

\end{itemize}.

He reviewed some of the major users of \ac{GDS}.  He listed some 
expected enhancements under three categories:

\begin{enumerate}
  \item Continuing refinements
  \item Station data export
  \item Items under consideration.
\end{enumerate}

Finally he gave some thoughts on the direction \ac{OPeNDAP} might take
from a \ac{COLA} perspective, including the new \ac{DAP}, Anagram/Server
side Java, Analysis Protocol and interoperability, modelers' needs,
and \ac{ESG}.

% $Id$
%
% $Log: wielgoszsummary.tex,v $
% Revision 1.2  2003/05/05 16:40:45  tom
% edited, LaTeX-ized
%
% Revision 1.1  2003/04/29 14:09:01  paul
% Starting OPeNDAP Technical Working Conference 2003 Report
%
%

%%% Local Variables: 
%%% mode: latex
%%% TeX-master: t
%%% End: 
