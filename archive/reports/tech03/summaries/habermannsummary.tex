Ted Habermann reviewed a plan to enable GIS users to access
OPeNDAP data via \ac{geoTIFF} files.  The system will use
float-sized floating point values and WGS80 geo-referencing
information.  All pixels will represent ``area'' rather than ``point''
data. The system will correctly size and create geoTIFF files for
\ac{COARDS} (and \ac{CF}-1.0 compliant NetCDF files (2D only currently).

The current limitations are:

\begin{itemize}
  \item As of today does not slice in Z,
  \item Dose not work with constrained URLs,
  \item No mechanism other than constrained URLs to limit the size
        of a request,
  \item The usual problem of discovering and constraining an OPeNDAP
         URL.
\end{itemize}

All of these problems are being addressed.
The technical implementation was outlined.

Conclusion: Several avenues for connecting GIS tools to OPeNDAP-served
data are underway.  The intermediary of conversion to \ac{geoTIFF}
format seems to be the most promising at the moment.
% $Id$
%
% $Log: habermannsummary.tex,v $
% Revision 1.3  2003/05/05 16:40:45  tom
% edited, LaTeX-ized
%
% Revision 1.2  2003/04/29 14:50:23  tom
% fixed for LaTeX and hylerlatex
%
% Revision 1.1  2003/04/29 14:09:01  paul
% Starting OPeNDAP Technical Working Conference 2003 Report
%
%

%%% Local Variables: 
%%% mode: latex
%%% TeX-master: t
%%% End: 
