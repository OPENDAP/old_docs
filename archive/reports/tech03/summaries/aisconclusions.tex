The \ac{AIS} is an element of \ac{OPeNDAP} that is critical for making
varied and disparate datasets conform to community and individual
standards and conventions as required by specific applications, and
to insure machine-to-machine interoperability in some cases.  An example 
is the ingestion of observational data from satellite and in-situ instruments 
into climatological models, where the models require \ac{COARDS} units
and structure for the input data.  The current prototype adresses
only the addition and/or modification of attributes.  The full
\ac{AIS} must address the addition and/or modification of
variables as well.

From the Breakout: Conclusions and Directions

There is a parallel effort underway at Unidata connected with the development
of NCML (NetCDF Markup Language). John Caron et al. have built tools which
provide a way to edit the contents of a NetCDF file by editing the NCML that
describes that file. It is a system that is very similar to the AIS. The main
difference is that NCML is limited to NetCDF files. We decided that both
efforts should stay in touch so that if we can merge them at some future
time, we do. For the time being, these ideas are new enough that keeping two
separate efforts is probably a good idea.

An integral part of a system like the AIS is documentation about data source
conventions (e.g. COARDS) and how to make a data source fit within a
particular convention. We should plan on supporting people who write AIS
resources by, for example, providing validators so they can see if the
information they have added is at least \emph{structurally} correct WRT some
convention. Also, we should add Convention and Version information
[Attributes? jhrg] so that ambiguity is reduced; most conventions were not
developed with the idea that one data source (aka data set, aka file) might
support more than just one! While DTDs or Schemas are an obvious way to
validate the syntax of an XML document (and we should probably use than in
any validation tool we provide) we should \emph{not} require each XML document be
validated by client-side XML parsers.

There are existing sources of information about data sets such as the GCMD,
FGDC clearing houses, etc., that the AIS should be able to draw from. It
seems that the best strategy for using information from these sources is to
build gateways between them and the AIS.

Another key feature of the AIS is a way for users to upload AIS resources
once they have developed them. This point has come up ever since we started
talking about the AIS.

Other comments:

\begin{itemize}
\item Add an audit trail so that documents that are the result of the merge
  operations will say so (and so that the exact source of merged
  information can be determined).
\end{itemize}
  

% $Id$
%
% $Log: aisconclusions.tex,v $
% Revision 1.2  2003/05/05 16:40:45  tom
% edited, LaTeX-ized
%
% Revision 1.1  2003/04/29 14:09:01  paul
% Starting OPeNDAP Technical Working Conference 2003 Report
%
%

%%% Local Variables: 
%%% mode: latex
%%% TeX-master: t
%%% End: 
