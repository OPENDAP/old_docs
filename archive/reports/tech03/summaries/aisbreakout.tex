% $Id$
%
% $Log: aisbreakout.tex,v $
% Revision 1.2  2003/05/05 16:40:45  tom
% edited, LaTeX-ized
%
% Revision 1.1  2003/04/29 14:09:01  paul
% Starting OPeNDAP Technical Working Conference 2003 Report
%
%

(Session leader: James Gallagher, notes by: John Chamberlain.
Participants included: James Gallagher, John Chamberlain, John Caron,
Tom Sgouros, Jon Callahan, Nathan Potter, Ted Habermann, Peter
Cornillon, guy from North Carolina, Paul Hemenway.)

Here are the minutes of the Breakout session:

Caron: described NCML implementation, noting in particular that it was
entirely in Java and that sub-setting was not supported yet.

Gallagher: AIS implementation currently entirely in C++ in the DAP core

Caron: NCML and AIS are functionally the same so we should consider
taking the best ideas from both designs and come up with a common
framework; he expressed admiration for the databse / pattern matching
capability of the AIS

Sgouros: suggest feature of an audit trail; this would allow the user
to be able to determine which attributes were added and would allow
the user to know what the original attributes were; would go a long
way to allaying the anticipated resistance by some critics to allow
metadata to be redefined.

Callahan: think the AIS should support attributes of convention and
version (such as COARDS) so that user does not have to scan the
structure to know if it is COARDS-compliant; also, the AIS should have
good documentation with extensive examples so that users are not
defining metadata in a vacuum.

Gallagher: the plan is provide validators

Potter: must be able to support not just one convention attribute but a
list of supported conventions; am totally against the use of DTDs
[general agreement that DTDs are bad]

Gallagher: the validators might only validate specific things like
conventions and would not be done by DTD but by a user proactively
running an external program that would do the validation; in addition to
convention validators we could also have application validators such a
program to verify that a ddx was Matlab-compliant for example.

Ted:  My concern is that we integrate existing metadata repositories
such as GCMD (DIF), FGOC, DC.

Callahan: DIFs are search metadata, not data metadata. We can build in a
reference to existing ontologies.

Gallagher: My general plan is to implement outside ontologies by having
them go through a gateway which would convert them to a DDX-compatible
format which would then allow them to be integrated like any other AIS
information.

Ted: What about field-by-field queries where you specify a single
location by ...

Chamberlain:  xpath for example

Ted: Right, basically showing the location of a single field in the
ontology.

Gallagher: AIS isn't meant to be a query and response system. Basically
you have to merge an entire set of semantic information. You can't just
query one item.

Caron: [seconds Gallagher] The conversion/gateway mechanism makes sense
for external ontologies.

Callahan: Can NCML add additional complex structures [as opposed to
simple string attributes]?

Gallagher: AIS can support any metadata that can be expressed in DAP
form. For example, to include a GIF you could store it as a short array
(of bytes) or you could store a file path to the GIF as a string. You
can store complex structures in DAP Structures just like any other data.

Caron: One desired capability for the NCML/THREDDS is to allow a file to
be uploaded onto a receptive server; this could be done for AIS also to
allow users to store their metadata for use by others.

Gallagher: That is an important feature. We are also looking at
peer-to-peer mechanisms for sharing metadata files.

Chamberlain: The name "Ancillary Information Service" may be too vague.
What is the consensus on possible other names with more specific
meaning, for example using words "semantic", "definitional", "ontology"
etc? [general consensus is that more specific words receive political
backlash and the existing name is OK with everybody]

Caron/Cornillon:  discussion of C++/Java issue; conclusions were that it
is OK to have some redundancy such as dual solutions in C++ and Java and
that it would be difficult to use Java across the board.


%%% Local Variables: 
%%% mode: latex
%%% TeX-master: t
%%% End: 
