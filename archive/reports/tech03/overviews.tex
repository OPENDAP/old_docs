
\subsubsection{Peter Cornillon: The \ac{PI} Overview}

Peter Cornillon, the project \ac{PI}, gave an overview of the project,
including the current status, and a background overview for the
newcomers to the meeting.

\subsubsection{Steve Hankin:  Role of \ac{OPeNDAP} in \ac{IOOS}}

Steve Hankin, \ac{DMAC}, \ac{PI}, gave an overview of the needs of the
\ac{IOOS}, and the roll \ac{OPeNDAP} already plays in the plans for
\ac{IOOS}.  In particular, \ac{OPeNDAP} is the ``operational
component'' for moving gridded data within the system, and is a
``pilot component'' for in-situ data within the system.  The challenge
will be to bring all the varied components together to make the
\ac{DMAC} work to provide the infrastructure necessary to move data in
a smooth and timely fashion, to achieve the goals of IOOS.

Specific to the topics of this conference, IOOS needs to
be able to serve large quantities of in-situ data, which will
require aggregation of large collections, and the development of
efficient ways of accessing the varied formats for storing in-situ
data.  IOOS also needs a comprehensive semantic ocean data model
which will take advantage of the OPeNDAP discipline-neutral and
general design, but will need controlled vocabularies to insure
machine-to-machine interoperability.  Implementation will require
a powerful \ac{AIS} capability on top of the \ac{OPeNDAP} \ac{DAP} 
transport mechanism.

\subsubsection{Peter Fox:  The Use of \ac{OPeNDAP} in the \ac{ESG} Project}

Peter Fox presented an overview of the use of \ac{OPeNDAP} in the
\ac{ESG} project to make available very large distributed climate
datasets.  He reviewed the areas of development and the overall design
architecture, including the institutions responsible for each part.
Then he compared the older http-cgi based design with the new
\ac{OPeNDAP} design, which will be protocol-independent, allowing for
implementations on multiple network transfer protocols.

Peter also reviewed \ac{ESG} services, and ended with some graphic
examples of user interfaces.  The \ac{ESG} use of the \ac{DAP}
requires capabilities that are distinct from the original
implementation of the \ac{DAP} in C++ using HTTP.  Therefore, to make
\ac{ESG} use of the \ac{DAP} compatible with other implementations,
the new \ac{DAP} Specification must be completed so that future
implementations can conform to the new Specification.

% $Id$
%
% $Log: overviews.tex,v $
% Revision 1.2  2003/05/05 16:40:45  tom
% edited, LaTeX-ized
%
% Revision 1.1  2003/04/29 14:09:43  paul
% Starting OPeNDAP Technical Working Conference 2003 Report
%
%

%%% Local Variables: 
%%% mode: latex
%%% TeX-master: t
%%% End: 
