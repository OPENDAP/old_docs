%\documentstyle[12pt,html,psfig]{article}
%%
% hacked from art12.doc
% 
% makes 1 , 1.25, 1.25 and 1.25 in. margins on the top, left, right and
% bottom resp. when using \documentstyle[12pt,...]{article}
%
% jhrg 1/6/94

% SIDE MARGINS:
\if@twoside               % Values for two-sided printing:
   \oddsidemargin 0pt     %   Left margin on odd-numbered pages.
   \evensidemargin 38pt   %   Left margin on even-numbered pages.
   \marginparwidth 85pt   %   Width of marginal notes.
\else                     % Values for one-sided printing:
   \oddsidemargin 18.5pt  %   Note that \oddsidemargin = \evensidemargin
   \evensidemargin 18.5pt
   \marginparwidth 68pt 
\fi
\marginparsep 10pt        % Horizontal space between outer margin and 
                          % marginal note
 
 
% VERTICAL SPACING:        
\topmargin 0pt            %    Nominal distance from top of page to top
                          %    of box containing running head.
\headheight 0pt           %    Height of box containing running head.
\headsep 0pt              %    Space between running head and text.
\topskip = 12pt           %    '\baselineskip' for first line of page.

\textheight = 635pt % 36\baselineskip
\advance\textheight by \topskip
\textwidth 435pt         % Width of text line.

% Adding this fixes the top and bottom margin sizes. jhrg

\textheight 8.75in

\newcommand{\postscript}[2]{
        \par
        \hbox{
                \vbox to #1{
                        \vfil
                        \special{ps: plotfile #2.ps}
                }
        }
}







%\begin{document}              
\Large
\begin{center}
{\bf Distributed Oceanographic Data System\\
Workshop\\
\large
September 29 - October 1, 1993\\
W. Alton Jones Conference Center\\ 
University of Rhode Island}\\
\end{center}
\begin{latexonly}
\vskip4pt\hrule height4pt\vskip4pt\hrule height2pt\smallskip
\vskip .35in
\end{latexonly}
\centerline {\bf Day 1:}
\normalsize

\vskip .25in
\begin{center}
 {Introduction by Dr. Cornillon at 8:50}
\end{center}
\medskip
Visions of "Utopia", a perfect oceanographic data system, which will 
then be limited by some of the realities. 
 
I need access to various data sets to design an algorithm to estimate 
mixed layer depths globally from scatterometer winds and AVHRR-derived 
day-night SST differences.

I will need other data:
mooring data with good vertical resolution to help design the algorithm
XBT/CTD data to help validate the algorithm

will also need the data to be coincident with scatterometer data and clear 
AVHRR day and preceding or night fields

have the best geographic coverage possible
Such data does now exist at NODC, WHOI, JPL , URI and other sites

We envision a system accessed from our workstation that will
allow us to locate and acquire these data sets including those held on our 
site quickly and easily

provide these data to our favorite analysis package in a format that the 
package recognizes.

We envision others accessing our data the same way (reducing the load of 
providing data to others that is currently an  impediment to data 
exchange)
\bigskip

\noindent {\bf Assumptions:}

The system will: 
\begin{itemize}
	\item focus on oceanographic data (Schramm- does this include met
data?)  
	\item be designed to serve the researcher
	\item be distributed (available to any data sites)
	\item based on a client-server model
	\item be dynamic and changing daily
	\item be easy to install (approximately 1 day for systems programmer)
	\item contain a variety of different data sets and data types
	\item not support distributed processing (at least not designed with
this as a requirement.
	\item upwardly compatible
\end{itemize}

Question about acceptable time delay in locating and receiving data.\\
\smallskip
{\bf B. Douglas -} It's easier to locate a large data set than a highly
delimited small data base, for these are "well supported".\\
\smallskip
{\bf G. Flierl -} question about "install times"\\ 
\smallskip
{\bf E. Dobinson -} two other assumptions need to be added:
\begin{description}
	\item{1)} do we assume the user knows what data he/she wants?  

\smallskip
	{\bf P. Cornillon -} we will need a "front end" director to provide a
location function
\smallskip
	\item{2)} second assumption - is it always better to move GB of data
around, or do we move the request to the site where this data is stored and
do the processing on-site (move the request to the data site rather that the
data to the request site)
\end{description}

\medskip
{\bf W. Schramm -} if your data are stored at multiple sites, this would be
very hard. 

\medskip
{\bf D. Glover -} Distributed Computing Environment - actual processing 
location should be transparent, system will optimize.  Do we want to 
consider such as system?\\
{\bf P. Cornillon -} We will restrict this morning's discussion to more 
traditional move request to data.

\medskip
{\bf H. Debaugh -} As long as you get your data, you don't care where you get 
your data from - it's much more complex to move your "program" to an 
unknown processing environment.

\smallskip
{\bf J. Corbin -} we should not preclude the ability to do distributed
processing but our system should not require it.\\
{\bf P. Cornillon -} Lets assume for the near term discussion we do not have 
distributed processing---
\smallskip
The Client/Server architecture allows flexibility in
\begin{itemize}
	\item The data formats served the oceanographic data systems may be
linked to other disciplines data systems in the future 

	\item The user interface other user groups can build their own
interface to the system (e.g. school teachers) 
\end{itemize}

\noindent {\bf The workshop series:}
\begin{itemize}
	\item Objective - {\bf To develop the base for our distributed
oceanographic data system}
	\item Two workshops 
    	   \begin{itemize}
		\item Workshop \#1 (now) - communications
		\item Workshop \#2 (summer 1994) servers
    	   \end{itemize}
	\item Fourth TOS meeting - spring 1995 the rest of the world
	(systems programmer to implement the system designed in workshop 
	\#1 and to help data providers build servers)

	\item Coordinated by TOS (not a URI of JGOFS effort!)
	\item Steered by a steering committee
     	   \begin{itemize}
		\item Glenn Flierl - MIT
		\item Ken McDonald - NASA
		\item Jim Holbrook - NOAA
		\item Peter Cornillon - URI
     	   \end{itemize}
\end{itemize}

\medskip
\noindent Question about use and value of metadata\\

{\bf W. Brown -} should capacity to have/use metadata be an assumption?\\
{\bf P. Cornillon -} metadata varies so much in descriptions, ranging from
what it is to how it was calculated.  Issues of what the metadata is are very 
important.

\medskip
{\bf R. Wilson -} it is very important to have a browse capability of the
host.  This lets you verify the suitability and applicability of the data.\\
{\bf P. Cornillon -} data browse will be an important requirement in our
system.  This will be an internet accessible system.

\medskip
{\bf B. Douglas -} question about data a PI is still working on and has not 
published. What happens when this data loses its identity in the system - 
since his career depends on proper recognition.  How will originators get 
credit for participating in this system?\\
{\bf P. Cornillon -} we will have a "voluntary" system.  We want to have
people use it because our system will help them. We're seeing a real change in 
how oceanographers make their data available.\\
{\bf B. Douglas -} reality says we must have citations of how much our data is 
being used, and by whom.

\medskip
{\bf B. Schramm -} researchers need access to "operational" Navy and NWS data, 
yet we have constraints on who in the international community will have 
free access to this data.\\
{\bf P. Cornillon -} we don't want "password" protection...

\medskip
\noindent back to introduction...

This workshop is sponsored by TOS, not URI...

Time line viewgraph of workshop sequencing, with the goal of eventually 
developing our prototype client:

\begin{center}
 {\bf Timeline Figure}
\end{center}
\bigskip
\noindent Workshop \#1 Communication Objective 
\begin{itemize}
	\item Define the architecture of the System form of the Client/Server 
System
    	   \begin{itemize}
		\item focus on the first day
    	   \end{itemize}

	\item Choose the Communications Protocol for Data Objects passed 
between clients and servers - 
    	   \begin{itemize}
		\item focus of the second day 
    	   \end{itemize}
\end{itemize}
\smallskip
{\bf E. Dobinson -} without rank-ordered system requirements from the 
scientists, its hard to come up with a desired system architecture.\\
{\bf P. Cornillon -} how the different architectures scale will be an issue.\\
{\bf R. Chinman -} isn't agenda for "data providers" much wider than just
"what data will be available"?\\
{\bf P. Cornillon -} We will let breakout groups define their own agendas.

\bigskip
\noindent {\bf G. Flierl:  Discussion of Distributed Systems Architectures}

\medskip
Where do existing systems fit in the context of the models we have discussed?
(Slides are included in notes) 
\begin{description}
	\item{Slide 1:}  Conventional Data Base Management System --

	   \begin{itemize} 
		\item Data held in a few formats specific to that system.
		\item Multiple data formats pose problems.
	   \end{itemize}

\begin{figure}[h]
\centerline{\psfig{figure=slide1.xfig.ps,height=5.0in}}
\end{figure}

	\item{Slide 2:}  Client-Server model -- Moves some of the processing
functions to the client.  No distinction between "data" and "metadata" . 
\begin{figure}[h]
\centerline{\psfig{figure=slide2.xfig.ps,height=5.0in}}
\end{figure}
\clearpage

	\item{Slide 3:} Client-Servers  (One client supporting multiple
servers,  with possible multiple protocols) 
\begin{figure}[h]
\centerline{\psfig{figure=slide3.xfig.ps,height=5.0in}}
\end{figure}
\clearpage

	\item{Slide 4:} Multiple Clients and one server; clients at different
levels of complexity (e.g. PI, industry, educational)
\begin{figure}[h]
\centerline{\psfig{figure=slide4.xfig.ps,height=5.0in}}
\end{figure}
\clearpage

	\item{Slide 5:} Multiple Clients and Multiple Servers.  Servers must
interpret queries and retrieve data.  
\begin{figure}[h]
\centerline{\psfig{figure=slide5.xfig.ps,height=5.0in}}
\end{figure}


\smallskip
{\bf R. Chinman -} if master directory acts as repository of data, is it
serving as a  layer between two clients?   
Issues - locating information; clients/local dictionary; master dictionary 
server; dictionaries in servers; polling

\smallskip
{\bf H. Debaugh -} can we have an additional "open" server between clients.  
(will be addressed in subsequent slide)
\clearpage
\item{Slide 6:}
	   \begin{itemize}
		\item Types of Queries --
	  	   \begin{itemize}
			\item sub-selections
			\item other SQL functions
			\item incremental requests
	  	   \end{itemize}

		\item Types of Responses--
	  	   \begin{itemize}
			\item documentation of data set
			\item identifiers
			\item attributes
			\item values/types (real numbers, integers..)
			\item structure (most oceanographic data is not highly
structured) 
   	   	   \end{itemize}

\begin{figure}[h]
\centerline{\psfig{figure=slide6.xfig.ps,height=5.0in}}
\end{figure}
\clearpage
	   \end{itemize}
\item{Slide 7:} Data Objects -- Object Oriented Programming overview,
	\begin{itemize}
		\item program is built up of objects
		\item encapsulation (data hiding - you do not see internal
information) 
		\item polymorphism
		\item different objects accept same message but react
appropriately and differently

		\item Inheritance -- you can have a "parent" class with its
own methods and data, and a sub-class with less capability.  If the lower
level cannot process the request, it will be passed up to the "parent" for
processing.  
	\end{itemize}

\begin{figure}[h]
\centerline{\psfig{figure=slide7.xfig.ps,height=5.0in}}
\end{figure}
\clearpage

\item{Slide 8:} Object Oriented Data Bases (OODB) --
	\begin{itemize}
		\item encapsulation
		\item polymorphism
		\item inheritance
		\item nested structure (inheritance)
		\item query optimization
		\item extensible
		\item new data types
		\item new operations
	\end{itemize}

{\em The vision which we are looking for is to define an appropriate 
combinations of the "clients-servers" structure in an OODB environment.}
\normalsize

\begin{figure}[h]
\centerline{\psfig{figure=slide8.xfig.ps,height=5.0in}}
\end{figure}
\clearpage
\item{Slide 9:} Other Aspects--
	\begin{description}
		\item{I.} Pre- and post- filters  (data compression)

		\item{II.} Telephone Operators - one object may talk to
multiple objects or sets to get the information out.  Standard way to get
data if you don't know where it is would be to call up master directory
which will locate the data you want.  Remote directory may update your local
directory once the data are located. - keeping local system current will be a
challenge, depending on frequency of use.
		\end{description}

\begin{figure}[h]
\centerline{\psfig{figure=slide9.xfig.ps,height=5.0in}}
\end{figure}
\clearpage
\end{description}

\medskip
{\bf H. Debaugh -} concern about data server doing compression and other tasks 
may cause problems.  There may be other machines/programs in the link.

\medskip
{\bf B. Schramm -} a variety of providers will be available.  Do we want to
have  different "classes" of servers instead of trying to put them all in one 
"box"?  

\medskip
{\bf G. McConaughy -} are "server" and "object" the same thing?\\
{\bf G. Flierl -} Yes,  at this point in discussion.

\medskip
{\bf H. Debaugh} presented viewgraph interpretation of open server. (Slide 6 
included in notes)

\medskip
{\bf G. McConaughy -} "scalability" is an issue of concern.  We want to keep 
knowledge close to the data.  "Intelligent" servers,  "minimally-
intelligent" routers, and "stupid" clients....

\medskip
{\bf T. Kelly -} open server lets you run more efficiently.  The open server only 
has to order data once, and can then redistribute the data multiple times.

\medskip
{\bf D. Collins -} the open server provides an opportunity to bypass the 
"bottleneck" once you have established a repeated need for similar data.  
The "left" direct arrow on the diagram can be used then.

\medskip
{\bf N. Soreide -} what is in the open server? Does it have metadata, or is it 
just a "traffic cop"?\\
{\bf G. Flierl -} Specifically, it would know where the 
data is, and would be able to respond to some data directly, but would 
have to go out and get other data.

\medskip
Three important functions:{\bf  Location, Routing, and Caching}

\medskip
{\bf J. Gallagher -} maintaining open server with small number of formats on 
the bottom will be easy, but once the number of formats is large this will 
be hard.

\medskip
(back to Flierl's slides)
\begin{description}
     \begin{description}
	\item{III.} Feedback, results from previous queries used to
constrain/optimize current query 
	\item{IV.} Hypertext
      \end{description}

\item{} Issues:
     \begin{itemize}
	\item which elements?
	\item directory protocol
	\item updates
	\item long-term stability 
	\item redundancy
	     \begin{itemize}
		\item versions
		\item updates
	     \end{itemize}
	\item protection/privilege
	\item credit for data sets
     \end{itemize}
\end{description}

\medskip
{\bf R. Chinman -} what happens when an oceanographer takes their workstation 
with them - can be an issue for stability of a data base which may reside 
on that system.

\medskip
{\bf D. Fulker -} if result of a query gives you location of many different 
versions of the same data, then there is a problem.  Must uniquely identify 
data sets.

\medskip
{\bf B. Douglas -} AGU has issued requirements to clearly identify data cited
in research, in a manner such that another researcher could uniquely access 
it. 

\medskip
{\bf B. Douglas -} "30-year" rule - will the data still have value in 30
years.

\medskip
{\bf W. Brown -} can't archived data be a distributed set of centers?

\medskip
{\bf G. McConaughy -} offered software (EOSDIS Version 0) as a base for the 
system we are trying to develop.

\medskip
{\bf E. Dobinson -} can we get a JGOFS overview?\\
{\bf P. Cornillon -} JGOFS is just an approach we're looking at.  It is not
the model for the system we're developing.  JGOFS requires a UNIX
workstation, there is no automatic update. There are many things about JGOFS
which should be improved. 

\medskip
{\bf B. Schramm -} CD ROMs are becoming a common data format, and would be 
especially useful for maintaining "local" data.  Can this group foster a 
format for CD ROMs?  

\medskip
{\bf J. Corbin -} Do we envision unrestricted access and no data charges for
this system?\\
{\bf P. Cornillon -} it's an issue we need to address, but one which is 
fairly easy to look at.

\medskip
{\bf L. Walstad -} there must be some restrictions so that someone doesn't ask 
to download a 30 GB data set on internet.\\
{\bf P. Cornillon -} X-Browse system allows certain access to the data for
free, but you can't take the entire image archive and download it.  The
objective is to limit how long the line is used by one person.	

\begin{latexonly}
\newpage
\end{latexonly}
\Large
\begin{center}
{\bf Afternoon session}
\end{center}
\normalsize
\vskip .25in
Broke into working groups for the afternoon

\medskip
\noindent Working Groups:
\begin{itemize}
\item System Designers
\item Data providers
\item Data Users/Scientists 
\item Data Objects
\end{itemize}
\medskip
\vskip .25in
\Large
\begin{center}
{\bf Evening Session}
\end{center}
\begin{center}
{\bf Presentations by working groups}
\end{center}
\normalsize

\medskip
\begin{center}
{\bf Data Users Working Group}
\end{center}
\begin{center}
 {W. Brown-- Chairman}
\end{center}


\medskip
\underline{Data User Requirements}

\medskip
\begin{center}
 {\bf "Data" - primary "observations" and information}
\end{center}
   \begin{description}
	\item{(O)} {\em Must provide access to both investigator and
archive-held data}
	\item{(1)} Data Selection
	   \begin{itemize}
		\item Ability to "locate" specified data in terms of:
parameter, time-space window, type, source, QUERIES 

		\item Refinement of search
		\item Coincident parameter search
		\item Flexibility to handle "unknown" parameter selections
	   \end{itemize}
	\item{(2)} Data Acquisition
	   \begin{itemize}
		\item Timeliness - electronic and/or mail delivery possible
		\item "Useful" Form - interactive and/or batch access
		\item Interface with simple programming languages
		\item Different formats
		\item Data subset selection possible
	   \end{itemize}
	\item{(3)} Redundant Data Screening

	\item{(4)} Data transfer protocol(s)-- Investigator to Archive
\end{description}
\medskip
\begin{center}
{----------------------------}
\end{center}
\vskip .25in
\begin{center}
{\bf Data Providers Working Group}
\end{center}
\begin{center}
 {R. Chinman--Chairman}
\end{center}

\medskip
\noindent Representation of the group:
\begin{itemize}
	\item NOA-NOS, FNOC
	\item NODC
	\item NOAA-NOS (Ocean, Lake Level Div)
	\item PODAAC
	\item PI
	\item NOAA-NESDIS (DMSP, ERSY, METEOSAT, GMS)
	\item UNOLS R/V Tech
	\item Global Change Data Center GDAAC
	\item TOGA COARE
\end{itemize}
Dataset Classification:
\begin{itemize}
	\item Open-ended
	\item Project (Closed)
	\item Rotating
	\item Orphan (incidental data to another study)
	\item Real-time
\end{itemize}

Providers - PI\\
Users -	Data Center\\
			repetitive data distribution tool
			data discovery tool (e.g. sociologists)

The types and kinds of data and providers and users {\em suggest/requires a 
relatively wide range of options/capabilities for distributing data}

What data providers need from or will do for the distributed ocean data 
system:
\begin{itemize}
   	\item Data for a wider community than ocean community alone
   	\item Internet based
	\item Generation and distribution of metadata critical!! including
citation info , algorithms, when and where data collected, data set version
caveats 
	\item MD-like locator needed for this system
	\item DIF-like metadata file needed for the locator, for the data
file to be findable on the system
	\item Provider-specific capabilities for instituting restrictions,
privileges, protection of datasets including log of users and activities
	\item Distribution of processing capabilities restricted to data
format translation and subsetting
\end{itemize}
\medskip
Discussion about data sets including appropriate credit for the PI who 
provided the data to the data center, and the requirement that a second 
investigator can obtain the exact same data set, apply the stated model or 
algorithm to it, and derive the same result.


\medskip
{\bf P. Cornillon -} MD is not populated as much as it should/could be because
of the excessive documentation requirements for data sets 
\medskip
\begin{itemize}
	\item Ease of installation - higher overhead of server installation
acceptable, but ease still important

	\item Data format translators necessary, with at least the following
translations: 
	   \begin{itemize}
		\item GRIB/BUFR
		\item HDF
		\item net CDF
	   \end{itemize}
\end{itemize}

\medskip
\begin{center}
{----------------------------}
\end{center}
\vskip .25in
\begin{center}
{\bf System Architecture Working Group}
\end{center}
\begin{center}
 {E. Dobinson-- Chairwoman}
\end{center}

\medskip
Basic Constraints
\begin{description}
	\item{1)} Cost - reusable code, functions,  systems
	\item{2)} Schedule - 9 working months to build system
	\item{3)} Extensibility (grow with time) and Scalability (start small
then grow dynamically)
	\item{4)} Simple to use, easy to install, easy to use
	\item{5)} Dynamic - easy to grow and administrate
\end{description}
\smallskip
Given these constraints, what is essential functionality?
\begin{itemize}
	\item Location and Search function primary - user must be able to
locate and find data (look at USGS system?)
	\item Need an order function to provide the user with the data he or
she wants..  
	\item Services a server can provide may vary, based on the data set
(BASIC CORE SYSTEM)
\end{itemize}
Discussion of Client/Server viewgraphs:

\medskip
\begin{center}
{\bf End of Day 1}
\end{center}
\bigskip
\bigskip
\begin{latexonly}
\vskip4pt\hrule height2pt\smallskip
\end{latexonly}
\bigskip
\Large
\begin{center}
{\bf Day 2:}
\end{center}
\vskip .25in
\normalsize
\smallskip
\begin{center}
{\bf Data Objects Working Group}
\end{center}
\begin{center}
{G. McConaughy-- Chairwoman} 
\end{center}

Started with an overview of existing systems.
\begin{description}
	\item{} Search to data access is range of service for most existing
systems.  EOSDIS is mostly search, JGOFS much more focused on data access.
IDBMS mostly one large data base.  
				
\smallskip
{\bf B. Schramm -} net CDF (UniData), NEONS were left out.

	\item{} EOSDIS Version 0 assumption is that you're sitting at your
client, and it sends out a search message and gets a response message based
on who it is talking to.  None of these messages is visible to user.  It is
using ODL (Object Definition Language).  A lot of info isn't seen by client.

	\item{}MEDS/Gopher system - messages coming back are menus which are
data services at sites.  Inventory search and results and data order - you
get back a "form" to fill out with data order and processing options.  Some
software is provided in the client for looking at data.  It is less of a
search system and more of an ordering system.  Burden on data provider to
hook in would be very small.

	\item{} IDBMS has 9 nodes (ORACLE-based) with centralized node with
Master Catalog where data are replicated and also used for data ingest.  Each
server holds a different type of data, user does not even know which node
data is being accessed.  Search can provide either metadata or primary data -
similar to how EOSDIS works.

	\item{} JGOFS - client has access to servers through calls.  There is
a server and a data dictionary which has a list of objects, as well as a path
to tell how to execute the method.  There will be different entries in the
dictionary for data and methods.  Tagged with the data are metadata.  The
client program connects to an executable remote -- all of the dictionary
serves to make this happen.  Metadata can be searched to come up with data
objects that may satisfy the users.  Can list all variables which can be
extracted from the data - including looking at multiple servers to do this
based upon the entry in the dictionary.  Focus of system is to support a PI.
Every server has a data dictionary for all holdings.

\end{description}

\smallskip
{\bf R. Mairs -} how would/could JGOFS send a message to EOSDIS? \\
{\bf G. Flierl -} suppose you wanted SST within some latitude bounds, 
and you seek a yes/no answer about the data. This is very much the kind of 
information EOSDIS handles.

\smallskip
{\bf P. Cornillon -} discussion about "capability" of a system, at the
frequent expense of ease of use and menu-driven operations.  EOSDIS may be
the system designed for the broad range of users who desire ease of use,
while a system like JGOFS is "harder" to use but of higher value to a PI who
has more specific data requirements.

\smallskip
{\bf B. Starek -} discussion of "query by example" on IDBMS. ARCInfo keys to 
databases (Common Production Tools - like methods).  

\smallskip
{\bf P. Cornillon -} our system must provide a very basic level of easy-to-use 
services yet have expert capability to support science users.  A modified 
JGOFS may be a good starting point.

\smallskip
{\bf G. Flierl -} discussion of JGOFS "Problems":
\begin{description}
	\item{1)} Feedback (in "client"?)

	\item{2)} Limitations on query: $<$, $=$, $>$, $<=$,  $>=$, (  ), \&, $|$ or
\~ \quad but no more than 20 strings grouped together 

	\item{3)} Data types: all ASCII doesn't fully deal with
matrices/tensors 

	\item{4)} No interactive retrieval (like XBROWSE which lets you
"look" at an image as you begin to receive it and decide if it will meet your needs .  
\end{description}
\medskip

{\bf G. Flierl -} Summary of support requirements
\begin{description}
	\item{1)} Multiple Clients/User Interfaces
	   \begin{itemize}
		\item menus/hypertext \quad\quad        ...feedback
		\item Fortran programmers \quad\quad ...simple full query
specification
	   \end{itemize}

	\item {2)} Data servers
	   \begin{description}
		\item{a)} inventory and location
		\item{b)} browse
		\item{c)} data, including points to other data 
	   \end{description}
	\item{3)} Pre/Post filters

	\item{4)} Multiple Servers-- routing

	\item{5)} PIs adding servers-- 	support multiple DBMS
\end{description}
\smallskip
{\bf L. Walstad -} How will we handle requests for 2-3 GB of data?\\
{\bf G. Flierl -} Server will respond, but indicate the data block is too
big.  It may suggest transfer by mail (tape).  

\smallskip
{\bf P. Cornillon -} list of problems he has encountered in using systems:
One person wants "SST",  a second searches for "sea surface 
temperatures".  Variable names need standardization, and we need to make 
our locator object be "intelligent"


{\bf G. Flierl -} discussion of data structure:

\begin{center}
{\bf Hierarchical Data Structure Diagram} 
\end{center}
\medskip

\Large
\begin{center}
{\bf Afternoon session}
\end{center}
\normalsize

The workshop will not try and endorse a specific data format in the 
afternoon discussions - this question cannot be resolved in a workshop of 
this scale.  

We will break up into Users and Providers.  

Discussion:  How much of the definition of communication protocol must 
this group consider?  

Developers can send out a  bit stream from their servers in some format, 
and along with this is a structure (really a series of arguments) which 
defines this format.  On the other end, the user must be able to 
"understand" this structure and apply it.

We're confusing communications protocols with the applications 
interface.  

{\bf G. Flierl -}  API has two calls---
\begin{description}
	\item{} call and get variable names (strings, numbers)
	\item{} call and get values (strings, numbers)

\begin{center}
{\bf RPC vs Data Stream Diagram}
\end{center}

	           Top figure illustrates an RPC-based system while the lower 
		illustrates the data stream approach

\smallskip
{\bf N. Soreide -} is "data volume" the amount of data (MBs) or number of data 
requests?\\
\smallskip
{\bf G. Flierl -} A lot of research is being done via individual exchange of
small data sets -- not transfers of large volumes of data from archives.  It
is important the system support these "small" users...

\smallskip
{\bf J. Gallagher -} MB and GB per day is "large"; less than MB is "small"  
(although these definitions are more commonly based on how long it takes 
to receive the data, which is driven by communications technology...)
\medskip

\noindent Data Providers Subgroup Meeting   - R. Chinman 
\smallskip
Questions:  What Objects?  What Structure?
\smallskip
What is an object? - An object includes 
\begin{description}
	\item{a.}  inventory,
	\item{b.}  browse
	\item{c.}  data  
\end{description}
		(these are its attributes).  
\smallskip
To what extent are data providers willing to provide these?  
\smallskip
Inventory (for an image, for example) might be "is there an image?"
\smallskip
Browse - defined as a mechanism for sub-sampling.

\smallskip
{\bf B. Schramm -}- We need to look at two different cases.  The Data Center 
will be significantly different from the PI in its implementation of these 
capabilities.  


DATA:	standard file format \quad\quad location \quad\quad disk space

\Large
\begin{center}
{\bf Evening Session September 30}
\end{center}
\normalsize
\smallskip
{\bf P. Cornillon} presiding 

\noindent {\bf System Functions}
\begin{description} 
	\item{} Selection
	   \begin{itemize} 
		\item Search (space/time, parameters, source, sensor, other),
cross-inventory  
		\item Refine search - optional but necessary (browse,
quality, log) 
	   \end{itemize}
	\item {} Order 
	   \begin{itemize}
		\item Means of delivery and where
		\item Timelines
		\item Format
		\item Either from the selection process or directly (
bypassing selection process)	  
	   \end{itemize}
\noindent {\bf Principles}
	   \begin{description}
		\item Must provide access to PI data sets in addition to
national archives (corollary) --- people should want to use the system to
manage their own data
	   \end{description}
\end{description}

\noindent {\bf Data Issues}
\begin{description}
	\item{} Provider of the data must be able to determine what data
objects (and order) to be delivered

	\item{} Mechanism  to resolve keyword conflicts (auto detection)

	\item{} Global naming procedure for data sets

	\item{} Central clearing house - system management (log)

	\item{} We must avoid giving people a reason to NOT want to put their
data into the server
\end{description}
\noindent {\bf Messages}

\begin{description}
	\item{} Functional description of data, e.g.,  for location or
defined range of formats
	\item{} Options available from provider
	\item{} Must allow (and in some cases encourage) a description of
data set or provide a pointer to a description (search) -- example, the "Gulf
Stream Paths" data set.
	\item{} Log File - Data set PI invites comments on data sets - central
storage 
	\item{} Be able to explore the system
\end{description}


\noindent Discussion --


{\bf D. Fulker -} provision of data needs to be in a form suitable for use in
an applications program  (API)


{\bf Data Providers  - R. Chinman}
\begin{description}
	\item {}Desire to remove all impediments to using data

	\item{}Desirable functions and structure
\begin{description}
	\item{} inventory - locator
   	\item{} browse - refinement
	\item{} data ---{\em server}--- standard format
\end{description}
\noindent inventory:
	\begin{description}
		\item{} id
		\item{} latitude
		\item{} long
		\item{} date/time
		\item{} PI
		\item{} sensor
		\item{} parameters
	\end{description}
\end{description}
cannot translate all native format data into one of the standard formats - 
PIs nor Data Centers
So... use server software to do that and inventory (if necessary) 
Use existing inventories via server translators on new system.

\noindent {\bf Benefits}


{\bf PI:}\quad fulfillment of contractual obligation to make data available
to National Data Centers, \quad access to Data Center data and other PI data

{\bf Data Center:}\quad get access to field project/PI-based data for their
users 


From the Data Providers (PI and Data Centers) will come a range of 
services


Ocean community should prioritize DATASETS and tell J. Gallagher
\smallskip

Data Centers provide an existing array of services and fold 2-3 into new 
system, e.g., EOSDIS design group, Emery EOSDIS testbed, NOAA DAC  
(includes NODC)


{\bf E. Dobinson -} question about why this system (DODS)  seems initially 
focused on large existing systems (EOSDIS, MEDS,NEONS.....) which have or 
will have established mechanisms for getting their data, instead of 
looking at developing a system uniquely but not exclusively designed to 
gain access to small, non-automated datasets held by PIs .  Response from 
P. Cornillon centered on difficulty in becoming familiar with all of these 
different client/server system (or even knowing of them as they multiply) 
, making the data received from them suitable for use on your system, and 
having to wait until some of the big systems (e.g., EOSDIS (July 94)) are 
ready to become operational. 
\medskip
\begin{center}
{\bf End of Day 2}
\end{center}
\bigskip
\bigskip
\begin{latexonly}
\vskip4pt\hrule height2pt\smallskip

\end{latexonly}\bigskip
\Large
\begin{center}
{\bf Day 3}
\end{center}
\begin{center}
{\bf Conclusions}
\end{center}
\normalsize
\medskip
{\bf G. Flierl:}

\noindent Next Steps:
\begin{description}
	\item{0)}  Meeting report(s); mailing lists and telemail

	\item{1)} Architecture Strawman / Message Strawman
	   \begin{itemize}
		\item tell Gallagher
		\item dialog or draft
		\item join the development  group
	   \end{itemize}
	\item{2)} Testbeds
	   \begin{itemize}
		\item provide data/ work with URI/MIT
		\item provide support (moral and personnel)
	   \end{itemize}
	\item{3)} Next Workshop (planned elsewhere than Alton?)
	   \begin{itemize}
		\item focus on people who want to develop servers and clients
	   \end{itemize}
	\item{4)} Colleagues
\end{description}
\medskip

Software development will be undertaken in a phased approach. (Slide 10)
\smallskip
\begin{figure}[h]
\centerline{\psfig{figure=slide10.xfig.ps,height=5.0in}}

\end{figure}
\clearpage
{\bf R. Chinman -} Is the name "DODS" appropriate? It focuses on oceans, yet 
there is a significant community outside "oceans" who may have interest 
in this work.

\smallskip
{\bf G. McConaughy -} there may be an advantage to keeping it as an "oceans" 
system, for if we broaden its basis (as implied by a more encompassing 
name) it may sound like it will try and do things which are already being 
done.

\smallskip
Discussion and refinement of system graphic (Slide 11).
\smallskip

\begin{figure}[h]
\centerline{\psfig{figure=slide11.xfig.ps,height=5.0in}}
\end{figure}
\clearpage

Is there a base level communication procedure?  Will there always be a 
match?
\begin{description}
	\item{1)} May not be efficient
	   \begin{itemize}
		\item ASCII
		\item XDR binary
	   \end{itemize}

	\item{2)} Must be buildable from DODS distribution
\end{description}


\smallskip
{\bf P. Cornillon -}  One implementation option might be to just put an 
additional server on a site.  We didn't consider this as much as we perhaps 
should have.

\smallskip
{\bf D. Fulker -} discussed his concerns about locating translators on the 
"client" side.


The meeting concluded in general agreement of the system goals and 
functionality.  

\begin{center}
{\bf End of Day 3}
\end{center}
\begin{latexonly}
\vskip4pt\hrule height2pt\smallskip
\end{latexonly}
\end{description}
%\end{document}