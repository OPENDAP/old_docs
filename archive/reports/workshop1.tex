%
% This file contains the report of the first DODS Workshop.
%
% $Id$
% 
% From grf
%
% I changed - to -- so that it prints as a dash
%
% changed text - sections marked by %grf In some cases, these were
% just to avoid ``he'' --- after all, we had a fair number of female
% participants.
% In the minutes:
% p2 l7 G. Flierl - question 
% p3 mid looses <- loses
% p5 near bottom  polymorphism [one word]
% p6 mid H. Debaugh - concern about data server doing compression ...
% p8 mid but you can't take the entire image archive and download
% it...
% p8 bottom Timliness <- Timeliness
% p9 bottom finable <- findable
% p10 bottom Need an order function ...
%            Services a server can provide may vary, ...
% p12 middle all ASCII doesn't fully deal with matrices/tensors
% p14 figure needs caption - suggest
%   Top figure illustrates an RPC-based system while the lower
%   illustrates the data-stream aaproach
% p15 top differernt <- different
% p17 bottom peoplewho <- people who
% p18 - perhaps delete the reference to minutes being distributed the
% next week!
% switch Slide 10 and Slide 11 ?
%
%
% jhrg:
% I merged my changes with Glenn's. 
% I made all the hyphens, en-dashes, ... consistent with each-other and with
% Knuth's description in the TeXbook. `-' for hyphenated words, -- to
% separate ranges and --- for the sentence punctuation symbol.
%
% ********* To switch from \input to \postscript fig, change the comments
% where the figures are added (appendix III).
%
% 12/22/93:
% Fixed appendix numbering (used \appendix correctly)
% Fixed figure placement (well sort of...). The figures a;; appear together,
% but the first one is not at the bottom of the page as per grf's suggestion.

\documentstyle[12pt,html,psfig]{article}
%
% hacked from art12.doc
% 
% makes 1 , 1.25, 1.25 and 1.25 in. margins on the top, left, right and
% bottom resp. when using \documentstyle[12pt,...]{article}
%
% jhrg 1/6/94

% SIDE MARGINS:
\if@twoside               % Values for two-sided printing:
   \oddsidemargin 0pt     %   Left margin on odd-numbered pages.
   \evensidemargin 38pt   %   Left margin on even-numbered pages.
   \marginparwidth 85pt   %   Width of marginal notes.
\else                     % Values for one-sided printing:
   \oddsidemargin 18.5pt  %   Note that \oddsidemargin = \evensidemargin
   \evensidemargin 18.5pt
   \marginparwidth 68pt 
\fi
\marginparsep 10pt        % Horizontal space between outer margin and 
                          % marginal note
 
 
% VERTICAL SPACING:        
\topmargin 0pt            %    Nominal distance from top of page to top
                          %    of box containing running head.
\headheight 0pt           %    Height of box containing running head.
\headsep 0pt              %    Space between running head and text.
\topskip = 12pt           %    '\baselineskip' for first line of page.

\textheight = 635pt % 36\baselineskip
\advance\textheight by \topskip
\textwidth 435pt         % Width of text line.

% Adding this fixes the top and bottom margin sizes. jhrg

\textheight 8.75in

\newcommand{\postscript}[2]{
        \par
        \hbox{
                \vbox to #1{
                        \vfil
                        \special{ps: plotfile #2.ps}
                }
        }
}







% Adding this fixes the top and bottom margin sizes. jhrg

\textheight 8.75in

\newcommand{\postscript}[2]{
        \par
        \hbox{
                \vbox to #1{
                        \vfil
                        \special{ps: plotfile #2.ps}
                }
        }
}

\begin{document}
\pagenumbering{roman}

\begin{titlepage}

\begin{latexonly}

% Title page for hard copy

\vbox{}
\vskip 1in
\begin{center}
\begin{LARGE}
Report on the First Workshop for the 

\vskip 1ex
Distributed Oceanographic Data System
\end{LARGE}
\begin{large}
\vskip 1ex
29 September -- 1 October 1993

\vskip 1ex
W. Alton Jones Campus

\vskip 1ex
University of Rhode Island

\vskip 1ex
West Greenwich, Rhode Island
\end{large}
\end{center}

\vskip 3in
\hskip 3in
Edited by:

\hskip 3.25in
Peter Cornillon

\hskip 3.25in
Glenn Flierl

\hskip 3.25in
James Gallagher

\hskip 3.25in
George Milkowski

\end{latexonly}

\begin{htmlonly}

% Title page that latex2html will understand

\title{Report on the First Workshop for the \\
Distributed Oceanographic Data System}

\author{Peter Cornillon\\
\and Glenn Flierl\\
\and James Gallagher\\
\and George Milkowski}

\date{29 September -- 1 October 1993}

\maketitle

\end{htmlonly}

\end{titlepage}

\newpage

\begin{center} 
{\bf EXECUTIVE SUMMARY}
\end{center}

\noindent
The Oceanography Society, with funding from NASA and NOAA, is
planning a workshop series to define the structure of a client-server based
distributed system for access to oceanographic data over the Internet and to
develop and test a prototype.  The underlying idea of such a system is that
individual scientists as well as national archives will become providers of
data.  Any scientist or archive willing to make their data generally
available over the network would install a server on their CPU providing
access to their data. Researchers would then be able to obtain any of the
data available to this system through a client running on their own CPU.

The series will consist of three workshops.

The first workshop was held at the University of Rhode Island's Alton
Jones Campus from 28 September to 1 October 1993, reported herein. The
first day was dedicated to questions related to system architecture.
The second day focused on communication objects and the third on
defining a prototype based on the previous day's discussions. A system
programmer hired as part of this effort will be responsible for
implementing the prototype.

The second workshop will be held in the summer of 1994. It will focus 
on progress made in implementing the plans/issues discussed in the first 
workshop or problems encountered in this. In particular, the prototype 
outlined in the first meeting will be presented and discussed with some 
data sets. At this point it should be clear what needs to be done at 
each data archive site. Following the second workshop, the programmer 
will go for short periods to the participants' labs to help them get 
things going or will work with them over the Internet.

The third workshop will be held at the next TOS meeting, late spring 1995. 
Its purpose will be to present the model to the oceanographic community.

The workshops will be run administratively by The Oceanography Society. 
Scientific direction will be provided by a steering committee consisting 
of Glenn Flierl of MIT, Ken MacDonald of NASA (EOSDIS), Jim Holbrook of 
NOAA's Pacific Marine Environmental Laboratory and Peter Cornillon of the 
University of Rhode Island.

This is the report of the first workshop, intended to begin the
development of a Distributed Oceanographic Data System (DODS). The
body of the report is a summary with liberal interpretation of the
proceedings of the workshop. Appendix A is a list of requirements that
was abstracted from the minutes of the workshop (Appendix B) and from
the flip charts developed in each of the breakout groups.

During the meeting, we developed a set of requirements --- what the
users and data providers would like to see from a DODS. These can be
summarized as follows:

\begin{list}{--}{}

\item To be successful, scientists must find the system useful and turn
to it when they wish to collect data together for their research. This
implies that the system must provide access to both archive and PI-held
data sets. It must be easy and natural to work with, not only for
other persons' data but for their own as well. The system must provide
simple ways for scientists to distribute their data and to submit it
to the archive centers.

\item The system must make it easy to locate the appropriate information.
Various procedures for searching, browsing, and refining searches must
be possible.

\item Interfaces for programming languages must be provided so that
data can be incorporated directly into analysis routines. The system
must support multiple interfaces, from simple commands to a GUI.

\item To be successful, archive centers must perceive the system as a
good way for individuals to retrieve information from their archives.
In addition, it should encourage and facilitate submission of data.

\item The system must be easy to install and extensible (meaning that new
servers, new data types, new browsing procedures, new user interfaces,
new data filters, ... can be added at any time).

\end{list}

Originally it was intended that the architecture to be used for this
system would be defined at the workshop. This was however
not practical given the time available, and discussions with regard to 
system architecture begun at the meeting have continued in the 
interim. Appendix C contains a brief outline of the three different 
systems under consideration. The prototyping effort has begun and the
communications structure for the system is being defined and
implemented.

Each participant was asked to prepare a short summary of their current 
interests. Appendix E contains these summaries. Appendix D is a 
listing of the participants.

Everyone involved in the meeting with an Internet address was placed on a
distribution list. The purpose of this list was for messages of very broad
interest to the group such as when the next meeting would be held.  In
addition, several participants expressed an interest in being involved in the
more detailed discussions related to the development of the DODS that were to
follow the meeting. A smaller distribution list was formed for this
group. Some of the discussions alluded to in the previous paragraph have
taken place via this list. There has also been some discussion in this forum
related to data structures. 

To receive reports and other updates via the dods-report distribution list,
send an email message to {\tt dods-report-request@dcz.gso.uri.edu} with
`subscribe' as the subject or in the body of the message. If you would like
to participate in the more detailed discussions between designers and
implementors, join the `dods' list by sending a similar message to\\
{\tt dods-request@dcz.gso.uri.edu}.

A postscript version of this report can be obtained via anonymous ftp at\\
{\tt zeno.gso.uri.edu} in {\tt pub/workshop1/}.
\newpage

\tableofcontents
\listoffigures

\newpage

\section{\bf Introduction}
\pagenumbering{arabic}

The Oceanography Society, with funding from NASA and NOAA, organized a
workshop to explore issues associated with a distributed data management
system for oceanographic researchers.  The workshop took place at the
W. Alton Jones Campus of the University of Rhode Island on 29, 30 October and
1 September 1993. The workshop was the first in a series to promote the
development of a Distributed Oceanographic Data System (DODS). The long term
goal of this series of workshops is to develop a system which will provide
direct access to oceanographic research data over the Internet. The first
workshop focused on the specification of the system requirements as the
initial step in the development of the DODS.

Data providers, systems developers and research scientists from government
agencies, academic institutions and private corporations attended the
workshop. They provided a comprehensive perspective on the current state of
oceanographic data management systems as well as the expertise for productive
discussions. The format of the workshop included both small focus groups
tasked to address specific issues in detail and plenary sessions which
provided a forum for the general discussion of topics and review of
presentations made by the focus groups.

This report of the first DODS workshop summarizes the discussions that took
place at the meeting and documents the system requirements derived from those
discussions. The report is organized into six sections: Workshop Goals,
Workshop Organization, Motivation, Vision, General Implementation Issues and
Requirements.

\section{\bf Workshop Goals}

The three primary goals of the workshop were: 1) to develop a vision of a
distributed oceanographic data system, 2) to specify the requirements for
that system, and 3) to define a system architecture capable of accommodating
these requirements. The vision expresses the overall objective of the
workshop series. It also provides a model that can be decomposed into its
separate parts for the purposes of planning a development strategy.
Specifying the system requirements and system architecture are tangible,
short term goals that are viewed as the necessary first steps in the
development of the DODS.

\subsection {System Vision}

Developing a vision of the DODS at the workshop was important for a number of
reasons, it helped to:
\begin{itemize}

\item Define the problem

\item Clarify the solution 

\item Abstract different functions

\item Show the synergism of components

\item Form the foundation of the workshop discussions
\end{itemize}
\medskip

The first two points were critical to establishing a common level of
understanding regarding the purpose of the meeting as a whole. This was a
difficult task considering the diverse perspectives and backgrounds of the
attendees. For example the term `distributed system' has one meaning to a
systems engineer and a very different meaning to a person who is principally
a data provider. Systems engineers who work with computer networks generally
think of a distributed system as a system where ``...the existence of
multiple autonomous computers is transparent to the user'' (Tanenbaum,
1989). Many data providers use a much broader definition which includes
multiple data systems residing on different computers that may be connected
via a computer network.  The vision helped to clarify the meaning of words
and concepts.

As expected, the system vision continuously evolved throughout the course of
the workshop. However, considerable emphasis was placed on maintaining a
research oceanographer's perspective as the focus during the vision's
development. There was a simple reason for this; the purpose of the workshop
was to develop a system that will be used! Intentionally focusing on
researchers' problems helped ensure that the requirements produced by the
workshop would address those issues. This problem-oriented approach provides
a basis for developing tools that address a known research problem and are
immediately useful for that purpose.

Feedback from system developers and data center managers, within the
framework of the research oceanographer's perspective, was used to constrain
the system vision. By constantly referencing the system vision, the
discussions were prevented from becoming debates of system applications in
the abstract. By participating in the development of the vision, meeting
attendees were encouraged to explore new and innovative solutions to
scientific data management problems. Defining these solutions lead to
descriptions of the requirements necessary to develop a system.

\subsection {System Requirements}

The second goal of the the workshop was the specification of requirements for
a system based on the vision. The requirements for the DODS are derived from
the discussions that took place at the workshop and are meant to describe the
external behavior of the system. Most were formulated within the focus groups
and then presented in the plenary sessions for general discussion and
review. This process was very dynamic since there were significant
interdependencies between how one focus group's requirement would modify the
issues another focus group was addressing. Therefore, the plenary sessions
served as a forum to resolve discrepancies between different focus group
approaches. Appendix I gives the raw requirements as extracted from the
meeting. The raw requirements were used to develop the formal set of
requirements which are presented in the Requirements Section. These
requirements are the basis for the development of the DODS design plan, to be
generated following the workshop.

\subsection {System Architecture}

The final goal of the workshop was to specify a system architecture that
would support the DODS vision and satisfy the requirements. Constraints
placed on the architecture are that it must be compatible with current
systems that would participate in the DODS and be capable of accommodating
future systems (large and small) as based on the system vision. This is
obviously an important consideration if the DODS is to be successful.

Discussions related to the system architecture helped to resolve where in the
system specific functions might take place. In some instances this provided
relevant feedback to the requirements specification process. For example, we
considered whether data manipulation functions, such as file decompression,
should take place at the remote or local CPU and the trade off in each
instance.

\section{\bf Workshop Organization}

The workshop was two and a half days long. The first two days were devoted to
the discussion of oceanographic data systems and their functions. The half
day session at the end of the meeting was spent summarizing the previous two
days' discussions and synthesizing them into requirements for the development
of a distributed oceanographic data system. Each of the first two days
focused on a single topic. The topic for Day One was system architecture,
and for Day Two it was data objects and communication protocols.

Each of the main topics for the first and second day of the workshop was
initially introduced by a plenary session. These introductions framed the
topic in the perspective of a research oceanographer. Following the morning
plenary sessions, the workshop participants broke up into focus groups to
examine the topic from a different and more narrowly focused perspective.
Afternoon or evening plenary sessions were used to summarize the results of
each of the focus groups to the workshop as a whole.

There were three focus groups which met on both of the first two days of the
workshop; data providers, system developers and data users. Each of these
groups corresponded loosely to the different roles of participants at the
workshop. A fourth focus group met on the first day of the workshop to
discuss data objects and protocols in use by existing distributed data
systems. Each focus group had a group leader who stimulated discussion and
summarized the groups' debates for the plenary sessions.

\section{\bf Motivation}

The workshop series was motivated by the rapid increase in the number of 
data sets available on the Internet coupled with a lack of coordination 
in the systems being developed to access and/or distribute these data.

Recently, many oceanographers with large data sets have been making 
their data sets available on-line. At the same time federal agencies 
have begun exploring the development of a distributed data system for 
global change research and federal archivists have begun investigating 
on-line access to data in federal archives. These efforts are 
however being undertaken with little to no coordination. For example,
most research oceanographers are potential data providers as well as 
data users; a point overlooked by the federal agencies in their 
development of an earth science data system for the 1990's. In 
addition, because research oceanographers collect, calibrate, 
process and analyze their own data, they are intimately familiar with 
its strengths and weaknesses. This means that data sets derived from 
the raw data by the researcher/collector often include his largely 
``undocumented'' knowledge of the raw data. So, although the raw data 
may be transferred to federal archives for distribution to the 
community at large, oceanographers will often prefer to deal with data 
sets that are acquired from a colleague whom they know and whose judgment 
about data quality they accept. This is especially true in large 
interdisciplinary experiments such as SYNOP, JGOFS, or WOCE that have 
been conceived and undertaken by a group of scientists at a number of 
different institutions. To make full use of their data, each of the 
scientists will in general require data from one or more of their 
colleagues. In order to address this problem, groups of researchers 
have begun investigating or have already developed data systems that 
provide for easy distribution of data between colleagues on the same 
project. The JGOFS data system is an example of this. The difficulty 
is that these systems are being developed independently and, in 
particular, accessing data held in one of these systems by another 
system is difficult to impossible. 

Although many research efforts require the acquisition of new data, 
most also make use of existing data held in the national archives. 
Access to these data is an absolute necessity. A consortium of federal 
agencies recognize this and have indicated that they will work toward a 
system providing seamless access to data held in federal archive, but
progress toward the actual development of such a system has been slow at 
best. The perception from the outside is that each agency is focusing 
on systems that will meet their needs with little actual interaction 
between the agencies with regard to these systems and how they will 
operate together. 

Much of the above became evident through a group of demonstrations 
of on-line data systems assembled at the recent Oceanography Society 
meeting held in Seattle. Of the eighteen systems presented, ten of 
which were operational at the time of the meeting, no pair could 
communicate with ease! It also became evident at the meeting that the 
rate of development of such systems was going to increase rapidly over 
the next several years. The potential explosion of available 
data sets along with the fact that there currently exist few systems 
capable of delivering the data, point to the present as a window of 
opportunity with regard to the development of a truly distributed 
data system: the hardware exists, individuals as well as institutions 
are willing to make their data widely available and advances in software 
design make a distributed system practical. This workshop, the
development effort following, and the later workshops in the series
have been designed to address the remaining required ingredient --- 
cooperation in the design of a system to access these data.

\section{\bf Vision}

The vision of data access that emerged from the workshop was that of
oceanographers interactively moving data from data sets located on remote
systems directly into their analysis packages for examination or using local
applications to acquire and process these data in programs tailored to their 
research problems.  From the oceanographer's perspective, the distributed data
to be accessed has a consistent form and structure making it straightforward 
to locally manipulate data from different sources.  The system is viewed
as being integrated with the oceanographer's own data system and applications
environment so that commands, operations and applications for accessing,
processing and analyzing research data are the same whether accessing 
locally stored data or data held remotely.  In addition, the system is a tool
which enables researcher to solve problems by providing access to data both
as they experiment with analysis ideas and when they are ready to actually
process the data.

It became clear at the workshop that this last point, the research
scientist's approach to problem solving was not well understood.  Scientific
research is generally carried out in a tentative fashion at first and, as the
researcher gains a better understanding of the issues and the data involved,
in a more assertive mode.  In the early stages the scientist repeatedly asks
``what if'' questions, often searching out additional data to corroborate an
observation or to answer a question not originally posed but clearly related
to the problem at hand.  This approach to problem solving is seen as being
fundamental to the design of a system that will efficiently meet the
scientist's research needs.

To help render the vision presented in the first paragraph of this section
more tangible, we present it in the context of a system in which all data are
either held locally or reside on remotely mounted disks.  First, however,
consider how a scientist accesses and analyzes data held locally on-line;
this may help clarify the point raised in the previous paragraph. There are
two general models to data access in commercially available and/or public
domain analysis packages.  We refer to those based on an
identify-load-command-manipulate paradigm (e.g., MatLab) as Class I
systems, and those based on an identify-command-load-manipulate paradigm
(e.g., FERRET) as Class II systems.  The major distinctions between the two
classes are when and where data that the user wants to access can be
sub-sampled.  In Class I applications the user defines the data (location
and format), the identify step, then directs the application to load the
data.  The application reads the entire data resource into memory; then the
user is free to issue commands and manipulate, sub-sample or display the
data.  For Class II applications the user again identifies the data resource
location and format, but instead of loading the data immediately, the
application provides verification of the resource's existence.  When the user
issues a command to manipulate and/or display the data, the application
retrieves only the data which has been specified.  In Class I systems the
data are sub-sampled within the application program's memory, for Class II
systems the data are sub-sampled prior to being placed in memory.  In both
cases, the user identifies the data resources of interest and then specifies
operations to be performed with the data.  Both systems allow the researcher
to easily access locally held data and provide a straightforward method for
operating on multiple types and multiple formats of data resources.

Now consider the case where one of the data resources resides on a disk 
that is connected to another, remote system.  Assuming that the 
structure of these data is well defined (and known) and that the data 
can be ingested by the analysis package, the disk containing the data
might be remotely mounted via the network file system (NFS\footnote{NFS, 
because it is widely used and well known, is used here to
illustrate the concept of a virtual file system.  There are other
implementations of virtual file systems that may in fact be more appropriate
to the case in hand}) at which point the data appear as a locally accessible
resource to the researcher's analysis application (regardless of class).  The
user specifies these data resources in the same fashion as was done in the
simple case of locally held data and the same range of operations is available.

Next, imagine that all disks with oceanographic data are NFS mounted to the
user's system and that the analysis application contains a listing of all
data sets, their formats and the NFS mounted disk on which they 
reside. In such a world, if the
researcher was interested in contouring all XBT derived temperatures at 500m
and plotting the results on a satellite derived SST field, the 
analysis application would access all the NFS mounted disks with XBT data,
import the appropriate subsets, remove duplicate XBT values from different
sources, perform the 2D contouring and plot the results on the SST derived
field.  Conceptually, all oceanographic data resources are local and
available to the investigator's analysis application.

\section{\bf General Notes on Implementation of the DODS}

The DODS is envisioned as a system with the functionality presented in the
previous section. The actual implementation of the system need not however
rely on remote mounting of disks containing the data of interest; there are
other software implementations that would accomplish the same objectives. The
image of remotely mounted disks was used in that it is a simple extension of
concepts that we all use in our current approach to data analysis and hence
renders an understanding of the vision straightforward.

In choosing an actual implementation strategy, an important underlying 
principle of the DODS effort is the belief that much
of the work of putting the system together will also be distributed (as are
the data).  How compatible specific system implementations, such as the NFS
virtual file system example above or an object oriented client-server system,
either of which could support the functionality required for the DODS, are
with regards to distributed development must be considered in the design of
the DODS.  The relevant issues are how will DODS evolve within a rapidly
changing technological environment and how the responsibility for that 
evolution is distributed? Virtual file systems or client-server approaches 
will provide different answers to these problems.  For example, in the case 
of virtual file system packages, system-to-system communication software is
maintained and controlled by a third party, whereas a client-server system
developed in-house would presume that the communication software is developed
by the DODS development team and maintained by the DODS community.  This
issue brings to light questions concerning software maintainability,
extensibility and obsolescence.  

A second consideration with regard to the choice of a system relates to the
requirement that it handle very large data sets, data volumes too large to
acquire over communal networks. In these cases, data would be staged by the
resource system to removable media (e.g., tape, CD-ROM or optical disk) that
are then made accessible to the user's local system.  Sophisticated users,
who must process large volumes of data in the assertive phase of their
research, will require DODS to provide such capabilities. This means that the
system must be capable of dealing not only with random access devices, but
serial devices and, more importantly, there must be a way of transferring
along with the data a means of accessing it. In a client-server
implementation, the server, providing data from a remote system, resides on
that system. The server may perform format transformations when the data are
requested by a client. If a request for a large volume of data on removable
media does not pass through the server, (i.e., if it is a straight file
transfer to the media) a format transformation may be required on the user's
system. Without knowledge of the structure of the data this is impossible.
Furthermore, moving the server to the user's system may be quite
difficult. There are several different fashions in which this problem could
be addressed; some impact the architecture of the system itself while others
impact the supplier or the user.

As indicated above, central to the success of the DODS is cooperation; 
the hardware, approaches to the software and accessibility of data are 
all realities. Although oceanographers require data from other 
disciplines and those in other disciplines may require access to 
oceanographic data, it was agreed at the meeting that the system would 
be designed to meet the needs of the oceanographer and would focus on 
access to oceanographic data. The group felt that a larger audience 
would jeopardize the success of the effort. This does not mean that 
non-oceanographers would be denied access to the system, nor that 
non-oceanographic data would be excluded from it.

Clearly, in developing the DODS different approaches must be investigated
within the context of the issues raised above while at the same time
satisfying to the extent possible the requirements defined at the meeting and
presented in the following section.

\section{\bf Requirements}

Software requirements describe the external behavior of a system.  Ideally,
requirements reduce ambiguity in the description of work to be done so that
developers can evaluate the completeness of a design and users know what to
expect in the finished system.

This section contains the requirements for the DODS as described by the DODS
Workshop minutes.  These requirements were developed by refining basic
questions about the system's existence. The answers to 
those questions came both from the minutes and from our vision of the system.

The basic question that emerged at the workshop was: {\it What will it 
take to make the DODS to succeed?} It quickly became clear that the answer 
is quite straightforward. The scientist must want to use the system; 
it must be the first place that he or she will think of going to satisfy 
a data need. This then gave rise to a host of other questions: why 
will the scientist want to use the system? why will data archives want 
to make data accessible to the system? etc. The answer to each of 
these raises still more questions. This section is organized in terms 
of these questions beginning with the most general and moving to the 
requirements. The section is divided into subsections defined by a set 
of high level questions. For clarity, these are listed below prior to 
entering into the details of the responses.

Many of the items in this section have a tag that shows which items from the
minutes match them. Appendix I contains a numbered list of requirements taken
from the minutes (Appendix II). The tag numbers in this section match the 
item numbers in Appendix I. 

\subsection {Why will scientists want to use the system?}

\subsubsection {It provides access to data sets held by other scientists
and to data sets held in the archives.}

\newcounter{e}
\begin{list}{\arabic{e}.}{\usecounter{e}}

\item The system provides a self-consistent view of data.  

\item The system supports browsing and searching the data. R:~21

\item The system provides a straightforward API that lets scientists write
FORTRAN subprograms and C functions for use with off-the-shelf analysis
packages as well as their own in-house analysis software. R:~23,~29,~51

\item The system provides command line interface tools that can be used
from the UNIX shell. R:~51

\item The system will support other types of user interfaces including a
GUI. R:~50

\item Data accessed are presented in a consistent, usable form,
regardless of their storage form or location. R:~16

\end{list}

\newcounter{m}
\begin{list}{}{\usecounter{m}}

\item To obtain a self-consistent view of  data requires:
\newcounter{f}
\begin{list}{--} {\usecounter{f}}

\item All data (remote and local) are accessed using the same commands.
R:~12,~13  

\item The system can resolve synonyms when getting data. R:~6

\item The system will support data set version numbering. R:~54

\item The data can be selected with a simple query

\begin{list}{--} {\usecounter{f}}

     \item Using boolean, relational, and functional operators.

     \item The result of a query is the data it describes. R:~58

\end{list}

\item The storage format of the data is hidden from the user.

\end{list}

\item To support browsing requires:
\begin{list}{--}{\usecounter{f}}

\item Individual datasets may provide browse capabilities in various forms;
the user should be able to take advantage of these

\end{list}

\item To support searching and location of data requires:
\begin{list}{--}{\usecounter{f}}

\item The system provides a system-wide directory function.  R:~2,~3,~7

\begin{list}{--}{\usecounter{f}}

     \item The system knows about all of its available resources.

     \item The system's data-set database is fully distributed (i.e., it
           is transparently spread out over some or all of the machines
           which make up the system). R:~8

     \item The system's directory information is maintained dynamically. R:~9

     \item The system automatically propagates information about additions,   
           modifications or deletions to its data resources to the
           rest of the system.

     \item The system may support search refinement. R:~5

\end{list}

\item The location of the data is hidden --- Software that the scientist
uses will know how to communicate with other parts of the system --- the
scientist does not have to know. The scientist never has to think `this is on
another CPU'.

\item The system enables querying of its data resources with user defined, 
multi-parameter searches. 

\begin{list}{--}{\usecounter{f}}

     \item The scientist can search `keyword {\tt relop$|$binop} value' of
           information extracted from data sets managed by the system.

     \item The system supports `unknown parameter' searches. R:~10

     \item The system is able to resolve keyword synonyms in queries. R:~6

\end{list}

\item The scientist can search plain text descriptions of data holdings.

\item Co-located searches are fully supported by the system. R:~4

\item If more than one data set must be used to satisfy a query, then the
system must do that. R:~11

\item Once found, the same data set may be accessed may times without
repeating the initial search process. R:~24

\item If a scientist knows exactly where a data set resides, the system can
go directly to that data set without searching. R:~18

\item Whenever data are managed by the system, they are automatically
accessible to other users of the system. R:~30

\item A provider may set limits to the remote access of his or her data sets
managed by the system. Such access limits may apply to everyone or to
everyone except a group of privileged users.  R:~48,~56

\end{list}

\item The data is usable because:
\begin{list}{--}{\usecounter{f}}

\item The system provides direct electronic access to data. The system
       uses the Internet to move data to the user. The existence of
       multiple computers within the system is transparent to the user.
       R:~28

\begin{list}{--}{\usecounter{f}}

     \item There may be limits on the total quantity of information
           that can be transmitted over the network in response to
           any single query. These will be set by the individual
           provider.
           R:~25

     \item Very large data sets may be transferred by mail; the user
           will be able to use these when they arrive with the same
           commands used over the network.

     \item The scientist can access parts of a data set using a query.

\end{list}

\item When the scientist gets data, it is returned in a self-describing
      form. By transforming all data, regardless of storage format,
      into a (canonical) self describing form the system only has to
      provide a parser for that form to correctly handle all data
      managed by the system. 

\item For this reason, it is simpler to combine different data sets and
      to co-locate data from two or more data sets. 

\item Once data has been accessed, the system makes it easy to import data 
      into other applications. The operations performed to retrieve
      data make it straightforward to use those data with analysis
      packages. The API interface makes it simple to use systems like
      MatLab, ... and the command line interface makes it simple to
      pipe a data stream into UNIX or home-made filter programs. 
      R:~50

\item Data can be saved in files. The resulting files are self-describing
      and may accessed using the same API as remote data.

\item Data users will be allowed to provide comments regarding individual 
      data resources.
      R:~47

\begin{list}{--}{\usecounter{f}}

     \item Feedback to data providers on problems with data quality or
           access. 

     \item Comments will be attributed to their authors, so the incentive to 
           make meaningful comments will be fairly great.

     \item The user comment capability would be enabled at the discretion of
           the data provider.
           R:~26
\end{list}

\end{list}

\subsubsection {It provides an easy tool for managing their own data.}

\newcounter{g}
\begin{list}{\arabic{g}.}{\usecounter{g}}

\item A single `interface semantics' is used to locate and access data.

\item The system can use already developed data management tools or work
       with data organized using files and the UNIX file system.
       R:~50

\item The scientist can choose to ignore any of the features of the system
       while using any other features.
       R:~17,~20

\item Analysis tools can be added

\begin{list}{--}{\usecounter{f}}

     \item The system's transformation of data includes format translation
           and subsampling. 
           R:~55

     \item The system includes the capability of adding new
           transformations

\end{list}

\item All data sets managed by the system are uniquely identified.
       R:~31

\item Although the system supports many features, a scientist is never
{\em required\/} to make use of those features --- A choice of options is
always given. These options allow the scientist to choose a level of
participation in the system.  R:~26
      
\end{list}

\subsubsection {It provides an easy means for them to make their data 
available to others.}

\newcounter{h}
\begin{list}{\arabic{h}.}{\usecounter{h}}

\item Scientists may choose from self describing data formats --- both
       standard and language-based types. 
       R:~14,~22

\item Other storage formats may be added at any time

\item The system will be able to support providing data in a DBMS.

\item Data providers retain complete control of their data resources

\end{list}

\item The self-describing formats are often what the scientist is
already using; then, files need only be moved to the appropriate
directories within the system. Formats to be considered initially
include
\begin{list}{--}{\usecounter{f}}

\item netCDF, HDF, GRIB/BUFR, various ASCII tables
       R:~34

\item Data providers will be encouraged to write text describing each data 
       set and to use a self describing data format.

\item Choosing to make the data set description file and/or using a
       self-describing data format will make it possible to use more of
       the system's features with the data. This is true for both local
       and remote users.

\end{list}

\item For data not already stored in a supported format, language tools
will be provided so that the scientist can describe the data's
structure to the system.
\begin{list}{--}{\usecounter{f}}

\item Some ways of structuring data can be described to the system by
       creating template files using a text editor. This will {\em not\/}
       require any programming knowledge at all.

\item Other ways of structuring data can only be described by writing
       programs.

\begin{list}{--}{\usecounter{f}}

     \item If data needs unusual access procedures (e.g. to be accessed
           efficiently) the scientist will have to write a program that
           implements that procedure. We will make documented source
           code available to the scientist that will illustrate how we
           wrote the code that access files.  The documentation will
           be more than just commented source code, it will be both
           annotated source code and a reference manual. 
             R:~35

\end{list}

\end{list}

\item Data providers can include general and/or technical documentation on
their data resources as an aid to users accessing their resources.
Such documentation is provided at the discretion of the data provider
for data users.  They can choose to limit access to the data
to only designated users.
       R:~26


\subsubsection {It provides an easy method for submitting data to the 
archives.}

\begin{list}{--}{\usecounter{f}}

\item Since the national archives will be on the system, the archive
       center can use it to transfer data and write it into their
       own storage system. 
       R:~57

\end{list}

\subsection {Why will the data archive centers want to use the system?}

\begin{list}{$\bullet$}{\usecounter{f}}

\item It allows them to distribute data with minimum effort and expense.

\item It allows them to acquire datasets with little work.

\end{list}

\begin{list}{--}{\usecounter{f}}

\item Data centers still have difficulties fulfilling their mandate ---
        to provide data. 
        R:~19,~30

\begin{list}{--}{\usecounter{f}}

     \item Most data centers currently provide data through hard media
           (i.e., tapes and CD-ROMS).  A number recognize the
           advantage to the research community of having data on-line
           and accessible over the Internet.

\end{list}

\item Many data sets held in data center archives are high profile items.
       Providing access to these data sets will jump start the system.

\item  One of the responsibilities of data archival centers is to acquire 
       data from scientist and programs.  The system supports data
       centers not only providing data but also acquiring data for
       archival purposes.

\item Data centers do not have to make use of all of the system's
       features when managing a given data set with the system.
       R:~17

\item The system will support different kinds of data resource documentation. 

\begin{list}{--}{\usecounter{f}}

     \item The system will support maintenance of access statistics.  Use
           statistics will be internally maintained at the discretion of
           the data provider. 
           R:~53

     \item The usage log for each data set will be available to the same set of
           users as the associated data set.
           R:~49

\end{list}

\end{list}

\subsection {How will the system be installed?}

\newcounter{i}
\begin{list}{\arabic{i}.}{\usecounter{i}}

\item Installing the system is easy

\item Extending the system is straightforward

\item The system is an open system.

\end{list}

\item The system is only supported on the UNIX operating system. Some
features may not be available without a workstation and X11R5.
       R:~1
\begin{list}{--}{\usecounter{f}}

\item All the programs that make up the system can be copied from an 
       anonymous ftp server. Any third party libraries required to
       build the software will be available at this site as well.
       R:~36

\item The system will be distributed in two forms: 1) precompiled binaries
       for popular workstations, and 2) source code.
       R:~36

\begin{list}{--}{\usecounter{f}}

     \item The (supported) workstations include
           Sun Sparc: SunOS 4.x; DECstation: Ultrix; DEC alpha:
           OSF/1; SGI: IRIX; IBM RS6000: AIX
           R:~37

     \item Source code will be available so that other platforms can run
          the system, but we will only support additional UNIX
          platforms if there is a significant call for such support
          and we have the resources.
          R:~38

\end{list}

\item The bulk of the system will be written in ANSI C. Parts may use
       FORTRAN if standard, widely used, source code exits for common
       algorithms and if there are significant reasons not to recode
       it in ANSI C. We may also use a different language for any X11
       interface we produce. Whether the system {\em can\/} be ported to OS `Q'
       depends mostly on how similar `Q' is to UNIX. Without knowing `Q' in
       advance, it is impossible to say if the port can be done easily. Any
       cross-OS port will almost certainly require a systems programmer
       with experience.

\item The software used to manage data may be more complex to build on
       unsupported platforms than the software used to access data on
       supported platforms. R:~38

\item Enough system documentation will be provided so that others may develop
       software that will work with `stock' parts of the system. This
       explicitly includes user interface software.
       R:~39,~50,~51

\end{list}

\item There will be no central management authority for the system. All
nodes on the system appear equal. Responsibility for how the system is
used and evolves is shared mutually by all participants in the system.

\subsection {Possible design features of the system.}
\newcounter{j}
\begin{list}{\arabic{j}.}{\usecounter{j}}

\item The system will not exclude the use of protocol translators located at
  the data source.
  R:~40

\item The design will not preclude adding protocol translators for
  efficiency on the user's software, but those translators must be optional
  (i.e., they must not be required to access any data).  R:~41

\item The system must support filtering the data stream, both at the data and
  at the user end.
  R:~42

\item The system must support a base level communication procedure by which
  all parts of the system may communicate.
  R:~43

\begin{list}{--}{\usecounter{f}}

   \item This communication procedure must include: Data set documentation
    (which may be null), Identifiers, attributes, data values/types, and
    structure.
    R:~44

   \item This does not have to be the most efficient means of communication
    between parts of the system.
    R:~46
 
\end{list}

\item The system's components must support negotiation. Each component
must be able to determine from other parts what level of support they provide
for the various features of the system.  R:~45

\item The design of the system must be both extensible and scalable. R:~52

\end{list}

\end{list}

\newpage
\begin{thebibliography}{99}
\bibitem{1} Tanenbaum, Andrew S. 
{\em Computer Networks, 2ed.},
Prentice-Hall:1989
\end{thebibliography}

\newpage

% appendix introduces the appendix section of a document --- each appendix is
% startes with section.

\appendix

\section{\bf DODS Requirements from the Minutes}

\bigskip

The following are requirements for the DODS from the workshop minutes. The
list is complete in the sense that everything in the minutes that is clearly
a requirement is included here. However, there may be important requirements
that are not on this list. See Section 7 of this report for the complete
requirements. 

\begin{enumerate}

\item The system must be accessible from a workstation.

\item The system must make the location of remote data easy.

\item The system must make the location of local data easy.
 
\item The system must be capable of performing searches and co-location
searches. 

\item Refined searches are optional.

\item The system must be able to resolve keyword synonyms. This is true for
both the object locator and data queries.

\item The system must provide `meta data' in some way (i.e., location of data
through descriptions of each server site's contents).

\item Data location will be supported with a distributed database.

\item The distributed locator database will be automatically updated.

\item The system must be flexible enough to handle unknown parameter searches.

\item If several data sets should be examined to satisfy a given query, the
system must facilitate that.

\item The system must make acquisition of remote data easy.

\item The system must make acquisition of local data easy.
 
\item The system must support a variety of different data types.

\item The system must provide access to field/project data.

\item Data Providers --- PI and centers --- can provide a range of services.

\item The system will provide access to data held by individual scientists.   

\item The system will provide access to data from government archive centers.

\item Not all data sets will be treated equally by the system.

\item The system will be able to browse data.

\item Data will be provided in a small set of standard formats.

\item Data will be directly accessible via API without first saving it to a
file. 

\item Data must be accessible via the search process or directly without first
searching.

\item Servers may respond to some queries with a message rather than data
(e.g., ``You requested too much data'', ``Get data by ftp'',...).

\item Data servers can provide varying service depending on the data set.

\item The system must support redundant data screening.

\item The system must deliver data electronically when possible.

\item Data acquired should be easily accessible to analysis packages (e.g.,
MatLab).

\item The system reduce the load of providing data to others.

\item The system must provide a global naming procedure for data sets (reduce
name-space pollution). 

\item The system is specifically for oceanographic data.

\item The system will allow other types (i.e., non-oceanographic)  of data, but
will not be deliberately designed with such data in mind.

\item The system must provide data translators for at least the following file
formats: GRIB/BUFR, HDF, netCDF.

\item The system will provide support for data-archive resident software (e.g.,
servers) written by the DODS development team, others working in conjunction
with the DODS Development Team, and others all on their own.

\item Software distribution must be easy for users to build --- it must not
require that they have a suite of other libraries. Instead use binary
distribution or source distribution with all of the necessary libraries
included.

\item Source code can be non-trivial to build if binary software is provided
for Sun Sparc SunOS 4.1.x, DECstation Ultrix, Alpha OSF, IBM RS6000 AIX, SGI
IRIX.

\item Server source code and/or binary software can be more complex to
build/install than client given the relative complexities of the systems (but
not more than 1 Programmer-Day).

\item The system should provide enough support for programmers that additional
user interfaces can be constructed.

\item The design will not preclude accessing existing systems via protocol
translators located at the server.

\item The design will not preclude {\em optional\/} client side protocol
translators that improve efficiency. However, such translators {\em must\/} be
entirely optional. 

\item The system must support pre and post filters for the data stream.

\item The system must support a base level communication procedure.

\item The server's responses must include: Data set documentation, identifiers,
attributes, data values/types, and structure.

\item The system's servers must support protocol negotiation.

\item The base level communication procedure does not have to be `efficient'.

\item The system's log file allows users to comment on data sets.

\item The data provider must be able to determine what data objects are
available and the order in which those data objects are delivered.

\item Usage log should be centralized (physically or logically).

\item The system must provide easy-to-use features for the novice and
sophisticated features for advanced users.

\item The system must support several user interfaces: menus/hypertext and
a programming language API.

\item The system design must be both extensible and scalable.

\item PI/providers must be credited for making data sets accessible.

\item The system must provide data set version numbers (along with processing
algorithm version/revision control).

\item The system's distributed computing capabilities are limited to format
translation and subsampling.

\item The provider must be able to restrict access in a flexible way.

\item The system should support data transmission from servers to national
archives.

\item The system should not make unnecessary distinctions between `data' and
`metadata'.

\end{enumerate}

\newpage

\section{\bf Meeting Minutes}


%\documentstyle[12pt,html,psfig]{article}
%%
% hacked from art12.doc
% 
% makes 1 , 1.25, 1.25 and 1.25 in. margins on the top, left, right and
% bottom resp. when using \documentstyle[12pt,...]{article}
%
% jhrg 1/6/94

% SIDE MARGINS:
\if@twoside               % Values for two-sided printing:
   \oddsidemargin 0pt     %   Left margin on odd-numbered pages.
   \evensidemargin 38pt   %   Left margin on even-numbered pages.
   \marginparwidth 85pt   %   Width of marginal notes.
\else                     % Values for one-sided printing:
   \oddsidemargin 18.5pt  %   Note that \oddsidemargin = \evensidemargin
   \evensidemargin 18.5pt
   \marginparwidth 68pt 
\fi
\marginparsep 10pt        % Horizontal space between outer margin and 
                          % marginal note
 
 
% VERTICAL SPACING:        
\topmargin 0pt            %    Nominal distance from top of page to top
                          %    of box containing running head.
\headheight 0pt           %    Height of box containing running head.
\headsep 0pt              %    Space between running head and text.
\topskip = 12pt           %    '\baselineskip' for first line of page.

\textheight = 635pt % 36\baselineskip
\advance\textheight by \topskip
\textwidth 435pt         % Width of text line.

% Adding this fixes the top and bottom margin sizes. jhrg

\textheight 8.75in

\newcommand{\postscript}[2]{
        \par
        \hbox{
                \vbox to #1{
                        \vfil
                        \special{ps: plotfile #2.ps}
                }
        }
}







%\begin{document}              
\Large
\begin{center}
{\bf Distributed Oceanographic Data System\\
Workshop\\
\large
September 29 - October 1, 1993\\
W. Alton Jones Conference Center\\ 
University of Rhode Island}\\
\end{center}
\begin{latexonly}
\vskip4pt\hrule height4pt\vskip4pt\hrule height2pt\smallskip
\vskip .35in
\end{latexonly}
\centerline {\bf Day 1:}
\normalsize

\vskip .25in
\begin{center}
 {Introduction by Dr. Cornillon at 8:50}
\end{center}
\medskip
Visions of "Utopia", a perfect oceanographic data system, which will 
then be limited by some of the realities. 
 
I need access to various data sets to design an algorithm to estimate 
mixed layer depths globally from scatterometer winds and AVHRR-derived 
day-night SST differences.

I will need other data:
mooring data with good vertical resolution to help design the algorithm
XBT/CTD data to help validate the algorithm

will also need the data to be coincident with scatterometer data and clear 
AVHRR day and preceding or night fields

have the best geographic coverage possible
Such data does now exist at NODC, WHOI, JPL , URI and other sites

We envision a system accessed from our workstation that will
allow us to locate and acquire these data sets including those held on our 
site quickly and easily

provide these data to our favorite analysis package in a format that the 
package recognizes.

We envision others accessing our data the same way (reducing the load of 
providing data to others that is currently an  impediment to data 
exchange)
\bigskip

\noindent {\bf Assumptions:}

The system will: 
\begin{itemize}
	\item focus on oceanographic data (Schramm- does this include met
data?)  
	\item be designed to serve the researcher
	\item be distributed (available to any data sites)
	\item based on a client-server model
	\item be dynamic and changing daily
	\item be easy to install (approximately 1 day for systems programmer)
	\item contain a variety of different data sets and data types
	\item not support distributed processing (at least not designed with
this as a requirement.
	\item upwardly compatible
\end{itemize}

Question about acceptable time delay in locating and receiving data.\\
\smallskip
{\bf B. Douglas -} It's easier to locate a large data set than a highly
delimited small data base, for these are "well supported".\\
\smallskip
{\bf G. Flierl -} question about "install times"\\ 
\smallskip
{\bf E. Dobinson -} two other assumptions need to be added:
\begin{description}
	\item{1)} do we assume the user knows what data he/she wants?  

\smallskip
	{\bf P. Cornillon -} we will need a "front end" director to provide a
location function
\smallskip
	\item{2)} second assumption - is it always better to move GB of data
around, or do we move the request to the site where this data is stored and
do the processing on-site (move the request to the data site rather that the
data to the request site)
\end{description}

\medskip
{\bf W. Schramm -} if your data are stored at multiple sites, this would be
very hard. 

\medskip
{\bf D. Glover -} Distributed Computing Environment - actual processing 
location should be transparent, system will optimize.  Do we want to 
consider such as system?\\
{\bf P. Cornillon -} We will restrict this morning's discussion to more 
traditional move request to data.

\medskip
{\bf H. Debaugh -} As long as you get your data, you don't care where you get 
your data from - it's much more complex to move your "program" to an 
unknown processing environment.

\smallskip
{\bf J. Corbin -} we should not preclude the ability to do distributed
processing but our system should not require it.\\
{\bf P. Cornillon -} Lets assume for the near term discussion we do not have 
distributed processing---
\smallskip
The Client/Server architecture allows flexibility in
\begin{itemize}
	\item The data formats served the oceanographic data systems may be
linked to other disciplines data systems in the future 

	\item The user interface other user groups can build their own
interface to the system (e.g. school teachers) 
\end{itemize}

\noindent {\bf The workshop series:}
\begin{itemize}
	\item Objective - {\bf To develop the base for our distributed
oceanographic data system}
	\item Two workshops 
    	   \begin{itemize}
		\item Workshop \#1 (now) - communications
		\item Workshop \#2 (summer 1994) servers
    	   \end{itemize}
	\item Fourth TOS meeting - spring 1995 the rest of the world
	(systems programmer to implement the system designed in workshop 
	\#1 and to help data providers build servers)

	\item Coordinated by TOS (not a URI of JGOFS effort!)
	\item Steered by a steering committee
     	   \begin{itemize}
		\item Glenn Flierl - MIT
		\item Ken McDonald - NASA
		\item Jim Holbrook - NOAA
		\item Peter Cornillon - URI
     	   \end{itemize}
\end{itemize}

\medskip
\noindent Question about use and value of metadata\\

{\bf W. Brown -} should capacity to have/use metadata be an assumption?\\
{\bf P. Cornillon -} metadata varies so much in descriptions, ranging from
what it is to how it was calculated.  Issues of what the metadata is are very 
important.

\medskip
{\bf R. Wilson -} it is very important to have a browse capability of the
host.  This lets you verify the suitability and applicability of the data.\\
{\bf P. Cornillon -} data browse will be an important requirement in our
system.  This will be an internet accessible system.

\medskip
{\bf B. Douglas -} question about data a PI is still working on and has not 
published. What happens when this data loses its identity in the system - 
since his career depends on proper recognition.  How will originators get 
credit for participating in this system?\\
{\bf P. Cornillon -} we will have a "voluntary" system.  We want to have
people use it because our system will help them. We're seeing a real change in 
how oceanographers make their data available.\\
{\bf B. Douglas -} reality says we must have citations of how much our data is 
being used, and by whom.

\medskip
{\bf B. Schramm -} researchers need access to "operational" Navy and NWS data, 
yet we have constraints on who in the international community will have 
free access to this data.\\
{\bf P. Cornillon -} we don't want "password" protection...

\medskip
\noindent back to introduction...

This workshop is sponsored by TOS, not URI...

Time line viewgraph of workshop sequencing, with the goal of eventually 
developing our prototype client:

\begin{center}
 {\bf Timeline Figure}
\end{center}
\bigskip
\noindent Workshop \#1 Communication Objective 
\begin{itemize}
	\item Define the architecture of the System form of the Client/Server 
System
    	   \begin{itemize}
		\item focus on the first day
    	   \end{itemize}

	\item Choose the Communications Protocol for Data Objects passed 
between clients and servers - 
    	   \begin{itemize}
		\item focus of the second day 
    	   \end{itemize}
\end{itemize}
\smallskip
{\bf E. Dobinson -} without rank-ordered system requirements from the 
scientists, its hard to come up with a desired system architecture.\\
{\bf P. Cornillon -} how the different architectures scale will be an issue.\\
{\bf R. Chinman -} isn't agenda for "data providers" much wider than just
"what data will be available"?\\
{\bf P. Cornillon -} We will let breakout groups define their own agendas.

\bigskip
\noindent {\bf G. Flierl:  Discussion of Distributed Systems Architectures}

\medskip
Where do existing systems fit in the context of the models we have discussed?
(Slides are included in notes) 
\begin{description}
	\item{Slide 1:}  Conventional Data Base Management System --

	   \begin{itemize} 
		\item Data held in a few formats specific to that system.
		\item Multiple data formats pose problems.
	   \end{itemize}

\begin{figure}[h]
\centerline{\psfig{figure=slide1.xfig.ps,height=5.0in}}
\end{figure}

	\item{Slide 2:}  Client-Server model -- Moves some of the processing
functions to the client.  No distinction between "data" and "metadata" . 
\begin{figure}[h]
\centerline{\psfig{figure=slide2.xfig.ps,height=5.0in}}
\end{figure}
\clearpage

	\item{Slide 3:} Client-Servers  (One client supporting multiple
servers,  with possible multiple protocols) 
\begin{figure}[h]
\centerline{\psfig{figure=slide3.xfig.ps,height=5.0in}}
\end{figure}
\clearpage

	\item{Slide 4:} Multiple Clients and one server; clients at different
levels of complexity (e.g. PI, industry, educational)
\begin{figure}[h]
\centerline{\psfig{figure=slide4.xfig.ps,height=5.0in}}
\end{figure}
\clearpage

	\item{Slide 5:} Multiple Clients and Multiple Servers.  Servers must
interpret queries and retrieve data.  
\begin{figure}[h]
\centerline{\psfig{figure=slide5.xfig.ps,height=5.0in}}
\end{figure}


\smallskip
{\bf R. Chinman -} if master directory acts as repository of data, is it
serving as a  layer between two clients?   
Issues - locating information; clients/local dictionary; master dictionary 
server; dictionaries in servers; polling

\smallskip
{\bf H. Debaugh -} can we have an additional "open" server between clients.  
(will be addressed in subsequent slide)
\clearpage
\item{Slide 6:}
	   \begin{itemize}
		\item Types of Queries --
	  	   \begin{itemize}
			\item sub-selections
			\item other SQL functions
			\item incremental requests
	  	   \end{itemize}

		\item Types of Responses--
	  	   \begin{itemize}
			\item documentation of data set
			\item identifiers
			\item attributes
			\item values/types (real numbers, integers..)
			\item structure (most oceanographic data is not highly
structured) 
   	   	   \end{itemize}

\begin{figure}[h]
\centerline{\psfig{figure=slide6.xfig.ps,height=5.0in}}
\end{figure}
\clearpage
	   \end{itemize}
\item{Slide 7:} Data Objects -- Object Oriented Programming overview,
	\begin{itemize}
		\item program is built up of objects
		\item encapsulation (data hiding - you do not see internal
information) 
		\item polymorphism
		\item different objects accept same message but react
appropriately and differently

		\item Inheritance -- you can have a "parent" class with its
own methods and data, and a sub-class with less capability.  If the lower
level cannot process the request, it will be passed up to the "parent" for
processing.  
	\end{itemize}

\begin{figure}[h]
\centerline{\psfig{figure=slide7.xfig.ps,height=5.0in}}
\end{figure}
\clearpage

\item{Slide 8:} Object Oriented Data Bases (OODB) --
	\begin{itemize}
		\item encapsulation
		\item polymorphism
		\item inheritance
		\item nested structure (inheritance)
		\item query optimization
		\item extensible
		\item new data types
		\item new operations
	\end{itemize}

{\em The vision which we are looking for is to define an appropriate 
combinations of the "clients-servers" structure in an OODB environment.}
\normalsize

\begin{figure}[h]
\centerline{\psfig{figure=slide8.xfig.ps,height=5.0in}}
\end{figure}
\clearpage
\item{Slide 9:} Other Aspects--
	\begin{description}
		\item{I.} Pre- and post- filters  (data compression)

		\item{II.} Telephone Operators - one object may talk to
multiple objects or sets to get the information out.  Standard way to get
data if you don't know where it is would be to call up master directory
which will locate the data you want.  Remote directory may update your local
directory once the data are located. - keeping local system current will be a
challenge, depending on frequency of use.
		\end{description}

\begin{figure}[h]
\centerline{\psfig{figure=slide9.xfig.ps,height=5.0in}}
\end{figure}
\clearpage
\end{description}

\medskip
{\bf H. Debaugh -} concern about data server doing compression and other tasks 
may cause problems.  There may be other machines/programs in the link.

\medskip
{\bf B. Schramm -} a variety of providers will be available.  Do we want to
have  different "classes" of servers instead of trying to put them all in one 
"box"?  

\medskip
{\bf G. McConaughy -} are "server" and "object" the same thing?\\
{\bf G. Flierl -} Yes,  at this point in discussion.

\medskip
{\bf H. Debaugh} presented viewgraph interpretation of open server. (Slide 6 
included in notes)

\medskip
{\bf G. McConaughy -} "scalability" is an issue of concern.  We want to keep 
knowledge close to the data.  "Intelligent" servers,  "minimally-
intelligent" routers, and "stupid" clients....

\medskip
{\bf T. Kelly -} open server lets you run more efficiently.  The open server only 
has to order data once, and can then redistribute the data multiple times.

\medskip
{\bf D. Collins -} the open server provides an opportunity to bypass the 
"bottleneck" once you have established a repeated need for similar data.  
The "left" direct arrow on the diagram can be used then.

\medskip
{\bf N. Soreide -} what is in the open server? Does it have metadata, or is it 
just a "traffic cop"?\\
{\bf G. Flierl -} Specifically, it would know where the 
data is, and would be able to respond to some data directly, but would 
have to go out and get other data.

\medskip
Three important functions:{\bf  Location, Routing, and Caching}

\medskip
{\bf J. Gallagher -} maintaining open server with small number of formats on 
the bottom will be easy, but once the number of formats is large this will 
be hard.

\medskip
(back to Flierl's slides)
\begin{description}
     \begin{description}
	\item{III.} Feedback, results from previous queries used to
constrain/optimize current query 
	\item{IV.} Hypertext
      \end{description}

\item{} Issues:
     \begin{itemize}
	\item which elements?
	\item directory protocol
	\item updates
	\item long-term stability 
	\item redundancy
	     \begin{itemize}
		\item versions
		\item updates
	     \end{itemize}
	\item protection/privilege
	\item credit for data sets
     \end{itemize}
\end{description}

\medskip
{\bf R. Chinman -} what happens when an oceanographer takes their workstation 
with them - can be an issue for stability of a data base which may reside 
on that system.

\medskip
{\bf D. Fulker -} if result of a query gives you location of many different 
versions of the same data, then there is a problem.  Must uniquely identify 
data sets.

\medskip
{\bf B. Douglas -} AGU has issued requirements to clearly identify data cited
in research, in a manner such that another researcher could uniquely access 
it. 

\medskip
{\bf B. Douglas -} "30-year" rule - will the data still have value in 30
years.

\medskip
{\bf W. Brown -} can't archived data be a distributed set of centers?

\medskip
{\bf G. McConaughy -} offered software (EOSDIS Version 0) as a base for the 
system we are trying to develop.

\medskip
{\bf E. Dobinson -} can we get a JGOFS overview?\\
{\bf P. Cornillon -} JGOFS is just an approach we're looking at.  It is not
the model for the system we're developing.  JGOFS requires a UNIX
workstation, there is no automatic update. There are many things about JGOFS
which should be improved. 

\medskip
{\bf B. Schramm -} CD ROMs are becoming a common data format, and would be 
especially useful for maintaining "local" data.  Can this group foster a 
format for CD ROMs?  

\medskip
{\bf J. Corbin -} Do we envision unrestricted access and no data charges for
this system?\\
{\bf P. Cornillon -} it's an issue we need to address, but one which is 
fairly easy to look at.

\medskip
{\bf L. Walstad -} there must be some restrictions so that someone doesn't ask 
to download a 30 GB data set on internet.\\
{\bf P. Cornillon -} X-Browse system allows certain access to the data for
free, but you can't take the entire image archive and download it.  The
objective is to limit how long the line is used by one person.	

\begin{latexonly}
\newpage
\end{latexonly}
\Large
\begin{center}
{\bf Afternoon session}
\end{center}
\normalsize
\vskip .25in
Broke into working groups for the afternoon

\medskip
\noindent Working Groups:
\begin{itemize}
\item System Designers
\item Data providers
\item Data Users/Scientists 
\item Data Objects
\end{itemize}
\medskip
\vskip .25in
\Large
\begin{center}
{\bf Evening Session}
\end{center}
\begin{center}
{\bf Presentations by working groups}
\end{center}
\normalsize

\medskip
\begin{center}
{\bf Data Users Working Group}
\end{center}
\begin{center}
 {W. Brown-- Chairman}
\end{center}


\medskip
\underline{Data User Requirements}

\medskip
\begin{center}
 {\bf "Data" - primary "observations" and information}
\end{center}
   \begin{description}
	\item{(O)} {\em Must provide access to both investigator and
archive-held data}
	\item{(1)} Data Selection
	   \begin{itemize}
		\item Ability to "locate" specified data in terms of:
parameter, time-space window, type, source, QUERIES 

		\item Refinement of search
		\item Coincident parameter search
		\item Flexibility to handle "unknown" parameter selections
	   \end{itemize}
	\item{(2)} Data Acquisition
	   \begin{itemize}
		\item Timeliness - electronic and/or mail delivery possible
		\item "Useful" Form - interactive and/or batch access
		\item Interface with simple programming languages
		\item Different formats
		\item Data subset selection possible
	   \end{itemize}
	\item{(3)} Redundant Data Screening

	\item{(4)} Data transfer protocol(s)-- Investigator to Archive
\end{description}
\medskip
\begin{center}
{----------------------------}
\end{center}
\vskip .25in
\begin{center}
{\bf Data Providers Working Group}
\end{center}
\begin{center}
 {R. Chinman--Chairman}
\end{center}

\medskip
\noindent Representation of the group:
\begin{itemize}
	\item NOA-NOS, FNOC
	\item NODC
	\item NOAA-NOS (Ocean, Lake Level Div)
	\item PODAAC
	\item PI
	\item NOAA-NESDIS (DMSP, ERSY, METEOSAT, GMS)
	\item UNOLS R/V Tech
	\item Global Change Data Center GDAAC
	\item TOGA COARE
\end{itemize}
Dataset Classification:
\begin{itemize}
	\item Open-ended
	\item Project (Closed)
	\item Rotating
	\item Orphan (incidental data to another study)
	\item Real-time
\end{itemize}

Providers - PI\\
Users -	Data Center\\
			repetitive data distribution tool
			data discovery tool (e.g. sociologists)

The types and kinds of data and providers and users {\em suggest/requires a 
relatively wide range of options/capabilities for distributing data}

What data providers need from or will do for the distributed ocean data 
system:
\begin{itemize}
   	\item Data for a wider community than ocean community alone
   	\item Internet based
	\item Generation and distribution of metadata critical!! including
citation info , algorithms, when and where data collected, data set version
caveats 
	\item MD-like locator needed for this system
	\item DIF-like metadata file needed for the locator, for the data
file to be findable on the system
	\item Provider-specific capabilities for instituting restrictions,
privileges, protection of datasets including log of users and activities
	\item Distribution of processing capabilities restricted to data
format translation and subsetting
\end{itemize}
\medskip
Discussion about data sets including appropriate credit for the PI who 
provided the data to the data center, and the requirement that a second 
investigator can obtain the exact same data set, apply the stated model or 
algorithm to it, and derive the same result.


\medskip
{\bf P. Cornillon -} MD is not populated as much as it should/could be because
of the excessive documentation requirements for data sets 
\medskip
\begin{itemize}
	\item Ease of installation - higher overhead of server installation
acceptable, but ease still important

	\item Data format translators necessary, with at least the following
translations: 
	   \begin{itemize}
		\item GRIB/BUFR
		\item HDF
		\item net CDF
	   \end{itemize}
\end{itemize}

\medskip
\begin{center}
{----------------------------}
\end{center}
\vskip .25in
\begin{center}
{\bf System Architecture Working Group}
\end{center}
\begin{center}
 {E. Dobinson-- Chairwoman}
\end{center}

\medskip
Basic Constraints
\begin{description}
	\item{1)} Cost - reusable code, functions,  systems
	\item{2)} Schedule - 9 working months to build system
	\item{3)} Extensibility (grow with time) and Scalability (start small
then grow dynamically)
	\item{4)} Simple to use, easy to install, easy to use
	\item{5)} Dynamic - easy to grow and administrate
\end{description}
\smallskip
Given these constraints, what is essential functionality?
\begin{itemize}
	\item Location and Search function primary - user must be able to
locate and find data (look at USGS system?)
	\item Need an order function to provide the user with the data he or
she wants..  
	\item Services a server can provide may vary, based on the data set
(BASIC CORE SYSTEM)
\end{itemize}
Discussion of Client/Server viewgraphs:

\medskip
\begin{center}
{\bf End of Day 1}
\end{center}
\bigskip
\bigskip
\begin{latexonly}
\vskip4pt\hrule height2pt\smallskip
\end{latexonly}
\bigskip
\Large
\begin{center}
{\bf Day 2:}
\end{center}
\vskip .25in
\normalsize
\smallskip
\begin{center}
{\bf Data Objects Working Group}
\end{center}
\begin{center}
{G. McConaughy-- Chairwoman} 
\end{center}

Started with an overview of existing systems.
\begin{description}
	\item{} Search to data access is range of service for most existing
systems.  EOSDIS is mostly search, JGOFS much more focused on data access.
IDBMS mostly one large data base.  
				
\smallskip
{\bf B. Schramm -} net CDF (UniData), NEONS were left out.

	\item{} EOSDIS Version 0 assumption is that you're sitting at your
client, and it sends out a search message and gets a response message based
on who it is talking to.  None of these messages is visible to user.  It is
using ODL (Object Definition Language).  A lot of info isn't seen by client.

	\item{}MEDS/Gopher system - messages coming back are menus which are
data services at sites.  Inventory search and results and data order - you
get back a "form" to fill out with data order and processing options.  Some
software is provided in the client for looking at data.  It is less of a
search system and more of an ordering system.  Burden on data provider to
hook in would be very small.

	\item{} IDBMS has 9 nodes (ORACLE-based) with centralized node with
Master Catalog where data are replicated and also used for data ingest.  Each
server holds a different type of data, user does not even know which node
data is being accessed.  Search can provide either metadata or primary data -
similar to how EOSDIS works.

	\item{} JGOFS - client has access to servers through calls.  There is
a server and a data dictionary which has a list of objects, as well as a path
to tell how to execute the method.  There will be different entries in the
dictionary for data and methods.  Tagged with the data are metadata.  The
client program connects to an executable remote -- all of the dictionary
serves to make this happen.  Metadata can be searched to come up with data
objects that may satisfy the users.  Can list all variables which can be
extracted from the data - including looking at multiple servers to do this
based upon the entry in the dictionary.  Focus of system is to support a PI.
Every server has a data dictionary for all holdings.

\end{description}

\smallskip
{\bf R. Mairs -} how would/could JGOFS send a message to EOSDIS? \\
{\bf G. Flierl -} suppose you wanted SST within some latitude bounds, 
and you seek a yes/no answer about the data. This is very much the kind of 
information EOSDIS handles.

\smallskip
{\bf P. Cornillon -} discussion about "capability" of a system, at the
frequent expense of ease of use and menu-driven operations.  EOSDIS may be
the system designed for the broad range of users who desire ease of use,
while a system like JGOFS is "harder" to use but of higher value to a PI who
has more specific data requirements.

\smallskip
{\bf B. Starek -} discussion of "query by example" on IDBMS. ARCInfo keys to 
databases (Common Production Tools - like methods).  

\smallskip
{\bf P. Cornillon -} our system must provide a very basic level of easy-to-use 
services yet have expert capability to support science users.  A modified 
JGOFS may be a good starting point.

\smallskip
{\bf G. Flierl -} discussion of JGOFS "Problems":
\begin{description}
	\item{1)} Feedback (in "client"?)

	\item{2)} Limitations on query: $<$, $=$, $>$, $<=$,  $>=$, (  ), \&, $|$ or
\~ \quad but no more than 20 strings grouped together 

	\item{3)} Data types: all ASCII doesn't fully deal with
matrices/tensors 

	\item{4)} No interactive retrieval (like XBROWSE which lets you
"look" at an image as you begin to receive it and decide if it will meet your needs .  
\end{description}
\medskip

{\bf G. Flierl -} Summary of support requirements
\begin{description}
	\item{1)} Multiple Clients/User Interfaces
	   \begin{itemize}
		\item menus/hypertext \quad\quad        ...feedback
		\item Fortran programmers \quad\quad ...simple full query
specification
	   \end{itemize}

	\item {2)} Data servers
	   \begin{description}
		\item{a)} inventory and location
		\item{b)} browse
		\item{c)} data, including points to other data 
	   \end{description}
	\item{3)} Pre/Post filters

	\item{4)} Multiple Servers-- routing

	\item{5)} PIs adding servers-- 	support multiple DBMS
\end{description}
\smallskip
{\bf L. Walstad -} How will we handle requests for 2-3 GB of data?\\
{\bf G. Flierl -} Server will respond, but indicate the data block is too
big.  It may suggest transfer by mail (tape).  

\smallskip
{\bf P. Cornillon -} list of problems he has encountered in using systems:
One person wants "SST",  a second searches for "sea surface 
temperatures".  Variable names need standardization, and we need to make 
our locator object be "intelligent"


{\bf G. Flierl -} discussion of data structure:

\begin{center}
{\bf Hierarchical Data Structure Diagram} 
\end{center}
\medskip

\Large
\begin{center}
{\bf Afternoon session}
\end{center}
\normalsize

The workshop will not try and endorse a specific data format in the 
afternoon discussions - this question cannot be resolved in a workshop of 
this scale.  

We will break up into Users and Providers.  

Discussion:  How much of the definition of communication protocol must 
this group consider?  

Developers can send out a  bit stream from their servers in some format, 
and along with this is a structure (really a series of arguments) which 
defines this format.  On the other end, the user must be able to 
"understand" this structure and apply it.

We're confusing communications protocols with the applications 
interface.  

{\bf G. Flierl -}  API has two calls---
\begin{description}
	\item{} call and get variable names (strings, numbers)
	\item{} call and get values (strings, numbers)

\begin{center}
{\bf RPC vs Data Stream Diagram}
\end{center}

	           Top figure illustrates an RPC-based system while the lower 
		illustrates the data stream approach

\smallskip
{\bf N. Soreide -} is "data volume" the amount of data (MBs) or number of data 
requests?\\
\smallskip
{\bf G. Flierl -} A lot of research is being done via individual exchange of
small data sets -- not transfers of large volumes of data from archives.  It
is important the system support these "small" users...

\smallskip
{\bf J. Gallagher -} MB and GB per day is "large"; less than MB is "small"  
(although these definitions are more commonly based on how long it takes 
to receive the data, which is driven by communications technology...)
\medskip

\noindent Data Providers Subgroup Meeting   - R. Chinman 
\smallskip
Questions:  What Objects?  What Structure?
\smallskip
What is an object? - An object includes 
\begin{description}
	\item{a.}  inventory,
	\item{b.}  browse
	\item{c.}  data  
\end{description}
		(these are its attributes).  
\smallskip
To what extent are data providers willing to provide these?  
\smallskip
Inventory (for an image, for example) might be "is there an image?"
\smallskip
Browse - defined as a mechanism for sub-sampling.

\smallskip
{\bf B. Schramm -}- We need to look at two different cases.  The Data Center 
will be significantly different from the PI in its implementation of these 
capabilities.  


DATA:	standard file format \quad\quad location \quad\quad disk space

\Large
\begin{center}
{\bf Evening Session September 30}
\end{center}
\normalsize
\smallskip
{\bf P. Cornillon} presiding 

\noindent {\bf System Functions}
\begin{description} 
	\item{} Selection
	   \begin{itemize} 
		\item Search (space/time, parameters, source, sensor, other),
cross-inventory  
		\item Refine search - optional but necessary (browse,
quality, log) 
	   \end{itemize}
	\item {} Order 
	   \begin{itemize}
		\item Means of delivery and where
		\item Timelines
		\item Format
		\item Either from the selection process or directly (
bypassing selection process)	  
	   \end{itemize}
\noindent {\bf Principles}
	   \begin{description}
		\item Must provide access to PI data sets in addition to
national archives (corollary) --- people should want to use the system to
manage their own data
	   \end{description}
\end{description}

\noindent {\bf Data Issues}
\begin{description}
	\item{} Provider of the data must be able to determine what data
objects (and order) to be delivered

	\item{} Mechanism  to resolve keyword conflicts (auto detection)

	\item{} Global naming procedure for data sets

	\item{} Central clearing house - system management (log)

	\item{} We must avoid giving people a reason to NOT want to put their
data into the server
\end{description}
\noindent {\bf Messages}

\begin{description}
	\item{} Functional description of data, e.g.,  for location or
defined range of formats
	\item{} Options available from provider
	\item{} Must allow (and in some cases encourage) a description of
data set or provide a pointer to a description (search) -- example, the "Gulf
Stream Paths" data set.
	\item{} Log File - Data set PI invites comments on data sets - central
storage 
	\item{} Be able to explore the system
\end{description}


\noindent Discussion --


{\bf D. Fulker -} provision of data needs to be in a form suitable for use in
an applications program  (API)


{\bf Data Providers  - R. Chinman}
\begin{description}
	\item {}Desire to remove all impediments to using data

	\item{}Desirable functions and structure
\begin{description}
	\item{} inventory - locator
   	\item{} browse - refinement
	\item{} data ---{\em server}--- standard format
\end{description}
\noindent inventory:
	\begin{description}
		\item{} id
		\item{} latitude
		\item{} long
		\item{} date/time
		\item{} PI
		\item{} sensor
		\item{} parameters
	\end{description}
\end{description}
cannot translate all native format data into one of the standard formats - 
PIs nor Data Centers
So... use server software to do that and inventory (if necessary) 
Use existing inventories via server translators on new system.

\noindent {\bf Benefits}


{\bf PI:}\quad fulfillment of contractual obligation to make data available
to National Data Centers, \quad access to Data Center data and other PI data

{\bf Data Center:}\quad get access to field project/PI-based data for their
users 


From the Data Providers (PI and Data Centers) will come a range of 
services


Ocean community should prioritize DATASETS and tell J. Gallagher
\smallskip

Data Centers provide an existing array of services and fold 2-3 into new 
system, e.g., EOSDIS design group, Emery EOSDIS testbed, NOAA DAC  
(includes NODC)


{\bf E. Dobinson -} question about why this system (DODS)  seems initially 
focused on large existing systems (EOSDIS, MEDS,NEONS.....) which have or 
will have established mechanisms for getting their data, instead of 
looking at developing a system uniquely but not exclusively designed to 
gain access to small, non-automated datasets held by PIs .  Response from 
P. Cornillon centered on difficulty in becoming familiar with all of these 
different client/server system (or even knowing of them as they multiply) 
, making the data received from them suitable for use on your system, and 
having to wait until some of the big systems (e.g., EOSDIS (July 94)) are 
ready to become operational. 
\medskip
\begin{center}
{\bf End of Day 2}
\end{center}
\bigskip
\bigskip
\begin{latexonly}
\vskip4pt\hrule height2pt\smallskip

\end{latexonly}\bigskip
\Large
\begin{center}
{\bf Day 3}
\end{center}
\begin{center}
{\bf Conclusions}
\end{center}
\normalsize
\medskip
{\bf G. Flierl:}

\noindent Next Steps:
\begin{description}
	\item{0)}  Meeting report(s); mailing lists and telemail

	\item{1)} Architecture Strawman / Message Strawman
	   \begin{itemize}
		\item tell Gallagher
		\item dialog or draft
		\item join the development  group
	   \end{itemize}
	\item{2)} Testbeds
	   \begin{itemize}
		\item provide data/ work with URI/MIT
		\item provide support (moral and personnel)
	   \end{itemize}
	\item{3)} Next Workshop (planned elsewhere than Alton?)
	   \begin{itemize}
		\item focus on people who want to develop servers and clients
	   \end{itemize}
	\item{4)} Colleagues
\end{description}
\medskip

Software development will be undertaken in a phased approach. (Slide 10)
\smallskip
\begin{figure}[h]
\centerline{\psfig{figure=slide10.xfig.ps,height=5.0in}}

\end{figure}
\clearpage
{\bf R. Chinman -} Is the name "DODS" appropriate? It focuses on oceans, yet 
there is a significant community outside "oceans" who may have interest 
in this work.

\smallskip
{\bf G. McConaughy -} there may be an advantage to keeping it as an "oceans" 
system, for if we broaden its basis (as implied by a more encompassing 
name) it may sound like it will try and do things which are already being 
done.

\smallskip
Discussion and refinement of system graphic (Slide 11).
\smallskip

\begin{figure}[h]
\centerline{\psfig{figure=slide11.xfig.ps,height=5.0in}}
\end{figure}
\clearpage

Is there a base level communication procedure?  Will there always be a 
match?
\begin{description}
	\item{1)} May not be efficient
	   \begin{itemize}
		\item ASCII
		\item XDR binary
	   \end{itemize}

	\item{2)} Must be buildable from DODS distribution
\end{description}


\smallskip
{\bf P. Cornillon -}  One implementation option might be to just put an 
additional server on a site.  We didn't consider this as much as we perhaps 
should have.

\smallskip
{\bf D. Fulker -} discussed his concerns about locating translators on the 
"client" side.


The meeting concluded in general agreement of the system goals and 
functionality.  

\begin{center}
{\bf End of Day 3}
\end{center}
\begin{latexonly}
\vskip4pt\hrule height2pt\smallskip
\end{latexonly}
\end{description}
%\end{document}
\newpage
\section{\bf Proposed System Architectures}

%\addtocounter{page}{27}

One of the explicit goals of the workshop was to recommend a system
architecture for the implementation of the DODS. The system developers focus
group discussed in detail several different system models on the first day.
In their report to the plenary session, they recommended the client-server
model as appropriate for the implementation of DODS. In this Appendix we
elaborate on the client-server architecture and show three different ways in
which it can be implemented.

The system shown in Figure 1 has translators located at the data servers
which translate the format of the data resource into a canonical intermediate
format used for transmission. The client component of this system reads and
parses the data stream. User programs access the parsed data using a
API. The data model implicit in the semantics of this API will closely match
that of the canonical format used for transmission (although it could, in
theory, be quite different, there is little reason to make it so).

In figure 2, the read and parse operations are moved  out of the client API
and into the data server. The data format implicit in the API is mapped to
the data resource format either directly or using an intermediate format.
Using an intermediate format would add complexity to design and might
restrict the total number of data resource formats accessible. However, the
presence of an explicit intermediate format would no doubt simplify support
for different data resource formats and (possibly) APIs. This design is also
capable of supporting limited random file access calls since it is not
constrained by a serialized intermediate format.

The system in Figure 3 is significantly different from either of two
preceding figures. Both figures 1 and 2 share a crucial feature; the
lowest level of access to the system is through an API we provide. In order
for programs to make use of the system directly they must be, at minimum,
relinked with our API. The system in Figure 3 is designed to overcome this
problem. Rather than develop a data specific API, the system in Figure 3 uses
a special file system which has translators located in both the client and
server components. The translators on the server side produce a data stream
in response to requests for data from the translators on the client side. The
translators on the client side are accessed not from an API we supply but
from the UNIX file system calls (open(), read(), ...). In this system, users
could choose from one of several translators on the client side so that the
same file could be accessed as ascii, html, or netcdf, for example. The
choice of translators would be accomplished using a special syntax for the
file names.  Implementation of this system requires modification to the Unix
kernel.

Each of the three systems pictured here represent different tradeoffs in
extensibility, generality and simplicity. The last design is the most complex
to implement and maintain, since creating a file system and supporting that
file system on several platforms is complex. System 1 and 2, however, are
fairly simple to implement. This is balanced by the relative generality of
the three basic designs. Systems 1 and 2 can only be used with programs we
(or others) explicitly modify to use the DODS API. System 3, however, provides
access to DODS data using Unix system calls (think of NFS) so programs which
access DODS data will not have to be specially modified. Existing software
would access DODS data using a special pathname which specifies the remote
data (like pathnames of NFS volumes specify remote data) and both client and
server format translators.

The above discussion presumes that arbitrary format translation can be
accomplished (i.e., that any format can be translated to any other format).
This is not true. However, if the choice of intermediate format(s) in the above
systems is made wisely, a wide variety of data formats can be supported.


\begin{figure}[btp]
\psfig{figure=fig1a.xfig.ps}
\caption{The Data Stream Client-Server Model for DODS}
\end{figure}

\begin{figure}[btp]
\psfig{figure=fig2a.xfig.ps}
\caption{The API Client-Server Model for DODS}
\end{figure}

\begin{figure}[btp]
%\postscript{4.3in}{fig3}
\psfig{figure=fig3a.xfig.ps}
\caption{The Virtual File System Client-Server Model for DODS}
\end{figure}




% clearpage tells latex to put the remaining figures on blank pages, newpage
% does not. That causes all sorts of unpleasentness (figure pop up at the top
% of the next appendix, ...). Always put clearpage at the end of a section
% with figures if you don't want the figures to bleed over into the next
% section.

\clearpage

\section{\bf Participants}

Addresses of participants can be found with their background statements.

\begin{htmlonly}
\begin{description}
\item{\htmladdnormallink{BACON, Ian}{file://localhost/home/dcz/george/tex/dods/html/bacon/bacon.html}} ian@getafix.tsg.com \\
\item{\htmladdnormallink{BASS, Bill}{file://localhost/home/dcz/george/tex/dods/html/bass/bass.html}}  bill@eos.hac.com \\
\item{\htmladdnormallink{BROWN, Wendell}{file://localhost/home/dcz/george/tex/dods/html/brown/brown.html}}  wsb@panthr.unh.edu \\
CHINMAN, Richard  chinman@ucar.ncar.edu \\
\item{\htmladdnormallink{COLLINS, Donald}{file://localhost/home/dcz/george/tex/dods/html/collins/collins.html}}  djc@shrimp.jpl.nasa.gov \\
\item{\htmladdnormallink{CORBIN, Jim}{file://localhost/home/dcz/george/tex/dods/html/corbin/corbin.html}}  corbin@cast.msstate.edu \\
\item{\htmladdnormallink{CORNILLON, Peter}{file://localhost/home/dcz/george/tex/dods/html/cornillon/cornillon.html}}  pete@petes.gso.uri.edu \\
\item{\htmladdnormallink{DEBAUGH, Henry}{file://localhost/home/dcz/george/tex/dods/html/debaugh/debaugh.html}}\\
\item{\htmladdnormallink{DOBINSON, Elaine}{file://localhost/home/dcz/george/tex/dods/html/dobinson/dobinson.html}}  elaine\_dobinson@isd.jpl.nasa.gov \\
DOUGLAS, Bruce B. DOUGLAS@omnet.nasa.gov \\
ENLOE, Yonsook  yonsook@killians.gsfc.nasa.gov \\
\item{\htmladdnormallink{FLIERL, Glenn}{file://localhost/home/dcz/george/tex/dods/html/flierl/flierl.html}}  glenn@lake.mit.edu \\
\item{\htmladdnormallink{FRANK, George}{file://localhost/home/dcz/george/tex/dods/html/frank/frank.html}}  george@galaxy.ngs.noaa.gov \\
\item{\htmladdnormallink{FULKER, Dave}{file://localhost/home/dcz/george/tex/dods/html/fulker/fulker.html}}  dfulker@unidata.ucar.edu \\
\item{\htmladdnormallink{GALLAGHER, James}{file://localhost/home/dcz/george/tex/dods/html/gallagher/gallagher.html}}  jimg@dcz.gso.uri.edu \\ 
GILL, Bob  gill@colts.nodc.noaa.gov \\
\item{\htmladdnormallink{GIVEN, Jeffrey}{file://localhost/home/dcz/george/tex/dods/html/given/given.html}}  jeff@gso.saic.com \\
\item{\htmladdnormallink{GLOVER, Dave}{file://localhost/home/dcz/george/tex/dods/html/glover/glover.html}}  david@plaid.whoi.edu \\
\item{\htmladdnormallink{HANKIN, Steve}{file://localhost/home/dcz/george/tex/dods/html/hankin/hankin.html}}  hankin@ferret.pmel.noaa.gov \\
\item{\htmladdnormallink{HOGG, Roen}{file://localhost/home/dcz/george/tex/dods/html/hogg/hogg.html}}  roen@oce.orst.edu \\
HOLBROOK, Jim J. HOLBROOK@omnet.nasa.gov \\
\item{\htmladdnormallink{IRISH, Jim}{file://localhost/home/dcz/george/tex/dods/html/irish/irish.html}}  jirish@whoi.edu \\
KELLEY, Tim  kelley@sanddunes.scd.ucar.edu \\
MAIRS, Rob  rmairs@saars1.fb4.noaa.gov \\
\item{\htmladdnormallink{MILKOWSKI, George}{file://localhost/home/dcz/george/tex/dods/html/milkowski/milkowski.html}}  george@zeno.gso.uri.edu \\
\item{\htmladdnormallink{MILLER, Chris}{file://localhost/home/dcz/george/tex/dods/html/miller/miller.html}}  miller@esdim1.nodc.noaa.gov \\
McCONAUGHY, Gail  gailmcc@boa.gsfc.nasa.gov \\
McDONALD, Ken  mcdonald@nssdca.gsfc.nasa.gov \\
NEKOVEI, Reza  reza@uri.gos.uri.edu   \\
OLSEN, Lola  olsen@eosdata.gsfc.nasa.gov \\
RHODES, Judi  OCEANOGRAPHY.SOCIETY@omnet.nasa.gov \\
\item{\htmladdnormallink{SCHRAMM, Bill W.}{file://localhost/home/dcz/george/tex/dods/html/schramm/schramm.html}} SCHRAMM@omnet.nasa.gov \\ 
SCHWENKE, George G. schwenke@se.hq.nasa.gov \\
\item{\htmladdnormallink{SOREIDE, Nancy}{file://localhost/home/dcz/george/tex/dods/html/soreide/soreide.html}}  nns@noaapmel.gov \\
\item{\htmladdnormallink{STAREK, Bob}{file://localhost/home/dcz/george/tex/dods/html/starek/starek.html}}\\
\item{\htmladdnormallink{WALSTAD, Leonard}{file://localhost/home/dcz/george/tex/dods/html/walstad/walstad.html}}  walstad@oce.orst.edu \\
\item{\htmladdnormallink{WHITE, Warren}{file://localhost/home/dcz/george/tex/dods/html/white/white.html}}  wbwhite@ucsd.edu \\
\item{\htmladdnormallink{WILSON, J. R.}{file://localhost/home/dcz/george/tex/dods/html/wilson/wilson.html}}  R.WILSON.MEDS@telemail.nasa.gov 
\end{description}
\end{htmlonly}

\begin{latexonly}
\begin{tabbing}
namezzzzzzzzzzzzzzzzzzzz \= address \kill

BACON, Ian \> {\tt ian@getafix.tsg.com} \\
BASS, Bill \> {\tt bill@eos.hac.com} \\
BROWN, Wendell \> {\tt wsb@panthr.unh.edu} \\
CHINMAN, Richard \> {\tt chinman@ghoti.coare.ncar.edu} \\
COLLINS, Donald \> {\tt djc@shrimp.jpl.nasa.gov} \\
CORBIN, Jim \> {\tt corbin@cast.msstate.edu} \\
CORNILLON, Peter \> {\tt pete@petes.gso.uri.edu} \\
DEBAUGH, Henry \\
DOBINSON, Elaine \> {\tt elaine\_dobinson@isd.jpl.nasa.gov} \\
DOUGLAS, Bruce B.\> {\tt DOUGLAS@omnet.nasa.gov} \\
ENLOE, Yonsook \> {\tt yonsook@killians.gsfc.nasa.gov} \\
FLIERL, Glenn \> {\tt glenn@lake.mit.edu} \\
FRANK, George \> {\tt george@galaxy.ngs.noaa.gov} \\
FULKER, Dave \> {\tt dfulker@unidata.ucar.edu} \\
GALLAGHER, James \> {\tt jimg@dcz.gso.uri.edu} \\
GILL, Bob \> {\tt gill@colts.nodc.noaa.gov} \\
GIVEN, Jeffrey \> {\tt jeff@gso.saic.com} \\
GLOVER, Dave \> {\tt david@plaid.whoi.edu} \\
HANKIN, Steve \> {\tt hankin@ferret.pmel.noaa.gov} \\
HOGG, Roen \> {\tt roen@oce.orst.edu} \\
HOLBROOK, Jim J.\> {\tt holbrook@pmel.noaa.gov} \\
IRISH, Jim \> {\tt jirish@whoi.edu} \\
KELLEY, Tim \> {\tt kelley@sanddunes.scd.ucar.edu} \\
MAIRS, Rob \> {\tt rmairs@saars1.fb4.noaa.gov} \\
MILKOWSKI, George \> {\tt george@zeno.gso.uri.edu} \\
MILLER, Chris \> {\tt miller@esdim1.nodc.noaa.gov} \\
McCONAUGHY, Gail \> {\tt gailmcc@boa.gsfc.nasa.gov} \\
McDONALD, Ken \> {\tt mcdonald@nssdca.gsfc.nasa.gov} \\
NEKOVEI, Reza \> {\tt reza@uri.gos.uri.edu  } \\
OLSEN, Lola \> {\tt olsen@eosdata.gsfc.nasa.gov} \\
RHODES, Judi \> {\tt OCEANOGRAPHY.SOCIETY@omnet.nasa.gov} \\
SCHRAMM, Bill W.\> {\tt SCHRAMM@omnet.nasa.gov} \\
SCHWENKE, George G.\> {\tt schwenke@se.hq.nasa.gov} \\
SOREIDE, Nancy \> {\tt nns@noaapmel.gov} \\
STAREK, Bob \\
WALSTAD, Leonard \> {\tt walstad@oce.orst.edu} \\
WHITE, Warren \> {\tt wbwhite@ucsd.edu} \\
WILSON, J R. \> {\tt R.WILSON.MEDS@telemail.nasa.gov}
\end{tabbing}

\end{latexonly}

\newpage

\section{\bf Participant Background Statements}
\begin{htmlonly}
\htmladdnormallink{BACON, Ian}{file://localhost/home/dcz/george/tex/dods/html/bacon/bacon.html} \\
\htmladdnormallink{BASS, Bill}{file://localhost/home/dcz/george/tex/dods/html/bass/bass.html} \\
\htmladdnormallink{BROWN, Wendell}{file://localhost/home/dcz/george/tex/dods/html/brown/brown.html} \\
\htmladdnormallink{COLLINS, Donald}{file://localhost/home/dcz/george/tex/dods/html/collins/collins.html} \\
\htmladdnormallink{CORBIN, Jim}{file://localhost/home/dcz/george/tex/dods/html/corbin/corbin.html} \\
\htmladdnormallink{CORNILLON, Peter}{file://localhost/home/dcz/george/tex/dods/html/cornillon/cornillon.html} \\
\htmladdnormallink{DEBAUGH, Henry}{file://localhost/home/dcz/george/tex/dods/html/debaugh/debaugh.html}\\
\htmladdnormallink{DOBINSON, Elaine}{file://localhost/home/dcz/george/tex/dods/html/dobinson/dobinson.html} \\
\htmladdnormallink{FLIERL, Glenn}{file://localhost/home/dcz/george/tex/dods/html/flierl/flierl.html} \\
\htmladdnormallink{FRANK, George}{file://localhost/home/dcz/george/tex/dods/html/frank/frank.html} \\
\htmladdnormallink{FULKER, Dave}{file://localhost/home/dcz/george/tex/dods/html/fulker/fulker.html} \\
\htmladdnormallink{GALLAGHER, James}{file://localhost/home/dcz/george/tex/dods/html/gallagher/gallagher.html}\\ 
\htmladdnormallink{GIVEN, Jeffrey}{file://localhost/home/dcz/george/tex/dods/html/given/given.html} \\
\htmladdnormallink{GLOVER, Dave}{file://localhost/home/dcz/george/tex/dods/html/glover/glover.html} \\
\htmladdnormallink{HANKIN, Steve}{file://localhost/home/dcz/george/tex/dods/html/hankin/hankin.html} \\
\htmladdnormallink{HOGG, Roen}{file://localhost/home/dcz/george/tex/dods/html/hogg/hogg.html} \\
\htmladdnormallink{IRISH, Jim}{file://localhost/home/dcz/george/tex/dods/html/irish/irish.html} \\
\htmladdnormallink{MILKOWSKI, George}{file://localhost/home/dcz/george/tex/dods/html/milkowski/milkowski.html} \\
\htmladdnormallink{MILLER, Chris}{file://localhost/home/dcz/george/tex/dods/html/miller/miller.html} \\
\htmladdnormallink{SCHRAMM, Bill W.}{file://localhost/home/dcz/george/tex/dods/html/schramm/schramm.html}\\ 
\htmladdnormallink{SOREIDE, Nancy}{file://localhost/home/dcz/george/tex/dods/html/soreide/soreide.html} \\
\htmladdnormallink{STAREK, Bob}{file://localhost/home/dcz/george/tex/dods/html/starek/starek.html}\\
\htmladdnormallink{WALSTAD, Leonard}{file://localhost/home/dcz/george/tex/dods/html/walstad/walstad.html} \\
\htmladdnormallink{WHITE, Warren}{file://localhost/home/dcz/george/tex/dods/html/white/white.html} \\
\htmladdnormallink{WILSON, J. R.}{file://localhost/home/dcz/george/tex/dods/html/wilson/wilson.html} \\
 
\end{htmlonly}

% Input files for Participants abstracts in Latex Version of Document
\begin{latexonly}
\begin{center}
\LARGE
 {\bf Ian Bacon}
\end{center}

\bigskip
\large
\noindent{\bf Personal-}

%\externalref{bass_src} {\bf Bill Bass}
\normalsize
\smallskip
\begin{description}
\item{Name:}  Ian Bacon
\item{Title:}  Program Manager, NASA Programs
\item{Affiliation:}  Telos Systems Group
\item{Address:}  14585 Avion Parkway, Chantilly, VA  22021
\item{email:}  ian@getafix.tsg.com
\item{phone:}  (703)802-1730
\item{fax:}  (703)802-0718
\end{description}
\medskip
\large
\noindent{\bf Data System-}
\normalsize
\medskip
\begin{description}

\item{Data system name:}  Space Flight Operations Center, JPL
\item{Discipline:}  Planetary Science
\item{Data Managed:}
	\begin{description}
	\item{Type of Data:}  all types returned from deep space probes
	\item{Inventory Meta Data [Y/N]:}  Y
	\item{Digital Data, Data Products [Y/N]:}  Y
	\item{Number of Data Granules:}
	\item{Total volume of Data [Megabytes]:}  more like Gigabytes
	\end{description}
\end{description}

\medskip
\large
\noindent {\bf Data Management Activities Summary-}
\normalsize

\medskip
 	Telos has experience with both distributed data systems and the
  object oriented world, through our work at JPL, and also through internal
  R\&D projects.  At JPL, we have been responsible for the Space Flight
  Operations Centre, previously called SFOC, now called AMMOS. This is a
  distributed system, comprised of over 500 UNIX workstations, spread across
  the country, networked together over the Internet. Telos designed a
  significant portion of this system. It is used to control spacecraft, but
  also to collect the level zero data, process them to level 1, and make them
  available in quick-look form to interested parties.  We have also been
  responsible for the Command Sequence Generator, used for Magellan and other
  systems. It takes maneuver profiles for various spacecraft, and converts
  them into the command sequences which are uploaded to the vehicles.  SeqGen
  was designed using object oriented design techniques, and was developed in
  C++. It includes a small object oriented database which was developed
  specifically for the project by Telos.

  	On the East Coast, Telos has been working with object oriented
  database systems on a project originally designed to meet the needs of one
  of the Intelligence agencies.  Project Headroom, as it is currently known,
  functions as a distributed database system, capable of extracting data from
  a variety of different database types (including Sybase and Oracle, as well
  as flat files, network databases, and others). The data can be accessed
  over a network, and are joined into a single data object, in an object
  oriented database.  We are currently working on a prototype of this system
  for use with scientific datasets. The version which we are developing
  allows analysts to select data from multiple datasets without having to
  know specifically where they are located, effectively creating a fileless
  system.  When the data are contained in monolithic files (such as HDF
  files) the system browses Metadata files, and extracts only those data
  points of interest to the analyst, so it is not be necessary to pull huge
  files over the Internet, and then manually extract the data.  Data from
  several datasets can be joined into a single data object, when there are
  areas of commonality between them (such as dates or geographical locations,
  for example). The data, once retrieved, are available for processing by
  public domain or user specific tools, which can be incorporated into the
  object oriented database as methods.  The client side includes a very
  powerful search tool, which is capable of performing contextual searches of
  text based information. We are also currently binding IDL into the client,
  to allow the analyst to use its powerful tools on the data object which the
  system builds.  Our goal would be to distribute this system over the ECS
  network, with client systems at the SCFs, and servers at each of the DAACs.

\newpage




\begin{center}
\LARGE
{\bf  Bill Bass}
\end{center}
\bigskip
\large
\noindent{\bf Personal-}
\normalsize
\smallskip
\begin{description}

\item{Name:}  Bill Bass
\item{Title:}  Principal Scientist
\item{Affiliation:}  Hughes Applied Information Systems, Inc.
\item{Address:}  1616A McCormick Drive
\item{email:}  bill@eos.hac.com
\item{phone:}  (301)925-0304
\item{fax:}  (301)925-0327
\end{description}
\medskip
\large
\noindent{\bf Data System-}
\normalsize
\medskip
\begin{description}

\item{Data system name:}  Eos Data and Information System Core System 
(ECS)
\item{Discipline:}  Atmosphere, Ocean, Land, Cryosphere, Interdisciplinary
\item{Data Managed:}
	\begin{description}
	\item{Type of Data:}  From Raw Remote Sensed Data to Geophysical 
Parameters
	\item{Inventory Meta Data [Y/N]:}  Y
	\item{Digital Data, Data Products [Y/N]:}  Y
	\item{Number of Data Granules:}  200000/day
	\item{Total volume of Data [Megabytes]:}  1 TB/day
	\end{description}
\end{description}

\medskip
\large
\noindent {\bf Data Management Activities Summary-}
\normalsize
\medskip

	I am a system engineer working on the development of a data and 
information system for a repository and distribution system for a large 
quantity of remotely sensed data.  I am currently investigating means of 
generalizing this NASA system to a global change data and information 
system and, beyond than, to a system in which users may easily become 
data providers as well as data users (sometimes called User-DIS).
\newpage

\begin{center}
\LARGE
{\bf  Wendell S. Brown}
\end{center}
\large
\noindent{\bf Personal-}
\normalsize
\smallskip
\begin{description}
\item{Name:}  Wendell S. Brown
\item{Title:}  Professor of Oceanography
\item{Affiliation:}  Inst. for the Study of Earth Oceans and Space, UNH
\item{Address:}  OPAL/EOS Morse Hall, UNH, Durham, NH  03824
\item{email:}  kmg@kepler.unh.edu
\item{phone:}  (603)862-3153
\item{fax:}  (603)862-0243
\end{description}
\medskip
\large
\noindent{\bf Data System-}
\normalsize
\medskip
\begin{description}

\item{Data system name:}  Environmental Data and Information 
Management System (EDIMS)
\item{Discipline:}  Marine
\item{Data Managed:}
	\begin{description}
	\item{Type of Data:}	NOAA/realtime meteorological data
	\item{Inventory Meta Data [Y/N]:}  y
	\item{Digital Data, Data Products ]Y/N]:}  y
	\item{Number of Data Granules:}  8 files/day
	\item{Total volume of Data [Megabytes]:}  0.002/file
\medskip
	\item{Type of Data:}  Coastline/bathymetry
	\item{Inventory Meta Data [Y/N]:}  y
	\item{Digital Data, Data Products [Y/N]:}  y
	\item{Number of Data Granules:}  30 files
	\item{Total volume of Data [Megabytes]:}  1
\medskip
	\item{Type of Data:}  UNB riverflow data
	\item{Inventory Meta Data [Y/N]:}  y
	\item{Digital Data, Data Products [Y/N]:}  y
	\item{Number of Data Granules:}  1 file/day
	\item{Total volume of Data [Megabytes]:}  0.001/file
\medskip
	\item{Type of Data:}  AFAP dataset
	\item{Inventory Meta Data [Y/N]:}  y
	\item{Digital Data, Data Products [Y/N]:}  y
	\item{Number of Data Granules:}  1 file
	\item{Total volume of Data [Megabytes]:}  30
\medskip
	\item{Type of Data:}  Massachusetts Bay dataset
	\item{Inventory Meta Data [Y/N]:}  y
	\item{Digital Data, Data Products [Y/N]:}  y
	\item{Number of Data Granules:}  745 files
	\item{Total volume of Data [Megabytes]:}  8
\medskip
	\item{Type of Data:}  NOAA/NOCN SST dataset
	\item{Inventory Meta Data [Y/N]:}  y
	\item{Digital Data, Data Products [Y/N]:}  y
	\item{Number of Data Granules:}  0-8 files/day
	\item{Total volume of Data [Megabytes]:}  0.3/file
	\end{description}
\end{description}

\medskip
\large
\noindent {\bf Data Management Activities Summary-}
\normalsize
\bigskip

\noindent {\bf EDIMS CONTEXT}

	The Gulf of Maine Council on the Marine Environment (GOM/CME) has 
developed a ten-year Gulf Action Plan to address issues concerning the 
environmental health of the Gulf and management of the marine resource.  
Toward that end, the GOM/CME Working Group established a Data and 
Information Management Committee (DIMC) to develop an Environmental 
Data and Information Management System (EDIMS) for the Gulf of Maine 
region.  I was chosen to lead a University of New Hampshire (UNH) effort 
to design and implement a prototype EDIMS which is now available for 
broad community use.

\bigskip
\noindent {\bf EDIMS CONCEPT}

	The EDIMS is designed to make a data directory and a relevant set of 
data bases accessible to EDIMS users.  The prototype EDIMS is designed to 
be simple and flexible enough to accommodate anticipated changes.  We 
expect that the eventual user group will include marine environment and 
resource managers; state and provincial planners; and ocean scientists and 
engineers.  As users become more familiar with the prototype EDIMS they 
will articulate their needs more clearly.  The development of EDIMS will 
be guided by that feedback.

	The fully operational EDIMS will need to deliver to its users a large 
and diverse blend of archived and real-time data from operational and 
research sources.  Thus EDIMS is structured around a decentralized 
database.  Figure 1 illustrates schematically how data users/suppliers 
from the states, provinces and federal agencies will be able to exchange 
data and information via the Internet network.  This approach relies on the 
effort and resources of the data users/suppliers and thus can be expanded 
relatively easily.

	While the EDIMS data directory and a few special data sets will 
reside at a host computer site, most of the EDIMS databases will reside at 
remote locations.  EDIMS data users/suppliers will link to the network and 
thus have access to all of the information and data at the EDIMS host site 
(presently UNH), as well as the remote EDIMS sites.  As envisioned, EDIMS 
data users/suppliers will be asked to assume a major share of the 
responsibility for database quality and maintenance.  They will be 
assisted by an EDIMS manager at the host site who will oversee the 
operations of the EDIMS and implement improvements to EDIMS.  We expect 
that different data users/suppliers in the region will commit to and 
support such an EDIMS because of their need to access the comprehensive 
EDIMS database.  This information will enable them to conduct research, 
protect public health, and/or manage the Gulf of Maine marine resource 
better than ever before.

\bigskip
\noindent {\bf PROTOTYPE EDIMS}

	A UNH development team has constructed and implemented the 
prototype EDIMS.  The prototype EDIMS consists of a few representative 
databases (see below), which can be accessed by a broad user community.  
We have developed a set of protocols that will enable the group of data 
users/suppliers to (a) query a directory of the regional Gulf of Maine data 
sets; (b) electronically transfer a selected subset of these data to their 
own computing environment; and (c) communicate generally with the 
prototype EDIMS user community.  We have selected a diverse set of 
databases to be part of the prototype EDIMS.  They are:

\begin{itemize}
\item    A Gulf of Maine database directory
\item   Documentation (incl. EDIMS User Manual)
\item   Gulf of Maine maps and bathymetry
\item   Massachusetts Bays Program physical oceanographic data archive
\item   Real-time satellite imagery and meteorology from the NOAA/NOS 
		Ocean Products Division
\item   The New Brunswick Department of Environment real-time river 
		discharges
\item   The USGS sediment texture data archive
\item   Dartmouth model "data" for the Gulf of Maine
\item   The Bedford Institute of Oceanography Atlantic Fisheries 	
		Adjustment Program (AFAP) hydrographic data archive
\end{itemize}
\smallskip
	Our goal in the next six months is to add:
\smallskip
\begin{itemize}
\item   A "Who's Who" in the Gulf of Maine
\item   The Gulf Watch mussel data
\item   The EOEA Massachusetts shellfish data archive
\end{itemize}
\smallskip

	EDIMS includes a SQL/ORACLE-based data directory.  A simple query 
system enables EDIMS users to browse the EDIMS database directory for 
information in user-specified time and space domains.  This 
documentation  directory and "capture" any of the electronic EDIMS 
databases.  The prototype EDIMS electronic databases consist of a data 
description header and a flat ASCII data file.

	In the prototype Gulf of Maine database directory, documentation and 
all but the Dartmouth databases reside on the EDIMS host client/server 
computer at the University of New Hampshire.  (When fully implemented, 
most the EDIMS data bases will reside at their remote storage sites.)  
During the prototype EDIMS development, most the databases will be 
static, for the regularly updated NOAA and river discharge databases.

	Internet is the conduit for the prototype EDIMS data and information.  
It is an established and well-documented international network, with data 
and mail transfer protocols that can be implemented on a variety of 
platforms.  The Internet File Transfer Protocol (FTP) feature is used to 
retrieve selected data from the host and/or remote storage sites.  In an 
effort to keep track of EDIMS use, we will monitor access to the EDIMS.
\newpage

\begin{center}
\LARGE
{\bf  Donald Collins}
\end{center}
\large
\noindent{\bf Personal-}
\normalsize
\smallskip
\begin{description}
\item{Name:}  Donald J. Collins
\item{Title:}  Manager
\item{Affiliation:}  Physical Oceanography Distributed Active Archive 
Center
\item{Address:}  Jet Propulsion Laboratory
			m/s  300-323
			4800 Oak Grove Drive
			Pasadena, California  91109
\item{email:}  D.Collins/OMNET     djc@shrimp.jpl.nasa.gov
\item{phone:}  (818) 354-3473
\item{fax:}  (818) 393-6720
\end{description}
\medskip
\large
\noindent{\bf Data System-}
\normalsize
\medskip
\begin{description}

\item{Data system name:}  Physical Oceanography Distributed Active 
Archive Center
\item{Discipline:}  Physical Oceanography
\item{Data Managed:}
	\begin{description}
	\item{Type of Data:}  Satellite data of the oceans, including
		supporting data for verification.  Higher level data products.
	\item{Inventory Meta Data [Y/N]:}   Y
	\item{Digital Data, Data Products [Y/N]:}  Y
	\item{Number of Data Granules:}
	\item{Total volume of Data [Megabytes]:}  10 Tb by 1995
	\end{description}
\end{description}

\medskip
\large
\noindent {\bf Data Management Activities Summary-}
\normalsize
\medskip

	The goal of the PO.DAAC is to serve the needs of the oceanographic, 
geophysical, and interdisciplinary science communities which require 
physical information about the oceans.  This goal will be accomplished 
through the acquisition, processing, archiving, and distribution of data 
obtained through remote sensing, or by conventional means, and through 
the provision of higher level data products to the scientific community.

	The PO.DAAC presently serves the broad scientific community, 
responding, without charge to the user, to requests for complete data sets 
and for subsets of data based on temporal and spatial criteria specified by 
the user. The PO.DAAC is responsible for the acquisition of well 
documented satellite ocean data products at all levels, from existing 
visible, infrared, passive and active microwave sensors, the distribution 
of these holdings to the scientific community, and the provision of data 
product documentation.

	The Earth Observing System (EOS) Project at the Goddard Space 
Flight Center (GSFC) contains as one element the Earth Science Data and 
Information System (ESDIS).  The ESDIS has been formulated as a 
distributed system, consisting of central functions at GSFC and 
Distributed Active Archive Centers (DAAC) at eight sites throughout the 
United States.  The Jet Propulsion Laboratory (JPL) has been designated as 
the site for the Physical Oceanography Distributed Active Archive Center 
(PO.DAAC).  The activities at each of the DAACs have been separated into 
Version 0 activities and into later versions.  The concept for Version 0 is 
that these activities evolve from the capabilities of the present systems 
that have existed at the selected sites from discipline specific data 
systems and data activities.  Version 0 will be operational in July, 1994 
as a, "working prototype with operational elements", with prototype 
capabilities for an Information Management System (IMS), a Data Archive 
and Distribution System (DADS), and a Product Generation System (PGS).   

	Version 1 will be operational at the end of FY-97, utilizing hardware 
and software delivered by the EOS Core System contractor.  The transition 
between Version 0 and Version 1 will be conducted without an 
interruption of service to the scientific community.  Version 1 operations 
will be conducted by the PO.DAAC.

	The activities of the PO.DAAC are focused on the provision of pre-
EOS data sets to the scientific community during Version 0, on the 
establishment of an operational status for the PO.DAAC by July, 1994, and 
on the transition between Version 0 and Version 1.

	The PO.DAAC will provide the data archive and distribution services 
for the \\TOPEX/POSEIDON mission, including the generation and publication 
of the Merged Geophysical Data Record, through the end of the nominal 
mission in August, 1995, and through the end of any extended mission 
period, presently assumed to be August, 1997.

	The PO.DAAC will archive and distribute the data from the AVHRR 
Oceans Pathfinder, and will assume the continued AVHRR Oceans 
Pathfinder data production following the initial production of the data 
sets for the period 1981-1995 by the Pathfinder Task.  The transition is 
assumed to occur at the beginning of FY-96.  The Pathfinder Task will 
assume responsibility for a 1 km U.S. coastal data product in FY-95, and 
will continue to produce this data product after that time.

	The PO.DAAC will continue the provision of data archive and 
distribution services to the U.S. WOCE and Scatterometry teams for the 
ERS-1 low bit rate data sets, and the provision of these same services to 
the U.S. teams for ERS-2, with a probable launch in mid-1995.

	The PO.DAAC will assume responsibility for data packaging and 
formats for the NSCAT data products, and for the archiving and 
distribution of these products to the NSCAT Science Working Team, and to 
the scientific community.  The PO.DAAC activities include preparation for, 
and the support of, the NSCAT mission, with launch in February, 1996.  The 
PO.DAAC will also assume responsibility for the development of higher 
level data products as determined by the SWT and the PO.DAAC User 
Working Group.

	The PO.DAAC will be responsible for data processing, archiving, and 
distribution for the EOS Altimeter mission, including all data products 
from level 0 through the Sensor Data Record and Geophysical Data Record.  
This responsibility will include the production of value added data 
products, and the archiving of ancillary data and algorithms required for 
reprocessing of data.  The extent of the PO.DAAC role will be determined 
during the Phase A and Phase B studies, scheduled to begin in late FY-93.  
During this period, the respective roles of the PO.DAAC and NOAA will be 
determined relative to the mission data.

	The PO.DAAC will be responsible for data processing, archiving, and 
distribution for the SeaWinds  mission, including all data products from 
level 0 through the Sensor Data Record and Geophysical Data Record.  This 
responsibility will include the production of value added data products, 
and the archiving of ancillary data and algorithms required for 
reprocessing of the data.  The extent of the PO.DAAC role will be 
determined during the Phase A and Phase B studies, scheduled to begin in 
late FY-94.  During this period, the role of the PO.DAAC will be determined 
relative to the mission data.

	The PO.DAAC will continue to publish the TOGA CD-ROM series 
throughout the International TOGA period, ending in FY-96.  The PO.DAAC 
will publish other data sets as recommended by the Science Working 
Teams of the missions which we support, and by the User Working Group, 
including the SSM/I Oceans Products in FY-94, the Altimetric CD-ROM in 
FY-95, the West Coast Time Series CD-ROM in FY-95, and additional data 
sets as identified.
\newpage

\begin{center}
\LARGE
{\bf  James H. Corbin}
\end{center}
\large
\noindent{\bf Personal-}
\normalsize
\smallskip
\begin{description}
\item{Name:}  James H. Corbin
\item{Title:}  Director
\item{Affiliation:}  Center for Air Sea Technology, Mississippi State 
University
\item{Address:}  MSU - CAST, Bldg. 1103, Room 233, Stennis Space Center, 
MS  39529-5005
\item{email:}  j.corbin (OMNET)
		corbin@cast.msstate.edu (INTERNET)
\item{phone:}  (601)688-2561
\item{fax:}  (601)688-7100
\end{description}
\medskip
\large
\noindent{\bf Data System-}
\normalsize
\medskip
\begin{description}

\item{Data System Name:}  Navy Environmental Observational Nowcast 
System (NEONS)
\item{Discipline:}  Oceanography, Meteorology
\item{Data Managed:}
	\begin{description}
	\item{Type of Data:}  Operational model runs from FNOC
	\item{Inventory Meta Data [Y/N]:}  N (?)
	\item{Digital Data, Data Products [Y/N]:}  Y
	\item{Number of Data Granules:}  ~150 per day; X/Y grid
	\item{Total Volume of Data [Megabytes]:}  ~20MB per day; 21 day 
		rotating archive
\medskip
	\item{Type of Data:}  Model results from Data Assimilation and 
		Model Evaluation Experiments (DAMEE) project
	\item{Inventory Meta Data [Y/N]:}  Y (?)
	\item{Digital Data, Data Products [Y/N]:}  Y
	\item{Number of Data Granules:}  (?)  ~10**5
	\item{Total Volume of Data [Megabytes]:}  (?)  ~GB
\medskip
	\item{Type of Data:}  Observational data for verification in DAMEE
	\item{Inventory Meta Data [Y/N]:}  Y  (?)
	\item{Digital Data, Data Products [Y/N]:}  Y
	\item{Number of Data Granules:}  (?)  ~10**5; profiles, tracks, etc.
	\item{Total Volume of Data [Megabytes]:}  (?)
	\end{description}
\end{description}

\medskip
\large
\noindent {\bf Data Management Activities Summary-}
\normalsize
\medskip

	A major effort of CAST is development of distributed data-bases 
and data systems.  For the last four years the CAST research staff has 
made major contributions to the enhancement and applications of the 
Naval Environmental Operational Nowcast System (NEONS) developed by 
the Naval Research Lab at Monterey California.  NEONS is an interface to a 
relational database management system.  It provides "C" and "FORTRAN" 
Applications Programming Interface (API) to the RDBMS, and also models 
typical oceanographic data types, namely grids, point observations in 
time, profiles, tracks, and images.  Through CAST efforts, NEONS has been 
implemented at the U.S. Naval Oceanographic Office (NAVOCEANO), the 
Fleet Numerical Oceanography Center (FNOC), the National Climate Data 
Center, and the Naval Research Lab Stennis Space Center (NRL/SSC) among 
others.  Through the support efforts to NRL/SSC, NAVOCEANO and FNOC, 
CAST will have near real-time access to meteorological and 
oceanographic observational and Navy operational model output data sets 
that could be included in the distributed system.

	CAST is also in the final developmental phases of a "client-server" 
version of NEONS called "netNEONS" which obviates the need for running a 
local RDBMS.  This allows local client applications to access the remote 
RDBMS transparently via the remote NEONS server.  The experience gained 
with netNEONS, coupled with what CAST has learned in the development of 
a prototype Network Data Browser and associated applications could prove 
valuable input to this workshop.  This prototype allows the scientists to 
browse and retrieve model output fields stored in UNIX files scattered 
across the network.  The meta information on the contents of the files are 
registered in an RDBMS which the user queries via a GUI.  This is a 
contents based browse system being developed for the NRL/SSC Ocean 
Sciences Group.  The NRL scientists run many numerical models of 
differing versions.  The Network Data Browser helps keep track and 
manage outputs from the different model runs for each of the different 
modeling groups and facilitates cross-group data browsing and retrieval.  
The final version is intended to be an end-to-end client-server system.

	CAST has also developed and implemented an operational data-base 
for NAVOCEANO to manage and distribute the ocean profile data from the 
Master Oceanographic Observational Data Set (MOODS). This entailed 
migrating nearly two and a half million ocean profiles from files-based 
system on UNISYS to the EMPRESS Distributed RDBMS running on a Cray Y-
MP and SUN front end.  CAST is also teaming with University of Colorado 
on development of an Altimetry Data Processing and Analysis System 
(ideally this will be a distributed system).
\newpage

\begin{center}
\LARGE
{\bf  Peter Cornillon}
\end{center}
\large
\noindent{\bf Personal-}
\normalsize
\smallskip
\begin{description}
\item{Name:}  Peter Cornillon
\item{Title:}  Professor of Oceanography
\item{Affiliation:}  URI/GSO
\item{Address:}  112 Watkins, Narragansett Bay Campus, URI, 
Narragansett, RI  02882
\item{email:}  pete@petes.gso.uri.edu
\item{phone:}  (401)792-6283
\item{fax:}  (401)792-6728
\end{description}
\medskip
\large
\noindent{\bf Data System-}
\normalsize
\medskip
\begin{description}

\item{Data System Name:}  xbrowse
\item{Discipline:}  Physical Oceanography
\medskip
\item{Data System Name:}  Global AVHRR Database
\item{Discipline:}  Physical Oceanography
\medskip
\item{Data System Name:}  InSitu
\item{Discipline:}  Physical Oceanography

\item{Data Managed:}
	\begin{description}
	\item{Total Volume of Data [Megabytes]:}  Gigabytes
	\end{description}
\end{description}

\medskip
\large
\noindent {\bf Data Management Activities Summary:}
\normalsize
\medskip

	At The University of Rhode Island I am involved with three different 
data access systems.  Each system has a different scope, but all are 
accessible using the Internet.

	The Global AVHRR database locates existing HRPT and LAC passes 
both at a number of institutions around the world.  The database is 
automatically updated using the Internet.  Users can submit SQL queries to 
the database using a captured account.  In addition, the database can be 
used to co-locate AVHRR and XBT data using a data server we developed 
and installed at NODC.  The AVHRR database currently points to over 
140,000 HRPT/LAC passes.

	The Xbrowse system provides realtime access to our 20,000+ 
archive of 5km AVHRR data of the western North Atlantic.  This client-
server system provides the user with a simple interface to a flatfile 
database which can be searched by date only.  In response to a query the 
user is presented with a list of image names, any one of which may be 
examined.  Xbrowse uses progressive transmission to ameliorate different 
(often low) levels of available network bandwidth at users sites.

	We have also developed a set of small in situ data servers which are 
accessed transparently using a captured account.  This system provides 
access to several different types of in situ and model data.  It uses a 
hierarchical searching mechanism specifically tailored to these data sets.

	In addition, we have an extensive in house archive of satellite data, 
both processed and raw.
\newpage

\begin{center}
\LARGE
{\bf  Henry A. Debaugh}
\end{center}
\large
\noindent{\bf Personal-}
\normalsize
\smallskip
\begin{description}
\item{Name:}  Henry A. Debaugh
\item{Title:}  Database Administrator for the Ocean and Lake Levels 
Division (OLLD)
\item{Affiliation:}  NOAA
\item{Address:}  NOAA/NOS, Routing code:  N/OES2x1, Room 7209, 1305 
East-West Highway, Silver Spring, MD  20910
\item{email:}  Internet not available yet (We are hoping for November.)  
Use omnet, care of d.beaumariage
\item{phone:}  (301)713-2884
\item{fax:}  (301)713-4437
\end{description}
\medskip
\large
\noindent{\bf Data System-}
\normalsize
\medskip
\begin{description}

\item{Data system name:}  National Water Level Data Center (proposed 
name)
\item{Discipline:}  Oceanography
\item{Data Managed:}
	\begin{description}
	\item{Type of Data:}  Water Level Time Series data
	\item{Inventory of Meta Data [Y/N]:}  Y
	\item{Digital Data, Data Products [Y/N]:}  Y
	\item{Number of Data Granules:}  ?  (OLLD maintains a world wide 
network of water level observation stations app:  200 at present)
	\item{Total Volume of Data:}  app. 3000 megabytes
\medskip
	\item{Type of Data:}  Ancillary Atmospheric and Oceanographic 
Times Series Data
	\item{Inventory of Meta Data [Y/N]:}  Y
	\item{Digital Data, Data Products [Y/N]:}  Y
	\item{Number of Data Granules:}  ?  (OLLD maintains a world wide 
network of water level observation stations app:  200 at present)
	\item{Total Volume of Data:}  app. 1000 megabytes
\medskip
	\item{Type of Data:}  Water Temperature \& Density
	\item{Inventory of Meta Data [Y/N]:}  Y
	\item{Digital Data, Data Products [Y/N]:}  Y
	\item{Number of Data Granules:}  ?  (OLLD maintains a world wide 
network of water level observation stations app. 90 collect Temperature 
\& Density data)
	\item{Total Volume of Data:}  app. 200 megabytes
	\end{description}
\end{description}

\medskip
\large
\noindent {\bf Data Management Activities Summary:}
\normalsize
\medskip

	I am the database administrator for the Ocean and Lake Levels 
Division.  I have a primary responsibility for the management of OLLD's 
data resources.

\medskip
OVERVIEW OF OLLD'S NATIONAL WATER LEVEL DATA CENTER \footnote {proposed 
name}

\medskip
	The Ocean and Lake Levels Division in the Office of Ocean and Earth 
Sciences, National Ocean Service, NOAA, is presently developing the Data 
Processing and Analysis Subsystem (DPAS), a critical component of the 
Next Generation Water Level Measurement System (NGWLMS).  DPAS is a 
fully integrated, state-of-the-art computer system that will perform the 
following functions of the NGWLMS; data acquisition, data processing, 
analysis and quality control, database management, field requirements 
assessments, logistics control, administrative activities, and data 
dissemination.  DPAS is scheduled to be completed early in 1994 and 
operational after thorough system testing has been successfully 
completed.  Some parts of DPAS, such as data acquisition functions, have 
been operational since early in 1992.  DPAS will provide on-line access to 
the Division's very large and valuable database for both in-house and 
external users.  OLLD analysts and oceanographers will use DPAS software 
to derive standard Division products and also to perform ad-hoc analyses 
and perform specialized services.

	The Division is responsible for the collection and subsequent 
processing and analysis of water level and related data from the coastal 
areas of the United States, including Alaska, Hawaii, and U.S. territories 
in the Pacific, and the Great Lakes.  Through OLLD's participation in the 
Global Sea Level Program field stations have already been installed in 
several foreign nations and more installations are planned for the near 
future.  NGWLMS field units, which are replacing existing water level 
gages based on antiquated technologies, automatically measure and record 
water level data and associated data quality assurance parameters, and 
other ancillary environmental data (such as wind speed, direction and 
gusts, barometric pressure, etc.).  These data are transmitted from each 
remote site every three hours via GOES satellites to NESDIS satellite 
downlink facilities at Wallops Island, Virginia.  DPAS automatically calls 
NESDIS every hour to download data collected since the last call and then 
automatically performs preliminary quality control checks and generates 
several reports which advise NOS personnel on the operational status of 
the entire system.  Every 2 weeks (as presently scheduled) DPAS also 
automatically interrogates they field units directly and downloads data 
that was not collected via satellite transmission for whatever reason.

	The nucleus of DPAS is Sybase, a high-performance relational 
database management system (DBMS) which manages and maintains OLLD's 
database which will exceed 10Gb of data in a few years.  This database 
will store and maintain not only data from the new NGWLMS field units but 
much of OLLD's historical data as well.  Sybase provides many other 
important features such as security and control, server enforced 
integrity, high data availability, and window-based tools.  Sybase also 
runs on a variety of hardware platforms and operating systems making it 
highly portable and providing an easy migration path to more powerful 
platforms.  Sybase is used in a client-server architecture.  For DPAS 
development, a VAX 4000 Model 500 computer acts as a database server 
running Sybase server software; this processor will be upgraded when the 
system becomes fully operational to increase performance.  Other VAXes 
are networked to the database server to provide various network services.  
Application software, consisting of integrated commercial and 
customized software, runs on client workstations which are 486-based 
PC's running the OS/2 operating system for in-house work.  DPAS will 
have the flexibility to allow external client workstations with other 
hardware and software configurations to access the database server via 
wide-area network (WAN) and download selected data to be used as 
needed.

	When completed, DPAS will automatically perform many functions of 
OLLD that are now manual processes.  Routine data processing and quality 
assurance tasks will be done autonomously.  DPAS will provide more 
capabilities to both in-house and external users and allow ad-hoc analyses 
to be accomplished with relative ease.  The extensive data archive of OLLD 
will be directly and readily available to external users and data from 
NGWLMS field units will become available in near-real time.
\newpage

\begin{center}
\LARGE
{\bf  Elaine Dobinson}
\end{center}
\large
\noindent{\bf Personal-}
\normalsize
\smallskip
\begin{description}
\item{Name:}  Elaine Dobinson
\item{Title:}  PO.DAAC Deputy Task Manager
\item{Affiliation:}  JPL
\item{Address:}  4800 Oak Grove Drive, Pasadena, A  91109
\item{email:}  elaine\_dobinson@isd.jpl.nasa.gov
\item{phone:}  (818)306-6269
\item{fax:}  (818)306-6929
\end{description}
\medskip
\large
\noindent{\bf Data System-}
\normalsize
\medskip
\begin{description}
\item{Data System Name:}  EOSDIS Physical Oceanography DAAC
\item{Discipline:}  Computer Science
\item{Data Managed:}
	\begin{description}
	\item{Type of Data:}  Physical Oceanographic Data Products
	\item{Inventory Meta Data [Y/N]:}  Y
	\item{Digital Data, Data Products [Y/N]:}  Y
	\item{Number of Data Granules:}  Approx. 30,000 Granules
	\item{Total Volume of Data [Megabytes]:}  Approx. 125 Gigabytes
	\end{description}
\end{description}

\medskip
\large
\noindent {\bf Data Management Activities Summary-}
\normalsize
\medskip

	I am currently working on the Version 0 implementation of EOSDIS.  I 
led the IMS work at the PO.DAAC, co-lead the IMS system-level data 
dictionary development activities, and am currently deputy manager for 
the DAAC.  The PO.DAAC archive is an upgrade of the NODS (NASA Oceans 
Data System) implemented in INGRES and about to be ported over to UNIX 
on SGI.

	Prior to my working on EOSDIS I was the lead database designer of 
the science data catalog for the Planetary Data System.  I am also the 
supervisor of the Archive Data Management Group in the Science Data 
Systems Section of JPL.
\newpage

\begin{center}
\LARGE
{\bf  Glenn R. Flierl}
\end{center}
\large
\noindent{\bf Personal-}
\normalsize
\smallskip
\begin{description}
\item{Name:}  Glenn R. Flierl
\item{Title:}  Professor of Physical Oceanography
\item{Affiliation:}  MIT
\item{Address:}  54-1426, MIT, Cambridge, MA 02139
\item{email:}  glenn@lake.mit.edu
\item{phone:}  (617) 344-2728
\item{fax:}
\end{description}
\medskip
\large
\noindent{\bf Data System-}
\normalsize
\medskip
\begin{description}

\item{Data system name:}  JGOFS
\item{Discipline:}  Bio, Chem, Phys Oceanogr.
\item{Data Managed:}
	\begin{description}
	\item{Type of Data:}  station, time-series, model ...
	\item{Inventory Meta Data [Y/N]:}  y
	\item{Digital Data, Data Products [Y/N]:}  y
	\item{Number of Data Granules:}
	\item{Total volume of Data [Megabytes]:}  growing, but not huge
	\end{description}
\end{description}

\medskip
\large
\noindent {\bf Data Management Activities Summary-}
\medskip

\centerline{\bf A Distributed, Object-based Data Management System}
\centerline {\bf for JGOFS}
\centerline {Glenn Flierl, James Bishop, David Glover, Satish Paranjpe}

\medskip
\normalsize
Large oceanographic programs such as JGOFS (The Joint Global Ocean Flux 
Study) require data management systems which enable the exchange and 
synthesis of extremely diverse and widely spread data sets. We have 
developed a distributed, object-based data management system for 
multidisciplinary, multi-institutional programs. It provides the capability 
for all JGOFS scientists to work with the data without regard for the 
storage format or for the actual location where the data resides.  The 
approach used yields a powerful and extensible system (in the sense that 
data manipulation operations are not predefined) for managing and 
working with data from large scale, on-going field experiments.

In the ``object-based'' system, user programs obtain data by 
communicating with a program (the ``method'') which can interpret the 
particular data base.  Since the communication protocol is standard and 
can be passed over a network, user programs can obtain data from any data 
object anywhere in the system. Data base operations and data 
transformations are handled by methods which read from one or more data 
objects, process that information, and write to the user program
(or to the next filter in the series).

We have written methods for various ASCII and binary databases, and built 
transformation routines for doing mathematical operations, dynamic 
height calculations, and data joins.
\newpage

\begin{center}
\LARGE
{\bf  George M. Frank}
\end{center}
\large
\noindent{\bf Personal-}
\normalsize
\smallskip
\begin{description}
\item{Name:}  George M. Frank
\item{Title:}  National Geodetic Survey Database Administrator
\item{Affiliation:}  DOC/NOAA/NOS/C\&GS/NGS
\item{Address:}  1315 East West Highway - Station 9127
			Silver Spring, Md. 20910-3282
\item{Email:}  george@galaxy.ngs.noaa.gov
\item{Phone:}  301-713-3251
\item{Fax:}  301-713-4172
\end{description}
\medskip
\large
\noindent{\bf Data System-}
\normalsize
\medskip
\begin{description}

\item{Data System Name:}  National Geodetic Survey Integrated Data Base 
(NGSIDB)
\item{Discipline:}  Geodesy
\item{Data Managed:}
	\begin{description}
	\item{Type of Data:}  Geodetic Survey Data
	\item{Inventory Meta Data:}  Yes
	\item{Digital Data, Data Products:}  Yes
	\item{Number of Data Granules:}  
	\item{Total Volume of Data (MB):}  6000
	\end{description}
\end{description}

\medskip
\large
\noindent {\bf Data Management Activities Summary:}
\normalsize
\medskip

	The National Geodetic Survey is a Federal organization that was 
established in 1807.  It was given the responsibility to map the coastline 
of the United States so that the nation's shipping industry could safely 
navigate its waters.  This mandate has resulted in the determination of 
horizontal coordinates of latitude and longitude for over 300K points and 
the vertical coordinates for approximately 1000K points.  This 
information is the basis for the National Geodetic Reference System from 
which all U.S. mapping efforts should begin.

	The NGS data holdings consists of all the geodetic surveying 
information that NGS has accumulated from private, local and federal 
government and its own efforts during the period since 1807.  The data 
types consist of coordinates of latitude, longitude, and elevation; 
descriptive text describing the location and physical characteristics of 
the points; and observational data such as gravity, directions, distances, 
elevation differences, and satellite observations including doppler, very 
long baseline interferometry (VLBI), and global positioning system (GPS).

	The NGS data is stored and managed by a relational database 
machine, a Britton LEE IDM 700, otherwise known as a Sharebase or 
Teradata.  This database server is accessed through a local area network 
of Unix workstations and PCs using the Structured Query Language (SQL).  
SQL is used in either an interactive mode or in Fortran or C application 
programs.  The NGS LAN is a TCP/IP twisted pair ethernet consisting of 
over 100 PCs and 35 UNIX workstations such as Suns, HPs, and Sun clones.  
The NGS LAN can be accessed with dial-in capability or through Internet.

	For those outside users that have not been granted direct access to 
the NGS LAN or database system, information can be obtained by 
contacting the NGS Information Center by telephone.  The Information 
Center provides a wide variety of data in various formats.  GPS orbital 
data which is not in the database can be obtained through the Information 
Center or through the Coast Guard bulletin board.

	The NGS database system is currently undergoing a transition to a 
Sybase DBMS system that resides on a Sun multiprocessor server.  Sybase 
possesses features such as ANSI SQL and Open Server, that will permit 
NGS to more easily share its data with other database systems.  The 
multiprocessing capability of this system will also increase the 
performance and therefore the availability of the NGS data.

	NGS and another organization within NOAA, the Ocean Lake and Level 
Division (OLLD),is proposing to create a distributed data system.  NGS and 
OLLD independently maintain their own data holdings using Sybase DBMS.  
Each realizes that the other possesses data that they could use.  It is their 
goal to create a system that would provide a link between them.  This 
would require the creation of metadata for their databases, a network 
accessible graphical user interface (GUI) that would allow users to locate 
and obtain data easily and rapidly, and the physical link that could be 
provided through Internet or dial-in access.
\newpage

\begin{center}
\LARGE
{\bf  David Fulker}
\end{center}
\large
\noindent{\bf Personal-}
\normalsize
\smallskip
\begin{description}
\item{Name:}  David Fulker
\item{Title:}  Director, Unidata Program Center
\item{Affiliation:}  University Corporation for Atmospheric Research 
(UCAR)
\item{Address:}  P.O. Box 3000, Boulder, CO  80307 
			or for UPS, etc:  3300 Mitchell Lane, Suite 170, Boulder, 
			CO  80301
\item{email:}  fulker@unidata.ucar.edu
\item{phone:}  (303)497-8650
\item{fax:}  (3)497-8690
\end{description}
\medskip
\large
\noindent{\bf Data System-}
\normalsize
\medskip
\begin{description}

\item{Data System Name:}  Unidata
\item{Discipline:}  Atmospheric Science
\item{Data Managed:}
	\begin{description}
	\item{Type of Data:}  Surface, Soundings, Grids, Images
	\item{Inventory Meta Data [Y/N]:}  N
	\item{Digital Data, Data Products [Y/N]:}  Y
	\item{Number of Data Granules:}  NA (real-time flow)
	\item{Total Volume of Data [Megabytes]:}  >100 MB/day
	\end{description}
\end{description}

\medskip
\large
\noindent {\bf Data Management Activities Summary-}
\normalsize
\medskip
\large
\begin{description}
\item{\bf *} {\bf UNIDATA Overview Factsheet}
\end{description}
\normalsize

	Unidata is a nationwide program to help university departments 
acquire and use atmospheric data.  The Unidata Program Center (UPC) is 
managed by the University Corporation for Atmospheric Research (UCAR) 
in Boulder, Colorado, and sponsored by the National Science Foundation 
(NSF).  Personnel and other resources required for university participation 
are provided by the institutions themselves.  The NSF and university 
resources are complemented by contributions from private industry.

\bigskip
\noindent{\bf Unidata Provides:}
\begin{itemize}
	\item Real-time weather data (via satellite) at group discount rates;
	\item Software to display and analyze those data;
	\item Consultation on the necessary hardware and software;
	\item Training workshops on how to install and use Unidata software;
	\item Ongoing support by and for the Unidata community of users.
\end{itemize}

\noindent {\bf Products and Services}
\begin{itemize}
\item Current Data via Satellite:

\noindent Established Data Products.  Unidata participants receive National 
Weather Service data at discounted rates.  These include the Domestic 
Data Service (conventional surface and upper-air observations for the 
U.S.), the International Data Service, and the Numerical Product Service 
(gridded analyses and forecasts from the National Meteorological Center 
and the European Centre fro Medium-Range Weather Forecasting).

\noindent Research Data Products.  A special Unidata broadcast channel carries 
a range of data prepared at the University of Wisconsin-Madison under
contract with the UPC.  These include satellite and radar images, as well as
the conventional meteorological data used with the Unidata McIDAS software/
This channel also carries special products not generally available through
the National Weather Service, such as wind profiler data.

\item Software Tools:

\noindent netCDF [described below]

\noindent The Local Data Manager (LDM) [described below]

\end{itemize}

	To licensed universities, Unidata distributes software applications 
for displaying and analyzing the data captured by the LDM.  These packages 
are WXP, the Weather Processor (developed by Purdue University), GEMPAK 
(developed at NASA/Goddard Space Flight Center), and YNOT (developed by 
MacDonald Dettwiler and Associates).  Unidata also distributes McIDAS-X 
and McIDAS-OS2 (developed at the University of Wisconsin-Madison Space 
Science and Engineering Center), which analyze and display data from the 
special Unidata/Wisconsin broadcast channel.

\bigskip
\noindent {\bf Support:}
\medskip

	The Unidata Program Center provides full support for all the 
software packages it distributes.  Full support includes:  consultation, 
training workshops, software maintenance, and documentation.

\noindent Costs:

	Unidata software, training, and support are provided at no charge to 
universities. Unidata arranges a group discount rate for data services.  
Universities assume the costs of purchasing equipment, hiring site and 
system administrators, subscribing to data services, and all travel and 
accommodations associated with training workshops.

	Some of the Unidata activities most relevant to a distributed data 
systems workshop are associated with two software systems, netCDF and 
LDM, described below.

\large
\begin{description}
\item{\bf *} {\bf UNIDATA netCDF Factsheet}
\end{description}
\normalsize

\noindent {\bf Overview:}
\medskip

	The Network Common Data Form, or netCDF, package is a software 
package that standardizes how scientific data are stored and retrieved. 
More than a data format, the netCDF package is a set of programming 
interfaces that can be used with widely varying scientific data sets by 
machines of widely varying architecture.

\bigskip
\noindent {\bf Features}
\begin{itemize}
\item Standardized Data Access

\noindent Unidata's library of netCDF subroutines insulates applications from 
the underlying data format.  Multidimensional floating- point, integer, and 
character data can be stored and retrieved using these functions.  Data are 
accessed by specifying a netCDF file, a variable name, and a description of 
what part of the multidimensional data is to be accessed.  
Multidimensional data may be accessed one point at a time, in cross-
sections, or all at once.

\item Self-Describing

\noindent Variables and their associated dimensions are named.  Information 
about the data, such as what units are used what is the valid range of data 
values, can be stored in attributes associated with each variable.  The 
processing history of a data set can be stored with the data.

\item C and FORTRAN Compatible

\noindent The netCDF subroutines can be invoked from either C or FORTRAN, 
and data stored using one language may be retrieved in the other.

\item Machine-Independent

\noindent The format underlying the netCDF package employs an open standard 
known as XDR (for eXternal Data Representation) that renders netCDF files 
machine independent.  The netCDF package is particularly useful at sites 
with a mix of computers connected by a network.  Data stored on one 
computer may be read directly from another without explicit conversion.

\item Portable

\noindent The software has been used successfully on a broad range of 
computers, from PCs to supercomputers.

\item Benefits
    \begin{itemize}
	\item Reusable Applications

\noindent Unidata's purpose in creating the netCDF library is to generalize 
access to scientific data so that the methods used for storing and 
accessing data are independent of the computer architecture and the 
applications being used.  In addition, the library minimizes the fraction of 
development effort devoted to dealing with data formats.

	\item Reusable Data

\noindent Standardized data access facilitates the sharing of data.  Since the 
netCDF package is quite general, a wide variety of analysis and display 
applications can use it.  The netCDF library is suitable, for example, for 
use with satellite images, surface observations, upper-air soundings, and 
grids.  By using the netCDF package, researchers in one academic 
discipline can access and use data generated in another discipline.
    \end{itemize}
\end{itemize}

\large
\begin{description}
\item{\bf *} {\bf UNIDATA LDM Factsheet}
\end{description}
\normalsize


\begin{description}
	\item {}{\bf Overview}

\noindent Unidata's Local Data Manager (LDM) software acquires 
meteorological data and shares these data with other computers on a 
network.  The LDM handles data from National Weather Service data 
streams, including gridded data from numerical forecast models.

\noindent A client ingester handles a specific data feed.  It scans the data 
stream, determines product boundaries, and extracts products, passing 
selected products to one or more LDM servers.

\noindent The LDM server processes the raw data passed to it by one or more 
ingesters and converts that data into a form that can be used by 
applications programs.

	\item {} {\bf Features --}  The LDM is

\noindent User configurable:  the LDM server can be instructed to append a 
particular product to a file; save a product in a particular form, such as 
Unidata's netCDF; execute an arbitrary program with the data product as 
input; store or retrieve products in a simple database by key; and/or pass 
data along to other running client programs.

\noindent Site configurable:  data captured on one machine can be stored on 
other machines on a network. This means that data ingest functions can be 
separated from storage and use functions, allowing sites to tailor their 
LDM system to their capacity.

\noindent Extensible:  new client decoders can be added easily, including 
decoders for archival data.

\noindent Event-driven:  the system captures data in real time.
\end{description}

	Unidata is currently engaged in exploiting the architecture of the 
LDM software to build a system for distributing real-time data via the 
Internet.  The principle is that products will fan out from the source 
through several tiers of cooperating LDM computers, each of which relays 
data to several "neighbors" on an event-driven basis.  In this way, 
hundreds of end-user sites can be served promptly (i.e., within a few 
seconds of data arrival) without the kinds of traffic jams that would 
arise if all sites were to contact a single server at the same time (i.e., at 
the time of data arrival).

	The distributed LDM system is undergoing tests at a dozen or so 
sites, and the results are sufficiently encouraging that we are planning 
eventually to replace the satellite data broadcast service with this 
Internet Data Distribution (IDD) system.
\newpage

\begin{center}
\LARGE
{\bf  James Gallagher}
\end{center}
\large
\noindent{\bf Personal-}
\normalsize
\smallskip
\begin{description}
\item{Name:}  James Gallagher
\item{Title:}  Programmer/Analyst
\item{Affiliation:}  URI/GSO
\item{Address:}  110 Watkins, Narragansett bay campus, URI, 
Narragansett, RI. 02882
\item{email:}  jimg@dccz.gso.uri.edu
\item{phone:}  401.792.6939
\item{fax:}  401.792.6728
\end{description}
\medskip
\large
\noindent{\bf Data System-}
\normalsize
\medskip
\begin{description}

\item{Data system name:}  xbrowse
\item{Discipline:}  Physical oceanography
\item{Data Managed:}
	\begin{description}
	\item{Type of Data:}  Sea surface temperaure (SST)
	\item{Inventory Meta Data [Y/N]:}  Y
	\item{Digital Data, Data Products [Y/N]:}  Y
	\item{Number of Data Granules:}  O(10k)
	\item{Total volume of Data [Megabytes]:}  O(10)
\medskip
	\item{Type of Data:}  SST, Raw satellite
	\item{Inventory Meta Data [Y/N]:}  Y
	\item{Digital Data, Data Products [Y/N]:}  Y
	\item{Number of Data Granules:}  O(10)
	\item{Total volume of Data [Megabytes]:}  O(1)
	\end{description}
\end{description}

\medskip
\large
\noindent {\bf Data Management Activities Summary-}
\normalsize
\medskip

	Xbrowse was designed so that users who are not physically at an 
archive site can efficiently access, review and retrieve image and in-situ 
data necessary for their work. Xbrowse uses the Internet to provide real-
time access to image archives. Xbrowse is different from other remote 
browsing systems in that it provides more than a fixed resolution preview 
of each image being browsed.  With xbrowse, the user views the image at 
progressively increasing levels of resolution -- up to the resolution of the 
archived data if desired.  Images that are of no interest to the user, for 
example because of cloud cover in a region of specific interest or because 
of data drop out in part of the image, can quickly be passed by at low 
resolution.  Images of greater interest can be allowed to progress to 
higher levels of resolution.  Further, the user can stop the progressive 
transmission of the image at a (low) level of resolution and mark out an 
inset area in the main image for viewing at higher levels of resolution.  
Since researchers often need only a portion of the image area at full 
resolution, this progressive transmission/composite image approach to 
browsing makes it feasible to provide effective real-time, remote access 
to the archived data and allows xbrowse to be used as a data access tool 
and not just a data ordering tool. In addition, xbrowse can access in situ 
metadata and overlay that data on the imagery being displayed.

	The xbrowse software is based on the client/server model of 
cooperating, independent processes.  Client software, essentially the user 
interface, is installed on a user's machine.  Server software at archive 
sites responds to commands from the client software to send the data 
requested by the user.  The client software on the user's machine can 
interact with many different servers at different sites.  Although at 
present only the two data servers mentioned earlier are operational -- the 
server at the University of Rhode Island which provides AVHRR SST 
images and a server at the National Ocean Data Center which provides XBT 
metadata -- it is expected that other data servers will be added at other 
archives in the future.

	Both the client and data server are available via anonymous ftp at 
zeno.gso.uri.edu. 
\newpage

\begin{center}
\LARGE
{\bf  Jeffrey Given}
\end{center}
\large
\noindent{\bf Personal-}
\normalsize
\smallskip
\begin{description}
\item{Name:}  Jeffrey Given
\item{Title:}  Geophysicist
\item{Affiliation:}  Science Applications International Corporation
\item{Address:}  MS A-2 10260 Campus Point Drive, San Diego, CA  92121
\item{email:}  jeff@gso.saic.com
\item{phone:}  (619)458-2656
\item{fax:}  (619)458-4993
\end{description}

\medskip
\large
\noindent {\bf Data Management Activities Summary-}
\normalsize
\medskip

	Our group at Science Applications International Corporation has 
approximately 40 professionals involved with all aspects of the 
management, processing, and wide-area distribution of geophysical data.  
The representative data management activities include:
\begin{description}

\item{\bf 1.}	The Center for Seismic Studies (CSS), Arlington, VA.  This is
a seismological data center that we have developed for ARPA over the past
decade.  The activities of the CSS include (1) Near real-time, world-wide
data acquisition over a TCP/IP WAN; (2) Data processing and analysis by
independent, cooperating institutions in Norway, Russia, and the US, with
automated data migration between the sites; (3) data-centric software
architecture based on a distributed Oracle DBMS and supported by several
application interfaces; (4) many gigabytes of randomly and readily accessible
time-series data stored on an optical jukebox and tightly integrated into a
near-real time processing system; (5) User interfaces built with X/Motif for
data selection, browsing, and interpretation; (6) Distribution of these data
to the world-wide community of seismologists.

\item{\bf 2.}	Sequoia 2000.  SAIC has been an industrial partner for two
years in this state-wide project of the University of California.  We are
deeply involved in developing DBMS and application software to support the
project oceanographers and climate researchers involved in global change
science.  We are working especially closely with NOAA staff and NOAA-
supported researchers at SIO.

\item{\bf 3.}	NOAA.  Under contract to Scripps Institution of Oceanography,
SAIC is prototyping a distributed data access system.

\item{\bf 4.}	Acoustic Thermometry of Ocean Climate (ATOC).  Under contract
to Scripps Institution of Oceanography, SAIC is developing a data center for
the ARPA-sponsored ATOC project.  The data to be managed includes large
volumes of real-time acoustic data collected from distant, distributed, data
acquisition points, and numerous other oceanographic and atmospheric data
sets.  A fundamental project requirement is that access to these data be
available to a global community of users for acoustic analysis and
global-change studies.

\end{description}
\newpage

\begin{center}
\LARGE
{\bf  David M. Glover}
\end{center}
\large
\noindent{\bf Personal-}
\normalsize
\smallskip
\begin{description}
\item{Name:}  David M. Glover
\item{Title:}  Research Specialist
\item{Affiliation:}  Woods Hole Oceanographic Inst.
\item{Address:}  Dept. of MC\&G, WHOI, Woods Hole, MA 02543
\item{email:}  david@plaid.whoi.edu
\item{phone:}  (508) 457-2000
\end{description}
\medskip
\large
\noindent{\bf Data System-}
\normalsize
\medskip
\begin{description}

\item{Data System Name:}  JGOFS DBMS
\item{Discipline:}  Biological, Chemical, and Physical Oceanography
\item{Data Managed:}
	\begin{description}
	\item{Type of Data:}  station data and time series data
	\item{Inventory Meta Data [Y/N]:}  y
	\item{Digital Data, Data Products [Y/N]:}  y
	\item{Number of Data Granules:}
	\item{Total Volume of Data [MB]:}  approx. 50MB and growing
	\end{description}
\end{description}

\medskip
\large
\noindent {\bf Data Management Activities Summary-}
\medskip

\centerline{\bf A Distributed, Object-based Data Management System}
\centerline {\bf for JGOFS}
\centerline{Glenn Flierl, PI, James Bishop, David Glover, Satish Paranjpe}
\normalsize
\medskip
	We have been involved in developing a distributed, object-based, 
multiple client, multiple server front-end for a relational DBMS.  This 
system provides to the user (the scientists in the JGOFS project) a 
format/location transparent means of access to the continually growing 
JGOFS database. This access allows the synthesis of fairly diverse data 
set types into new science-driven products without the user worrying 
about where the data is located or in what format it is stored. This is 
achieved by using "methods" that are executable bits of code that know 
about the data set's particulars and the requests are directed by a "server" 
that acts as a telephone operator knowing where the data is located. Since 
none of the data manipulation operations are predefined the system is 
flexible and extensible.

	I also sit on the EOSDIS Advisory Panel (aka the Data Panel) for the 
EOS Investigator Working Group (IWG) and chair the Users Working Group 
(UWG) for the JPL Physical Oceanography DAAC. The Data Panel has been 
instrumental in the crafting of the EOSDIS requirements that went into 
the RFP that was consequently won by Hughes. Now the Data Panel "stands 
guard", as it were, to insure that the distributed, evolving DIS we 
recommended actually comes about. The Users Working Group (UWG) 
advises the physical oceanography DAAC at JPL.
\newpage

\begin{center}
\LARGE
{\bf  Steve Hankin}
\end{center}
\large
\noindent{\bf Personal-}
\normalsize
\smallskip
\begin{description}
\item{Name:}  Steve Hankin
\item{Title:}  Computer Scientist
\item{Affiliation:}  NOAA/Pacific Marine Environmental Laboratory
\item{Address:}  7600 Sand Point Way NE, Seattle WA, 98115
\item{email:}  hankin@pmel.noaa.gov
\item{phone:}  (206)526-6080
\item{fax:}  (206)526-6744
\end{description}
\medskip
\large
\noindent{\bf Data System-}
\normalsize
\medskip
\begin{description}

\item{Data system name:}  FERRET and TMAP Data Base
\item{Discipline:}  Oceanography
\item{Data Managed:}
	\begin{description}
	\item{Type of Data:}  gridded data products
	\item{Inventory Meta Data [Y/N]:}  Y (exists)
	\item{Digital Data, Data Products [Y/N]:}  Y
	\item{Number of Data Granules:}  250 (files)
	\item{Total volume of Data [Megabytes]:}  5000 Mbytes	
\medskip
	\item{Type of Data:}  gridded model outputs
	\item{Inventory Meta Data [Y/N]:}  Y (exists)
	\item{Digital Data, Data Products [Y/N]:}  Y
	\item{Number of Data Granules:}  3000 (files)
	\item{Total volume of Data [Megabytes]:}  250,000 Mbytes
	\end{description}
\end{description}

\medskip
\large
\noindent {\bf Data Management Activities Summary-}
\normalsize
\medskip

	The Thermal Modeling and Analysis Project (TMAP) at NOAA/PMEL in 
Seattle is a numerical ocean modeling and data analysis group 
emphasizing upper ocean processes and ocean climate physics.  Typically 
TMAP's model experiments are performed on Cray and Cyber 
supercomputers and produce mid-size output data sets (one to five 
gigabytes).  TMAP also performs observational data studies to produce 
data sets suitable for forcing and validating the models.  These analyses 
are carried out on networked workstations.  The program FERRET was 
developed by TMAP as the primary software tool for analysis.

	FERRET is a workstation-based, interactive visualization and 
analysis environment that permits users to explore large and complex 
gridded data sets.  'Gridded data sets' in the FERRET framework include 
climatological data products (e.g. Levitus climatologies), binned monthly 
observation summaries (e.g. COADS), operational model outputs (e.g. NMC \& 
ECMWF), diagnostic model outputs (e.g. GFDL MOM), suitably prepared 
'section' data, and time series and vertical profiles (viewed as singly 
dimensioned grids).  FERRET's gridded data sets can be one to four 
dimensions - usually (but not necessarily) longitude, latitude, depth, and 
time - where the coordinates along each axis may be regular or irregularly 
spaced.  A data set may contain mixed 1, 2, 3, and 4-dimensional 
variables.  Axes of the same orientation may also differ - models often 
require staggered grids and gridded data products have few standards for 
temporal or spacial resolution.

	FERRET was designed to function in a distributed data environment.  
Since gridded data sets are frequently multi-gigabyte in size the program 
contains logic to manage a data set as a network-wide, distributed 
collection of files.  Furthermore, FERRET communicates with its data base 
using a two phase approach that is suited to optimized, wide-area access.  
In phase one FERRET queries the meta-data to determine the description 
and range of the available variables.  In phase two, after a user has 
specified a calculation, FERRET issues requests for only the minimally 
required subset of the data needed for the calculation, optimizing with 
respect to parameters that control access speed.  Memory caching is used 
to minimize the network bandwidth requirements. 

	FERRET is an excellent environment for browsing gridded data sets.  
Graphical displays of data sections (and extractions of these sections to 
files) may be created with single commands.  Graphical displays are 
automatically labelled with complete and unambiguous documentation.  
Additional variables may be overlaid - again with full documentation 
provided automatically.  A (prototype) point and click graphical user 
interface (GUI) has been developed to make browsing simple for novice 
users.  The development of this GUI is expected to continue under a project 
funded by the NOAA ESDIM program to provide Internet-wide gridded data 
browsing and analysis services from the NOAA Seattle campus. 

	FERRET development has emphasized interoperability standards.  
FERRET data sets are normally stored in the Unidata netCDF format, a 
self-documenting, publicly available format supported by the meteorology 
community.  FERRET graphics are layered upon the International Standards 
Organization's Graphical Kernel System (ISO/GKS) which provides network 
compatibility through the X-windows standard, and also supports a wide 
collection of other protocols: PostScript, HPGL, CGM, Tektronix, and 
Versatek to name a few.  FERRET is written in transportable FORTRAN 77 
and ANSI C.  It has been ported to SUN, DEC/Ultrix, SGI, VAX, and Macintosh 
(beta version) with IBM RS6000 and DEC/Alpha versions planned for the 
near future. 

	FERRET offers a flexible environment for data analysis;  new 
variables may be defined interactively as mathematical transformations 
of variables from data sets.  Complex analyses may be defined through 
hierarchical variable definitions.  A symmetric, 4-dimensional command 
syntax is used to designate arbitrary rectangular regions in 4-space and 
"IF-THEN-ELSE" logic permits calculations to be applied over arbitrarily 
shaped regions.  An assortment of data smoothers and data gap fillers 
compliments the normal range of mathematical analysis tools.  FERRET's 
scripting language facilitates large, batch-style calculations and enables 
FERRET to interact with specialized packages such as Matlab or GMT and to 
use custom-written routines. 

	The TMAP group maintains a large and growing data base of gridded 
data products - approximately 5 Gigabytes in size presently.  These data 
are stored in random access format for quick, record-level access by 
FERRET and are maintained on-line.  Much of this data base is expected 
shortly to be available over Internet through FERRET.

A partial list of the data sets includes:
\begin{itemize}
\item COADS monthly average surface marine observations (1946-1991)
\item Esbensen \& Kushnir global ocean surface heat budget
\item ECMWF/TOGA global 12-hourly surface analysis (1985-92)
\item FSU tropical Pacific wind stress (1961-92)
\item NODC Levitus climatological global ocean atlas
\item NGDC ETOPO 5 minute relief of the surface of the earth
\item NMC blended monthly average SST (1982-1992)
\item NMC monthly upper air winds and OLR analysis (1968-88)
\item Oberhuber atlas of heat, buoyancy and turbulent kinetic energy 
\item Rasmusson and Carpenter tropical Pacific El Nino composite 
analysis
\item Richardson monthly climatological global ocean surface currents
\item Sadler tropical Pacific winds (1979-90)
\item US Navy FNOC global 6-hourly surface winds (1982-92)
\end{itemize}
\smallskip
A partial list of the sites using FERRET includes:
\smallskip
\begin{itemize}
\item NCAR
\item MIT
\item Los Alamos National Laboratories
\item NOAA/Pacific Marine Environmental Laboratory
\item NOAA/Alaska Fisheries Science Center
\item NOAA/Geophysical Fluid Dynamics Laboratory
\item NOAA/Atlantic Ocean Marine Laboratory
\item University of Washington (6 departments)
\item University of Hawaii
\item Naval Post-graduate School
\item Florida State University
\item University of Rhode Island
\item University of British Columbia
\item Texas A and M
\end{itemize}
\smallskip
	FERRET is freely available over the Internet via anonymous FTP on 
node abyss.pmel.noaa.gov.
\newpage

\begin{center}
\LARGE
{\bf  Roen Hogg}
\end{center}
\large
\noindent{\bf Personal-}
\normalsize
\smallskip
\begin{description}
\item{Name:}  Roen Hogg
\item{Title:}  Faculty Research Assistant
\item{Affiliation:}  Oregon State University
\item{Address:}
		College of Oceanography
		Oceanography Administration -- Building 104
		Corvallis, OR  97331-5503
\item{email:}  roen@oce.orst.edu
\item{phone:}  503-737-4414
\item{fax:}  503-737-2064
\end{description}

\medskip
\large
\noindent {\bf Data Management Activities Summary-}
\normalsize
\medskip

	The objective of our project is to use object-oriented technology to 
develop a user-friendly system that will allow scientists to interact with 
oceanic data and test hypotheses at the workstation.  In addition, this 
system will facilitate multidisciplinary ocean field experiments by 
allowing scientists to communicate the results of their research to 
internal and external user groups.

	The proposed system will provide intuitive and relatively 
transparent access to existing analysis systems, numerical models, 
imagery data, data acquisition systems, heterogeneous databases, and 
communication systems.  Given the large disk storage and computing 
capability (between 2-3 Giga flops) required by existing 3-D modeling and 
analytical tasks, the system will provide access to the Oregon State 
University Oceanography computing facilities (SUN Sparc and IBM UNIX-
based workstations, massively parallel CM-5 Connection Machine, and 
100GB optical data storage capabilities).

	Some of the major features associated with the system include the 
following:
\begin{description}
\item{1.} Derivation Analysis

\noindent The system will process requests for the selection and formatting 
of relevant data.  In particular, the system will process requests to 
transform:
\begin{description}
	\item{(1)}	level 1 data (sensor data) into level 2 data
(geophysical data) 
	\item{(2)}	level 2 data (geophysical data) into level 3 data
(higher-order data)
\end{description}

\item{2.} Research Analysis

\noindent The system will support the research analysis common to most 
research projects.  This includes the following:
\begin{description}
	\item{(1)}	useful numerical analysis and display of data sets
	\item{(2)}	useful data queries 
\end{description}

\noindent In addition, the system will store a description of how each data
set was derived.

\item{3.}	Visualization

\noindent The system will support graphic and video images.  These images 
will include plots (e.g., profile, time series, drifter, imagery, grids, 
station), satellite images, and video.  The system will provide the 
following functionality:
\begin{description}
	\item{(1)}	interface with the appropriate graphic/video generating 
programs
	\item{(2)}	store a description of how each image was derived
	\item{(3)}	store any free-form text a scientist wants to
associate with a particular image
	\item{(4)}	provide useful query capabilities
\end{description}

\item{4.}	Communication

\noindent The system will provide computer networking and transmission of 
data and results to internal and external user groups.  The system will 
support the following types of communication:
\begin{description}
	\item{(1)}	e-mail
	\item{(2)}	video
	\item{(3)}	audio
	\item{(4)}	data (files)
\end{description}

\item{5.}	Ocean Model

\noindent The system will interact with various existing models by providing 
specifications of input and output parameters.
\end{description}
\newpage

\begin{center}
\LARGE
{\bf  James D. Irish}
\end{center}
\large
\noindent{\bf Personal-}
\normalsize
\smallskip
\begin{description}
\item{Name:}  James D. Irish
\item{Title:}  Research Specialist
\item{Affiliation:}  Woods Hole Oceanographic Institution
\item{Address:}  307 Smith
\item{email:}  jirish@whoi.edu
\item{phone:}  (508)457-2000 Ext. 2732
\item{fax:}  (508)457-2195
\end{description}
\medskip
\large
\noindent{\bf Data System-}
\normalsize
\medskip
\begin{description}

\item{Data system name:}  Not using a formal data system at present, but 
starting to formulate one for use in 1 year.
\item{Discipline:}  Physical Oceanography
\item{Data Managed:}  Water Velocity, Temperature, Pressure, 
Conductivity, Optical and Acoustical Time Series
\end{description}

\medskip
\large
\noindent {\bf Data Management Activities Summary-}
\normalsize
\medskip

	As a student at Scripps I worked with Walter Munk and his BOMM 
system of data analysis and archiving.  I carried many ideas (and much 
code) to APL/UW where I constructed a similar system to analyze and 
store data using the computer systems there.  At UNH Wendell Brown and I 
merged our ideas and the SIO routines to an Ocean Analysis Software 
Package with archiving capability, but the system had no on-line search 
capability.  A search and retrieval system was started, but not completed 
before I left UNH.  At UNH I also developed moorings systems which 
telemetered data via ARGOS and GOES satellites, and packet radio back to 
the laboratory in real-time.  Since I have been at WHOI, I have also used 
cellular phone techniques for returning data from the field.  A PC and 
mainframe based system was developed at UNH for retrieving the near 
real-time data daily from the field, editing, normalizing, and storing it in 
ASCII and binary data files for analysis and archiving.  However, we were 
never able to get funding to set up a real-time data base for others to 
access.  As part of my funded GLOBEC activities on Georges Bank, I am 
again deploying moorings with GOES and ARGOS telemetry (and possibly 
acoustic telemetry from bottom instruments) for several years, and plan 
to work with the Georges Bank Data Management Office at WHOI (Wiebe, 
Flierl, Brown, and Lynch) in establishing a real-time data retrieval, 
processing and distribution system on my computers for use within the 
Georges Bank GLOBEC data system using the ideas and concepts developed 
and tested by this group.
\newpage

\begin{center}
\LARGE
{\bf  George Milkowski}
\end{center}
\large
\noindent{\bf Personal-}
\normalsize
\smallskip
\begin{description}
\item{Name:}  George Milkowski
\item{Title:}  Data Systems Manager/Developer
\item{Affiliation:}  University of Rhode Island
\item{Address:}  Graduate School of Oceanography
			Narragansett Bay Campus
			Narragansett, RI 02882-1197
\item{email:}  george@zeno.gso.uri.edu OMNET{g.milkowski}
\item{phone:}  401 792 6939
\item{fax:}  401 792 6728
\end{description}
\medskip
\large
\noindent{\bf Data System-}
\normalsize
\medskip
\begin{description}

\item{Data system name:}  Global 1km AVHRR Inventory
\item{Discipline:}  Physical Oceanography
\item{Data Managed:}
	\begin{description}
	\item{Type of Data:}  High Resolution Raw/Level 1b AVHRR
	\item{Inventory Meta Data [Y/N]:}  Y
	\item{Digital Data, Data Products [Y/N]:}  Some
	\item{Number of Data Granules:}  $>$200,000
	\item{Total volume of Data [Megabytes]:}  700 Gigabytes
	\end{description}
\end{description}

\medskip
\large
\noindent {\bf Data Management Activities Summary-}
\normalsize
\medskip

	The Global 1km AVHRR Inventory was developed to provide, within a 
single inventory, a comprehensive listing of available high resolution 
AVHRR data from major archives around the world.  The inventory includes 
listings of Level 1b data holdings from U.S. and foreign archives.  Those 
archives with holdings listed are; NOAA/NESDIS, USGS EROS Data Center, 
University of Rhode Island, University of Miami,  European Space Agency 
(ESA's inventory is composed of AVHRR data collected and archived by 
contributing European country agencies and their data centers),      
Australian CSIRO.  The Japanese Weather Service is currently working on 
providing listings of their holdings.  An on-line, simple to use, user 
interface provide access to the Global 1km Inventory and permits user 
specified searches based on time, geographic location, sensor, archive, 
platform, sensor data collection mode and data processing level.  The      
inventory is held with a relational data base management system.  
Information on how to access and use the inventory can be obtained 
through anonymous ftp at zeno.gso.uri.edu in /usr/spool/ftp/pub/.       
Access to the inventory is also possible through the NASA Climate and 
Global Change Master Directory.

	Contributing archives periodically transfer listings of their new 
acquisitions to the Global 1km AVHRR Inventory.  These updates are coded 
in the internationally recognized format developed by the Committee on 
Earth Satellites Work Group on Data Catalog Subgroup specifically for 
AVHRR inventory update exchange.  

	Other scientific data management activities include the 
development and management of an XBT client-server application with 
NOAA/NODC that provides the location of XBT drops based on user supplied 
spatial and temporal windows.  This application was designed such that 
the client application can either stand alone or be incorporated within 
another application.  An example of this latter implementation is URI's 
XBROWSE where the XBT client-server application dynamically displays 
the location of XBT data correlated with sea surface temperature imagery.  
\newpage

\begin{center}
\LARGE
{\bf  Christopher Miller}
\end{center}
\large
\noindent{\bf Personal-}
\normalsize
\smallskip
\begin{description}
\item{Name:}  Christopher Miller
\item{Title:}  Physical Scientist
\item{Affiliation:}  NOAA/NESDIS/Environmental Information Services
\item{Address:}  1825 Connecticut Ave., NW, Suite 506, Washington, D.C.  
20235
\item{email:}  C.Miller.NOAA (Omnet)
\item{phone:}  (202)606-5012
\item{fax:}  (202)606-0509
\end{description}

\medskip
\large
\noindent {\bf Data Management Activities Summary-}
\normalsize
\medskip

	The responsibilities of the Environmental Information Services 
Office encompass the activities of the NOAA National Data Centers 
(National Climatic Data Center, National Oceanographic Data Center and 
the National Geophysical Data Center) and the Environmental Services Data 
and Information Management (ESDIM) Program, which is a cross-cutting 
program addressing NOAA-wide data management issues.  ESDIM's current 
focus is rescuing critical NOAA data at risk of being lost and improving 
access to NOAA data and information.  The head of Environmental 
Information Services, Mr. Gregory Withee, is, also, responsible for the 
Information Management element of the NOAA Climate \& Global Change 
Program.  Information Management serves the Climate \& Global Change 
Program through its support of the needs of the multiple science elements 
of that program (generation of long-term climate and global change data 
sets; development of information management systems in connection with 
specific science program objectives).  Information Management is, also, a 
focal point for the Global Change Data and Information System (GCDIS), 
which is envisioned as an interagency gateway for access and delivery of 
data and information products.  A key goal is to promote interoperability 
among the existing heterogeneous systems.
\newpage

\begin{center}
\LARGE
{\bf  William Schramm}
\end{center}
\large
\noindent{\bf Personal-}
\normalsize
\smallskip
\begin{description}
\item{Name:} William Schramm
\item{Title:}  Chief, Ocean Applications Branch
\item{Affiliation:}  NOAA
\item{Address:}  2560 Garden Road, Monterey, CA 93940
\item{email:}  OMNET/W.SCHRAMM
\item{Phone:}  (408) 647-4206
\item{Fax:}  (408) 647-4225
\end{description}
\medskip
\large
\noindent{\bf Data System-}
\normalsize
\medskip
\begin{description}

\item{Data system name:}  NEONS database management system
\item{Discipline:}  Synoptic Oceanography and Meteorology
\item{Data Managed:}
	\begin{description}
	\item{Type of data:}  Numerical ocean and atmospheric products 
(gridded fields from Navy and NOAA sources)
	\item{Inventory Meta Data?}  yes
	\item{Digital data, data products?}  yes
	\item{Number of data granules:}  7025 fields
	\item{Total volume of data:}  84 MB
	\item{Note:} 1035 new fields (12.5 MB) are entered into the 
DBMS/day.  Some are retained for two days and some for thirty days.
\medskip
	\item{Type of data:}  SSM/I (SDRs, TDRs and EDRs)
	\item{Inventory Meta Data?}  yes
	\item{Digital data, data products?}  yes
	\item{Number of data granules:}  14 satellite passes/day
	\item{Total volume of data:}  150 MB
	\item{Note:}  This data changes daily.
\medskip
	\item{Type of data:}  Synoptic ocean and atmospheric observations
	\item{Inventory Meta Data?}  yes
	\item{Digital data, data products?}  yes
	\item{Number of data granules:}  16,000 observations
	\item{Total volume of data:}  1.8 MB
	\item{Note:}  Observations are retained for two days.
\medskip
	\item{Type of data:}  DMSP IR and visual images
	\item{Inventory Meta Data?}  yes
	\item{Digital data, data products?}  yes
	\item{Number of data granules:}  79 images
	\item{Total volume of data:}  1.6 MB
	\item{Note:}  Images are retained for two days in compressed 
format.
	\end{description}
\end{description}

\medskip
\large
\noindent {\bf Data Management Activities Summary-}
\normalsize
\bigskip

\noindent{\bf BACKGROUND}
\bigskip

	The Naval Research Laboratory (NRL) in Monterey CA has developed a 
powerful database management system for environmental data called the 
Naval Environmental Operational Nowcasting System (NEONS).  The system 
was developed to manage the three basic types of environmental data; 
observations, images and gridded data.  NRL uses NEONS to support 
research and development programs such as the development of satellite 
data processing software.  A typical R\&D application of the system at NRL 
is to develop "virtual sensor" information based on combinations of 
satellite data that do not exist from any single satellite.  It is the design 
of NEONS, however, and not the NRL applications that makes the system of 
interest to others including NOAA facilities.

	Unlike most data management initiatives, which are developed for a 
specific application, NEONS was developed to be very flexible and 
versatile.  As a result other institutions have found the system to be 
useful and because of the Navy's Technology Transfer program NRL has 
been very cooperative in making the software available.  Recently the Navy 
announced that NEONS will be used for operational database management 
on Cray supercomputers at the Fleet Numerical Oceanography Center 
(FNOC) and the Naval Oceanographic Office at Stennis, MS.  Within NOAA 
the NEONS software was first installed at the Ocean Applications Branch 
(OAB) of the National Ocean Service.  OAB was established in Monterey to 
support Navy/NOAA cooperative programs and NEONS is being used to 
support civilian distribution of FNOC data, analyses and forecasts via the 
Navy/NOAA Oceanographic Data Distribution System (NODDS).  OAB later 
contacted other NOAA offices to make them aware of the availability of 
NEONS and to promote sharing of software and data resources between 
Navy and NOAA.  In late 1991 OAB arranged for the installation of NEONS 
at the National Climatic Data Center (NCDC) in Asheville NC for support of 
the Global Climate Perspectives System (GCPS).  Other participants in 
GCPS are the NOAA Climate Monitoring and Diagnostics Lab (CMDL) in 
Boulder CO and the Climate Analysis Center (CAC) in Washington DC.  In 
February 1992 OAB helped install NEONS at CMDL and in August 1992 at 
CAC.  In December of 1992 there was a second installation in Boulder at 
the
Forecast Systems Laboratory where NEONS will be used in the MADER 
project.  The most recent NEONS installation in NOAA was at the NMFS 
Laboratory in Hawaii.  In addition to installations in the Navy and in NOAA, 
the system has been provided to; 1) Canadian Atmospheric and 
Environmental Services in Toronto and Vancouver, 2) Bureau of 
Meteorology in Melbourne, Australia, 3) South Dakota School of Mines and 
Technology, 4) Woods Hole Oceanographic Institution, 5) Cray Research 
Inc, 6) British Meteorological Office, 7) French Meteorological service and 
8) World Laboratory in Italy.  
 
\bigskip
\noindent {\bf NEONS TECHNICAL DESIGN}
\bigskip

	NRL designed NEONS for fast, efficient operation and compatibility 
with computer industry and international data exchange standards such as 
BUFR and GRIB.  The system is built around the commercial database 
management system, EMPRESS, and operates on a variety of computers, 
from UNIX workstations to Cray supercomputers.  Computer industry 
standards used in the design of NEONS include UNIX and SQL.

	An important feature of NEONS is the storage of data in variable 
length binary strings.  This is important because environmental data 
comes in a variety of record lengths.  Another important advantage of the 
way in which NEONS stores data is that the data are addressed only to a 
minimal level of information in contrast to many other database systems 
which address data deeply, down to the report or even data value level.  
The approach used by NEONS greatly speeds up searches compared to other 
systems that are often burdened with high system overhead and frequent 
disk accesses.  The third advantage of the NEONS approach is that by using 
binary compaction the system takes advantage of the great CPU speed of 
new RISC computers while at the same time minimizing the I/O time 
which is the critical limiting factor in modern DBMS systems.   

	The international data exchange standards used by NEONS are the 
binary formats adopted by the World Meteorological Organization (WMO) 
for global exchange of real-time weather data:  Binary Universal Format 
for data Representation (BUFR) for observations and GRIB for GRIdded 
Binary numerical fields.  
 
\bigskip
\noindent {\bf A NEONS NETWORK}
\bigskip

	With the expanding use of NEONS within the Navy, in NOAA and in 
other countries such as Canada and Australia, NRL and OAB have promoted 
the concept of a distributed network of NEON systems to facilitate the 
global exchange of weather and ocean data.  In this concept, each office 
would continue to load and process its own data and in addition, would 
make the data easily available to others over INTERNET.  

	NRL has developed an X-Windows Data Browser to interactively 
browse files on NEON systems, search for particular data sets, and 
download data of interest.  Using the browser, a user can specify the time 
and area where he wishes information.  The browser then searches the 
database, either locally or remotely over INTERNET, to find satellite 
images, gridded model outputs or observations that fall in the desired 
time/space window.  The user then interacts with the database to narrow 
the search to the actual data required.

	The data can be downloaded in any of a wide variety of formats.  The 
potential for such a network can be demonstrated by considering the 
gigabytes of climate data now being loaded by NCDC into their NEONS.  
Early this year NCDC described their work in the GCPS as follows:  "NEONS 
is up and running.  Sequences and parameters have been defined for the 
Global Historical Climatological Network (GHCN) data set (monthly global 
surface temperature, precipitation, and station and sea level pressure), 
Global Precipitation (GCPC) dataset, Cooperative Summary of the Day 
(TD3200) (daily max and min temps, precipitation, snow fall and snow 
depth), and CARDS.  We are presently working on tying in the system with 
the Metadata portion of the STORM system (also an EMPRESS database 
system) to link the data to the station histories. We are also discussing 
with NRL the strong possibility of working with them on the development 
of an interactive interface between NEONS and NCAR Graphics for 
Lat/Lon/Time data) a system like the one they have developed for gridded 
data."

	To promote the idea of a network of NEON systems, OAB started an 
OMNET bulletin board for NEONS users and, in cooperation with NRL and 
FNOC, hosted a NEONS Users Conferences in April 1992 and again in April 
1993.  The next NEONS users meeting will be held in conjunction with the 
AMS conference in January 1994.

\bigskip
\noindent {\bf SUMMARY}
\bigskip

	The Navy, through the Naval Space Warfare Systems Command, has 
invested over \$4M in the NEONS program.  Other government agencies 
should and can take advantage of this investment.  Offices wanting more 
information about NEONS or wanting to install the system should write to 
OAB/NOAA, 2560 Garden Road, Monterey CA 93940.
\newpage


\begin{center}
\LARGE
{\bf Nancy Soreide}
\end{center}
\large
\noindent{\bf Personal-}
\normalsize
\smallskip
\begin{description}
\item{Name:}  Nancy Soreide
\item{Title:}  System Analyst
\item{Affiliation:}  NOAA/PMEL
\item{Address:}  7600 Sand Point Wy NE, Seattle WA 98115
\item{email:}  nns@noaapmel.gov
\item{phone:}  206-526-6728
\item{fax:}  206-526-6774
\end{description}
\medskip
\large
\noindent{\bf Data System-}
\normalsize
\medskip
\begin{description}

\item{Data system name:}  TOGA-TAO Display Software, EPIC
\item{Discipline:}  Oceanogaphy
\item{Data Managed:}
	\begin{description}
	\item{Type of Data:}  oceanographic time series
	\item{Inventory Meta Data [Y/N]:}  Y
	\item{Digital Data, Data Products [Y/N]:}  Y
	\item{Number of Data Granules:}  aprox 3000 time series
	\item{Total volume of Data [Megabytes]:}  aprox 175 MBytes
\medskip
	\item{Type of Data:}  oceanographic profile (depth-indexed) data
	\item{Inventory Meta Data [Y/N]:}  Y
	\item{Digital Data, Data Products [Y/N]:}  Y
	\item{Number of Data Granules:}  aprox 35,000 profiles (1 CTD=1
			profile) 
	\item{Total volume of Data [Megabytes]:}  aprox 530 MBytes
	\end{description}
\end{description}

\medskip
\large
\noindent {\bf Data Management Activities Summary-}
\normalsize
\medskip


\bigskip
\noindent {\bf TOGA-TAO DISPLAY SOFTWARE}
\bigskip

	The TOGA-TAO Array consists of 63 moored ATLAS wind and 
thermistor chain and current meter buoys, spanning the Pacific Basin from 
95W in the eastern Pacific to 137E in the west, transmitting data in real-
time via the Argos satellite system.  PMEL has developed the TAO 
workstation display software for distribution, and display, of the TOGA-
TAO buoy data in a point-and-click environment.  Data displays include 
buoy summary plots, buoy sensor plots, vertical stacks of plots from any 
collection of buoy/sensor pairs, as well as animations of TAO buoy data, 
operational ocean model analyses from NMC, TOGA drifting buoys, and 
climatological averages.  TOGA-TAO buoy data displayed on the 
workstation are updated by acquisition from the Argos satellite shore-
based computer during the previous night.  The TOGA-TAO data sets and 
display software are available on PMEL's anonymous FTP on the Internet 
network, and have been distributed world-wide.  Automated procedures 
provide remote users with updated data and graphics files. The TAO 
software is based on X-windows, and can be run on a local or wide-area 
network.


\bigskip
\noindent {\bf EPIC}
\bigskip


	The EPIC system was designed to manage the large volume of 
oceanographic time series and hydrographic data being collected by PMEL 
oceanographers participating in NOAA's large scale ocean climate study 
programs, such as EPOCS and TOGA.  At present, over 35,000 data sets are 
on-line in EPIC at PMEL for retrospective analysis.  EPIC is a complete 
system including a data selection module and a suite of over 100 graphics 
display and analysis programs.  Supported data types include time series 
data, (such as temperature, wind and current time series from moored 
buoys), drifter data, acoustic doppler data, and profile data, (such as CTD, 
bottle and XBT data).  System elements for data selection, data display, 
and data analysis, function independently.  The system is well 
documented, with a user manual and extensive on-line help.
\newpage

\begin{center}
\LARGE
{\bf  Robert Starek}
\end{center}
\large
\noindent{\bf Personal-}
\normalsize
\smallskip
\begin{description}
\item{Name:}  Robert Starek
\item{Title:}  Program Manager
\item{Affiliation:}  Naval Oceanographic Office
\item{Address:}  Code OTM, 1002 Balch Blvd, Stennis Space Center, MS  
39522
\item{phone:}  (601)688-5189
\item{fax:}  (601)688-5701
\end{description}

\medskip
\large
\noindent {\bf Data Management Activities Summary-}
\normalsize
\medskip

	I am the program manager for an effort to develop a geo-referenced, 
multi-disciplinary data base for the Naval Oceanographic Office.  The 
system, entitled the Integrated Data Base Management System (IDBMS), 
will consist of Department of Defense (DoD) oceanographic and Mapping, 
Charting and Geodesy data.  The architecture of the IDBMS is distributed in 
that data will be physically located closest to the point of greatest use on 
server class computers.  A majority of the data will be stored in a 
relational data base and will be accessed and managed using a client-
server based concept.  Tightly coupled to the actual data will be a catalog 
consisted of meta data.  A graphical user interface using visual references 
such as maps will allow for spatial browse and query of the catalog.  
Access to IDBMS data by non-Naval Oceanographic Office persons will 
require explicit DoD approval.
\newpage

\begin{center}
\LARGE
{\bf  Leonard Walstad}
\end{center}
\large
\noindent{\bf Personal-}
\normalsize
\smallskip
\begin{description}
\item{Name:}  Leonard Walstad
\item{Title:}  Dr.
\item{Affiliation:}  College of Oceanic and Atmospheric Science, Oregon 
State U.
\item{Address:}  Ocean Admin Bldg 104, Corvallis, OR  97331-5503  
\item{email:}  lwalstad@oce.orst.edu  l.walstad/omnet
\item{phone:}  (503) 737-2070
\item{fax:}  (503) 737-2064
\end{description}

\medskip
\large
\noindent {\bf Data Management Activities Summary-}
\normalsize
\medskip

	I am currently chairman of the GLOBEC data management committee.  
The GLOBEC steering committee has endorsed the objective of choosing a 
data management framework which makes use of a distributed data 
management system.  Furthermore, we believe that the system must 
encourage interaction, rather than hinder the exchange of data.  

	As an ocean modeler, my data sets are generally stored as flat files 
in machine format floating point and integer numbers.  However, I 
generally make extensive use of the output of these numerical models for 
purposes beyond the usual graphical display.  Typical uses are forcing of 
biological systems,  analysis of float tracks, and vorticity and energy 
dynamics.  Because I make extensive use of the output and use several 
numerical models, an analysis system would be of great benefit.  A key 
concern is the ability of databases to deal with single entities which 
involve hundreds of MB of data (i.e. the output of a single snapshot of a 
biophysical model).

	I use data assimilation to provide initial and boundary conditions, 
and also to update the physical fields within numerical models.  This 
component of my research would be substantially simplified if an 
efficient interface to data was provided.

	I believe that oceanographers need communication tools more than 
traditional database tools. Several key components of a valuable system 
are:
\begin{description}
	\item {1.} extensibility
	\item {2.} explorability
	\item {3.} modularity
	\item {4.} efficiency 
	\item {5.} abstraction
\end{description}
\newpage

\begin{center}
\LARGE
{\bf  Warren B. White}
\end{center}
\large
\noindent{\bf Personal-}
\normalsize
\smallskip
\begin{description}
\item{Name:}  Warren B. White
\item{Title:}  Physical Oceanographer
\item{Affiliation:}  Scripps Institution of Oceanography
\item{Address:}  SIO/UCSD, La Jolla, CA 92093-0230
\item{email:}  wbwhite@ucsd.edu
\item{phone:}  619-534-4826
\item{fax:}  619-534-8041
\end{description}
\medskip
\large
\noindent{\bf Data System-}
\normalsize
\medskip
\begin{description}

\item{Data system name:}  JEDA Center
\item{Discipline:}  quality-controled temperature-depth observations over 
the globe; monthly mean global upper ocean temperature gridded fields 
(1979-1993) 
\item{Data Managed:}
	\begin{description}
	\item{Type of Data:}  temperature-depth observations
	\item{Inventory Meta Data [Y/N]:}  yes, in the GTSPP format
	\item{Digital Data, Data Products [Y/N]:}  yes, in the NetCDF format
	\item{Number of Data Granules:}  1-2 million
	\item{Total volume of Data [Megabytes]:}  1.5-2.0 GBytes
	\end{description}
\end{description}

\medskip
\large
\noindent {\bf Data Management Activities Summary-}
\normalsize
\medskip

	We are a WOCE DAC for upper ocean temperature profiles collected 
for the period 1990-1995. Presently, we have conducted quality control on 
the historical file for the 11-year period 1979-1989. We are presently 
conducted QC on the delayed-mode data collated by NODC for the period 
1990. One of the steps in quality control is the gridding of upper ocean 
temperature anomalies. We have accomplished this for the period 1979-
1989 at 11 standard levels in the upper 400 m. The standard grid is 2o 
latitude by 5o longitude by month over as much of the ocean as the 
observations will allow.  Interpolation errors also accompany these 
gridded estimates. Our climatological reference is computed on this same 
spatial grid each month for the 10-year period from 1979-1988.  We plan 
to publish to the oceanographic community (over the Internet) the QC 
profiles, the gridded anomaly products, and the climatological reference.
\newpage

\begin{center}
\LARGE
{\bf  J. R. Wilson}
\end{center}
\large
\noindent{\bf Personal-}
\normalsize
\smallskip
\begin{description}
\item{Name:}  J. R. Wilson
\item{Title:}  Director
\item{Affiliation:}  Marine Environmental Data Service
\item{Address:}  Department of Fisheries and Oceans, 200 Kent Street, 
Ottawa, Ontario, K1A   0E6, Canada
\item{email:}  R.Wilson.MEDS (Omnet)
\item{phone:}  (613)990-3009
\item{fax:}  (613)990-5510
\end{description}
\medskip
\large
\noindent{\bf Data System-}
\normalsize
\medskip
\begin{description}

\item{Data system name:}  MEDS Oceanographic Database
\item{Discipline:}  Physical Oceanography
\item{Data Managed:}
	\begin{description}
	\item{Type of Data:}  Water column phy and chem properties
	\item{Inventory Meta Data [Y/N]:}  Y
	\item{Digital Data, Data Products [Y/N]:}  Y
	\item{Number of Data Granules:}  600,000 multi-variable stations
	\item{Total volume of Data [Megabytes]:}  1,000
\medskip
	\item{Type of Data:}  Drifting buoy data
	\item{Inventory Meta Data [Y/N]:}  Y
	\item{Digital Data, Data Products [Y/N]:}  Y
	\item{Number of Data Granules:}  5,000,000 observations
	\item{Total volume of Data [Megabytes]:}  2,000
\medskip
	\item{Type of Data:}  Wave measured and hindcast data
	\item{Inventory Meta Data [Y/N]:}  Y
	\item{Digital Data, Data Products [Y/N]:}  Y
	\item{Number of Data Granules:}  4000 station years
	\item{Total volume of Data (Megabytes]:}  30,000
\medskip
	\item{Type of Data:}  Inland and coastal water levels
	\item{Inventory Meta Data [Y/N]:}  Y
	\item{Digital Data, Data Products [Y/N]:}  Y
	\item{Number of Data Granules:}  7,500 station years
	\item{Total volume of Data [Megabytes]:}  1,000
	\end{description}
\medskip
\item{Data system name:}  DFO Chemical Contaminants Data Directory
\item{Discipline:}  Chemical Oceanography
\item{Data Managed:}
	\begin{description}
	\item{Type of Data:}  National Directory of Data Sources in DFO
	\item{Inventory Meta Data [Y/N]:}  Y
	\item{Digital Data, Data Products [Y/N]:}  Y
	\item{Number of Data Granules:}  7 regional inventories
	\item{Total volume of Data [Megabytes]:}  .05
	\end{description}
\end{description}

\bigskip
\large
\noindent {\bf Data Management Activities Summary-}
\normalsize
\medskip

	MEDS is involved in the acquisition, processing, quality control, 
archival, and dissemination by broadcast and on request of ocean station, 
wave, and water level data.  These activities support national and 
international research, engineering, regulation, and management 
activities.

	Typical applications of our data include, real time water level data 
for use in setting flow rates along the St. Lawrence Seaway, real time 
wave data for forecasts and warnings, engineering quality wave data for 
design of fixed and floating structures and ships, data in support of 
national research activities, and regulation and management of fisheries.  
MEDS also provides data management and dissemination services in 
support of international data exchange programs in general and the WOCE 
Surface Velocity and Upper Ocean Thermal Programs.

	Data systems used include sequential tape systems for the 
voluminous wave and water level time series data, on-line VMS ISAMs for 
the drifting buoy, ocean station, water level, and wave spectral data, and 
Oracle relational databases for inventories and the contaminants data 
directory.

	Over the past two years work has been carried out on the 
development of distributed client/server database capabilities using 
commercial software such as Oracle and SQLnet, and on a more general 
system utilizing the exchange of standard request, menu, raster, graphic, 
text, and data objects in a style that accommodates dissimilar database 
technologies including home built ones.

Design criteria for the system are as follows:
\begin{description}
\item {1.} Client/Server Architecture (minimize network traffic).

\item {2.} Software in the field (client) not data or product specific.  Host
or server services can be changed and enhanced without modifying client
software.

\item {3.} Not necessary for user to have a personal account and password on 
the server.  User does not login on the server in the traditional fashion.

\item {4.} System allows a user to browse an inventory, identify data, and
copy data back to the client from an associated database.

\item {5.} The client will be a workstation operating in a "windows" mode 
usually with various tools available for such things as GIS, spreadsheet, 
database.
\end{description}

	Implementation of such a system was found to involve the 
development of a number of standard objects including the list given 
above.  Objects are passed in both directions between client and server.  
Both client and server have installed processing and communications 
modules.  The communications modules consist of network "listeners" 
which detect the arrival of standard objects in some fashion.  In the 
prototype system in DEC VMS the "listener" consisted of a DCL procedure 
which woke up every 10 seconds and tested for the presence of a file with 
a certain name.  Once the file was detected a program was invoked that 
opened the file and discovered the type of data, the type of product 
required, space-time ranges, parameter identifications, etc.  The arriving 
object also carried the node address to which the reply object would be 
sent.  Menu and request objects carried further information such as 
programs to be run to process the request or reply.

	Such a system was found to be relatively generic and simple.  It 
relieved the need for someone to login to the server in the traditional 
sense and the associated security risks.  It also relieved the need for 
large number of user licenses on the server as the listener/processor 
operated as a single user.  Since the standardization was in the exchange 
objects and their format was cast in concrete, software could be coded on 
the server to deal with any type of database including home built ones 
such as the ISAMs used in MEDS.

	One purpose of the system and the design was to facilitate the 
development of ad hoc GIS databases.  Point or polygon data could be 
returned to the client with raster, text, or vector graphic objects 
attached to it.  The point and polygon data would be loaded up into the GIS.  
The user could click on a point or polygon on the map, see what raster, 
text, or vector objects were attached to it, select one, and have a window 
popped with a photograph, vector graphic, or textual information 
displayed.  Similarly the user could have retrieved an object from the 
server formatted for uploading into a spreadsheet or Dbase application on 
the client.

	This development and the pilot project were done over the past two 
years.  We have not yet managed to find the resources to code and 
implement a production system.
\newpage

\end{latexonly}

\end{document}


