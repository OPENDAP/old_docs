\documentstyle[12pt,html,psfig]{article}
%
% hacked from art12.doc
% 
% makes 1 , 1.25, 1.25 and 1.25 in. margins on the top, left, right and
% bottom resp. when using \documentstyle[12pt,...]{article}
%
% jhrg 1/6/94

% SIDE MARGINS:
\if@twoside               % Values for two-sided printing:
   \oddsidemargin 0pt     %   Left margin on odd-numbered pages.
   \evensidemargin 38pt   %   Left margin on even-numbered pages.
   \marginparwidth 85pt   %   Width of marginal notes.
\else                     % Values for one-sided printing:
   \oddsidemargin 18.5pt  %   Note that \oddsidemargin = \evensidemargin
   \evensidemargin 18.5pt
   \marginparwidth 68pt 
\fi
\marginparsep 10pt        % Horizontal space between outer margin and 
                          % marginal note
 
 
% VERTICAL SPACING:        
\topmargin 0pt            %    Nominal distance from top of page to top
                          %    of box containing running head.
\headheight 0pt           %    Height of box containing running head.
\headsep 0pt              %    Space between running head and text.
\topskip = 12pt           %    '\baselineskip' for first line of page.

\textheight = 635pt % 36\baselineskip
\advance\textheight by \topskip
\textwidth 435pt         % Width of text line.

% Adding this fixes the top and bottom margin sizes. jhrg

\textheight 8.75in

\newcommand{\postscript}[2]{
        \par
        \hbox{
                \vbox to #1{
                        \vfil
                        \special{ps: plotfile #2.ps}
                }
        }
}







% Adding this fixes the top and bottom margin sizes. jhrg

\textheight 8.75in

\newcommand{\postscript}[2]{
	\par
	\hbox{
		\vbox to #1{
			\vfil
			\special{ps: plotfile #2.ps}
		}
	}
}

\begin{document}

\LARGE
\centerline{Meeting to Discuss Possible Collaborative}
\centerline{Efforts between Hughes ECS Development and}
\centerline{DODS Development}
\vskip 6em
\large  
\centerline{M.I.T.  22-23 March 1994}

\normalsize
\bigskip
      A meeting between members of the Hughes Information Technology Company
developing the EOSDIS Core System, ECS and the DODS development team was held
at M.I.T. on 22 and 23 March 1994.  Attending the meeting were, from Hughes,
Mark Elkington and Ron Williamson and for DODS, James Gallagher, Glen Flierl
and George Milkowski.  The purpose of the meeting was to learn more about
each of the systems being developed, (ECS and DODS), determining areas of
commonality within the system models and discussing where cooperative efforts
might provide mutual benefit.  

      During the first day of the meeting, technical description of the two
systems were presented.  The second day consisted of exploring those areas
were the two groups felt it would be useful to collaborate in either
developing a common approach or exchanging information. 

\section{Day One}

      Jim and Glenn gave overviews of the the most recent design documents
related to DODS.  Jim spoke on the Data delivery model and the rational for
our focus on the API as the basis for the server design.  Glenn presented a
discussion of the DODS data model that reflected the `theoretical approach'
which came out of the DODS data model working group meeting.  These two
presentations were based on DODS documents that are available from the DODS
Mosaic documentation repository.  

       Mark gave an overview of the ECS program and talked briefly about the
main drivers which are producing requirements for the development of ECS.  He
expressed his concern regarding the current volatility within ECS because of
the recent redesign of the system and the paradigm shift in the approach.
What was clear from his comments was the fact that ECS has a very full plate
in terms of what requirements have been specified.  Both Mark and Ron feel
that over time some requirements will fall away as it becomes clear that
resources are limited and priorities need to be imposed.  However, at this
stage they must design with the goal of full requirement implementation.  The
most fundamental difference between the new ECS and the old ECS is that fact
that the new ECS will allow users to get on-line data from data providers.
In terms of what data is going to be provided the primary emphasis is still
on the EOS satellite data.  Thus while there is a significant level of
additional complexity that must be built into ECS in order to support users
getting the data on-line, the main data providers that ECS will be focusing
on will continue to be the DAACS.  The capability for others to make data
available through the system is a requirement.  

    Ron detailed the ECS Data Management Component.  He discussed the
three sub-architectures that make up the Data Management Component, the
Distributed Information Manager, the Local Information Manager and the Data
Server sub-architectures.  He also outlined the interaction of the different
sub-architectures with the client, trader and object request broker components
during data location.  They are using an objected oriented model as the
basis for their design approach.  The current design architecture is fairly
complex and like the DODS parts of it are still be defined.  Mark and Ron's
presentations were derived from the documentation that they brought to the
meeting and provided to us for reference.

\section{Day Two}

      The following were felt to be logical areas were collaborating would
provide mutual benefit;

\begin{enumerate}
   \item  Definition of Universal Reference, UR.  Both DODS and the ECS will
   define some kind of Universal Reference Locator similar to that
   used by Mosaic.  However, both DODS and ECS developers see the need for an
   extended syntax that goes beyond Mosaic's which would support operator and
   parameter specification.  Both DODS and ECS would also envision being able
   to encapsulate the specification of UR within another UR.  It was felt
   that at a minimum the two systems should be able to recognize the others
   UR and be able to support some parts of it.  Neither system has settled on
   a UR definition yet although DODS has a working definition (see DODS
   documentation on Virtual File Systems/Databases).  Glenn detailed his
   current thinking on the implementation of a DODS UR and Mark and Ron
   indicated that it was parallel with theirs.  Jim commented that web
   parsers already exist and that they can be used to parse a URL like name. 

   \item Protocol Negotiation.  In the DODS model protocol negotiation will
   be handled by the Protocol Arbitrator.  In the ECS protocol negotiation
   will be handled by the Object Resource Broker.  For the near future DODS
   will not be doing much design and development work on the Protocol
   Arbitrator choosing instead to focus on implementing servers and handling
   the task of negotiating server-to-server capabilities manually.  ECS
   developers will be more active in the near term on designing a protocol
   negotiation model.  They are looking into available applications that
   support Object protocol negotiation.  There are two important aspects of a
   protocol negotiation model that ECS will be addressing near term, server
   definition and client-server mismatch handling.   Hughes recommended that
   we look at Postgres which has a good object class library and might lend
   itself to being used for server definition purposes. 

   Parts of the information necessary for protocol negotiation will be
   distributed within the Data Information Manager, the Local Information
   Manager and the Data server with the Hughes model.  What is required in
   order to provide a consistent view of these managers across the system is
   a formally defined schema for each level.  Hughes sees benefit to
   following our near term development of a translator for the JGOFS-NetCDF
   API and the DODS Ancillary Data.  They felt that it would be relevant to
   the definition of their schemas.

   \item  Location Function.  Hughes is currently researching what approaches
   are available for supporting the data location function.  They are 
   in the intermediate stages of this discovery process and are hoping to
   have some recommendations in the May time frame.  They expressed a
   willingness to share and coordinate development in this area with us.

   George mentioned that we are funded to pursue development of a location
   function within DODS in cooperation with with NASA's GCMD and intend to
   meet with Lola Olsen in the near future to discuss this.  In particular,
   DODS is interested in discussing how the DIF could be used to support
   accessing distributed data resources.  Hughes is also interested in
   coordination with the GCMD but mentioned that the current structure of the
   DIF and the information that it contains is incomplete from their
   perspective.  George agreed and stated that one of the issues that needed
   to be addressed with Lola was the possibility of modifying the DIF to
   accommodate referencing distributed data sets and systems explicitly.


   \item Data Model.  The data model that Glenn has developed was of interest
   to Mark and Ron.  Hughes does not currently have a well defined
data model and the methodology that Glenn presented was felt to be a
useful way of formally defining a data model its structures and operators.  

   \item Error handling was another issue raised.  Jim discussed the
   ad hoc solution currently implemented with the servers that he is
   developing (simple error flag) and pointed out the fact that it was
   not easy to capture and transfer back to the client server side
   errors and diagnostics.  RPCs do not handle errors well within the
   framework on many APIs because those APIs rely on global variables
   and system calls to report errors verbosely. The function calls
   only tell the calling function that an error occurred, not what
   error occurred. For that information you need to inspect a global
   (e.g., UNIX's errno global). When an error occurs it happens in the
   context of the server process (and thus the server's copy of
   errno). The client cannot access the server's global variables
   easily using RPCs. Other IPC technologies like shared memory might
   facilitate such access.  

   \item The issue of asynchronous communication was also raised.
   RPCs are not asynchronous. RPCs are based on the procedure call model
   and it just doesn't support asynchronous behavior without becoming
   very distorted.  This was considered a significant limitiation to RPCs.


   \item  Server development and exchange.  Hughes is developing an HDF
   'Wrapper'.  This will amount to ECS HDF style (i.e.; API) which defines a
   subset of the available HDF functions.  DODS will be developing an HDF
   server pending  acceptance of its NASA MTPE proposal.  It was agreed that
   we will need to keep each other up-to-date on overlapping work on HDF.  

   Hughes expressed and interest in possibly utilizing the DODS NetCDF server
   for enveloping EOSDIS V0 datasets.  Mark and Ron will alert those within
   Hughes working on the V0 to V1 migration of our work.

   Hughes mentioned that there is a possibility of its developing servers for
   GRIB and BUFR which it would be willing to provide to DODS.


\end{enumerate}


\section{Future Work}

	At the end of the meeting the following cooperative
arrangements had been agreed to:

\begin{itemize}  
  \item DODS and Hughes will work toward a common Universal Reference.  This
will be accomplished through the following: 
    \begin{itemize}

      \item  Mark will send us the currently evolving ECS specification for a
UR for our comment. 
      \item  We will generate a position paper on our present requirements
for UR based on the data model that we are developing.  We will provide
Hughes with a draft for comment.
      \item One of the DODS developers (Glenn or Jim) was invited to
participate in a meeting to be held at Hughes in Landover on Protocols for
CEOS. The time of the meeting has not been decided most likely it will be
held the week of the 16 May or 23 May.

    \end{itemize}

  \item  DODS and ECS will try to coordinate their activities with regard to
development of a Locator Service function.  The following activities are
planned;
    \begin{itemize}
      
      \item DODS is funded to work with the GCMD to develop a Locator Service
for DODS.  We will be meeting with Lola Olsen in the near future to discuss
coordination and development.  At that meeting we will be exploring possible
ways of using the DIF as the basis for Locator Service.  We will be drafting
a agenda for the meeting and will circulate to Hughes for comment.
Hughes wondered whether it would be appropriate for them to also work
from their side to recommend modification to the DIF structure.  It
was felt that for the time being DODS should take the lead but would
keep in close touch with Hughes on developments.

     \item Hughes is working on designing a Locator Service using  Object
Resource Broker, Advertising Service  and Trader components within their
model.  This work is under development.  Hughes recommended that DODS come to
Hughes in Landover for a formal presentation and discussion of their design
in the near future.  The suggestion was made that we piggyback on the May
CEOS Protocol Meeting.  

    \end{itemize}
\end{itemize}

\end{document}