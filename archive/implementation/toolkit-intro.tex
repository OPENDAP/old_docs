\section{Introduction} 
\label{introduction}

This paper describes the DODS core, or toolkit, software. The toolkit
software contains one implementation of the \Dap\ and can be used when
building both DODS data servers and DODS surrogate libraries. The software
described in this paper is written in \Cpp.

The toolkit software is intended to serve as a foundation on which DODS
surrogate libraries and data servers can be built. The toolkit software
consists of two \Cpp\ class libraries and a small C library. Using the
toolkit software, a data server and surrogate library for a new, previously
unsupported, API can be built.  Although it is possible to build the client
and server components from scratch, the toolkit represents a significant body
of software, and using it can save considerable development time.

The DODS toolkit software is used by specializing classes in both of the \Cpp\
libraries so that they contain features required for a particular API. The
two libraries contain classes which implement the \dap and which provide
network I/O capabilities. The library which contains the DAP software is
referred to in this document as the `DAP library' while the library which
caontins the network I/O software is referred to as the `netio library'. 

To use either the DAP library or the netio library, new \Cpp\ classes will
have to be written. The DAP library is a collection of {\em abstract\/}
classes\cite{stroustrup:cpp} and as such cannot be directly instantiated. In
order to be used in any program, each of the abstract classes must be
specialized. The netio library classes are not abstract, but are designed to
provide only the bare network I/O capability needed by all DODS servers or
clients (surrogate libraries). In oder to make effective use of the netio
libary, it will need to be specialized. 

Specialization of the libraries is straightforward \Cpp\ programming. The dap
and netio libraries are themselves much larger than the specializations that
must be written to make a complete surrogate library and/or data server. For
release 1.x of DODS, the dap and netio libraries contain 3291 and 1188 lines
of code respectively, while the specializations of the dap and netio
libraries for NetCDF contain 595 and 178 lines each\footnote{Lines of
code counts only the actual \Cpp\ statments, not comments and blank
lines.}\footnote{In order to build a functioning client library and server
approximately 1200 extra lines of code were required. These were 80% derived
from the Unidata, inc. NetCDF software distribution}.

The sharing of code by the the client and server reflects the object-oriented
nature of the toolkit and its data-model centric design. There are three
basic objects which the system manipulates: a variable which contains data,
the DDS which describes the logical structure of a data set, and the DAS
which contains attribute (i.e., auxillary) information about variables in the
data set.  The DAP bases the translation of supported APIs on a common data
model used for all APIs. The class hierarchy with {\tt BaseType} at its root
(See Figure~\ref{fig:basetype-hier}) is the set of classes which form this
model.  Furthermore, it is a goal of the toolkit software to be as general as
possible; other classes, such as the DAS and DDS, are also used by both
clients and servers because they contain member functions which apply to both
the client and server side of the system. Each of the objects (data variable,
DAS and DDS) has both a binary representation which can be manipulated by
\Cpp\ software {\em and\/} a network representation (using MIME) that can be
transmitted from one machine to another.

In this document it is assumed that the reader is adding complete support for
a new API to DODS. However, it is possible to build only a DODS server or
client using the DODS toolkit. In most client-server systems, that would make
little sense because without a matched client and server there is no
system. Because the DODS client and server communicate using a prorocol that
is shared by several clients and servers, there is already a pool of both
clients and servers which could be used in conjunction with either a new
server or a new client.

The remainder of this document contains a description of the DAP library (see
Section~\ref{dap-library}, the netio library (see Section~\ref{netio-library}
and a general discussion of the construction of DODS data servers and clients
(surrogate libraries). Unfortunately, there is neither the time nor space
present to discuss \Cpp, object-oriented programming or client-server
systems. However, the following references do address those topics:
\Cpp\cite{stroustrup:cpp}\cite{meyers:cpp}\cite{cargill:cpp}, object-oriented
programming\cite{meyer:oop}, client-server systems\cite{steven:unp}.

