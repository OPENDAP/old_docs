\section{Building DODS Data Servers}
\label{toolkits:servers}

A DODS data server is really made of five executable programs; httpd, a CGI
program (usually implemented as a UNIX shell script) and three filter
programs (made using \Cpp\ and possibly C software) used to read various
types of information from the dataset. Figure~\ref{toolkits:server-design}
shows the data server design. 

When httpd recieves a URL which references one of the programs in its
ScriptAlias directories it executes that program with information taken from
the URL as arguments\cite{httpd:cgi}. The program, know as a CGI program, is
started such that its standard input and standard output descriptors are the
same at those belonging to httpd. Thus, the CGI can send information to the
httpd's client (the program that is fetching the URL from the httpd) by
writing to its standard output. 

The three filter programs are started by the CGI depending on the extension
given with the URL. If the URL ends with `.das' then the DAS filter program is
started. Similarly, the extension `.dds' will cause the DDS filter to run and
the extension `.dods' will cause the DODS filter to run. The CGI program,
which serves to dispatch the request to one of the three filter programs, can
be very simple. In both of the sample servers distributed with DODS version
1.x, the CGI takes its name and catenates `_das', `_dds' or `_dods' depending
on the URL extension.

Each of the three filter prgrams returns one of the three DAP responses
descriped in \DAP. The DAS filter program returns the DAS, the DDS filter
returns the DDS and the DODS filter returns the data cooresponding to a
particular variable. 

URLs that access DODS data (i.e., that access a DODS data server and request
iformation) differ from those typed in by users in that users never supply
the extensions `.das', etc. Instead those extensions are added by the
client-library (see Section~\ref{toolkits:clients} for more information about
where this takes place). A URL supplied by a user accesses data from the
user's perspective. However, the client library must access the three
different servers to satisfy any given dat arequest from the user. Thus, URLs
which access one of the three filter programs has the extension added to it.

A utility called `geturl' is supplied with the DAP core software (in the
version 1.x software distribution it is in the \dollar(DODS_ROOT)/etc
directory). Ths utility can be used to test each of the three filter programs
by supplying it with a URL that will access that filter program (i.e., a URL
with the extension allready appended). For example, to access the netCDF DDS
filter included in the DODS version 1.x netCDF software geturl would be
invoked as {\tt geturl -s http://dods.gso.uri.edu/cgi-bin/nc/test.dds}. In
this example the {\tt -s} option causes geturl to access the server
synchronously, {\tt dods.gso.uri.edu} is the name of the machine on which the
server is running, {\tt cgi-bin} is the httpd's ScriptAlias directory, {\tt
nc} is the DODS dispatch CGI script (whcih users think of as ``the server'')
and {\tt test} is the data set. If a user were to access this same data set
with a program they would use the same URL {\em excpet\/} for the `{\tt
.dds}' extension.

\begin{figure}
\label{toolkits:server-design}
\caption{The Architecture of a DODS Data Server.}
\end{figure}

\subsection{Writting the dispatch CGI}

The dispatch CGI must take a URL intended for a DODS data server filter
program and invoke the correct filter program based on the URL extesion. The
CGI should be written so that it can invoke not only the three standard DODS
filter programs (DAS, DDS and DODS) but also additional filter programs so
that sites can customize their server to retrun other types of data (e.g., a
fourth filter program might return a GIF rpresenation of the data).

\begin{figure}
\label{toolkits:cgi}
\begin{verbatim}

\end{verbatim}
\caption{The CGI distributed with the version 1.x DODS-netCDF software
(editted).}
\end{figure}

\subsection{The DAS filter program}  

\section{Building Client Libraries}
\label{toolkits:clients}
