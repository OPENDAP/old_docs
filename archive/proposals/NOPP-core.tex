
% The system software section of the NOAA NOPP proposal. 11/10/99 jhrg
%
% $Id$

\documentclass[10pt]{article}
\bibliographystyle{plain}
\newcommand{\Cpp}{{\rm {\small C}\raise.5ex\hbox{\footnotesize ++}}}
\begin{document}
\title{NOAA NOPP Software}

\maketitle

\section{DODS Software}

The DODS system software can be divided into three categories. First there is
the software that implements the Data Access Protocol, DODS' unique data
transmission protocol. Second there is the software that implements servers
and clients. This software uses the DAP and also adds other features that any
functional system must have but which do not belong in a data transmission
protocol. Lastly, some features of the DODS system software don't make any
use of the DAP but instead use features of the World Wide Web (WWW). We
propose to extend DODS' software by continuing development in each of these
three categories. In the following, the current status and near-term plans for
the software is described and new development is proposed.

% \begin{quote}
% \emph{Describe the software as middle-ware. Define middle-ware. Explain its
% importance. Relate to other ventures (e.g., WWW). Relate our project to the
% CERN/W3C WWW project?}
% \end{quote}

\subsection{The Data Access Protocol}

The Data Access Protocol (DAP) is a simple request/response protocol which
provides access to data. The DAP is used to transfer three types of objects
from remote sites (servers) to a client: 1) An object which describes the
variables contained in a dataset by listing their name, datatype and
relationship with other variables in the dataset, 2) An object that lists
attributes for each of those variables and, 3) A data object which contains
both variable names, types and relationships along with numeric values for
each variable. A request for a data object can be accompanied by a Constraint
Expression (CE) which lists which of the variables, and of those variables
which values or ranges of values, to return. Thus the DAP provides both a
simple way to ask a dataset about it contents and to selectively read from
that dataset.

The DAP has now been implemented in two different languages on a variety of
different computing platforms. As it stands now, the DAP has been fully
implemented in \Cpp\ and runs on a variety of UNIX platforms. This software
is stable and has been used by our group and others since 1997. In addition,
our partners at JPL have ported this software to the Microsoft Windows
operating system (Windows 95, 98 and NT). This port is currently available
only to developers principally because some of the client software is not
working well. However, the DAP portion of the port is very robust. We have
also developed, through our partners at JPL and OSU, a second implementation
of the DAP using the Java programming language. This implementation of the
DAP is not complete, but we expect that it will be complete by January 2000.
Thus the DAP can be used on both UNIX and MS Windows and by \Cpp, C and
Java programs.

The DAP as a specification has also been evolving during the last two years.
The original design called for a limited capability to add functions to
individual servers and have those functions called from within the CE. This
capability has been expanded to include a standard set of functions which
provide access to new datatypes. Unlike the basic datatypes around which DODS
was designed, these types carry a lots of implicit metadata. While this was
seen as a weakness in general, for certain datatypes it proves to be the most
pragmatic way to transport information. For example, the DAP now supports a
set of CE Functions for processing dates. These functions provide an
interface which hides a dataset's internal representation of the date
information while providing a standard external representation for the rest
of the world. To implement these CE function the specification of the DAP was
changed to generalize functions.

There are some significant enhancements to the DAP on which we and others are
currently working. These include server-based datatype translation,
geographical selection functions, response caching, greater control over the
transmission process, better error handling and more efficient memory
management. Each of these improvements does not fundamentally alter the DAP
specification itself but, instead, simplifies using, or improves performance
of, the DAP. Most of these suggestion have come from requests from the
developer community based on their experience.\footnote{For example, response
caching is being added because our partners at PMEL commented that users in
Australia found the DODS-enhanced Ferret hard to use because of long network
latencies.} Because each of these improvements, while not radically altering
the DAP, have come from the user/developer community, they will almost
certainly have a huge impact on the utility of the protocol.

\subsection{DODS Client and Server Software}
\label{sec:c-s}

The DODS project software most visible to users are its clients and servers.
The DODS clients we have written fall into three categories:
client-libraries, command line clients and Graphical User Interface (GUI)
clients.  The client-libraries are software libraries that can be used by
programmers and sophisticated users to build clients for DODS with existing
software.  These client-libraries implement \emph{already existing} APIs but
do so in a way that enables them to read information from DODS servers. We
have two client-libraries currently available; one for NetCDF 3.x and one for
the JGOFS API. The NetCDF client-library has been used very successfully,
enabling most notably, Ferret\cite{Ferret} to access DODS servers. In
addition, this client library has been tested with GrADS, Perl (using the
netCDF/Perl interface), MexCDF, GMT and others. Client-libraries are a
powerful software tool because they make it simple to leverage existing work
in the important process of building a useful body of client software.

The command-line clients we provide work with the UNIX and Windows operating
systems and with the Matlab and IDL commercial analysis packages. These
clients require users to compose DODS URLs, which many find a complex task,
but provide a useful basis for testing servers, and more importantly, writing
scripts which read from DODS servers. In effect the command-line clients for
both Matlab and IDL are similar in function to the client-libraries because
they provide a programatic interface to DODS servers. These command line
tools can (and have) be used to build complex graphical interfaces and
scripts which automate access to data served by DODS under program control.

The last group of clients for DODS are Graphical User Interfaces. We have
built a GUI fro Matlab, based on our Matlab command-line client. In addition,
our partners at HAO have developed a similar GUI for IDL using the IDL
command-line client. These interfaces are very popular with users because
they simplify choosing datasets and fetching data by insulating them from the
URLs used to retrieve data. Other GUI clients that read from DODS servers
have already been mentioned; Ferret is the most prominent example. In all
these cases, the GUI clients are built using either a client-library or one
of our command-line clients. In essence the GUI clients are almost (or are,
in fact, in the case of Ferret) separate projects which make use of DODS
as middle-ware.

DODS currently provides six different servers which make it possible for
clients to access data stored in many different ways. We have servers for
NetCDF, HDF4 and HDF EOS, JGOFS, FreeForm, DSP and Matlab. The FreeForm and
JGOFS servers are easily customizable to various institution-specific ways of
storing data. 

In the near-term we plan on expanding the capabilities of some of our
existing servers and adding servers for some new file formats. Both our
FreeForm and JGOFS servers will undergo changes to make them more usable.  To
the FreeForm server we will add capabilities to read data files with ragged
lines, and comma separated values. The JGOFS server will be modified so that
it can run our more generalized for of CE function. We plan to add servers
for GRIB, BUFR and HDF5 within the next year. Our efforts to build GRIB and
BUFR servers have been hampered so far by lack of experience with these
formats within our group. However, we are going to pursue building these
servers through our partners at GMU/COLA and UCAR. similarly, we are going to
pursue building a server for HDF5 though our partnership with NCSA.

Until recently DODS was limited to the UNIX family of platforms. We provide
source code as well as binaries for Sun, Dec, SGI and Intel-based UNIX
workstations. However, the software has recently been ported to the Windows
platform by our partners at JPL. While this port is still in its infancy, it
holds great promise to open up many doors for DODS. We will continue working
with JPL to get the port to the stage where it can be released.

DODS currently includes minimal support for datasets comprised of collections
of files. Because this is an important class of dataset, we have worked to
broaden its support. Our first revision file servers do this by providing a
way to index the files which make up a multi-file dataset by date, time
and/or parameter. In the next year we plan to extend this so that accessing
the catalog and accessing data in any of the files it lists can be made in a
single step, effectively aggregating the files into a single dataset from the
perspective of a client program.

DODS' focus to date has been on data accessibility. However, we are now
extending the system to provide ease-of-use features. When users are
unfamiliar with a dataset, using that dataset is greatly simplified if some
relatively uniform set of information about it is available. This is hardly a
new concept and DODS' DAP has always provided ample ways to add this sort of
information easily and without actually modifying the dataset itself. Our new
ease-of-use conventions will help data providers make data available that
will not only be easy to access, but also easy to use. These features will be
based on the COARDS standard\cite{COARDS} which has the added benefit of
increasing the viability of our netCDF client-library\footnote{Ferret, GrADS
  and similar programs which use NetCDF are much more powerful tools when
  used with COARDS-compliant datasets.}.  However, these new features, which
amount to a new, though small, set of conventions for datasets, will not be
limited to COARDS.

\subsection{The World Wide Web and DODS Software}

Is is possible to run the DAP over any basic transport protocol. Our
implementations use HTTP. Because our implementation is layered on HTTP our
system can take advantage of many features of the evolving WWW
infrastructure. Because the DAP and HTTP are not tightly coupled, changes in
the DAP (and \emph{vice versa}) do not affect the purely-HTTP based services
of DODS.

Because DODS uses HTTP to send requests and responses to and from servers,
any web browser can be considered a potential DODS client. We have capitalized
on this in several ways. DODS servers now support a variety of HTTP/WWW
based accesses. These are:
\begin{itemize}
\item All of the basic objects returned by servers in request to a response
  are pseudo-MIME\cite{rfc:mime} documents. The objects describing variables
  and their attributes are text documents and can be displayed by any web
  browser. The data object is a binary document and browsers will offer to
  save it to a file.

\item It is possible to request data in ASCII form from a DODS server. This
information is returned as a plain text document which can be viewed in a
browser, saved to a file or read by a spreadsheet. All the CE features which
can be used to tailor a request when asking for a data object can be used
when asking for ASCII data.

\item Another WWW feature recently introduced to our servers is a simple HTML
form-based query interface. This interface provides a way for users to view
the variables and their attributes from a given dataset and request data
without first mastering the somewhat complex syntax of DODS.\footnote{DODS
uses URLs to reference data, but most users find that the CE notation is
complicated---the web interface helps users write URLs.}

\item Server version information, help and general information can also be
requested.
\end{itemize}

One component missing from DODS is a way to find data. Because many other
groups had approached the distributed dataset problem from this
end---building systems to locate datasets---we felt it was acceptable to not
include a dataset registry. However, it has become apparent that such a
registry is really needed even if it provides fairly limited features. First
and foremost, a registry will provide an interface for those other projects
which already have years of experience in building dataset search tools to
learn about new DODS datasets as they come on line. We have implemented a
prototype registry which uses standard WWW technology to interface and HTML
form with a simple database system. We will finish this prototype to produce
a working system and register our existing datasets with it. We will also use
this registry to provide information about DODS datasets to the GCMD who will
incorporate information about DODS datasets in their catalog. Our goal is to
complete this first level of catalog within one year.

\section{Future Directions for the DODS Software}

We propose four areas of expansion for the DODS software:
\begin{itemize}
\item Add support for GIS by adding both client and server capabilities,
\item Add support for user specified virtual datasets by extending our File
  Server software,
\item Continue expanding our ease-of-use features and,
\item Enhance the general capabilities of our DAP implementations.
\end{itemize}

\subsection{GIS systems}
Geographical Information System (GIS) analysis packages such as ArcView can
benefit greatly from access to the types of science data served by DODS. We
propose to work with ESRI, Inc. to build an interface for its GIS analysis
packages using a similar approach to the command-line clients we have already
built for IDL and Matlab. This client will enable the huge GIS audience to
access DODS data and merge it with GIS data.

We also propose to build servers for the GEOTIFF and ARC-GRID formats.
Support of these formats within DODS appears to be straightforward and will
enable the science analysis tools such as Matlab, Ferret, etc. to use the
extensive body of GIS data in concert with the science data already served by
DODS.  

The overall goal of both planned GIS efforts is to bridge the gap between GIS
data and GIS analysis tools and the corresponding data and analysis tools in
the science community. We have some exploratory work currently underway, but
the scope of this effort will require between one full-time person for
two to three years, depending on level of support we can secure from outside
the project.

\subsection{Virtual datasets}
We want to extend our current and developing file server technology so that
datasets spread across several computers can be combined and accessed as if
they we a single object. This activity is a natural outgrowth of our work
with file servers (see Section~\ref{sec:c-s}). File servers aggregate
multiple files into a single, cohesive, dataset. While our current file
servers are fairly primitive, we are in the process of making their use
indistinguishable from servers used to access single-file datasets. If DODS
can see both single- and multi-file datasets as equivalent using a server
that automatically aggregates the components of the later, then it should
also be possible to apply the same \emph{essential} process to multiple
remote datasets. However, this will be complicated by the nature of
distributed systems. It has become apparent to us that techniques which work
well on single host systems and within intranets are often not well suited to
distribution across the Internet. This observation has been backed up by
others\cite{waldo:dist-comp}. Thus, adapting our file server so it can be
used to aggregate remote datasets will require that we make significant
extensions to the file server technology. This will take one person two years.

\subsection{Additional ease-of-use features}

The DODS project plans on developing an initial set of dataset conventions
for oceanography within a year and then working to extend those conventions
to increase the scope of datasets they address. Because DODS is moving
towards providing ease-of-use features in addition to ease-of-access,
continuing this effort is extremely important. Furthermore, while this part
of the project may appear to be a technical one, it is more an adaptation of
technology to long-held practices in the oceanographic community. 

We propose to address the communal aspect of data by extending our
server-based ease-of-use features (which includes, in part, a metadata
convention) to hub sites. These sites will hold additional information about
various datasets contributed by the dataset providers themselves as well as
by other people; data users and DODS project data specialists. It is feasible
with current DODS technology to build a functional prototype system which
could hold several different types of information about the 80--100 datasets
currently in our Matlab GUI client. Users, or other DODS client programs,
could go to one hub site for COARDS-compliant metadata for those datasets,
another hub site for FGDC metadata about those same datasets.

We have found that in DODS (and other system implementors have echoed this
observation) the toughest issue to resolve is the balance between the burden
placed on data providers by requiring that they include information about
their datasets and the usefulness of those datasets to the naive user.  The
advantage of using hub sites over storing the metadata with the dataset is
that users can contribute information about datasets, and very rich bodies of
metadata can be built, without taxing the provider of those data. Similarly,
hub sites are different from centralized catalog systems because it has been
impossible for individual users to set up and run their own site\footnote{The
  GCMD has made some moves towards this, but they currently don't store the
  type of ease-of-use information we are interested in. In fact, we have a
  partnership with the GCMD and it is certainly conceivable that they would
  be (or run) one, of many, of these hub sites.} and it has been the practice
that these centralized catalog systems do not allow third-party submissions.

The development of metadata hub sites, and the continued development of our
ease-of-use features, will require a person for three years. 

\subsection{DAP Improvements}

Improvements to the DODS DAP can affect the entire system in positive ways.
We propose to expand or enhance the DAP in three ways. By adding support XML
encodings the DAP will be more generally accessible. By developing a CORBA
2.x toolkit both CORBA-based servers and clients can be participants in
DODS. Finally, by enhancing our support for HTTP, our servers can transfer
data more efficiently. While each of these changes won't directly change what
users can or cannot do with DODS, they will give other developers more
flexibility and power which, in turn, will improve the user software they
build.

Support for XML can be easily integrated into DODS because, although DODS'
design predates XML\cite{w3c:xml} it can accommodate and use XML in a
seamless way. DODS currently transfers information using a small collection
of WWW documents. Three of these are MIME-encapsulated \Cpp\ objects or, more
correctly, the persistent representation of those \Cpp\ objects. When DODS
was originally designed, we chose a simple text format for that persistent
representation. To accommodate XML we will add a persistent representation
that uses XML instead of plain text. Thus, it will be fairly straightforward
to extend our servers (by extending the DAP) so that XML representations of
objects used by the DAP are available.

We plan to build interfaces to CORBA for both the client and server ends of
DODS. CORBA can be thought of as both a transport and encapsulation
technology\cite{siegel:corba-prog}. Because the DAP can easily be used with
transports other than HTTP and because both CORBA and DODS are object-based
technologies, it is possible to build interfaces that go from systems
which use one (either DODS or CORBA) to the other. We propose to build a
CORBA object which can be used to access DODS datasets by encapsulating the
three basic DODS objects in a single CORBA object. This would provide a way
for CORBA-based clients to interact with DODS servers. We also propose to
build a toolkit which will enable a CORBA object to act as a DODS server.

The last of three planned improvements to the DAP will add full support for
HTTP 1.1. This will improve the efficiency of our servers in two important
ways. First, our servers will be able to reuse TCP/IP connections which are
time consuming to set up. This will benefit DODS clients because they will
not need any modification and will see significant performance improvements,
especially for clients, like Ferret, that access servers through one of our
client-libraries. These clients typically ask for several objects in a row
and currently our software rebuilds the TCP/IP connection for each request.
Full support of HTTP 1.1 will eliminate this waste resulting in shorter
response times for users. 

Full support of HTTP 1.1 will have another benefit; users who stop
transmissions in mid-stream will be able to restart those transmissions for
the place they stopped. This provides users with much flexibility because
they often don't know ahead of time how long a given request will take or how
much data will be returned. being able to stop a request and later on restart
provides an important level of control for users over the data access process.

To accomplish these goals will require one person for three years.

\raggedright
\bibliography{../../boiler/dods}

\end{document}