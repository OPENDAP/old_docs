% Chapter to the DODS Programmer's Guide
%
% $Id$
%
% $Log: ch01.tex,v $
% Revision 1.7  2004/02/18 06:38:48  jimg
% Various changes, mostly for the DODS --> OPD macros.
%
% Revision 1.6  2000/10/04 15:00:23  tom
% changed \figureplace definition, cleaning up...
%
% Revision 1.5  2000/03/21 22:04:33  tom
% added server spec appendixes, fixed xref tags
%
% Revision 1.4  1999/07/22 18:54:56  tom
% fixed errors
%
% Revision 1.3  1999/02/04 17:46:05  tom
% Modified for dods-book.cls and Hyperlatex
%
% Revision 1.2  1998/12/07 15:51:33  tom
% updated for DODSv2.19
%
% Revision 1.1  1998/03/13 20:50:02  tom
% created API manual from James's Toolkit document
%
%

\chapter{The DODS Client and Server Toolkit}
\label{tk,overview}

The Distributed Oceanographic Data System (DODS) is a system used to
facilitate access to scientific data on the internet.  Using DODS, you
can turn any data analysis program that uses one of several data
access APIs into a powerful, internet-friendly data browser.  For basic
information about the structure of DODS and its components, please see
\OPDuser. 

The DODS Toolkit software consists of a collection of \Cpp\ classes used
to build DODS data servers and clients:
\indc{toolkit!contents}\indc{DODS!toolkit contents}

\begin{description}
\item[Type classes] A complete implementation of the DODS data access
  protocol (DAP).  This consists of a set of virtual classes for
  different types of data, which must be sub-classed to use.
  
\item[Data classes] This is a set of classes, including both the Data
  Descriptor Structure (\class{DDS}) and the Data Attribute Structure
  (\class{DAS}), designed to contain information about a dataset's
  data.

\item[Connect classes] These classes are used by a DODS client to
  mimic a durable connection to a DODS server.

\item[\class{DODSFilter}] This class consists of several utility
  functions useful for writing DODS servers.

\item[CGI Utilities] The file \lit{cgi\_util.h} describes a small
  number of other common functions needed by DODS servers.

\end{description}

The DAP toolkit contains three class hierarchies: one for each of the
\class{DAS} and \class{DDS} objects, rooted in classes with those names,
and one for the variables, rooted at the class \class{BaseType}. The
following sections present an overview of these class hierarchies and
describe, in general terms, how they are used to build the
client-library and data server. See \sectionref{tk,das-dds-objects}
for information about the \class{DDS} and \class{DAS} classes.
To use the type classes, they must be subclassed.
\chapterref{tk,subclassing} provides a detailed description of this
process.

Detailed information about the \class{DAS} and \class{DDS} themselves
is available in the \OPDuser.  Descriptions of the classes and their
member functions are in the reference material
\texorhtml{, \chapterref{tk,classref}}{, \OPDapiref}.

The DODS toolkit also contains two classes used by DODS clients to
manage network connections between themselves and a server. The
\class{Connect} object manages a single connection with some DODS
server, and the \class{Connections} object manages a group of
\class{Connect} objects.  The Connect class is meant to be subclassed
when used by a real client library. See \chapterref{tk,manage-conns}.

\section{The \class{DAS} and \class{DDS} Objects}
\label{tk,das-dds-objects}

\indc{DAS class}\indc{DDS class}\indc{attribute data}\indc{data
  attributes} \indc{data model}\indc{data description} The dataset
attribute structure (DAS\texonly{:~\pagexref{tk,das}}) and dataset
descriptor structure (DDS\texonly{:~\pagexref{tk,dds}}) objects are
used to store information about a data set's variables. These objects
are used on both the client and server sides, although there are class
features that only pertain to one or another of the roles.  They can
be thought of as metadata objects.  In this book, however, we will
avoid the term \emph{metadata} because often this is \emph{data} to
many users.

It might be said that neither the \class{DAS} nor the \class{DDS}
contain actual science data --- the \class{DAS} contains attribute
information from the data set while the \class{DDS} contains
structural information about the data set and variables in the data
set.  Since the boundary between data and metadata (or data
attributes) is often a blurry one, this is not a distinction we will
insist on. 

To build both the \class{DAS} and \class{DDS}, the server either reads
information directly from the dataset or from DODS-specific
\new{ancillary data} files, depending on the capabilities of the data
access API used to access the data.  The \class{DAS} and \class{DDS}
server \new{filter} programs do this and then transmit the resulting
object to the client.

On the client side, the DODS client\footnote{Whatever it is.  The DODS
  client can be another server, a user application linked with a
  DODS-compliant API, or a standalone program using the DODS data
  access protocl API.  In any case, the use of the class libraries
  described in this document is identical.} uses information in the
DAS and DDS to satisfy API calls issued by the user program requesting
information about variables, their type, shape, and attributes.  The
client requests both of these objects when it first contacts
the remote data set. The \class{DAS} and \class{DDS} objects are then
stored as part of a \new{virtual connection} to that data set and can
be used repeatedly by the client library without retransmission.

\indc{Data Attribute Structure|see(DAS)}
\indc{Data Description Structure|see(DDS)}
\indc{DAS!internal vs. external representation}
\indc{DDS!internal vs. external representation}
The \class{DAS} and \class{DDS} objects have both an internal and an
external representation. Internal to the DODS client or server, these
structures are stored as \Cpp\ objects, while their external
representation is as text. The object is sent from the server to the
client using this text representation. Each of the two classes
contains a parser which can read the text representation and recreate
the object's internal representation. In addition, it is possible to
write the text representation for either object (using a text editor)
and then use the parser to create the internal, \Cpp, object.
Furthermore, the text representation is a type of persistence and can
be used to build a flexible object caching mechanism.
\indc{persistence!object}\indc{object persistence}
\indc{caching}\indc{performance improvement}

One possible use for this caching mechanism is to store the \class{DAS} and
\class{DDS} for a dataset and use the stored versions in place of opening the
data set and reading information about it and its contents. For large
data sets with many variables this can result in a significant
performance improvement, and several of the packaged DODS servers use
it.  

The caching mechanism may also be used on the server-side to
store extra information about the data set---information that is not
present in the data set proper, but which the data provider would like
included when people access the data set via DODS.  In this scenario,
the data server first integrates the data set file(s) using the API
and builds the \class{DAS} and \class{DDS}. Once an initial version of
the \class{DAS} and \class{DDS} are resident in memory, the parser is
used to read an external text file which contains additions to, or
corrections of, the information extracted from the data file. This
information is the ancillary data introduced above. Several of the
DODS servers use this mechanism. 

\subsection{The \class{DAS} Object}
\label{tk,das}

\indc{DAS object}\indc{object!DAS}\indc{attribute information}
The \class{DAS} contains attribute information that is generally
\emph{not} used by software when processing the variables; this object
is specifically designed to hold all the information in a data set
that has no where else to go.  Each variable can have an unlimited
number of attributes. There are also `global' attributes that apply to
the dataset as a whole. Each attribute is a set of three elements: the
attribute name, type and value. The supported types for an attribute
are: Byte, Int32, Float64, String and URL, and vectors of these types.
These types have the same range of values as the corresponding types
in the \dap.

\indc{global attribute}
A variable's attributes can be any qualities that are not part of the
\class{DDS}.  Thus, the type and shape of a variable are \emph{not}
attributes; they are characteristics of the variable and are part of
the \class{DDS} (see Section~\ref{tk,dds}). However, users may
store any other information as attributes\footnote{Actually, there is
no reason that type, etc. cannot be stored as an attribute; however,
it must be in the \class{DDS} regardless}, including paragraphs of
text, vectors of integers and floating point values, and so on. There
are some standard attributes required by the DODS standard.  The
system \emph{could} work without these attributes, but they are
required anyway to make other aspects of the systems work better.

Many APIs support the concept of information in the data set that is
not structured, and provide function calls to access that information.
The \class{DAS} object provides a place for all that information. The
data server can interrogate the data set and build the \class{DAS} and
the client-library can use the \class{DAS} object to satisfy many of
the API calls requesting information about the variables in a data
set.

Figure~\ref{fig,das-object-diagram} is a diagram of the \class{DAS}
object. The \class{DAS} consists of the following components:
\indc{DAS!diagram}

\begin{itemize}

\item A list of attribute tables, each of which is a list of
  attributes for a particular data variable.  Since data variables can
  contain other data variables (as with a compound data type, for
  example), an attribute table can contain other attribute tables.

\item A parser and scanner to read a text version of the
  \class{DAS} object and create the corresponding \Cpp\ (binary)
  representation in memory.

\item A printer with which you can create a text version of the
  \class{DAS} object.

\end{itemize}

\figureplace{A \class{DAS} Object}{htb}
{fig,das-object-diagram}{DAS.ps}{DAS.gif}{}

The \class{DAS} printer (\lit{DAS::print()}) is used to create a
textual representation of the \Cpp object.  The object is transmitted
from server to client using this representation.  The scanner and
parser (\lit{DAS::parse()}) reads the printed representation and
creates a \Cpp object to match this specification.  This object method
can be used to read the text version from a disk or from a network
connection, depending on the input stream identified for it.
\indc{DAS!transmission}

The third part of the \class{DAS}, the attribute table list, points to
a series of \new{attribute tables}.  An attribute table is a list of
name-type-value triples used to describe a data variable.  Each
attribute describes some aspect of the data variable.  The example
list in figure~\ref{fig,attr-table-diagram} shows what a table might
look like for one of the data variables in
figure~\ref{fig,das-object-diagram}. (The types have been left out of
the diagrams for clarity.)

\figureplace{An Attribute Table}{hptb}%
{fig,attr-table-diagram}{attrtable.ps}%
{attrtable.gif}{}

The \class{DAS} object contains member functions to add or retrieve
\class{AttrTable} objects as well as individual attributes based on a
variable name. In addition, the \class{AttrTable} object may be
traversed using a \class{Pix}\footnote{A \class{Pix} is a %libg++
``pseudoindex'' object. See the libstdc++ documentation for more
information.}.
\indc{AttrTable class}\indc{attribute!table}

The value of an attribute can itself be another attribute table.  So,
for example, an aggregate variable that contains other
variables, such as a Structure, might have an attribute table for each
of its member variables.  In figure~\ref{fig,nest-attr-table-diagram},
you can see illustrated the attributes of an aggregate data variable.
The first three attributes apply to the collection of data (it was
taken with a CTD instrument, from the good ship R/V Endeavour, and so
on), while the next three attributes reveal attributes of the
constituent data variables.  Each of the values of these attributes is
itself an attribute table, containing attribute data about that data
type. 

\figureplace{A Nested Attribute Table}{hptb}%
{fig,nest-attr-table-diagram}{nest-attrtable.ps}%
{nest-attrtable.gif}{}

DODS attribute tables are modelled with the \class{AttrTable} class.
Each \class{AttrTable} object contains a doubly-linked list of
attribute triples, of a structure named \lit{entry}. Each entry object
contains a Name, Type and vector of values.  \class{AttrTable}
provides methods for reading, writing, and modifying the attribute
table.

The DODS definition of dataset attributes contains a ``Global''
attribute.  This has nothing to do with the data structure of the
\class{DAS} object, but with its use in DODS.  Global attributes apply
to all the variables in the dataset.  They can also be thought of as
being attributes of the dataset itself.

\pagebreak
\tbd{pagebreak command here.}
\subsection{The \class{DDS} Object}
\label{tk,dds}

\indc{DDS object}\indc{object!DDS}
The \class{DDS} is used to store information about the organization of
the data set and its variables. It contains information about the type
and shape of variables. While the \class{DDS} is similar to the
\class{DAS} in that it is used to store information about the data
set, it is used quite differently by both the client and
server components of DODS. The \class{DAS} is a stand-alone object and
is used solely for the purpose of storing attributes of variables and
the dataset. The \class{DDS}, however, stores type information about
a data set's variables by storing actual instances of those variables.

The DODS \dap\ variable objects have methods that can be used to read
values from a data set or transfer the variable's value over the
network.  This makes it convenient to use the \class{DDS} object
itself to hold data, and on the server side, the \class{DDS} object is
used by both the \class{DDS} filter program and the data filter
program.  (See \sectionref{tk,servers} for more information about the
structure of the DODS server.)
\indc{DDS object!diagram}

\figureplace{The \class{DDS} Object}{htb}
{fig,dds-diagram}{DDS.ps}{DDS.gif}{}

Figure~\ref{fig,dds-diagram} shows the structure of the \class{DDS}
class objects. The \class{DDS} consists of the following components:

\begin{itemize}
  
\item A \class{String} object used to store the name of the dataset to
  which this \class{DDS} refers.

\item A singly-linked list to store the variables in the data set.
  Each variable is stored in an instance of one of the
  \class{BaseType} class's descendants, and can be a simple or
  compound data type.

\item A ``printer'' method to create a textual representation of the
  \class{DDS}. Among other uses, this is used to transmit the
  \class{DDS} from server to client.
  
\item A parser and scanner to read a text \class{DDS} and convert it
  into its \Cpp form.  This is used to read \class{DDS} information
  from a disk, and also to receive it over the network.
  
\item A singly-linked list of parsed constraint expression clauses.
  The clauses wait here to be evaluated.
  
\item A constraint expression parser and scanner to extract constraint
  expression clauses from the input constraint expression.  This
  component is responsible for creating the list of constraint
  expression clauses.

\end{itemize}

The \class{DDS} object provides methods to access and operate each of
these components.  The two lists can be traversed with a g++
\class{Pix} object.

The \class{DDS} is `lexically scoped' so that two \class{Structure}
variables may have components with the same names; each component will
be referred to using the \class{Structure} name and the dot operator.
So for example, if a \class{DDS} called \lit{ralph} contains two
structures, \lit{vitals} and \lit{new\_hip}, and each structure
contains a variable called \lit{age}, you can differentiate the two by
referring to one as \lit{ralph.vitals.age} and the other as
\lit{ralph.new\_hip.age}.

\section{The Type Hierarchy}
\label{tk,type-hier}

\indc{DODS types}\indc{data types!DODS}
The \new{Type Hierarchy} is the set of classes that form the hierarchy
used to build objects that contain data. These classes comprise the
data model for DODS. They contain \new{simple} data types such as
integer and floating point values as well as \new{compound} types like
structure and sequence.  Each type is embodied by a \Cpp class, and
the classes are arranged in a class hierarchy, with a \class{BaseType}
defining properties inherited by all the type classes.

This section contains a brief description of the different types,
their relation to one another, and how they are used in an application
program.  For detailed descriptions of the characteristics of each
type, including inheritance diagrams, please see
\texorhtml{\chapterref{tk,classref}}{the \OPDapiref}.

The DODS types can be divided into four categories:

\begin{itemize}
\item \class{BaseType}

\item Simple Types

\begin{tabular}{lllllllll}
  \class{Byte} & \class{Int16} & \class{UInt16} & \class{Int32} & \class{UInt32} 
  & \class{Float32} & \class{Float64} & \class{Str} & \class{Url} \\
\end{tabular} 

\item Vector Types

  \begin{tabular}{ll}
  \class{List} & \class{Array}
  \end{tabular}

\item Compound Types

  \begin{tabular}{llll}
  \class{Structure}
  & \class{Sequence}
  & \class{Grid}
  \end{tabular}

\end{itemize}


Unlike the other classes in the DODS toolkit, the type classes are abstract
classes---in order to be used by a program, you must subclass the hierarchy
and create concrete classes to instantiate.

\subsection{Common Ancestor: BaseType}

\indc{BaseType class}\indc{class!BaseType}
The root of the type hierarchy is the abstract class \class{BaseType}.
This class, because it is abstract, is never instantiated itself.
\class{BaseType} is used as the base class for all of the different
types of variables, and contains common member functions used by all
the other type classes. For simple variables such as \class{Int32},
only the abstract virtual functions in \class{BaseType} need to be
added to complete the class definition. Compound types like
\class{Structure} usually require more. A compound type contains one
or more instances of \class{BaseType}, and requires methods to access,
add and remove these member variables.

\subsection{Simple Types}

\indc{data types!simple}\indc{DODS types!simple}
\indc{Byte class}\indc{class!Byte}
\indc{Int32 class}\indc{class!Int16}
\indc{UInt32 class}\indc{class!UInt16}
\indc{Float64 class}\indc{class!Float32}
\indc{Int32 class}\indc{class!Int32}
\indc{UInt32 class}\indc{class!UInt32}
\indc{Float64 class}\indc{class!Float64}
\indc{Str class}\indc{class!Str}\indc{URL class}\indc{class!URL}
The DODS simple data types consist of \class{Byte}, \class{Int16},
\class{UInt16}, \class{Int32}, \class{UInt32}, \class{Float32},
\class{Float64}, \class{Str} and \class{Url}. These data types match
very closely the corresponding types in C or Fortran.  Note
that---internally---each of these types uses either the C or \Cpp\ 
representation to hold the value of the object.

These abstract classes are direct descendents of \class{BaseType}
which contain only their definitions for \class{BaseType}'s abstract
member functions and a single constructor.

\begin{description}

\item [Byte] Variables which store bytes. Equivalent to unsigned
char on most UNIX workstations.

\item [Int16] Variables which store integer values as 16-bit
twos-complement  signed integers. Equivalent to \lit{int} on
most 16-bit UNIX workstations.

\item [UInt16] Unsigned integer.  A 16-bit unsigned integer value.

\item [Int32] Variables which store integer values as 32-bit
twos-complement  signed integers. Equivalent to \lit{int} on
most 32-bit UNIX workstations.

\item [UInt32] Unsigned integer.  A 32-bit unsigned integer value.

\item [Float32] Variables which store floating point data.
Defined as the  IEEE 32-bit floating point data type, equivalent
to a \lit{float} in ANSI  C.

\item [Float64] Variables which store floating point data.
Defined as the  IEEE 64-bit floating point data type, equivalent
to a \lit{double} in ANSI  C.

\item [Str] Variables which store string information. A DODS string is
\emph{not} a sequence of characters referenced using a pointer (as it is in
C), it is represented using a \Cpp\ object of the class \class{String}. 

\item [Url] Variables which store references to network
resources. This is a sub-class of \class{Str}.

\end{description}

\subsection{Vector Types: Array, List}

\indc{Vector class}\indc{class!Vector}
\indc{Array class}\indc{class!Array}
\indc{data types!vector}\indc{DODS types!vector}
The vector data types are \class{Array} and \class{List}. A
 \class{List} is a simple ordering of elements of a single type.  An
 \class{Array} arranges the elements so that they can be easily
 accessed with one or more indices.  

\begin{description}
  
\item[Array] Instances of \class{Array} have different semantics than
  arrays in C.  They are multi-dimensional data structures which
  contain \emph{N} instances of some data type. Each instance in the
  array can be referred to by an integer index between \emph{0 and
    N-1}. Multidimensional arrays are declared using C's syntax of a
  sequence of bracketed integer values: \lit{Int32 a[10][20]} declares
  an array of 10 arrays of 20 integers. However, unlike C arrays, the
  \class{Array} class supports named dimensions. In the preceding
  example, the array could have been declared: \lit{Int32 a[row =
    10][col = 20]} where \lit{row} and \lit{col} are the names of the
  first and second dimension, respectively. You can use the
  \lit{dimension\_name} and \lit{dimension\_size} member functions of
  the \class{Array} class to determine the name and size for the
  $i^{th}$ dimension.
  
  \class{Array}, like all of the compound types, contains a reference to a
  component variable. In the preceding example, the instance of \class{Array}
  would contain information about the dimensions of the array (10 by 20), but
  not the type of the elements (Int32). The element type information is
  stored in the component variable which the instance of \class{Array}
  references.
  
  When creating an array, the dimension sizes (and optionally their
  names) must be set.  Regardless of the shape of the array, it is
  always stored as a vector.  In order to access the element of a
  multidimensional array it is necessary to calculate the offset for a
  given element.
  
\indc{List class}\indc{class!List}
\item[List] A \class{List} is an ordered collection of elements of
  unknown length.  This is in contrast to the \class{Array}, whose
  size is always known in advance. When a List type is declared no
  size information is supplied, but in order to transmit a
  \class{List} object the length of that list must be known.  Thus,
  internally the number of elements in the current value \emph{is}
  stored.

\end{description}

\subsection{Compound Types: Structure, Sequence, Function, Grid}

\indc{data types!compound}\indc{DODS types!compound}
The compound data types are used to build new types as aggregates of
other types, including other compound types. (Note that \class{List}
and \class{Array} are compound types, as well, but contain only a
single type of data.) \class{Structure}, \class{Sequence} and
\class{Function} all contain a list of \class{BaseType} objects.
However, they have different semantics; a Structure is a simple
aggregate; nothing other than aggregation is implied, while Sequence
and Function define templates for relational objects. A Grid combines
several \class{Array} objects so that nonlinear values may be applied
to the indices of an array.

\begin{description}
  
\indc{Structure class}\indc{class!Structure}
\item[Structure] A Structure is an ordered collection of variables
  that conveys no relational information other than grouping. The
  variables that are members of a Structure may be of different types.
  In addition to the (possible) benefit of added organization,
  Structure may be used to supply information to the system that may
  be useful in optimizing the access or translation operations.

\indc{Sequence class}\indc{class!Sequence}  
\item[Sequence] A \class{Sequence} is similar to a \class{Structure}
  in that it consists of an ordered collection of variables which may
  be of different types.  However, where an instance of a
  \class{Structure} object describes a single set of data variables,
  an instance of a \class{Sequence} object describes a set of data
  variables, each of which is an entry in an ordered series of similar
  data variables.

  Consider a \class{Sequence} named \emph{S}, where each instance is
  called \emph{s}:

\begin{displaymath} 
\begin{array}{cccc}  
        s_{0 0} & s_{0 1} & \cdots & s_{0 n} \\  
        s_{1 0} & s_{1 1} & \cdots & s_{0 n} \\  
        \vdots  & \vdots  & \cdots & \vdots \\  
        s_{i 0} & s_{i 1} & \cdots & s_{i n} \\
        \vdots  & \vdots  &        & \vdots   
\end{array} 
\end{displaymath}

Every instance $s_{i}$ of \emph{S} has the same number, order, and
class of variables. A \class{Sequence} implies that each of the
\emph{n} variables is related to each other in some logical way.
Because a \class{Sequence} has several values for each of its
variables it has an implied \new{state}, or position in the sequence,
in addition to the instance data values.

\begin{table}[ht]
\caption{Table of relational data.}
\label{toolkit,seq2} 
\begin{center} 
\begin{tabular}{|c|c|c|} \hline 
\tblhd{Name} & \tblhd{Age} & \tblhd{Weight} \\ \hline 
James & 32 & 165 \\ 
Charlie & 7 & 65.4 \\ 
Bob & 10 & 80 \\ \hline
\end{tabular} 
\end{center} 
\end{table}

For example given the the information in Table~\ref{toolkit,seq2},
$s_{0}$ is \lit{James}, \lit{32}, \lit{165}, $s_{1}$ is \lit{Charlie},
\lit{7}, \lit{65.4}, \ldots. The data in the table might have the
following Sequence declaration:

\begin{vcode}{ib}
Sequence {
    Str name;
    Int32 age;
    Float64 weight;
} people;
\end{vcode}

%\indc{Function class}\indc{class!Function}
%\item[Function] A \class{Function} is a special case of a
%  \class{Sequence} that separates its variables into two categories:
%  \emph{independent} and \emph{dependent} corresponding to the
%  independent and dependent variables in an experiment. Variables in a
%  \class{Function} may be of differing classes. The mathematical
%  description of this functional relation is not specified.  Instead
%  the \class{Function} type is used to indicate that one of the two
%  sets constitute the independent variables and the other the
%  dependent variables.
  
\indc{Grid class}\indc{class!Grid}
\item[Grid] A \class{Grid} is an association of an \emph{N}
  dimensional \class{Array} with \emph{N} named vectors (\new{map
    vectors}), each of which has the same number of elements as the
  corresponding dimension of the \class{Array}. Each vector is used to
  map indices of one of the \class{Array}'s dimensions to a set of
  values which are normally non-integral (e.g., floating point
  values). Two map vectors may be members of different classes.


\begin{figure}[ht]
\W\label{toolkit,grid} 
\begin{center} 
\begin{ifhtml}
Grid =
\begin{tabular}{|c|c|c|c|c|c|} \hline 
32.0 & 31.5 & 31.1 & 30.8 & 29.2 \\ \hline
32.3 & 31.8 & 31.4 & 30.9 & 29.8 \\ \hline
32.9 & 32.3 & 31.8 & 29.7 & 30.2 \\ \hline
32.3 & 31.8 & 31.5 & 30.7 & 29.9 \\ \hline
\end{tabular} 
\\ \strut \\ M =
\begin{tabular}{|c|c|c|c|c|} \hline
1.2 & 1.4 & 1.8 & 3.9 & 4.5 \\ \hline
\end{tabular}
\\ \strut \\ N =
\begin{tabular}{|c|c|c|c|} \hline
67.8 & 68.7 & 92.3 & 95.2 \\ \hline
\end{tabular}
\end{ifhtml}
\begin{iftex}
\begin{tabular}{r|c|c|c|c|c|c|c|} \cline{2-6} 
M =    & 1.2 & 1.4 & 1.8 & 3.9 & 4.5 & \multicolumn{2}{c}{}\\ \cline{2-6}
\multicolumn{8}{r}{N=}\\ \cline{2-6}\cline{8-8}
Grid = & 32.0 & 31.5 & 31.1 & 30.8 & 29.2 & & 67.8 \\ \cline{2-6}\cline{8-8}
       & 32.3 & 31.8 & 31.4 & 30.9 & 29.8 & & 68.7 \\ \cline{2-6}\cline{8-8}
       & 32.9 & 32.3 & 31.8 & 29.7 & 30.2 & & 92.3 \\ \cline{2-6}\cline{8-8}
       & 32.3 & 31.8 & 31.5 & 30.7 & 29.9 & & 95.2 \\ \cline{2-6}\cline{8-8}
\end{tabular} 
\end{iftex}
\end{center} 
\caption{A sample Grid.}
\T\label{toolkit,grid} 
\end{figure}

In \figureref{toolkit,grid}, the grid element indicated by
\lit{Grid[2][3]} corresponds to \lit{N[2]} and \lit{M[3]}, or $N =
92.3$ and $M = 3.9$ respectively.  The element has a value of 29.7.

\end{description}





%%% Local Variables: 
%%% mode: latex
%%% TeX-master: "../pguide.tex"
%%% End: 




