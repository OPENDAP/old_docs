% Chapter 6 of the DODS Programmer's Guide
%  A Server functional specification
%
% $Id$
%
% $Log: ch06.tex,v $
% Revision 1.3  2004/02/18 06:38:48  jimg
% Various changes, mostly for the DODS --> OPD macros.
%
% Revision 1.2  2000/10/04 15:00:23  tom
% changed \figureplace definition, cleaning up...
%
% Revision 1.1  2000/03/21 22:04:33  tom
% added server spec appendixes, fixed xref tags
%
%

\chapter{Overview of the DODS Server Architecture}
\label{pguide,server}

This appendix describes in functional terms the operation of a server.

Note that descriptions of server outputs naturally and unavoidably
refer to certain kinds of inputs and vice versa.  This makes it
difficult to create definitions without forward and backward
cross-references.  Please be prepared to read this specification with
a \htmlonly{virtual} thumb between the pages as you flip back and
forth from the output to the input sections.

\section{Outputs}
\label{pguide,server,output}

A server sends to a client one of three different sorts of messages:

\begin{itemize}
\item HTML;
\item ASCII data; or
\item Binary data.
\end{itemize}

\subsection{HTML Data}

There are three different kinds of HTML data returned by a DODS server.
Clients that request these data should be prepared to display the HTML
to the user.

\begin{description}
\item[Server Information (\lit{usage})] 
  
  One kind of HTML data is returned in response to a request for
  server information (using the \lit{.info} URL suffix), and contains
  usage information for the server, including a formatted version of
  the dataset DAS and DDS.  This information is formatted by the
  \lit{usage} service program, invoked by the server.

\item[WWW Interface] 
  
  The forms-based DODS WWW interface returns HTML data to the client.
  This can either be a form a user can use to create a DODS contraint
  expression or a DODS directory listing, depending on whether the URL
  indicates a file with the \lit{.html} suffix, or a directory
  containing other files.

\item[Error messages] 

  A badly formed URL will result in a DODS error message, which is
  simply some HTML text describing the supported URL suffixes.  (See
  \lit{DODS\_Dispatch.pm}.)  Note that though an error message could
  in theory be returned to any client, whether or not they can display
  HTML, in practice, only web browser clients are prone to these kinds
  of errors.  Aside from web browser clients (e.g. Netscape), the DODS
  clients issue their server requests through the DODS client core
  libraries, which format the URL according to the DODS conventions.  

\end{description}

\note{There are two kinds of DODS error messages.  The particular one
  described above is issued in response to a badly formed URL.   Other
  kinds of errors are returned as ASCII data within a DODS data
  document.  See \sectionref{pguide,server,binary}.}


\subsection{ASCII Data (Text)}

These are the different kinds of ASCII data returned by a
fully-equipped DODS server.  (``Fully equipped'' implies that the
server has, on its execution path, all the supported DODS service
programs.)

\begin{description}
\item[DDS] 
  
  The DODS Data Descriptor Structure is a description of the data
  contained in the dataset.  It contains information about the data
  types and names represented in the dataset.  Programmers can think
  of the DDS as containing the type declarations for the data.  The
  DDS is fully described in \OPDuser .  The DDS is created with a
  service program called \lit{*\_dds}, where the \lit{*} is the same
  two-letter abbreviation used in the server name (\lit{nph-*}).

%Need xref to UG here.

\item[DAS] 
  
  The DODS Data Attribute Structure contains information \emph{about}
  the data in the dataset---the metadata.  It is a hierarchical list
  of name-type-value triples, where the names of the containers
  correspond to entries in the DDS.  The DAS is fully described in
  \OPDuser .  The DAS is created with a service program called
  \lit{*\_das}, where the \lit{*} is the same two-letter abbreviation
  used in the server name (\lit{nph-*}).

\item[ASCII Data] 

  If a DODS server is equipped with a service program called
  \lit{asciival}, it can convert binary data to ascii data on the fly,
  allowing you to use a standard web browser (such as Netscape) to
  examine data.  This feature can also be used to import data into a
  client that may not be able to process DODS data encoded in the
  standard binary format.

\end{description}

\subsection{Binary Data}
\label{pguide,server,binary}

The DODS data document consists of two parts, a DDS describing the
data returned, and the data itself.  It may also consist of ASCII data
describing an error condition.

\begin{description}
\item[DDS] 

  The DDS returned with the data differs slightly from the DDS
  returned by a simple request using the \lit{.dds} URL suffix.
  Whereas the DDS returned in response to that URL is a description of
  the dataset available, the DDS returned in a data document describes
  the data being returned.  If you ask for an entire dataset, there
  will be no difference between these two DDS's.  However, if you
  request only a \emph{part} of a dataset, the DDS included in the
  data document will only reflect the part of the dataset returned to
  you. 

  As an example, consider a dataset containing an array with one
  hundred elements.  The DDS for this dataset will contain an array
  with one hundred elements.  However, if you send a server a request
  for just the first fifty elements of the array, the DODS data
  document returned will contain a DDS with only fifty elements.  This
  is illustrated in \figureref{fig,DataDDS}. 

\item[Data] 

  The data requested from a server is (unsurprisingly) also contained
  in the DODS server's response to a data request.  The data is
  encoded using the XDR standard (eXternal Data Representation), and
  packed into the second part of the response document.  For more
  information on XDR, see \xlink{Internet RFC
  1014}{http://www.faqs.org/rfcs/rfc1014.html}.  For further details,
  see \sectionref{pguide,server-spec,data-doc}, below.

\item[Error Messages] 
  
  A DODS data document may contain an error message.  This is a DODS
  Error object (containing an ASCII error message and some other data)
  stuck into the HTTP document where the data would usually go.  This
  is where error conditions on the server are noted, as well as badly
  formed constraint expressions and file names.

\end{description}


\figureplace[The DODS Data Document and the DDS]{The DODS Data
  Document and the DDS.  For the dataset containing the vector
  \lit{temp}, with 100 elements, the top ``Simple DDS Request'' shows
  what the DDS might look like for that dataset.  The bottom ``DODS
  Data Document'' shows what might be returned by a request for all
  the even-numbered elements of the \lit{temp} array.  Note that the
  DDS has been altered to allow for the reduced number of elements in
  the returned data array.\htmlonly{\html{hr}}}
  {htb}{fig,DataDDS}{DataDDS.ps}{DataDDS.gif}{}

The HTTP response containing the DODS data document is formatted like
this: 

\begin{example}
HTTP/1.0 200 OK
XDODS-Server: <server/dap version string>
Content-type: application/octet-stream
Content-Description: dods-data
Content-Encoding: deflate
\var{<blank line>}
\var{<DDS>}
Data:
\var{<binary data, each variable given in the order it is listed in the DDS>}
\var{<EOF>}
\end{example}

Note the following:

\begin{itemize}
\item DODS servers use HTTP 1.0.
  
\item The version string (after the \lit{XDODS-Server:} header) must
  be <text>/<version number> where version number is x.y or x.z.y.
  This version number is parsed on the client side to ensure that a
  client is communicating with a compatible server.
  
\item The \lit{Content-Encoding} is used only when the document is
  compressed using `deflate'.  (This is the same as the compression
  used by the \lit{gzip} program, and is implemented in \lit{libz.a}. 
  
\item \lit{Content-Description} is used by the DODS client to figure
  out if the object is an error instead of a data object.  If it is,
  the \lit{Content-Description} field should read \lit{dods-error}.

\end{itemize}

\subsubsection{Encoding the DAP Data Types}
\label{pguide,server-spec,data-doc}

The DODS transmission protocol spearates variables into three classes:
scalars, arrays, and sequences.  A scalar is included in the DODS data
document simply by writing its XDR representation. An array is sent by
writing the number of elements (as an XDR \lit{int}) followed by the
elements themselves, also in the XDR format.  However, arrays of Byte,
Int16, Int32, UInt16, UInt32, Float32, and Float64 actually have their
size sent twice, once by the DAP software and once by the XDR
software. Here's a hex dump of some Int32 array data that shows this
behavior:

\begin{vcode}{xib}
00000000: 4461 7461 7365 7420 7b0a 2020 2020 496e  Dataset {.    In
00000010: 7433 3220 755b 7469 6d65 5f61 203d 2031  t32 u[time_a = 1
00000020: 5d5b 6c61 7420 3d20 315d 5b6c 6f6e 203d  ][lat = 1][lon =
00000030: 2032 315d 3b0a 7d20 666e 6f63 313b 0a44   21];.} fnoc1;.D
00000040: 6174 613a 0a00 0000 1500 0000 15ff fff9  ata:............
00000050: 40ff fff6 6fff fff3 e5ff fff1 ffff fff3  @...o...........
00000060: 4aff fff6 9aff fffb 1c00 0002 9600 0009  J...............
00000070: b300 000b 5e00 000b 0300 000b 8200 000a  ....^...........
00000080: b900 000a ae00 000b 7300 000a 2900 0008  ........s...)...
00000090: 5b00 0007 3500 0006 da00 0007 6900 0007  [...5.......i...
000000a0: 3e                                       >
\end{vcode}

The length bytes for the <data> section start at address 0x45 (The
length is 0x00000015.) and is repeated at 0x49.  Here's how you can
see this:

\begin{vcode}{xib}
geturl "http://dods.gso.uri.edu/cgi-bin/nph-nc/data/fnoc1.nc?u[0:0][0:0][0:20]"
\end{vcode}

The DDS for this dataset is (use \lit{geturl -d}):

\begin{vcode}{ib}
Dataset {
    Int32 u[time_a = 16][lat = 17][lon = 21];
    Int32 v[time_a = 16][lat = 17][lon = 21];
    Float64 lat[lat = 17];
    Float64 lon[lon = 21];
    Float64 time[time = 16];
} fnoc1;
\end{vcode}

Structures and Grids are sent by serializing their components, one by
one, in the order of their declaration in the DDS.

Sequences have a much more complex encoding scheme because one
sequence may contain another sequence \emph{and} because the sequence
type provides no information about the length of a given instance.  To
send sequences the DAP uses two different algorithms, one up to DODS
version 2.14 servers and clients and a better (more compact one) after
that.  DODS clients at version 2.15 and later can all read from both
servers \emph{but} they assume that a server that does not announce
its version is very old and will attempt to read Sequences using the
old algorithm.  (So if you're writing a new DODS server that serves
Sequences, you must remember to address the version requirement.)

Here's how the new algorithm encodes a sequence:

\begin{enumerate}

\item Serialize row of the sequence by: 

\begin{enumerate}
\item Writing the start of instance marker (0x5A), and 
\item Serializing (writing the XDR encoding of) each element in the
  row in the order of appearance, then
\end{enumerate}
\item Write the end of sequence marker (0x5A).
\end{enumerate}








%1) Serialize each row of the sequence
%  a. Write the start of instance marker (0x5A; // binary pattern 0101 1010)
%  b. Serialize each element in the row in the order of appearance.
%3) Write the end of sequence marker (0x5A; // binary pattern 0101 1010)

%Here's an example (note that the servers on dods.gso.uri.edu are from version
%2.14, so they send sequences the old way).

%    [dcz:/usr/local/DODS/bin] geturl -d "http://dcz/test/nph-jg/test"
%    Dataset {
%       Sequence {
%           String leg;
%           String year;
%           String month;
%           Sequence {
%               String station;
%               String lat;
%               String lon;
%               Sequence {
%                   String press;
%                   String temp;
%                   String sal;
%                   String o2;
%                   String sigth;
%               } Level_2;
%           } Level_1;
%       } Level_0;
%    } test;

%    [dcz:/usr/local/DODS/bin] geturl "http://dcz/test/nph-jg/test.asc?station,lat,lon,press&station=10"
%    station, lat, lon, press
%    "10", "37.44", "-71.96", "5.000"
%    "10", "37.44", "-71.96", "25.000"
%    "10", "37.44", "-71.96", "49.000"
%    "10", "37.44", "-71.96", "99.000"
%    "10", "37.44", "-71.96", "149.000"
%    "10", "37.44", "-71.96", "199.000"
%    "10", "37.44", "-71.96", "300.000"
%    "10", "37.44", "-71.96", "400.000"
%    "10", "37.44", "-71.96", "500.000"
%    "10", "37.44", "-71.96", "600.000"
%    "10", "37.44", "-71.96", "701.000"
%    "10", "37.44", "-71.96", "801.000"
%    "10", "37.44", "-71.96", "901.000"

%For clarity's sake I'm going to interject comments in the hex listing:

%    [dcz:/usr/local/DODS/bin] geturl "http://dcz/test/nph-jg/test.dods?station,lat,lon,press&station=10" > dump

%    00000000: 4461 7461 7365 7420 7b0a 2020 2020 5365  Dataset {.    Se
%    00000010: 7175 656e 6365 207b 0a20 2020 2020 2020  quence {.       
%    00000020: 2053 6571 7565 6e63 6520 7b0a 2020 2020   Sequence {.    
%    00000030: 2020 2020 2020 2020 5374 7269 6e67 2073          String s
%    00000040: 7461 7469 6f6e 3b0a 2020 2020 2020 2020  tation;.        
%    00000050: 2020 2020 5374 7269 6e67 206c 6174 3b0a      String lat;.
%    00000060: 2020 2020 2020 2020 2020 2020 5374 7269              Stri
%    00000070: 6e67 206c 6f6e 3b0a 2020 2020 2020 2020  ng lon;.        
%    00000080: 2020 2020 5365 7175 656e 6365 207b 0a20      Sequence {. 
%    00000090: 2020 2020 2020 2020 2020 2020 2020 2053                 S
%    000000a0: 7472 696e 6720 7072 6573 733b 0a20 2020  tring press;.   
%    000000b0: 2020 2020 2020 2020 207d 204c 6576 656c           } Level
%    000000c0: 5f32 3b0a 2020 2020 2020 2020 7d20 4c65  _2;.        } Le
%    000000d0: 7665 6c5f 313b 0a20 2020 207d 204c 6576  vel_1;.    } Lev
%    000000e0: 656c 5f30 3b0a 7d20 7465 7374 3b0a 4461  el_0;.} test;.Da
%    000000f0: 7461 3a0a 5a00 0000 0000 0002 3130 0000  ta:.Z.......10..

%The start of instance marker is at 0xf4. Also note that at 0xf3 the <cr>
%separates the `Data:' marker from the start of the XDR encoded information.

%    00000100: 0000 0005 3337 2e34 3400 0000 0000 0006  ....37.44.......
%    00000110: 2d37 312e 3936 0000 0000 0005 352e 3030  -71.96......5.00
%    00000120: 3000 0000 5a00 0000 0000 0002 3130 0000  0...Z.......10..

%The next start of instnace marker is at 0x124.

%Note that you might expect this dataset to look a little differently, that
%the 10, 37.44 and -71.96 would be sent once and only the pressure inforamtion
%would appear more than once (Since it is different for each row. That is, in
%fact, why the start of instance and end of seqeunce markers were introduced.
%However, our JGOFS servers `flatten' sequences before sending them.

%    00000130: 0000 0005 3337 2e34 3400 0000 0000 0006  ....37.44.......
%    00000140: 2d37 312e 3936 0000 0000 0006 3235 2e30  -71.96......25.0
%    00000150: 3030 0000 5a00 0000 0000 0002 3130 0000  00..Z.......10..
%    00000160: 0000 0005 3337 2e34 3400 0000 0000 0006  ....37.44.......
%    00000170: 2d37 312e 3936 0000 0000 0006 3439 2e30  -71.96......49.0
%    00000180: 3030 0000 5a00 0000 0000 0002 3130 0000  00..Z.......10..
%    00000190: 0000 0005 3337 2e34 3400 0000 0000 0006  ....37.44.......
%    000001a0: 2d37 312e 3936 0000 0000 0006 3939 2e30  -71.96......99.0
%    000001b0: 3030 0000 5a00 0000 0000 0002 3130 0000  00..Z.......10..
%    000001c0: 0000 0005 3337 2e34 3400 0000 0000 0006  ....37.44.......
%    000001d0: 2d37 312e 3936 0000 0000 0007 3134 392e  -71.96......149.
%    000001e0: 3030 3000 5a00 0000 0000 0002 3130 0000  000.Z.......10..
%    000001f0: 0000 0005 3337 2e34 3400 0000 0000 0006  ....37.44.......
%    00000200: 2d37 312e 3936 0000 0000 0007 3139 392e  -71.96......199.
%    00000210: 3030 3000 5a00 0000 0000 0002 3130 0000  000.Z.......10..
%    00000220: 0000 0005 3337 2e34 3400 0000 0000 0006  ....37.44.......
%    00000230: 2d37 312e 3936 0000 0000 0007 3330 302e  -71.96......300.
%    00000240: 3030 3000 5a00 0000 0000 0002 3130 0000  000.Z.......10..
%    00000250: 0000 0005 3337 2e34 3400 0000 0000 0006  ....37.44.......
%    00000260: 2d37 312e 3936 0000 0000 0007 3430 302e  -71.96......400.
%    00000270: 3030 3000 5a00 0000 0000 0002 3130 0000  000.Z.......10..
%    00000280: 0000 0005 3337 2e34 3400 0000 0000 0006  ....37.44.......
%    00000290: 2d37 312e 3936 0000 0000 0007 3530 302e  -71.96......500.
%    000002a0: 3030 3000 5a00 0000 0000 0002 3130 0000  000.Z.......10..
%    000002b0: 0000 0005 3337 2e34 3400 0000 0000 0006  ....37.44.......
%    000002c0: 2d37 312e 3936 0000 0000 0007 3630 302e  -71.96......600.
%    000002d0: 3030 3000 5a00 0000 0000 0002 3130 0000  000.Z.......10..
%    000002e0: 0000 0005 3337 2e34 3400 0000 0000 0006  ....37.44.......
%    000002f0: 2d37 312e 3936 0000 0000 0007 3730 312e  -71.96......701.
%    00000300: 3030 3000 5a00 0000 0000 0002 3130 0000  000.Z.......10..
%    00000310: 0000 0005 3337 2e34 3400 0000 0000 0006  ....37.44.......
%    00000320: 2d37 312e 3936 0000 0000 0007 3830 312e  -71.96......801.
%    00000330: 3030 3000 5a00 0000 0000 0002 3130 0000  000.Z.......10..
%    00000340: 0000 0005 3337 2e34 3400 0000 0000 0006  ....37.44.......
%    00000350: 2d37 312e 3936 0000 0000 0007 3930 312e  -71.96......901.
%    00000360: 3030 3000 81a5 0000 00                   000......









\section{Inputs}
\label{pguide,server,input}

The input to a DODS server is contained in an HTTP ``GET'' request.
Unlike a POST, the information in this kind of request is all in the
URL.  Consequently, examining the parts of a DODS URL will illustrate
all the different sorts of requests a DODS server can handle.

\Figureref{dods-api,fig,url-parts} contains a description of the parts
of a DODS URL, not including the ``Constraint Expression.''  The
constraint expression and the parts of the DODS URL are described in
detail in \OPDuser[constraint] and \OPDuser[url-parts.html].

\begin{figure}[h]
\texorhtml
{\small
${}\overbrace{\tt http}^{Protocol}://
\overbrace{\tt dods.gso.uri.edu}^{Machine Name}/
\overbrace{\tt cgi-bin/nph-nc}^{Server}/
\overbrace{\tt data}^{Directory}/
\overbrace{\tt fnoc1.nc}^{Filename}/
\overbrace{\tt .das}^{URL Suffix}$}
{\begin{vcode}{cb}
> http://dods.gso.uri.edu/cgi-bin/nph-nc/data/fnoc1.nc.das
   ^     ^                        ^      ^    ^        ^
   |     |                        |      |    |        |
Protocol |                        |      |    |        |
Machine Name                      |      |    |        |
Server-----------------------------      |    |        |
Directory---------------------------------    |        |
Filename---------------------------------------        |
URL Suffix----------------------------------------------
\end{vcode}}
\caption{Parts of a DODS URL (without a constraint expression)}
\label{dods-api,fig,url-parts}
\end{figure}

\subsection{Request types}

A DODS server is equipped to respond to several different request
types.  Each request type is signified by a different URL suffix.  The
server itself is just a dispatch script, that determines the type of a
request, and dispatches the request to the appropriate service
program, and relays its result back to the client.

In \figureref{pguide,fig,server-design}, a DODS client makes a GET request
to a DODS server (which is just an \lit{httpd} daemon equipped with a
bunch of CGI programs).  The daemon invokes the DODS Server, which is
a simple-minded dispatch script which in turn invokes the DAS, DDS,
Data, or other service program.

\figureplace{The Architecture of a DODS Data Server.}{htbp}
{pguide,fig,server-design}{arch.ps}{arch.gif}{}

The request types, their suffixes, and the service programs are listed
in \tableref{server-spec,suffix}.


\begin{table}[ht]
\caption{Table of URL suffixes.}
\label{server-spec,suffix} 
\begin{center} 
\begin{tabular}{|l|l|p{3in}|} \hline 
\tblhd{Suffix} & \tblhd{Service Program} & \tblhd{Description} \\ \hline 
\lit{.dds} & \lit{*\_dds} & Returns the Data Descriptor Structure for
                            the specified dataset. \\ \hline
\lit{.das} & \lit{*\_das} & Returns the Data Attribute Structure for
                            the specified dataset. \\ \hline
\lit{.dods} & \lit{*\_dods} & Returns binary data in the form of a
                              DODS data document.  See
                              \sectionref{pguide,server,binary}.
                              \\ \hline
\lit{.asc} & \lit{asciival} & Converts data requests to ASCII values
                              before sending them back to the client.
                              This service is useful for invoking from
                              simple web browsers. \\ \hline
\lit{.info} & \lit{usage} & Returns an HTML formatted version of the
                            dataset DDS and DAS, and any other server
                            and dataset information provided in *.ovr
                            files. \\ \hline
\lit{.html} & none & Returns a URL constraint expression builder form,
                     based on the dataset DDS and DAS.  This is the
                     DODS WWW interface. \\ \hline
\lit{.ver} & none & Returns the server version information. \\ \hline
\end{tabular} 
\end{center} 
\end{table}

\subsection{Constraint expressions}

A DODS server can accept a ``constraint expression'' contained in the
URL query string.  The DODS constraint expression describes how a DODS
server should subsample a DODS dataset before sending the data back to
the client.  The details of the constraint expresssion syntax are
covered in \OPDuser[constraint].  What's important here is simply
that the constraint expression is a logical expression with two
clauses: projection and selection.  

The ``projection'' clause of a constraint expression specifies the
data variables requested by the client, and the ``selection'' clause
specifies the condition under which the client wants them.  That is,
the projection clause might specify that the client wants to see
oceanic temperature data, and the selection clause would specify that
only records from below 1000 meters should be returned.

\subsection{Server functions}

Within the context of a constraint expression, a server can implement
functions a client would use to specify data.  Since the constraint
expression has two kinds of clauses, there are two kinds of server
functions: projection and selection.

To implement another server-side constraint function, see .  Following
is a list of the canonical server-side functions implemented in all
DODS servers.

\subsubsection{Selection}

yes


\subsubsection{Projection}


yes

%%% Local Variables: 
%%% mode: latex
%%% TeX-master: t
%%% End: 
