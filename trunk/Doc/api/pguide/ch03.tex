% Chapter to the DODS Programmer's Guide
%
% $Id$
%
% $Log: ch03.tex,v $
% Revision 1.4  1999/07/22 18:54:56  tom
% fixed errors
%
% Revision 1.3  1999/02/04 17:46:05  tom
% Modified for dods-book.cls and Hyperlatex
%
% Revision 1.2  1998/12/07 15:51:33  tom
% updated for DODSv2.19
%
% Revision 1.1  1998/03/13 20:50:02  tom
% created API manual from James's Toolkit document
%
%

\chapter{Using the DODS Library Classes}
\label{tk,subclassing}

In order to be used by real client-libraries and data servers, many of
the classes in the DODS toolkit must be subclassed. For example, the
Type Hierarchy classes, which represent the data types in the DODS
data model, are all abstract classes. In order to use them in a
program they must be subclassed. The \class{Connect} class also must
be subclassed if it to hold additional information about the
connection. 

\section{Sub-classing the Type Hierarchy}

\indc{DODS types!using}\indc{sub-classing!type classes}
In order to link a program with the DODS Toolkit library, the DAP's
abstract classes must be subclassed and those subclasses must ensure
that all of the member functions of those classes have valid
definitions. This is necessary because of \Cpp's rules governing
abstract classes. The next three sections cover sub-classing the
simple, vector and compound classes, respectively. In addition, a
sample set of classes (called the \lit{Test} classes because they are
used by the DAP tests) is included with the DAP distribution. You can
read the source code for those classes to find out how they were
created.

Each of the sub-classed types must supply:

\begin{itemize}
\item A constructor that takes a \class{String} argument and returns
  an object with that name.

\item A virtual destructor.

\indc{ptr\_duplicate}\indc{copy function}
\item A copy function called \lit{ptr\_duplicate}.
  
\indc{read function}
\item The \lit{read} function, to read data from a disk and load it
  into the type class object.

\end{itemize}

The \lit{ptr\_duplicate} member function returns a pointer to a new
instance of the type of object from which it was invoked. This member
function exists so that objects that are referenced through pointers
to \class{BaseType} can correctly copy themselves. (If you were to use
the operator \lit{new} to copy an object referenced through
\class{BaseType}, you would get a \class{BaseType}, not a new instance
of the type of the referenced object.) Note that \lit{ptr\_duplicate}
is a virtual function so an object which is a descendent of
\class{BaseType} will get the most specific definition of that
function.

The \lit{read} function is much more complicated to write than
\lit{ptr\_duplicate}, and the difficulty varies depending on the data
type to be read. However, this function is only used on the server
side of the system and not by the client.  That is, it can be
implemented with a null function body if all you are building is a
client library. The \lit{read} function takes two arguments, the
dataset name, and an error flag.  It must read from that dataset the
values specified by the current constraint expression. The error flag
is passed by reference so that read can set its value and callers can
test it.  The class \class{BaseType} contains a number of member
functions to facilitate writing a read function.  The return value of
\lit{read} is TRUE is there is more data to be read (by additional
calls to \lit{read}) or FALSE otherwise.

\subsection{Sub-classing the Simple Types}

\indc{sub-classing!simple types}\indc{simple types!sub-classing}
Creating a \lit{read} function for the simple type classes is a fairly
straightforward operation.  Simply read the function using the data
access API's standard protocol, and use the type class's \lit{val2buf}
function to load the data into the type object.

The following function is taken from the DODS implementation of the
\netcdf\ data access library.  (See the file
\lit{src/nc-dods/NCByte.cc}.)
\indc{sub-classing!example}

\begin{vcode}{ins}
NCByte::read(const String &dataset, int &error) 
{
   int varid;                  /* variable Id */
   nc_type datatype;           /* variable data type */
   long cor[MAX_NC_DIMS];      /* corner coordinates */
   int num_dim;                /* number of dim. in variable */
   long nels = -1;             /* number of elements in buffer */
   int id;

   if (read_p()) // already done
     return true;

   int ncid = lncopen(dataset, NC_NOWRITE); /* netCDF id */

   if (ncid == -1) { 
     cerr << "ncopen failed on " << dataset<< endl;
     return false;
   }
 
   varid = lncvarid( ncid, name());
   (void)lncvarinq( ncid, varid, (char *)0, &datatype, 
                    &num_dim, (int *)0, (int *)0);

   if(nels == -1){  
     for (id = 0; id < num_dim; id++) 
       cor[id] = 0;
   }

   if (datatype == NC_BYTE){
     dods_byte Dbyte;

     (void) lncvarget1 (ncid, varid, cor, &Dbyte);
     set_read_p(true);
          
     val2buf( &Dbyte );

     (void) lncclose(ncid);  
     return true;
   }
   return false;
}
\end{vcode}

The following points are worth consideration about the above example.

\begin{description}
\item[line 10] Check to see if this variable has already been read.
\item[line 13] The \lit{lncopen()} function call is simply the
  \lit{ncopen()} function from the \netcdf\ library. Also, the
  \lit{lncvarid} function is renamed \lit{ncvarid} and so on.  The
  names have been changed to avoid link-time problems.  (Remember that
  the whole point of this exercise is to create a new \lit{ncopen()}
  function and its friends.
\item[line 33] This sets the flag tested with the \lit{read\_p}
  function.
\item[line 35] This command transfers values from the dataset's
  variable to the DODS instance.  This is the point where the read
  actually happens.
\end{description}

Note the use of \lit{dods\_byte} in the code example.  The DODS
configuration process creates definitions for the simple data types
like this one. They are stored in \lit{config\_dap.h}.

The \lit{read} member function is used by the constraint expression
evaluator to extract data from a dataset during evaluation of the
constraint expression.  This is particularly important to remember
because the \lit{read} member function for a simple data type will be
called when reading an aggregate type such as \class{Structure}.

\subsection{Sub-classing the Vector Types}

\indc{sub-classing!vector types}\indc{vector types!sub-classing}
The vector data types require the same abstract member functions be
defined as the simple types. The definition for \lit{ptr\_duplicate}
is also the same for vector as for simple types.  However, the
\lit{read} member function for the vector types (classes \class{Array}
and \class{List}) is more complicated than for the simple types
because vectors of values are represented in two ways in DODS,
depending on the type of variable in the vector. Arrays are stored as
C would store them for the simple types such as \class{Byte},
\class{Int32} and \class{Float64}.  However, compound types are stored
as arrays of the DODS \Cpp\ objects (with the exception of arrays
themselves, but more on that later).

When reading an array of Byte values, the \lit{val2buf} member function
should be passed a pointer to values stored in a contiguous piece of
memory. For example, when \lit{read} is called to read a variable
\lit{byte-array}, it must determine how much memory to allocate to
hold that much information, use \lit{new} to allocate an adequate
amount of memory, use the dataset's API calls to read
\lit{byte-array} into the newly allocated memory and then pass that
memory to \lit{val2buf}. This same procedure can be followed for all
the simple types.
\indc{val2buf}

However, when reading an array of \class{Structure}, for example, the
values must be stored in the DODS \class{Array} object one at a time
using the \class{Array} member function \lit{set\_vec}. An
\class{Array} object containing a 3x4 array of \class{Structure}
objects will actually point to 12 different instances of that class.
An \class{Array} object containing \class{Byte} objects contains
only one instance of the \class{Byte} class, as a template for the
array elements.

Arrays in DODS are unlike arrays in C in that an array object may have
more than one dimension. In terms of the way a value is stored,
however, an Array is a single dimensional object. When an Array is
declared as having two or more dimensions, those are mapped onto a
single vector. To access the element $A_ij$ of array \emph{A}, you
must know the size of the first dimension, \emph{I}, and use the
expression $i * I + j$ to compute the offset into the vector.

Since the class \class{List} is a single dimension array without
\emph{declared} size (the size of each value of the List object is
stored in the object) the rules for \class{Array}'s read member
function apply.

\subsection{Sub-classing the Compound Types}

\indc{sub-classing!compound types}\indc{compound types!sub-classing}
The \lit{read} member function of the compound data types simply
iterates over the contained variables calling their \lit{read} member
functions. In the future, this definition will move into the supplied
classes (That is, \lit{read} will no longer be a abstract member
function for the compound types.).

The two constructor types \class{Sequence} and \class{Function} are
different from all the other types in the DAP in that they have
\emph{state}. That is, the value of a sequence depends on how many
values have been read previously.  This is very different from an
array where the $i^{th}$ element has the same value regardless of what
has happened before. When you write implementations for \lit{read} in
the \class{Sequence} and \class{Function} classes, you must be sure to
write those member functions so that they can be called repeatedly and
that each call to read returns the next value of the corresponding
data Sequence or Function.

This is true because the constraint expression evaluator must be able to
apply certain constraints to values of individual sequence elements and is
actually implemented in the DDS class by first calling the read member
function, evaluating the constraint expression based on the values and, if
they constraint expression is satisfied, calling the serialize member
function.  See the member function \lit{DDS::send} (See
section~\ref{tk,ce-evaluation} for more information on evaluation of
constraint expressions).

If, for some reason, it is not possible to write \lit{read} so that it
gets called once for each sequence value\footnote{A single entry in a
  sequence, modelled as a row in a relational table, is sometimes
  called an instance of the sequence.  This is useful terminology, but
  is occasionally confusing when we are also talking about instances
  of objects.}, then you must re-implement \lit{DDS::send} so that its
functions are performed. For example, you could implement
\lit{Sequence::read} so that the entire sequence is read in and
overload \lit{Sequence::serialize} and \lit{Sequence::deserialize} so
that the next set of values are sent/received. You would then build a
send that called the \lit{Sequence::read} member function once and
extracted each successive value, evaluated the constraint expression
using on that value and used the result of that evaluation to
determine whether to send the value or not.

\section{Sub-classing the \class{Connect} Class}
\label{tk,subclass-netio}

\indc{sub-classing!Connect}\indc{Connect class!sub-classing}
The DODS API defines a connection-less protocol in which a server
keeps no information for a client in between data requests.  This is
in contrast to most data access APIs, which maintain state information
about the files or datasets that a user program currently has open. To
simulate an API's connection, the client library for that API must
create a \emph{virtual connection} using information about the data
set it has read from the server.  That is, the client library must
maintain the \emph{illusion} of a connection (state) for each open
data object (typically a file) even though no such connection actually
exists.

DODS provides the \class{Connect} class to make this a simple process.
This class contains a variety of information about a dataset and its
location on the internet, including the dataset's URL.  Most of
the information necessary to fake a connection is contained here.

Data access APIs differ widely however, and it is usually necessary to
add some information to the \class{Connect} class to make a workable
virtual connection. This information can be stored in the subclassed
\class{Connect}.  The API's ``open'' call must be completely recoded
so that the \class{DAS} and \class{DDS} objects, as well as any other
information, are requested from the dataset server and stored locally
in whatever form is most convenient.

\indc{Connect class!sub-classing!example}
\Figureref{fig,NCConnect} shows the sub-class of \class{Connect} used
to store information extracted from the DAS and DDS objects (for the
\netcdf\ library), and used
to simulate the storage of information in the original
API.\footnote{The \netcdf\ client library has been rewritten a couple
  of times since these examples were taken from it.  Most of the
  changes involve additional functionality not necessary for the
  illustrations here.  We have also removed some error-checking to
  make the intention clearer.  Therefore the actual code in the
  \netcdf software may not match these examples. However, these
  examples will work.}

\begin{figure}
\begin{vcode}{cb}
class NCConnect: public Connect {
private:
    int _ncid;        
    int _nvars;       
    int _ndims;       
    String _dim_name[MAX_NC_DIMS];
    int _dim_size[MAX_NC_DIMS];
    int _das_loc[MAX_NC_VARS];
    void init_list(int i);
    void parse_array_dims();
    void parse_grid_dims();

public:
    NCConnect(const String &name, const int mode);
    ~NCConnect();

    int &ncid();
    int ndims();
    int nvars();
    int dim_size(const int dimid);
    const String &dim_name(const int dimid);
    int das_loc(const int varid);
    void parse_das_loc();
    void parse_dims();
};
\end{vcode}
\caption{The subclass of Connect used with the NetCDF client library.}
\label{fig,NCConnect}
\end{figure}

The private data added to the \class{Connect} object with this class
is as follows:

\begin{description}

\item[\lit{\_ncid}]  This is only used for access to local files.  It
  is the \netcdf file descriptor, or file ID.

\item[\lit{\_nvars}] The number of variables in data file.

\item[\lit{\_ndims}] The number of dimensions found.  Note that not
  all the variables in a \netcdf file have the same number or set of
  dimensions. This number is the list of all the different ones.

\item[\lit{\_dim\_name[MAX\_NC\_DIMS]}] The names of the dimensions
  found in the data file.

\item[\lit{\_dim\_size[MAX\_NC\_DIMS]}] An array containing the size
  of each dimension found.

\item[\lit{\_das\_loc[MAX\_NC\_VARS]}] An index into a table of
  attributes. 

\end{description}

This additional data allows the client library to emulate most of the
\netcdf\ query functions properly and \emph{locally}.

Now look at the recoded open call of the \netcdf library in
\figureref{fig,open}. The new function has exactly the same type as
the original implementation; it takes the same number and type of
arguments and returns the same type. The first operation performed by
the new open call is to create a \class{Connect} object (actually an
\class{NCConnect} object) using the arguments passed to the open call.

\indc{open!using Connect class}
After the open call creates the new \class{NCConnect} object, it makes
sure that the user is not trying to open the remote dataset for
writing and, if they are not, reads the \class{DAS} and \class{DDS}
from the dataset. Once read, the \class{DAS} and \class{DDS} objects
are parsed using \class{NCConnect}'s member functions
\lit{parse\_das\_loc} and \lit{parse\_dims}. These two member
functions, along with the additional state variables in
\class{NCConnect}, effectively create the virtual connection.
Subsequent calls to the client library for information about the
variables (e.g., their size, names, etc.) will be answered using
information stored in the symbol table in the \class{NCConnect}
object.

\begin{figure}[htb]
\begin{vcode}{cb}
int
ncopen(const char *path, int cmode)
{
    int id;
    NCConnect *c = new NCConnect(path);

    if(cmode != NC_NOWRITE) {    
        delete c;  
        return(-1);
    }

    c->request_das();
    c->request_dds();

    c->parse_das_loc();
    c->parse_dims();

    return(conns.add_connect(c)) ;
}
\end{vcode}
\caption{The recoded open call of NetCDF.}
\label{fig,open}
\end{figure}

The code above checks to see whether the user is trying to open a
connection for writing.  Since DODS is a read-only protocol, this
attempt must fail.

%%% Local Variables: 
%%% mode: latex
%%% TeX-master: "../pguide.tex"
%%% End: 
