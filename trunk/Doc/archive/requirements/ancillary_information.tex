
% A SRS for the Ancillary Information/Metadata Services.
% 10/15/2001 jhrg
%
% $Id$

\documentclass{article}
\usepackage{psfig}
\usepackage{changebar}
\usepackage{acronym}
\usepackage{gloss}
\usepackage{vcode}

% Note: to get the glossary to work, run bibtex on the *.gls.aux file,
% then latex, then bibtex *.gls, then latex. Also, make sure to set
% your BST and BIBINPUTS environment variables so that the BST and BIB files
% will be found.
\makegloss

% Change paragraph typesetting; eliminate indenting and add more space between
% paragraphs.
\setlength{\parindent}{0em}     % Amount of indentation
\addtolength{\parskip}{1ex}     % Vertical separation
\setcounter{secnumdepth}{4}     % Put \paragraph in TOC
\setcounter{tocdepth}{4}


\begin{document}

\title{DODS Ancillary Information Service SRS, Draft}
\author{James Gallagher\thanks{The University of Rhode Island,
    jgallagher@gso.uri.edu}}
\date{$Revision$ \\ Printed \today}

\maketitle
\tableofcontents

\section{Introduction}
This is the \acl{SRS} for the \acl{DODS} \acl{AIS}. 

This document conforms to the IEEE~830-1998 \ac{SRS} recommended practice.
Since the recommended practice covers a wide range of possible projects, some
of the information in it is not appropriate for this part of \acs{DODS}.
Where that is the case, that section has been marked N/A.

\textbf{Bold face} type is used to indicate a word or phrase which can be
found in the glossary.

\emph{Emphasized} text contained in square brackets ([]) is used to indicate
an editorial comment, especially about the information that should be
provided in a part of the \ac{SRS}.

\subsection{Purpose}
This \ac{SRS} describes the requirements for the \ac{AIS} and changes in the
requirements for other pieces of already existing software.

The intended audience is the entire developer community of the \acl{DODS}.

\subsection{Scope}
\label{sec:scope}

The software to be developed is the \ac{AIS} for \acs{DODS}. Some existing
software will need modification.

The overriding goal of development of the \ac{AIS} is to achieve level 3
interoperability at the data level in the DODS environment. That is, the
\ac{AIS} will provide semantically meaningful machine-to-machine
interoperability for the use of data (as opposed to the discovery of
datasets) with distributed, heterogeneous data sets accessed via the DODS DAP

To achieve the goal, the \ac{AIS} will provide ways for users to add and
change the Attributes of a dataset. It will also make it possible for users
to add new Variables to a dataset. Finally, the \ac{AIS} will be designed so
that users may add this information dirrectly in a URL which references a
dataset or by creating a new virtual dataset using an \emph{\ac{AIS} server}.

The motivation for the \ac{AIS} stems from two observations made
regarding \ac{DODS}:
\begin{enumerate}
\item Complete and consistent syntactic and semantic use 
   metadata are required to achieve Level 3 interoperability.

\item We will not be able to achieve complete and consistent
   semantic use metadata by adding information at only data server locations.
\end{enumerate}

\subsection{Overview}

The remainder of this document is organized as follows:
\begin{enumerate}
\item Section~\ref{sec:overall} provides background for the specific
  requirements and relates those requirements to the rest of DODS.
\item Section~\ref{sec:specific} lists the specific requirements for the
  \ac{AIS}.
\item Following Section~\ref{sec:specific} are a list of acronyms and
  abbreviations, a change log, a glossary and references.
\end{enumerate}

% Section 2 ---------------------------------------------------------------

\section{Overall Description}
\label{sec:overall}

\emph{[This section of the SRS should describe the general factors that
  affect the product and its requirements. This section should not state
  specific requirements. Instead, it provides a background for those
  requirements, which should be defined in Section~\ref{sec:specific}.]}

\cbstart
\subsection{Product functions}
The \ac{AIS}:
\begin{enumerate}
\item MUST provide a way to change for consistency or to add
   for completeness Attributes (translational use metadata)
   of or to existing datasets.

\item MUST provide a way to add for completeness (independent)
   Variables to existing datasets. 

\item MUST support making these changes on the computer serving
   the data, on the client computer or on any other computer 
   on the network.

\begin{enumerate}
\item SHOULD support users making those changes without running 
   a network server.
\end{enumerate}

\item Access to an augmented dataset MUST not be logically different 
   from a plain dataset. Clients should be able to use URLs which 
   reference both types of datasets interchangeably.

\item The user MUST be allowed to make use of any combination of 
   AIS DDSs and/or DASs to modify the DDS and/or DAS of a 
   requested data set.

\item There MUST be a way to associate public sites providing 
   additions/changes to a given dataset. 

\item Legacy clients MUST work with AIS URLs, with the possible exception of
  those URLs which refer to local AIS resources.

\item URLs which contain AIS reosurces SHOULD NOT break legacy servers if at
  all possible.

\end{enumerate}
\cbend

\subsection{Product perspective}

\cbstart
The \acl{AIS} will meet the goals of Section~\ref{sec:scope} by providing a
way to augment the content of the \ac{DAS}, \ac{DDS} and \ac{DataDDS} objects
returned by \ac{DAP} servers. Together these objects model a data source as a
collection of variables, each of which has a name, a datatype, one or more
values and a collection of attributes.\footnote{See the DAP
  Specification~\cite{gallagher:dap-spec} for a more detailed description of
  these objects and the other possible requests that can be made to a DAP
  data source.} The \ac{AIS} provides a way for users to add information to
these responses to tailor their content to a particular use.
\cbend

The following sections describe how the software operates within
the constraints imposed by the existing \ac{DODS}/\ac{DAP} software.

\subsubsection{System interfaces}
\paragraph{Textual representation of the \ac{DAS} and \ac{DDS} objects}
The text representation of the \ac{DAS} and \ac{DDS} objects are described in
(Section~\ref{sec:ui:text-das-dds}), under the category of \emph{User
  Interfaces}. However, since these will be used by the software which builds
these objects, they are also system interfaces.

\paragraph{The \ac{DODS} \ac{URL}}
The \ac{DODS} \ac{URL} (see Section~\ref{sec:ui:url}), like the textual
\ac{DAS} and \ac{DDS} objects, plays a dual role as both system and user
interface. \cbstart This is true because the \ac{DODS} software itself uses
\ac{URL}s and because users at varioues levels ``write URLs.''\footnote{I'm
  using \emph{users} in a broad sense; it includes people who set up \ac{AIS}
    servers as well as people who cut and paste URLs using a web browser.}
  \cbend

\paragraph{Augmenting the \ac{DDS} object}
\label{sec:augment-dds}
The \ac{DDS} grammar will need modification in order to add virtual
variables. These variables will get their values either from other variables
or attributes already in the dataset or from external files. The grammar for
the \ac{DDS} object's external representation will need modification. See
Section~\ref{sec:DDS}.

\subsubsection{User interfaces}
\label{sec:ui}

\paragraph{The \ac{DODS} \ac{URL}}
\label{sec:ui:url}
Clients of \ac{DODS} must be able to specify, in the \ac{URL} that references
a data source, which \ac{AIS} resource(s) should be used with that data
source . This is the most direct way to support all possible interfaces to
DODS. This is listed under \emph{User Interfaces} because it is possible that
users will type these into a web browser, script file, \emph{et~cetera}.

\cbstart
\paragraph{The AIS Server}
\label{sec:ais-server}
The \emph{AIS Server} is a DODS server that provides access to a virtual
dataset which includes \ac{AIS} resources. This server provides a way for
people to use the \ac{AIS} without having to explicitly embed \ac{AIS}
resource references in  URLs. The \ac{AIS} Server URL is a regular DODS URL;
the AIS Server hides the additional complexity of explicit AIS resource
references. \cbend

\paragraph{The text representation of the \ac{DAS} and \ac{DDS}}
\label{sec:ui:text-das-dds}
Information will be provided to the \ac{AIS} using text \cbstart documents.
These documents may be files or they may be the result of dereferncing a URL.
\cbend The \ac{DAP} uses a textual representation for the external form of
the \ac{DAS} and \ac{DDS} objects. This representation will be used to supply
the extra information about specific instances of these objects that
constitute the information accessed through the \ac{AIS}.

\paragraph{Graphical User Interfaces}
Various graphical user interfaces which have been developed for \ac{DODS} or
have been modified to take advantage of \ac{DODS} \cbstart may need
modification to take full advantage of the \ac{AIS}. This will depend on the
particular GUI in question, describing how each should be enhanced is beyond
the scope of this \ac{SRS}.\cbend

\subsubsection{Software interfaces}
\paragraph{DAP~2.0} 
The \ac{DAP} protocol defines the \ac{DAS}, \ac{DDS} and \ac{DataDDS} objects.
\paragraph{HTTP~1.0} 
\ac{DODS} clients and servers use \ac{HTTP} for all their network I/O.

\subsubsection{Operations}
\paragraph{User defined \ac{AIS} files} 
Users may write their own \ac{AIS} resources. This will require resources from
\emph{User Support}.

\paragraph{Collections of \ac{AIS} resources should be listed}
Data users (people principally interested in accessing data using the
\ac{DAP}) will need to be able to locate \ac{AIS} resources or \ac{AIS}
servers. It seems likely that various groups or individuals will create
\ac{AIS} resources for the groups of data sources in which they are
interested. \acl{AIS} resource collections should be listed, if their
authors agree to it, so that other users may find them.

\subsubsection{Site adaptation requirements}
Users will need to upgrade their clients \cbstart to use local \ac{AIS}
resources. \cbend

Providers may need to upgrade their servers.

\subsection{User characteristics}
There is a range of possible users of the \ac{AIS}. One group of user will be
involved in writing the \ac{AIS} resources. They will be looking at data
sources, seeing what information they contain and thinking about the types of
additional information that would make the data source more broadly useful.
In many cases these users will be people using the \ac{DAP} to get data. They
will have a good idea of the type of information that needs to be added to
the data source to make its contents more useful to them. 

Some \ac{AIS} users will be involved in adding functions to servers so that
derived variables can be accessed. These users are essentially an extension
of the programmers who will implement this SRS and will be the most technical
group of users.

Another type of user will be solely interested in accessing data and will
never create their own \ac{AIS} resources. Some of these users will be aware
of the \ac{AIS} because they are writing URLs that appear in scripts or in
configuration files for analysis programs. Others may access data through web
systems that hide the use of the \ac{AIS}.\footnote{Note that even though
  some users might not know they are using the \ac{AIS}, its use can be
  traced by going back to client programs and looking at the URLs, which will
  contain the references to any \ac{AIS} resources used \cbstart or by
  getting information from an \ac{AIS} Server about the \ac{AIS} resources it
  uses\cbend.}

Most users of \ac{DODS} will not want to write \ac{AIS} resources. Writing
this type of information so that various standards and conventions are
satisfied is tedious.

\subsection{Constraints}
\emph{[This section lists any other items that may limit the developer's
  options.]}

Performance should be comparable to plain access. Response time 
increase should be minimal. 

Changes to the current URL syntax should be limited. \cbstart Access to a
\ac{DAS} object currently does not use a \ac{CE}, however, the \ac{AIS} may
add this. The `function call' syntax already present in the DODS URL's syntax
should be used; another syntax should not be added. It's OK if the \ac{AIS}
adds a special function which both clients and servers can look for.\cbend

The AIS should use as much of the WWW/HTTP infrastructure as possible,
building its own infrastructure only when necessary. This will lead to a
system that is first and foremost `document based,' which will make
deployment and testing simpler.

\subsection{Assumptions and dependencies}
\label{sec:assumptions}
The system designers can assume that the following are true:
\begin{enumerate}
  
\item People providing \ac{AIS} resources for use by others expect to install
  some sort of network server software (e.g., httpd) \cbstart or to upgrade
  existing servers.\cbend

\item The \ac{DAP} specification (version 2.0) will be changed to allow
  \ac{AIS} resources to be referenced in a \ac{DODS} \ac{URL}.

\cbstart
\item A new server will be written to handle the \ac{AIS} Server operations.
\cbend
\end{enumerate}

% Section 3 ---------------------------------------------------------------

\section{Specific Requirements}
\label{sec:specific}
\emph{[This section of the \ac{SRS} lists all of the software requirements to
  a level of detail sufficient to enable designers to design a system to
  satisfy those requirements, and testers to test that the systems satisfies
  those requirements.]}

This section is organized using template A.5 of IEEE STD 830-1998,
organization by feature.

\subsection{External interfaces}
\subsubsection{User interfaces}
Each of the following user interfaces have been discussed
previously in Section~\ref{sec:ui}. Note that these are also system
interfaces. 
\begin{enumerate}
\item The \ac{DODS} \ac{URL}~\cite{gallagher:dap-spec}.
\item The text representation of the \ac{DAS} and \ac{DDS}
  objects~\cite{gallagher:dap-spec}.
\item The \ac{AIS} resource collection configuration file for remote
  resources. 
\end{enumerate}

\subsubsection{Software interfaces}
\label{sec:si}
Software to process \ac{DODS} \ac{URL}s is located on both the client- and
server-side of the system in object-oriented class libraries. Two classes,
one each for clients and servers, will be principally involved in processing
\ac{AIS} references embedded in a \ac{URL}.
\begin{enumerate}
\item \texttt{Connect}. The \texttt{Connect} class handles the task of
  dereferencing a \ac{DODS} \ac{URL} and processing the response,
  creating other classes and invoking their methods as needed. The
  implementation of this class will require modification for local \ac{AIS}
  resource access.
\item \texttt{DODSFilter}. The \texttt{DODSFilter} class processes the
  request for a given object. The class uses information passed from the
  \texttt{DODS\_Dispatch.pm} Perl module to build an object and generate the
  response using that object. This object will require modification for
  server-side \ac{AIS} resource access.
\end{enumerate}

In addition, the \ac{DAS} and \ac{DDS} classes contain an interface that
provides a way for information contained in a text file to be merged with an
existing instance.  Each class supports a \texttt{parse(\ldots)} method which
can be invoked with either a pointer to an open file or the name of a file.
The file's contents will be merged with the object if the contents parse as a
valid object of the corresponding type. That is, if \texttt{dds} is an
instance of \ac{DDS}, calling \texttt{dds.parse("myfile")} will merge into
\texttt{dds} the information in \texttt{myfile} if its contents are a valid
\ac{DDS} object description using the text notation described in the \ac{DAP}
specification, version 2.0~\cite{gallagher:dap-spec}.

\ac{DODS} servers interface with \ac{HTTP} daemons using
\ac{CGI}~1.1~\cite{w3c:cgi}. The specification describes how different parts
of a \ac{URL} are broken up and passed to \ac{CGI} programs.

\subsubsection{Communications interfaces}
\ac{DODS} client programs use HTTP~1.0 to communicate with server programs.

\subsection{System features}

\subsubsection{Specifying \ac{AIS} resources in the \ac{URL}}
% *** What about values in the URL?
\label{sec:url}
To request that the object (\ac{DAS}, \ac{DDS} or \ac{DataDDS}) returned from
a data source be augmented using information from a particular \ac{AIS}
resource, that resource must be \cbstart identified \cbend in the \ac{DODS}
\ac{URL} requesting the object.\footnote{This scheme, of passing the \ac{AIS}
  resource using the \ac{URL}, was suggested by Steve Hankin.} To do this the
syntax of a \ac{DODS} \ac{URL} will be modified so that the \texttt{query}
component can contain both a \ac{CE} and a list of \ac{AIS} resources. In the
\ac{DAP}~2.0 specification, the \texttt{query} part of the \ac{DODS} \ac{URL}
contained only the \ac{CE}.  To accommodate one or more \ac{AIS} references
it becomes:
\cbstart
\begin{ttfamily}
\begin{center}
\begin{tabular}{lll}
query & = & CE | AIS | CE "\&" AIS\\
AIS & = & "ais(" AIS\_REF *("," AIS\_REF) ")"\\
\end{tabular}
\end{center}
\end{ttfamily}

The complete grammar is:
\medskip

\begin{minipage}{5in}
\begin{ttfamily}
\begin{center}
\begin{tabular}{lll}
DAP\_URL & = & "http:" "//" host [ ":" port ] [ abs\_path ] \\
abs\_path & = & server\_path dataset\_id "." ext [ "?" query ] \\
server\_path & = & \\
dataset\_id &= & \\
ext & = & "das" | "dds" | "dods" | "ver" | "html" | "info" \\
    &   & | "asc" \\
\\
query & = & CE | AIS | CE "\&" AIS\\
AIS & = & "ais(" AIS\_REF *(, AIS\_REF) ")"\\
AIS\_REF & = & *(merge) AIS\_URL \\
merge & = & "default" | "replace" | "fallback" \\
AIS\_URL & = & "http:" "//" host [ ":" port ] [ ais\_path ] \\
         &   & | "file:" "//" [ ais\_path ]\\
ais\_path & = & "/" | 1*("/" <word>) ".ais"\footnote{See
  Section~\ref{sec:groups} for information about AIS resource references that
  end in a slash.}\\
\\
CE & = & *(projection) *(\& selection) \\
projection & = & ids | function \\
ids & = & id | id , ids \\
function & = & id ( args ) \\
args & = & arg | args arg \\
arg & = & id | <quoted string> | <int> | <float> \\
    &   & | [ deref ] URL \\
\end{tabular}
\end{center}
\end{ttfamily}
\end{minipage}
\cbend

This syntax will correctly parse both \ac{URL}s which reference \ac{AIS}
resources as well as older \ac{URL}s which may have already been embedded in
application configuration files, web sites, \emph{et cetera}.

The \ac{AIS} \ac{URL} is a regular \ac{URL} as described by RFC~2396,
``Uniform Resource Identifiers (URL): Generic Syntax''~\cite{rfc2396}.
\cbstart In practice remote resources will only be accessed using the
\ac{HTTP}, HTTPS or FILE protocols, but this issue should be handled by the
underlying network I/O layer, not the \ac{DODS} or \ac{DAP} software. For
example, \texttt{libwww} supports a variety of transport protocols. The
`documents'\footnote{The information returned by dereferencing the AIS URL
  may be held in a file, a database, generated by a CGI, \emph{et cetera.}}
referenced will have the extension \texttt{.ais}. Information that augments a
DAS and/or a DDS may appear in an \texttt{.ais} document. The textual
representations for these will used and their current form allows both to be
int he same file unambiguously (see the DAP 2.0 Specification). Note that the
URL may describe a remote resource accessed via HTTP, \emph{et c.}, or a
local resource accessed using the \texttt{file:} protocol identifier. \cbend

Example 1:\\
\begin{vcode}{it}
http://www.dods.org/dods-3.2/nph-dods/data/fnoc1.nc.das?
    &ais(http://unidata.ucar.edu/dods/fnoc1_fgdc.adas)
\end{vcode}

In Example 1 the DAS object for the FNOC1 data set is accessed and merged
with the extra DAS information in the \ac{AIS} resource referenced by
fnoc1\_fgdc.adas held at Unidata.

When a user requests a DAS object, the only permissible \ac{AIS} resource is
one that augments a DAS object. See Section~\ref{sec:DDS} for a discussion of
DDS objects affected by attribute (DAS) information.

\subsubsection{Augmenting the DAS}
\label{sec:DAS}

\paragraph{Writing \ac{AIS} resources for the \ac{DAS} object}
An \ac{AIS} resource that augments a \ac{DAS} object is a text file that
contains structured text using the grammar for the external representation of
a \ac{DAS} object as described in ``DODS DAP
2.0-Draft''~\cite{gallagher:dap-spec}. 

Example 2:\\
\begin{vcode}{it}
Attributes {
  u {
    String long_name "U-component of the vector wind";
  }
}
\end{vcode}

\paragraph{Supporting parallel sets of attributes}

\cbstart
The \ac{AIS} will not extend the current DAS syntax to support parallel sets
of attributes. Instead \ac{AIS} resource authors will have to write separate
\ac{AIS} resources for each set of attributes. 
\cbend

If a \ac{URL} writer includes two or more \ac{AIS} resources, and both
provide an attribute with the same name but a different value, the \ac{AIS}
will behave as follows: 
\begin{enumerate}
\item The resource listed \cbstart first \cbend will take precedence; its
  value will be the value of the attribute. This is the default behavior; no
  special notation is needed.
\item Attributes are replaced. Only the attributes from the \ac{AIS} resource
  are used. This is called `replacement behavior;' to get it the \ac{AIS} URL
  must be prefixed by \texttt{replace}.
\item Attributes from the \ac{AIS} resource are used if and only if they are not defined
  in the data source. This is called `fallback behavior;' to get it the
  \ac{AIS} URL must be prefixed by \texttt{fallback}.
\end{enumerate}

\subsubsection{Augmenting the \ac{DDS}}
\label{sec:DDS}

The \ac{AIS} \ac{DDS} will contain the textual representation of a \ac{DDS}
as described in the DAP 2.0 specification~\cite{gallagher:dap-spec} 
as amended by this document.

\ac{AIS} resources which augment a data source's \ac{DDS} object will
add new variables to a data source. Client programs will be able to request
these variables just as they can request the `real' variables. 

To denote that a variable has been added to a data source by an \ac{AIS}
resource, that variable's declaration will be prefixed by the type modifier
\texttt{virtual}. For the sake of convenience these variables will be
referred to as \textbf{virtual variables}\onlygloss{virtual-variables}.

\paragraph{Virtual variable values}
\label{sec:virtual-var-vals}
\cbstart While there may be other sources of virtual variables supported in
the future, in this version of the \ac{AIS} SRS only values read from files
or other variables or attributes already present in the dataset will be
allowed.

When variables or attributes are used as a source of values for virtual
variables, their names should be prefixed by, `\texttt{Dataset.}' and
`\texttt{Attributes.}, respectively.\footnote{This extends the concept that
  these are container objects; it seems like a syntactically natural way to
  unambiguously denote the source of the values.}

When the values for a virtual variable are read from a file, the name of the
file is given using a function-like syntax. This syntax is used to simplify
future expansion to other sources for values.

\paragraph{Virtual variable value files}
\label{sec:virtual-var-val-files}
The files which hold the values for a virtual variable will use a CSV
format. Most virtual variables will be either arrays of simple-types or
additional columns in Sequences. The format of the virtual variable file will
only address adding values for those datatypes. In the future this may be
expanded. 
\begin{enumerate}
\item A \texttt{\#} in the file will start a comment which will run to the end
  of the line.
\item Values in the file will be either numbers (using the same syntax for
  integer and floating values as is used by the DAS) or strings. Strings
  which contain spaces must be quoted using double quotes, otherwise the
  quotes are optional.
\item For one and two dimensional arrays of simple-types, each row of the
  file will hold the values for a row of the array. Rows will be delimited by
  a carriage return ('\verb+\n+'). Values for each column position will be
  separated by commas ('\texttt{,}').
\item For $N$-dimensional arrays of simple-types, dimensions $3, \cdots, N$
  will be represented by repeating the pattern for dimensions $2, \cdots, N-1$
  separated by a blank line. (Note that the reader will know the
  dimensionality of the data because that information is in the DDS.)
\item For a column in a Sequence, each value will appear on a separate row.
\end{enumerate}
\cbend

The Example below shows how the virtual variable feature can be used to
transform a simple array (Called \texttt{SST} in the data source) into
a Grid.\footnote{See the DAP 2.0 specification~\cite{gallagher:dap-spec}
for an explanation of data types supported by the \ac{DODS}
\ac{DAP}.} The Grid is called \texttt{sea\_surface\_temperature} and
its array part is a virtual variable called \texttt{SST} which gets
its value from the data source's \texttt{temp} variable. The two map
vectors for the \texttt{sea\_surface\_temperature} Grid are also
virtual variables; their values read from files.

\cbstart
Example:\\
\begin{vcode}{it}
Dataset {
    virtual Grid {
      Array:
        virtual Byte SST[longitude = 512][latitude = 512] = Dataset.temp;
      Maps:
        virtual Float64 longitude[512] = file("long.dat");
        virtual Float64 latitude[512] = file("lat.dat");
    } sea_surface_temperature;
}
\end{vcode}

A partial grammar, updated to include specification of values for virtual
variables, for the DDS object's external representation is shown below. This
grammar does not show the keyword's various literal representations and a few
other details that are not relevant to virtual variables. For the complete
grammar see \cite{gallagher:dap-spec}.

\begin{minipage}{5in}
\begin{ttfamily}
\begin{center}
\begin{tabular}{lll}
datasets & = & +(dataset) \\
dataset  & = & dataset "\{" *(decl | v\_decl) "\}" name ";" \\
decl & = & list decl  \\
            & &   | base-type var ";" \\
            & &   | structure  "\{" *(declaration) "\}" var ";" \\
            & &   | sequence "\{" *(declaration) "\}" var ";" \\
            & &   | grid "\{" array : *(declaration) \\
            & &     maps : *(declaration) "\}" var ";" \\
v\_decl & = & base-type var "=" value ";"\footnote{Even though the grammar
  allows any base-type (both scalars and vectors) initial versions of the
  \ac{AIS} will limit these to arrays and scalars within Sequence variables.}
  \\
base-type & = & byte | int16 | uint16 | int32 | uint32 \\ 
          & &   | float32 | float64 | string | url \\
var     & = & id | var array-decl \\
value & = & "Dataset." name\footnote{\emph{name} is the name of a
  variable as it would appear in a \ac{CE}.}
          | "Attributes." name\footnote{\emph{name} is the fully-qualified
  name of an attribute.}\\ 
      & & | "file(" string ")"\\
array-decl & = & "[" integer "]" | "[" id "=" integer "]" \\
\end{tabular}
\end{center}
\end{ttfamily}
\end{minipage}
\cbend

\subsubsection{\ac{AIS} Servers}
\label{sec:ais-servers}

The \ac{AIS} Server provides a way to hide the use of \ac{AIS} resources
behind an interface that acts like a regular DODS server. 

The \ac{AIS} Server provides access to one or more DODS datasets using
regular DODS URLs. 

Each dataset to which an \ac{AIS} Server provides access is really served by
another DODS server (e.g., the DODS-HDF4 server).\footnote{Think of an AIS
  Server as a kind of proxy server.} A configuration file provides the
\ac{AIS} Server with the information it needs to match URLs to specific
\ac{AIS} resources. 

\subsubsection{Groups of data sources}
\label{sec:groups}
\emph{[Remove the idea of groups; make this about sites. Allow for RDBs to
  maintain the list of things. That is, redefine in the guise of a service.
  Add a note about REs when using a file.]}

It will be possible to use a single \ac{AIS} resource token with a group of
data sources. For example, a collection of FNOC data sources might have a
matching collection of \ac{AIS} resources. As a convenience to clients, it
will be possible to \emph{reference the collection} of \ac{AIS} resources in
the \ac{DODS} \ac{URL} in place of a specific \ac{AIS} resource reference.
The \ac{AIS} will choose the appropriate resource from the collection for a
particular data source. \cbstart This will simplify using the \ac{AIS} since
users can name the collection and let the \ac{AIS} figure out which parts of
the collection apply to the given data source. \cbend

To indicate that a particular reference is to a collection of \ac{AIS}
resources, the reference must end in a slash (\texttt{/}). See Example 1.

Example 1:\\
\begin{vcode}{it}
http://www.dods.org/dods-3.2/nph-dods/data/fnoc17.nc.das?
&ais(http://unidata.ucar.edu/fnoc_fdgc/)
\end{vcode}

\acl{AIS} resource collections may be either local or remote. Note
that a remote resource collection must refer only to resources local
to itself due to the prohibition against chaining. \emph{[Is this
really true?]}

\paragraph{Selection of specific \ac{AIS} resources from collections}
Whether local or remote, \ac{AIS} resource collections must contain a
mapping of the elements of the collection to individual data
sources. For local collections, this mapping will be held in the local
configuration file. Remote resources will use a configuration file at
the same location as the resources themselves. In both cases the same
syntax will be used.

\emph{[Remove this example; add grammar; add better examples.]}

Example:\\
\begin{vcode}{it}
fnoc_fdgc {
    http://www.dods.org/dods-3.2/nph-dods/data/fnoc1.nc.das =
        http://unidata.ucar.edu/fnoc_fdgc/fnoc1_fgdc.ais;
    .
    .
    .
    http://www.dods.org/dods-3.2/nph-dods/data/fnoc17.nc.das =
        http://unidata.ucar.edu/fnoc_fdgc/fnoc17_fgdc.ais;
    .
    .
    .
}
\end{vcode}

\emph{[Grammar here.]}

\begin{minipage}{5in}
\begin{ttfamily}
\begin{center}
\begin{tabular}{lll}

\end{tabular}
\end{center}
\end{ttfamily}
\end{minipage}

\subsubsection{AIS access logging}
\label{sec:logging}

AIS accesses at a site will be recorded. This means that accesses made using
the `\ac{AIS} resource in the URL' and through an \ac{AIS} server will both
be recorded. 

\ac{AIS} Servers will provide access to the logfile (as an ASCII text file)
using the special URL
\texttt{http://$<host>$/$<cgi_bin>$/nph\_dods/ais\_log\_info}. Sites that
host an \ac{AIS} server will be able to disable this feature.

\ac{AIS} resource references embedded in a URL will be logged by the machine
that serves those resources. It will be possible to request \ac{AIS} resource
logging information using the same URL as for the \ac{AIS} Server. Sites will
be able to disable this feature.

In both cases accesses shown in the \texttt{ais\_log\_info} document will
reflect only accesses to \ac{AIS} resources.

\subsection{Performance requirements}
N/A

\subsection{Design constraints}
Our use of the query string in a \ac{URL} flies in the face of
RFC~2396~\cite{rfc2396}.  This is potentially a problem although so far no
servers have failed because of the technically illegal characters we use. If
we were to require strict adherence to the RFC, then many characters in the
query string portion of the \ac{URL} would have to be escaped according to
RFC~2396~\cite{rfc2396}. This make writing URLs by hand virtually impossible
and greatly hinders reading them, even by programmers, as well. For these
reasons, I decided to relax the constraints on characters describe by
RFC~2396 for DODS constraint expressions. We'll do the same
here.\footnote{This does not mean I think all users should be writing URLs,
  just that I think we should spend some effort to minimize the
  unpleasantness of that experience.}

\addcontentsline{toc}{section}{References}
\raggedright
\bibliography{dods}
\bibliographystyle{plain}

%\addcontentsline{toc}{section}{Glossary}
\printgloss{dods-glossary}

\appendix

\section{Acronyms and Abbreviations}
\begin{acronym}
%
% Make one entry per line, even if they are long lines that wrap and look
% ugly. This makes it simple to sort the list using emacs' sort-lines
% command. 3/27/2000 jhrg
%
% $Id$

%\acro{AS}{Aggregation Server}
\acro{AIS}{Ancillary Information Service}
\acro{BNF} {Backus-Naur Form}
\acro{CE}{Constraint Expression}
\acro{CGI}{Common Gateway Interface}
%\acro{CSV}{Comma Separated Values}
\acro{DAP}{Data Access Protocol}
\acro{DAS}{Dataset Attribute Structure}
\acro{DDS}{Dataset Descriptor Structure}
\acro{DODS}{Distributed Oceanographic Data System}, See the DODS home page: \texttt{http://\-unidata.ucar.edu\-/packages\-/dods/}
\acro{DataDDS}{Data Dataset Descriptor Structure}
\acro{HTML}{Hypertext Markup Language}
\acro{HTTP}{HyperText Transfer Protocol}
\acro{MIME}{Multimedia Internet Mail Extensions}
\acro{SRS}{Software Requirements Specification}, See IEEE 830--1998
\acro{URI}{Uniform Resource Identifiers}
\acro{URL}{Uniform Resource Locator}
\acro{W3C}{The World Wide Web Consortium}, See http://www.w3c.org/
%\acro{WWW}{The World Wide Web}
%\acro{XDR}{External Data Representation}
\acro{XML}{Extensible Markup Language}
%\acro{COARDS}{Cooperative Ocean/Atmosphere Research Data Service}
\acro{FGDC}{Federal Geographic Data Community}

\end{acronym}

\section{Change log}

\begin{verbatim}
$Log: ancillary_information.tex,v $
Revision 1.17  2004/04/24 21:37:22  jimg
I added every directory in preparation for adding everyting. This is
part of getting the opendap web pages going...

Revision 1.16  2004/03/03 17:22:48  jimg
Needs to be looked at... I changed the date to fix an error with CVS. This
needs some updating.

Revision 1.15  2003/10/15 15:26:29  jimg
check point

Revision 1.14  2002/01/03 14:34:35  jimg
Most of section three is now complete.

Revision 1.13  2002/01/01 03:10:55  jimg
Sections 1 and 2 are mostly redone.

Revision 1.12  2002/01/01 01:27:14  jimg
I removed much of the old text that had been moved to comments (mostly about
tokens).

Revision 1.11  2002/01/01 01:24:29  jimg
Added information from the ais_goals.txt document.

Revision 1.10  2001/11/30 23:50:26  jimg
This version of the paper is still incomplete WRT changes that have been
suggested. However, a new architecture has been proposed for the AIS and it
seems like that should be investigated before more time is spent refining the
ideas here.

Revision 1.9  2001/11/14 00:37:56  jimg
Fixed a bunch of spelling errors.

Revision 1.8  2001/11/13 07:19:27  jimg
Spelling & Grammar fixes.

Revision 1.7  2001/11/11 20:23:26  jimg
Added information about resource collections (a way to reference an AIS
site and have it choose the correct resource for a particular file/
dataset). Also added more information about AIS resources for the DDS
object. Finally, added a constraint that chaining not be allowed
(although this is somewhat vague to me).

Revision 1.6  2001/11/08 00:41:55  jimg
Added a description of the way variables will be added to datasets based on
the discussions on DODS-tech@unidata.ucar.edu.
Still to go: Prohibition against chaining (?), Mappings at the AIS site(s),
Mappings at client, Translation (NB: Per conversation with Dan, maybe this is
a separate beast?).

Revision 1.5  2001/11/03 05:26:24  jimg
Changes from Peter and Paul. More changes to come. See the changebars.

Revision 1.4  2001/10/23 05:39:45  jimg
Fixed up section three. Added outline to DDS part.

Revision 1.3  2001/10/20 00:07:28  jimg
Added first crack at Section 3.
Still to do: Good examples, DDS.

Revision 1.2  2001/10/18 23:30:10  jimg
Completed first go at section two.

Revision 1.1  2001/10/18 21:22:02  jimg
First crack at the introduction.

\end{verbatim}

\end{document}
