\documentclass[12pt,titlepage]{dods-paper}
\usepackage{acronym}

% $Id$

\W\newcommand{\vskip}{} % `\W' means `only-in-html'

\newcommand{\htmlnote}[1]{\texorhtml{\footnote{#1}}{ (\xlink{#1}{#1})}}
\newcommand{\Cpp}{\texorhtml{{\rm {\small C}\raise.5ex\hbox{\footnotesize ++}}}{C++}}
\newcommand{\dap}{OPeNDAP$^2$}
\newcommand{\thirdparty}{third--party}
\newcommand{\thirdpartysite}{\thirdparty\ metadata site}
\newcommand{\gdaac}{Goddard DAAC}
\newcommand{\semanticuse}{{\it semantic use metadata}}
\newcommand{\semanticsearch}{{\it semantic search metadata}}
\newcommand{\Semanticuse}{{\it Semantic use metadata}}
\newcommand{\Semanticsearch}{{\it Semantic search metadata}}

\title{\acl{NVODS} \acs{NVODS} Status Report}
\author{Peter Cornillon and Paul Hemenway}
\date{24 February 2002}

\begin{document}
\maketitle

\T\pagestyle{plain}
\T\pagenumbering{roman}

\pdflink{nv.pdf}

\texorhtml{\tableofcontents\pagebreak\section*{Executive Summary}}%
{\htmlmenu{2}\begin{center}\bf\Large Executive Summary\end{center}}

This is a status report for the \acl{NVODS} project that includes an
overview of the operation of the project.

\ac{NVODS} is envisioned as \emph{a system that facilitates access to 
oceanographic data and data products via the Internet} regardless 
of:  data type, where the data are stored, the format in which they 
are stored, the user's visualization package or the user's level 
of expertise.

\ac{NVODS} is to be a component of the \ac{GOOS}, but its use is
expected to extend well beyond the \ac{GOOS} effort. It is to be a
system that will evolve with the Internet, with computer hardware,
with computer software and with the interests of its users.

In order to satisfy this vision, the system must provide:

\begin{enumerate}
\item A mechanism for people with data (or data products) to serve
  these data in a consistent form via the Internet without having to
  reformat them;

\item A data transport mechanism independent of platform, storage
  format, or user application;
  
\item A set of client software applications including basic network
  browsing tools from which the user may access any data or data
  product in the system and with which the user may visualize the
  data; and

\item One or more data location sites which will allow the user to
  discover in a simple and logical manner any datasets served via the
  system containing data of interest to the user.
\end{enumerate}

\texorhtml{\begin{center}\rule{3in}{.005in}\end{center}}{\htmlrule}

The \ac{NVODS} effort began with a year of intense planning.  The
process consisted of regional workshops, a national meeting, a
detailed analysis/synthesis by the \ac{NVODS} development team of the
recommendations arising from the various meetings, development of a
plan by the same group for the next two years of the \ac{NOPP}-funded
effort and an \ac{ExCom} meeting in which this plan was finalized.
This report summarizes each of the primary components of the planning
effort and concludes with the plan endorsed by the \ac{NVODS}
\ac{ExCom}.

The \ac{NVODS} effort is based on the ongoing \ac{DODS} 
project. The most important distinction between the \ac{DODS} project
and \ac{NVODS} is that the \ac{DODS} focus is on the data access
protocol, i.e., how data are moved around over the network, while \ac{NVODS}
is envisioned as a more complete system consisting of the basic data
access protocol, extensive population of the system with oceanographic 
data, web-based and stand-alone user interfaces, data location sites, 
intermediate processing sites and ancillary metadata sites. 

The \ac{DODS} component of the system is well developed. Therefore the
focus of the \ac{NVODS} effort over the next two years is on
development of the remainder of the required services and on system
population. At the end of the current \ac{NOPP} funding we would like
to see:

\begin{enumerate}
\item A large and diverse group of users that depend on the system to
  access ocean data for a wide variety of applications; and
  
\item Use of the underlying data access protocol in a suite of
  regional, national, and international systems of analysis and
  display which require real-time, near real-time, and/or analysis of
  terabyte databases.
\end{enumerate}

During the first year planning effort, it became clear that the 
community of users of ocean related data, the community that \ac{NVODS}
must serve, is extremely broad, consisting of groups with very different
data and data product needs and with very different visualization and
analysis capabilities. Specifically, the \ac{NVODS} user community
consists of:

\begin{enumerate}
\item The scientific community desiring access to the entire range of
  data and data products (from low level data to high level data
  products) via the entire range of data analysis tools (from low end
  web browser interfaces to high end analysis packages);
  
\item Planners and managers with a need to access middle and high
  level ocean data products also using a broad range of tools (from
  low end web browser interfaces to high end \ac{GIS} packages); and
  
\item The lay public often interested in the highest level data
  products accessed via middle to high end web browser interfaces.
\end{enumerate}

Given the extremely broad range of \ac{NVODS} user needs, it is clear
that a centralized source of funding will not be adequate to build all
of the tools required by all of the communities nor will it be
adequate to fund the installation of data servers at all possible data
provider sites. The planning effort therefore focused on developing a
population strategy that will be ``self-seeding,'' on the development
of an open source infrastructure to which others can add necessary
software components, and on assembling --- with existing funding --- a
community of software developers who will build some of the more basic
user interfaces. With these elements in place we believe that the
number of users will grow rapidly, as will the numbers of those
providing their data via the system and those developing interfaces to
the system.  This then defines the project strategy at the highest
level.

These high level goals will be met by:

\begin{enumerate}
\item Increasing user and potential user awareness and involvement
  through committees, working groups, pilot projects and presentations
  at national and international meetings; and
  
\item Developing the necessary technological infrastructure both by
  the development of new software and by linking the system with
  existing and emerging related technologies for data transport and
  data product generation.
\end{enumerate}

Specific tasks that will undertaken in the remaining two years of the 
currently funded \ac{NVODS} effort include:

\begin{description}
\item[Population] A number of population efforts will be undertaken
  by project regional coordinators with the goal of doubling the
  number of sites serving data via the system by the end of the
  project.
  
\item[Committees and Working Groups] The Executive Committee
  identified the need for several Committees and Working Groups
  following the recommendations of the workshops and national meeting.
  These groups will be set up as appropriate.
  
\item[Pilots] The Executive Committee recommended pursuing several
  pilot projects to demonstrate the end-to-end utility of \ac{NVODS}
  within specific user communities as well as across a range of user
  communities to address regional, national, and global problems.
  
\item[Software Development] The \ac{NVODS} software is an extension
  of the Distributed Oceanographic Data System \ac{DODS} software that
  has been developed over the last decade.  The data transport
  mechanism is robust, general, and maturing.  Clients and servers are
  available in many formats widely used in the oceanographic
  community.  Server-side services provide needed capabilities such as
  allowing the user to access a browser page which can be used to
  access and subset any dataset in the \ac{NVODS}.  Because many
  planners, managers, and educators use \ac{GIS}, \ac{NVODS} is
  commissioning a \ac{GIS} client to allow \ac{GIS} users to access
  \ac{NVODS} data.  A process that allows the addition of ancillary
  metadata to a dataset is being designed and will be implemented
  during the course of the project.  The \ac{NVODS} project is working
  with the \ac{GCMD} to have a complete and accurate set of search
  metadata resident at the \ac{GCMD} for all \ac{NVODS} datasets.
  \ac{NVODS} includes an active user-support office operated by
  UNIDATA.  The \ac{NVODS} project includes a professional writer who
  writes and maintains user and technical documentation of the
  elements as they are produced.  Finally, to encourage the
  contribution of \ac{NVODS} software from outside the core
  developers, the project is stressing its open source development
  structure to the outside user community.
\end{description}

The \ac{NVODS} project is designed to provide a seamless methodology
for data provision, discovery, access, and processing for the
oceanographic user, and is the basis for information transfer within
the \ac{GOOS} for use by scientists, planners, and educators alike.
It is intended to enable regional, national, and global access to data
of all types, including real-time, global, and archival data.  In
addition, the burden of providing data to \ac{NVODS} is designed to be
light enough that both data centers and individual investigators will
find it desirable --- and feasible --- to participate.  The vision of
\ac{GOOS} is one in which we have the comprehensive knowledge of the
oceans we need to make the most intelligent use of our marine
environment, and \ac{NVODS} will provide the means to transfer the
information necessary to meet that goal.

\T\newpage
\T\pagenumbering{arabic}
\section{This Report -- Scope}
\label{This-Report}

This report presents an overview of the \acl{NVODS} effort as well as
its status as of 24 February 2002. For the sake of completeness, it
begins with background material associated with the \acl{NOPP} call
for proposals which gave rise to the \ac{NVODS} project.
Section~\ref{PLANNING-STAGE} presents a summary of the recently
completed first year planning stage of the project. In the next
section, the vision of \ac{NVODS} as articulated by the \ac{ExCom} at
their August 2001 meeting is presented. This is followed with an
overview of the organizational structure of the project recommended by
the \ac{ExCom} at the same meeting. Appendix~\ref{STATUS} details the
status of \ac{NVODS} at the conclusion of the planning year. The
remaining sections detail the tasks that will be undertaken in years 2
and 3 of this project to help achieve the broad objectives laid out in
Section~\ref{VISION}.

\section{Background}

In 1999 the \ac{NOPP} released a \ac{BAA} requesting proposals for the
``Planning and implementation of a `Virtual Ocean Data Hub' (VODHub)
activity as a key element of the full community-based `system' to
broaden and improve access to ocean data$\ldots$'' As described in the
\ac{BAA}, ``[t]he genesis for this initiative was a series of
workshops in 1997 (Nowlin) and 1998 (Powell) illuminating the need for
an \ac{ORSMP} which garnered further impetus and definition with the
publication of the \ac{NOPP} report `Toward a U.S. Plan for an
Integrated, Sustained Ocean Observing System' in April 1999. These
reports are available at the NOPP web 
sites[: \xlink{http://nopp.org}{http://nopp.org}]''.

A group of 23 primary and 27 secondary partners, led by the \ac{DODS}\ 
group at \ac{URI}, successfully responded to this request for
proposals\htmlnote{http://www.unidata.ucar.edu/packages/dods/archive/proposals/nopp-html/}.
The requested funding was for three years and began in the summer of
2000. The first year of the effort was, as requested in the \ac{BAA},
designed to obtain community consensus on the basics of the system to
be implemented, as well as to design this system. The second and third
years were to be devoted to implementation of the system.

\section{The Planning stage - The First Year}
\label{PLANNING-STAGE}

The first year has been completed. It involved a suite of four
regional workshops one held in the
Northeast\htmlnote{http://www.po.gso.uri.edu/tracking/vodhub/reports/synth/synthesis\_100.html},
one in the
Southeast\htmlnote{http://www.po.gso.uri.edu/tracking/vodhub/reports/synth/synthesis\_79.html},
one on the Gulf
Coast\htmlnote{http://www.po.gso.uri.edu/tracking/vodhub/reports/synth/synthesis\_18.html}
and one on the West
Coast\htmlnote{http://www.po.gso.uri.edu/tracking/vodhub/reports/synth/synthesis\_56.html},
each led by a regional coordinator who is also a Co-I on the project.
The focus of each of the workshops was to identify the requirements
for the data dissemination elements of a distributed data system.  In
the original proposal a ``Great Lakes'' region was also identified, but as
a result of a lack of interest in the project on the part of the Great
Lakes coordinator, this associated regional meeting was replaced with
a meeting devoted to issues related to \ac{GIS} access to data
available via the
system\htmlnote{http://www.po.gso.uri.edu/tracking/vodhub/reports/synth/synthesis\_120.html}.
Following the workshops, the five workshop reports were synthesized
into one document that summarized the ensemble of workshop
recommendations\htmlnote{http://www.po.gso.uri.edu/tracking/vodhub/reports/synth}.
The synthesis report served as background material for a National
meeting of project primary PIs that was held in Washington, D.C. from
25 to 27 April
2001\htmlnote{http://www.po.gso.uri.edu/tracking/vodhub/reports/NVODSnationalreport.pdf}.
%The recommendations of this meeting were then used to define 
%the course of the project over the next two years. 

James Gallagher, the technical lead on the project, Peter Cornillon,
the project \ac{PI}, and project staff met at \ac{URI} on 11-13 June
2001 to determine the technical elements that were needed to implement
the recommendations that emerged from the workshops and the National
Meeting\footnote{The list of resulting elements is presented in
  Appendix~\ref{WORKSHOP-RECOMMENDATIONS}}. The group then assigned
priorities to the elements of this list for presentation to the
\ac{ExCom}\footnote{The Executive Committee is made up of Peter
  Cornillon of \ac{URI}, the project \ac{PI}, James Gallagher of
  \acs{OPeNDAP}, the project technical lead, Glenn Flierl of
  \acs{MIT}, Steve Hankin of \acs{PMEL}, Linda Mercer of the State of
  Maine, and Worth Nowlin of Texas A\&M.}
based on the group's perception of the overall goals of the project.

The \ac{ExCom} met on 13-14 August 2001. In this meeting the committee:

\begin{itemize}
\item Formalized a vision for the project (Section~\ref{VISION});

\item Made recommendations concerning the overall organization of the project 
into committees and working groups (Sections~\ref{ORGANIZATION} and
\ref{WORKING-GROUPS});

\item Finalized the list of recommendations and the prioritization of 
the associated software tasks (Appendix~\ref{PRIORITIZED-LIST});

\item Identified regional and other pilot projects that might enhance 
the development and the visibility of \ac{NVODS} capabilities
(Section~\ref{PILOTS});

\item Discussed issues associated with the continuance of the project beyond 
the three-year lifetime of the grant (Appendix~\ref{OPENDAP}); and

\item Identified relationships that should be pursued with ongoing 
oceanographic projects in the nation and the world, and their project 
data distribution needs. 
\end{itemize}

The recommendations made by the \ac{ExCom} in this meeting and the
actions taken on these recommendations to-date are summarized in
Appendix~\ref{EXCOM-RECOMMENDATIONS}.

\section{Niche, User Community and Project Objectives}
\label{VISION}

At the end of the \ac{NVODS} National Meeting, Mel Briscoe presented the 
\ac{NOPP} perspective on \ac{NVODS} and where it should be going. In his 
presentation, he emphasized that the ultimate \ac{NVODS} user was to be
the ``management'' community, not the scientific community, and that \ac{NVODS}
was to focus on serving data products rather than the raw data from which
these products were made. He defined the management community to be those 
requiring access to ocean data products to make management decisions such
as oil spill response, fisheries management, etc. This was not inconsistent 
with the issues underling many of the recommendations that emerged from the 
regional meetings, although his presentation did raise some questions about 
precisely what the bounds on the \ac{NVODS} effort were. This issue was
revisited by the \ac{ExCom}. In the ensuing discussion the \ac{ExCom} 
outlined where \ac{NVODS} fits in with regard to an end-to-end data 
management system (effectively a vision for \ac{NVODS}), who comprises 
the \ac{NVODS} user community and where the project should be at the end of
the current \ac{NOPP} funding. The \ac{ExCom} is well aware that 
there will be some adjustment to each of these elements as the project 
evolves.

\subsection{The \ac{NVODS} Niche}

At the \ac{NVODS} National Meeting, Neville Smith presented an overview of the 
various elements of an end-to-end data system and the deficiencies associated
with the elements as they exist today as well as with the linkages between 
these elements. An end-to-end data system as envisioned by Smith extends from
data acquisition at the sensor through processing, distribution, visualization,
and analysis to a ``deep'' archive such as the \ac{NODC}. The \ac{ExCom} sees 
\ac{NVODS} as the infrastructure in the middle necessary to facilitate access 
to data. \ac{NVODS} as envisioned by the \ac{ExCom} is:

\begin{itemize}
\item a system that \emph{facilitates access to oceanographic data and 
data products via the Internet} regardless of data type, where the data are 
stored, the format in which they are stored, the user's visualization package 
or the user's level of expertise;

\item a system that is a component of \ac{GOOS}, but is also expected to 
extend well beyond the \ac{GOOS} effort (see Appendix~\ref{GOOS-VISION} for
an overview of the national and international efforts associated with the
\acl{GOOS} and a brief summary of the role that we envision \ac{NVODS} 
playing in \ac{GOOS}); and  

\item a system that will evolve with the Internet, with computer hardware, 
with computer software and with the interests of its users.
\end{itemize}

The \ac{ExCom} does not see \ac{NVODS} as dealing with the acquisition
of data at the sensor nor as dealing with issues related to the long
term storage of data. \ac{NVODS} should however be interfaced to the
long term archives and could well be interfaced to sensors acquiring
the data. \ac{NVODS} is also not seen as dealing with the system
processing capability, the actual network infrastructure, nor the very
low level data transport protocols, such as TCP/IP.

The basic components required of the system are:

\begin{enumerate}
\item A mechanism for people with data (or data products) to serve
  these data in a consistent form via the Internet without having to
  reformat them;

\item A data transport mechanism independent of platform, storage
  format, or user application;
  
\item A set of client software applications including basic network
  browsing tools from which the user may access any data or data
  product in the system and with which the user may visualize the
  data; and

\item One or more data location sites which will allow the user to
  discover in a simple and logical manner any datasets served via the
  system containing data of interest to the user.
\end{enumerate}

Other components of a data system that lie between acquisition and
a ``deep'' archive that were discussed are data product generation, 
visualization software and processing or analysis software. The 
\ac{ExCom} felt that, although very important, significant efforts in 
these areas were beyond the scope of the current funding. They did
however feel that to help ``jump start'' the system, low level efforts 
associated with these might be appropriate, for example, a generic web 
display tool for \ac{NVODS} accessible data sets or the generation of 
some low level data products.

\subsection{The \ac{NVODS} User Community}

The community of users targeted for \ac{NVODS} is as critical in designing
the system as is the niche that the system is to occupy in the end-to-end
data system environment. The \ac{ExCom} recognized that the community of 
potential users of ocean related data is extremely broad consisting of 
groups with very different data and data product needs and with very 
different visualization and analysis capabilities. Specifically, the 
eventual \ac{NVODS} user community is expected to consist of:

\begin{enumerate}
\item the scientific community desiring access to the entire range of 
data and data products (from low level data to high level data products) 
via the entire range of data analysis tools (from low end web browser 
interfaces to high end analysis packages);

\item planners and managers with a need to access mid- and high-level 
ocean data products also using a broad range of tools (from low-end
web browser interfaces to high-end \ac{GIS} packages);

\item the educational community with a need to access data and data 
products at all levels as well as analysis tools at all levels (the level
depending of course on the level of the students involved); and

\item the lay public often interested in the highest level data 
products accessed via mid- to high-end web browser interfaces. 
\end{enumerate}

The \ac{ExCom} believes that this is the community of users that
\ac{NVODS} should serve \emph{in the long run}. At the same time, the
committee recognizes that this group is so broad that it is
impractical to design and implement a system that will \emph{at the
  outset}, i.e., during the current funding cycle, meet the needs of
all of the identified user groups. As indicated above, Mel Briscoe
made it clear in his presentation that the \ac{ORAP} of \ac{NOPP} felt
that the focus should be on the management community rather than on
the general public or the scientific communities. The \ac{ExCom}
agrees that this is the community where the system will have the
largest impact --- ``the biggest bang for the buck.''  At the same time
there was a recognition in the committee that meeting the needs of the
scientific community is likely to benefit the project more in the
short term than meeting the needs of the management community, in that
the scientific community is much more likely to develop tools that
will benefit all users of the system than either of the other
communities. Put another way, the funding is not sufficient to develop
all of the interfaces, servers, intermediate processors, product
generators, and other miscellany that such a system must support. The
only way that the effort will succeed is if the community contributes
to the development of these tools as has been done with the web.

The \ac{ExCom} feels that, again as was the case with the web itself,
it is the scientific community that will develop these tools early on
\emph{if} the basic infrastructure is sufficiently flexible and robust
to accommodate a very broad range of activities and \emph{if} the
system is open source. Because of the competing requirements here ---
focusing on the management community as the one that offers the
greatest return on investment as envisioned by the \ac{ORAP} of
\ac{NOPP} and on the scientific community as the one most likely to
quickly evolve the system as envisioned by the \ac{ExCom} --- a two
fold approach is recommended.  First, the project will support a suite of
pilot efforts at the regional level that will be designed to meet
management/planning community needs, and second, the project will support
efforts that are tied to large national or international scientific
programs such as \acs{WOCE}/\acs{CLIVAR} or \acs{GODAE}.

In the discussion associated with user groups, the \ac{ExCom} also addressed
the range of data/data product types that \ac{NVODS} must be designed to
handle. Specifically, it must support data set transactions in all of the 
oceanographic sub-disciplines as well as data in related fields such as 
meteorology, it must handle sequences of data (e.g., point measurements
in time) as well as gridded data sets (e.g., model output or satellite
data products) and it must be capable of handling real-time data as well
as retrospective data sets.

\subsection{Project Objectives}

Based on the recommendations of the regional workshops and the National
Meeting, the \ac{ExCom} believes that the success of this effort by
the end of the current funding cycle will require:

\begin{enumerate}
\item The number of users should be increasing rapidly both in the 
research and management communities.

\item The \ac{DAP} on which the basic infrastructure of the system is
founded should be accepted as a data/product access ``standard'' 
by suppliers and users of oceanographic data and data products in 
general and Ocean.US in particular. Specifically, a non-profit consortium 
with significant community buy-in should have formed and be responsible 
for developing and maintaining the \ac{DAP}. (See Appendix~\ref{OPENDAP}.)

\item There should be a number of non-NOPP funded groups with 
significant experience writing both client and server software for the 
system.

\item A plan must be in place to broaden the user community to 
educational users and to the general public.

\item A sustained funding base for continued maintenance and 
development of the system must be identified.

\item There should be 4 to 5 pilots with significant use from 
different groups representative of the communities that \ac{NVODS}
is trying to serve. These groups need not be large but they should be 
obtaining significant benefit from using the system. They should serve 
as specific examples of the capability of \ac{NVODS}. 

\end{enumerate}

With these elements in place the \ac{ExCom} believes that the user 
community will grow rapidly as will those providing their data via the 
system. This then defines the project strategy at the highest level.
Appendix~\ref{EXCOM-RECOMMENDATIONS} of this report presents the details of the
recommendations of specific actions that should be taken to meet these
objectives.

\section{Organization}
\label{ORGANIZATION}

An organizational structure for the \ac{NVODS} project was defined 
in the \ac{NOPP} proposal. This structure consisted of an Executive 
Committee (\acs{ExCom}) which was to provide input to the project 
\ac{PI} with regard to the overall direction of the project, a 
Project Manager reporting to the \ac{PI} with day-to-day administrative 
responsibility and five regional coordinators responsible for organizing 
activities in the five regions identified in the proposal --- the 
Northeast, the Southeast, the Gulf Coast, the West Coast and the Great 
Lakes. As a result of some personnel changes the details of this 
structure have changed somewhat. 

The regional coordinator for the Great Lakes, Anders Andren, chose not to 
continue participation in the project following notification of funding 
from \ac{NOPP}. At the recommendation of the \ac{ExCom} and with the 
concurrence of the \ac{NOPP} program manager funding originally
allocated for this region was allocated to other regions and
this region was not included in the first year planning process.

The regional coordinator identified in the proposal for the Southeast,
Margaret Davidson, assumed a new job shortly before this project was 
funded and was unable to continue in her role as Southeast regional 
coordinator. She designated a replacement, Anne Ball, who lead the 
southeast regional workshops in the planning year. Unfortunately however, 
Anne felt that the level of effort that was required in the out years 
would be too large for her to continue in this role and she withdrew
following the National Meeting. The \ac{ExCom} decided not to replace
her and instead recommended that the Southeast region be absorbed into 
the Gulf Coast region. 

As a result of the death of a critical member of the West Coast region, 
this region had little involvement from California at the time of the 
National Meeting. In light of this, the \ac{ExCom} recommended that an 
effort be made to contact Steve Weisberg of the 
\ac{SCCWRP}\htmlnote{http://www.sccwrp.org/}. The West Coast regional 
coordinator, Mark Abbott, suggested that the project \ac{PI} do this. 
This has been done and there is significant interest in the \ac{SCCWRP}
in collaboration. The level of collaboration will be identified over
the next several months. At present this effort is still viewed as 
falling under the West Coast regional coordinator although the 
\ac{ExCom} has designated additional funds (taken from those set 
aside for the Great Lakes region) for this effort, so one might 
view this as a spur of the West Coast region.

Taken together these changes mean that there are now three regions,
the Northeast, the Southeast/Gulf Coast and the West Coast, plus a
possible separate subregion, and that funding allocated for the five 
regions in the out years will be used predominantly by the remaining 
three regions. The focus of the regional efforts was also changed as 
a result of Mel Briscoe's comments at the National Meeting 
(Section~\ref{VISION}). Specifically, the regions were asked to 
identify one or two pilot projects that would entrain users from the 
management community. 

The Project Manager, Richard Chinman, elected to withdraw from the 
project in the middle of the first year. The \ac{PI} has assumed the 
associated responsibilities with the help of Paul Hemenway who has
been hired as an Assistant Project Manager. The Assistant Project
Manager is an ex officio member of the \ac{ExCom}.

\section{Communication}
\label{COMMUNICATION}

The \ac{NVODS} project maintains a number of different communication
mechanisms. 

\subsection{e-mail Lists}

There are two e-mail lists that were set up for the previous
DODS effort with \acs{NASA} funding. These lists are still used for
communication about technical issues related to \ac{DODS} software.
They are: 

\subsubsection{The \ac{DODS} mailing list} 

This is a low traffic list that
was set up for user discussions of \ac{DODS} software, e.g., questions 
on installation, setup, and use of \ac{DODS}. Software release and other 
general announcements will be made to this list. 

\subsubsection{The \ac{DODS}-Tech mailing list} 

This is a medium traffic list.
It is used for technical discussions concerning \ac{DODS} development with
an emphasis on those issues related to the \ac{DAP}. \\

\noindent Those interested can join either of these e-mail 
lists from the \ac{DODS} home page by clicking on the ``Mailing Lists'' 
entry under the ``Support'' tab. 

\subsection{Telecons}

Two sets of regular telecons are run by the project. These are:

\subsubsection{Management Telecon} 

This weekly telecon consists of the 
Project Manager (Cornillon), the Project Technical Lead (Gallagher), the 
Project Lead for Client Issues (Holloway) and the Assistant Project 
Manager (Hemenway). Issues dealt with by this group relate to all
software issues related to the project. The group focuses on near
term (one week to three months) aspects of these issues.

\subsubsection{\ac{DODS}-Tech Telecon} 

This is a weekly dial-in telecon and
consists of anyone who wants to participate. Agenda items as well
as scheduling changes associated with the telecon are announced on
the \ac{DODS}-Tech e-mail list. Issues discussed here tend to focus on
software schedules as well as software related issues. Minutes for
these meetings are available at 
\xlink{http://unidata.ucar.edu/packages/dods/home/developers/meetings/develTC/}{http://unidata.ucar.edu/packages/dods/home/developers/meetings/develTC/}.


\subsection{Meetings}

In the implementation phase, the NVODS effort will involve three classes 
of meetings: annual or semiannual meetings involving long range planning
for the project, semiannual meeting involving a broad range of technical 
issues and ad hoc meetings designed to focus on very specific technical
or logistic issues.

\subsubsection{Planning Meetings}

\paragraph {Co-I Planning Meetings:} 

The primary planning meeting is the Co-I assembly held in the summer
of each year. This meeting will involve all Co-Is on the project. The
meeting will consist of two parts, an informational section in which
the status of the project is summarized and a discussion phase in
which recommendations are made for subsequent work to be undertaken in
the effort.

\paragraph {Executive Committee Meetings:} 

These are semiannual meetings.  held in the late winter and late
summer. The focus of the late summer meeting is to finalize plans for
the upcoming year; i.e., to address the recommendations made at the
Co-I Planning Meeting.

\subsubsection{Technical Meetings}

\paragraph {Semi-annual Technical Meeting:} 

This is a meeting that is held in mid-winter and mid-summer. It
involves any technical group interested in the evolution of the
\ac{DAP}. The purpose of these meetings is threefold:

\begin{description}
\item[Coordination]  The \ac{NVODS} effort involves not only
those directly funded by \ac{NOPP}, but a number of other groups that
are developing software related to the \ac{DODS} \ac{DAP} or software
that is of interest to the \ac{NVODS} effort. Many of these
``external'' groups are (by choice) not involved in the telecons and
e-mail lists set up for technical communication within the project.
This meeting therefore serves as a way to better integrate the efforts
of these groups with the rest of the project. 

\item[Evolution of the System]  The \ac{NVODS} effort is 
addressing a broad range of difficult problems associated with access
to a very heterogeneous array of data from a large number of different
sources. The solution to some of these problems is well defined and
software is being written to implement these solutions. There are however
a number of other problems for which a solution has not been identified.
Another focus of the \ac{DODS}-Tech meeting is to begin or to continue 
discussions related to these problems or to more clearly identify what 
the problems actually are.

\item[Project Momentum]  The \ac{NVODS} effort is very complicated. 
We have found that despite the various e-mail lists and telecons set up for 
communication among the developers, conveying a view of the overall 
effort and how one's specific project fits in the project as a whole 
is difficult at best; issues discussed in the telecons and on the e-mail 
lists are generally buried in the details. The \ac{DODS}-Tech meeting 
therefore serves as an important way for all those on the project to come 
to appreciate the contribution that they are making in the context of the 
project as a whole. We have found that this is very important with regard
to generating or maintaining enthusiasm among those developing software 
for the project.
\end{description}

A brief summary of the most recent \ac{DODS}-Tech meeting held in
Boulder CO from 9 to 11 January 2002 is available in Appendix~\ref{DODS-TECH}.

\paragraph {Special Purpose Meetings:} 

Some elements of the project involve several components of the system
working together or could benefit with input from a number of project
participants. In special cases we have decided to hold focussed
topical meetings that involve a small group of developers. The
development of the \ac{AS} is a good example of this. The general
objectives for this server were outlined via e-mail in early 2001.
Then a meeting was held at Unidata in Boulder on 14 March 2001.
Participating in the meeting were: James Gallagher (\ac{NVODS}), John
Caron (Unidata), Russ Rew (Unidata), Steve Hankin (PMEL), Roland
Schweitzer (CDC) and Ethan Davis (Unidata). At this meeting, classes
of data aggregation were formalized, a suite of test datasets was
selected, an outline for the implementation of the server was roughed
out as was the implementation schedule. Implementation proceeded
smoothly from there.  Several NVODS aggregated datasets are currently
available via DODS.

\subsection{Software Development}

The personnel responsible for developing and maintaining \ac{NVODS}
related software fall into two groups, the \emph{Core Group}, responsible
for the \ac{DAP} and some of the clients and servers and others writing 
to the \ac{DAP} who are not funded by either \ac{NVODS} or \ac{OPeNDAP},
generally dealing with specialized clients or servers. The developers in 
both groups are distributed around the country. Development of software 
in this environment requires communication related to software elements
as well as to problems with these elements. To this end, two basic
software tracking and development mechanisms are used for software 
developed by the Core Group:

\begin{itemize}
\item \emph{Code} -- The basic code (computer programs) is maintained in 
\ac{CVS}\htmlnote{http://www.gnu.org/software/cvs/cvs.html}. Developers 
check software out of the \ac{CVS} archive, make the appropriate 
changes and check the software back into the \ac{CVS} repository, resolving 
conflicts that may have arisen from modifications made by other programmers 
when doing so. 

\item \emph{Bug Reports} -- Problems that arise with these software elements 
are reported using Bugzilla\htmlnote{http://bugzilla.mozilla.org/}, a 
free-ware package designed for managing software bugs in a distributed 
development environment. Bugzilla is used to report all problems related
to software developed by the Core Group as well as some problems related 
to software developed by non-Core participants.
\end{itemize}

\section{Working Groups}
\label{WORKING-GROUPS}

There are a number of issues being addressed on the project that
could benefit from the input of a broader range of participants 
than the core development group. The most important of these
issues were identified at the \ac{NVODS} National Meeting and
it was recommended that working groups be formed to address these
issues. The issues identified are:

\begin{itemize}
\item Data Location -- a searchable list of project accessible data sets. 
\item Standards -- both for metadata required to achieve Level 3 
interoperability\footnote{This is machine-to-machine interoperability
with semantic meaning.}
\item Web Client -- the ability to, via a web browser, select subsets 
and display these for any \ac{NVODS} accessible data sets. 
\item GIS Technical Oversight.
\end{itemize}

\noindent As of the writing of this report, these working groups have not
been formed. It is also likely that there will be other efforts 
that require standing working groups for the project.

\section{Pilots}
\label{PILOTS}
%*******

As a result of comments made by Mel Briscoe at the \ac{NVODS}
National Meeting, the \ac{ExCom} requested that each regional coordinator 
identify one or two pilot projects that would be undertaken
to attract users to the system, to exercise various components 
of the system and to demonstrate the value of the system to the
management/planning communities. These pilot projects should, where possible,
identify the end user of the data as well as the primary sources of
data and should use funding allocated for the regions to assemble all
of the basic components of the system: data servers, data integration
and data display for the target user group. To be successful, the pilot
projects will feed their technical needs back to the \ac{NVODS}
developers so that the technical capabilities of NVODS will meet
the needs of the individual pilots.  Pilots that have been
proposed as of 24 February 2002 are:

\subsection {Nearshore Surface Current Pilot\label{CODAR-PILOT}}

This pilot would provide near real-time access to measurements of 
nearshore surface currents. Users of these data are thought to range 
from decision makers addressing nearshore issues such as those involved
in  search and rescue and hazardous spill mitigation to recreational 
boaters to the scientific community. The primary source of data 
envisioned for this effort will be HF radar data beginning with 
\ac{CODAR} data. The effort will begin by focusing on the west coast
and time and funds permitting will be extended to other HF radar sites
along the U.S.~coast. Consideration will also be given to including
current meter data in the pilot. The pilot will make the data available
via OPeNDAP$^2$ as well as through a web-based interface.

\subsection {Shrimp Pilot}

This pilot is designed to provide fisheries management groups in the
northeast with access to shrimp biological and fisheries data that will 
allow them to study the effect of changes in the environment on the 
northern shrimp population in the Gulf of Maine. The information 
exists to do a very thorough study of the effects of changes in the 
environment on the northern shrimp population in the Gulf of Maine,
but these data exist at a variety of sites in a variety of formats. What 
is needed is a mechanism for making all this information available in a 
format that can be readily used to conduct appropriate analyses. 

\subsection {Modeling Pilot}

There are several modeling efforts currently underway in the Gulf
of Mexico that are used or will be used for planning purposes. \ac{TAMU} 
is proposing a pilot to produce an end-to-end demonstration of the value of 
using real time data in models to improve the now-cast and forecast of 
shelf currents in the Gulf of Mexico. This will build on the effort to 
serve all physical data from operational observing systems in the Gulf. 
Transfer of the model results will be via DODS. Initially the effort
will focus on the \ac{TAMU} operational forecast model for the Gulf 
shelves from Cape San Blas, FL to the Texas-Mexican border which is 
funded by the Texas General Land Office.

\subsection{\ac{SCCWRP}}

\acs{SCCWRP} is a joint powers agency\footnote{A joint powers agency is 
one that is formed when several government agencies have a common mission 
that can be better addressed by pooling resources and knowledge. In the
\ac{SCCWRP} case, the common mission is to gather the necessary scientific 
information so that our member agencies can effectively, and cost-efficiently, 
protect the Southern California marine environment. An important part of 
the \ac{SCCWRP} mission is to ensure that the data they collect and 
synthesize effectively reaches decision-makers, scientists and the public.} 
focusing on marine environmental research. The Executive Director of
\ac{SCCWRP}, Stephen Weisberg, has indicated that \ac{SCCWRP} would 
like to undertake a pilot project in which they would serve a subset
of their data holdings via OPeNDAP$^2$. This effort will be undertaken
directly with the \ac{NVODS} Project Office as opposed to through the
West Coast Coordinator, although close communication with regard to this
effort will be maintained with others on the west coast.

\T\newpage
\appendix    

\section{Acronyms}
\label{Acronyms}
\label{page-acro}
{\begin{acronym}
\small
\T\parskip 0pt
 \acro{AIS}{Ancillary Metadata Services}.
 \acro{AGU}{American Geophysical Union}.
 \acro{AS}{Aggregation Server}.
 \acro{ASCII}{American Standard Code for Information Exchange}.
 \acro{BAA}{Broad Agency Announcement}.
 \acro{CDC}{Climate Diagnostics Center}.
 \acro{CGI}{Common Gateway Interface}.
 \acro{CLIVAR}{Climate Variability and Predictability}. 
 \acro{COARDS}{Cooperative Ocean-Atmosphere Research Data Standard}.
 \acro{CODAR}{Coastal Ocean Dynamics Applications Radar}.
 \acro{COLA}{Center for Ocean-Land-Atmosphere Studies}.
 \acro{CSDGM}{Content Standard for Digital Geospatial Metadata}.
 \acro{CVS}{Concurrent Versions System}.
 \acro{DAAC}{Distribute Active Archive Center}.
 \acro{DAP}{Data Access Protocol}.
 \acro{dds}{Data Descriptor Structure}.
 \acro{das}{Data Attribute Structure}.
 \acro{DIF}{Directory Interchange Format}.
 \acro{DLESE}{Digital Library for Earth System Education}. 
 \acro{DODS}{Distributed Oceanographic Data System}.
 \acro{DRDS}{DODS Relational Database Server}.
 \acro{EDMI}{Earth Data Multimedia Instrument}.
 \acro{EOS}{Earth Observing System}.
 \acro{ESIP}{Earth Science Information Partner}.
 \acro{ESRI}{Environmental Systems Research Institute}.
 \acro{ExCom}{Executive Committee}.
 \acro{FGDC}{Federal Geographic Data Committee}.
 \acro{FITS}{Flexible Image Transport System}.
 \acro{GCMD}{Global Change Master Directory}.
 \acro{GDS}{GrADS-DODS Server}.
 \acro{GIF}{Graphics Interchange Format}.
 \acro{GIS}{Geographical Information System}.
 \acro{GODAE}{Global Ocean Data Assimilation Experiment}.
 \acro{GOOS}{Global Ocean Observing System}.
 \acro{GLOBEC}{Global Ocean Ecosystem Dynamics}.
 \acro{GRIB}{GRid in Binary}.
 \acro{GrADS}{Grid Analysis and Display System}.
 \acro{GSFC}{Goddard Space Flight Center}.
 \acro{GUI}{Graphical User Interface}.
 \acro{HAO}{High Altitude Observatory}.
 \acro{HDF}{Hierarchical Data Format}.
 \acro{HDF-EOS}{Hierarchical Data Format - EOS}.
 \acro{http}{Hypertext Transfer Protocol}.
 \acro{httpd}{http deamon}.
 \acro{IDL}{Interactive Display Language}.
 \acro{IGOS-P}{Integrated Global Observing Strategy-Partnership}.
 \acro{IFREMER}{Institut Fran�ais de Recherche pour l'Exploitation de la Mer}.
 \acro{JGOFS}{Joint Global Ocean Flux Experiment}.
 \acro{LAS}{Live Access Server}.
 \acro{MIT}{Massachusetts Institute of Technology}.
 \acro{NASA}{National Aeronautics and Space Administration}.
 \acro{netCDF}{Network Common Data Format}.
 \acro{NOAA}{National Oceanic and Atmospheric Administration}.
 \acro{NODC}{National Oceanographic Data Center}.
 \acro{NOPP}{National Ocean Partnership Program}.
 \acro{NORLC}{National Ocean Research Leadership Council}.
 \acro{NSDL}{National SMETE Digital Library}.
 \acro{NITF}{National Imagery Transmission Format}.
 \acro{NSF}{National Science Foundation}.
 \acro{NVODS}{National Virtual Ocean Data System}.
 \acro{OPeNDAP}{Open Source Project for a Network Data Access Protocol}.
 \acro{ORAP}{Ocean Research Advisory Panel}.
 \acro{ORSMP}{Ocean Research Synthesis and Modeling Program}.
 \acro{PES}{Planet Earth Science}.
 \acro{PI}{Principal Investigator}.
 \acro{PMEL}{Pacific Marine Environmental Laboratory}.
 \acro{RFP}{Request For Proposal}.
 \acro{SCCWRP}{Southern California Coastal Water Research Project}.
 \acro{SDTS}{Spatial Data Transfer Standard}.
 \acro{SMETE}{Science, Mathematics, Engineering and Technology Education}. 
 \acro{TAMU}{Texas A\&M University}.
 \acro{TEDS}{Tactical Environmental Data Server}.
 \acro{UCAR}{University Center for Atmospheric Research}.
 \acro{UDRS}{Universal Data Reduction Server}.
 \acro{UN}{United Nations}.
 \acro{UNCED}{United Nations Conference on Environment and Development}.
 \acro{URI}{University of Rhode Island}.
 \acro{URL}{Universal Resource Locator}.
 \acro{US}{United States of America}.
 \acro{WCS}{Web Coverage Server}.
 \acro{WMS}{Web Mapping Server}.
 \acro{WOCE}{World Ocean Circulation Experiment}.
 \acro{XML}{eXtensible Markup Language}.
\end{acronym}
}

\T\newpage
\section{The \ac{GOOS} Vision and Objectives }
\label{GOOS-VISION}

\subsection{The Global Imperative}

In 1992, the \ac{UN} \ac{UNCED} mandated the establishment of a global
ocean observing system that would enable effective and sustainable management
and utilization of the marine environment and its natural resources.
The \ac{GOOS} is an international effort to respond to this mandate.  

The \ac{IGOS-P} was established in 1998 to assist with the implementation
of several global observing systems, including GOOS.  Under the auspices of the 
\ac{IGOS-P} a team has developed ``an Ocean Theme'' (Ocean Theme Team 2000)
with the vision ``to develop and maintain continuity of observing capabilities
for the global ocean and to advance to a permanent global ocean observing
system.'' That report specifies satellite missions needed in support of the 
Ocean Theme, ongoing in situ components of the \ac{GOOS} for which 
commitments have been made, major pilot projects specifically acknowledged
as a part of \ac{GOOS}, and examples of data, products, and services available
or anticipated.

\subsection{The National Imperative}

In response to the global initiative and to address the security needs of the
\ac{US}, the \ac{US} Congress mandated the formation of the \ac{NORLC}.
In turn, the \ac{NORLC} formulated a plan for the \ac{US} to meet the
challenges of continuously monitoring and assessing marine resources
and environment for national and global health and safety, and for
increased understanding and public awareness.  The plan is embodied in
the \ac{NOPP}.  A working group prepared
a report, ``Toward a U.S. Plan for an Integrated, Sustained Ocean Observing 
System'' (\xlink{http://core.cast.msstate.edu/NOPPobsplan.html}{http://core.cast.msstate.edu/NOPPobsplan.html}), 
in response to
a Congressional request for such a plan.  The goal is an integrated, 
sustained national ocean observing system, and the report establishes
the following seven areas of focus for such a system (taken from the
Executive Summary of the above report):
\begin{itemize}
\item Detecting and forecasting oceanic components of climate variability.
\item Facilitating safe and efficient marine operations.
\item Ensuring national security.
\item Managing living resources for sustainable use.
\item Preserving healthy and restoring degraded marine ecosystems.
\item Mitigating natural hazards.
\item Ensuring public health. 
\end{itemize}

The report (section 3: Design and Status of a U.S. Ocean Observing 
System, subsection 3.1: Infrastructure) calls for creation of a prototype 
of a data management system which ``could be a combination of DODS, LABNET, 
Master Environmental Library, the hub-node concept described in Powell 
(1998), and the objectives of a distributed data system in which 
individuals and institutions serve our their data in accordance with 
community conventions (called a National Virtual Ocean Data System; 
CORE, 1998).'' 

The U.S. Ocean Observing System comprises many elements, including
\begin{itemize}
\item people
\begin{itemize}
\item Data collectors, including scientists, fisherman, managers operating
monitors of all types, 

\item Data providers, who are the data collectors in
many cases, but who may be data archivers, or system administrators.

\item Intermediate users, who convert raw data into data products such
as maps, graphs, images, animations, and web-based information used
by the myriad of end users.

\item End users, who make use of the data for their own purposes, which range 
from research on all levels to fishery managers, who have to make decisions
concerning fishing restrictions and the application of resources in marine
habitats, to short and long range planners, to search and rescue operations,
to educators at all levels, to government decision makers.  

\item Developers and maintainers of the hardware and software to allow 
the discovery, collection, storage 
and access, transfer, and transformations of the information to take place 
in a timely and efficient manner, dictated by the needs of the end users, 
rather than the inherent limitations of the system.

\end{itemize}
\item Hardware and software.
\begin{itemize}
\item Hardware for the collection and analysis of the raw data.
\item Data location mechanisms to allow users to locate data relevant to their
needs.
\item Data transmission mechanisms for getting data from providers to users,
\item Hardware and software analysis tools for users to obtain the products
they need.
\end{itemize}
\end{itemize}

A summary of many of the user constituencies and products of
the Ocean Observing System are given in Table 3.2 of the report, 
arranged to address the areas of focus itemized above.

However, the U.S. Ocean Observing System is more than the sum of its 
parts, because it embodies the end uses to which the elements are put.  It 
encompasses the determination of lobster population migrations with time and 
with resource changes in the Gulf of Maine.  It encompasses the determination
of coastal currents in real-time for search and rescue operations as well
as general shipping navigation.  It encompasses the generation of plots
of changing sea surface temperature that are downloaded in classrooms to
describe the variations of El Ni\~{n}o and La Ni\~{n}a, and their climatic
effects on local weather patterns.  It encompasses the improvement of
weather and ocean modeling systems that result in more accurate 
weather forecasts and climate modeling.  It encompasses the contributions
to society that come from a fuller knowledge of, understanding of, and
use of the marine environment which surrounds and flows through the
human habitat.  

\subsection{The Role of \ac{NVODS} in \ac{GOOS}}

\ac{NVODS} provides the infrastructure for moving the data
and metadata and data products (i.e. the information), necessary for the 
Ocean Observing System to function.

\ac{NVODS} has at its base the OPeNDAP data access protocol capable of
addressing data transport needs in \ac{GOOS}. As such it forms a
fundamental component of this system. Its contribution is however not
limited to this system in that it will also provide access to data
that are not formally part of \ac{GOOS} thus offering the possibility
for interoperability between \ac{GOOS} data holding and other data of
oceanographic interest.

In light of the role that \ac{NVODS} will play in Coastal \ac{GOOS}, project
personnel should work closely with the OCEAN.US office in establishing
the metadata and data access standards that will be required for the
\ac{GOOS} data system.

In that \ac{NVODS} is envisioned as a fundamental component of
\ac{GOOS}, the \ac{NVODS} user community should include, but not be limited to
those involved in ocean management.  The \ac{NVODS} effort should begin
working with the ocean management community immediately.

\T\newpage
\section{Workshop and National Meeting Recommendations}
\label{WORKSHOP-RECOMMENDATIONS}

A number of recommendations were made as a result of the first year
workshops and National Meeting. These recommendations were regrouped
following the National Meeting by Gallagher, Cornillon, Holloway and
Hemenway and software tasks were identified to address each of the
recommendations. Regrouping allowed for a more systematic approach
to the software effort. These software tasks were then rank ordered
in terms of their perceived importance to the task as a whole. The
rank ordered list was then presented to the \ac{ExCom} which, after
some changes, endorsed the list. The prioritized list endorsed by
the \ac{ExCom} is presented in Appendix~\ref{PRIORITIZED-LIST}. The
regrouped recommendations resulting from the workshops and National
meeting are presented below:

\begin{enumerate}
\item DAP-related modifications 
   \begin{itemize}
   \item Modify DODS to handle: [24 - other web tools handle these well.] 
      \begin{enumerate}
         \item maps 
         \item graphs 
         \item images 
         \item animations 
      \end{enumerate}

   \item Allow for more projection types on the earth's surface. [28,31] 

   \item Define data rules to make computer models and output accessible 
      [The DAP already handles model output well] 

   \item Provide password protection for DODS-served datasets. [23] 

   \item Provide the ability to display a disclaimer for certain datasets. [33] 

   \item Provide secure software for DODS servers. [11] 

   \item Intra and inter data set aggregation. [In progress for gridded data sets, 30] 

   
   \end{itemize}

\item NVODS web/browser interface and metadata issues. (Data location and use.) 

   \begin{itemize}
   \item Provide a web-based catalog that: [1,3,8,10,15,19,21,27,37] 
        \begin{enumerate}
           \item is searchable by keyword. [6,16,38] 
           \item is searchable by geography. [6,9,15] 
           \item lists all available datasets (not just DODS). [15,21] 
          \item Contains regional datasets, model output, and products 
people are willing to serve. [6] 
        \end{enumerate}

   \item Provide a web browser interface that allows: [1,9,14,17,18] 
        \begin{enumerate}
           \item location [1,9] 
           \item subsettng [1,9] 
           \item manipulation [??] 
           \item display [1,9] 
           \item data downloading [4,16] 
        \end{enumerate}

   \item Users needs to find the data they need: see 1 above. 

   \item Need a place for data quality flags [user feedback re data quality,45] 

   \item Generic display [5] 

   \item Coordinate metadata efforts with FGDC (and GCMD -ed) [15] 
   \end{itemize}


\item Client-side issues 
   \begin{itemize}
   \item Need a GIS client [2] 

   \item URL builder [1,5,6,9,15,17,18,22,29,39,40,42]
   \end{itemize}


\item NVODS user-support issues 
   \begin{itemize}
   \item Need regional help desks for installation of DODS servers and clients. 
      [A regional issue - Unidata support provides this at the national level.] 

   \item Need high quality, easy to understand documentation. [11] 

   \item Provide a means to record problems users have using DODS and installing servers.
      [Unidata support provides this.] 

   \item Establish mailing lists for regional participants. [A regional issue] 
   \end{itemize}


\item NVODS outreach efforts 
   \begin{itemize}
    \item Create pilot projects to demonstrate the usefulness of DODS for 
current regional players and others not in attendance at the 
workshops. [13] 

   \item Find ways to identify other groups who would benefit from the 
use of DODS. 
      [An ExCom issue] 

   \item Perform demonstrations of the use of DODS [27] 
   \end{itemize}


\item Other issues 
   \begin{itemize}
   \item Standards related to NVODS and DODS [6,12,15,20,31,36,37,44] 

   \item Push-pull linkage [34] 

   \item Mirror servers [38 - not really a project issue.] 

   \item Develop functionality to address: 
        \begin{enumerate}
            \item dataset certification [27] 
            \item data archaeology and rescue [Not a project priority] 
            \item long term data archiving [Not a project priority] 
        \end{enumerate}
   \end{itemize}

\item Project structure/support issues 
   \begin{itemize}
      \item Create regional web-sites and/or list servers to facilitate 
greater communication within regions. [A regional issue] 
      \item Project working groups: [An ExCom issue] 
        \begin{enumerate}
      \item Web-client [We have a first crack at this with the LAS evaluation group.] 
      \item Regional/Topical efforts/pilots 
      \item Data location/discovery 
      \item Standards [Transport and metadata.] 
      \item Strategic [Is this the role of the ExCom?] 
        \end{enumerate}
   \end{itemize}

\item Issues not identified but are of importance to the project 
   \begin{itemize}
   \item Data server metrics [7] 

   \item Repeating axes [25] 

   \item BUFR server[35] 

   \item HDF API client library[41] 

   \item Macintosh OS[43] 
   \end{itemize}

\end{enumerate}

\T\newpage
\section{Executive Committee Recommendations}
\label{EXCOM-RECOMMENDATIONS}

In addition to addressing the list of specific recommendations listed
in Appendices~\ref{WORKSHOP-RECOMMENDATIONS} and \ref{PRIORITIZED-LIST}, 
the \ac{ExCom} made a number of other recommendations. These are
summarized here along with action taken to address them.

\begin{itemize}
\item The highest short-term priority for the project should be to expand
the \ac{NVODS} user base. To effect this, a three pronged approach was
outline by the \ac{ExCom}: 

\begin{enumerate}
\item New Users must be entrained in the system.

\emph{Status 24 February 2002} -- \ac{NVODS} staff negotiated with the
\ac{AGU} for a booth at the Ocean Sciences Meeting held in Honolulu, HI
11-15 February 2002.

\item \ac{NVODS} capabilities must be demonstrated through pilot projects,

\emph{Status 24 February 2002} -- See Section~\ref{PILOTS}

\item \ac{NVODS} must entrain organizations, particularly at the national
and regional levels, to be aware of and to take advantage of the \ac{NVODS}
capabilities for moving data and data products to users who need them.
To this end, the \ac{NVODS} project should have representatives at \ac{GOOS}
and other national meetings.

\emph{Status 24 February 2002} -- A 2-day meeting was held with Navy
personnel responsible for \ac{TEDS}\htmlnote{http://www.nrlmry.navy.mil/~lande/TEDS/}.
\end{enumerate}

\item The \ac{NVODS} project must enhance the software and metadata 
capabilities it has available to add critical components needed 
to fulfill the \ac{NVODS} vision.  These include:

\begin{enumerate}
\item Completing the highest priority items in the prioritized software
list.  

\emph{Status 24 February 2002} -- See Appendix~\ref{PRIORITIZED-LIST}.

\item Partner with an organization/agency to define a controlled
vocabulary for the \ac{NVODS} project.

\emph{Status 24 February 2002} -- No action.

\item \ac{NVODS} should have a site for consistent use metadata for
\ac{NVODS} datasets.

\emph{Status 24 February 2002} -- No action, waiting for the development
of the \ac{AIS}.

\item \ac{NVODS} should establish a supporting partnership with an
organization/agency to set up a data discovery mechanism for \ac{NVODS}
datasets.

\emph{Status 24 February 2002} -- The \ac{NVODS} project office has
worked closely with the \ac{GCMD} to coordinate the lists of OPeNDAP$^2$
accessible datasets. Approximately 2/3 of the known datasets are now
searchable through the \ac{GCMD}.

\item The \ac{GIS} capabilities need to be developed as openly as possible.

\emph{Status 24 February 2002} -- The approach taken toward the 
\ac{OPeNDAP}--ARCview interface has been to modularize it as much
as possible so that some of the code can be reused with other \ac{GIS}
vendors. 

\item The \ac{NVODS} software developers must provide a set of clear
server installation guidelines.

\emph{Status 24 February 2002} -- No action.

\item The \ac{NVODS} installation package should include a browser that
can be used to access an arbitrary dataset, and a data registration tool
that will help the data providers identify their user communities and
provide a means to let those communities know that the new data are
available on the \ac{NVODS} network.

\emph{Status 24 February 2002} -- No action.

\item The \ac{NVODS} project needs to consider how to address the needs
of users who require access to real-time and near real-time data, and
that UNIDATA be asked to give a demonstration of their ``push'' capabilities
applied to an oceanographic dataset.

\emph{Status 24 February 2002} -- No action.

\item The \ac{NVODS} project should design the metrics gathering
capabilities with the \ac{LAS} project to take advantage of the \ac{LAS}
experience with end users.

\emph{Status 24 February 2002} -- No action.
\end{enumerate}

\item The \ac{NVODS} project should make some management decisions which
will enhance the project's ability to fulfill the \ac{NVODS} vision.
\begin{itemize}
\item Consider reducing the number of platforms supported within the
project to allow more resources to be directed toward meeting the goals
of the project.

\emph{Status 24 February 2002} -- It is not clear which platforms to
eliminate. No action.

\item The \ac{NVODS} project should adopt an ``open source'' structure
for software development, and encourage outside users to develop and contribute
their own software to interface with the existing \ac{NVODS} infrastructure,
thereby increasing and expanding that infrastructure.  Such contributions
should be recognized and publicized as part of the \ac{NVODS} project.

\emph{Status 24 February 2002} -- The first draft of a document that 
details how one writes to the \ac{DAP} has been written. Work has also
begun on facilitating the build process, in particular which compliers
are used.
\end{itemize}

\end{itemize}

\T\newpage
\section{Additional Software Capabilities -- Prioritized Task List}
\label{PRIORITIZED-LIST}

The following is a partially prioritized list\footnote{An updated
version of this list is available at: 
\xlink{http://mail.po.gso.uri.edu/tracking/nvods/priorities.html}{http://mail.po.gso.uri.edu/tracking/nvods/priorities.html}.} 
of software capabilities beyond those which already exist in 
\ac{NVODS} that are required to meet the 
recommendations of the workshops and the \ac{ExCom}. Items through
\#25 are prioritized. Item \#26 and following are of lower priority
than the first 25, but are not ordered relative to one another. When
the first 25 items have been addressed the remaining issues will be
revisited and addressed as appropriate. In addition to listing the 
items here, their status as of 24 February 2002 is provided. 

\begin{enumerate}
\item \emph{\ac{AIS}} -- The primary problem currently faced in the project 
is the lack of a consistent set of what we refer to as \emph{translational 
use metadata}\footnote{See \xlink{http://www.unidata.ucar.edu/packages/dods/archive/proposals/nopp-html/nopp\_10.html}{http://www.unidata.ucar.edu/packages/dods/archive/proposals/nopp-html/nopp\_10.html} 
for a discussion of different metadata types.}, those metadata needed for 
Level 3 interoperability, machine-to-machine interoperability with semantic
meaning. 

\emph{Status 24 February 2002} -- This issue has been discussed extensively
on the \ac{DODS} e-mail list. A design for this system component will be
completed and the system implemented in the May-June 2002 time frame.
 
\item \emph{GIS Client} -- This refers to the need for \ac{GIS} access
to data sets accessible via \ac{NVODS}. 

\emph{Status 24 February 2002} -- A subcontract has been let to \ac{ESRI}
to develop an \ac{NVODS} client for \ac{ESRI} products.

\item \emph{.HTML page (a top level directory page at the server)} -- Over
the course of the project we have found that we can not rely on data providers
to advertise the datasets that they are serving with OPeNDAP$^2$ servers.
This item refers to automatic generation of a web page at the top of the
htdocs directory that will advertise the existence of \ac{OPeNDAP} 
servers. The data provider will be able to disable the installation of
this web page.

\emph{Status 24 February 2002} -- A preliminary discussion was held at
the \ac{DODS}-Tech meeting. This item will be combined with item \#10
below.

\item \emph{File-out Tool:  DataDODS - output to - \acs{netCDF}, \acs{HDF},
MATLAB, \acs{IDL}, etc.}
-- Although the primary goal of \ac{NVODS} is for data accessible in the
system to be accessible directly from the user's application this may not
always be practical or desirable. The \emph{file-out} capability is 
envisioned as a server-side service that will provide the desired data
in one of a number of specified formats. 

\emph{Status 24 February 2002} -- a \acs{netCDF} file-out capability is
in beta test.

\item \emph{Generic Display} -- Simple plots of any \ac{NVODS} accessible
datasets should be possible from a web browser, \emph{even} if there is
insufficient metadata for the plotting program to plot the data in a
semantically meaningful context. 

\emph{Status 24 February 2002} -- No action.

\item \emph{Adopt a Controlled Vocabulary} -- One of the fields that is 
important in translational use metadata is the mapping from the names of
variables used in the dataset to names that have more geophysical meaning.
Unfortunately, we do not yet have a list of acceptable variable names.
This task involves either adopting one or if need be developing one.

\emph{Status 24 February 2002} -- No action.

\item \emph{Build Metrics-gathering into Servers} -- Knowledge of OPeNDAP$^2$
data accesses is important for the project. Much of this information is
currently available in the \acs{httpd} logs of systems maintaining 
OPeNDAP$^2$ servers. This information is however tedious to collect and
parse and is not as complete as we would like. 

\emph{Status 24 February 2002} -- No action.

\item \emph{Data Registration Tool} -- A simple mechanism should exist for 
those willing to list their OPeNDAP$^2$-accessible datasets at one of
the \ac{NVODS} directory sites.

\emph{Status 24 February 2002} -- A basic tool has been built and is available
on the web page that lists all know OPeNDAP$^2$-accessible datasets. Click
on the ``Click here to submit a dataset'' button at the top of the datasets
page:

\noindent http://www.unidata.ucar.edu/cgi-bin/dods/datasets/datasets.cgi?
\newline xmlfilename=datasets.xml

\item {A \acs{GUI} Geospatial/temporal Selection Capability} -- For most 
datasets accessible via the system, the lack of a consistent set of 
translational use metadata make it difficult to impossible to develop a 
general tool that will allow users to graphically (or even with a
form) select subsets of the desired datasets. There are however several
clients (\ac{LAS} and the Matlab \ac{GUI}) that do provide such a capability
for a subset of the datasets accessible via the system. In both cases the
needed metadata is stored at the client. With the advent of the \ac{AIS}
it should be possible to generalize the approach taken in these interfaces
to a more general interface that would work for a wide range of datasets
in the system.

\emph{Status 24 February 2002} -- No action.

\item \emph{DODS-dir} -- This refers to the ability to traverse the
directory structure on systems with OPeNDAP$^2$ servers searching
for datasets accessible to \ac{NVODS} on these systems.  

\emph{Status 24 February 2002} -- A preliminary version of this has been
released and is accessible at most OPeNDAP$^2$ server sites. The version
that currently exists is however not robust to non-OPeNDAP$^2$ datasets
in these directories nor does it differentiate between files in the
directories and file-servers, inventories of OPeNDAP$^2$ accessible
datasets.

\item \emph{Secure Server Software for Datasets} -- An issue of concern
is the security of the \acs{CGI} scripts that are part of the 
OPeNDAP$^2$ servers.

\emph{Status 24 February 2002} -- Done, OPeNDAP$^2$ servers have been
evaluated by two different groups for potential holes and none were found.

\item \emph{\ac{XML} version of the \ac{DAP}} -- With the rapid adoption of
\ac{XML} as a web based encoding scheme for text, it has become clear that
the project would benefit from at the very least encoding the \ac{dds} and
\ac{das} in \ac{XML}.

\emph{Status 24 February 2002} -- Separate funding has been obtained by
\ac{OPeNDAP} to explore this. Work on this will be completed this 
Federal fiscal year.

\item \emph{Enable serving \ac{CODAR} Data} -- This refers to a pilot project
more than to any specific software required by the core. The idea is
to develop a web interface to \ac{CODAR} data from a variety of sites.

\emph{Status 24 February 2002} -- See Section~\ref{CODAR-PILOT}.

\item \emph{Geospatial-Temporal Display of NVODS Datasets} -- The fact
that all datasets accessible via OPeNDAP$^2$ in \ac{NVODS} have a
common structural representation makes it possible to automatically
manipulate these in client-side applications. For those that are
semantically consistent, fusion or overplotting is possible.

\emph{Status 24 February 2002} -- \ac{LAS} provides this functionality
for a number of \ac{NVODS} accessible data.

\item \emph{\acs{FGDC} Reader into the \ac{das}} -- Many datasets have been 
described via the \acs{FGDC} clearinghouse. Unfortunately there is currently
no way to link these descriptions with the datasets themselves via the
\ac{das}.

\emph{Status 24 February 2002} -- No action.

\item \emph{Servers to Support Individual Investigators} -- Two data servers
have been built for \ac{PI} datasets: FreeForm and \acs{JGOFS}. Neither of
these works well for certain classes of ``simple'' datasets. 

\emph{Status 24 February 2002} -- No action.

\item \emph{\ac{GIS}: Toolbox} -- Building \acs{URL}s can be difficult.
Tools that build \acs{URL}s for some of the more popular subsets may be
needed.

\emph{Status 24 February 2002} -- No action. We believe that this is
ultimately coupled with item~\#1.

\item \emph{Client Toolbox, including GUI-ettes} -- In addition to \acs{URL}
builders, returned data may require transformation to be in a consistent
set of geophysical units or variable names may require transformation. 
Routines that perform these functions for existing clients: MATLAB, IDL,
Ferret, etc. would make the data significantly more useful.

\emph{Status 24 February 2002} -- No action. We believe that this is
ultimately coupled with item~\#1.

\item \emph{Web Crawler} -- This task would be better termed \emph{site
crawler} in that it refers to an inventory of all datasets at a given
site once the site is known.

\emph{Status 24 February 2002} -- A prototype site crawler has been developed.
This crawler does not however differentiate between files in a dataset and
datasets. No action has been taken on addressing the latter.

\item \emph{Multi-URL Protocol}

\emph{Status 24 February 2002} -- 

\item \emph{Provide DODS-dir with Dataset list}  -- 

\emph{Status 24 February 2002} -- Done.

\item \emph{Grid() Server-side function}  -- Datasets that consist of
arrays with map vectors should be subsettable using the maps rather than
only by array subscript.

\emph{Status 24 February 2002} -- This server-side function has been developed
and is available in release 3.2. The function is however simple in that it 
does not know about axes that repeat such as longitude.

\item \emph{Control Access to DODS-served Datasets}  -- Password protection.

\emph{Status 24 February 2002} -- Is available in release 3.2.

\item \emph{Design NVODS to handle: maps, graphs, images and animations}
-- The \ac{ExCom} felt that this should be done via related \acs{URL}s
rather than modifying the \ac{DAP}. This is seen as an NVODS issue probably 
to be handled by judicious use of placement of metadata 
images/maps/graphs/annimations.

\emph{Status 24 February 2002} -- No action.

\item \emph{Concatenation of Modulo Requests} -- Maps of arrays that
repeat are special cases of map vectors. Making this information 
available to clients would be useful

\emph{Status 24 February 2002} -- No action. It is assumed that this
information is in the \ac{das} (or can be added to the \ac{das}) or
is obvious such as in the case of longitude when the geophysical 
meaning of the independent variable is clear.

THE FOLLOWING ITEMS ARE NOT RANKED WITHIN THEIR PRIORITIES:
(Items 26-31 are second priority, Items after 31 are third priority.)

\emph{Status 24 February 2002} -- No Action.

\item Server of GIS data into the NVODS system.

\emph{Status 24 February 2002} -- No Action.

\item Provide Unique Dataset Tags.

\emph{Status 24 February 2002} -- No Action.

\item Allow for more projection types on the Earth's Surface.

\emph{Status 24 February 2002} -- No Action.

\item Fileserver Interface

\emph{Status 24 February 2002} -- No Action.

\item Point Aggregation Server for non-gridded data. 

\emph{Status 24 February 2002} -- No Action.

\item Interface for standardized representations  of independent variables 

\emph{Status 24 February 2002} -- No Action.

\item Support a swath data type (NGrid?).

\emph{Status 24 February 2002} -- Modifications to the \emph{grid} data
type may support swath data directly.

\item Display a Disclaimer on access to certain datasets. 

\emph{Status 24 February 2002} -- No Action.

\item Enable Push serving data into NVODS. 

\emph{Status 24 February 2002} -- No Action.

\item BUFR Server

\emph{Status 24 February 2002} -- No Action.

\item Develop an ESML-based server. (See: Independent Investigator Server)

\emph{Status 24 February 2002} -- No Action.

\item Develop an NVODS Variable-Name Controlled Vocabulary. 

\emph{Status 24 February 2002} -- No Action.

\item Mirror Servers

\emph{Status 24 February 2002} -- No Action.

\item IDL GUI

\emph{Status 24 February 2002} -- No Action.

\item MATLAB GUI

\emph{Status 24 February 2002} -- An updated MATLAB GUI is currently
available in (beta) test form and is scheduled for full release by
15 March 2002.

\item HDF API -- Client Library

\emph{Status 24 February 2002} -- No Action.

\item Develop Depth interfaces Similar to Time

\emph{Status 24 February 2002} -- No Action.

\item Macintosh OS-X

\emph{Status 24 February 2002} -- No Action.

\item CORBA Gateway

\emph{Status 24 February 2002} -- No Action.

\item Data Quality Specification

\emph{Status 24 February 2002} -- No Action.

\end{enumerate}

\T\newpage
\section{Formation of \ac{OPeNDAP}}
\label{OPENDAP}

In response to the \ac{ExCom} recommendation for the formation of a 
separate entity to handle the \ac{DAP} used by \ac{NVODS}, 
a non-profit corporation with 501~(c)(3) 
status called \acl{OPeNDAP} has been formed. The goal of this corporation 
is to develop and promote software that facilitates access to data via 
the network. The \ac{OPeNDAP} Data Access Protocol, OPeNDAP$^2$, forms 
the core of the \ac{OPeNDAP} effort. The \ac{DODS} \ac{DAP} will become 
OPeNDAP$^2$. \ac{OPeNDAP} is not affiliated with a particular discipline 
data system such as \ac{NVODS}; it is envisioned as a generic data exchange 
mechanism that lies at the core of a variety of discipline data systems. 
The evolution of \ac{OPeNDAP} will be guided by input from a variety of 
discipline data systems. \ac{OPeNDAP} software will be open source. 
\ac{OPeNDAP} will seek funding independently of discipline data systems. 

\noindent Replacing \ac{DODS} with OPeNDAP$^2$ underlines the following:

\begin{itemize}
\item The discipline-independent nature of the \ac{DAP} $\Rightarrow$
\emph{oceanographic} has been removed from the name,
\item The focus on only one part of an end-to-end system, $\Rightarrow$,
\emph{system} has been removed from the name, and
\item The open source nature of the effort. 
\end{itemize}

\noindent \ac{OPeNDAP} should also facilitate the hiring of computer 
programmers and systems developers when compared with hiring in a 
university setting where hiring is now severely hampered by salary 
constraints.

A for-profit organization is envisioned to work in concert with
the non-profit consortium, to develop products from the NVODS capabilities 
that would be of use to specific user communities.  The combination of the
two corporations would provide the maximum continued expansion and 
development of the NVODS system.

\T\newpage
\section{\ac{DODS}-Tech Meeting }
\label{DODS-TECH}

A gathering of those involved in the development of the \ac{DODS} 
\ac{DAP}, associated clients and servers and data discovery tools,
took place in Boulder CO at \ac{UCAR} facilities on 9--11 January
2002. This meeting is referred to as the \ac{DODS}-Tech meeting.
The proceedings of this meeting will be available on CD-ROM from
Unidata shortly. Requests for the CD-ROM may be made from the 
\ac{DODS} home page. 

In addition to a number of interesting presentations, the meeting
also involved a session in which the following issues related to 
weaknesses in the current implementation of the system or areas for 
future development were raised:

\begin{itemize}
\item \emph{Inventory access for datasets consisting of a large number of
elements of different size.} -- This problem relates to datasets such
as profile data bases, ship cruises, mooring records, etc. There is
at present no consistent way of dealing with such data bases and there
exist a significant number of such datasets that are of great interest
to the oceanographic and meteorological communities. This is an area 
that will be pursued initially via the \ac{DODS} e-mail list and then 
possibly by a focus group.

\item \emph{\ac{XML} representations of the \ac{dds} and \ac{das}} -- 
It became increasingly clear at the workshop that at a minimum the
\ac{dds} and \ac{das} must be represented in \ac{XML}. This issue
will be pursued by the core group (and the \ac{PMEL} group) with 
NOAA funding.

\item \emph{Caching} -- Concern was raised with regard to caching
by the core. Beyond raising the issue, no decision was made on how
to proceed with this one.
 
\item \emph{Evolution of the core} -- Two issues were raised with
regard to the core and how it is implemented:

\begin{enumerate}
\item \emph{Performance} -- Although the \ac{DAP} as implemented
works well for small to moderate sized datasets, it may not be
as efficient as needed for very high volume transfers. \ac{HAO}
with funding from the Earth System Grid project will investigate
options for a more efficient encoding of the transport protocol
for high data rate (grid) transfers.

\item \emph{Modularity} -- A concern was raised about the monolithic
nature of the core which makes it hard to maintain and to modify.
Beyond raising the issue, no decision was made on how to proceed with 
this one.
\end{enumerate}

\item \emph{Services Discovery} -- As servers become more sophisticated,
the need to identify services provided by or at these servers, becomes
more pressing. This is an area that will be pursued initially via the 
\ac{DODS} e-mail list and then possibly by a focus group.

\item \emph{Synthetic Variables} -- This relates to variable that are
created from existing variables in a dataset by a server. It is in fact
a subset of the \emph{Services Discovery} in that the creation of new
variables can be viewed as a server-side service.

\item \emph{Support for the \ac{DRDS}} -- The server provides each
Table in the Database as a separate dataset. 
\end{itemize}

\T\newpage
\section{Current Status}
\label{STATUS}

While this section is intended to give the status of the \ac{NVODS} project
at the end of the first year (August 2001),  some additional progress is
included as of this writing (February 2002).  
        
\subsection{Current Data Providers}

Over 250 datasets from 24 sites already are served into the \ac{NVODS}
system.  The datasets range from \ac{GLOBEC} cruise data to mooring data
to large collections of satellite data which are continuously updated,
including sea surface temperature, height, and winds. For a list of the 
known datasets provided via \ac{DODS}, see the ``Datasets'' list on the 
\ac{DODS} web-site:

\begin{center}
http://unidata.ucar.edu/packages/dods/
\end{center}

\subsection{Data Discovery}

There exist several data discovery mechanisms within the \ac{NVODS}
effort:

\begin{itemize}

\item \emph{The Data Set List} -- This is a manually maintained 
web-site list
of \ac{DODS}-accessible data sets. The list is only searchable by
words in the data set name. It is organized by data provider.

\item \emph{The \ac{DODS} Portal of the \acl{GCMD}} -- This is a
specialized view of the data holdings in the \ac{GCMD}. It allows
the user to search for \ac{DODS}-accessible data sets meeting user 
defined criteria. Approximately two thirds of all known \ac{DODS}-accessible
data sets are described in the GCMD. The remainder lack the metadata 
required by the \ac{GCMD} for data discovery.

\item \emph{The Matlab \ac{GUI}} -- This Matlab \ac{GUI} provides
a graphically searchable list of data sets that are \ac{DODS}-accessible.
This list comprises only a small fraction of all \ac{DODS}-accessible
data sets.
\end{itemize}

\subsection{Data Servers}

Standard OPeNDAP$^2$ data servers exist for: \acs{netCDF}, \acs{HDF}-4,
\acs{HDF}-5, \acs{HDF}-\acs{EOS}, Matlab, DSP, \ac{JGOFS} and FreeForm.
In addition, an JDBC server exists for relational data bases.

Two specialized servers have also been developed: the \acl{AS} and
the \acl{GDS}. The \ac{AS} provides a single view of multi-file 
datasets of arrays. Three types of aggregation have been defined:
\begin{enumerate}

\item Aggregation of arrays where all of the arrays are projected
on the same n-dimensional space. Aggregation is on a dimension not
represented in the original n dimensions; the dimensionality of
the resulting array is increased by one.

\item Aggregation of arrays that represent different variables.
In this case m n-dimensional arrays, each representing a different
parameter are assembled in one dataset, but not into one array.

\item Aggregation of arrays for in which the aggregated dimension
already exists. In this case the dimensionality of the aggregated
array is the same as that of the input arrays, one of the dimensions
is simply larger, the sum across all files that are to be aggregated
of the elements in the dimension that is to be aggregated.
\end{enumerate}

The \ac{GDS} was developed by the \ac{COLA} group to facilitate 
internal data transfers within the project. At the same time, this
server provides in two areas added functionality for the \ac{NVODS} 
effort:

\begin{enumerate}
\item Because \ac{GrADS} reads \ac{GRIB} datasets, the \ac{GDS}
adds \ac{GRIB} to the data types that may be served via 
OPeNDAP$^2$.

\item The \ac{GDS} also allows for analysis at the server; any
\ac{GrADS} function may be applied to datasets accessible to
the \ac{GDS} prior to the data being delivered to the OPeNDAP$^2$
client.
\end{enumerate}

\subsection{Data Clients}

OPeNDAP$^2$ clients have been written for Matlab and IDL. In addition,
an OPeNDAP$^2$ client library has been written for the \Cpp\ 
implementation of \acs{netCDF}. This means in that any application
that makes use of the \Cpp\ \acs{netCDF} library may be relinked with
the OPeNDAP$^2$ \acs{netCDF} client and the application is then
OPeNDAP$^2$-enabled. Ferret and GrADS have been relinked this way.
The Java version of the \acs{netCDF} library has been
OPeNDAP$^2$-enabled at Unidata, the developers of \acs{netCDF}. This
means that any application that makes use of this library is
automatically an OPeNDAP$^2$ client.  Two such clients that we are
aware of are ncBrowse and VisAD.  Finally, Excel can import data via
\acs{http} if it is comma-delimited.  In that most OPeNDAP$^2$ servers
have associated with them an ascii service that provides
comma-delimited output, all of these OPeNDAP$^2$-accessible data sets
are also accessible to Excel.

\end{document}


%%% Local Variables: 
%%% mode: latex
%%% TeX-master: t
%%% End: 
