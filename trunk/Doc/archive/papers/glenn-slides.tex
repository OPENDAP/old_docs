
% Slides for a general presenation on DODS.
% From Glenn Flierl. 3/21/94

\nopagenumbers
\magnification=\magstep3
\tolerance=1000
\baselineskip 15 pt
\def\ni{\noindent}
\def\s#1{\medskip\noindent{\bf #1}}
\def\ss#1{\medskip\noindent{\it #1}}

\centerline{\bf Distributed Ocean Data System (DODS)}
\centerline{Sponsored by The Oceanography Society}

\s{Purpose:}

\item{$\cdot$} To develop a client--server system which will provide
researchers with a simple way to locate, acquire, and use ocean data
from a wide variety of sources.

\item{$\cdot$} To give researchers and data archives an easy way to
distribute their information.


\s{Process:}

\item{$\cdot$} Workshop series to define system and guide development

\itemitem{$\cdot$} \# 1: Alton Jones Campus, URI, 9/28/93--10/1/93.
Outline goals, architecture, communications protocols

\itemitem{$\cdot$} \# 2: 94. Presentation of prototype, requirements
for developing data servers

\itemitem{$\cdot$} \# 3: TOS, 95. Demonstration for oceanographic
community.

\item{$\cdot$} Systems programmer (James Gallagher) plus scientists
guiding development.

\vfill\eject

\centerline{\bf Approach: Data Access}

\s{Application Program Interfaces (API):}

User programs generally access data through some kind of
application program interface which is responsible for reading the
data files or querying the database and returning the results (usually
as numbers) to the main program. Such a subroutine library

\item{$\cdot$} provides the access to data

\item{$\cdot$} hides details of the data format

\item{$\cdot$} is formally defined

\item{$\cdot$} may specify data model

\s{Examples}

\item{$\cdot$}netCDF

\item{$\cdot$}JGOFS

\item{$\cdot$}hdf

\item{$\cdot$}MATLAB

\item{$\cdot$}stdio (open, read, close)

\vfill\eject

\s{Remote Procedure Calls (RPC):}

RPC's split the API between ``client stubs'' and ``server stubs.'' The
former package the inputs from the application code's calls and send
them over the network to the server stubs which execute a version of
the original calls to get the data. The results are then packaged and
returned to the client stub.  It, in turn, passes them back to the
application code.

\item{$\cdot$} permits components to be distributed

\item{$\cdot$} hides details of communications

\item{$\cdot$} can handle different machine formats for numbers (XDR)

\s{Simple Data Server:}

By working at the API level, applications need only be re--linked in
order to be able to use data over the network. Neither the application
code nor the data format needs to be changed. Calls for local and
remote access appear the same (only the ``file name'' is replaced by a
``universal resource locator''--like string giving the extra
information about what machine is serving the information, etc.)

\vfill\eject

\s{Translating Server:}

In some cases, the data to be served may not be accessible with the
chosen API (e.g., one could choose to serve data written as
unformatted records by a FORTRAN program as though it were netCDF). 

\vfill\eject

\s{Translators:}

In many cases, the application programs users are working with will
not have an API matching that expected by the server of the data. In
this case, an additional process will be interposed. This
``translator'' accepts calls from one API and builds the required
information by making calls to a second, different API --- that
required by the server.

\vfill\eject
\centerline{\bf DODS Data Model}

Often the data model for a particular data set is not specified in the
data set itself. Yet this kind of information is important in
determining what operations are supported by a server and in
translating from one API to another. DODS needs

\item{$\cdot$} A method for expressing the data model of a server

\item{$\cdot$} Procedure calls for obtaining that information.

\item{$\cdot$} Ways for the application to specify operations and for
the server to get this information.


\vfill\eject

\s{Scientific Data:}

\ss{Functions:} Much of our data can be expressed as vector functions
$$
\eqalign{
data=\vec f(\vec x)~~~~ &{\rm with}~~
f_1=u,~f_2=v,~f_3=T,...\cr
&x_1=lon,~x_2=lat,~x_3=press,~x_4=time,...\cr
}
$$
For these, we can subset by (a) projection [picking only particular
components of $\vec f$], (b) subsetting the domain [usually in terms
of comparisons and Boolean operations], (c) subsetting the range.

\ss{Grids:} A special subclass of functions are gridded data in which
$$
X=\vec x \otimes \vec y...~~~~ {\rm and}~~ data=\vec f(X,...)
$$
In some cases, the grid may be mapped to a distorted structure.
Congruent functions (e.g. hyperslab,...)transform gridded data into
another gridded set. Range operations and many domain operations will
not preserve the gridded character.

\vfill\eject
\ss{Sequences:} Here each point is a tuple of values
$$
data=\{\vec g_1,~ \vec g_2,...\}
$$
including both independent and dependent variables on the same
footing. While functions can be ``unravelled'' into sequences, the
reverse operation may not be possible. Operations include projection
and selection [subsetting by value]. There are no distinctions between
independent and dependent variables. A ``relation'' is a sequence with
the order being immaterial; in much of our data, however, the ordering
does have implicit meaning.

\ss{Data Set:} A data set is a collection of related elements of the
types described above.

\ss{Aggregation:} Often data can be usefully aggregated so that
repeated information need not be transferred.

\ss{Unions:} Different forms may be used for the same
fundamental quantity, most notably time. The data model needs to
transmit the information that universal time, or dd/mm/yy or yymmdd
can be used interchangeably.

\vfill\eject
\s{Data Model Information:}

\item{$\cdot$} Variable Information
\itemitem{$\cdot$} Names
\itemitem{$\cdot$} Types (real, string,...)
\itemitem{$\cdot$} Attributes
\item{$\cdot$} Functional Dependences
\item{$\cdot$} Grid Definitions
\item{$\cdot$} Aggregation Information

\s{Operations:}

With these definitions, we should be able to request that the server
perform projections, selections [Boolean combinations of comparisons,
regular expressions] on the data. In the case of gridded data, we may
have more complex hyperslab operations. Since many of the API's do not
explicitly include such operations, DODS will specify them as part of
the URL (object name). 

\vfill\eject
\s{Virtual File System:}

In some cases, possible selections can be presented to the users as a
directory structure, e.g.

{\tt
\ni\quad cruise-AII

\ni\quad\quad station01

\ni\quad\quad\quad cast1

\ni\quad\quad\quad cast2

\ni\quad\quad\quad cast3

\ni\quad\quad\quad cast4

\ni\quad\quad station02

\ni\quad\quad\quad cast1

\ni\quad\quad\quad cast2

\ni\quad...
}

\ni so that selecting {\tt /cruise-AII/station02/cast2} would actually
be doing a database operation 

{\tt cruise==AII\&station==02\&cast==2}.

\ni The pseudodirectory provides a convenient, familiar, and intuitive
organizational structure; in addition, it should be possible to
provide the equivalent of symbolic links, equating a short name to a
complex URL.


\bye
\item{$\cdot$} 
\item{$\cdot$} 
\item{$\cdot$} 
\item{$\cdot$} 
\item{$\cdot$} 
