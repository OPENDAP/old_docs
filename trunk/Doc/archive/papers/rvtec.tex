
% Paper for submission to TOS. This could also be the beginings of some user
% documentation or a `welcome' page in our HTML doc tree...
%
% I think this actually appeared in a Journal called RV Tech. and was written
% by George Milkowski and Peter Cornillon. 2/5/99 jhrg
%
% $Id$


\documentstyle[12pt,psfig,margins]{article}

\psfigurepath{tos1-figs}
\topmargin -8ex
\footskip 4ex
\headheight 8ex
\footheight -4ex
\textwidth 6.25in
\textheight 8.25in
\oddsidemargin 0em
%\parskip 1ex
\def \dods{Distributed Oceanographic Data System }
\def \dodsns{Distributed Oceanographic Data System}
\def \1em{\hskip 1em}
\def \2em{\hskip 2em}
\def \3em{\hskip 3em}
\def \4em{\hskip 4em}
\def \dap{data access protocol }
\def \dapns{data access protocol}

\begin{document}
\bibliographystyle{chicago}
\Large
\centerline {\bf DODS: Providing direct access }
\centerline {\bf to distributed research data resources.}
\normalsize

\large
\bigskip
\noindent
{\bf Introduction}
\medskip
\normalsize 

It doesn't take long using the World Wide Web (WWW) to discover many hundreds
of scientific datasets that are currently available on-line to researchers.
For oceanographers the volume and diversity of data from national archives,
special program offices or other researchers that is of potential value to
their research is overwhelming.  However, while a large number of oceanographic
datasets are on-line, from the research oceanographer's point of view there
are significant impediments which often make acquiring and using these
on-line data anything but easy.  What are the problems?

To begin with, the storage format of data is many times system specific
making it difficult to view or combine datasets from different systems.
Also, a majority of data archives have developed their own data management
systems with specialized interfaces for navigating their data resources.
With the notable exception of NASA's EOSDIS Version 0 Information Management
System\footnote{NASA's EOSDIS V0 IMS provides a single interface to 8 NASA
  earth science data centers and the CIESIN social science data center}, none
of these data systems interoperate with each other, making it necessary for a
user to visit many systems and `learn' multiple interfaces in order to
acquire data.  Finally, even after the data has been successfully transfered
to the researcher's local system in order to {\em use} the data a researcher
must convert that data into the format that his or her data analysis
application requires or alternatively modify and recompile the analysis
application's source code.

To address these distributed data access problems, researchers at the
University of Rhode Island and the Massachusetts Institute of Technology are
creating a network tool that, while taking advantage of WWW data resources,
resolves the issues of multiple data formats and different data systems
interfaces.  This network tool, called the Distributed Oceanographic Data
System or DODS, enables oceanographers to interactively access distributed,
on-line science data using the one interface that a researcher is already
most familiar with, his or her own personal data analysis application.  The
architecture and design of DODS will make it possible for a researcher to
open, read, subsample and import directly into his or her data analysis
applications scientific data resources over the WWW.  The researcher will not
need to know either what format is used to store the data or how the data is
actually accessed and served by the remote data system.

\large
\bigskip
\noindent
{\bf DODS: An extension to existing interfaces}
\medskip
\normalsize 

\noindent The Distributed Oceanographic Data System is not a data management
system, nor is it used for creating data base schema and inventories.
Rather, DODS is a specification for directly accessing, representing and
transferring data on a network (the Internet).  It is a {\em data access
  protocol} which defines both a common functional interface to data systems
and a data model for representing data on those systems.  It is designed to
be integrated {\em with\/} already existing user applications and resource
management systems, not to replace them.  DODS extends the operational
capabilities of existing systems in two important ways; first, through its
network interface it transforms a user's application into an agent of a
distributed client-server system and second, it provides a system
independent, integrated view of their resources.

For transforming a user's analysis application into network client DODS
includes client libraries which when linked with the application provide the
client-server network interface and necessary client-side operations
supporting client-server communications.  

The key to providing an integrated view of data resources is the DODS data
model.  The DODS data model defines a set of data types and constructs which
are used to represent the content and structure of data files (analogous to
data types that are used in programming languages such as C and Fortran).
All data transfered by a server to a DODS client is first translated into a
canonical representation using the DODS data model.  When the client receives
the data it then translates the DODS canonical representation to the data
representation that is valid for the application.  For any new data format
to be supported by DODS two translators need to be created.  One that
translates from the data format to the DODS data model and one to translate
from the DODS data model to the format.  The benefit of this approach is that
newly created client DODS-to-application translator can immediately access
any DODS data resources regardless of the format used by the remote system
for storage of the data.  Likewise, a new server format-to-DODS translator
will immediately support access by already existing DODS client applications. 


Systems wishing to use DODS as a means of providing access to their data
resources and/or in accessing other's resources through DODS do so through
the implementation of the DODS client-server libraries.  The server libraries
enable data systems developers to install servers which: 
  \begin{itemize}
    \item Utilize httpd as the network server for client-server interactions.
    \item Define a DODS network server interface to the remote system that
      then allows DODS client applications to query and access a server's data
      resources.
    \item Provide detailed information on the structure and attributes of a
      server's data resources.
    \item Translate data resources utilizing the DODS data model into a DODS
      canonical representation for transfer. 
  \end{itemize}

%

DODS compliant user applications use client libraries developed for specific
common Application Programmers Interfaces or APIs\footnote{ An API is itself
  a software tool that is typically a discrete set of procedures or commands
  which provide a formal method of access to a computational resources such
  as a disk file, a data management system or a device such as a display
  screen}.  For a DODS compliant API the software libraries provide:
  \begin{itemize}
    \item A DODS surrogate library which maps API procedure calls to DODS API
      procedure calls.
    \item Means for encoding of user query for transmission to DODS servers 
    \item Customized translators for decoding from the DODS canonical
      representation of data to API specific data format.
  \end{itemize}

% DODS Architecture Figure

 
\large
\bigskip
\noindent
{\bf DODS: A Means of Accessing PI Data}
\medskip
\normalsize 

One of the motivating factors leading to the development of DODS was the
desire to access researcher held datasets as well as the the data resource of
large data centers.  The large data systems being developed by agencies
within the government were previously seen as focusing only on the data
resources of those agencies and having no integrated plan to include datasets
of individual researchers.  Datasets that are considered scientifically
valuable to the research community.  The model for these large data systems
was that researchers were data users and the agencies data providers.  Within
DODS researchers are seen as both data providers and data users.  Therefore,
designing a system which provides an effective option for researchers to
serve as well as access data is one of the primary goals of DODS.




\large
\bigskip
\noindent
{\bf Status of DODS Development Effort}
\medskip
\normalsize

A beta version of a NetCDF/DODS client-server libraries is now available.  In
the very near future JGOFS/DODS libraries will also be distributed.  The
client libraries are easily linked with existing applications without the
need to modify application source code (i.e., the API procedure call names
that are used by an application remain the same in DODS, relinking the
application with the DODS client libraries causes the procedure calls to
reference the DODS surrogate procedures).  DODS server libraries are
installed behind a system's httpd server with a minimum of effort.  Binaries
for DEC/ULTRIX, DEC-ALPHA/OSF, SUN/OS, SUN/SOLARIS, SGI/IRIX and IBM/AIX will
be available.  Detailed technical information, how to get binaries or source
code for the client-server libraries and papers documenting the development
of DODS are available via the WWW at {\em http://dcz.gso.uri.edu/}.


The \dods is being developed to help address the need within the
oceanographic community for researchers to easily access and serve data
resources over the Internet.  DODS developers are creating a data access
protocol based on a client-server architecture in order to take advantage of
and promote the availability of on-line research data.  Two unique aspects of
the DODS system design are that it transforms researchers' data analysis
applications into client agents within DODS and that it provides a system 
independent view of the data resources available through its servers.  Rather
than trying to replace or build a system from scratch, DODS is purposefully
designed to be integrated with already existing data systems and user
applications, taking advantage of and extending those systems' capabilities.

\end{document}
