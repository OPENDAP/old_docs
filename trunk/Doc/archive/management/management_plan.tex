\documentclass{article}

\begin{document}

\title{Management of DODS}
\maketitle

\section*{Project Goal}

The primary goal of the DODS project is to provide an infrastructure
for interoperability at the data level that \emph{will be used} 
by the oceanographic community for their data exchange needs.

A secondary objective is to become the data interoperability layer 
for the NASA ESIP Federation.

\begin{quote}
  To meet both objectives, we have to extend our servers so that 
  they provide `ease of use' services in addition to `ease of 
  access'. Simultaneously, we will develop a simple data location 
  tool to which others can add value and a comprehensive suite of 
  client programs. 

  In addition, to meet the primary objective, we will have to
  seed the system with a suite of general interest oceanographic
  data sets and build a cadre of users who will either entrain
  others into using the system and/or become data providers in
  the future.
\end{quote}

\section*{Management}
\subsection*{Strategic Management}
The strategic position of the project will be reviewed by the steering
committee\footnote{Historically this has been called the Gang of Four.}
(Glenn Fleirl, Peter Cornillon, James Gallagher, and Dan Holloway with
Richard Chinman \emph{ex officio}) every 6 months at an in-person meeting.
  
\subsection*{Project Lead}
Overall organization and direction will be provided by Peter Cornillon. His
responsibilities are:
\begin{enumerate}

  \item Establish the overall direction of the project and evaluate our 
  progress toward it. 

  \begin{enumerate}
    \item Use DODS. Problems found should be fixed, but deficiencies cannot
    necessarily be addressed right away. Resist the urge to adopt quick fixes.

    \item Evaluate new tasks in light of our goal(s). Does the proposed thing
    really further our objective(s) or does it just look cool? The evaluation
    must balance the research nature of our group (which excels at developing
    prototypes of new ideas) with the long-term needs of developing a stable
    system. 
  \end{enumerate}

  \item Identify and pursue funding for the continuation of the project. 

  \item Define the oceanographic (and possible meteorological) data sets
  that we will pursue to achieve a critical mass in the seeding of the 
  system.

  \item Same for non-oceanographic Federation data sets. In order for
  DODS to become the interoperable layer, a number of Federation data 
  sets will have to be served via DODS. Peter will identify which ones
  to pursue.

  \item Work within the oceanographic community to promote the use of
  DODS and its population.

  \item Work with the non-oceanographic community to promote the use of
  DODS and its population --- As we have seen, when other groups adopt
  DODS, it increases our visibility which facilitates the acquisition 
  of future funding and it brings new resources to the project in the
  form of developed tools and of ideas. 

\end{enumerate}
  
\subsection*{Project Manager}
Management for DODS will be provided by Richard Chinman. His responsibilities
are: 
\begin{enumerate}

\item Coordinate communication within the DODS project

  \begin{itemize}
  \item Organize telephone conferences
  \item Organize semi-annual developer's meetings
  \end{itemize}

\item Track the progress on project tasks

  \begin{itemize}
  \item Organize and maintain a list of tasks (web, project)
  \item Organize meetings (telecons and in-person).
  \end{itemize}

\item Track expenditures of funds received in a gross sense

  \begin{itemize}
  \item Encumber expenditures for subcontracts and salaries
  \item Identify the rate of spending in general: Do we have extra \$s for x?
  \end{itemize}

\item Monitor system usability, especially WRT UIs

  \begin{itemize}
  \item Is the software and documentation easy to get?
  \item Are obvious flaws reported?
  \end{itemize}

\item Interact with the Federation

  \begin{itemize}
  \item Metrics
  \item Clusters - ours and others
  \item Meetings
  \end{itemize}
        
\item Organize our participation at AGU meetings - the face that we
  present at these meetings (posters, booth, handouts, demos)

\item Monitor compliance with funding

  \begin{itemize}
  \item Track the milestones in the proposals
  \item Submit reports, etc. to funding agencies
  \end{itemize}

\end{enumerate}

\subsection*{Technical Management}
  DODS will divide itself into two main groups: The Core, or system, software
  group and the Clients and Data group. The core group will be lead by James
  Gallagher and the Clients and Data group will be lead by Dan Holloway.
  
  The overall technical coordination of the project will be James'
  responsibility. This means choosing which goals should be addressed by
  software and, of those, how they fit in with the rest of the system. In
  addition, this will include choosing when to support new technology.
  
  Dan will be responsible for overseeing design and implementation of those
  parts of the project that fall under the Client and Data umbrella.
  Similarly, James will oversee those parts of the project that are part of
  the system's infrastructure (which we have called the `core').  Their
  responsibilities will include assigning people to work on, and scheduling
  of, various parts of the project. However, the Client/Data and Core parts of
  the project will not be two completely separate entities; there will be
  considerable crosstalk between the two groups.
  
  When there's a question as to where a particular thing falls, Dan and James
  will assign it to one of the two groups, even though the assignment might
  be arbitrary.

  The two main groups will each meet via teleconference every two weeks.
  
  Some people will be in both the Core and Clients and Data groups.

\subsubsection*{Technical Groups}
Each member of the the project staff is part of either the UI/Data or Core
groups, or both. This grouping simplifies coordination of tasks within the
project; it is not meant to limit what any staff member does or the parts of
the project they work on. In Table~\ref{table:staff} the current (\today)
associations within the staff are shown. Rob Morris will work on the PC
port until the end of the year at which time he will move to the IDL 
GUI hence his membership in both groups. Ethan Davis' functions consist
of builds, metrics and user services. The first two of these are Core
functions, the last one associates him with the UI/Data group. Tom
Sgouros will document the core as well as server installation.

\begin{table}
\caption{Staff group associations within the DODS project}
\label{table:staff}
\begin{center}
\begin{tabular}{|l|c|c|} \hline
Staff           & UI/Data & Core \\ \hline
Brent Baker     &       & *     \\           
Deirdre Byrne   & *     &       \\
Ethan Davis     & *     & *     \\
James Gallagher &       & Lead  \\
Jose Garcia     &       & *     \\
Paul Hemenway   & *     &       \\
Dan Holloway    & Lead  &       \\
Kwoklin Lee     & *     &       \\
Rob Morris      & *     & *     \\
Chris Mulhearn  &       & *     \\
Reza Nekovei    &       & *     \\
Ruth Platner    & *     &       \\
Nathan Potter   &       & *     \\
Ton Sgouros     & *     & *     \\ 
Pippin Wolfe    & *     &       \\ \hline
\end{tabular}
\end{center}
\end{table}

\section*{Policy}
\subsection*{Project direction}
The project direction will be defined by a steering committee which will
consist of the Project Lead, Technical Leads and one or more advisors with
the Project Manager \emph{ex officio}. The members of the steering committee
must spelled out in the Management section of this document.

\begin{itemize}
\item This must be an in-person meeting
\item Duration: about a day
\end{itemize}

\subsection*{Other groups}
When interacting with other groups, solidify the relation with a Memorandum
of Understanding (MOU).
\begin{itemize}
\item Always involve both technical and management in a MOU
\item Consider the impacts on other efforts
\end{itemize}

\subsection*{Time estimates}
Make realistic time assessments of activities.
\begin{itemize}
\item The people doing work should provide an estimate of the time required.
\begin{quote}
  Before we start a new task, there should be an estimate of the time it will
  require. This estimate should be made by the person(s) who will actually do
  the task. The estimate should also define exactly what will be included in
  the `first version' or what it means to `complete' this task. In the former
  case, what will/should be included in follow on versions should also be
  described, albeit more briefly than the initial version's description. If
  this to be an on-going process, there should be a description of the
  continued work needed.
\end{quote}

\item The estimate should be broken into small blocks of time
\begin{quote}  
  The estimate should break the problem into small things that will take
  no more than two weeks to do. If that's not possible (and in some cases it
  is really not) then we should understand that there is a high degree of
  variability in the estimate. That's OK, so long as we know and accept that.
\end{quote}

\item Tie the estimation to the work description
\begin{quote}  
  Breaking the estimate down into small pieces will help flesh out ideas
  which appear simple but are, in fact, quite complicated.
\end{quote}

\item Estimates should be written down
\begin{quote}  
  Time estimates are very hard to get correct, and this is exacerbated when
  the thing you're estimating has never been done before. However, by writing
  down an estimate so that it can be reviewed later, it is possible to build
  up skill in estimation and build up knowledge about the types of problems
  that appear over and over again in a particular project. Put another way,
  you can learn estimation. It is important to make sure people feel
  comfortable (or minimally uncomfortable) with this since it can feel as if
  people are being graded.
\end{quote}
\end{itemize}

\end{document}
