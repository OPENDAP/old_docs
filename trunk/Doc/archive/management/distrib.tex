
% This document describes how source code distributions will be
% made within DODS. By documenting this procedure we hope that others
% that contribute to DODS will find it simpler to do so in a way that
% 'gels' with the rest of DODS.
%       
% jhrg 4/28/95

% $Id$

\documentstyle[12pt,html,psfig,margins]{article}

%
% These are html links which are used often enough in writing about DODS to
% merit an input file.
% jhrg. 4/17/94
%
% File rationalized and updated while writing the DODS User
% Guide. Also includes other useful abbreviations.
% tomfool 3/15/96
%
% $Id$
%
% HTML references to DODS documents
%
% For most of the DODS documentation there are three html references defined:
% 1) The upper case \newcommand produces the title of the paper, an
%    html link in the online documentation and a footnote in the paper
%    version.  
% 2) A capitalized version produces a capitalized description of the
%    paper and a link in the online version (but no footnote in the
%    paper version). 
% 3) A lower case version which produces a lower case description of
%    the paper and a html link, but no footnote.

% External references to documents:
%
% In order for external references to work (via latex2html) the labels used
% throughout the documents must be unique. The text labels are all prefixed
% with a document identifier (e.g., ddd) with a colon separater. The
% \externalrefs below (way below...) includes the perl files html.sty uses to
% make the cross refs.

\newcommand{\DODS}{\htmladdnormallink{Distributed Oceanographic Data System}
  {http://www.unidata.ucar.edu/packages/dods/}}

\newcommand{\Dods}{\htmladdnormallink{DODS}{http://www.unidata.ucar.edu/packages/dods/}}

\newcommand{\wrkshp}{\htmladdnormallink{Report on the First Workshop for the
    Distributed Oceanographic Data System, Proposed System Architectures}
  {http://www.unidata.ucar.edu/packages/dods/archive/reports/workshop1/section3.13.html}}

% not DODS URLs, but useful all the same...

\newcommand{\www}{\htmladdnormallink{World Wide Web} 
  {http://www.w3.org/hypertext/WWW/TheProject.html}}
\newcommand{\url}{\htmladdnormallink{Uniform Resource Locator}
  {http://www.w3.org/hypertext/WWW/Addressing/URL/url-spec.html}}

\newcommand{\uri}{\htmladdnormallinkfoot{Uniform Resource Identifiers}
  {http://www.w3.org/hypertext/WWW/Addressing/URL/uri-spec.html}}

\newcommand{\urn}{\htmladdnormallinkfoot{Uniform Resource Name}
  {http://www.acl.lanl.gov/URI/archive/uri-archive.messages/1143.html}}

\newcommand{\HTTPD}{\htmladdnormallinkfoot{HTTPD}
  {http://hoohoo.ncsa.uiuc.edu/docs/Overview.html}}

\newcommand{\HTTP}{\htmladdnormallinkfoot{HyperText Transfer Protocol}
  {http://www.w3.org/hypertext/WWW/Protocols/HTTP/HTTP2.html}}

%\newcommand{\HTML}{\htmladdnormallinkfoot{HyperText Markup Language}
%  {http://www.w3.org/hypertext/WWW/MarkUp/MarkUp.html}}

% CERN used to be `Conseil Europeen pour la Recherche Nucleaire'
\newcommand{\CERN}
  {\htmladdnormallink{European Laboratory for Particle Physics}
  {http://www.cern.ch/}}

\newcommand{\NCSA}
  {\htmladdnormallink{National Center for Supercomputing Applications}
  {http://www.ncsa.uiuc.edu/}}

\newcommand{\WWWC}{\htmladdnormallink{World Widw Web Consortium}
  {http://www.w3.org/}}

\newcommand{\CGI}{\htmladdnormallinkfoot{Common Gateway Interface}
  {http://hoohoo.ncsa.uiuc.edu/cgi/overview.html}}

\newcommand{\MIME}{\htmladdnormallink{Multipurpose Internet Mail Extensions}
  {http://www.cis.ohio-state.edu/htbin/rfc/rfc1590.html}}

\newcommand{\netcdf}{\htmladdnormallink{NetCDF}
  {http://www.unidata.ucar.edu/packages/netcdf/guide.txn_toc.html}}

\newcommand{\JGOFS}{\htmladdnormallink{Joint Geophysical Ocean Flux Study}
  {http://www1.whoi.edu/jgofs.html}}

\newcommand{\jgofs}{\htmladdnormallink{JGOFS}
  {http://www1.whoi.edu/jgofs.html}}

\newcommand{\hdf}{\htmladdnormallink{HDF}
  {http://www.ncsa.uiuc.edu/SDG/Software/HDF/HDFIntro.html}}

\newcommand{\Cpp}{{\rm {\small C}\raise.5ex\hbox{\footnotesize ++}}}

% Commands

\newcommand{\pslink}[1]{\small
\begin{quote}
  A \htmladdnormallink{postscript}{#1} version of this document is
  available.  You may also use \htmladdnormallink{anonymous
  ftp}{ftp://ftp.unidata.ucar.edu/pub/dods/ps-docs/} to access postscript files
  of all of the DODS documentation.
\end{quote}
\normalsize
}

\newcommand{\declaration}[1]{\small
{\tt {#1}}
\normalsize
}

% All the links from here down are for the white papers. I'm not sure that any
% of these are still valid given the reorganization of the web site. 5/12/98
% jhrg.

\newcommand{\DDA}{\htmladdnormallinkfoot{DODS---Data Delivery Architecture}
  {http://www.unidata.ucar.edu/packages/dods/archive/design/data-delivery-arch/data-delivery-arch.html}}
\newcommand{\Dda}{\htmladdnormallink{Data Delivery Architecture}
  {http://www.unidata.ucar.edu/packages/dods/archive/design/data-delivery-arch/data-delivery-arch.html}}
\newcommand{\dda}{\htmladdnormallink{data delivery architecture}
  {http://www.unidata.ucar.edu/packages/dods/archive/design/data-delivery-arch/data-delivery-arch.html}}

\newcommand{\DDD}{\htmladdnormallinkfoot{DODS---Data Delivery Design}
  {http://www.unidata.ucar.edu/packages/dods/archive/design/data-delivery-design/data-delivery-design.html}}
\newcommand{\Ddd}{\htmladdnormallink{Data Delivery Design}
  {http://www.unidata.ucar.edu/packages/dods/archive/design/data-delivery-design/data-delivery-design.html}}
\newcommand{\ddd}{\htmladdnormallink{data delivery design}
  {http://www.unidata.ucar.edu/packages/dods/archive/design/data-delivery-design/data-delivery-design.html}}

\newcommand{\URL}{\htmladdnormallinkfoot{DODS---Uniform Resource Locator}
  {http://www.unidata.ucar.edu/packages/dods/archive/design/urls/urls.html}}
\newcommand{\Url}{\htmladdnormallink{Uniform Resource Locator}
  {http://www.unidata.ucar.edu/packages/dods/archive/design/urls/urls.html}}
\newcommand{\dodsurl}{\htmladdnormallink{uniform resource locator}
  {http://www.unidata.ucar.edu/packages/dods/archive/design/urls/urls.html}}

\newcommand{\DAP}{\htmladdnormallinkfoot{DODS---Data Access Protocol}
  {http://www.unidata.ucar.edu/packages/dods/archive/design/api/api.html}}
\newcommand{\Dap}{\htmladdnormallink{Data Access Protocol}
  {http://www.unidata.ucar.edu/packages/dods/archive/design/api/api.html}}
\newcommand{\dap}{\htmladdnormallink{data access protocol}
  {http://www.unidata.ucar.edu/packages/dods/archive/design/api/api.html}}

\newcommand{\SOFT}
        {\htmladdnormallinkfoot{DODS---Software Development Environment}
  {http://www.unidata.ucar.edu/packages/dods/archive/managment/software/software.html}}
\newcommand{\Soft}{\htmladdnormallink{Software Development Environment}
  {http://www.unidata.ucar.edu/packages/dods/archive/managment/software/software.html}}
\newcommand{\soft}{\htmladdnormallinkfoot{software development environment}
  {http://www.unidata.ucar.edu/packages/dods/archive/managment/software/software.html}}

% I used to call the DAP the API, then for a few days I called it the
% DTP. These def's were used then. Rather than find everywhere they are used
% (not that hard, but...) I just hacked the macros. NB: the source file for
% the DAP document is still called `api.tex'

\newcommand{\API}{\htmladdnormallinkfoot{DODS---Data Access Protocol}
  {http://www.unidata.ucar.edu/packages/dods/archive/design/api/api.html}}
\newcommand{\api}{\htmladdnormallink{data access protocol}
  {http://www.unidata.ucar.edu/packages/dods/archive/design/api/api.html}}

\newcommand{\DTP}{\htmladdnormallinkfoot{DODS---Data Access Protocol}
  {http://www.unidata.ucar.edu/packages/dods/archive/design/api/api.html}}
\newcommand{\dtp}{\htmladdnormallink{data access protocol}
  {http://www.unidata.ucar.edu/packages/dods/archive/design/api/api.html}}

\newcommand{\TOOLKIT}{\htmladdnormallinkfoot{DODS---Client and Server Toolkit}
  {http://www.unidata.ucar.edu/packages/dods/archive/implementation/toolkits/toolkits.html}}
\newcommand{\Toolkit}{\htmladdnormallink{Client and Server Toolkit}
  {http://www.unidata.ucar.edu/packages/dods/archive/implementation/toolkits/toolkits.html}}
\newcommand{\toolkit}{\htmladdnormallink{client and server toolkit}
  {http://www.unidata.ucar.edu/packages/dods/archive/implementation/toolkits/toolkits.html}}

\newcommand{\TKR}{\htmladdnormallinkfoot{DODS---DAP Toolkit Reference}
  {http://www.unidata.ucar.edu/packages/dods/archive/implementation/toolkits/toolkits.html}}
\newcommand{\Tkr}{\htmladdnormallink{DAP Toolkit Reference}
  {http://www.unidata.ucar.edu/packages/dods/archive/implementation/toolkits/toolkits.html}}
\newcommand{\tkr}{\htmladdnormallink{DAP toolkit reference}
  {http://www.unidata.ucar.edu/packages/dods/archive/implementation/toolkits/toolkits.html}}

% I know that these are wrong. The papers have been moved to the gso web
% site, but I'm not sure exactly where. 5/12/98 jhrg

\newcommand{\DM}{\htmladdnormallinkfoot{DODS---Data Model}
  {http://lake.mit.edu/dods-dir/dodsdm2.html}}
\newcommand{\Dm}{\htmladdnormallink{Data Model}
  {http://lake.mit.edu/dods-dir/dodsdm2.html}}
\newcommand{\dm}{\htmladdnormallink{data model}
  {http://lake.mit.edu/dods-dir/dodsdm2.html}}

\newcommand{\DD}{\htmladdnormallinkfoot{DODS---Data Delivery}
  {http://lake.mit.edu/dods-dir/dods-dd.html}}

% external refs for DODS documents

\externallabels{http://www.unidata.ucar.edu/packages/dods/design/api}
  {/home/www/packages/dods/archive/design/api/labels.pl}
\externallabels{http://www.unidata.ucar.edu/packages/dods/design/data-delivery-arch}
  {/home/www/packages/dods/archive/design/data-delivery-arch/labels.pl}
\externallabels{http://www.unidata.ucar.edu/packages/dods/design/data-delivery-design}
  {/home/www/packages/dods/archive/design/data-delivery-design/labels.pl}
\externallabels{http://www.unidata.ucar.edu/packages/dods/implementation/toolkits}
  {/home/www/packages/dods/archive/implementation/toolkits/labels.pl}

%\renewcommand{\externalref}[2]{#2}

%%% Local Variables: 
%%% mode: latex
%%% TeX-master: t
%%% End: 


\begin{document}

\title{Producing Software Distributions for DODS}
\author{}
\date{\today}

\maketitle

\begin{abstract}
 
  This document describes the parts of a software `distribution' within DODS
  and provides information on using various freeware tools to help produce
  them. By documenting these parts and suggesting how they might be made we
  hope to unify the DODS software developed at URI with DODS software
  developed at other places so that all DODS software distributions will work
  well together. There are five parts to a DODS software distribution: The
  source code, Pre-compiled binary versions for select computer/operating
  system combinations, documentation, tests and a problem reporting
  mechanism. Each of these parts of a distribution are discussed in this
  paper.

\end{abstract}


% Biolerplate text warning that anything can (and almost certainly will)
% change. 
%
% jhrg 3/25/94
%
% $Id$

\small

\centerline{{\bf Note}} 

\begin{quotation}

\begin{htmlonly}
  This is a working document made available to solicit comments. To receive
  e-mail notice of new or significantly changed documents, send e-mail
  to \htmladdnormallink{majordomo@unidata.ucar.edu}
  {mailto:majordomo@unidata.ucar.edu} with 'subscribe dods' in the
  body of the message.
\end{htmlonly}

\begin{latexonly}
  This is a working document made available to solicit comments. To receive
  e-mail notice of new or significantly changed documents, send e-mail
  to majordomo@unidata.ucar.edu with 'subscribe dods' in the body of
  the message. 

  Hypertext references in the hardcopy version of this document are
  represented using footnotes that contain a Universal Resource Locator (URL)
  to the referenced document.  To read the online documentation, open the URL
  `http://www.unidata.ucar.edu/' with a World Wide Web browser.
\end{latexonly}

\end{quotation}

\normalsize

%
% Biolerplate text expaining that this document is intended for programmers.
%
% jhrg 4/17/94
%
% $Id$

\small

\begin{quotation}
  This document is written for people interested in implementing software
  that will be part of, or work in parallel with, the Distributed
  Oceanographic Data System. It is also intended for people who will manage
  those developers. It assumes familiarity with the development of small to
  medium sized software systems.
\end{quotation}

\normalsize

\begin{htmlonly}
\pslink{file://dcz.gso.uri.edu/pub/DODS/distrib.ps}
\end{htmlonly}

\clearpage

\tableofcontents

\clearpage

\section{Introduction}
\label{introduction}

\section{Documentation}

Each distribution of DODS software must have accompanying
documentation which describes how the distribution fits in with the
larger body of DODS software, how it is installed and how it is used.
The description of the relation of a particular distribution to the
rest of DODS and the information about installing the distribution
should be in available in ASCII text. Conventionally, this type of
documentation is stored in files named README and INSTALL; DODS
distributions should use this convention.

A single README and INSTALL file may suffice for an entire
distribution. However, if a distribution is organized into several
directories within the source code repository, then it might be
clearer to provide documentation for each of those directories.

The file `README' should contain information describing how the software in
the particular distribution relates to the other software in DODS. The file
`INSTALL' should contain information describing how to compile, link and
install the software.

Describing how a particular piece of software is used requires much
more effort than the simple documentation provided by the README and
INSTALL files. This type of documentation must address the use of
software from a conceptual point of view. It is not sufficient to list
command line options or GUI buttons and menus and explain what they
do. Instead, user documentation must organize the use of a particular
piece of software by tasks that the user is likely to want to do and
then explain how to accomplish those tasks.

It is best if user documentation is formatted using something more
elaborate than ASCII text. In particular, it is important for user
documentation to be available in hardcopy and on-line forms. At URI we
use \LaTeX\ since it is straightforward to translate most documents
into both PostScript and HTML. Thus, a single set of documentation
sources can provide both hardcopy and on-line documents. User
documentation should be distributed in both source form (whatever is
necessary to make the document that the user reads) as well as
something simple for most people to print such as
PostScript\footnote{PostScript files produced on the Macintosh are
notoriously hard for people with generic PS printers to print - avoid
this if at all possible}.

\subsection{Required Documentation}

The minimum documentation for any distribution are the ASCII text
files README and INSTALL; one each for the entire distribution.

User documentation is not required for a distribution, but its absence
will dramatically limit the usefulness of the software. This type of
documentation is optional only because it is very time consuming to
produce. By making it optional, software developers will hopefully be
able to make DODS software available more quickly and can then
subsequently update the distribution by adding documentation written
during the beta test phase. 

\section{Source Code}

All DODS distributions must contain source code. That is, there are no
binary-only distributions in DODS. This is important because DODS is a
community effort and thus must be completely accessible to all users.

The DODS software is divided into two parts. The first part is the
core software which contains the toolkits used to build translators
and servers for various data systems and the network I/O toolkit. The
second part is a collection of rebuilt libraries and data servers for
various data systems and APIs which are built using the core software.
A user of DODS who wants to get the source code distribution should
not have to get a different version of the core software for each API
they want to use with DODS. Instead, the core software should be
available separately from each of the specific API or data system
implementations. 

In order to manage building DODS source code on a number of platforms,
the GNU autoconf system will be used. Each distribution should come
with a single `configure' script which will correctly configure that
distribution.

Third party libraries should be provided as-is from the original
source. DODS distributions should provide a version (whichever one the
distribution used) as well as a pointer to the original source. Thus
users may choose to get a newer version or use the exact version DODS
uses. In some cases users may already have the third party software
and the source code distributions should accommodate this as well.
Third party libraries that are used by DODS should not become a part
of the DODS distribution unless there is good reason to include them
in the DODS source code. For example, much of the DODS core software
uses libg++ --- the GNU C++ class library. The library itself is not
part of the core distribution, however.

In some cases it is simpler to include the source code to a particular
library within DODS. For example, most of the WWW library is included
in the DODS netio library. This was done to simplify building the DODS
library because few systems already have the library and it is not
widely used. On the other hand, most users of g++ (the GNU C++
compiler) will already have libg++.

\subsection{Source Code Requirements}

All source code must be available.

Third party software should be available, but as an extra item; users
must be able to use their own copy or the original developers copy
instead of the DODS-supplied version.

Some third party code may be integrated into the DODS core software
(but not the API or data systems software).

\section{Binary Distributions}
\label{binary-dist}

All binaries (executables and libraries) built from source code must also be
available for the {\em supported\/} platforms in binary form so that users of
those platforms do not need to compile the source code. The binary
distributions will mirror the source code distributions so that users can get
one copy of the core software and then get copies of new client libraries and
servers which make use of that core software. For example, if a users wants
the DODS version of netCDF, they can get the DODS core software and DODS
netCDF binary distributions. Later if that same user wants the DODS JGOFS
software, they need only get the JGOFS specific distribution---the single
copy of the core software initially obtained is sufficient.

\subsection{Binary Distribution Requirements}

DODS will support, according the the requirements contained in the \wrkshp\
and suggestions from the DODS developer community, the following platforms:

\begin{itemize}

\item IBM RS600 running A/IX 
\item Dec Alpha running OSF/1
\item Dec Decstation running Ultrix 
\item SGI running IRIX
\item Sun SPARC running SunOS 4.1.x
\item Sun SPARC running Solaris 2.x

\end{itemize}

While not part of the original requirements, support for the Sun SPARC running
Solaris has been added at the suggestion of members of the DODS development
community. In addition, given that DEC no longer supports Ultrix, support for
that platform may be stopped.

\section{Problem Reports}

Each DODS distribution must have a mechanism in place to handle problem
reports. It is not sufficient to rely on email alone for problem reporting.
The developers at URI will need to manage reports of problems with several
different distributions and do so in a way that reduces the chance that
reported problems are never investigated. In addition, it is very likely that
developers outside of those at URI will need to forward some problem reports
they receive to URI.

The DODS software developed at URI will use the GNU gnats (yes I know its a
funny name \ldots) problem reporting system which maintains a simple database
of problem reports and provides a form which users can fill out when
submitting a report. The software uses email to actually send in the report;
for a system which already requires the Internet, this is a very
non-intrusive mechanism. Several of the GNU software distributions already
use this mechanism, so users may already be familiar with it. In any case,
few users should have trouble with gnats; it is far simple than the simplest
email program.

\section{Testing}

Testing software is a hard job to do well. Tests are often as hard to write
as the code begin tested (which makes it easy to question the validity of
such an action in the first place) and determining which parts of a large
piece of software are covered by a given set of tests is very hard to do by
hand. 

However, for a project like DODS which necessarily involves several different
groups writing different components, it is imperative that software be
distributed with tests which other developers can also run. DODS will use GNU
DejaGNU to handle testing. When, for one reason or another, DejaGNU cannot
(or has not) been used, other test software will be included with the
component.

A potential disadvantage of using DejaGNU is that a separate package is
required to run the tests. This is in contrast to tests that are packaged in
a single executable which is built solely for testing. However, the DejaGNU
testing framework simplifies the construction of, and addition to, test
suites to a degree that warrants its use.

\subsection{Testing Requirements}

Each DODS software distribution must come with tests which can be run by
users. The tests can be either executable programs built specially for testing
purposes or suites of tests within the GNU DejaGNU test framework.

\section{Conclusion}

For each piece of software distributed by DODS there are four identifiable
components:

\begin{itemize}

\item Documentation
\item Source code
\item Binaries (ready-to-run software)
\item Problem reporting scheme
\item Tests

\end{itemize}

Because documentation is very hard to create, a DODS distribution can be
released before the documentation is complete. However, to be useful to a
most users each distribution must contain documentation.

\appendix

\section{Distribution Checklist}

\begin{enumerate}

\item Documentation

  \begin{itemize}
  \item README
  \item INSTALL
  \item NEWS
  \item User documentation
  \end{itemize}

\item Source Code
  
  \begin{itemize}
  \item configure.in
  \item Makefile.in
  \item configuration header
  \end{itemize}

\item Pre-Built Binary Software (see Section~\ref{binary-dist})

\item Problem Reporting Software

\item Tests

\end{enumerate}

\section{The GNU `Library' Copyright Notice}

The following is a sample of the copyright notice that appears at the top of
the DODS core software libraries. It references the GNU library copyright,
which is included in the source code and binary directories. Software that is
part of DODS does not have to be copyrighted this way (or any way), but using
the GNU copyright saves developers the time and effort of writing one of
their own as well as increasing the chances that the copyright will actually
do what is intended. This boiler plate text is included as a suggestion, not
a requirement.


\begin{verbatim}
//      The DODS Data Access Protocol (DAP) Library
//      Copyright (C) 1994, 1995 The University of Rhode Island, Graduate
//      School of Oceanography and The Massachusetts Institute of Technology
//     
//      This library is free software; you can redistribute it and/or
//      modify it under the terms of the GNU Library General Public
//      License as published by the Free Software Foundation; either
//      version 2 of the License, or (at your option) any later version.
//     
//      This library is distributed in the hope that it will be useful,
//      but WITHOUT ANY WARRANTY; without even the implied warranty of
//      MERCHANTABILITY or FITNESS FOR A PARTICULAR PURPOSE.  See the GNU
//      Library General Public License for more details.
//     
//      You should have received a copy of the GNU Library General Public
//      License along with this library; if not, write to the
//      Free Software Foundation, Inc., 675 Mass Ave, Cambridge,
//      MA 02139, USA.
//
//      Authors: James Gallagher (jimg@dcz.gso.uri.edu), 
//               Dan Holloway (dan@hollywood.gso.uri.edu), 
//               Reza Nekovei (reza@hadaf.gso.uri.edu)
//
//               The University of Rhode Island
//               South Ferry Rd.
//               Narragansett, RI. 02882
//               U.S.A.
//
//               Glenn Flierl (glenn@lake.mit.edu)
//
//               The Massachusetts Institute of Technology
//               Cambride, MA. 02139
//               U.S.A.
\end{verbatim}

\end{document}