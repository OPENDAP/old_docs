
% This document describes what software will be used to build DODS and why.
% It also should lay out basic guidelins for software development.
%
% 3/25/94 jhrg
%
% $Id$

\documentstyle[12pt,psfig,html,margins]{article}

%
% These are html links which are used often enough in writing about DODS to
% merit an input file.
% jhrg. 4/17/94
%
% File rationalized and updated while writing the DODS User
% Guide. Also includes other useful abbreviations.
% tomfool 3/15/96
%
% $Id$
%
% HTML references to DODS documents
%
% For most of the DODS documentation there are three html references defined:
% 1) The upper case \newcommand produces the title of the paper, an
%    html link in the online documentation and a footnote in the paper
%    version.  
% 2) A capitalized version produces a capitalized description of the
%    paper and a link in the online version (but no footnote in the
%    paper version). 
% 3) A lower case version which produces a lower case description of
%    the paper and a html link, but no footnote.

% External references to documents:
%
% In order for external references to work (via latex2html) the labels used
% throughout the documents must be unique. The text labels are all prefixed
% with a document identifier (e.g., ddd) with a colon separater. The
% \externalrefs below (way below...) includes the perl files html.sty uses to
% make the cross refs.

\newcommand{\DODS}{\htmladdnormallink{Distributed Oceanographic Data System}
  {http://www.unidata.ucar.edu/packages/dods/}}

\newcommand{\Dods}{\htmladdnormallink{DODS}{http://www.unidata.ucar.edu/packages/dods/}}

\newcommand{\wrkshp}{\htmladdnormallink{Report on the First Workshop for the
    Distributed Oceanographic Data System, Proposed System Architectures}
  {http://www.unidata.ucar.edu/packages/dods/archive/reports/workshop1/section3.13.html}}

% not DODS URLs, but useful all the same...

\newcommand{\www}{\htmladdnormallink{World Wide Web} 
  {http://www.w3.org/hypertext/WWW/TheProject.html}}
\newcommand{\url}{\htmladdnormallink{Uniform Resource Locator}
  {http://www.w3.org/hypertext/WWW/Addressing/URL/url-spec.html}}

\newcommand{\uri}{\htmladdnormallinkfoot{Uniform Resource Identifiers}
  {http://www.w3.org/hypertext/WWW/Addressing/URL/uri-spec.html}}

\newcommand{\urn}{\htmladdnormallinkfoot{Uniform Resource Name}
  {http://www.acl.lanl.gov/URI/archive/uri-archive.messages/1143.html}}

\newcommand{\HTTPD}{\htmladdnormallinkfoot{HTTPD}
  {http://hoohoo.ncsa.uiuc.edu/docs/Overview.html}}

\newcommand{\HTTP}{\htmladdnormallinkfoot{HyperText Transfer Protocol}
  {http://www.w3.org/hypertext/WWW/Protocols/HTTP/HTTP2.html}}

%\newcommand{\HTML}{\htmladdnormallinkfoot{HyperText Markup Language}
%  {http://www.w3.org/hypertext/WWW/MarkUp/MarkUp.html}}

% CERN used to be `Conseil Europeen pour la Recherche Nucleaire'
\newcommand{\CERN}
  {\htmladdnormallink{European Laboratory for Particle Physics}
  {http://www.cern.ch/}}

\newcommand{\NCSA}
  {\htmladdnormallink{National Center for Supercomputing Applications}
  {http://www.ncsa.uiuc.edu/}}

\newcommand{\WWWC}{\htmladdnormallink{World Widw Web Consortium}
  {http://www.w3.org/}}

\newcommand{\CGI}{\htmladdnormallinkfoot{Common Gateway Interface}
  {http://hoohoo.ncsa.uiuc.edu/cgi/overview.html}}

\newcommand{\MIME}{\htmladdnormallink{Multipurpose Internet Mail Extensions}
  {http://www.cis.ohio-state.edu/htbin/rfc/rfc1590.html}}

\newcommand{\netcdf}{\htmladdnormallink{NetCDF}
  {http://www.unidata.ucar.edu/packages/netcdf/guide.txn_toc.html}}

\newcommand{\JGOFS}{\htmladdnormallink{Joint Geophysical Ocean Flux Study}
  {http://www1.whoi.edu/jgofs.html}}

\newcommand{\jgofs}{\htmladdnormallink{JGOFS}
  {http://www1.whoi.edu/jgofs.html}}

\newcommand{\hdf}{\htmladdnormallink{HDF}
  {http://www.ncsa.uiuc.edu/SDG/Software/HDF/HDFIntro.html}}

\newcommand{\Cpp}{{\rm {\small C}\raise.5ex\hbox{\footnotesize ++}}}

% Commands

\newcommand{\pslink}[1]{\small
\begin{quote}
  A \htmladdnormallink{postscript}{#1} version of this document is
  available.  You may also use \htmladdnormallink{anonymous
  ftp}{ftp://ftp.unidata.ucar.edu/pub/dods/ps-docs/} to access postscript files
  of all of the DODS documentation.
\end{quote}
\normalsize
}

\newcommand{\declaration}[1]{\small
{\tt {#1}}
\normalsize
}

% All the links from here down are for the white papers. I'm not sure that any
% of these are still valid given the reorganization of the web site. 5/12/98
% jhrg.

\newcommand{\DDA}{\htmladdnormallinkfoot{DODS---Data Delivery Architecture}
  {http://www.unidata.ucar.edu/packages/dods/archive/design/data-delivery-arch/data-delivery-arch.html}}
\newcommand{\Dda}{\htmladdnormallink{Data Delivery Architecture}
  {http://www.unidata.ucar.edu/packages/dods/archive/design/data-delivery-arch/data-delivery-arch.html}}
\newcommand{\dda}{\htmladdnormallink{data delivery architecture}
  {http://www.unidata.ucar.edu/packages/dods/archive/design/data-delivery-arch/data-delivery-arch.html}}

\newcommand{\DDD}{\htmladdnormallinkfoot{DODS---Data Delivery Design}
  {http://www.unidata.ucar.edu/packages/dods/archive/design/data-delivery-design/data-delivery-design.html}}
\newcommand{\Ddd}{\htmladdnormallink{Data Delivery Design}
  {http://www.unidata.ucar.edu/packages/dods/archive/design/data-delivery-design/data-delivery-design.html}}
\newcommand{\ddd}{\htmladdnormallink{data delivery design}
  {http://www.unidata.ucar.edu/packages/dods/archive/design/data-delivery-design/data-delivery-design.html}}

\newcommand{\URL}{\htmladdnormallinkfoot{DODS---Uniform Resource Locator}
  {http://www.unidata.ucar.edu/packages/dods/archive/design/urls/urls.html}}
\newcommand{\Url}{\htmladdnormallink{Uniform Resource Locator}
  {http://www.unidata.ucar.edu/packages/dods/archive/design/urls/urls.html}}
\newcommand{\dodsurl}{\htmladdnormallink{uniform resource locator}
  {http://www.unidata.ucar.edu/packages/dods/archive/design/urls/urls.html}}

\newcommand{\DAP}{\htmladdnormallinkfoot{DODS---Data Access Protocol}
  {http://www.unidata.ucar.edu/packages/dods/archive/design/api/api.html}}
\newcommand{\Dap}{\htmladdnormallink{Data Access Protocol}
  {http://www.unidata.ucar.edu/packages/dods/archive/design/api/api.html}}
\newcommand{\dap}{\htmladdnormallink{data access protocol}
  {http://www.unidata.ucar.edu/packages/dods/archive/design/api/api.html}}

\newcommand{\SOFT}
        {\htmladdnormallinkfoot{DODS---Software Development Environment}
  {http://www.unidata.ucar.edu/packages/dods/archive/managment/software/software.html}}
\newcommand{\Soft}{\htmladdnormallink{Software Development Environment}
  {http://www.unidata.ucar.edu/packages/dods/archive/managment/software/software.html}}
\newcommand{\soft}{\htmladdnormallinkfoot{software development environment}
  {http://www.unidata.ucar.edu/packages/dods/archive/managment/software/software.html}}

% I used to call the DAP the API, then for a few days I called it the
% DTP. These def's were used then. Rather than find everywhere they are used
% (not that hard, but...) I just hacked the macros. NB: the source file for
% the DAP document is still called `api.tex'

\newcommand{\API}{\htmladdnormallinkfoot{DODS---Data Access Protocol}
  {http://www.unidata.ucar.edu/packages/dods/archive/design/api/api.html}}
\newcommand{\api}{\htmladdnormallink{data access protocol}
  {http://www.unidata.ucar.edu/packages/dods/archive/design/api/api.html}}

\newcommand{\DTP}{\htmladdnormallinkfoot{DODS---Data Access Protocol}
  {http://www.unidata.ucar.edu/packages/dods/archive/design/api/api.html}}
\newcommand{\dtp}{\htmladdnormallink{data access protocol}
  {http://www.unidata.ucar.edu/packages/dods/archive/design/api/api.html}}

\newcommand{\TOOLKIT}{\htmladdnormallinkfoot{DODS---Client and Server Toolkit}
  {http://www.unidata.ucar.edu/packages/dods/archive/implementation/toolkits/toolkits.html}}
\newcommand{\Toolkit}{\htmladdnormallink{Client and Server Toolkit}
  {http://www.unidata.ucar.edu/packages/dods/archive/implementation/toolkits/toolkits.html}}
\newcommand{\toolkit}{\htmladdnormallink{client and server toolkit}
  {http://www.unidata.ucar.edu/packages/dods/archive/implementation/toolkits/toolkits.html}}

\newcommand{\TKR}{\htmladdnormallinkfoot{DODS---DAP Toolkit Reference}
  {http://www.unidata.ucar.edu/packages/dods/archive/implementation/toolkits/toolkits.html}}
\newcommand{\Tkr}{\htmladdnormallink{DAP Toolkit Reference}
  {http://www.unidata.ucar.edu/packages/dods/archive/implementation/toolkits/toolkits.html}}
\newcommand{\tkr}{\htmladdnormallink{DAP toolkit reference}
  {http://www.unidata.ucar.edu/packages/dods/archive/implementation/toolkits/toolkits.html}}

% I know that these are wrong. The papers have been moved to the gso web
% site, but I'm not sure exactly where. 5/12/98 jhrg

\newcommand{\DM}{\htmladdnormallinkfoot{DODS---Data Model}
  {http://lake.mit.edu/dods-dir/dodsdm2.html}}
\newcommand{\Dm}{\htmladdnormallink{Data Model}
  {http://lake.mit.edu/dods-dir/dodsdm2.html}}
\newcommand{\dm}{\htmladdnormallink{data model}
  {http://lake.mit.edu/dods-dir/dodsdm2.html}}

\newcommand{\DD}{\htmladdnormallinkfoot{DODS---Data Delivery}
  {http://lake.mit.edu/dods-dir/dods-dd.html}}

% external refs for DODS documents

\externallabels{http://www.unidata.ucar.edu/packages/dods/design/api}
  {/home/www/packages/dods/archive/design/api/labels.pl}
\externallabels{http://www.unidata.ucar.edu/packages/dods/design/data-delivery-arch}
  {/home/www/packages/dods/archive/design/data-delivery-arch/labels.pl}
\externallabels{http://www.unidata.ucar.edu/packages/dods/design/data-delivery-design}
  {/home/www/packages/dods/archive/design/data-delivery-design/labels.pl}
\externallabels{http://www.unidata.ucar.edu/packages/dods/implementation/toolkits}
  {/home/www/packages/dods/archive/implementation/toolkits/labels.pl}

%\renewcommand{\externalref}[2]{#2}

%%% Local Variables: 
%%% mode: latex
%%% TeX-master: t
%%% End: 


\begin{document}

\title{DODS---Software Development Guidelines}
\author{}
\date{\today}

\maketitle

\begin{abstract}
Work in progress.
\end{abstract}


% Biolerplate text warning that anything can (and almost certainly will)
% change. 
%
% jhrg 3/25/94
%
% $Id$

\small

\centerline{{\bf Note}} 

\begin{quotation}

\begin{htmlonly}
  This is a working document made available to solicit comments. To receive
  e-mail notice of new or significantly changed documents, send e-mail
  to \htmladdnormallink{majordomo@unidata.ucar.edu}
  {mailto:majordomo@unidata.ucar.edu} with 'subscribe dods' in the
  body of the message.
\end{htmlonly}

\begin{latexonly}
  This is a working document made available to solicit comments. To receive
  e-mail notice of new or significantly changed documents, send e-mail
  to majordomo@unidata.ucar.edu with 'subscribe dods' in the body of
  the message. 

  Hypertext references in the hardcopy version of this document are
  represented using footnotes that contain a Universal Resource Locator (URL)
  to the referenced document.  To read the online documentation, open the URL
  `http://www.unidata.ucar.edu/' with a World Wide Web browser.
\end{latexonly}

\end{quotation}

\normalsize

%
% Biolerplate text expaining that this document is intended for programmers.
%
% jhrg 4/17/94
%
% $Id$

\small

\begin{quotation}
  This document is written for people interested in implementing software
  that will be part of, or work in parallel with, the Distributed
  Oceanographic Data System. It is also intended for people who will manage
  those developers. It assumes familiarity with the development of small to
  medium sized software systems.
\end{quotation}

\normalsize


\tableofcontents

\newpage

\section{Implementation}

Each of the institutions which participate in this effort will develop
servers using a similar implementation plan. The first year of work will
focus on the development of the data servers and software tools that can be
used to build additional servers. While each of the servers developed during
this phase of the DODS effort are intended to provide access to a data set
that is of great importance to the oceanographic community, a second goal of
equal importance is the development of high quality software tools that will
ease server development in the future. Each of the servers developed during
year one should be written so that they are an example of applying the
DODS concept of distributed data to a particular type of data archive. Others
who have data to contribute, but who do not have the resources to build data
servers from scratch, should be able to use the products from year one to
significantly ease the burden of full participation in DODS.

\subsection{Coordination with DODS}

Coordination of the separate development groups has two goals: maximizing
each group's leverage of the other's efforts and to impose as little as
possible on the normal work patterns of each group. Because of the
geographical dispersion of the four groups, formal meetings will be kept to
an absolute minimum. However, some form of organized communication is
necessary.

Currently, DODS developers at MIT and URI have used both in-person meetings
and the Internet to coordinate their efforts. In-person meetings are most
useful to discuss complex highlevel design issues. The distribution of
documentation and software has been entirely coordinated using a central,
publically accessible, ftp archive, on-line documentation (accessible via
NCSA Mosaic), and mailing lists.

We envision that networked workstations which support email (for both
person-to-person and mailing lists), NCSA Mosaic (for documentation access)
and anonymous ftp (for access to software) will be sufficient for most of the
communication needs of the year one effort.  It is anticipated though
that the members would get together for a formal meeting once a year
in coordination with the DODS Workshops.  This would allow the data
servers developers to be involved with the DODS development efforts
and provide an opportunity for focused work as a group. 

\subsection{Data Server Development}

\begin{description}

\item{Design}

  Because of the DODS `loose collaboration' working model, there are no
  guidelines for design processes. The effort underway at MIT and URI follows
  a spiral development processes in which pieces are designed, developed and
  tested one after the other, with some overlap. However a site chooses to
  develop software, it is important that the designs be made available as
  soon as possible. Documentation should be accessible as HTML (for network
  access via NCSA Mosaic) and postscript (for hardcopy).

  The design paradigm for data servers is discussed at some length in
  \ddd_html.  Each server is built using an existing data model/format API
  and Sun Microsystem's RPC version 4.0. For every entry point in the target
  API, a RPC client and server stub is defined. The programmer then
  implements these stubs so that they correctly marshal and unmarshal each of
  the API's entry point's arguments and results. The goal is to produce a new
  library with exactly the same entry points as the API, but which makes a
  remote procedure call to the matched server so that remote data can be
  accessed.  This new library (called a `client library') should emulate the
  original API so that programs which were written without any knowledge of
  DODS can link with the library and access remote data.

\item{Code Development}

  General guidelines for the DODS software development effort are available
  in \wrkshp_html. Programs developed for DODS will be distributed
  as binary images for a selected set of architectures\footnote{Sun Sparc
    (SunOS 4.1.3 \& Solaris 5.x), Dec Alpha OSF-1.x, \dots} and thus {\em
    must\/} compile correctly on those computers. In addition, all software
  will be available in source code form. Both the binary and source code
  forms of DODS software will be available on the DODS anonymous ftp archive.
  Since both source and binary forms are available, the source code is not
  required to be the most general possible. However, every effort should be
  made to reduce or remove artificial barriers to building the source code.
  This includes a prohibition against source code that can only be compiled
  using commercially available development tools (i.e., users must be able to
  compile source code using freely available tools).

\item{Testing}

  Programs that are part of the DODS effort must be thoroughly tested at the
  site where they are written. Developers should expect that the only testing
  their code receives is the testing they do themselves.  Each program and
  each relevant source file that makes up a program must include code that
  will test the associated program of module. It is especially important that
  these tests be included in the source distributions so that researchers who
  use DODS know what has and has not been tested. In addition, bundling test
  code permits others to find and fix `testing bugs'.

\end{description}

General documentation plan.
     Each major component has an architecture, design and implmentation doc
     Each piece of software has documentation that explains testing,
     installation and use

Document the process of building source code:
     Use of make
     Use of autoconf, autoheader
     Use of dejagnu
     Use of gnats
     Languages (C, C++, flex, bison)
     Debugging tools (dbnew, dbmalloc, gdb, assert)

Every directory that contains an implementation should:
Each API client-lib and server should have a README that describes:
     Which files from the original API implmentation are required
     How to modify them to support LOCAL access
     Tell how to run the tests

A Makefile should include the following targets
     Build (all, which includes the test code)
     test
     install
     clean
     depend
     tar

A Makefile should not have any absolute paths
A Makefile should work on many platforms

Use autoconf
Use autoheader

Coding standards:
Code that uses malloc should support dbmalloc. e.g.:
#ifdef DBMALLOC
#include <stdlib.h>
#include <dbmalloc.h>
#endif

#ifdef DBMALLOC
static unsigned long  histid1, histid2, orig_size, current_size;
#endif

#ifdef DBMALLOC
    static int flag = 0;

    if (!flag) {
        orig_size = malloc_inuse(&histid1);
        flag = 1;
    }
#endif

#ifdef DBMALLOC
    current_size = malloc_inuse(&histid2);

    if( current_size != orig_size )
    {
        malloc_list(2,histid1,histid2);
    }
#endif  

\end{document}