The Asian Pacific Data Research Center (APDRC) is designed to provide
one-stop shopping of climate data for a wide variety of users.  The
APRDC is running OPeNDAP, EPIC, LAS, and Catalog/Aggregation (CAS) servers.
The main problems with serving in-situ data are the irregularities
of data formats, and the volume of data files.  Their solutions are
to store the data in NetCDF, and use the EPIC System developed by
NOAA/PMEL.  Gang gave an overview of the data that are served, and
examples of the acquisiton and analysis of served data using EPIC.
Then he outlined accessing data via OPeNDAP.  The problem with OPeNDAP
is that it is difficult to locate an individual data file, and only
one file is accessed by a single URL.  Their solution is to configure
OPeNDAP into EPIC by exporting an APDRC/OPeNDAP pointer file to
serve numerous in-situ data files.  Gang showed how to access data using
this modified system, and gave a MATLAB comparison between WOCE CTD
and GFDL model data.  

WISH LIST:  

\begin{itemize}
\item  Ability to access a sequence of in-situ data files
\item  Ability to aggregate a bunch of in-situ profiles.
\end{itemize}


% $Id$
%
% $Log: gangyuansummary.tex,v $
% Revision 1.2  2003/05/05 16:40:45  tom
% edited, LaTeX-ized
%
% Revision 1.1  2003/04/29 14:09:01  paul
% Starting OPeNDAP Technical Working Conference 2003 Report
%
%

%%% Local Variables: 
%%% mode: latex
%%% TeX-master: t
%%% End: 
