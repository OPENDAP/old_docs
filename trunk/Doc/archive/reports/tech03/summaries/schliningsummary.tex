Brian Schlining from \ac{MBARI} was unable to attend due to the blizzard.  
However, Kevin Gomes from \ac{MBARI} gave Brian's presentation: ``Using
OPeNDAP for Oceanographic Observatory Data Management at \ac{MBARI}.''

\ac{MBARI} is operating and expanding an ocean observatory that
collects, analyses, and disceminates a wide variety of oceanographic
data and data products under the umbrella of \ac{MOOS}.  The problem 
consists of several components:

\begin{itemize}
   \item  Large number of data sources
   \item  Large variety of data sources
   \item  Dynamic System:  data sources may appear and disappear
   \item  No standard data format:  data may be "instrument-native"
\end{itemize}

The solution:

\begin{itemize}
   \item Data providers send description of data before sending the data.
   \item Store the "raw" data,
   \item Where possible convert "raw" data to NetCDF,
   \item Support distriuted data sources.
\end{itemize}

An instrumental connector, \ac{PUCK}, was described which transmits data from
an instrument after the appropriate metadata for the instrument and date
are transmitted.  Based on the input from \ac{PUCK}, and similar
metadata, a data system, \ac{SSDS}, is being developed to allow a user to
store and access the data with the appropriate metadata.  The up-side
and down-side of  \ac{LAS} in the system were described.  Then the
use of HOOVES and \ac{JSP} uas described.   The current state of the
system and future expectations were mentioned. 

% $Id$
%
% $Log: schliningsummary.tex,v $
% Revision 1.2  2003/05/05 16:40:45  tom
% edited, LaTeX-ized
%
% Revision 1.1  2003/04/29 14:09:01  paul
% Starting OPeNDAP Technical Working Conference 2003 Report
%
%

%%% Local Variables: 
%%% mode: latex
%%% TeX-master: t
%%% End: 
