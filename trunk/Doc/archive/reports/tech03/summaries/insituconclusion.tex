An interesting, non-trivial, and critical challenge to \ac{OPeNDAP} developers
is storing and accessing in-situ data in a fashion that will allow
fast access to those data by selecting on a large variety of
conditions which cut across the stored indices and variables
in unforseen ways.  In-situ data are stored with varying structures
that are determined more by the database they are stored in (in some
cases) or by the needs or knowledge of the data provider, than by
any consideration for access by other, ``outside'' users.  As users
try to access in-situ data stored in \ac{OPeNDAP}-served datasets,
determining access times or even how to make appropriate requests
across different datasets for similar data have become problematic.
Therefore, an ad-hoc working group of interested parties has been
addressing the general problem of handling in-situ data within
\ac{OPeNDAP}.  The breakout was a result of some of those considerations
and is one step in a continuing process.

Conclusions:

\begin{enumerate}
 \item The field chosen for indexing is not always the most obvious or
advantageous for accessing particular subsets of data.  Searchable
header fields, usually identified ahead of time to allow caching
and extraction.  However, such identification is not consistent.

 \item The data model is at least adequate to handle general in-situ
data.  

 \item One possible extension of the architecture is to have data files
with a separate store/cache/database for header information.

 \item Having a server that can aggregate in-situ data is extremely
important.  Joe Wielgosz will consider developing a \ac{NetCDF}
in-situ aggregation capability.
\end{enumerate}

% $Id$
%
% $Log: insituconclusion.tex,v $
% Revision 1.2  2003/05/05 16:40:45  tom
% edited, LaTeX-ized
%
% Revision 1.1  2003/04/29 14:09:01  paul
% Starting OPeNDAP Technical Working Conference 2003 Report
%
%

%%% Local Variables: 
%%% mode: latex
%%% TeX-master: t
%%% End: 
