Yingshuo Shen presented the work of a consortium of Japanese and
Hawaiian investigators in which they performed some benchmark
tests on access times for data served locally vs the same data
served via the aggregation server at \ac{PMEL}.  The data used
were NCEP reanalysis daily air temperature and pressure from
1948 to 2002.  The server systems were described.  In general,
local data access were faster than remote accesses, although
in some cases the acess times were comparable.  Access to 
data locally served via OPeNDAP was significantly faster than
access to data remotely served via OPeNDAP.

Conclusions and wish list for the developers:

\begin{itemize}
\item Improve access speed to frequently used data while keeping the
  complete dataset in service and reducing the burden of storage.
  
\item The file access method does a better job than the OPeNDAP
  method.  But the file method is only limited to \ac{NetCDF} type.
  It would be nice if the aggregation server could also do the same
  job for \ac{HDF} files, or at least via the \ac{HDF}- \ac{OPeNDAP}
  Server.
\item Dynamic configuration - Frontier and \ac{APDRC} are considering
  improving the combination (local/remote or fast/slow) accesses by
  using dynamic configuration - a routine check of log file to adjust
  the data configuration.
\item Handle the Earth Simulator's model outputs (binary form) - also
  aggregating via OPeNDAP free format?
\item Note: this testing involves only 1-2 Mb data volume.  The result
  can be very different if the data volume reaches Giga- or Terabyte
  volumes.  More testing is needed.
\end{itemize}

From the discussion after the paper, the source of the difference
in access times was not known to result more from processing
times at the servers or internet access times.  Several
more detailed tests were indicated by the discussion.

% $Id$
%
% $Log: yingshuosummary.tex,v $
% Revision 1.2  2003/05/05 16:40:45  tom
% edited, LaTeX-ized
%
% Revision 1.1  2003/04/29 14:09:01  paul
% Starting OPeNDAP Technical Working Conference 2003 Report
%
%

%%% Local Variables: 
%%% mode: latex
%%% TeX-master: t
%%% End: 
