Since the minutes of the breakout session are somewhat confused, here
is a distillation of what we perceive to have been the main thrusts of
discussion. 

\begin{itemize}
\item The specification documents should contain detailed information about the
  interactions that take place within the protocol, as well as the
  cirrent details about the data model.  State diagrams would be
  useful to document how requests are processed and responses are generated.
  For example, when a request is received, at what point(s) can an error
  object be returned in place of the requested object?

\item Error delivery must be guaranteed. The biggest problem is that errors
  discovered while sending data are impossible to deliver using the current
  protocol. It may be that the organization of the data response document
  needs to be changed so that reliable error delivery is achieved.

\item To complete the separation of the DAP from transport protocols, we must
  specify the `headers' now used by requests and responses and make those
  part of the DAP, in some way. One way to do this is to define all requests
  and responses in XML and include tags for the information now contained in
  the headers. This would also serve as the basis for a lower level of
  abstraction where both requests and responses could be more completely
  defined, with an emphasis on making the request definition(s) on a par
  with the definitions of the responses.

\item All requests and responses in the DAP should be made using XML. The older
  interface, which uses HTTP GET in most implementations, can be supported
  without much effort as a special case of the more general XML forms.

\item Document each `service' separately and make www.opendap.org the official
  site/registry for these services and their documentation. Background:
  The current suite of planned documentation for version 4 contains two
  volumes: one for the data model, constraints and objects and one for the
  services. These volumes correspond roughly to the transport-independent and
  -dependent parts of the DAP. Really, the first document \emph{is} the DAP while
  the second is other stuff that's used by various implementations to make
  things like a web server that happens to be a DAP server, too, work the way
  people expect them to work.

\item Modify the DAP to better support server-side analysis. A client should be
  able to specify a sequence of processing steps, and this seems to be the
  place to add support for server-side analysis. A client should also be able
  to make requests for several data sets served by a single server.
  Furthermore, server directives should also be possible (such as `compress
  the response', etc.). If a request is sent to a server using XML, then
  these things should be easy to implement. It should also be easy to
  transform a request made in the current syntax (URL/HTTP-GET) into this new
  scheme.

\item Things we should express in XML (this is effectively a list of things that
  should be in the DAP and not a service): 

  \begin{itemize}
  \item Dataset structure, 
  \item Dataset attribute, 
  \item Data values (although this need not be the primary way to get
    values), 
  \item Sub-setting constraints, 
  \item Service identifiers, 
  \item Errors, 
  \item Service requests, 
  \item Behavioral directives, 
  \item Post-processing directives, 
  \item Deltas (e.g.  AIS).
  \end{itemize}
  
  Another way of thinking of this is to say that we should be able to
  request: names of datasets, variable names for a given dataset, data for
  variables and a server manifest. The lists are slightly different, but the
  basic concepts are the same. The differences also illustrate the degree of
  latitude present in developing a design. [The second list is from a
  conversation with Joe W after the breakout. jhrg]

\item Consider adding ragged arrays. Intention: Support Sequence-like variables
  which cannot be constrained. This would provide a way for a client to know
  that it cannot select using certain variables. This is an issue with some
  relational databases of profile data. [From a conversation with Joe W after
  the breakout. jhrg]

\item Add a Permission Object to the set of XML things that a server can
  manipulate. This would be used in conjunction with server-wide
  authentication (which could be implemented using some third-party solution)
  and could be used to limit access to specific resources within a server.
  [From a conversation with Joe W after the breakout. jhrg]


\item Other comments about the DAP:

  \begin{itemize}
  \item Eliminate the client-side validation of XML documents in the current Java
    prototype.
  \item Remove the * operator for URLs
  \item Add access by record number for Sequences
  \item Add selection by value for arrays
  \item Add grid selection using map vectors [This might be hard if the maps are
    added using the AIS. This is one instance of a general problem when
    variables are added to a relational data type. It might be better to
    build a front-end library that can perform these operations at the
    client. jhrg]
  \item Add some way to get data without having to get the global attributes for
    the dataset (as is the case with the prototype code developed in
    Java).
  \item Use/include a checksum for the XML documents.
  \end{itemize}
\end{itemize}
                                
% $Id$
%
% $Log: dapspecconclusion.tex,v $
% Revision 1.2  2003/05/05 16:40:45  tom
% edited, LaTeX-ized
%
% Revision 1.1  2003/04/29 14:09:01  paul
% Starting OPeNDAP Technical Working Conference 2003 Report
%
%

%%% Local Variables: 
%%% mode: latex
%%% TeX-master: t
%%% End: 
