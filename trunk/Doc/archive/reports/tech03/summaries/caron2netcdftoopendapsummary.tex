John Caron talked about mapping the Java-\ac{NetCDF} \ac{API} to \ac{OPeNDAP} Datasets.
He reviewed the Java-\ac{NetCDF} packages.  Then he reviewed the \ac{NetCDF}
and \ac{OPeNDAP} 3.0 data models, with particular emphasis on the differing
structural elements, and the application of attributes within the 
models.  He compared \ac{OPeNDAP} and \ac{NetCDF} primitives, and \ac{OPeNDAP}
types the \ac{NetCDF} Objects.  He described how structures are flattened,
and then showed the relationship among the ``DODS-NetCDF'' objects in
the extended \ac{NetCDF} \ac{API}.  He presented outstanding issues and possible
solutions in 4 areas:

\begin{enumerate}
   \item  Strings vs char [n],
   \item  Lack of globally scoped Dimension,
   \item  Performance,
   \item  No access to sequences through the standard \ac{API}.
\end{enumerate}

Subtle issues include:

\begin{itemize}
  \item  Mutability,
  \item  State,
  \item  Efficiency of an operation.
\end{itemize}

Conclusions:

\begin{itemize}
   \item Java-\ac{NetCDF} 2.1 ``extended \ac{API}'' has full access to \ac{OPeNDAP}
         Data.
   \item ``standard'' \ac{API} has access to everything except sequences,
         but some efficiency may be lost.
   \item The data model mismatch is relatively minor and may be
         fixed in DAP 4.0.
\end{itemize}

Finally he gave his view of the \ac{NetCDF} ``convention factory''
whereby a \ac{NetCDF} file is dumped in, and a standardized
\ac{NetCDF} Dataset results from the application of the appropriate
units for coordinate system identification and conversion, conforming
to any of a number of specified ``standard'' conventions.

% $Id$
%
% $Log: caron2netcdftoopendapsummary.tex,v $
% Revision 1.2  2003/05/05 16:40:45  tom
% edited, LaTeX-ized
%
% Revision 1.1  2003/04/29 14:09:01  paul
% Starting OPeNDAP Technical Working Conference 2003 Report
%
%

%%% Local Variables: 
%%% mode: latex
%%% TeX-master: t
%%% End: 
