James Gallagher reviewed the ongoing development of a new
specification of the Data Access Protocol (DAP 4.0).  The goals are:

\begin{enumerate}
  \item Separate different logical parts into different specifications,
  \item Adopt \ac{XML},
  \item Add a \ac{SOAP} interface, or something similar, to provide a
        web services interface for \ac{OPeNDAP} servers, 
  \item Facilitate implementations using different transport protocols,
  \item Provide a complete specification, not just a reference 
        implementation.
\end{enumerate}

The ultimate goal is to get people to \emph{use} the \ac{DAP}.
The \ac{DAP} 3.0 and earlier were intertwined with \ac{HTTP}.
In \ac{DAP} 4.0, the intent is to support all transport protocols
equally, e.g. \ac{GridFTP} and local file accesses, as well as 
\ac{HTTP}.  

An organization of the specification documents was proposed:

\begin{itemize}
  \item Overview
  \item Data model and Objects
  \item Services
\end{itemize}

The object representation was described.  While the \ac{DAP} 
architecture is fairly old, it is so close to XML that it seems
foolish not to adopt XML.  The advantages of an XML representation
were described, as well as the distinction between objects and
services.  The data model was described including the atomic and
constructor data types.  The four objects in the proposed \ac{DAP}
4.0 were listed:

\begin{enumerate}
  \item \ac{DDX}
  \item \ac{BLOB}
  \item ErrorX
  \item Version
\end{enumerate}

The end goals of DAP 4.0 are:

\begin{itemize}
  \item Promote adoption by other groups,
  \item Use standard technologies.
\end{itemize}

Expected topics for the breakout session were:

\begin{itemize}
   \item Persistence
   \item Constraint expressions
   \item Services
\end{itemize}

% $Id$
%
% $Log: gallagherdapspecsummary.tex,v $
% Revision 1.3  2003/05/09 20:59:09  paul
% Address James's comments: consistently consistently; SOAP word changes.
%
% Revision 1.2  2003/05/05 16:40:45  tom
% edited, LaTeX-ized
%
% Revision 1.1  2003/04/29 14:09:01  paul
% Starting OPeNDAP Technical Working Conference 2003 Report
%
%
