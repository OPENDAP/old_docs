Breakout Leader: Dan Holloway (took notes also)

\subsubsection{Summary:}
 
  A consensus was reached that web search engines, such as Google, 
are becoming a standard capability that users expect will be able 
to assist them in discovering data resources.  Given this expectation,
the group discussed different strategies and requirements to 
facilitate web search engines in providing a search capability for 
OPeNDAP-accessible data.

\subsubsection{Discussion:}

\begin{description}
\item[Duplication of Effort:]  The group acknowledged that there are several 
  data discovery efforts underway involving the GCMD, THREDDS, and 
  OPeNDAP Dataset List and was concerned with potential duplication of 
  effort or insufficent communication between the various efforts.  Also, 
  how to link client applications to the discovery services these groups 
  have developed.

\item[Web Search Engines:]  The group generally agreed that web search engines,
  such as Google, are becoming an accepted capability for locating data
  resources available on the internet.  The group discussed what web
  search engines could and could not provide as part of the data 
  discovery process.  Search engines are good at providing keyword based
  searches of persistant HTML pages but could not readily address the
  typical Geospatial/Temporal search requirements to identify specific
  data granules.  Additionally, the concept of `trusted data' was raised
  and whether data resources located using standard web search engines 
  might meet this criteria.  

\item[Service Framework:]  The discussion of exploiting web search engines 
  to facilitate data discovery concluded that discovery services should
  provide persistant HTML representations that standard web search 
  engines could harvest.  One approach might be to develop a set of 
  discovery services, layered from simple services such as DODS-DIR, 
  to more complete data discovery search services such as THREDDS catalogs
  or the GCMD query interface, thus constituting a discovery service 
  framework.  And that each service in the framework should present a 
  persistant HTML representation identifying itself so that a web search 
  engine like Google could harvest those pages.

\item[Best Practices Document:]  The group acknowledged the importance of 
  facilitating Data Providers in making their OPeNDAP-accessible data
  readily locatable thru web search engines.  Several topics were
  discussed.  First, that OPeNDAP-accessible data servers will provide
  differing levels of discovery services, ranging at the lowest-level
  of a simple URI identifying an OPeNDAP server instance, to filesystem
  traversal services, and potentially onto more geophysically-aware 
  search interfaces identifying specific granules within large data 
  collections, such as THREDDS catalogs, and that there will be best 
  practices which will facilitate each of these discovery services.  
  To facilitate the Data Provider, whenever possible, the services 
  should link to existing resources within the Provider's facility.  
  Additionally, the required information to facilitate data location 
  through web search engines should be kept to a minimal set.  There 
  was a discussion of the applicability of existing metadata conventions, 
  such as the Dublin Core Lite or GCMD DIF Lite, but a consensus was 
  not reached on the applicability to web search engine use. The 
  requirement to support existing metadata conventions employed by 
  various Data Providers, such as GCMD DIF or FGDC CSDGM, was 
  acknowledged to minimize the cost of participation.

  Several group participants suggested the development of tools, and/or
  user interfaces to assist Data Providers in describing their data
  to facilitate discovery through various discovery services.  In 
  particular, groups such as the GCMD and FGDC which have experience
  in this area should be collaborated with.
\end{description}

% $Id$
%
% $Log: discoverysummary.tex,v $
% Revision 1.2  2003/05/05 16:40:45  tom
% edited, LaTeX-ized
%
% Revision 1.1  2003/04/29 14:09:01  paul
% Starting OPeNDAP Technical Working Conference 2003 Report
%
%

%%% Local Variables: 
%%% mode: latex
%%% TeX-master: t
%%% End: 
