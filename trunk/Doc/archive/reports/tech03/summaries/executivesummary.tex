The \ac{OPeNDAP} Technical Working Conference of 2003 was held in Boulder,
Colorado on the 19th through the 21st of March 2003.  

After an overview of the current state of \ac{OPeNDAP} and its use,
two presentations put the current state of the \ac{OPeNDAP}
project in the context of broader issues: the use of \ac{OPeNDAP} in
the \ac{ESG} Project and the use of \ac{OPeNDAP} in the \ac{USGOOS}
Project, including requirements by those two project on the developments
to be discussed at the meeting.

The \ac{OPeNDAP} and the software are both going through a phase of
transition and expansion, and the conference was designed to address
both, while concentrating on the latter.  

\ac{OPeNDAP} itself is going from a classical software development
project in an academic environment to the open-source development project
of a loosely affiliated not-for-profit corporation, while trying to 
entrain new users and user communities as well as outside developers.  

A technical working conference was held in 2002.  Many of the 
developers and data providers met for the first time there.  That meeting
resulted more in demonstrations of work done than in specific
recommendations for future work, although some direction came
from the discussions.

The \ac{OPeNDAP} and \ac{NVODS} executive committees decided that
the time was ripe for another conference, but this time the
emphasis would be on discussions and presentations concerning several
specifically identified topics.  To that end, the conference was
designed to address the following software changes and issues, and
a breakout discussion session was planned for each:

\begin{itemize}
  
\item The \ac{OPeNDAP} protocol is moving from a definition dependent
  on a specific transport technology (HTTP) to one which can be
  implemented consistently in a variety of ways using a
  variety of transfer protocols.
  
\item Scientific communities that will have many users for common
  datasets need the ability to impose standards and conventions on
  their common datasets so as to facilitate ease of community use, and
  to allow machine-to-machine interoperability for large or continuing
  projects, such as environmental monitoring and management.  To that
  end, the \ac{AIS} was conceived and is being implemented.
  
\item The \ac{GIS} community of communities is very large but has no
  direct access to \ac{OPeNDAP} served data.  A connector from
  \ac{OPeNDAP}-served data to one or more \ac{GIS} enabled clients is
  of paramount importance in making \ac{OPeNDAP} data available to
  that segment of the scientific/management/education public facile
  with \ac{GIS} tools.
  
\item Many \ac{OPeNDAP} datasets are coming on line, particularly for
  internal use within single organizations, which are not advertised
  to the outside world, even though they are open and the data are
  freely available.  Several mechanisms and tools are available or are
  in advanced stages of development that will make discovery and
  inventory information readily available to users, including the
  capabilities of \ac{THREDDS} and the \ac{ODC}.
  
\item One challenge for pragmatic use of the system is how to serve
  and access in-situ data in a way that allows free selection of the
  variables to be transmitted, while obtaining the selection in a
  timely fashion that does not bind computing and network resources.
  Because of the variety of data and data types and formats used for
  the storage of in-situ data, an organized attack on the problem
  within \ac{OPeNDAP} is underway.

\end{itemize}

Detailed specific recommendations and/or plans for collaborations
emerged from each of the breakouts, and are given in this report.


The main conclusions concerning the software development in these areas
are:

\begin{enumerate}

   \item \ac{DAP} Specification: 

     \begin{itemize}
       
     \item The direction the new \ac{DAP} Specification is taking was
       well received.
       
     \item Several items received extensive discussion with specific
       recommendations, including: XML request/response, ``service''
       documentation, a ``permission object,'' ragged arrays, various
       other things listed in the detailed conclusions.

     \end{itemize}
     
   \item \ac{AIS}: It is a critical for making varied and disparate
     datasets conform to community and individual standards and
     conventions as required by specific applications, and to insure
     machine-to-machine interoperability in some cases.  Parallel
     efforts by \ac{OPeNDAP} and Unidata should be pursued
     independently, but with cross-knowledge so that the two efforts
     can be merged if and when that becomes propitious.  Concerning
     the use of the \ac{AIS}: Documentation should be provided for any
     specific convention or standard, inclucing how validators for
     conventions should be written and applied.  However, XML
     documents should not be required to be validated by clients.  The
     \ac{AIS} should draw from standards sources such as \ac{GCMD} and
     \ac{FGDC}.  \emph{An audit trail should be included.}

   \item The Treatment of In-situ Data:  

     \begin{enumerate}

     \item The field chosen for indexing is not always the most
       obvious or advantageous for accessing particular subsets of
       data.  Searchable header fields, usually identified ahead of
       time to allow caching and extraction.  However, such
       identification is not consistent.
       
     \item The data model is at least adequate to handle general
       in-situ data.
       
     \item One possible extension of the architecture is to have data
       files with a separate store/cache/database for header
       information.
       
     \item Having a server that can aggregate in-situ data is
       extremely important.  Joe Wielgosz will consider developing a
       \ac{NetCDF} in-situ aggregation capability.
     \end{enumerate}
   
   \item Discovery and Inventory Tools: The group concluded that web
     search engines are a viable facility to discover
     OPeNDAP-accessible resources on the internet, and that existing
     services and any new services developed to facilitate data
     discovery should generate persistant HTML representations that
     web search engines can harvest.  To roll-out this capability the
     group suggested developing a very simple server identification
     service as an initial test, and that `Best Practices' documents
     and tools should be developed to assist Data Providers.

     
   \item \ac{GIS}: Several collaborations were established to make
     \ac{OPeNDAP}-\ac{GIS} connections, and to populate some specific
     datasets in \ac{OPeNDAP} to exercise and test those connections.
   \end{enumerate}


Finally, the pure \ac{OPeNDAP} users presented several examples
of \ac{OPeNDAP} servers in use, tests of those servers, 
integration of \ac{OPeNDAP} into other systems, and the stages
of development of systems and projects that use or will use
\ac{OPeNDAP}.

The meeting was a good mix of presentations of the order of
20 minutes each, and breakout discussions of the order of one
or two hours each, which lead to a clear sense of participation
by all, and most with a sense of their individual work being 
significant contributions toward the progress and goals of the
whole \ac{OPeNDAP} endeavor.  The result was a feeling that the
project is vibrant and growing;  the current software is being
used extensively and successfully, and the discussions point
to major improvements that will enhance the use of \ac{OPeNDAP}
by high performance computer, modelers, users of in-situ data,
and users who need to make sure their data conform to uniform
standards of their own design and/or choosing.

% $Id$
%
% $Log: executivesummary.tex,v $
% Revision 1.3  2003/05/09 20:59:09  paul
% Address James's comments: consistently consistently; SOAP word changes.
%
% Revision 1.2  2003/05/05 16:40:45  tom
% edited, LaTeX-ized
%
% Revision 1.1  2003/04/29 14:09:01  paul
% Starting OPeNDAP Technical Working Conference 2003 Report
%
%

%%% Local Variables: 
%%% mode: latex
%%% TeX-master: t
%%% End: 
