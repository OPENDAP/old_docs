The \ac{OPeNDAP} Technical Working Conference of 2003 was held in
Boulder Colorado on the 19th through the 21st of March, 2003, during
one of the largest snowfalls in the record of the eastern foothills of
the Colorado Rocky Mountains.  The blizzard conditions led to a change
of venue for the first day of the meeting, and the non-attendance of
roughly half of the pre-registered participants.  In spite of the dire
predictions on Monday, the 17th of March, and because some
participants were en-route from around the world, we decided to
announce that the meeting would take place, but that travel would be
problematic, and people should make their own arrangements based on
their own schedules and local conditions.

Hence, two of the presentations were given \emph{in absentia}, and the
other scheduled presentations, which were not given because the
presenters could not attend, are included in the report, on the web
site:
\xlink{http://www.po.gso.uri.edu/tracking/meetings/tech03/}{http://www.po.gso.uri.edu/tracking/meetings/tech03/}
and will be available from the web site.

While the evolution of \ac{OPeNDAP} is toward a broader use by other
communities serving other data, the genesis of \ac{OPeNDAP} was in the
oceanographic community, and its main use continues to be the
distribution of oceanographic data.  Therefore, many of the technical
presentations deal with oceanographic-specific data, even though the
technical details address problems of data distribution and their
solutions in a generic and general fashion.

The objective of the conference was to bring together three groups
of people intimately involved in the development and use of the
\ac{DAP}:  The \ac{OPeNDAP} developers, other developers who are
making direct use of and contributions to the \ac{OPeNDAP} system,
and data providers and users who have experience with providing
data and/or accessing and using data via the \ac{OPeNDAP} system.
Many elements of the system are being developed, not the least of
which is a re-specification of the \ac{DAP}.  Several of these elements
were culled out for specific presentations and breakout discussions.

Two breakout discussions were held on the first day: one discussed the
\ac{AIS} and the other concerned itself with the problems associated
with serving and accessing in-situ data.  Three breakout discussions
were held on the second day: \ac{GIS} issues, discovery and inventory
tools, and the new \ac{DAP} Specification.  Many people came to the
conference with specific recommendations concerning these topics, and
the ensuing discussions demonstrated how concepts evolve, merge, and
how new concepts emerge with face-to-face discussions among
knowledgable people.

This report is divided into sections by topic.  Within each topic
are summaries of the presentations relevant to that topic,
a report of the breakout session, if appropriate, and a section
giving the conclusions with regard to that topic.

Then a final summary of the meeting is given, and the overall
conclusions are presented.

The conference produced several specific solutions to some problems,
and some recommendations for future changes.  However, the most
important result of the conference was the kindling and re-kindling of
discussion on new and old topics which will change the direction of
the \ac{OPeNDAP} project, and improve the provision and access to
distributed data beyond what would have happened without the direct
interaction of the participants.  We wish to thank all the
participants for persevering through the travails of the inclement
weather to attend the meeting.  We believe the result was well worth
everyone's effort.




% $Id$
%
% $Log: introduction.tex,v $
% Revision 1.3  2003/05/05 16:40:45  tom
% edited, LaTeX-ized
%
% Revision 1.2  2003/04/29 14:50:23  tom
% fixed for LaTeX and hylerlatex
%
% Revision 1.1  2003/04/29 14:09:43  paul
% Starting OPeNDAP Technical Working Conference 2003 Report
%
%

%%% Local Variables: 
%%% mode: plain-tex
%%% TeX-master: t
%%% TeX-master: t
%%% End: 
