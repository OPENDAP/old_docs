\documentstyle[12pt,html,psfig]{article}
%
% hacked from art12.doc
% 
% makes 1 , 1.25, 1.25 and 1.25 in. margins on the top, left, right and
% bottom resp. when using \documentstyle[12pt,...]{article}
%
% jhrg 1/6/94

% SIDE MARGINS:
\if@twoside               % Values for two-sided printing:
   \oddsidemargin 0pt     %   Left margin on odd-numbered pages.
   \evensidemargin 38pt   %   Left margin on even-numbered pages.
   \marginparwidth 85pt   %   Width of marginal notes.
\else                     % Values for one-sided printing:
   \oddsidemargin 18.5pt  %   Note that \oddsidemargin = \evensidemargin
   \evensidemargin 18.5pt
   \marginparwidth 68pt 
\fi
\marginparsep 10pt        % Horizontal space between outer margin and 
                          % marginal note
 
 
% VERTICAL SPACING:        
\topmargin 0pt            %    Nominal distance from top of page to top
                          %    of box containing running head.
\headheight 0pt           %    Height of box containing running head.
\headsep 0pt              %    Space between running head and text.
\topskip = 12pt           %    '\baselineskip' for first line of page.

\textheight = 635pt % 36\baselineskip
\advance\textheight by \topskip
\textwidth 435pt         % Width of text line.

% Adding this fixes the top and bottom margin sizes. jhrg

\textheight 8.75in

\newcommand{\postscript}[2]{
        \par
        \hbox{
                \vbox to #1{
                        \vfil
                        \special{ps: plotfile #2.ps}
                }
        }
}







% Adding this fixes the top and bottom margin sizes. jhrg

\textheight 8.75in

\newcommand{\postscript}[2]{
        \par
        \hbox{
                \vbox to #1{
                        \vfil
                        \special{ps: plotfile #2.ps}
                }
        }
}

\begin{document}

\LARGE
\centerline{Discussion of DODS Development Status}
\centerline{Current and Evolving Efforts}
\large
\centerline{Meeting at URI \hfil 20 June 1994}
\centerline{Peter, Jim and George}
\vskip 6em


\noindent {\bf Discussion of the move to use the World Wide Web (WWW) server
as the network transport agent within DODS.}
\vskip 1em
\normalsize

Jim reported on his meeting with Glenn on Friday, 17 June 1994, during which
they discussed the advantage of abandoning Remote Procedure Calls, (RPC), as
the network transport agent in favor of using the WWW server.  Glenn has
been working on using the WWW server with JGOFS and is convinced that it can
support the kinds of functionality that will be necessary for DODS.  The
change to the WWW server is viewed as an implementation decision and does not
change the fundamental DODS architecture.  Jim felt that the redesign would
cost us about 4-6 weeks in terms of the development schedule and raised the
issue of whether or not to push back the the October DODS meeting.  In
addition to addressing the change in schedule Jim also wanted to the clarify
what the objectives of the meeting are.

Jim summarized the main points of the WWW server implementation as:

\begin{itemize}

  \item The WWW server would be installed at each data provider site.
  \item The DODS WWW server implementation will use API procedure calls as
        the data delivery mechanism, just as the RPC designed implementation
        did.  The move to the WWW server reflects more of an implementation
        decision than a redesign of the basic DODS architecture.  
  \item Linked to the back-end of the WWW servers would be a translating
        module (CGI), that included the local data set's API library and a DODS
        API library. 
  \item For a client procedure request, the WWW server translating module
        would perform the following operations: 
  \begin{itemize}
	\item Translate from the incoming (WWW server) DODS API procedure
              call to the local data sets API procedure call(s) (e.g.; JGOFS). 
	\item Execute the appropriate local API procedure(s) on the data.
	\item Translate the retrieved data from the local API structure to
              an appropriate DODS API structure.
	\item Pass the DODS API data structure to the WWW server for
	      transport to the client application.
  \end{itemize}
  \item A user's client applications that accesses a DODS supported WWW
	servers would perform the following operations:
  \begin{itemize}
	\item A user's application makes a procedure call using a supported
	      API. 
	\item The API procedure call is translated into the appropriate DODS
	      API procedure call(s) by the client translating module.
	\item The client translating module passes the DODS API structure to
	      the WWW network client module  which transmits it to the
	      appropriate WWW server. 
	\item Responses from the WWW server, in DODS API structures, are
	      captured by the WWW network client module and passed to client
	      translating module.
	\item The client translating module translates the DODS API surrogate
	      procedure call(s) to the appropriate user application API
	      procedure call. 
  \end{itemize}
\end{itemize}

Jim emphasized the following key issues with regard to this implementation:

\begin{enumerate}
  \item  Utilizing WWW servers decreases the amount of software that DODS would
         be responsible for writing, debugging and maintaining. 
  \item  We will need to develop translators regardless of whether we
         implement RPC or the WWW server, in the WWW implementation
         the number of translators is a linear function of the number of
         supported APIs.
  \item  This approach requires that the DODS API be a fully realized API and
         not a patch to supplement other APIs that may not have the breadth
         of functional capabilities that the DODS data model model supports.
  \item  The DODS API supports more data types than either JGOFS or NetCDF.
  \item  The DODS API is simpler in terms of the number of separate procedure
         calls that it implements.
\end{enumerate}


Jim reported that Glenn felt concern about the upcoming
workshop\footnote{Besides the need to reschedule on account of the switch to
WWW server implementation} because our NASA proposal for server
development had not been funded.  Glenn had anticipated that the DODS design
and implementation architecture would be presented at the workshop to the
other server developers in the proposal.  Peter felt that we could still
focus on server development at a DODS workshop if we restricted the workshop
attendees to a small select group that were invited primarily to do just
that, write code for DODS servers.  The need to coordinate the DODS server
development so that an operational version of DODS could be presented at the
TOS workshop in April 1995 was also discussed.  The following tentative plans
were made related to DODS server development:

\begin{itemize}

  \item Two small DODS workshops will be held.
  \item A small, select group of participants will be invited to each
        workshop\footnote{ We might consider requiring that participants
        write a short statement of intent or proposal indicating what data
        sets they are going to support with DODS servers and what their plans
        are for maintaining support for DODS}.  Between 4 and 5 organizations
        will be asked to attend with the request that each organization send
        at least one systems' programmer who will be asked to spend time at
        the workshop developing and installing a DODS server to support
        access to at their organization's data.
  \item Each workshop will be approximately 1 week long, with the 1-2 day
        relegated to a technical overview of DODS its design and
        implementation and the rest of the time devoted to DODS server
        development. 
  \item The first workshop would be scheduled for the week of 5-9 December
        1994.  Large federal organization would be invited to this meeting 
        {e.g; NODC, MEDS, JPL/PODAAC, NOAA/PMEL, NOAA/TOGA/COARE}.
  \item The second workshop would be scheduled for the week of 23-27 January
        1995.  Non federal organizations would be invited to this workshop 
        {e.g.; Scripps, WHOI, OSU, Hawaii, UW}.
  \item The primary objectives of the workshop would be:
  \begin{itemize}
	\item Write code and install DODS servers for on-line data sets that
	      are managed by the participating organizations.
	\item Coordinate the set-up of a stable operational DODS prototype.
	\item Develop working relationships with the technical personnel at
	      the different organizations who will be responsible for the day
	      to day management of the DODS servers.
  \end{itemize}
  \item A stable DODS prototype system will be operational for the April 1995
        TOS.  A presentation would be made at the meeting by Glenn.  A
        demonstration of DODS would highlight two different user applications
        (e.g.; JGOFS and MATLAB) accessing the same data over the Internet.
\end{itemize}

In the near term the following plans were made:

\begin{itemize} 
  \item Jim will organize his notes from the meeting with Glenn.
  \item Jim will need to modify the NetCDF RPC libraries in order to
        utilize the WWW http daemon.
  \item Translators for NetCDF to DODS, DODS to NetCDF, JGOFS to DODS and
        DODS to JGOFS need to be developed.
  \item Reza will be brought in to help with the software development for the
        translators.
  \item We will focus on the development of libraries for translators and the
        DODS API as well as system developers documentation at this time.
        Integration of third party systems (i.e.; MATLAB\footnote{It was
        pointed out that MATLAB supports NetCDF and therefore will be able to
        access DODS data through that API}) will be a lower priority for now. 
  \item The issue of how we announce the shift to the WWW server
        implementation was raised.  We want to avoid the perception of our
        taking a hit-or-miss approach to DODS development.  This is
	particularly the case with the NetCDF RPC libraries that have already
        been developed.  These libraries will no longer be supported by DODS
        in the WWW server implementation.  
  \item George will work on modifying current documentation to reflect the
        changes in the DODS implementation and finish high level overview
        documentation on DODS.
  \item George will be responsible for the planning and organization of the
        workshops.
  \item Peter spoke with Dan about whether he would be interested in
        participating in the DODS development activities.  Dan said that he
        could spend two days a week working on DODS.
\end{itemize}

\end{document}
 
