%
%  $Id$
%
\documentclass{dods-paper}
\W\usepackage{frames}

\newcommand{\DODSinstallUrl}{http://www.unidata.ucar.edu/packages/dods/user/install-html/}
\newcommand{\opendap}{OPeNDAP}

\htmldirectory{writing_server}

\title{Writing an \opendap\ Server}
\author{James Gallagher}

\begin{document}
\maketitle


\section{Preface}

Writing your own \opendap\ server is one way to serve data to clients
that understand the Data Access Protocol (DAP). This tutorial
describes the writing of such a server.


\section{Before you write any code}

Even if the data you want to serve are stored in some idiosyncratic
format, or accessed using an API that you or your group have
developed, there may already be a server which, with the correct
configuration, can work with your data.

The DODS project has developed two servers which can be customized to
read different types of data.  One, the FreeForm server can be used to
provide table and array data, with some restrictions on the data's
storage format.  The JGOFS server can only serve table-like data but
has a more flexible --- and more complicated --- customization scheme.
More introductory information about these two servers can be found in
the \xlink{Server Installation Guide}{\DODSinstallUrl/install\_A.html}.

  

\subsection{The FreeForm server}


The FreeForm server is configured for a specific dataset using a
`format specification file.' Detailed information about writing these
files for your data may be found in the \xlink{FreeForm server
  guide}{http://www.unidata.ucar.edu/packages/dods/user/servers/dff-html/}.
The main limitation of the FreeForm server is that it requires that
all information be rigidly column-oriented. For example, consider the
two ASCII data files shown below:

\begin{vcode}{sib}
  1 one 13.4                            1 one   13.4
  2 two 27.8                            2 two   27.8
  3 three 17.4                          3 three 17.4

ASCII data that cannot                  Data that can
be served with FreeForm
\end{vcode}


The first file cannot be served by the FreeForm server because the
third value in each record does not start at the same column position.
The same data can be served only if each datum in all the records `lines
up.' In addition to ASCII data, the FreeForm server can also serve
binary data and DBase files.  The FreeForm server cannot serve CSV
files.


\subsection{The JGOFS server}

The JGOFS server can serve data that is logically structured in a
table-like fashion. By table-like, we mean that the data are organized
in row and column form with the additional twist that the rightmost
column can itself be a table; it's not limited to the standard simple
data types such as Integer or floating point numbers. This enables the
JGOFS server to efficiently serve data that are hierarchical without
duplicating values at the outer levels.  Compare JGOFS' nested tables
with a regular flat table scheme:
   
\begin{vcode}{ib}
station depth temp                         station depth temp
1       10    10.2                         1       10    10.2
        20    10.3                         1       20    10.3
        30    10.0                         1       30    10.0
2       10    10.2                         2       10    10.2
        20    10.35                        2       20    10.35
        30    10.0                         2       30    10.0
                   
   Nested Tables                      A Flat Table Duplicates Values
\end{vcode}

The JGOFS server can also read data that's organized in many files,
making access to the individual files seamless. In practice this is
quite powerful.  If information about stations, continuing with the
example above, is stored in one file and information about each
depth/temperature sounding is stored in separate files, JGOFS can
easily be configured to serve these data.  In fact, the JGOFS server
comes with a standard `method' that can serve data stored in just this
configuration as long the data are stored as ASCII values.

However, JOGFS is far more flexible than just a scheme to serve ASCII
data.  The JGOFS server is customized by writing a writing a
specialized access method for a new data storage format. For more
information about the JGOFS system see the \xlink{JGOFS Data System
  Overview}{http://puddle.mit.edu/datasys/jgsys.html}.  There is
also a detailed \xlink{method writing
  guide}{http://puddle.mit.edu/datasys/jgsysdoc.html} with examples
available for JGOFS.


\section{Writing your own \opendap\ server}

If neither of the available flexible data-readers can work with your
data, or if you wish to add some features that are not otherwise
available, you may find it advisable to write your own \opendap\ server.
  

\subsection{Choose a language}

  
It is possible to take the \xlink{DAP
  specification}{http://www.unidata.ucar.edu/packages/dods/design/dap-rfc-html/}
and implement a server which DAP-aware clients can use as a data
source. For example, the data server hosted by the \xlink{IRI/LDEO
  Climate Data Library}{http://ingrid.ldeo.columbia.edu/} at Columbia
University is such a server. However, server writers don't have to
work from the DAP specification. The DODS Project has developed two
separate object-oriented class libraries which, along with other
software we provide, can greatly simplify building a server in most
cases.


The DODS project provides both a
\xlink{C++}{http://www.unidata.ucar.edu/packages/dods/api/pref-html/}
and a
\xlink{Java}{http://www.unidata.ucar.edu/packages/dods/api/javaDocs/}
implementation of the DAP. Each library includes both the classes that
implement the various objects which comprise the DAP and support
software that handles the mechanics of processing inbound requests and
generating the correct responses.

To choose one of the toolkits, several factors should be weighed.
First, with which of the two programming languages are you most
comfortable? Also to be considered are: What type of computer will the
server run on.  Java is equally supported on win32 and Unix (and mac,
in all probability) while the C++ code for server development is
supported only on Unix.  If you have an API that can read the data,
will it be easier to use it from C++ or Java?   Lastly, the Java
servers are implemented as servlets and typically spend less time on
startup tasks than the C++ servers which use the CGI mechanism.  If
you anticipate many small requests, then you can expect noticeable
performance improvements with the Java code, while larger average
requests will mitigate this difference.  

\note{Developers at UCAR have implemented DODS servers as Apache httpd
  modules; these DODS servers effectively run as Unix daemons and thus
  have none of the startup performance issues of CGI programs. These
  servers are noticeably faster---about one order of magnitude---for
  very small requests. In the future we plan to incorporate this
  software in our general distribution, contact \xlink{technical
    support}{mailto:support@unidata.ucar.edu} or the \xlink{
    dods-tech}{http://www.unidata.ucar.edu/packages/dods/home/mailLists/subscribe-dods-tech.html}
  list for information/help.}

\subsection{Server architecture}

The essence of the DODS server architecture is that a collection of
programs are used to handle various requests made to the servers. In
addition to these `handler' and `service' programs there's also a
dispatcher that interfaces to an http daemon. The actual requests are
made to the web daemon which then passes them along to the dispatcher.
The dispatcher examines the request and decides which handler or
service program should process it and how that program should be
passed parameters extracted from the request.

\note{The C++ software works exactly as described above; the Java code is
the same in principle but slightly different in practice, since it's
based on servlets.}

While the C++ toolkit uses an architecture based on CGI and the Java toolkit
uses servlets, both share many characteristics. If you understand how the
servers are built, it will be easy to see how your own server can be
implemented with minimal effort. The Server Installation Guide's section on
\xlink{Server Architecture}{\DODSinstallUrl/install\_B.html} provides an
excellent description of the CGI-based (C++) servers. The Server Installation
Guide contains a \xlink{short how-to that covers setting up the Java
  software}{\DODSinstallUrl/install\_C.html}.  It also explains the software
needed to run the servlet-based DODS servers.


\section{The DAP Architecture}

  

The DAP can be thought of as a layered protocol composed of MIME, HTTP, basic
objects, and complex, presentation-style, responses.
  

\subsection{The DAP uses HTTP which in turn uses MIME}

  

Clients use HTTP when they make requests of DAP servers. HTTP is a fairly
straightforward protocol (\xlink{General information on
  HTTP}{http://www.w3.org/Protocols/}, \xlink{The HTTP/1.1 specification in
  HTML}{http://www.w3.org/Protocols/rfc2616/rfc2616.html}).  It uses MIME
documents to encapsulate both the request sent from client to server and the
response sent back. This is important for the DAP because the DAP uses
headers in both the \xlink{
  request}{http://www.unidata.ucar.edu/packages/dods/design/dap-rfc-html/dap\_16.html}
and \xlink{
  response}{http://www.unidata.ucar.edu/packages/dods/design/dap-rfc-html/dap\_22.html}
documents to transfer information.  However, for a programmer who intends to
write a DAP server, exactly what gets written into those headers and how it
gets written is not important.  Both the C++ and Java class libraries will
handle these tasks for you (look at the \xlink{DODSFilter
  class}{http://www.unidata.ucar.edu/packages/dods/api/pref-html/DODSFilter.html}
to see how). It's important to know about, however, because if you decide not
to use the libraries, or the parts that automate generating the correct MIME
documents, then your server will have to generate the correct headers itself.

\subsection{The DAP defines three objects}

To transfer information from servers to clients, the DAP uses three
objects.  Whenever a client asks a server for information, it does so
by requesting one of these three objects (note: this is not strictly
true, but the whole truth will be told in just a bit. For now, assume
it's true). These are the Dataset Descriptor Structure (DDS), Dataset
Attribute Structure (DAS), and Data object (DataDDS). These are
described in considerable detail in other documentation. The
Programmer's Guide contains a description of the \xlink{DDS and DAS
  objects}{http://www.unidata.ucar.edu/packages/dods/api/pguide-html/pguide\_6.html}.
These objects contain the name and types of the variables in a
dataset, along with any attributes (name-value pairs) bound to the
variables. The DataDDS contains data values. We have implemented the
SDKs so that the DataDDS is a subclass of the DDS object that adds the
capacity to store values with each variable.

  

\begin{tabular}[c]{lll} \\
\xlink{COADS Climatology}{http://dodsdev.gso.uri.edu/dods-3.2/nph-dods/data/nc/coads\_climatology.nc.html} &
\xlink{DAS}{http://dodsdev.gso.uri.edu/dods-3.2/nph-dods/data/nc/coads\_climatology.nc.das} &
\xlink{DDS}{http://dodsdev.gso.uri.edu/dods-3.2/nph-dods/data/nc/coads\_climatology.nc.dds} \\
\xlink{NASA Scatterometer Data}{href="http://dodsdev.gso.uri.edu/dods-3.2/nph-dods/data/hdf/S2000415.HDF.ascii?Wind\_Speed\%5B0:1:457\%5D\%5B0:1:23\%5D\%5B0:1:3\%5D,Wind\_Dir\%5B0:1:457\%5D\%5B0:1:23\%5D\%5B0:1:3\%5D} &
\xlink{DAS}{http://dodsdev.gso.uri.edu/dods-3.2/nph-dods/data/hdf/S2000415.HDF.das} &
\xlink{DDS}{http://dodsdev.gso.uri.edu/dods-3.2/nph-dods/data/hdf/S2000415.HDF.dds} \\
Catalog of AVHRR Files &
\xlink{DAS}{http://dodsdev.gso.uri.edu/dods-3.2/nph-dods/data/ff/1998-6-avhrr.dat.das} &
\xlink{DDS}{http://dodsdev.gso.uri.edu/dods-3.2/nph-dods/data/ff/1998-6-avhrr.dat.dds} \\
\xlink{AHVRR Image}{http://dodsdev.gso.uri.edu/dods-3.2/nph-dods/data/dsp/east.coast.pvu.ascii?dsp\_band\_1\%5B0:1:511\%5D\%5B0:1:511\%5D} &
\xlink{DAS}{http://dodsdev.gso.uri.edu/dods-3.2/nph-dods/data/dsp/east.coast.pvu.das} &
\xlink{DDS}{http://dodsdev.gso.uri.edu/dods-3.2/nph-dods/data/dsp/east.coast.pvu.dds} \\
\end{tabular}   


The DAP models all datasets as collections of
\xlink{variables}{http://www.unidata.ucar.edu/packages/dods/api/pguide-html/pguide\_9.html}.
The \xlink{DDS and
  DataDDS}{http://www.unidata.ucar.edu/packages/dods/api/pref-html/DDS.html}
objects are containers for those variables.  How you represent your
dataset using the three objects and the variable's data type hierarchy
is covered in section~\ref{server-tut,implementing}.

\subsection{The DAP also defines services}

In the previous section we said that the DAP defined three objects and
all interaction with the server involved those three objects. In fact,
the DAP also defines other responses. They are:

\begin{description}
\item[ASCII] Data can be requested in CSV form.
\item[HTML] Each server can return an HTML form that facilitates building
 URLs.
\item[INFO] Each server can combine the DDS and DAS and present that as 
HTML.
\end{description}

In each case the server's response to these requests is built using
one or more of the basic three objects.  Here are some links to
various datasets' ASCII, HTML and INFO responses:

\begin{tabular}[c]{llll} \\
\xlink{COADS Climatology}{http://dodsdev.gso.uri.edu/dods-3.2/nph-dods/data/nc/coads\_climatology.nc.html} &
\xlink{ASCII for the SST variable}{http://dodsdev.gso.uri.edu/dods-3.2/nph-dods/data/nc/coads\_climatology.nc.asc?SST} &
\xlink{HTML}{http://dodsdev.gso.uri.edu/dods-3.2/nph-dods/data/nc/coads\_climatology.nc.html} &
\xlink{INFO}{http://dodsdev.gso.uri.edu/dods-3.2/nph-dods/data/nc/coads\_climatology.nc.info} \\
\xlink{NASA Scatterometer Data}{ref="http://localhost/dods-3.2/nph-dods/data/hdf/S2000415.HDF.ascii?Wind\_Speed\%5B0:1:457\%5D\%5B0:1:23\%5D\%5B0:1:3\%5D,Wind\_Dir\%5B0:1:457\%5D\%5B0:1:23\%5D\%5B0:1:3\%5D} &
\xlink{ASCII for wind speed and direction}{http://dodsdev.gso.uri.edu/dods-3.2/nph-dods/data/hdf/S2000415.HDF.ascii?Wind\_Speed\%5B0:1:457\%5D\%5B0:1:23\%5D\%5B0:1:3\%5D,Wind\_Dir\%5B0:1:457\%5D\%5B0:1:23\%5D\%5B0:1:3\%5D} &
\xlink{HTML}{http://dodsdev.gso.uri.edu/dods-3.2/nph-dods/data/hdf/S2000415.HDF.html} &
\xlink{INFO}{http://dodsdev.gso.uri.edu/dods-3.2/nph-dods/data/hdf/S2000415.HDF.info} \\
Catalog of AVHRR Files &
\xlink{ASCII for values within a date range}{http://dodsdev.gso.uri.edu/dods-3.2/nph-dods/data/ff/1998-6-avhrr.dat.ascii?year,day\_num,DODS\_URL&day\_num\%3C170} &
\xlink{HTML}{http://dodsdev.gso.uri.edu/dods-3.2/nph-dods/data/ff/1998-6-avhrr.dat.html} &
\xlink{INFO}{http://dodsdev.gso.uri.edu/dods-3.2/nph-dods/data/ff/1998-6-avhrr.dat.info} \\
\xlink{AHVRR Image}{http://dodsdev.gso.uri.edu/dods-3.2/nph-dods/data/dsp/east.coast.pvu.ascii?dsp\_band\_1\%5B0:1:511\%5D\%5B0:1:511\%5D} &
\xlink{ASCII for the SST}{http://dodsdev.gso.uri.edu/dods-3.2/nph-dods/data/dsp/east.coast.pvu.ascii?dsp\_band\_1\%5B0:1:511\%5D\%5B0:1:511\%5D} &
\xlink{HTML}{http://dodsdev.gso.uri.edu/dods-3.2/nph-dods/data/dsp/east.coast.pvu.html} &
\xlink{INFO}{http://dodsdev.gso.uri.edu/dods-3.2/nph-dods/data/dsp/east.coast.pvu.info} \\
\end{tabular}
  

\subsection{Parts of the server you don't have to write}

You do not have to write handlers for the ASCII, HTML or INFO
responses because the DODS server includes software that generates
these using the DAS, DDS and DataDDS objects. In addition, if you
follow a simple rule about how you name the programs that generate the
object responses, you'll be able to fit these within the existing
dispatch software and can avoid writing that as well.  The rule is
that the three objects are generated by programs named
\var{name}\lit{\_das}, \var{name}\lit{\_dds} and
\var{name}\lit{\_dods} (the last one generates the DataDDS object).
The \var{name} can be any text. In practice, it should be short and
describe the data with which it's designed to work.
 
Below is a snapshot of the directory which holds the programs that
make up the DODS servers on my development computer. The ASCII
response is generated by the \lit{asciival} program, The HTML and INFO
responses are generated by the \lit{www\_int} and \lit{usage}
programs. You can also see the dispatch program (\lit{nph-dods}) as
well as the DAS, DDS and DataDDS handlers for the netCDF (nc), HDF (hdf),
Matlab (mat), JGOFS (jg) and FreeForm (ff) servers.
 
\begin{vcode}{xib}
[jimg@zanzibar etc]$ ls
aclocal.m4            ftp_dods_source.html   MIME/
asciival*             handler_name.pm        nc_das*
ChangeLog             hdf_das@               nc_dds*
check_perl.sh*        hdf_dds@               nc_dods*
common_tests.exp      hdf_dods*              nightly_dods_build.conf
config.guess*         HTML/
nightly_dods_build.conf.example
config.sub*           HTTP/                  nightly_dods_build.sh*
COPYRIGHT             INSTALL-clients        nph-dods*
CVS/                  INSTALL-matlab-client  nph-dods.in*
cvsdate*              installServers         printenv*
def*                  INSTALL-servers        README
deflate*              install-sh*            README-Matlab-GUI
depend.sh*            jg_das*                tar-builder.pl*
DODS_Cache.pm         jg_dds*                test-dispatch.sh*
DODS_Dispatch.pm      jg_dods*               ud_aclocal.m4
dods.ini              jgofs_objects_readme*  update-manifest.pl*
ff_das*               localize.sh*           update-manifest.pl~*
ff_dds*               LWP/                   usage*
ff_dods*              Makefile.common        usage~*
FilterDirHTML.pm      mat_das*               usage-jg*
ftp_dods_binary.html  mat_dds*               www_int*
ftp_dods_ml_gui.html  mat_dods*
[jimg@zanzibar etc]$ 
\end{vcode}


\section{Getting ready to write your components}

The three object handlers are normally implemented in three separate
programs.  Each program has a \lit{main()} function that looks like:

\begin{vcode}{sib} 
#include <iostream>
#include <string>

#include "DDS.h"
#include "cgi_util.h"
#include "DODSFilter.h"

extern void read_descriptors(DDS &dds, const string &filename);
throw  (Error);

int 
main(int argc, char *argv[])
{
    DDS dds;
    DODSFilter df(argc, argv);
       
    try {
        if (!df.OK()) {
             
df.print_usage();
            return 1;
        }
        
        if (df.version()) {
             
df.send_version_info();
            return 0;
        }

// Read the netCDF file dataset descriptor in memory
        read_descriptors(dds,  df.get_dataset_name());
        df.read_ancillary_dds(dds);
        df.send_dds(dds, true);
    }

    catch (Error &e) {

        set_mime_text(cout, dods_error,
        df.get_cgi_version());

        e.print(cout);
        return 1;
    }

    return 0;
}
\end{vcode}


Most of the software is boiler plate. The first two lines of main

\begin{vcode}
DDS dds;
DODSFilter df(argc, argv);
\end{vcode}


declare an instance of DDS, to be used a little later as well as an
instance of DODSFilter. The latter is used to parse command line
arguments fed to the program by the dispatch script. By using this
class and the dispatch script, you can assume that the correct options
and arguments will be passed into your program and parsed. The instance
of DODSFilter, \lit{df}, contains accessors for all the switches that
the dispatch script might use, so by passing \lit{argc} and \lit{argv}
to it you're sure to parse them all.

\begin{vcode}{sib}
try {
   if (!df.OK()) {
         df.print_usage();
         return 1
   }
   
   if (df.version()) {
         df.send_version_info();
         return 0;
   }
\end{vcode}  

This code calls the DODSFilter invariant to check that the handler was
invoked correctly. If a malformed request was made to the server, this
will be flagged here and the server will return an error message
describing how to submit a correctly formed URL. This also tests to
see if the request is for version information. If so, the
DODSFilter object prints the server's version number and the handler
exits. Just about every server built with our code includes these
lines verbatim.

\begin{vcode}{sib}  
// Read the netCDF file dataset descriptor in memory

   read_descriptors(dds, df.get_dataset_name());
   df.read_ancillary_dds(dds);

   df.send_dds(dds, true);
\end{vcode}  
 
These lines are the heart of the handler. Exactly what's going on here
will be covered in more detail later. However, each of the three
object handlers contains similar code that builds the object to
returned as the response and then passes that object to the
\lit{DODSFilter::send\_dds}, \lit{send\_das} or \lit{send\_data}
method, depending on the type of object to be returned.

\begin{vcode}{sib} 
catch (Error &e) {
        set_mime_text(cout, dods_error, df.get_cgi_version());
        e.print(cout);

        return 1;
       }

       return 0
\end{vcode}
  
Rounding out the program is a catch block that picks up exceptions thrown by
the any of DAP library code. The DAP library throws two types of exceptions,
\lit{Error} and \lit{InternalErr}. The latter is a subclass of \lit{Error},
so catching just Error will get everything. Note that you should also catch
\lit{bad\_alloc} exceptions at this level (the library does not) unless you
catch them inside the function or method that builds the DDS, DAS or DataDDS.
If \lit{e} is an \lit{InternalErr}, then when it prints, you'll see
information about the file and line number where the problem was detected.
Regular \lit{Error} objects print something that's more useful to users. By
calling the \lit{set\_mime\_text} function (see the file cgi\_util.cc) you're
sure that the error message will be returned to the client in a form that
both web browsers and more sophisticated clients can use.

  

\section{Subclassing the data types}

  

The DAP defines a data type hierarchy as the core of its data model.
This collection of data types includes scalar, vector and
constructor types. Most of the types are available in all modern
programming languages with the exceptions being Url, Sequence and
Grid. In the DAP library, the class \lit{BaseType} is the root of the
data type tree. 

% Add a UML figure of the data type classes. I have
% one but dia seems hosed at the moment... Fix this in the AM


\subsection{A quick review of the data types supported by the DAP}

The DAP supports the common scalar data types such as Byte, 16- and 32-bit
signed and unsigned integers, and 32- and 64-bit floating point numbers. The
DAP also supports Strings and Urls as basic scalar types. The DAP includes
two vector data types, Arrays (of unlimited size and dimensionality) and
Lists. Lists in the DAP must be type-homogeneous; it does not support Lists
of Lists. The DAP also supports three type-constructors: Structure, Sequence
and Grid. A Structure on the DAP mimics a struct in C. A Sequence is a
table-like data structure inherited from the JGOFS data system. It can be
used to hold information that might be stored in relational databases or
tables, either flat or hierarchical. The JGOFS, FreeForm and HDF servers all
use the Sequence data type. Lastly, the Grid data type is used to bind an
array to a group of `map vectors,' single dimension arrays that provide
non-integral values for the indices of the array.  The most typical use of a
Grid is to provide latitude and longitude registration for some georeferenced
array data (e.g., a projected satellite image). The DAP does not have a
pointer data type, but in some cases the Url data type can be used as a
pointer to variables between files.  More information about the \xlink{DAP's
  data type
  hierarchy}{http://www.unidata.ucar.edu/packages/dods/api/pguide-html/pguide\_9.html}
is given in the Programmer's Guide.


\subsection{Creating the subclasses}

When you start building a DAP server, the first thing you must do is
create a collection of data type subclasses. That is, each of the leaf
classes in the preceding class diagram must be subclassed by your
server. This is pretty easy since a good bit of the work is rote.

First we'll illustrate the parts that are mechanical. Here's an
example from the Matlab server. The class is the Byte class. In the
case of the matlab server, this class doesn't do anything beyond the
bare minimum, so it's a good starting point:

\begin{vcode}{sib}
Byte *
NewByte(const string &n)
{
    return new MATByte(n);
}

MATByte::MATByte(const string &n) : Byte(n)
{
}

BaseType *
MATByte::ptr_duplicate()
{
    return new MATByte(*this);
}

bool
MATByte::read(const string &)
{
    throw InternalErr(__FILE__, __LINE__, "Unimplemented read method
    called.");
}
\end{vcode}

To create a child of any of the data type leaf classes, you must
define three methods and one function. Let's talk about the function
first. The function \lit{NewByte} is what Meyers calls a `virtual
constructor.'  It's similar to a low-budget factory class
(``low-budget'' because it's not a class). This function is undefined
in the DAP library but is used there when creating instances of Byte.
Thus the function \lit{NewByte} is an interface that can be used to
create instances of Byte without knowing in advance the static type of
the object that will be created. If all this sounds a little weird,
just remember that your Byte, Int16, ... Grid classes --- whatever
they may be called --- must all contain an implementation of this
function and they should all return a pointer to an instance of your
child classes. In this case, it's an instance of the MATByte class. If
you look in the files for the Matlab server, you'll see that Grid
returns a pointer to a new MATGrid, and so on.

Second, a constructor must be implemented and should take the name of the
variable as its sole argument.

Third, your child classes should also define the ptr\_duplicate() method.
This method returns a pointer to a new instance of an object in the same
class.  Occasionally, in the DAP library, objects are declared with pointers
specified as \lit{BaseType *}.  If the \lit{new} operator was used to copy
such an object, the copied object would be an instance of BaseType (the
static type of the object) not the type of the thing referenced (the dynamic
type)\footnote{This is the oft discussed phenomenon of `slicing,' see Meyers,
  Stroustrup, et cetera for a complete explaination.}. By using the
\lit{ptr\_duplicate()} method the DAP library is sure that when it copies an
object, it's getting an instance of the subclass defined by your server.

Finally, each of the child classes must provide an implementation of the
\lit{read} method. This method is called by code in the DAP library to read
values from the data set. It will be explained in more detail when we get to
building DataDDS responses. For now, it's enough to know that if a particular
server has no use for a given data type (it happens that the Matlab server
will never need to create an instance of Byte, because Matlab 5 files can
only store float64 matrices) this method should throw an InternalErr object.

\section{Implementing the DDS object}
\label{server-tut,implementing}

Building a DDS object is the heart of writing your own \opendap\ server. This
object will be used to generate the DDS response and it will be the basis of
the DataDDS response. You have to do two things to accomplish building the DDS.
First you must decide how the variables that comprise your dataset can be
represented using the data type hierarchy that is part of the DAP. Once you
have done this, you need to write code that can build an instance of DDS for
your dataset. In practice the hardest part of this the first part; once you
know how to map variables in your dataset to the DAP data types, writing code
to build the DDS instance is easy.

Many data sets are actually a representative of large group. In some cases
there may be an API that can read the datasets and there may even be a
formal data model. In such a case you're best off using the API and
writing general code to build the DDS while performing a depth-first
scan of the variables in the dataset.
 

Here's how the Matlab server builds a DDS object:

\begin{vcode}{sib}
void
read_descriptors(DDS &dds_table, string filename)
{
    MATFile *fp;
    Matrix *mp;
      
    // dataset name
    dds_table.set_dataset_name(name_path(filename));
 
    fp = matOpen(filename.c_str(), "r");
    if (fp == NULL)
        throw Error(string("Could not open the file: ") + filename);

    // Read all the matrices in file
    while ((mp = matGetNextMatrix(fp)) != NULL) {

      // String types are used as attributes
        if(mxIsNumeric(mp)) {
            if(mxIsComplex(mp)) {
                string Real = (string)mxGetName(mp) + "_Real";
                // real part
                MakeMatrix(&dds_table, Real, mxGetM(mp),mxGetN(mp)); 

                string Imag = (string)mxGetName(mp) + "_Imaginary";
                // imaginary part
                MakeMatrix(&dds_table, Imag, mxGetM(mp),mxGetN(mp)); 
         } else
                MakeMatrix(&dds_table, (string)mxGetName(mp), mxGetM(mp),
                            mxGetN(mp)); 
         }

          mxFreeMatrix(mp);
    }
    matClose(fp);

    return true;
}
\end{vcode}

This function iterates over all the variables in the Matlab file named
by \lit{filename} and creates a variable in the DDS for each numerical
array in the dataset. Matlab does not have the notion of attributes
bound to specific variables, but it is often the case that attribute
information is present in string variables, something for which this code checks. However, this function simply ignores the string variables since
attribute information is the job of a different object. Of course, a
function could be written to build both objects at the same time\ldots

\begin{vcode}{sib} 
void 
MakeMatrix(DDS *dds_table, string name, int row, int column)
{
    Array *ar; 
    string DimName;
    size_t pos;
        
    // complex matrices have common rows and columns
    if ((pos = name.find("_Real")) != name.npos)
        DimName = name.substr(0, pos);
    else{
        if ((pos = name.find("_Imaginary")) != name.npos)
            DimName = name.substr(0, pos);
        else
            DimName = name;
    }

    BaseType *bt =  NewFloat64(name);    
    ar = NewArray(name);
    ar->add_var(bt);
    ar->append_dim(row,DimName+"_row");
    ar->append_dim(column,DimName+"_column");
        
    if (!dds_table)
        throw InternalErr(__FILE__, __LINE__, "NULL DDS object.");
   
    dds_table->add_var(ar);
}
\end{vcode}

This function has two main parts, the first, which is of less interest
to this tutorial, checks to see if the matrix holds complex numbers
and does some special stuff if it does. The second part creates a new
array of 64 bit floating point numbers. Here's the code:

  
\begin{vcode}{sib}
   BaseType *bt = NewFloat64(name);   
   ar = NewArray(name);
   ar->add_var(bt);
   ar->append_dim(row, DimName+"_row");
   ar->append_dim(column, DimName+"_column");
\end{vcode}
 

The first line creates a new instance of the \lit{Float64} data type and
assigns it to a \lit{BaseType}, the parent of all the data types. There's no
reason it couldn't be bound to an instance of \lit{Float64}, but in a server
where there might be many types of arrays, it is easier to use a pointer to a
more general object.  The second line creates a new Array instance and the
third line binds the \lit{Float64} object to the new \lit{Array} object,
making the array an array of \lit{Float64}s. The last two lines set the sizes
of the Array's dimensions.  Because instances of Array occur frequently, it
is a good idea to be familiar with the \xlink{\lit{
    Array}}{http://www.unidata.ucar.edu/packages/dods/api/pref-html/Array.html}
and \xlink{\lit{
    Vector}}{http://www.unidata.ucar.edu/packages/dods/api/pref-html/Vector.html}
classes (\lit{Vector} is the parent of both \lit{Array} and \lit{List}).

Finally, the last line of the function,

\begin{vcode}{sib}
    dds_table->add_var(ar);
\end{vcode}

Adds variable \lit{ar} to the DDS.

Note that our Matlab server supports only the data types that can appear
in a Matlab 5 file. This means that the only numeric data type supported
is a matrix of 64 bit floating point numbers. Strings, as mentioned earlier,
are handled specially. In most other servers, the code to build the variables
and load them in a DDS object is more complex since it must handle mapping
the dataset's different types to the DAP's.

\section{Implementing the DAS object}

To implement the DAS handler, start with the same boiler-plate code
used to create a \lit{main()} for the DDS handler and replace the call
to the function that builds the DDS with one that builds a DAS. An
example of such a function from the Matlab server is shown below:

\begin{vcode}{sib}
void
read_attributes(DAS &das_table, string filename)
{
    AttrTable *attr_table = das_table.add_table("MAT_GLOBAL", new AttrTable);
    
    MATFile *fp = matOpen(filename.c_str(), "r");
    if (fp == NULL)
        throw Error(string("Could not open the file: ") + filename.c_str());

    // Read all the matrices in file
    Matrix *mp;
    while ((mp = matGetNextMatrix(fp)) != NULL) {
        // String types are used as attributes
        if(mxIsString(mp)) {
            // get size
            int X = mxGetN(mp);
            int Y = mxGetM(mp);

            char *str_rep = new char [X*Y+3];
      
            // quote the string for parser
            *str_rep = '"'; 
            mxGetString(mp, str_rep+1, X*Y+1);
            *(str_rep + X*Y + 1) = '"';
            *(str_rep + X*Y + 2 )= '\0';

            if (attr_table->append_attr(mxGetName(mp),
                "String",str_rep) == 0) {
                delete [] str_rep;
                mxFreeMatrix(mp);
                matClose(fp);
                throw Error(string("Couldn't output attribute: ")
                            + mxGetName(mp));
            }

            delete [] str_rep;
        }
        else {
            das_table.add_table(mxGetName(mp), new AttrTable);
        }
        mxFreeMatrix(mp);
    }
    matClose(fp);

    return true;
}
\end{vcode}
  
In this example the server creates a single attribute container called
MAT\_GLOBAL and loads all the data set attributes into it.

\begin{quote}
  
  Most data set types (e.g., hdf) have both global attributes (those that
  apply to the entire data set) and attributes that only apply to
  a particular variable. Such data sets have both a global attribute container
  \emph{and} separate attribute containers for each variable\footnote{By
    convention, global attribute containers which hold values read from the
    data set has the suffix \_GLOBAL; the first part of the container's name
    is either read from the dataset or is some appropriate string chosen by
    the server.}. Matlab has only global attributes.

\end{quote}

Here's what is going on inside this function:

First a new attribute table object is created and added to the DAS
object. An attribute table (AttrTable) is similar to a structure in
that it holds other things which may be attributes (typed name-value
pairs) or other attribute tables. The DAS is a container for AttrTable
objects. Take a look at the documentation for the
\xlink{AttrTable}{http://www.unidata.ucar.edu/packages/dods/api/pref-html/AttrTable.html}
and \xlink{DAS}{
  http://www.unidata.ucar.edu/packages/dods/api/pref-html/DAS.html}
classes in the Programmer's Reference Guide.

\begin{vcode}{sib}
    AttrTable *attr_table = das_table.add_table("MAT_GLOBAL", new AttrTable);
\end{vcode}

Following the creation of a table for global attributes, the function handles
the routine and API-dependent tasks of opening the data set and setting
things up to iterate over its variables.

\begin{vcode}{sib}
    MATFile *fp = matOpen(filename.c_str(), "r");
    if (fp == NULL)
        throw Error(string("Could not open the file: ") + filename.c_str());

    // Read all the matrices in file
    Matrix *mp;
    while ((mp = matGetNextMatrix(fp)) != NULL) {
\end{vcode}

Inside the \lit{while}-loop that iterates over the data set's
variables, we test for string variables. The Matlab server assumes
that all string variables in a data set are actually global attributes
for the data set (and not `data' variables). If a string variable is
found, the code uses the variable's name and value as the attribute
name and value (code that handles the case where a variable is not a
string is explained further down).

Attributes are all string-valued in the DAP. That is, even though the
DAP supports the full range of scalar data types for attributes, the
\emph{values} are stored as strings. The call to
\lit{AttrTable::append\_attr} adds the attribute tuple (Name, type and
value) to the AttrTable instance.

Note that if \lit{AttrTable::append\_attr} fails, it returns zero and
the code cleans up and throws an exception. I elided that from this
sniplet to focus the example.

\begin{vcode}{sib}
        // String types are used as attributes
        if(mxIsString(mp)) {
            // get size
            int X = mxGetN(mp);
            int Y = mxGetM(mp);

            char *str_rep = new char [X*Y+3];
      
            // quote the string for parser
            *str_rep = '"'; 
            mxGetString(mp,str_rep+1,X*Y+1);
            *(str_rep + X*Y + 1) = '"';
            *(str_rep + X*Y + 2 )= '\0';

            if (attr_table->append_attr(mxGetName(mp), "String", str_rep) == 0) {
\end{vcode}

If the variable is not a string, then the Matlab server creates an empty
attribute table for it. This is to conform to the \xlink{DAP 2.0 draft
  specification}{http://www.unidata.ucar.edu/packages/dods/design/dap-rfc-html/dap_32.html}
which states that all variable must have an attribute table, even if it is
empty.

\begin{vcode}{sib}
        else {
            das_table.add_table(mxGetName(mp), new AttrTable);
        }
\end{vcode}



\section{Implementing the DataDDS object}

Building the DataDDS object handler follows the same pattern as before
with the DDS and the DAS. In fact, the DDS handler can be modified and
used as the DataDDS handler by simply changing the call to
\lit{DODSFilter::send\_dds()} to a call to \lit{send\_data()}.
However, before the \lit{send\_data()} method will work, we must
return to the data type child classes and add more functionality.

In the data type child classes you created we must now implement the method
\lit{read()}. This method will be called by the DAP software that sends data
values. To understand how the \lit{read()} method will be used, it's
instructive to look at the code that calls it. In the DAP data type classes,
each of the scalar, vector and constructor types has a method called
\lit{serialize()}. Below is shown
\xlink{Byte's}{http://www.unidata.ucar.edu/packages/dods/api/pref-html/Byte.html}
version of this method (Look at the code for
\xlink{DDS}{http://www.unidata.ucar.edu/packages/dods/api/pref-html/DDS.html}
if you'd like to see how \lit{serialize()} is used and pay particular
attention to \lit{DDS::send\_data()}).

\begin{quote}
Note: You don't have to write your own version of \lit{serialize()}, this is
shown here to provide you with some background information about the role of
the \lit{read()} method in building the DataDDS response.
\end{quote}

\begin{vcode}{sib}
bool
Byte::serialize(const string &dataset, DDS &dds, XDR *sink, bool ce_eval)
{
    if (!read_p())
        read(dataset);          // read() throws Error and InternalErr

    if (ce_eval && !dds.eval_selection(dataset))
        return true;

    if (!xdr_char(sink, (char *)&_buf))
        throw Error(
"Network I/O Error. Could not send byte data.\n\
This may be due to a bug in DODS, on the server or a\n\
problem with the network connection.");

    return true;
}
\end{vcode}

The \lit{serialize()} method is broken into three parts:
\begin{itemize}
\item Read the data for this variable if that has not already been done.
  We'll see later that when \lit{read()} successfully completes, it sets a
  boolean that the DAP library tests with the \lit{read\_p()} method. If
  \lit{read\_p()} returns \lit{true} then data for this variable has already
  been read\footnote{In some cases, data for a variable is read while
    evaluating the constraint expression. So it can be the case that the
    values for a variable are read before the \lit{serialize()} method calls
    \lit{read()}.}.
  
\item Next the constraint expression (CE) is evaluated. The constraint
  expression was passed to the server as part of the DataDDS request. Parsing
  the CE and ensuring it's evaluated correctly is handled for you by the DAP
  library.
  
\item Finally, if the CE evaluates to \lit{true}, then we serialize
  the binary data that was read into this instance with the
  \lit{read()} call at the beginning of this method.
\end{itemize}

\note{CE evaluation actually happens in two phases. In the phase, the
  expression is parsed. During this process, variables that are `projected'
  are marked as such and a linked list of `selection' nodes is built. The
  \lit{serialize()} method is called only for variables that are part of the
  projections (that is, that are to be sent back to the client). The second
  phase of CE evaluation happens inside \lit{serialize()} when the
  \lit{DDS::ce\_eval} method is used to evaluate the clauses. If all the
  clauses evaluate to true, then the current variable is sent.}

Here's the \lit{read()} method for the Matlab server's \lit{Array} class
(\lit{MATArray}). This is by far the most complex looking piece of code in
this tutorial, but it's really not very complicated once broken down.

\begin{vcode}{sib}
bool
MATArray::read(const string &dataset)
{
    if (read_p())  // Nothing to do
        return false;

    MATFile *fp = matOpen(dataset.c_str(), "r");
    if (fp == NULL)
        throw Error(string("Could not open the file: ") + dataset.c_str());
  
    Pix p = first_dim();
    int start = dimension_start(p,true);
    int stride = dimension_stride(p, true);
    int stop = dimension_stop(p, true); 

    next_dim(p);
    int start_p = dimension_start(p,true);
    int stride_p = dimension_stride(p, true);
    int stop_p = dimension_stop(p, true); 


    // get real part of the complex  matrix
    double *DataPtr;
    Matrix *mp;
    size_t pos;
    if ((pos = name().find("_Real")) != name().npos) {  
        string Rname = name().substr(0,pos);
        mp = matGetMatrix(fp,Rname.data());
        DataPtr = mxGetPr(mp); 
    }
    else{
        // get Img part of the complex matrix
        if ((pos = name().find("_Imaginary")) != name().npos) {  
            string Iname = name().substr(0,pos);
            mp = matGetMatrix(fp,Iname.data());
            DataPtr = mxGetPi(mp); 
        }
        else{
            mp = matGetMatrix(fp,name().data());
            DataPtr = mxGetPr(mp); // get the matrix structure
        }
    }

    if (DataPtr == NULL)
        throw Error(string("Error reading matrix"));

    if(start+stop+stride == 0){ //default rows
        start = 0;
        stride = 1;
        stop = mxGetM(mp)-1;
    }
    if(start_p+stop_p+stride_p == 0){ //default columns
        start_p = 0;
        stride_p = 1;
        stop_p = mxGetN(mp)-1;
    }

    int Len = (((stop-start)/stride)+1)*(((stop_p-start_p)/stride_p)+1);
  
    int Tcount = 0;
    dods_float64 *BufFlt64 = new dods_float64 [Len];    
  
    for (int row = start; row <= stop; row +=3Dstride) {          
        for(int column = start_p; column <= stop_p; column+=stride_p) {
            *(BufFlt64+Tcount) = (dods_float64)
            *(DataPtr+row+column*mxGetM(mp));  
            Tcount++;
        }
    }

    set_read_p(true);      
    val2buf((void *)BufFlt64);
    delete [] BufFlt64;
          
    mxFreeMatrix(mp);
    matClose(fp);
    return false;
}
\end{vcode}

First, on entry into the method, we check to see if the data have already
been read. This can happen if the data were previously needed for the
evaluation of the CE. Note that in an earlier version of the DAP library, the
return value of \lit{read()} was used to signal whether the method needed to
be called again to read more data (\lit{false} indicated that all the data
had been read). Now calls to \lit{read()} always get all the data, but the
return type is still \lit{bool} because older code checks the return value.
In software that uses 3.2 or newer, \lit{read()} should always exit by
returning \lit{false} unless it encounters an error, in which case it should
throw an exception.

\begin{vcode}{sib}
    if (read_p())  // Nothing to do
        return false;
\end{vcode}

If the data has not yet been read, the method then opens the Matlab
data set.  Each data source is different, but conceptually, this
action has to be performed somewhere. In some cases, the data set
would be opened once someplace else and the \lit{read()} methods
would access some sort of pointer or other access token.

\begin{vcode}{sib}
    MATFile *fp = matOpen(dataset.c_str(), "r");
    if (fp == NULL)
        throw Error(string("Could not open the file: ") + dataset.c_str());
\end{vcode}

Read the data for the variable from the data set. Again, this will vary with
each type of data set. In the Matlab server, a complex matrix is represented
in the DAP by two different matrices, once with the suffix \lit{\_Imaginary}
and one with the suffix \lit{\_Real}. This code looks at the name of the
variable and uses that to find the correct variable and read its values. More
complex data sets will probably need a more sophisticated lookup scheme.

\begin{vcode}{sib}
    // get real part of the complex  matrix
    double *DataPtr;
    Matrix *mp;
    size_t pos;
    if ((pos = name().find("_Real")) != name().npos) {  
        string Rname = name().substr(0,pos);
        mp = matGetMatrix(fp, Rname.data());
        DataPtr = mxGetPr(mp); 
    }
    else{
        // get Img part of the complex matrix
        if ((pos = name().find("_Imaginary")) != name().npos) {  
            string Iname = name().substr(0,pos);
            mp = matGetMatrix(fp, Iname.data());
            DataPtr = mxGetPi(mp); 
        }
        else{
            mp = matGetMatrix(fp,name().data());
            DataPtr = mxGetPr(mp); // get the matrix structure
        }
    }

    if (DataPtr == NULL)
        throw Error(string("Error reading matrix"));
\end{vcode}

Once the data have been read from the data set we need to check for
sub-sampling that may have been specified by the client and passed to the
server via the CE. The CE was automatically parsed by the boilerplate code,
but we need to explicitly look at the values because data set types are
fairly idiosyncratic about how they use this information.

\begin{quote}
While the Matlab server reads the entire array from the dataset and then
applies the sub-sampling information, \emph{many} data set types provide ways
to subsample variables through their own API (e.g., HDF, NetCDF, \ldots). In
such cases, you'd read the CE information, then use it to read the data
values. The most important point is that \emph{you don't always have to read
  the entire variable from the data when using the DAP}. In fact, most
servers don't, they make sure to use the data set's underlying API in the
most efficient way possible, something that the DAP was designed to make
possible.
\end{quote}

Since the Matlab server supports only Matlab 5 and since Matlab 5 supported
only two dimensional matrices, we grab a pointer to the first and second
dimensions using the \lit{Array::first\_dim()} and \lit{Array::next\_dim()}
methods. The \lit{Array::dimension\_start()}, \lit{dimension\_stride()} and
\lit{dimension\_stop()} methods are used to the read the start and stop
indices and the sub-sampling stride for the dimension referenced by the
\lit{Pix} \lit{p}.

\begin{vcode}{sib}
    Pix p = first_dim();
    int start = dimension_start(p,true);
    int stride = dimension_stride(p, true);
    int stop = dimension_stop(p, true); 

    next_dim(p);
    int start_p = dimension_start(p,true);
    int stride_p = dimension_stride(p, true);
    int stop_p = dimension_stop(p, true); 

    if(start+stop+stride == 0){ //default rows
        start = 0;
        stride = 1;
        stop = mxGetM(mp)-1;
    }
    if(start_p+stop_p+stride_p == 0){ //default columns
        start_p = 0;
        stride_p = 1;
        stop_p = mxGetN(mp)-1;
    }
\end{vcode}

Using the information from the CE, the array values are sub-sampled and copied
to new storage. Again, this step is generally not necessary when it's
possible to subsample variables in the data set using an API, et cetera.

\begin{vcode}{sib}
    int Len = (((stop-start)/stride)+1)*(((stop_p-start_p)/stride_p)+1);
  
    int Tcount = 0;
    dods_float64 *BufFlt64 = new dods_float64 [Len];    
  
    for (int row = start; row <= stop; row += stride) {          
        for(int column = start_p; column <= stop_p; column+=stride_p) {
            *(BufFlt64+Tcount) = (dods_float64) *(DataPtr+row+column*mxGetM(mp));  
            Tcount++;
        }
    }
\end{vcode}

The values, now read and sub-sampled are copied into the DAP variable object.
The DAP methods enforce a strict policy that all memory allocated outside of
the library must be deleted outside the library, and vice versa. So values
sorted in the \lit{BufFlt64} array are copied to new storage allocated inside
the \lit{Array} object. This code deletes the \lit{BufFlt64} array.

Also important is the call to \lit{set\_read\_p()} with the value \lit{true}
this sets the \lit{read\_p} property\footnote{The the `p' stands for
  `Predicate'; maybe not the best naming convention given the prevalence of
  pointers in $C++$ software.} so that, should this function be run again
while building this particular response, it will know the data have already
been read.
 
\begin{vcode}{sib}
    val2buf((void *)BufFlt64);
    delete [] BufFlt64;
    set_read_p(true);      
\end{vcode}

The remaining software frees resources allocated via the Matlab data access
API. As was explained earlier, the \lit{false} return value is a hold over
from an earlier version of the DAP library. 

\begin{vcode}{sib}
    mxFreeMatrix(mp);
    matClose(fp);
    return false;
\end{vcode}

To build a \lit{main()} function that will return the DataDDS, copy the one
for the the DDS but change the call for \lit{DODSFilter::send\_dds()} to
\lit{DODSFilter::send\_data()}. This will work because the code leading up
to the \lit{send\_data()} call builds the DDS object, then the
\lit{send\_data()} call will arrange to build the DataDDS response using the
DDS. During this process it will parse and evaluate the CE and call the
\lit{read()} methods for the variables in the DDS. Thus the DDS will contain
variables loaded with values which can then be used to create the information
in the DataDDS object/response.

\section{Notes}

Here's a collection of information that might be important to specific
servers but is hard to fit into a general tutorial.

\begin{itemize}

\item It's possible to replace the three separate programs, one for the DAS,
  DDS and DataDDS, with a single program. Clearly there all share close to
  90\% of the same code. You can tell which object is being requested by
  either using the old Unix trick of making the programs symbolic links to a
  single executable and then checking to see which name was used to invoke
  it, or by modifying the dispatch script slightly to pass that information
  along with the other parameters.

\item Similarly, you can subclass the DAS and DDS objects adding methods to
  them to build up the attributes and variables instead of writing functions
  to do that task. 

\item For complex data sets, it's likely that you'll need to create some new
  fields in the data type classes you specialized.

\item If you're writing a server that will principally be for internal use, or
  you expect many many small accesses, you should look into using the Java
  software or implementing your server as an Apache module. Contact support
  or the dods-tech email list for help with the latter.
  
\item You can get help from the \xlink{DODS web
    site}{http://unidata.ucar.edu/packages/dods/}, the \xlink{dods-tech email
    list}{http://www.unidata.ucar.edu/packages/dods/home/mailLists/subscribe-dods-tech.html}
  and the \xlink{DODS user support desk}{mailto:support@unidata.ucar.edu}.
  
\item The DODS Project has developed two tools to help with serving datasets
  that contain many files. The first is to set up a `file server' a kind of
  catalog of URLs that is itself a DODS data set. The second is called the
  Aggregation Server (AS). The AS can automatically aggregate discrete
  datasets, accessed as either as files (in some cases) or URLs to produce a
  single data set. See the \xlink{DODS web
    page}{http://unidata.ucar.edu/packages/dods/} and/or contact \xlink{User
    Support}{mailto:support@unidata.ucar.edu} for help with this.

\item When you're returning error messages, be sure to trim file pathnames so
  that the real path to data is hidden.

\item If you have a data set and its API requires you to store a lot of state
  information that's not specific to a given variable or data type, consider
  creating a mixin class and making your data type children inherit from it in
  addition to the DAP classes. Because the `virtual contractors' are used to
  create instances of the data type classes, you can be sure that in your
  server, all instances of Byte, etc., contain the mixin. Use the
  \lit{dynamic\_cast} operator to access the mixin's methods.

\end{itemize}

   
\end{document}

% $Log: writing_server.tex,v $
% Revision 1.11  2003/10/15 15:26:29  jimg
% check point
%
% Revision 1.10  2002/05/13 22:00:08  jimg
% I made a bunch of edits while flying to BWI (you know you've got it
% bad when you use the airport's initials...). I hope I fixed things more
% than I broke them.
%
% Revision 1.9  2002/05/13 19:34:43  tom
% changed html directory
%
% Revision 1.8  2002/05/13 02:50:27  tom
% more grammar
%
% Revision 1.7  2002/05/13 02:17:08  tom
% fixed author credit
%
% Revision 1.6  2002/05/13 01:56:52  jimg
% Fixed the places marked by Tom.
%
% Revision 1.5  2002/05/13 00:38:52  tom
% grammar checking, etc etc
%
% Revision 1.4  2002/05/11 22:40:27  jimg
% Finished. Spell cheecked. Not read over, though.
%
% Revision 1.3  2002/05/10 23:26:25  jimg
% Added text for the DAS section.
%
% Revision 1.2  2002/05/10 21:00:17  jimg
% I ran a spell checker on this.
%
% Revision 1.1  2002/05/10 17:17:05  tom
% converted from html, checked in.
%

%%% Local Variables: 
%%% mode: latex
%%% TeX-master: t
%%% End: 
