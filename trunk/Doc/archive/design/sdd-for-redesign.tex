%
% $Id$
%
% This is a SDD template for software that has already been implemented but
% needs significant redesign and implementation. 7/19/2000 jhrg

\documentclass{article}
\usepackage{epsfig}
\usepackage{rotating}
\usepackage{subfigure}
\usepackage{vcode}
\usepackage{xspace}

% Change paragraph typesetting; eliminate indenting and add more space between
% paragraphs. 2/15/2000 jhrg
\setlength{\parindent}{0em}     % Amount of indentation
\addtolength{\parskip}{1ex}     % Vertical separation

% Sample macros including a URL which can be hyphenated. 
\newcommand{\Cpp}{\rm {\small C}\raise.5ex\hbox{\footnotesize ++}\xspace}
\newcommand{\dap}{\rm {\small DAP}\raise.5ex\hbox{\footnotesize ++}\xspace}
\newcommand{\maewesturl}{http://maewest.gso.uri.edu/\-cgi-bin/\-nph-dsp/\-htn\_sst\_decloud/\-1992/\-i92098065016.htn\_d.Z\xspace}

\begin{document}

\title{}
\author{James Gallagher\thanks{The University of Rhode Island,
    jgallagher@gso.uri.edu}}
\date{\today \\ $Revision$ }

\maketitle
\tableofcontents

\section{Introduction}

Section~\ref{sec:arch} describes the architecture of loaddods, including how
loaddods and writeval work together.

Section~\ref{sec:requirements} describes the new requirements.

Section~\ref{sec:design} describes how those requirements are satisfied.

\section{Architecture}
\label{sec:arch}

% Sample figure. For tables with their own footnotes, use the minipage
% environemnt. See the Latex companion.
\begin{figure}
\begin{center}
\epsfig{file=.eps,width=5in}
\caption{}
\label{fig:loaddods-component}
\end{center}
\end{figure}

\section{Requirements}
\label{sec:requirements}

\section{Design and Implementation}
\label{sec:design}

\clearpage
\appendix

\section{ChangeLog}
\begin{verbatim}

$Log: sdd-for-redesign.tex,v $
Revision 1.2  2000/07/20 23:55:05  jimg
*** empty log message ***


\end{verbatim}

\end{document}
