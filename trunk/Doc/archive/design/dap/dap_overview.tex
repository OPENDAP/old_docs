%
% Documentation for the DAP. Intended to be like an RFC document.
%

\documentclass[justify]{dods-paper}
\usepackage{acronym}
\usepackage{xspace}
\usepackage{gloss}

% latex and HTML macros. Some latex commands become nops for HTML. 4/10/2001
% jhrg 
\T\newcommand{\Cpp}{\rm {\small C}\raise.5ex\hbox{\footnotesize++}\xspace}
\T\newcommand{\C}{\rm {\small C}\xspace}
\W\newcommand{\Cpp}{C++}
\W\newcommand{\C}{C\xspace}
\W\newcommand{\cdots}{...}
\W\newcommand{\ddots}{}
\W\newcommand{\vdots}{.}
\W\newcommand{\pm}{+/-}
\W\newcommand{\times}{*}
\W\newcommand{\uppercase}[1]{\textsc{#1}}
\T\newcommand{\qt}{\lit{\char127}}
\W\newcommand{\qt}{"}

\texorhtml{\def\rearrangedate#1/#2/#3#4{\ifcase#2\or January\or February\or
  March\or April\or May\or June\or July\or August\or September\or
  October\or November\or December\fi\ \ifx0#3\relax\else#3\fi#4, #1}
\def\rcsdocumentdate{\expandafter\rearrangedate\rcsInfoDate}}%
{\HlxEval{
(put 'rearrangedate       'hyperlatex 'hyperlatex-ts-rearrange-date)

(defun hyperlatex-ts-rearrange-date ()
  (let ((date-string (hyperlatex-evaluate-string 
                       (hyperlatex-parse-required-argument))))
    (let ((year-string (substring date-string 0 4))
          (month-string (substring date-string 5 7))
          (day-string (substring date-string 8 10))
          (month-list '("January" "February" "March" "April"
                        "May" "June" "July" "August" 
                        "September" "October" "November" "December")))
       (insert
         (concat (elt month-list (1- (string-to-number month-string)))
                 " " (int-to-string (string-to-number day-string))
                 ", " year-string)))))
}
\newcommand{\rcsdocumentdate}{\rearrangedate{\rcsInfoDate}}}

\newcommand{\dapversion}{Version 4.0\xspace}
\newcommand{\opendap}{OPeNDAP\xspace}
\newcommand{\DAP}{DAP\xspace}
\newcommand{\DODS}{DODS\xspace}
\newcommand{\NVODS}{NVODS\xspace}
\newcommand{\CE}{constraint expression\xspace}
\newcommand{\CEs}{constraint expressions\xspace}
\newcommand{\ErrorX}{ErrorX\xspace}
\newcommand{\CapX}{Server Capabilities Document\xspace}
\newcommand{\Blob}{Blob\xspace}
\newcommand{\DDX}{DDX\xspace}
\newcommand{\DAX}{DAX\xspace}
\newcommand{\DDS}{DDS\xspace}
\newcommand{\DAS}{DAS\xspace}
\newcommand{\URI}{URI\xspace}
\newcommand{\DataDDS}{DataDDS\xspace}

\newcommand{\type}[1]{\emph{#1}}
\newcommand{\Alias}{\type{Alias}\xspace}
\newcommand{\Array}{\type{Array}\xspace}
\newcommand{\Attribute}{\type{Attribute}\xspace}
\newcommand{\ProcAttribute}{\type{Processing Attribute}\xspace}
\newcommand{\Map}{\type{Map}\xspace}
\newcommand{\Target}{\type{Target}\xspace}
\newcommand{\FQN}{fully qualified name\xspace}
\newcommand{\Grid}{\type{Grid}\xspace}
\newcommand{\Structure}{\type{Structure}\xspace}
\newcommand{\Dataset}{\type{Dataset}\xspace}
\newcommand{\Sequence}{\type{Sequence}\xspace}
\newcommand{\Container}{\type{Container}\xspace}
\newcommand{\Bi}{\type{Binary Image}\xspace}
\newcommand{\String}{\type{String}\xspace}
\newcommand{\URL}{\type{URL}\xspace}
\newcommand{\Boolean}{\type{Boolean}\xspace}
\newcommand{\Byte}{\type{Byte}\xspace}
\newcommand{\Enum}{\type{Enumeration}\xspace}
\newcommand{\Time}{\type{Time}\xspace}
\newcommand{\Function}{\type{Function}\xspace}
\newcommand{\Description}{\type{Description}\xspace}
\newcommand{\Parameter}{\type{Parameter}\xspace}
\newcommand{\Constraint}{\type{Constraint}\xspace}
\newcommand{\NoAttributes}{\type{NoAttributes}\xspace}
\newcommand{\Project}{\type{Project}\xspace}
\newcommand{\Select}{\type{Select}\xspace}
\newcommand{\Hyperslab}{\type{Hyperslab}\xspace}

\newcommand{\Aliases}{\type{Aliases}\xspace}
\newcommand{\Arrays}{\type{Arrays}\xspace}
\newcommand{\Attributes}{\type{Attributes}\xspace}
\newcommand{\ProcAttributes}{\type{Processing Attributes}\xspace}
\newcommand{\Maps}{\type{Maps}\xspace}
\newcommand{\Targets}{\type{Targets}\xspace}
\newcommand{\FQNs}{fully qualified names\xspace}
\newcommand{\Grids}{\type{Grids}\xspace}
\newcommand{\Structures}{\type{Structures}\xspace}
\newcommand{\Sequences}{\type{Sequences}\xspace}
\newcommand{\Containers}{\type{Containers}\xspace}
\newcommand{\Bis}{\type{Binary Images}\xspace}
\newcommand{\Strings}{\type{Strings}\xspace}
\newcommand{\URLs}{\type{URLs}\xspace}
\newcommand{\Booleans}{\type{Booleans}\xspace}
\newcommand{\Bytes}{\type{Bytes}\xspace}
\newcommand{\Enums}{\type{Enumerations}\xspace}
\newcommand{\Times}{\type{Times}\xspace}
\newcommand{\CSVs}{comma-separated values\xspace}
\newcommand{\Functions}{\type{Functions}\xspace}
\newcommand{\Descriptions}{\type{Descriptions}\xspace}
\newcommand{\Parameters}{\type{Parameters}\xspace}
\newcommand{\Constraints}{\type{Constraints}\xspace}
\newcommand{\Projects}{\type{Projects}\xspace}
\newcommand{\Selects}{\type{Selects}\xspace}
\newcommand{\Hyperslabs}{\type{Hyperslabs}\xspace}

\newcommand{\DIR}{\textbf{Directory}\xspace}
\newcommand{\TEXT}{\textbf{Text}\xspace}
\newcommand{\HTML}{\textbf{HTML}\xspace}
\newcommand{\HELP}{\textbf{Help}\xspace}
\newcommand{\VER}{\textbf{Version}\xspace}
\newcommand{\INFO}{\textbf{Info}\xspace}

%%%%%%%%%%%%%% Web Services Paper
\newcommand{\GetDDX}{\textbf{GetDDX}\xspace}
\newcommand{\GetData}{\textbf{GetData}\xspace}
\newcommand{\GetBlobData}{\textbf{GetBlobData}\xspace}
\newcommand{\GetBlob}{\textbf{GetBlob}\xspace}
\newcommand{\GetDir}{\textbf{GetDir}\xspace}
\newcommand{\GetInfo}{\textbf{GetInfo}\xspace}

\newcommand{\FSs}{\type{Foundation Services}\xspace}
\newcommand{\FS}{\type{Foundation Service}\xspace}
\newcommand{\ODSN}{\type{OPeNDAP}\xspace}

\newcommand{\blobdataelement}{\texttt{BlobData} element\xspace}
\newcommand{\blobelement}{\texttt{Blob} element\xspace}



%\setcounter{secnumdepth}{4}
%\setcounter{tocdepth}{4}

\newcommand{\Tableref}[1]{Table~\ref{#1}}%
\newcommand{\Figureref}[1]{Figure~\ref{#1}}%
\W\begin{iftex}
\newcommand{\Sectionref}[2]{Section~\ref{#1}%
  \ifx#2)%
    \ on page~\pageref{#1}%
  \else% 
    \ (page~\pageref{#1})\xspace%
  \fi\ifx#2\space\ \else #2\fi}%
\W\end{iftex}
\W\newcommand{\Sectionref}[1]{Section~\ref{#1}}
\W\newcommand{\raggedright}{}


%% Conveniences for documenting XML
\newcommand{\tag}[1]{\emph{#1}}
\newcommand{\element}[1]{\link{\tag{#1}}{sec-xml-#1}}
\newcommand{\attribute}[1]{\emph{#1}}
\newcommand{\currentelement}{}
\newcommand{\ELEMENT}[1]{\renewcommand{\currentelement}{#1 element}%
  \subsubsection{#1}\label{sec-xml-#1}\indc{\currentelement}%
  \indc{catalog tag!#1}\indc{aggregation tag!#1}%
  \indc{XML!#1 element}}
\newcommand{\ATTRIBUTE}[1]{\item{\lit{#1}}\indc{\currentelement!#1}
    \indc{#1 attribute!of \currentelement}%
    \indc{XML!#1 attribute}}

% Conveniences for examples
\newcounter{exampleno}
\setcounter{exampleno}{0}
\newcounter{examplerefno}
\setcounter{examplerefno}{0}
\newcommand{\examplelabel}[1]{\refstepcounter{exampleno}\label{#1}%
  \medskip Example \theexampleno :\smallskip}
\newcommand{\exampleref}[1]{\texorhtml{Example~\ref{#1}%
    \refstepcounter{examplerefno}\label{exref\theexamplerefno}%
    % This is a test whether r@exref... is defined.  If not, skip
    % anything with the \pageref macro.
    \catcode`\@=11%
    \expandafter\ifx\csname r@exref\theexamplerefno\endcsname\relax\else%
    \expandafter\ifx\csname r@#1\endcsname\relax\else%
    \bgroup\count100=\pageref{exref\theexamplerefno}%
    \count101=\pageref{#1}\ifnum\count100=\count101\else~%
    on page~\pageref{#1}\fi\egroup\fi\fi\xspace}%
  {\link{Example \ref{#1}}{#1}}}%
\T\setlength{\vcodeindent}{10pt}
\texorhtml{
%% LaTeX version
\newenvironment{textoutput}[1]{\ifx #1\relax%
  \medskip Output:\vspace{-\medskipamount}\else%
  \medskip #1\vspace{-\medskipamount}\fi%
  \begin{list}{}{\setlength{\leftmargin}{\vcodeindent}}\begin{ttfamily}\item}%
 {\end{ttfamily}\end{list}} %
}{%% Hyperlatex version
\newenvironment{textoutput}[1]{\xml{blockquote}Output:\\ \\ \xml{tt}}%
  {\xml{/tt}\xml{/blockquote}}%
\newenvironment{minipage}[4]{}{}
\newenvironment{ttfamily}{\xml{blockquote}\xml{tt}}%
  {\xml{/tt}\xml{/blockquote}}}

\newcommand{\DAPOverviewTitle}{DAP Specification Overview}
\newcommand{\DAPOverview}{\xlink{\textbf{\textit{\DAPOverviewTitle}}}%
  {dap.html}\xspace}

% Old macros; new values
\newcommand{\DAPObjectsTitle}{The Data Access Protocol---DAP 2.0}
\newcommand{\DAPObjects}{\xlink{\textbf{\textit{\DAPObjectsTitle}}}%
  {http://www.opendap.org/pdf/ESE-RFC-004v0.06.pdf}\xspace}
  
\newcommand{\DAPHTTPTitle}{Using DAP 2.0 with HTTP}
\newcommand{\DAPHTTP}{\xlink{\textbf{\textit{\DAPHTTPTitle}}}%
  {daph.html}\xspace}

% New macros.
\newcommand{\DAPDataModelTitle}{DAP Data Model Specification}
\newcommand{\DAPDataModel}{\xlink{\textbf{\textit{\DAPDataModelTitle}}}%
  {dapo.html}\xspace}
  
\newcommand{\DAPWebTitle}{DAP Web Services Specification}
\newcommand{\DAPWeb}{\xlink{\textbf{\textit{\DAPWebTitle}}}%
  {daph.html}\xspace}
  

\newcommand{\DAPASCIITitle}{DAP Formatted Data Specification}
\newcommand{\DAPASCII}{\xlink{\textbf{\textit{\DAPASCIITitle}}}%
  {dapa.html}\xspace}
\newcommand{\DAPHTMLTitle}{DAP HTTP Query Specification}
\newcommand{\DAPHTML}{\xlink{\textbf{\textit{\DAPHTMLTitle}}}%
  {dapm.html}\xspace}

% It probably doesn't matter what we call the macro, but SOAP does not have
% to run over HTTP and I think that's going to be important for other groups.
% 10/27/03 jhrg
\newcommand{\DAPServicesTitle}{DAP HTTP Services Specification}
\newcommand{\DAPServices}{\xlink{\textbf{\textit{\DAPServicesTitle}}}%
  {daps.html}\xspace}

\newcommand{\thirtytwobitlimit}[1]{$4,294,967,296$ #1 ($2^{32}$)}
%%% Local Variables: 
%%% mode: latex
%%% TeX-master: t
%%% End: 


\rcsInfo $Id$

% Note: to get the glossary to work, run bibtex on the *.gls.aux file,
% then latex the file, then bibtex *.gls, then latex again. Also, make
% sure to set your BST and BIBINPUTS environment variables so that the
% BST and BIB files will be found.
% \makegloss

\title{\DAPOverviewTitle\\ DRAFT}
\htmltitle{\DAPOverviewTitle\ -- DRAFT}
\author{James Gallagher\thanks{The University of Rhode Island,
    jgallagher@gso.uri.edu}}
\date{Printed \today \\ Revision \rcsInfoRevision}
\htmladdress{James Gallagher <jgallagher@gso.uri.edu>, \rcsInfoDate, 
  Revision: \rcsInfoRevision}
\htmldirectory{html}
\htmlname{dap}

\bibliographystyle{plain}

\begin{document}

\maketitle
\T\tableofcontents

\section{Organization of the Specification}

This is the specification of the \DAP, \dapversion.\footnote{Whenever
  `Data Access Protocol' or `DAP' is used, the reader should assume
  \dapversion of the protocol is being discussed. If a feature of
  another version is being discussed, that version will be explicitly
  given.}  The specification is divided among several documents, each
describing a different aspect of the protocol.  The documents are
arranged in a more-or-less concentric manner.  That is, an
implementation of the \DAP can satisfy the requirements in one or more
of the documents.

\begin{description}

\item[\DAPObjects] All implementations of the \DAP, regardless of the
  communication protocol used, must satisfy the requirements specified
  in this document.

\item[\DAPHTTP]  Implementations of the \DAP using HTTP as the
  communication protocol must satisfy the requirements of this
  document. 

\item[\DAPASCII] HTTP implementations of the \DAP that provide
  ASCII-formatted data should satisfy the requirements of this
  document.

\item[\DAPHTML] HTTP implementations of the \DAP that provide
  an interactive query mechanism should satisfy the requirements of
  this document.

%\item[\DAPServices] HTTP implementations of the \DAP that provide
%  information about the datasets served by a server should satisfy the
%  requirements of this document.

\end{description}

This is a complete list of the specification documents issued by the
\opendap\ group, as of \rcsdocumentdate.  Other specification documents,
referring to implementations using different communication protocols,
or providing different services, may be available from other groups.

\section{Introduction}

The \DAP is a presentation- and application-level protocol
\cite{IEC94} for access to distributed data organized as
name-type-value tuples. While the name-type-value model is a nearly
universal \emph{conceptual} organization of data, the actual
organization takes nearly as many forms as there are individual
collections. Nonetheless, information stored in a computer can often be
broken down into a set of variables that have names, data types and
values.  For example, data from a satellite might be stored in a named
array of bytes.  

Complicating the picture, however, there are many different file
formats, APIs and file/directory organizations used to house data. The
\DAP was designed to hide the implementation of different collections
of data behind a simple language-like interface based on the
name-type-value conceptual model.

The \DAP provides ways to sub-set the holdings of data sources so that
only the interesting parts are accessed. This makes the \DAP ideal for
remote access to data sources that are too large to replicate onto many
machines and to situations where many different data sources are used in
combination. In both cases it is both necessary to keep the data remote
(spreading the burden of its storage and management) and to be able to
selectively retrieve information.

\section{Requirements}

The key words ``MUST'', ``MUST NOT'', ``REQUIRED'', ``SHALL'', ``SHALL NOT'',
``SHOULD'', ``SHOULD NOT'', ``RECOMMENDED'', ``MAY'' and ``OPTIONAL'' in this
document are to be interpreted as described in RFC 2119~\cite{rfc2119}.

An implementation is not compliant if it fails to satisfy one or more of the
``MUST'' or ``REQUIRED'' level requirements for the protocols it implements.
An implementation that satisfies all of the ``MUST'' or ``REQUIRED'' level
and all of the ``SHOULD'' level requirements for its protocols is said to be
`unconditionally compliant'; one that satisfies all the ``MUST'' level
requirements but not all the ``SHOULD'' level requirements for its protocols
is said to be `conditionally compliant.'

\section{Overall Operation}

The \DAP is actually a collection of protocols that
provide a way to describe and access a remote data source. The
information within a data source is described using the \DAP Data
Model and those descriptions are held in objects defined by the \DAP
Objects specification.  The Data Model and Objects define an
interface to the information in a data source.  The interface may be
operated in different ways, depending on the implementation.

The \DAP does not define what constitutes a data source. Instead it
describes an interface to the information in a data source. A data
source can be anything for which this interface can be meaningfully
used. For example, a data source might be a single data file and a set
of programs which map the contents of that file onto the \DAP data
model and objects.  A data source might also be a collection of files
which can be logically grouped together and treated as a single entity
or it might be all or part of a relational data base. Literally any
collection of information that can be mapped onto the \DAP data model
and objects qualifies as a \DAP data source.

\subsection{The \DAP Data Model}

% Describe a data set as variables and attributes.

The \DAP models data sources as a collection of variables, each of
which has a name, a type, one or more values and zero or more
\new{attributes}.\footnote{A variable \emph{is} simply the binding
  between names, data types, and values.}  A client can interrogate
the data source to find out the names of its variables, their types,
and their attributes and it can then use that information to access
the values.  The \DAP defines a basic collection of data types
commonly found in general programming environments.  These include
integer and floating point number types, character strings and URLs
(atomic types), arrays, structures, sequences, grids (container types)
and binary images.\footnote{The \DAP versions 1 and 2 included a List
  data type which has been dropped because it can be implemented using
  Sequence.  \dapversion adds a binary image data type so that
  arbitrary collections of bytes can be manipulated. Binary images are
  unsized arrays of bytes and should not be confused with digital
  image formats such as JPEG.}  The container types Sequence,
Structure and Grid can be used to create different aggregations of the
types. Because there are few limits on ways the types can be combined,
there are very few data sources which cannot be described using the
\DAP's data types.


% Describe attributes

Like the variables they describe, the attributes corresponding to a
variable are also name-type-value tuples.  Attribute types are limited
to the atomic types, vectors (one dimensional arrays) of the atomic
types and structures (which are called \new{containers} for historical
reasons). Each variable must have at least one \new{attribute
  container} associated with it; that container holds the variable's
attributes. There may be zero or more attributes in the container,
including other containers. Beyond this there are no requirements for
a variable's attributes.

Attributes provide a way to bind information beyond a name and type to
a variable. While the name and type associated with a variable defines
all that a computer needs to know to store and retrieve the values
associated with a variable, attributes provide a way to store other
information which is generally required to \emph{use} the values of a
particular variable.  For example, a variable might hold a two
dimensional array of bytes which represent magnitudes derived from a
satellite-borne sensor.  All that a computer program need know to read
the data are the dimensions of the array and the type of its
constituents.  However, to make effective use of those values, a
person (or a program) must know what units those magnitudes represent
(say degrees Celsius), how to scale the byte values to real numbers,
or even just the proper way to spell the name of this kind of data
(e.g. ``T,'' ``Temperature,'' or ``Sea Surface Temperature'').
Another way to think about the difference between variables and
attributes is that the name-type-value properties of a variable are
not dependent on any particular use of the data, while the values of
attributes generally are.\footnote{In other documents the name and
  type of a variable are called \new{syntactic metadata} while
  that variable's attributes are referred to as
  \new{semantic metadata}.}

Access to values from a data source is accomplished by requesting one
or more variables. Because the \DAP was designed for efficient access
to a remote data source, it provides a set of sub-setting operators
which may be applied to each requested variable.  The exact operators
vary with the type of the data variable to be sampled; some provide
ways to request parts of arrays (sometime referred to as hyper-slabs)
while others provide ways to access values using relational
expressions.

A complete description of the \DAP data model can be found in
  \DAPObjects .


%\subsection{Services}
%\label{sec:services}

%The distinction between a \emph{service} and an
%\emph{object} is that the latter is part of the foundation on which the
%\DAP is built. A service is a way of accessing
%information from a server.  It might return the network representation of one
%of the objects or it might return data in ASCII ready to be imported into a
%spreadsheet.

%The \DAP Services protocol defines how programs interact with a remote
%data source. Each potential response from a remote data source is considered
%a service that data source offers. \textbf{SOAP exceptions \& our Error
%  object; a service???} Thus, when a program requests the \ac{DDX} object
%from a server and receives the XML-encoded persistent representation of that
%object, it is using a service just as it is when it requests the \ac{HTML}
%interface for the data source. This is one departure from the \DAP
%version 2 specification; all responses are explicitly described as services.

%In \DAP versions 1 and 2, interaction with a data source was assumed to
%take place using \ac{HTTP}. In \DAP version 3, interaction can take place
%using any transport protocol. Server and client toolkits implementing the
%\DAP are encouraged to take advantage of certain \ac{HTTP} features when
%that is the transport protocol. These include the reuse of socket connections
%between communicating processes and local (i.e., client-side) caching.

%The Services defined by \DAP version 3 are:

%Responses which contain the persistent representations of the four objects
%defined by \DAP version 3, encoded using XML or XDR as noted:

%\begin{description}
%\item[DDX] The \ac{DDX} object. XML.
%\item[Blob] Data values. XDR.
%\item[DAX] Ancillary information. XML.\footnote{I'm not sure if this is really a service\ldots jhrg}
%\item[ErrorX] Errors. XML.\footnote{Or this\ldots jhrg}
%\end{description}

%Responses which contain the persistent representations of the objects defined
%by versions 1 and 2 of the \DAP. These responses are ASCII documents or
%mixed-content (ASCII and application/octet-stream); object information is
%represented using a curly-brace notation similar to the C programming
%language. These are included in version 3 to provide backward compatibility
%for older clients. Given the similarity between these responses and the
%objects in version 3, generating them should be easy.

%\begin{description}
%\item[DDS] The \ac{DDS} object. ASCII.
%\item[DAS] The \ac{DAS} object. ASCII.
%\item[DataDDS] A pseudo-multi-part \acs{MIME} document that contains a
%    \ac{DDS} and a Blob. ASCII and XDR-encoded binary.
%\item[Error] Errors. ASCII. This differs from the ErrorX response because it
%  uses the curly-brace encoding scheme (ErrorX uses XML). It
%  also defines a slightly different set of error codes.
%\end{description}

%The remaining responses provide additional ways clients can interact with a
%data source. Some of these responses are optional and some are only loosely
%defeined.

%\begin{description}
%\item[ASCII] Get data values encoded as \ac{CSV} ASCII. This is designed to
%  be easy to parse, view in a web browser and import to a spreadsheet.
%  Required. 
%\item[Version] Return the \DAP protocol version. This may also include
%  the implementation name and version. Required.
%\item[Capabilities] Return a description of the features supported by this
%  server. Required.
%\item[HTML] An \ac{HTML} interface to the data source. This should display an
%  \ac{HTML} form, or other interactive user interface that can be rendered by
%  a web browser, which provides a way for people to examine the variables in
%  a particular data source.
%\item[Directory] Return information about different data sources accessible
%  through a particular server.
%\item[Info] Return an \ac{HTML} document that presents the information in the
%  \ac{DDX} object.
%\end{description}



%\printgloss{dods-glossary}

\T\addcontentsline{toc}{section}{References}
\T\raggedright
\bibliography{../../../boiler/dods}

\appendix

\section{Brief History}

The \DAP has been used by \DODS since 1996, \NVODS since 1999 and
several other projects. The first version of the protocol dealt only
with the three pseudo-objects used by DAP 1.0 to describe information
sources. Version 2.0 added constraint expressions to provide a way to
extract part of a data source. In addition, version 2.0 added services
that integrate the \DAP more with the rest of the \ac{WWW}.  
% Delete this if we go with version 3.0
Version 3.0 began the process of moving data descriptions to the XML
syntax. 

% XBrowse and JGOFS

The origin of the \DAP has its roots in the \DODS project. Work on
\DODS started in late 1993 after experiments to combine URI's XBrowse
software for remote access to satellite imagery with MIT's JGOFS data system
seemed to reach a point where further work would best proceed by developing a
completely new data system that combined the best aspects of both systems.

% RPC, netCDF and JGOFS

The first protocol used by \DODS was based on the netCDF and JGOFS APIs.
These APIs were reimplemented using Sun Microsystem's \ac{RPC} software. It
was clear that designing \DODS so that it could use existing programs as
clients was a powerful idea, because it meant a data system could be built
by leveraging the effort already put into existing software. However, as
Waldo, et al.~\cite{waldo:dist-comp} point out, schemes which hide network
access from programmers are brittle. In addition, the RPCized versions of
netCDF and JGOFS could not interoperate.

% Version 1, no CEs, a way to deal with translation

In early 1995 the \DODS project decided to scrap the initial
implementation based on \ac{RPC}s and develop a new mechanism based on
a more universal data model. This new mechanism was designed to
encompass the types of data found in XBrowse, netCDF and JGOFS and was
based on common data types found in most programming languages. The
`variables and attributes' organizational scheme found in netCDF and
HDF 4 (along with programming languages such as Lisp) was adopted as a
way to characterize data sources. In addition, the mechanism was
designed to minimize the number of interactions that would be required
between a client program and a remote data server.  Furthermore, the
\ac{RPC} network protocol was dropped in favor of \ac{HTTP}.  This
allowed the \DODS project to make use of web browsers as rudimentary
clients and provided a way to simplify construction of \DAP servers.
This new mechanism became the first version of the \DAP.

% Version 2, CEs

In mid 1996 \DAP version 1 was extended so that variables could be
sub-set using operators. The \DODS servers were extended so that
information could be requested using Constraint Expressions built up using
the sub-setting operators \cite{gallagher:dods}.

There have been several sub-versions of the \DAP 2 protocol. These
versions added services beyond the basic objects originally defined by
version 1.0. These included access to data as \ac{CSV} ASCII values, a
simple \ac{HTML} interface for building \ac{URL}s and a simple
directory interface for navigating collections of files served a one
logical data source \cite{gallagher:dap-spec}.

By early 2000 the \DODS project had six data server implementations
using one of the \DAP 2.x sub-versions, and there were several hundred
data sources on-line. In addition, other groups had developed their
own servers which used the \DAP. Some of these were used for
oceanographic data, but others were used to access data from other
disciplines. At this point it became clear that a formal description
of the protocol was needed. This need was partially addressed by the
\DAP 2.0 Draft Specification \cite{gallagher:dap-spec}. However, that
specification was not completed because by this time the \DAP's
overall design was beginning to show its age. In the intervening years
the \ac{WWW} had eclipsed other network infrastructures and \ac{XML}
had emerged as a dominant way to encode information for systems like
\DODS which used \ac{HTTP} to transport non-\ac{HTML} documents.

% Version 3, XML, protocols, explicit breakdown of parts, more formal
% specification 

\DAP Version 3.0 consisted of some additions to the 2.0 version.  The
\DAP software, meanwhile, had advanced to version 3.3.  To avoid
confusion about the version of the \DAP and the version of the
software, the \DAP version was incremented to \dapversion.

That brings us to \DAP \dapversion, described in this and the associated
Data Model, Objects and Services documents. This version of the \DAP
keeps the essentials in place, removing a few features of the 1.0 and 2.x
protocols that were rarely used and hard to implement. It augments the
current persistent representation of objects with one based on XML and
provides support for the old scheme using services. The new protocol is also
designed to function with other transport protocols besides HTTP in
response to requests from high-performance computing groups.

% Future, SOAP.

Planned future work on the \DAP includes adding a \ac{SOAP} service. This
will provide a way for clients and servers to communicate using the \ac{SOAP}
protocol. In addition, this may lead to the development of other
\ac{SOAP}/\ac{XML} based services.

%\section{Notational Conventions and Generic Grammar}
%\label{app:grammar}
%
% This was taken verbatim from rfc2616. The original section title was
% `Notational Conventions and Generic Grammar' 3/21/2001 jhrg

\subsection{Augmented BNF}
All of the mechanisms specified in this document are described in both prose
and an augmented Backus-Naur Form (BNF) similar to that used by RFC
822~\cite{rfc822}. Implementors will need to be familiar with the notation in
order to understand this specification. The augmented BNF includes the
following constructs:

\begin{description}
  
\item [\texttt{name = definition}] The name of a rule is simply the name
  itself (without any enclosing \texttt{"$<$"} and \texttt{"$>$"}) and is
  separated from its definition by the equal \texttt{"="} character. White
  space is only significant in that indentation of continuation lines is used
  to indicate a rule definition that spans more than one line. Certain basic
  rules are in uppercase, such as SP, LWS, HT, CRLF, DIGIT, ALPHA, etc. Angle
  brackets are used within definitions whenever their presence will
  facilitate discerning the use of rule names.
  
\item [\texttt{"literal"}] Quotation marks surround literal text. Unless
  stated otherwise, the text is case-insensitive.
      
\item [\texttt{rule1 | rule2}] Elements separated by a bar (\texttt{"|"}) are
  alternatives, e.g., \texttt{"yes | no"} will accept \texttt{yes} or
  \texttt{no}.
  
\item [\texttt{(rule1 rule2)}] Elements enclosed in parentheses are treated
  as a single element.  Thus, \texttt{"(elem (foo | bar) elem)"} allows the
  token sequences \texttt{"elem foo elem"} and \texttt{"elem bar elem"}.
  
\item [\texttt{*rule}] The character \texttt{"*"} preceding an element
  indicates repetition. The full form is \texttt{"$<$n$>$*$<$m$>$element"}
  indicating at least \texttt{$<$n$>$} and at most \texttt{$<$m$>$}
  occurrences of element. Default values are 0 and infinity so that
  \texttt{"*(element)"} allows any number, including zero;
  \texttt{"1*element"} requires at least one; and \texttt{"1*2element"}
  allows one or two.
  
\item [\texttt{[rule]}] Square brackets enclose optional elements;
  \texttt{"[foo bar]"} is equivalent to \texttt{"*1(foo bar)"}.
  
\item [\texttt{N rule}] Specific repetition: \texttt{"$<$n$>$(element)"} is
  equivalent to \texttt{"$<$n>*$<$n$>$(element)"}; that is, exactly
  \texttt{$<$n$>$} occurrences of (element).  Thus 2DIGIT is a 2-digit
  number, and 3ALPHA is a string of three alphabetic characters.
  
\item [\texttt{\#rule}] A construct \texttt{"\#"} is defined, similar to
  \texttt{"*"}, for defining lists of elements. The full form is
  \texttt{"$<$n$>$\#$<$m$>$element"} indicating at least \texttt{$<$n$>$} and
  at most \texttt{$<$m$>$} elements, each separated by one or more commas
  (\texttt{","}) and OPTIONAL linear white space (LWS). This makes the usual
  form of lists very easy; a rule such as \texttt{( *LWS element *( *LWS ","
    *LWS element ))} can be shown as \texttt{1\#element} Wherever this
  construct is used, null elements are allowed, but do not contribute to the
  count of elements present. That is, \texttt{"(element), , (element) "} is
  permitted, but counts as only two elements. Therefore, where at least one
  element is required, at least one non-null element MUST be present. Default
  values are 0 and infinity so that \texttt{"\#element"} allows any number,
  including zero; \texttt{"1\#element"} requires at least one; and
  \texttt{"1\#2element"} allows one or two.
  
\item [\texttt{;} comment] A semi-colon, set off some distance to the right
  of rule text, starts a comment that continues to the end of line. This is a
  simple way of including useful notes in parallel with the specifications.
  
\item [implied \texttt{*LWS}] The grammar described by this specification is
  word-based. Except where noted otherwise, linear white space (LWS) can be
  included between any two adjacent words (token or quoted-string), and
  between adjacent words and separators, without changing the interpretation
  of a field. At least one delimiter (LWS and/or separators) MUST exist
  between any two tokens (for the definition of "token" below), since they
  would otherwise be interpreted as a single token.
\end{description}

\subsection{Basic Rules}

   The following rules are used throughout this specification to
   describe basic parsing constructs. The US-ASCII coded character set
   is defined by ANSI X3.4-1986~\cite{ANSI:US-ASCII}.

\begin{vcode}{it}
       OCTET          = <any 8-bit sequence of data>
       CHAR           = <any US-ASCII character (octets 0 - 127)>
       UPALPHA        = <any US-ASCII uppercase letter "A".."Z">
       LOALPHA        = <any US-ASCII lowercase letter "a".."z">
       ALPHA          = UPALPHA | LOALPHA
       DIGIT          = <any US-ASCII digit "0".."9">
       CTL            = <any US-ASCII control character
                        (octets 0 - 31) and DEL (127)>
       CR             = <US-ASCII CR, carriage return (13)>
       LF             = <US-ASCII LF, linefeed (10)>
       SP             = <US-ASCII SP, space (32)>
       HT             = <US-ASCII HT, horizontal-tab (9)>
       <">            = <US-ASCII double-quote mark (34)>
\end{vcode}

HTTP/1.1 defines the sequence CR LF as the end-of-line marker for all
protocol elements except the entity-body (see Appendix 19.3 of RFC
2616[9] for tolerant applications). The end-of-line marker within an
entity-body is defined by its associated media type, as described in
Section 3.7 of RFC 2616[9].

% My original text:
% except the entity-body (see appendix
%    19.3\footnote{In RFC 2616\cite{rfc2616}.} for
%    tolerant applications). The end-of-line marker within an entity-body
%    is defined by its associated media type, as described in section
%    3.7.\footnote{In RFC 2616\cite{rfc2616}.}
% The replacement text above was suggested by Allan Doyle, 20 April
% 2005.

\begin{vcode}{it}
       CRLF           = CR LF
\end{vcode}

   HTTP/1.1 header field values can be folded onto multiple lines if the
   continuation line begins with a space or horizontal tab. All linear
   white space, including folding, has the same semantics as SP. A
   recipient MAY replace any linear white space with a single SP before
   interpreting the field value or forwarding the message downstream.

\begin{vcode}{it}
       LWS            = [CRLF] 1*( SP | HT )
\end{vcode}

   The TEXT rule is only used for descriptive field contents and values
   that are not intended to be interpreted by the message parser. Words
   of *TEXT MAY contain characters from character sets other than ISO-
   8859-1 [22] only when encoded according to the rules of RFC 2047
   [14].

\begin{vcode}{it}
       TEXT           = <any OCTET except CTLs,
                        but including LWS>
\end{vcode}

   A CRLF is allowed in the definition of TEXT only as part of a header
   field continuation. It is expected that the folding LWS will be
   replaced with a single SP before interpretation of the TEXT value.

   Hexadecimal numeric characters are used in several protocol elements.

\begin{vcode}{it}
       HEX            = "A" | "B" | "C" | "D" | "E" | "F"
                      | "a" | "b" | "c" | "d" | "e" | "f" | DIGIT
\end{vcode}

   Many HTTP/1.1 header field values consist of words separated by LWS
   or special characters. These special characters MUST be in a quoted
   string to be used within a parameter value (as defined in section
   3.6).

\begin{vcode}{it}
       token          = 1*<any CHAR except CTLs or separators>
       separators     = "(" | ")" | "<" | ">" | "@"
                      | "," | ";" | ":" | "\" | <">
                      | "/" | "[" | "]" | "?" | "="
                      | "{" | "}" | SP | HT
\end{vcode}

   Comments can be included in some HTTP header fields by surrounding
   the comment text with parentheses. Comments are only allowed in
   fields containing "comment" as part of their field value definition.
   In all other fields, parentheses are considered part of the field
   value.

\begin{vcode}{it}
       comment        = "(" *( ctext | quoted-pair | comment ) ")"
       ctext          = <any TEXT excluding "(" and ")">
\end{vcode}

   A string of text is parsed as a single word if it is quoted using
   double-quote marks.

\begin{vcode}{it}
       quoted-string  = ( <"> *(qdtext | quoted-pair ) <"> )
       qdtext         = <any TEXT except <">>
\end{vcode}

   The backslash character ("\verb+\+") MAY be used as a single-character
   quoting mechanism only within quoted-string and comment constructs.

\begin{vcode}{it}
       quoted-pair    = "\" CHAR
\end{vcode}

\begin{quote}
This appendix was copied from RFC 2616~\cite{rfc2616}. The copyright
from that document reads:

\begin{quote}
  Copyright (C) The Internet Society (1999).  All Rights Reserved.

   This document and translations of it may be copied and furnished to
   others, and derivative works that comment on or otherwise explain it
   or assist in its implementation may be prepared, copied, published
   and distributed, in whole or in part, without restriction of any
   kind, provided that the above copyright notice and this paragraph are
   included on all such copies and derivative works.  However, this
   document itself may not be modified in any way, such as by removing
   the copyright notice or references to the Internet Society or other
   Internet organizations, except as needed for the purpose of
   developing Internet standards in which case the procedures for
   copyrights defined in the Internet Standards process must be
   followed, or as required to translate it into languages other than
   English.
\end{quote}

\end{quote}


%\section{Acronyms and Abbreviations}
%\begin{acronym}
%%
% Make one entry per line, even if they are long lines that wrap and look
% ugly. This makes it simple to sort the list using emacs' sort-lines
% command. 3/27/2000 jhrg
%
% Find all the acronyms in a tex file using: 
%
%   grep -o '\\ac\(l\|s\)*{[A-Z,a-z]*}' dap_overview.tex | sort -u
%
% Alphabetize this using emacs' sort-lines on a region of the text.
%
% $Id$

%\acro{AS}{Aggregation Server}
%\acro{CE}{Constraint Expression}
%\acro{CGI}{Common Gateway Interface}
%\acro{COARDS}{Cooperative Ocean/Atmosphere Research Data Service}
%\acro{DataDDS}{Data Dataset Descriptor Structure}
%\acro{FGDC}{Federal Geographic Data Community}
%\acro{MIME}{Multimedia Internet Mail Extensions}
%\acro{SRS}{Software Requirements Specification}, See IEEE 830--1998
%\acro{W3C}{The World Wide Web Consortium}, See http://www.w3c.org/
%\acro{XDR}{External Data Representation}

\acro{BNF} {Backus-Naur Form}
\acro{CSV}{Comma Separated Values}
\acro{DAP}{Data Access Protocol}
\acro{DAS}{Dataset Attribute Structure}
\acro{DAX}{Dataset Ancillary Information}
\acro{DDS}{Dataset Descriptor Structure}
\acro{DDX}{Dataset Descriptor}
\acro{DODS}{Distributed Oceanographic Data System}, See the DODS home page: \texttt{http://\-unidata.ucar.edu\-/packages\-/dods/}
\acro{HTML}{Hypertext Markup Language}
\acro{HTTP}{HyperText Transfer Protocol}
\acro{NVODS}{National Virtual Ocean Data System}, See the NVODS home page: \texttt{http://\-www.po.gso.uri.edu/\-tracking/\-vodhub/\-vodhubhome.html}
\acro{RPC}{Remote Procedure Call}
\acro{SOAP}{Simple Object Access Protocol}
\acro{URI}{Uniform Resource Identifiers}
\acro{URL}{Uniform Resource Locator}
\acro{WAN}{Wide Area Network}
\acro{WWW}{The World Wide Web}
\acro{XML}{Extensible Markup Language}

%\end{acronym}

\section{Change log}

\begin{verbatim}
$Log: dap_overview.tex,v $
Revision 1.8  2003/07/24 22:32:12  tom
excised dap_services document references

Revision 1.7  2003/07/16 01:06:08  tom
progress on comments, fixed titles

Revision 1.6  2003/06/10 16:13:53  tom
incorporated suggestions from March meeting

Revision 1.5  2003/05/22 19:37:30  tom
rearranging

Revision 1.4  2003/03/17 17:45:10  tom
progress made.  draft for discussion 3/18/03

Revision 1.3  2002/11/19 05:37:52  jimg
Added a paragraph about sub-setting to the Purpose section.

Revision 1.2  2002/11/19 05:26:20  jimg
Fixed a few errors. Ready for comments. Still very much a draft.

Revision 1.1  2002/11/16 01:29:03  jimg
First version of the overview document.

\end{verbatim}

\end{document}
