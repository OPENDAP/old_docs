%
% Documentation for the DAP. Intended to be like an RFC document.
%

\documentclass[justify]{dods-paper}
\usepackage{longtable}
\usepackage{acronym}
\usepackage{xspace}
\usepackage{gloss}
\usepackage{changebar}
\rcsInfo $Id$

% latex and HTML macros. Some latex commands become nops for HTML. 4/10/2001
% jhrg 
\T\newcommand{\Cpp}{\rm {\small C}\raise.5ex\hbox{\footnotesize++}\xspace}
\T\newcommand{\C}{\rm {\small C}\xspace}
\W\newcommand{\Cpp}{C++}
\W\newcommand{\C}{C\xspace}
\W\newcommand{\cdots}{...}
\W\newcommand{\ddots}{}
\W\newcommand{\vdots}{.}
\W\newcommand{\pm}{+/-}
\W\newcommand{\times}{*}
\W\newcommand{\uppercase}[1]{\textsc{#1}}
\T\newcommand{\qt}{\lit{\char127}}
\W\newcommand{\qt}{"}

\texorhtml{\def\rearrangedate#1/#2/#3#4{\ifcase#2\or January\or February\or
  March\or April\or May\or June\or July\or August\or September\or
  October\or November\or December\fi\ \ifx0#3\relax\else#3\fi#4, #1}
\def\rcsdocumentdate{\expandafter\rearrangedate\rcsInfoDate}}%
{\HlxEval{
(put 'rearrangedate       'hyperlatex 'hyperlatex-ts-rearrange-date)

(defun hyperlatex-ts-rearrange-date ()
  (let ((date-string (hyperlatex-evaluate-string 
                       (hyperlatex-parse-required-argument))))
    (let ((year-string (substring date-string 0 4))
          (month-string (substring date-string 5 7))
          (day-string (substring date-string 8 10))
          (month-list '("January" "February" "March" "April"
                        "May" "June" "July" "August" 
                        "September" "October" "November" "December")))
       (insert
         (concat (elt month-list (1- (string-to-number month-string)))
                 " " (int-to-string (string-to-number day-string))
                 ", " year-string)))))
}
\newcommand{\rcsdocumentdate}{\rearrangedate{\rcsInfoDate}}}

\newcommand{\dapversion}{Version 4.0\xspace}
\newcommand{\opendap}{OPeNDAP\xspace}
\newcommand{\DAP}{DAP\xspace}
\newcommand{\DODS}{DODS\xspace}
\newcommand{\NVODS}{NVODS\xspace}
\newcommand{\CE}{constraint expression\xspace}
\newcommand{\CEs}{constraint expressions\xspace}
\newcommand{\ErrorX}{ErrorX\xspace}
\newcommand{\CapX}{Server Capabilities Document\xspace}
\newcommand{\Blob}{Blob\xspace}
\newcommand{\DDX}{DDX\xspace}
\newcommand{\DAX}{DAX\xspace}
\newcommand{\DDS}{DDS\xspace}
\newcommand{\DAS}{DAS\xspace}
\newcommand{\URI}{URI\xspace}
\newcommand{\DataDDS}{DataDDS\xspace}

\newcommand{\type}[1]{\emph{#1}}
\newcommand{\Alias}{\type{Alias}\xspace}
\newcommand{\Array}{\type{Array}\xspace}
\newcommand{\Attribute}{\type{Attribute}\xspace}
\newcommand{\ProcAttribute}{\type{Processing Attribute}\xspace}
\newcommand{\Map}{\type{Map}\xspace}
\newcommand{\Target}{\type{Target}\xspace}
\newcommand{\FQN}{fully qualified name\xspace}
\newcommand{\Grid}{\type{Grid}\xspace}
\newcommand{\Structure}{\type{Structure}\xspace}
\newcommand{\Dataset}{\type{Dataset}\xspace}
\newcommand{\Sequence}{\type{Sequence}\xspace}
\newcommand{\Container}{\type{Container}\xspace}
\newcommand{\Bi}{\type{Binary Image}\xspace}
\newcommand{\String}{\type{String}\xspace}
\newcommand{\URL}{\type{URL}\xspace}
\newcommand{\Boolean}{\type{Boolean}\xspace}
\newcommand{\Byte}{\type{Byte}\xspace}
\newcommand{\Enum}{\type{Enumeration}\xspace}
\newcommand{\Time}{\type{Time}\xspace}
\newcommand{\Function}{\type{Function}\xspace}
\newcommand{\Description}{\type{Description}\xspace}
\newcommand{\Parameter}{\type{Parameter}\xspace}
\newcommand{\Constraint}{\type{Constraint}\xspace}
\newcommand{\NoAttributes}{\type{NoAttributes}\xspace}
\newcommand{\Project}{\type{Project}\xspace}
\newcommand{\Select}{\type{Select}\xspace}
\newcommand{\Hyperslab}{\type{Hyperslab}\xspace}

\newcommand{\Aliases}{\type{Aliases}\xspace}
\newcommand{\Arrays}{\type{Arrays}\xspace}
\newcommand{\Attributes}{\type{Attributes}\xspace}
\newcommand{\ProcAttributes}{\type{Processing Attributes}\xspace}
\newcommand{\Maps}{\type{Maps}\xspace}
\newcommand{\Targets}{\type{Targets}\xspace}
\newcommand{\FQNs}{fully qualified names\xspace}
\newcommand{\Grids}{\type{Grids}\xspace}
\newcommand{\Structures}{\type{Structures}\xspace}
\newcommand{\Sequences}{\type{Sequences}\xspace}
\newcommand{\Containers}{\type{Containers}\xspace}
\newcommand{\Bis}{\type{Binary Images}\xspace}
\newcommand{\Strings}{\type{Strings}\xspace}
\newcommand{\URLs}{\type{URLs}\xspace}
\newcommand{\Booleans}{\type{Booleans}\xspace}
\newcommand{\Bytes}{\type{Bytes}\xspace}
\newcommand{\Enums}{\type{Enumerations}\xspace}
\newcommand{\Times}{\type{Times}\xspace}
\newcommand{\CSVs}{comma-separated values\xspace}
\newcommand{\Functions}{\type{Functions}\xspace}
\newcommand{\Descriptions}{\type{Descriptions}\xspace}
\newcommand{\Parameters}{\type{Parameters}\xspace}
\newcommand{\Constraints}{\type{Constraints}\xspace}
\newcommand{\Projects}{\type{Projects}\xspace}
\newcommand{\Selects}{\type{Selects}\xspace}
\newcommand{\Hyperslabs}{\type{Hyperslabs}\xspace}

\newcommand{\DIR}{\textbf{Directory}\xspace}
\newcommand{\TEXT}{\textbf{Text}\xspace}
\newcommand{\HTML}{\textbf{HTML}\xspace}
\newcommand{\HELP}{\textbf{Help}\xspace}
\newcommand{\VER}{\textbf{Version}\xspace}
\newcommand{\INFO}{\textbf{Info}\xspace}

%%%%%%%%%%%%%% Web Services Paper
\newcommand{\GetDDX}{\textbf{GetDDX}\xspace}
\newcommand{\GetData}{\textbf{GetData}\xspace}
\newcommand{\GetBlobData}{\textbf{GetBlobData}\xspace}
\newcommand{\GetBlob}{\textbf{GetBlob}\xspace}
\newcommand{\GetDir}{\textbf{GetDir}\xspace}
\newcommand{\GetInfo}{\textbf{GetInfo}\xspace}

\newcommand{\FSs}{\type{Foundation Services}\xspace}
\newcommand{\FS}{\type{Foundation Service}\xspace}
\newcommand{\ODSN}{\type{OPeNDAP}\xspace}

\newcommand{\blobdataelement}{\texttt{BlobData} element\xspace}
\newcommand{\blobelement}{\texttt{Blob} element\xspace}



%\setcounter{secnumdepth}{4}
%\setcounter{tocdepth}{4}

\newcommand{\Tableref}[1]{Table~\ref{#1}}%
\newcommand{\Figureref}[1]{Figure~\ref{#1}}%
\W\begin{iftex}
\newcommand{\Sectionref}[2]{Section~\ref{#1}%
  \ifx#2)%
    \ on page~\pageref{#1}%
  \else% 
    \ (page~\pageref{#1})\xspace%
  \fi\ifx#2\space\ \else #2\fi}%
\W\end{iftex}
\W\newcommand{\Sectionref}[1]{Section~\ref{#1}}
\W\newcommand{\raggedright}{}


%% Conveniences for documenting XML
\newcommand{\tag}[1]{\emph{#1}}
\newcommand{\element}[1]{\link{\tag{#1}}{sec-xml-#1}}
\newcommand{\attribute}[1]{\emph{#1}}
\newcommand{\currentelement}{}
\newcommand{\ELEMENT}[1]{\renewcommand{\currentelement}{#1 element}%
  \subsubsection{#1}\label{sec-xml-#1}\indc{\currentelement}%
  \indc{catalog tag!#1}\indc{aggregation tag!#1}%
  \indc{XML!#1 element}}
\newcommand{\ATTRIBUTE}[1]{\item{\lit{#1}}\indc{\currentelement!#1}
    \indc{#1 attribute!of \currentelement}%
    \indc{XML!#1 attribute}}

% Conveniences for examples
\newcounter{exampleno}
\setcounter{exampleno}{0}
\newcounter{examplerefno}
\setcounter{examplerefno}{0}
\newcommand{\examplelabel}[1]{\refstepcounter{exampleno}\label{#1}%
  \medskip Example \theexampleno :\smallskip}
\newcommand{\exampleref}[1]{\texorhtml{Example~\ref{#1}%
    \refstepcounter{examplerefno}\label{exref\theexamplerefno}%
    % This is a test whether r@exref... is defined.  If not, skip
    % anything with the \pageref macro.
    \catcode`\@=11%
    \expandafter\ifx\csname r@exref\theexamplerefno\endcsname\relax\else%
    \expandafter\ifx\csname r@#1\endcsname\relax\else%
    \bgroup\count100=\pageref{exref\theexamplerefno}%
    \count101=\pageref{#1}\ifnum\count100=\count101\else~%
    on page~\pageref{#1}\fi\egroup\fi\fi\xspace}%
  {\link{Example \ref{#1}}{#1}}}%
\T\setlength{\vcodeindent}{10pt}
\texorhtml{
%% LaTeX version
\newenvironment{textoutput}[1]{\ifx #1\relax%
  \medskip Output:\vspace{-\medskipamount}\else%
  \medskip #1\vspace{-\medskipamount}\fi%
  \begin{list}{}{\setlength{\leftmargin}{\vcodeindent}}\begin{ttfamily}\item}%
 {\end{ttfamily}\end{list}} %
}{%% Hyperlatex version
\newenvironment{textoutput}[1]{\xml{blockquote}Output:\\ \\ \xml{tt}}%
  {\xml{/tt}\xml{/blockquote}}%
\newenvironment{minipage}[4]{}{}
\newenvironment{ttfamily}{\xml{blockquote}\xml{tt}}%
  {\xml{/tt}\xml{/blockquote}}}

\newcommand{\DAPOverviewTitle}{DAP Specification Overview}
\newcommand{\DAPOverview}{\xlink{\textbf{\textit{\DAPOverviewTitle}}}%
  {dap.html}\xspace}

% Old macros; new values
\newcommand{\DAPObjectsTitle}{The Data Access Protocol---DAP 2.0}
\newcommand{\DAPObjects}{\xlink{\textbf{\textit{\DAPObjectsTitle}}}%
  {http://www.opendap.org/pdf/ESE-RFC-004v0.06.pdf}\xspace}
  
\newcommand{\DAPHTTPTitle}{Using DAP 2.0 with HTTP}
\newcommand{\DAPHTTP}{\xlink{\textbf{\textit{\DAPHTTPTitle}}}%
  {daph.html}\xspace}

% New macros.
\newcommand{\DAPDataModelTitle}{DAP Data Model Specification}
\newcommand{\DAPDataModel}{\xlink{\textbf{\textit{\DAPDataModelTitle}}}%
  {dapo.html}\xspace}
  
\newcommand{\DAPWebTitle}{DAP Web Services Specification}
\newcommand{\DAPWeb}{\xlink{\textbf{\textit{\DAPWebTitle}}}%
  {daph.html}\xspace}
  

\newcommand{\DAPASCIITitle}{DAP Formatted Data Specification}
\newcommand{\DAPASCII}{\xlink{\textbf{\textit{\DAPASCIITitle}}}%
  {dapa.html}\xspace}
\newcommand{\DAPHTMLTitle}{DAP HTTP Query Specification}
\newcommand{\DAPHTML}{\xlink{\textbf{\textit{\DAPHTMLTitle}}}%
  {dapm.html}\xspace}

% It probably doesn't matter what we call the macro, but SOAP does not have
% to run over HTTP and I think that's going to be important for other groups.
% 10/27/03 jhrg
\newcommand{\DAPServicesTitle}{DAP HTTP Services Specification}
\newcommand{\DAPServices}{\xlink{\textbf{\textit{\DAPServicesTitle}}}%
  {daps.html}\xspace}

\newcommand{\thirtytwobitlimit}[1]{$4,294,967,296$ #1 ($2^{32}$)}
%%% Local Variables: 
%%% mode: latex
%%% TeX-master: t
%%% End: 


% Note: to get the glossary to work, run bibtex on the *.gls.aux file,
% then latex the file, then bibtex *.gls, then latex again. Also, make
% sure to set your BST and BIBINPUTS environment variables so that the
% BST and BIB files will be found.
% \makegloss

\title{OPeNDAP Web Services Design Document\\(First SOAP Implementation)\\ DRAFT}
\htmltitle{\DAPWebTitle\ -- DRAFT}
\author{John Camberlin}
\date{Printed: \today \\ Revision: \rcsInfoRevision}
\htmladdress{James Gallagher <jgallagher@gso.uri.edu>, \rcsInfoDate, 
  Revision: \rcsInfoRevision}
\htmldirectory{html}
\htmlname{dapo}

\begin{document}

\maketitle
%\T\tableofcontents


%%%%%%%%%%%%%%%%%%%%%%%% Introduction %%%%%%%%%%%%%%%%%%%%%%%%
\section{Introduction}

The web approach to building distributed applications has been more rapidly and widely adopted than any other approach. A logical extension of this success is the development of web services technologies which are more loosely coupled than traditional distributed programming models like RPC, DCOM and CORBA. Web services create interactions between clients and servers which are simple and flexible. At the same time the technology includes the structural capability to do object transfers that are not possible under plain HTTP. Web services do these transfers by using SOAP messages (as opposed to the MIME messages typical of ordinary web applications). This is a more data-friendly mechanism for exchanging information over the web and is ideal for the purpose of creating a convenient interface to OPeNDAP servers.

The rationale for creating OPeNDAP web services is that it would allow programmatic clients to access OPeNDAP servers in a more reliable and structured way than the HTTP interface. Web services also make client implementations easier because the client developer need only work with objects rather than worry about constructing and parsing text messages. Traditionally all of the control structures in OPeNDAP have been passed as text which both client and server must construct/parse. We have provided tools such as the dods.dap package which allow the client to parse some of these messages, but this is a non-standard and incomplete solution. With web services OPeNDAP has a standard mechanism for reliable interfacing and object transfer without parsing.

An important benefit of a web services interface is that it provides the basis for directory services via UDDI. Using this technology server sites may have the option of publishing information about their content globally or locally. This would expand the ability of end users to locate data sources.

Another auxiliary capability of the web services implementation will be to provide for the possibility of hosting secure sites--something not possible with current incarnations of OPeNDAP. For those users with sensitive or restricted data this would allow them to publish that data over the web, but still enforce a comprehensive security policy. SOAP supports an integrated ability to exchange digital signatures and X.509 credentials in a standard way.

\section{Component Areas}

Implementing web services involves building and configuring several different types of components. Depending on the type of service desired and the nature of the underlying application these components can vary. For enabling OPeNDAP servers with web services we have focused on the following component areas:

\begin{itemize}
    \item [1] Server configuration and dispatching
    \item [2] Services
    \item [3] WSDL publication
    \item [4] Server-side code enhancement
    \item [5] Client support
    \item [6] New object design
    \item [7] Directory support
\end{itemize}
The next section of this document details the additions to an OPeNDAP 
server that enables web services by treating each of these areas separately.

\subsection{Server Configuration and Dispatching}

Because OPeNDAP Java servers are implemented with Apache and Tomcat we take 
advantage of Apache's integrated SOAP support to dispatch web service 
requests. The current configuration of the OPeNDAP Java server already 
contains most of the elements necessary to add this support. The additional 
steps are:

\begin{itemize}
  \item [1] \textbf{Deploy Apache SOAP to Tomcat:} Put the Apache SOAP 
  archive file (soap.war) 
  into the \%TOMCAT\_HOME\%/webapps directory. Add the libraries \lit{mail.jar}, 
  \lit{activation.jar}, \lit{xerces.jar} to the Tomcat library directory 
  (\%TOMCAT\_HOME\%/common/lib/).

  \item [2] \textbf{Deploy the service class:} The service class 
  OPeNDAP.services.ServiceDispatcher.class where the SOAP router can find it 
  (\%TOMCAT\_HOME\%/webapps/soap/WEB-INF/classes/opendap-services/).
  
  \item [3] \textbf{Deploy the service:} This can be done by a script 
  executing the command: 
  
  \lit{java org.apache.soap.server.ServiceManagerClient 
  http://localhost:8080/soap/servlet/rpcrouter deploy 
  DeploymentDescriptor\_OPeNDAP\_Services.xml.}
  
  The deployment descriptor file in this command is listed in Appendix A.
  
  \item [4] \textbf{Load ServiceInitializer class} Part of the configuration 
  is a class, dods.servlet.ServiceInitializer, 
  which serves as a dumb container for any configuration parameters needed by the 
  dispatcher. To load this class add an entry to the Apache SOAP servlet 
  configuration file (\%TOMCAT\_HOME\%/webapps/soap/WEB-INF/web.xml):\\[2mm]
  
  \begin{vcode}{t}
    <servlet>
        <servlet-name>serviceinitializer</servlet-name>
        <display-name>OPeNDAP Services Initializer</display-name>
        <description>Stores parameters used by the service dispatcher.</description>
        <servlet-class>dods.servlet.ServiceInitializer</servlet-class>
        <init-param>
            <param-name>DDScache</param-name>
            <param-value>/Java-DODS/sdds-testsuite/dds/</param-value>
        </init-param>
        <load-on-startup> 1 </load-on-startup>
    </servlet>
  \end{vcode}
  
  The "DDSCache" parameter should match the one used in the DODS servlet 
  configuration file. For a windows machine the slashes will be reversed 
  and a drive letter will be referenced.

\end{itemize}

Once these steps have been completed the new services (described in section 2 below) will be reachable. In general the approach taken on the server side is to use a hard-coded deployment descriptor to instruct Apache how to route incoming messages. This may not be as flexible as using dynamic invocation with JAX-RPC, for example, but is exactly appropriate for OPeNDAP servers which are expected to implement only a few services which will rarely change.


\subsection{Services}

The key feature of the web services design is the envisioned services themselves. These are all new interfaces that replace existing internal interfaces of the servlet code base with a web services version of the same function. The OPeNDAP Java server operates via the servlet dods.servlet.DODSServlet. The doGet method of the servlet dispatches the client request to one of eleven internal methods:

Existing Internal Methods:\\
\begin{itemize}
	\item doGetVER
	\item doGetHELP
	\item doGetDDS
	\item doGetDAS
	\item doGetDODS
	\item doGetDIR
	\item doGetASC
	\item doGetINFO
	\item doGetHTML
	\item doGetCatalog
	\item doGetStatus
\end{itemize}

New Web Service Interface (in OPeNDAP.services.ServiceDispatcher):
\begin{itemize}
	\item serviceVersion  - returns the server version as String
	\item serviceDDX      - returns the DDX XML document as a String
	\item serviceDataURL  - returns the URL to requested data as a String
\end{itemize}


The existing servlet dispatcher uses a final extension on the URL to route the request. For example, if the client wants the ASCII text of the data it appends ".ASC" to the end of the URL file. The web service interface is invoked directly by the Apache SOAP routing service(\lit{org.apache.soap.server.ServiceManagerClient}). The four service methods are in a new class \lit{OPeNDAP.services.ServiceDispatcher} which is registered via the Apache SOAP API. The Apache router determines the destination service by examing the body entry in the SOAP envelope from the client and then calls that method.

The methods that implement the services can be any normal Java method because the simplest method of invocation (plain RPC) is used. The only restriction is that the method parameters must be serializable. To comply with this restriction and make the parameters of the service methods more client-friendly, they differ somewhat from the parameter strategy found in the existing methods which use a special "requestState" object which bundles the individual parameters:

\begin{vcode}{t}
    String serviceVersion() throws SOAPException\\
    String serviceDDX( String sDodsURL ) throws SOAPException\\
    String serviceDataURL( String sDodsURL, String sConstraint ) throws SOAPException
\end{vcode}

The service methods cannot merely redirect the request to the existing internals because the service methods return SOAP envelopes, thus the service methods must re-implement the existing code for the new output format.


\subsection{Server-Side Code Enhancement}

To support web services the OPeNDAP server-side code requires certain enhancements beyond adding the service interfaces. One type of new code is required for retrieving and returning the DataDDS. In the existing servlet this is done via a special routine in the dods.dap.Server.CEEvaluator class which generates the octet stream. Instead the service helper needs to just create the dods.dap.DataDDS object itself and return it in its entirety.

In the current design there is no need for serializers/deserializers because the argument and return types are all strings in the interface. If additional interfaces were added that used special types (such as DDS, DAS and DataDDS, for example) then serializers would have to be written for those types. The serializers convert these objects to and from XML which allows them to be transferred via SOAP. The serializers do not need to convert each iota of field data into separate XML elements because the protocol allows attachments to the XML document or binary blobs. The XML acts simply as an envelope around the object blob. These classes are used by both the server and the client to work with the special return types of the services.


\subsection{WSDL Publication}

Since different versions of the server may support additional service calls or the parameters of those calls might change it is a good idea to publish a WSDL (Web Services Description Language) document. This XML document specifies the services interface to the server and is generated automatically from the code in the ServiceDispatcher class using a separate conversion tool. Examples of possible tools are: the IBM Web Services Toolkit's wsdlgen, Apache Axis' Java2WSDL tool, and the GLUE platform's Java2WSDL tool. Regardless of the tool chosen, the output should be relatively the same.

Clients can access the WSDL document from the UDDI directory (see component 7 below) or alternatively can download from a resource directory at OPeNDAP.org. If the latter method is used the client must match up exactly the server version with WSDL document to make sure they are compatible.

Sophisticated clients can use the WSDL document to make dynamic calls to the OPeNDAP services.


\subsection{Client Support}

To an extent the design for client support has the force of a recommendation because outside client programmers ultimately may implement the client action in any way they choose and various solutions may be open to them. For the OPeNDAP team it is important to have an idealized client implementation to ensure workability with the servers and to provide a model for client developers. By providing a client package along with example uses a programmer can readily implement an OPeNDAP web services client while having a minimal knowledge of the underlying protocols such as SOAP.

\subsection{New Object Design}

In the initial web services implementation there are no new data objects, but in a future edition additional objects such as an object to contain a directory structure may be desirable, for example, if the doGetDir method was implemented with a web services equivalent.

\subsection{Directory Services}
Supporting a UDDI server will not be practical for most OPeNDAP hosts because of the need to manage a database and other administration requirements. The approach taken, therefore, is for OPeNDAP.org to support a central registry which the service code can automatically contact and publish its WSDL and other information. Each time a given ServerDispatcher is instantiated and an expiration period has elapsed, the dispatcher will send a set of UDDI messages to the central UDDI registry with its information. The host administrator is able to turn this functionality off if they desire.

An important aspect of the UDDI server will be its taxonomy model. The standard taxonomy models are entirely commercial in focus so are inappropriate for categorizing research data. For this use a custom taxonomy is needed. Currently the OGC and related projects such as Digital Earth have been working on such taxonomies.

\subsection{Conclusion}

Adding a web services interface to the existing OPeNDAP Java server platform offers a range of benefits with little or no additional administrative burden on hosting organizations. The web services interface will allow better discovery of existing data resouces via UDDI and better programmatic access to those resources. This will enhance client support for OPeNDAP with greater functionality and reliability.



\appendix

\section{Apache SOAP Deployment Descriptor}

To deploy the service using the command line an xml deployment descriptor is required. An example of such a descriptor is:

\begin{vcode}{t}
<isd:service xmlns:isd="http://xml.apache.org/xml-soap/deployment"
    id="urn:opendap-services"
        <isd:provider 	type="java"
            scope="Request"
            methods="	serviceVersion
                        serviceDDX
                        serviceDataURL">
            <isd:java	class="dods.servlet.ServiceDispatcher" static="false"/>
        </isd:provider>
    <isd:faultListener>org.apache.soap.server.DOMFaultListener</isd:faultListener>
</isd:service>
\end{vcode}

To use the descriptor to deploy the OPeNDAP web services the administrator issues a command line which activates a java program from the SOAP package. This program then contacts the SOAP administrative servlet (RPCrouter) and sends it the descriptor instructing the deployment. A typical command line might look like this:

\lit{java org.apache.soap.server.ServiceManagerClient http://localhost:8080/soap/servlet/rpcrouter deploy DeploymentDescriptor.xml}

In practice additional classpath or path information might be required on the command line. If the administrator would rather use the web tool to do the deployment, Apache SOAP offers a form interface to the rpcrouter. For an example of this form see Appendix B.

Once the services are deployed you can test them using the example client shown in Appendix C. 


\section{Apache SOAP Service Deployment Form}

The Apache SOAP administrative functionality which comes automatically with the servlet has a deployment form which the administrator can fill out and submit in lieu of using a command to send the deployment descriptor to the router. This form filled out looks like this:

% insert graphic here

\section{Sample Client}

This is the source code for a sample client which illustrates how to make RPC calls to the OPeNDAP web services and receive a response.

\begin{vcode}{t}

package dods.clients.webservices;

import java.io.*;
import java.net.*;
import java.util.*;
import org.apache.soap.*;
import org.apache.soap.rpc.*;

public class ExampleClient {

	public static void main(String[] args) throws Exception {
		String sWebServicesURL = "http://127.0.0.1:8080/soap/servlet/rpcrouter"; // default url
		if( args.length > 0 ) sWebServicesURL = args[0]; // user can supply url if desired
		URL url = null;
		try {
			url = new URL(sWebServicesURL);
		} catch(Exception ex) {
			System.out.println("error evaluating web services URL [" + sWebServicesURL + "]: " + ex);
			System.exit(1);
		}
		System.out.println("testing web services for URL: " + sWebServicesURL);
		org.apache.soap.rpc.Call call = new Call();
		call.setTargetObjectURI("urn:opendap-services");
		call.setEncodingStyleURI(Constants.NS_URI_SOAP_ENC);
		System.out.println("version: " + getVersion(call, url));
		System.out.println("example blob url for http://localhost:8080/dods/dts/D1: \n" + 
		getDataURL(call, url, "http://localhost:8080/dods/dts/D1", ""));
		System.out.println("example ddx for http://localhost:8080/dods/dts/D1: \n" + 
		getDDX(call, url, "http://localhost:8080/dods/dts/D1"));
	}

	public static String getVersion( org.apache.soap.rpc.Call call, java.net.URL url ){
		call.setMethodName("serviceVersion");
		return getResponseString( call, url );
	}

	public static String getDDX(  org.apache.soap.rpc.Call call, java.net.URL url, String sOPeNDAPurl ){
		call.setMethodName("serviceDDX");
		Vector params = new Vector();
		params.addElement(new Parameter("sDodsURL", String.class, sOPeNDAPurl, null));
		call.setParams (params);
		return getResponseString(call, url);
	}

	public static String getDataURL(  org.apache.soap.rpc.Call call, java.net.URL url, String sOPeNDAPurl, String sConstraint ){
		call.setMethodName("serviceDataURL");
		Vector params = new Vector();
		params.addElement(new Parameter("sDodsURL", String.class, sOPeNDAPurl, null));
		params.addElement(new Parameter("sConstraint", String.class, sConstraint, null));
		call.setParams (params);
		return getResponseString(call, url);
	}

	public static String getResponseString( org.apache.soap.rpc.Call call, java.net.URL url ){

		Response response = null;
		try {
			response = call.invoke(url, "" );
		} catch(Exception ex) {
			System.out.println("error invoking service call: " + ex);
		}

		if( response == null ){
			System.out.println("no response from server");
			return null;
		}

		if( response.generatedFault() ) {
			Fault fault = response.getFault ();
			StringBuffer sbError = new StringBuffer();
			sbError.append("Error: " + fault.getFaultCode() + " " + fault.getFaultString());
			System.out.println(sbError.toString());
			return null;
		} else {
			Parameter result = response.getReturnValue();
			return (String)result.getValue();
		}
	}
}
		
\end{vcode}

Appendix D shows a test run of executing this client example.







\section{Example Test run}

Below is the results of a run of the ExampleClient against the test server.

\begin{vcode}{t}

testing web services for URL: http://localhost:8080/soap/servlet/rpcrouter

version: Server Version: DODS/3.2

example blob url for http://localhost:8080/dods/dts/D1: 
http://localhost:8080/dods/dts/D.blob

example ddx for http://localhost:8080/dods/dts/D1: 
<?xml version="1.0" encoding="UTF-8"?>
<Dataset name="EOSDB.DBO"
xmlns:xsi="http://www.w3.org/2001/XMLSchema-instance"
xmlns="http://www.dods.org/ns/DODS"
xsi:schemaLocation="http://www.dods.org/ns/DODS  http://argon.coas.oregonstate.edu/ndp/dods.xsd" >

	<Attribute name="_location" type="Container">
		<Attribute name="Description" type="String">
			<value>"String describing general location (southern ocean,oregon coast, etc.) of drifter deployment."</value>
		</Attribute>
	</Attribute>

	<Sequence name="Drifters">
		<String name="instrument_id"/>
		<String name="location"/>
		<Float64 name="latitude"/>
		<Float64 name="longitude"/>
	</Sequence>

	<dodsBLOB URL="http://localhost:8080/dods/dts/D.blob"/>
</Dataset>

\end{vcode}










\end{document}
