
\documentclass[justify]{dods-paper}
\usepackage{listings}
\usepackage{longtable}
\usepackage{acronym}
\usepackage{xspace}

\rcsInfo $Id: dap-webservices.tex 12461 2005-11-02 00:46:13Z jimg $
\T\lstset{numbers=left, numberstyle=\tiny, stepnumber=2, numbersep=5pt} 
\T\lstset{language=C++}
\T\lstset{escapeinside={(*@}{@*)}}

% \input{boiler}

% Note: to get the glossary to work, run bibtex on the *.gls.aux file,
% then latex the file, then bibtex *.gls, then latex again. Also, make
% sure to set your BST and BIBINPUTS environment variables so that the
% BST and BIB files will be found.
% \makegloss

\title{Porting Your Software to libdap 3.5}
\htmltitle{Porting Your Software to libdap 3.5}
\author{James Gallagher\thanks{OPeNDAP, Inc. jgallagher@opendap.org}}
\date{\rcsInfoDate \\ Revision: \rcsInfoRevision}
\htmladdress{James Gallagher <jgallagher@opendap.org>, \rcsInfoDate, 
  Revision: \rcsInfoRevision}
\htmldirectory{porting-libdap-3.5}
\htmlname{porting-libdap-3.5}

\begin{document}

\maketitle
\T\tableofcontents

\section{Use \lit{BaseTypeFactory} instead of Virtual Constructors}

In older versions of \lit{libdap} (3.2 and earlier), the variables
defined by your subclasses of \lit{Byte}, \ldots, \lit{Grid} were
created using `virtual constructors'\cite{meyers:ecpp}. The code
looked like:

% \begin{vcode}{t}
\begin{lstlisting}[caption={Old Style Virtual Constructor}]
    Byte *
    NewByte(const string &n)
    {
        return new NCByte(n);
    }
\end{lstlisting}
% \end{vcode}

In this example, the virtual constructor returns an instance of the
\lit{NCByte} class, which is apparently a subclass of \lit{Byte}
(although I don't show that explicitly, it's implied since the
\lit{NCByte} pointer is returned via a \lit{Byte} pointer).

The DDS parser knew to use those virtual constructors to make new
instances of those types so that the instances would include the
specializations you added.

This worked fine, why change it? Using the functions, only one set of
specializations could be created per process. The virtual constructors
have been replaced with a factory class. This enables one process to
create many different set of specializations by making several
different subclasses of the factory.

How it works in practice: The class \lit{BaseTypeFactory} defines a factory
class for the \lit{Byte}, \ldots, \lit{Grid} classes and provides a
default implementation. To make a factory for your specializations,
create a subclass of \lit{BaseTypeFactory}. Here's an example:

%\begin{vcode}{t}
\begin{lstlisting}[caption={New Factory Declaration}]
    class NCTypeFactory: public BaseTypeFactory {
    public:
        NCTypeFactory() {} 
        virtual ~NCTypeFactory() {}

        virtual Byte *NewByte(const string &n = "") const;
        // more 'New' methods
    };
\end{lstlisting}
%\end{vcode}

And the implementation:

%\begin{vcode}{t}
\begin{lstlisting}[caption={The Factory Implementation for NCByte}]
    Byte *
    NCTypeFactory::NewByte(const string &n ) const 
    { 
        return new NCByte(n);
    }
\end{lstlisting}
%\end{vcode}

\subsection{Using the factory}

It's great to have the factories, but how does the DDS class know to
use it? It now has a constructor that takes a pointer to a
\lit{BaseTypeFactory} as a formal parameter. This determines which
factory will be used to instantiate variables by \lit{DDS::parse()}.
It's the caller's responsibility to free the factory once the DDS
instance is deleted. See the documentation for those classes for more
information about the new constructors.

%\begin{vcode}{t}
\begin{lstlisting}[caption={DDS Constructor}]
    DDS(BaseTypeFactory *factory, const string &n = "");
\end{lstlisting}
%\end{vcode}

See Appendix~\ref{type-decl} for the complete declaration and
definition files.

\section{STL Iterators Versus \lit{Pix}}

Thanks to the work of Patrick West, \lit{libdap} now has a complete
set of STL iterators that act as replacements for the aging \lit{Pix}
iterators. There are new methods and types for all the classes that
act as containers (DDS, DAS, Constructor, AttrTable). Here's an
example of the old Pix code used to access things in an Structure
variable:\footnote{Example code from the gadods library written by Joe
  Wielgosz at COLA.}

%\begin{vcode}{t}
\begin{lstlisting}[caption={Pix iterators}]
    Structure *structure;
    BaseType *var;
    Pix p = structure->first_var();
    while (p) {
      var = structure->var(p);      
      // do stuff
      structure->next_var(p);
    }
\end{lstlisting}
%\end{vcode}

Here's the code rewritten using the STL iterators:

%\begin{vcode}{t}
\begin{lstlisting}[caption={STL iterators}]
    BaseType *var;
    Constructor::Vars_iter p = structure->var_begin();
    while (p != structure->var_end()) {
      var = *p;      
      // do stuff
      ++p;
    }
\end{lstlisting}
%\end{vcode}

Some things to note:
\begin{itemize}
\item Each of the container types has a method similar to STL's
  \lit{begin()} method which returns an iterator that references the
  start of the contained items.
\item Each of iterators 'points to a pointer.' To get the
  \lit{BaseType} pointer that the iterator \lit{p} references, use the
  dereferencing operator \lit{*} ({\it i.e.} \lit{var = *p}; note that
  \lit{var} is a pointer to a \lit{BaseType}).
\item To advance to the next item, increment the iterator.
\item To test for the end of the container, use the \lit{var\_end()}
  method. You can assign the value of \lit{var\_end()} to a variable
  and test against that if you think the cost of calling the method
  in a loop is too high.
\end{itemize}

Note that the iterators used for Structure and Sequence are defined in
their parent class \lit{Constructor} while \lit{Grid}, \lit{DDS},
\lit{DAS} and \lit{AttrTable} contain their own type definitions. Look
at the class' documentation to see which new methods replace the
`\lit{Pix} methods.'

Note that Patrick also created a set of \lit{Pix} adapters that enable
the old Pix types to work by adapting them to STL iterators. It's best
to treat all the \lit{Pix} code as deprecated, but the adapters ease
the work by enabling you to break it up over several releases.

\section{New Methods in Connect}

Connect no longer contains instances\footnote{Well, they are actually
  still there, but the methods that access them are deprecated and
  will be removed in 3.6.} of DAS, DDS, {\it et c.}, that can hold
values returned from the server. Instead it has methods which return
the objects using value-result parameters. Here are the new methods:

%\begin{vcode}{t}
\begin{lstlisting}[caption={Methods from Connect}]
    virtual void request_das(DAS &das) throw(Error, InternalErr);
    virtual void request_das_url(DAS &das) throw(Error, InternalErr);

    virtual void request_dds(DDS &dds, string expr = "") throw(Error, InternalErr);
    virtual void request_dds_url(DDS &dds) throw(Error, InternalErr);

    virtual void request_data(DataDDS &data, string expr = "") throw(Error, InternalErr);
    virtual void request_data_url(DataDDS &data) throw(Error, InternalErr);
\end{lstlisting}
%\end{vcode}

Note:
\begin{itemize}
\item The methods now take a reference to the object being requested;
  the return is via this parameter.
\item The methods signal errors using exceptions.
\item Patrick has added versions that do not modify the URL at all (the
  ones with the \lit{\_url} suffix).
\end{itemize}

Other changes: Be sure to scan the \lit{Connect} documentation for
other things that have been deprecated. The Connect class was one of
the poorest in \lit{libdap}. I have reduced it considerably and
factored many of the HTTP-specific methods into \lit{HTTPConnect} and
all of the caching into \lit{HTTPCache}. Also note that cache resource
management is not controlled by the \lit{Response}, \lit{HTTPResponse}
and \lit{HTTPCacheResponse} classes. Particularly, the release of
cache resources is now managed by the \lit{Response} class and its children.

\subsection{Connections removed}

In the past \lit{libdap} included a class that could be used to make a
collection of \lit{Connect} instances. This was used in code like the
NetCDF Client Library to store one instance of \lit{Connect} for each
`open file.' That class was called \lit{Connections}. It has been
removed from the library since it makes most sense to just put a copy
of it in your project. If you need it, download the NetCDF client
library source code (or use subversion) and put copies of
\lit{Connections.cc} and \lit{Connections.h} in your source.

\section{Automake support}

All of OPeNDAP's projects will use autoconf, automake and libtool to
build on Unix platforms. To make it easier to write configure.ac
scripts which detect libdap, we include an autoconf m4 macro. Look in
the \lit{m4} directory of libdap for the file \lit{libdap.m4}. Copy
this into your source. Here's an example of the macro's
use:\footnote{Thanks to Patrice Dumas for contributing the macro and
  many patches for our builds.}

%\begin{vcode}{t}
\begin{lstlisting}[caption={libdap.m4}]
    AC_CHECK_LIBDAP([3.5.0],
     [
      LIBS="$LIBS $DAP_LIBS"
      CPPFLAGS="$CPPFLAGS $DAP_CFLAGS"
     ],
     [ AC_MSG_ERROR([Cannot find libdap])
    ])
\end{lstlisting}
%\end{vcode}

\appendix

\section{NCTypeFactory}
\label{type-decl}

\W\textbf{NCTypeFactory.cc}
\begin{lstlisting}[caption={NCTypeFactory.h}]

// -*- mode: c++; c-basic-offset:4 -*-

// This file is part of libdap, A C++ implementation of the OPeNDAP Data
// Access Protocol.

// Copyright (c) 2005 OPeNDAP, Inc.
// Author: James Gallagher <jgallagher@opendap.org>
//
// This library is free software; you can redistribute it and/or
// modify it under the terms of the GNU Lesser General Public
// License as published by the Free Software Foundation; either
// version 2.1 of the License, or (at your option) any later version.
// 
// This library is distributed in the hope that it will be useful,
// but WITHOUT ANY WARRANTY; without even the implied warranty of
// MERCHANTABILITY or FITNESS FOR A PARTICULAR PURPOSE.  See the GNU
// Lesser General Public License for more details.
// 
// You should have received a copy of the GNU Lesser General Public
// License along with this library; if not, write to the Free Software
// Foundation, Inc., 59 Temple Place, Suite 330, Boston, MA  02111-1307  USA
//
// You can contact OPeNDAP, Inc. at PO Box 112, Saunderstown, RI. 02874-0112.

#ifndef nc_type_factory_h
#define nc_type_factory_h

#include <string>

// Class declarations; Make sure to include the corresponding headers in the
// implementation file.

class NCByte;
class NCInt16;
class NCUInt16;
class NCInt32;
class NCUInt32;
class NCFloat32;
class NCFloat64;
class NCStr;
class NCUrl;
class NCArray;
class NCStructure;
class NCSequence;
class NCGrid;

/** A factory for the NetCDF client library types.

    @author James Gallagher
    @see DDS */
class NCTypeFactory: public BaseTypeFactory {
public:
    NCTypeFactory() {} 
    virtual ~NCTypeFactory() {}

    virtual Byte *NewByte(const string &n = "") const;
    virtual Int16 *NewInt16(const string &n = "") const;
    virtual UInt16 *NewUInt16(const string &n = "") const;
    virtual Int32 *NewInt32(const string &n = "") const;
    virtual UInt32 *NewUInt32(const string &n = "") const;
    virtual Float32 *NewFloat32(const string &n = "") const;
    virtual Float64 *NewFloat64(const string &n = "") const;

    virtual Str *NewStr(const string &n = "") const;
    virtual Url *NewUrl(const string &n = "") const;

    virtual Array *NewArray(const string &n = "", BaseType *v = 0) const;
    virtual Structure *NewStructure(const string &n = "") const;
    virtual Sequence *NewSequence(const string &n = "") const;
    virtual Grid *NewGrid(const string &n = "") const;
};

#endif // nc_type_factory_h
\end{lstlisting}
%\end{verbatim}

\W\textbf{NCTypeFactory.cc}
\begin{lstlisting}[caption={NCTypefactory.cc}]

// -*- mode: c++; c-basic-offset:4 -*-

// This file is part of libdap, A C++ implementation of the OPeNDAP Data
// Access Protocol.

// Copyright (c) 2005 OPeNDAP, Inc.
// Author: James Gallagher <jgallagher@opendap.org>
//
// This library is free software; you can redistribute it and/or
// modify it under the terms of the GNU Lesser General Public
// License as published by the Free Software Foundation; either
// version 2.1 of the License, or (at your option) any later version.
// 
// This library is distributed in the hope that it will be useful,
// but WITHOUT ANY WARRANTY; without even the implied warranty of
// MERCHANTABILITY or FITNESS FOR A PARTICULAR PURPOSE.  See the GNU
// Lesser General Public License for more details.
// 
// You should have received a copy of the GNU Lesser General Public
// License along with this library; if not, write to the Free Software
// Foundation, Inc., 59 Temple Place, Suite 330, Boston, MA  02111-1307  USA
//
// You can contact OPeNDAP, Inc. at PO Box 112, Saunderstown, RI. 02874-0112.


#include <string>

#include "NCByte.h"
#include "NCInt16.h"
#include "NCUInt16.h"
#include "NCInt32.h"
#include "NCUInt32.h"
#include "NCFloat32.h"
#include "NCFloat64.h"
#include "NCStr.h"
#include "NCUrl.h"
#include "NCArray.h"
#include "NCStructure.h"
#include "NCSequence.h"
#include "NCGrid.h"

#include "NCTypeFactory.h"
#include "debug.h"

Byte *
NCTypeFactory::NewByte(const string &n ) const 
{ 
    return new NCByte(n);
}

Int16 *
NCTypeFactory::NewInt16(const string &n ) const 
{ 
    return new NCInt16(n); 
}

UInt16 *
NCTypeFactory::NewUInt16(const string &n ) const 
{ 
    return new NCUInt16(n);
}

Int32 *
NCTypeFactory::NewInt32(const string &n ) const 
{ 
    return new NCInt32(n);
}

UInt32 *
NCTypeFactory::NewUInt32(const string &n ) const 
{ 
    return new NCUInt32(n);
}

Float32 *
NCTypeFactory::NewFloat32(const string &n ) const 
{ 
    return new NCFloat32(n);
}

Float64 *
NCTypeFactory::NewFloat64(const string &n ) const 
{ 
    return new NCFloat64(n);
}

Str *
NCTypeFactory::NewStr(const string &n ) const 
{ 
    return new NCStr(n);
}

Url *
NCTypeFactory::NewUrl(const string &n ) const 
{ 
    return new NCUrl(n);
}

Array *
NCTypeFactory::NewArray(const string &n , BaseType *v) const 
{ 
    return new NCArray(n, v);
}

Structure *
NCTypeFactory::NewStructure(const string &n ) const 
{ 
    return new NCStructure(n);
}

Sequence *
NCTypeFactory::NewSequence(const string &n ) const 
{ 
    return new NCSequence(n);
}

Grid *
NCTypeFactory::NewGrid(const string &n ) const 
{ 
    return new NCGrid(n);
}
\end{lstlisting}
%\end{verbatim}

\bigskip

\bibliographystyle{plain}
\T\addcontentsline{toc}{section}{References}
\T\raggedright
\bibliography{../../../boiler/dods}

\end{document}
