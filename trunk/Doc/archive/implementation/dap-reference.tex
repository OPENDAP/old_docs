{\tt BaseType} contains the following member functions:

\begin{description}

\item {\bf Constructor}

\declaration{BaseType(const String \&n = (char *)0, const String \&t =
	(char *)0, xdrproc\_t xdr = NULL);} 

{\tt BaseType}'s constructor takes three arguments: a {\tt String} that is
the name of the variable, a {\tt String} that is the type of the variable,
and a {\tt xdrproc\_t} which is a xdr encoding/decoding function.
Descendents of {\tt BaseType} should define constructors that take the name
parameter as an argument. These constructors should then pass that parameter
to {\tt BaseType}'s constructor along with hardcoded values for {\tt
xdr\_coder} and the type name. The {\tt xdr\_coder} is not used to encode
or decode all data of a particular type; it is used only for certain types of
arrays. However, to simplify use of the DAP, each object's constructor should
provide a fuction pointer for {\tt xdr\_coder} which can encode on of those
objects. Even though {\tt xdr\_coder} is used for arrays, the function itself
should en/decode single objects, not arrays. See Sun's XDR manual for more
information\cite{sun:XDR}.

\item {\bf ptr\_duplicate}

\declaration{virtual BaseType *ptr\_duplicate() = 0;}

When {\tt ptr\_duplicate} is called it returns a pointer to a copy of itself
(not necessarily a {\tt BaseType *}). Since {\tt ptr\_duplicate} is a virtual
function, objects will be duplicated correctly when referred to via pointers
to {\tt BaseType}. 

This member function is abstract and thus must be defined for each descendent
of {\tt BaseType}.

\item {\bf name}

\declaration{String name() const;}

Returns a copy of the name of the variable represented by this instance.

\item {\bf set\_name}

\declaration{void set\_name(const String \&n);}

Given a {\tt String}, set the variable name to that {\tt String}.

\item {\bf type}

\declaration{String type() const;}

Returns a copy of the type of the variable represented by this instance.

\item {\bf set\_type}

\declaration{void set\_type(const String \&t);}

Given a {\tt String}, set the variable type to that {\tt String}.

\item {\bf card}

\declaration {virtual bool card() = 0;}

Returns true if the object is one of the cardinal types. Of particular use
since arrays of these objects are represented specially to improve
transmission efficiency.

This member function is abstract and thus must be defined for each descendent
of {\tt BaseType}.

\item {\bf xdr\_coder}

\declaration{xdrproc\_t xdr\_coder();}

Return the {\tt xdrproc\_t} which can encode/decode this variable ({\tt
xdrproc\_t} is a pointer to a xdr en/decoder function). This is {\em not\/} a
virtual function, nor is there a special member function to set the
`xdr\_coder'; it can only be set with the constructor.

\item {\bf var}

\declaration{virtual BaseType *var(const String \&name = (char *)0);}

Returns a {\tt BaseType *} to a variable contained in a constructor type
(e.g., in an array of Int32, var(...) returns a pointer to Int32).  

NB: Only meaningful for objects in {\tt Array}, {\tt List}, {\tt Structure},
{\tt Sequence}, {\tt Function}, or {\tt Grid}.

\item {\bf add\_var}

\declaration{virtual void add\_var(BaseType *v, Part p = nil);}

Add variable V to part P of the object.     Returns: void

NB: Only meaningful for objects in {\tt Array}, {\tt List},
{\tt    Structure}, {\tt Sequence}, {\tt Function}, or {\tt
Grid}.

\item {\bf width}

\declaration{virtual unsigned int width() = 0;}

Return the size in bytes required to store the variable.

This member function is abstract and thus must be defined for
each descendent of {\tt BaseType}.

\item {\bf read}

\declaration{virtual bool read(String dataset, String var\_name, String
	constraint) = 0;} 

Given {\tt String}s for the data set name, variable name, and constraint
expression, read the variable from the data set into the object.

This member function is abstract and thus must be defined for each descendent
of {\tt BaseType}. For each API this is point where data is moved from the
dataset into the DODS DAP software so that it can be sent to the client.

\item {\bf read\_val}

\declaration{virtual unsigned int read\_val(void **val) = 0;}

Reads the value of the variable {\em in an internal buffer\/} and places it
in the memory referenced by {\tt **VAL}.  Either the caller must allocate
enough storage for {\tt **VAL} or set it to null.  If set to null then {\tt
new} will be used to allocate storage and the caller is responsible for
deallocatinbg storage with {\tt delete}

This member function is abstract and thus must be defined for each descendent
of {\tt BaseType}.

\item {\bf store\_val}

\declaration{virtual unsigned int store\_val(void *val, bool reuse = false) =
	0;} 

Stores the value pointed to by VAL {\em in the objects internal buffer\/}.
It returns the size (in bytes) of the information copied.

This member function is abstract and thus must be defined for each descendent
of {\tt BaseType}.

\item {\bf serialize}

\declaration{virtual bool serialize(bool flush = false) = 0;}

Move the contents of the object (which should contain the value of a
variable) onto the network. If the data needs to be transformed from the
local representation to network representation (as defined by the \Dap), then
that is done as well.

This member function is abstract and thus must be defined for each descendent
of {\tt BaseType}.

NB: This member function does {\em not\/} check to see if valid data is in
the object.

\item {\bf deserialize}

\declaration{virtual bool deserialize(bool reuse = false) = 0;}

Copy data from the network to the object. If the data needs to be transformed
from network to local representation, that step is performed as well. If
REUSE is true, then the internal buffer is reused rather an being deleted and
reallocated using new. No checks are made to ensure that the old buffer is
large enough to hold the new incoming data; that is the caller's
responsibility. 

This member function is abstract and thus must be defined for each descendent
of {\tt BaseType}.

NB: This member function does {\em not\/} check to see if the contents of the
network are valid.

\item {\bf expunge()}

\declaration{bool expunge();}

Flush the \_out stream writing all data to the network.

Returns: true if the stream could be flushed, false otherwise.

\item {\bf print\_decl}

\declaration{virtual void print\_decl(ostream \&os, String space = " ",  bool
	print\_semi = true);} 

Print the declaration for this variable. This member function takes three
arguments: an {\tt ostream} where text output will be sent, a {\tt String} of
spaces which are prepended to the declaration (so they can be nicely
indented), and a {\tt bool} which controls output of a trailing semicolon and
newline ({\tt true} == print the semicolon and newline).

\item {\bf print\_val}

\declaration{virtual void print\_val(ostream \&os, String space = "", bool
	print\_decl\_p = true) = 0;} 

Print the value of the variable. This member function takes two arguments: an
{\tt ostream} where text output will be sent, and a {\tt String} of spaces
that are prepended to the value.  PRINT\_DECL\_P when true causes the
variable declaration to be prepended.

This member function is abstract and thus must be defined for each descendent
of {\tt BaseType}.

\item {\bf check\_semantics}

\declaration{virtual bool check\_semantics(bool all = false);}

Return {\tt true} if the semantics of this variable are correct. This member
function is next to useless for direct descendents of {\tt BaseType}, but for
children of {\tt CtorType}, it is very valuable (it is possible to declare
variable in the DDS that will parse but which make no sense).  This member
function takes a single argument: a {\tt bool} which when {\tt true} causes
constructor-type variables to be recursively checked.

\end{description}

Array:

\begin{description}

\item {\bf Constructor}

\declaration{Array(const String \&n = (char *)0,  BaseType *v = 0);}

Array's constructor takes two arguments: {\tt String}s for the variable name
and a {\tt BaseType} pointer which references the variable which is an
array. See above.

\item {\bf append\_dim}

\declaration{void append\_dim(int dim, String name = "");}

Add a dimension to the array. This member function takes an {\tt int}
argument for the size of the dimension and a String argument for its name.

Returns: void.

\item {\bf first\_dim}

\declaration{Pix first\_dim();}

Return a {\tt Pix} which references the first dimension of the {\tt Array}.

\item {\bf next\_dim}

\declaration{void next\_dim(Pix \&p);}

Given a {\tt Pix} which refers to one of the {\tt Array}'s dimensions, update
the {\tt Pix} so that it refers to the next dimension.

\item {\bf dimension\_size}

\declaration{int dimension\_size(Pix p);}

Given a {\tt Pix} which refers to a dimension, return the size of that
dimension as an {\tt int}.

\item {\bf dimension\_name}

\declaration{String dimension\_name(Pix p);}

Given a {\tt Pix} which refers to a dimension, return the name of that
dimension as an {\tt String}.

\item {\bf dimensions}

\declaration{unsigned int dimensions();}

Return the number of dimensions in this array.

\item {\bf print\_decl}

\declaration{virtual void print\_decl(ostream \&os, String space = " ", bool
	print\_semi = true);}

Print the declaration for this variable. If SPACE is given, us that as left
padding, otherwise use the default (which is one space). If PRINT\_SEMI is
true, end the declaration with a semicolon and a newline (the default).

Returns: void

\item {\bf print\_val}

\declaration{virtual void print\_val(ostream \&os, String space = "", bool
	print\_decl\_p = true);}

Print the declaration for the object (using print\_decl) and follow that with
the value contained in the variable. If PRINT\_DECL\_P is true, then the
declaration and an equal sign is prepended to the output and a semicolon and
newline are appended to the output.

Returns: Void

\end{description}

List:

\begin{description}

\item {\bf length}

\declaration{unsigned int length();}

Returns: The number of elements in the list.

NB: This is {\em not\/} the same as {\tt BaseType}'s {\tt width()} member
function---that returns the {\em number of bytes\/} required to store the
{\tt List}.

\item {\bf set\_length}    
\declaration{void set\_length(unsigned int l);}

Set the length of a {\tt List} object.

\item {\bf set\_size}    
\declaration{void set\_size(unsigned int s);}

Deprecated synonym for {\tt set\_length}.

\end{description}

{\Tt Structure}:

\begin{description}

\item {\bf var}    
\declaration{virtual BaseType *var(const String \&name);}  
\declaration{BaseType *var(Pix p);}

Get the {\tt BaseType} pointer which points to the named variable, or get the
{\tt BaseType} pointer which points to the variable referred to by {\tt Pix}
p. If the variable is not a member of the {\tt Structure}, return NULL.

\item {\bf add\_var}

\declaration{virtual void add\_var(BaseType *, Part p = nil);}

Add the variable referred to by the {\tt BaseType} pointer to {\tt Part} p of
the {\tt Structure}.

\item {\bf first\_var}

\declaration{Pix first\_var();}

Return a {\tt Pix} the refers to the first member of the {\tt Structure}.

\item {\bf next\_var}

\declaration{void next\_var(Pix \&p);}

Update {\tt Pix} p to refer to the next member of the {\tt Structure}. Set p
to NULL when there are no more members.

\end{description}

Sequence:

\begin{description}

\item {\bf var}    
\declaration{virtual BaseType *var(const String \&name);}  
\declaration{BaseType *var(Pix p);}

Get the {\tt BaseType} pointer which points to the named variable, or get the
{\tt BaseType} pointer which points to the variable referred to by {\tt Pix}
p. If the variable is not a member of the {\tt Sequence}, return NULL.

\item {\bf add\_var}

\declaration{virtual void add\_var(BaseType *, Part p = nil);}

Add the variable referred to by the {\tt BaseType} pointer to {\tt Part} p of
the {\tt Sequence}.

\item {\bf first\_var}

\declaration{Pix first\_var();}

Return a {\tt Pix} the refers to the first member of the {\tt Sequence}.

\item {\bf next\_var}

\declaration{void next\_var(Pix \&p);}

Update {\tt Pix} p to refer to the next member of the {\tt Sequence}. Set p
to NULL when there are no more members.

\end{description}

Function:

\begin{description}

\item {\bf first\_indep\_var}

\declaration{Pix first\_indep\_var();}

Returns a {\tt Pix} which refers to the first member in the {\em
independent\/} section of the {\tt Function}.

\item {\bf next\_indep\_var}

\declaration{void next\_indep\_var(Pix \&p);}

Update {\tt Pix} p to refer to the next member in the {\em independent\/}
section of the {\tt Function}. Set p to NULL when there are no more members.

\item {\bf indep\_var}

\declaration{BaseType *indep\_var(Pix p);}

Get the {\tt BaseType} pointer which points to the {\em independent\/}
variable referred to by {\tt Pix} p. If the variable is not a member of the
{\tt Function}, return NULL.

\item {\bf first\_dep\_var}

\declaration{Pix first\_dep\_var();}

Returns a {\tt Pix} which refers to the first member in the {\em dependent\/}
section of the {\tt Function}.

\item {\bf next\_dep\_var}

\declaration{void next\_dep\_var(Pix \&p);}

Update {\tt Pix} p to refer to the next member in the {\em dependent\/}
section of the {\tt Function}. Set p to NULL when there are no more members.

\item {\bf dep\_var}

\declaration{BaseType *dep\_var(Pix p);}

Get the {\tt BaseType} pointer which points to the {\em dependent\/} variable
referred to by {\tt Pix} p. If the variable is not a member of the {\tt
Function}, return NULL.

\end{description}

Grid:

\begin{description}

\item {\bf array\_var}

\declaration{BaseType *array\_var();}

Returns a {\tt BaseType} pointer to the $N$ dimensional variable that
contains the grid values.

\item {\bf first\_map\_var}

\declaration{Pix first\_map\_var();}

Returns a {\tt Pix} which refers to the first member in the {\em map/}
section of the {\tt Grid}.

\item {\bf next\_map\_var}

\declaration{void next\_map\_var(Pix \&p);}

Update {\tt Pix} p to refer to the next member in the {\em map/} section of
the {\tt Grid}. Set p to NULL when there are no more members.

\item {\bf map\_var}

\declaration{BaseType *map\_var(Pix p);}

Get the {\tt BaseType} pointer which points to the {\em map\/} variable
referred to by {\tt Pix} p. If the variable is not a member of the {\tt
Grid}, return NULL.

\end{description}

The class {\tt DAS} defines the following member functions:

\begin{description} \item {\bf Constructor}

\declaration{DAS(AttrTablePtr dflt=(void *)NULL, unsigned int
	sz=DEFAULT\_INITIAL\_CAPACITY);} 

Make a new, empty, instance of {\tt DAS}. You can specify a default
attribute table for each each variable and the size of the hash table {\tt
DAS} uses to reference the {\tt AttrTable} for each variable. However, the
internal representation for this class may change, and this parameter might
become meaningless.

\item {\bf first\_var}

\declaration{Pix first\_var();}    

Return a {\tt Pix} that refers to the first variable in the {\tt DAS}.

\item {\bf next\_var}

\declaration{void next\_var(Pix \&p);}  

Given a {\tt Pix} that refers to an entry in the {\tt DAS}, update that {\tt
Pix} to refer to the next variable. Set the {\tt Pix} to NULL if no more
variables remain in the {\tt DAS}.

\item {\bf get\_name}

\declaration{String \&get\_name(Pix p);}

Return a reference to the name of the variable referred to by {\tt Pix} p in
a {\tt String}.

\item {\bf get\_table}

\declaration{AttrTable *get\_table(Pix p);} 
\declaration{AttrTable *get\_table(const String \&name);}

\declaration{AttrTable *get\_table(const char *name);}\footnote{{\tt (char
*)} version. It is necessary to provide this definition because without
explicitly defining a member function for {\tt (char *)}, \Cpp\ will confuse
it with a {\tt Pix} since {\tt Pix} is represented by a {\tt (void *)}.}

Return the {\tt AttrTable} associated with variable NAME.

\item {\bf add\_table}

\declaration{AttrTable *add\_table(const String \&name, AttrTable *at);}
\declaration{AttrTable *add\_table(const char *name, AttrTable *at);}

Add the {\tt AttrTable} pointed to by AT at the {\tt AttrTable} for variable
NAME.

Return AT if not error is detected, NULL otherwise.

\item {\bf parse}

\declaration{bool parse(String fname);}
\declaration{bool parse(int fd);}
\declaration{bool parse(FILE *in=stdin);}

Parse the data set attribute structure(s) and then store the variables and
their attributes in the current {\tt DAS} object.  If any of the input
sources contains more than one data set attribute structure, the variables
and attributes they describe are all merged into the current {\tt DAS}
object.

Return TRUE if the input source was parsed without error, FALSE otherwise.

\item {\bf print}

\declaration{bool print(ostream \&os = cout);}

Print the contents of the data set attribute structure to OS. 

\end{description}

DDS:

\begin{description}

\item {\bf Constructor}

\declaration{DDS(const String \&n = (char *)0);}

Create a new {\tt DDS} object for the data set named N.

\item {\bf get\_dataset\_name}

\declaration{String get\_dataset\_name();}

Return the data set name.

\item {\bf set\_dataset\_name}

\declaration{void set\_dataset\_name(const String \&n);}

Set the data set name to be N.

\item {\bf add\_var}

\declaration{void add\_var(BaseType *bt);}

Add the variable pointed to by BT to the data set. Unlike the {\tt DAS}, the
{\tt DDS} retains information about the order of variables.

\item {\bf del\_var}

\declaration{void del\_var(const String \&n);}

Delete the variable named N from the data set.

\item {\bf first\_var}

\declaration{Pix first\_var();}

Return a {\tt Pix} the refers to the first variable in the data set.

\item {\bf next\_var}

\declaration{void next\_var(Pix \&p);}

Given a {\tt Pix} P that refers to a variable in the {\tt DDS}, update that
{\tt Pix} to refer to the next variable in the data set. If there are no more
variable, set P to NULL.

\item {\bf var}

\declaration{BaseType *var(Pix p);}
\declaration{BaseType *var(const String \&n);}
\declaration{BaseType *var(const char *n);} \footnote{{\tt (char *)}
version. As with the {\tt get\_table} member function in the {\tt DAS} class,
it is necessary to provide this definition since \Cpp\ will confuse it with a
{\tt Pix} unless it is explicitly defined as a member function for {\tt (char
*)}. This is because {\tt Pix} is represented by a {\tt (void *)}.}

Return a pointer to the variable referred to by {\tt Pix} P,
or with the name N, in the data set. If the variable does not
exist, return NULL.

\item {\bf parse}

\declaration{bool parse(String fname);}
\declaration{bool parse(int fd);}
\declaration{bool parse(FILE *in=stdin);}

Parse the data set descriptor structure(s) from FNAME, FD, or IN.  If more
than one DDS is parsed, the second DDS and any additional DDSs are merged
with the existing in-memory DDS.

Return TRUE if the DDSs can all be parsed, FALSE otherwise.

\item {\bf print}

\declaration{bool print(ostream \&os = cout);}

Print the {\tt DDS} on OS.

Return TRUE if the {\tt DDS} was printed, FALSE otherwise.

\item {\bf check\_semantics}

\declaration{bool check\_semantics(bool all = false);}

Check the semantics of the data set. If ALL is TRUE, recursively check the
semantics of the variables in the data set (e.g., descend into {\tt
Structure} variables and check the semantics of their components).

Return TRUE if the variables are OK, FALSE otherwise.

\end{description}

{\tt Connect} defines the following member functions:

\begin{description}

\item {\bf Constructor}

\declaration{Connect(const String \&name, const String \&api = "");}

Make a new instance of {\tt Connect}. NAME is the URL or local file name to
open and API is the name of the API. The API name is used to build URLs which
access different parts of the data set (See \DDD).

\item {\bf is\_local}

\declaration{bool is\_local();}

Return TRUE if the data set managed by this instance of {\tt Connect} is a
local data set, FALSE otherwise.

\item {\bf URL}

\declaration{const String \&URL();}

Return the URL which points to the remote data et managed by this instance of
{\tt Connect}. If this {\tt Connect} manages a local data set, return an
empty string (``''). This is the string passed to the constructor via the
NAME parameter.

\item {\bf DAS}

\declaration{DAS \&das();}

Return a reference to the {\tt DAS} object associated with the open data set
if this {\tt Connect} manages a remote data set, otherwise the return value
is undefined.

\item {\bf DDS}

\declaration{DDS \&dds();}

Returns a reference to the {\tt DDS} object associated with the open data set
if this {\tt Connect} manages a remote data set, otherwise the return value
is undefined.

\item {\bf request\_das}

\declaration{bool request\_das(const String \&ext = "das");}

Get the {\tt DAS} from the remote server.  The parameter EXT is a string that
tells the data server CGI which filter program is to be run to handle the
request.  The default is "das".

Returns TRUE if successful, FALSE if unsuccessful.

\item {\bf request\_dds}

\declaration{bool request\_dds(const String \&ext = "dds");}

Get the {\tt DDS} from the remote server. The parameter EXT is a string that
tells the data server CGI which filter program is to be run to handle the
request.  The default is "dds".

Returns TRUE if successful, FALSE if unsuccessful.

\item {\bf request\_data}

\declaration{bool request\_data(const String expr, bool asynch = false, const
	String \&ext = "dods");}

This function makes a data read request to the remote API server and links
the object's xdrin file pointers to the data stream.  This function does not
actually read the data, it only sets up the BaseType static class member so
the data can be read.

The string EXPR is for use with contraint expressions, which are not yet
implemented.

If ASYNCH is FALSE (default), then the data will be read synchronouly; if
ASYNCH is TRUE, then the data will be read asynchronouly.

The parameter EXT is a string that tells the data server CGI which filter
program is to be run to handle the request.  The default is "dods".

Returns TRUE if successful and FALSE if unsuccessful.

\end{description}

{\tt Connections} defines the following methods:

\begin{description}

\item {\bf Constructor}

\declaration{Connections(int max\_connections = MAX\_CONNECTIONS);}

Make a new instance of {\tt Connections}. MAX\_CONNECTIONS defaults to 64.

\item {\bf operator[]}

\declaration{ConnectPtr \&operator[](int i);}

Return the {\tt Connect} pointer associated with integer I. If you want to
manage {\tt Connect}s using something other than integer references, you must
subclass {\tt Connections}.

\item {\bf Add\_connect}

\declaration{int add\_connect(Connect *c);}

Add the {\tt Connect} object pointed to by C to the instance of {\tt
Connections}.

Returns the integer which refers to the managed {\tt Connect} pointer.

\item {\bf del\_connect}

\declaration{void del\_connect(int i);}

Remove the {\tt Connect} referred to by integer I from the
{\tt Connections}  object. 

\end{description}

