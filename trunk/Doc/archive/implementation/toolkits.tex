
% This file describes the client and server toolkits. These are used to build
% the server CGIs and the surrogate library for a new API.
%
% jhrg 11/15/94
%
% $Id$

\documentstyle[code,12pt,html,psfig,margins]{article}

\psfigurepath{toolkits-figs}

%
% These are html links which are used often enough in writing about DODS to
% merit an input file.
% jhrg. 4/17/94
%
% File rationalized and updated while writing the DODS User
% Guide. Also includes other useful abbreviations.
% tomfool 3/15/96
%
% $Id$
%
% HTML references to DODS documents
%
% For most of the DODS documentation there are three html references defined:
% 1) The upper case \newcommand produces the title of the paper, an
%    html link in the online documentation and a footnote in the paper
%    version.  
% 2) A capitalized version produces a capitalized description of the
%    paper and a link in the online version (but no footnote in the
%    paper version). 
% 3) A lower case version which produces a lower case description of
%    the paper and a html link, but no footnote.

% External references to documents:
%
% In order for external references to work (via latex2html) the labels used
% throughout the documents must be unique. The text labels are all prefixed
% with a document identifier (e.g., ddd) with a colon separater. The
% \externalrefs below (way below...) includes the perl files html.sty uses to
% make the cross refs.

\newcommand{\DODS}{\htmladdnormallink{Distributed Oceanographic Data System}
  {http://www.unidata.ucar.edu/packages/dods/}}

\newcommand{\Dods}{\htmladdnormallink{DODS}{http://www.unidata.ucar.edu/packages/dods/}}

\newcommand{\wrkshp}{\htmladdnormallink{Report on the First Workshop for the
    Distributed Oceanographic Data System, Proposed System Architectures}
  {http://www.unidata.ucar.edu/packages/dods/archive/reports/workshop1/section3.13.html}}

% not DODS URLs, but useful all the same...

\newcommand{\www}{\htmladdnormallink{World Wide Web} 
  {http://www.w3.org/hypertext/WWW/TheProject.html}}
\newcommand{\url}{\htmladdnormallink{Uniform Resource Locator}
  {http://www.w3.org/hypertext/WWW/Addressing/URL/url-spec.html}}

\newcommand{\uri}{\htmladdnormallinkfoot{Uniform Resource Identifiers}
  {http://www.w3.org/hypertext/WWW/Addressing/URL/uri-spec.html}}

\newcommand{\urn}{\htmladdnormallinkfoot{Uniform Resource Name}
  {http://www.acl.lanl.gov/URI/archive/uri-archive.messages/1143.html}}

\newcommand{\HTTPD}{\htmladdnormallinkfoot{HTTPD}
  {http://hoohoo.ncsa.uiuc.edu/docs/Overview.html}}

\newcommand{\HTTP}{\htmladdnormallinkfoot{HyperText Transfer Protocol}
  {http://www.w3.org/hypertext/WWW/Protocols/HTTP/HTTP2.html}}

%\newcommand{\HTML}{\htmladdnormallinkfoot{HyperText Markup Language}
%  {http://www.w3.org/hypertext/WWW/MarkUp/MarkUp.html}}

% CERN used to be `Conseil Europeen pour la Recherche Nucleaire'
\newcommand{\CERN}
  {\htmladdnormallink{European Laboratory for Particle Physics}
  {http://www.cern.ch/}}

\newcommand{\NCSA}
  {\htmladdnormallink{National Center for Supercomputing Applications}
  {http://www.ncsa.uiuc.edu/}}

\newcommand{\WWWC}{\htmladdnormallink{World Widw Web Consortium}
  {http://www.w3.org/}}

\newcommand{\CGI}{\htmladdnormallinkfoot{Common Gateway Interface}
  {http://hoohoo.ncsa.uiuc.edu/cgi/overview.html}}

\newcommand{\MIME}{\htmladdnormallink{Multipurpose Internet Mail Extensions}
  {http://www.cis.ohio-state.edu/htbin/rfc/rfc1590.html}}

\newcommand{\netcdf}{\htmladdnormallink{NetCDF}
  {http://www.unidata.ucar.edu/packages/netcdf/guide.txn_toc.html}}

\newcommand{\JGOFS}{\htmladdnormallink{Joint Geophysical Ocean Flux Study}
  {http://www1.whoi.edu/jgofs.html}}

\newcommand{\jgofs}{\htmladdnormallink{JGOFS}
  {http://www1.whoi.edu/jgofs.html}}

\newcommand{\hdf}{\htmladdnormallink{HDF}
  {http://www.ncsa.uiuc.edu/SDG/Software/HDF/HDFIntro.html}}

\newcommand{\Cpp}{{\rm {\small C}\raise.5ex\hbox{\footnotesize ++}}}

% Commands

\newcommand{\pslink}[1]{\small
\begin{quote}
  A \htmladdnormallink{postscript}{#1} version of this document is
  available.  You may also use \htmladdnormallink{anonymous
  ftp}{ftp://ftp.unidata.ucar.edu/pub/dods/ps-docs/} to access postscript files
  of all of the DODS documentation.
\end{quote}
\normalsize
}

\newcommand{\declaration}[1]{\small
{\tt {#1}}
\normalsize
}

% All the links from here down are for the white papers. I'm not sure that any
% of these are still valid given the reorganization of the web site. 5/12/98
% jhrg.

\newcommand{\DDA}{\htmladdnormallinkfoot{DODS---Data Delivery Architecture}
  {http://www.unidata.ucar.edu/packages/dods/archive/design/data-delivery-arch/data-delivery-arch.html}}
\newcommand{\Dda}{\htmladdnormallink{Data Delivery Architecture}
  {http://www.unidata.ucar.edu/packages/dods/archive/design/data-delivery-arch/data-delivery-arch.html}}
\newcommand{\dda}{\htmladdnormallink{data delivery architecture}
  {http://www.unidata.ucar.edu/packages/dods/archive/design/data-delivery-arch/data-delivery-arch.html}}

\newcommand{\DDD}{\htmladdnormallinkfoot{DODS---Data Delivery Design}
  {http://www.unidata.ucar.edu/packages/dods/archive/design/data-delivery-design/data-delivery-design.html}}
\newcommand{\Ddd}{\htmladdnormallink{Data Delivery Design}
  {http://www.unidata.ucar.edu/packages/dods/archive/design/data-delivery-design/data-delivery-design.html}}
\newcommand{\ddd}{\htmladdnormallink{data delivery design}
  {http://www.unidata.ucar.edu/packages/dods/archive/design/data-delivery-design/data-delivery-design.html}}

\newcommand{\URL}{\htmladdnormallinkfoot{DODS---Uniform Resource Locator}
  {http://www.unidata.ucar.edu/packages/dods/archive/design/urls/urls.html}}
\newcommand{\Url}{\htmladdnormallink{Uniform Resource Locator}
  {http://www.unidata.ucar.edu/packages/dods/archive/design/urls/urls.html}}
\newcommand{\dodsurl}{\htmladdnormallink{uniform resource locator}
  {http://www.unidata.ucar.edu/packages/dods/archive/design/urls/urls.html}}

\newcommand{\DAP}{\htmladdnormallinkfoot{DODS---Data Access Protocol}
  {http://www.unidata.ucar.edu/packages/dods/archive/design/api/api.html}}
\newcommand{\Dap}{\htmladdnormallink{Data Access Protocol}
  {http://www.unidata.ucar.edu/packages/dods/archive/design/api/api.html}}
\newcommand{\dap}{\htmladdnormallink{data access protocol}
  {http://www.unidata.ucar.edu/packages/dods/archive/design/api/api.html}}

\newcommand{\SOFT}
        {\htmladdnormallinkfoot{DODS---Software Development Environment}
  {http://www.unidata.ucar.edu/packages/dods/archive/managment/software/software.html}}
\newcommand{\Soft}{\htmladdnormallink{Software Development Environment}
  {http://www.unidata.ucar.edu/packages/dods/archive/managment/software/software.html}}
\newcommand{\soft}{\htmladdnormallinkfoot{software development environment}
  {http://www.unidata.ucar.edu/packages/dods/archive/managment/software/software.html}}

% I used to call the DAP the API, then for a few days I called it the
% DTP. These def's were used then. Rather than find everywhere they are used
% (not that hard, but...) I just hacked the macros. NB: the source file for
% the DAP document is still called `api.tex'

\newcommand{\API}{\htmladdnormallinkfoot{DODS---Data Access Protocol}
  {http://www.unidata.ucar.edu/packages/dods/archive/design/api/api.html}}
\newcommand{\api}{\htmladdnormallink{data access protocol}
  {http://www.unidata.ucar.edu/packages/dods/archive/design/api/api.html}}

\newcommand{\DTP}{\htmladdnormallinkfoot{DODS---Data Access Protocol}
  {http://www.unidata.ucar.edu/packages/dods/archive/design/api/api.html}}
\newcommand{\dtp}{\htmladdnormallink{data access protocol}
  {http://www.unidata.ucar.edu/packages/dods/archive/design/api/api.html}}

\newcommand{\TOOLKIT}{\htmladdnormallinkfoot{DODS---Client and Server Toolkit}
  {http://www.unidata.ucar.edu/packages/dods/archive/implementation/toolkits/toolkits.html}}
\newcommand{\Toolkit}{\htmladdnormallink{Client and Server Toolkit}
  {http://www.unidata.ucar.edu/packages/dods/archive/implementation/toolkits/toolkits.html}}
\newcommand{\toolkit}{\htmladdnormallink{client and server toolkit}
  {http://www.unidata.ucar.edu/packages/dods/archive/implementation/toolkits/toolkits.html}}

\newcommand{\TKR}{\htmladdnormallinkfoot{DODS---DAP Toolkit Reference}
  {http://www.unidata.ucar.edu/packages/dods/archive/implementation/toolkits/toolkits.html}}
\newcommand{\Tkr}{\htmladdnormallink{DAP Toolkit Reference}
  {http://www.unidata.ucar.edu/packages/dods/archive/implementation/toolkits/toolkits.html}}
\newcommand{\tkr}{\htmladdnormallink{DAP toolkit reference}
  {http://www.unidata.ucar.edu/packages/dods/archive/implementation/toolkits/toolkits.html}}

% I know that these are wrong. The papers have been moved to the gso web
% site, but I'm not sure exactly where. 5/12/98 jhrg

\newcommand{\DM}{\htmladdnormallinkfoot{DODS---Data Model}
  {http://lake.mit.edu/dods-dir/dodsdm2.html}}
\newcommand{\Dm}{\htmladdnormallink{Data Model}
  {http://lake.mit.edu/dods-dir/dodsdm2.html}}
\newcommand{\dm}{\htmladdnormallink{data model}
  {http://lake.mit.edu/dods-dir/dodsdm2.html}}

\newcommand{\DD}{\htmladdnormallinkfoot{DODS---Data Delivery}
  {http://lake.mit.edu/dods-dir/dods-dd.html}}

% external refs for DODS documents

\externallabels{http://www.unidata.ucar.edu/packages/dods/design/api}
  {/home/www/packages/dods/archive/design/api/labels.pl}
\externallabels{http://www.unidata.ucar.edu/packages/dods/design/data-delivery-arch}
  {/home/www/packages/dods/archive/design/data-delivery-arch/labels.pl}
\externallabels{http://www.unidata.ucar.edu/packages/dods/design/data-delivery-design}
  {/home/www/packages/dods/archive/design/data-delivery-design/labels.pl}
\externallabels{http://www.unidata.ucar.edu/packages/dods/implementation/toolkits}
  {/home/www/packages/dods/archive/implementation/toolkits/labels.pl}

%\renewcommand{\externalref}[2]{#2}

%%% Local Variables: 
%%% mode: latex
%%% TeX-master: t
%%% End: 


\begin{document}

\title{DODS---The Client and Server Toolkit} 
\author{}
\date{\today}

\maketitle

\begin{abstract}
  
  This paper describes how to use the DODS Client and Server Toolkit software
  to build both DODS client-libraries and data servers. The Client and Server
  Toolkit software contains an implementation of the DODS \Dap\ as well as
  additional software to simplify building DODS client-libraries and servers.
  This software is implemented in \Cpp\ as four groups of classes and some
  utility functions: one class hierarchy for the data types, one for each of
  the {\tt DAS} and {\tt DDS} described in the \dap\ and one to manage the
  virtual connection in the client library.  Each of these groups of classes
  are described in this paper. Note that this document is not a reference
  volume for the member functions in each of the classes; that can be found
  in ``\TKR''.

\end{abstract}


% Biolerplate text about asking questions using the extra button in the
% navigation section.
%
% jhrg 1/24/96
%
% $Id$

\small

\centerline{{\bf Questions?}} 

\begin{quotation}

\begin{htmlonly}
  If you have questions about anything in this document, please send me those
  questions. Each of the sections has a small navigation panel at both its
  start and end which contains a button marked `Questions ?'. Use that to
  send me email. Please try to include the section name or number
  corresponding to your question; I plan on adding information to the
  sections based on the questions people ask about them.
\end{htmlonly}

\begin{latexonly}
  If you have questions about anything in this document, please send me those
  questions at {\em jgallagher@gso.uri.edu}. Please try to include the
  section name or number corresponding to your qestion; I plan on adding
  information to the sections based on the questions people ask about them.
\end{latexonly}

\end{quotation}

\normalsize


% Biolerplate text warning that anything can (and almost certainly will)
% change. 
%
% jhrg 3/25/94
%
% $Id$

\small

\centerline{{\bf Note}} 

\begin{quotation}

\begin{htmlonly}
  This is a working document made available to solicit comments. To receive
  e-mail notice of new or significantly changed documents, send e-mail
  to \htmladdnormallink{majordomo@unidata.ucar.edu}
  {mailto:majordomo@unidata.ucar.edu} with 'subscribe dods' in the
  body of the message.
\end{htmlonly}

\begin{latexonly}
  This is a working document made available to solicit comments. To receive
  e-mail notice of new or significantly changed documents, send e-mail
  to majordomo@unidata.ucar.edu with 'subscribe dods' in the body of
  the message. 

  Hypertext references in the hardcopy version of this document are
  represented using footnotes that contain a Universal Resource Locator (URL)
  to the referenced document.  To read the online documentation, open the URL
  `http://www.unidata.ucar.edu/' with a World Wide Web browser.
\end{latexonly}

\end{quotation}

\normalsize
 

% Biolerplate text expaining that this document is intended for DODS
% component implementors.
%
% jhrg 8/2/94
%
% Requires html-refs.tex jhrg 11/15/94
%
% $Id$

\small

\begin{quotation}
  This document is written for people who will be implementing user libraries
  and/or data servers for the Distributed Oceanographic Data System. It
  assumes that the reader is familiar with the implementation of software
  systems using C and \Cpp\ within the UNIX programming environment.
\end{quotation}

\normalsize

\begin{htmlonly}
\pslink{file://dods.gso.uri.edu/pub/DODS/toolkits.ps}
\end{htmlonly}

\clearpage

\tableofcontents

\clearpage

\section{Introduction}

This document describes how to use the DODS toolkit software to build both
DODS data servers and client-libraries. Before using the toolkit software it
is important that the basic ideas behind DODS the design of the data
transport system be understood. For this information see ``\DDA'' and
``\DDD''. The data transport system using the \dap\ to move information from
a data server to a client-library; this protocol is described in ``\DAP''. In
addition, DODS uses as much of the existing \www\ infrastructure as it can;
to find out about DODS' use of URLs, see ``\URL''.

The body of this document is divided into five
sections. Section~\ref{tk:overview} provides background information on the
organization of the toolkit software. Section~\ref{tk:manage-conns} describes
how to use the Network I/O classes to manage virtual
connections. Section~\ref{tk:subclassing} discusses how to sub-class the
toolkit \Cpp\ classes so that they are specialized for your specific
use. Section~\ref{tk:using} describes in detail how to write certain sections
of both the data server and the client-library for a new API. Lastly,
Section~\ref{tk:linking} describes how to link user programs with the new
client-library implementation of an API. Note that this is {\em not\/} a
reference volume. See \TKR\ for a concise listing of the member functions in
each of the toolkit's classes.

Missing from this document is a discussion of translation between data
types. This  topic will be addressed in a future version of the toolkit
software and at that time this document will be updated.

Because the type of information presented in a document like this depends
to a large extent on the needs of its readers I would like as much feedback
as possible. In particular, if you have any questions about individual
sections, email me those questions and I'll send back an answer as well as
including that information in the next version of this document. I hope that
this will improve the overall content much more rapidly than if I try to
guess what people need to know. Note that the online version of the document
has `Question' buttons in the navigation panels of each frame which you can
use to send me email, otherwise send the questions to
jgallagher@gso.uri.edu. 

\section{Overview of the Toolkit Software} 
\label{tk:overview}

The \TOOLKIT\ software consists of a collection of \Cpp\ classes distributed
in two libraries and is intended to simplify building the data servers and
client-libraries described in ``\DDD''. The toolkit software contains an
implementation of the data access protocol described in ``\DAP'' as well as
additional software that simplifies maintenance of a virtual connection in
the client-library. The implementation of the DAP is contained in the 
\Cpp\ class library {\tt libdap++}. The network I/O software is contained in
the \Cpp\ class library {\tt libnetio}.

The DAP implementation contains three class hierarchies: one for each of the
{\tt DAS} and {\tt DDS} objects, rooted in classes with those names, and one
for the variables, rooted at the class {\tt BaseType}. The following sections
present an overview of these class hierarchies and describe, in general
terms, how they are used to build the client-library and data server. See
Section~\ref{das-dds-objects}.

The netio implementation consists of two classes, one that manages a single
connection ({\tt Connect}) and one that manages a group of those objects
({\tt Connections}). The Connect class is meant to be subclassed when used by
a real client-library. See Section~\ref{tk:manage-conns}.

\subsection{The DAS and DDS Objects}
\label{das-dds-objects}

The dataset attribute structure DAS~\externalref{api:das} and dataset
descriptor structure DDS~\externalref{api:dds} objects are used to store
information about the data set's variables. These can be thought of a
metadata objects, but it is worth noting that we shy away from the term {\em
  metadata\/} because often this is {\em data\/} to many users.

Neither the DAS nor the DDS contains science data---the DAS contains
attribute information from the data set while the DDS contains structural
information about the data set and variables in the data set.

To build both the DAS and DDS, the server must open the data set and use
whatever appropriate API calls to extract the necessary information about the
data set and its constituent variables. The DAS and DDS server filter
programs do this and then transmit the resulting \Cpp\ object to the client.

On the client-side, the surrogate library uses information in the DAS and DDS
to satisfy calls which are issued by the user program requesting information
about variables, their type, shape, and attributes.  The two existing
Client-libraries request both of these objects when they first contact the
remote data set. The DAS and DDS objects are then stored as part of a {\em
  virtual connection\/} to that data set and can be used repeatedly by the
client-library without retransmission.

The {\tt DAS} and {\tt DDS} objects have both an internal and an external
representation. Their internal representation is as a \Cpp\ object while
their external representation is as text. The object is sent from the server
to the client-library using this text representation. Each of the two classes
contains a parser which can read the text representation and (re)create the
object's internal representation. In addition, it is possible to write the
text representation for either object (using a text editor) and then use the
parser to create the internal, \Cpp, object. Furthermore, the text
representation is a type of persistence and can be used to build a flexible
object caching mechanism.

One possible use for this caching mechanism is to store the DAS and DDS for a
dataset and use the stored versions in place of opening the data set and
reading information about it and its contents. This can be tricky to
implement, but for large data sets with many variables can also result in a
significant performance improvement. The caching mechanism may also be used
on the server-side to store extra information about the data
set---information that is not present in the data set proper, but which the
data provider would like included when people access the data set over the
network. In this scenario, the data server first integrates the data set
file(s) using the API and builds the DAS and DDS. Once an initial version of
the {\tt DAS} and {\tt DDS} are resident in memory, the parser is used to
read an external text file which contains additions to, or corrections of,
the information extracted from the data file. In the \Ddd\ paper the extra
information is referred to as `ancillary information'.

\subsubsection{The DAS Object}
\label{toolkits:das}

The {\tt DAS} contains attribute information that is generally {\em not\/}
used by software when processing the variables; this object is specifically
designed to hold all the information in a data set that has no where else to
go. Each variable can have an unlimited number of attributes. There are also
`global' attributes that apply to the dataset as a whole. Each attribute is a
tuple of three elements: the attribute name, type and value. The supported
types for an attribute are: Byte, Int32, Float64, String and URL. These types
have the same range of values as the corresponding types in the \dap.

Attributes are any qualities that are not listed in the \dap\ in the section
discussing the DDS~\externalref{api:dds}.  Thus, the type, shape, and units
of a variable are {\em not\/} attributes; they are characteristics of the
variable and are part of the DDS (see Section~\ref{toolkits:dds}). However,
users make store any other information in as attributes (actually, there is
no reason that type, etc. cannot be stored as an attribute; however, it must
be in the DDS regardless) including paragraphs of text, vectors of integer
and floating point values, etc.

Many APIs support the concept of information in the data set that is not
computable and provide function calls to access that information. The DAS
object provides a place for all tat information. The data server can
interrogate the data set and build the DAS and the client-library can use the
DAS object to satisfy many of the API calls requesting information about the
variables in a data set.

Figure~\ref{fig:das-class-diagram} is a Booch~\cite{booch:oo-analysis}
diagram of the {\tt DAS} class and its related classes. The figure shows that
the class {\tt DAS} contains a parser and related utilities as well as an
instance of the class {\tt DASVHMap}. The parser utilities are can read the
textual representation of a {\tt DAS} object and create the \Cpp\ (or binary)
representation. The parser and scanner are built using GNU's {\tt bison} and
{\tt flex} respectively; source for them can be found in the dap source file
{\tt das.y} and {\tt das.lex}.

The {\tt DASVHMap} object deserves more explanation. It was created using
{\tt g++}'s {\tt genclass} utility; several other classes in the dap library
were also created that way. {\tt Genclass} is a tool that creates a
parameterized class from a template given certain parameters. The parameters
are class names and header file names. This utility predates the current
(\today) Draft ANSI \Cpp\ standard for template classes. It is used by DODS
because the project started using {\tt g++} before its template mechanism was
implemented on the DODS development platforms.

The {\tt DASVHMap} class uses the {\tt DASMap} class to map objects of type
{\tt String} onto objects of type {\tt AttrTable}. Each variable in the data
set has a name and that name is used as an index into the {\tt DASVHMap} to
get the matching {\tt AttrTable}. In addition to an attribute table for each
variable, the {\tt DASVHMap} is used to store the data set's global
attributes by using the string ``Global''.

Each {\tt AttrTable} object contains a doubly-linked list of attribute
triples, named {\tt entry}. Each entry contains a Name, Type and vector of
values.

The {\tt DAS} object contains member functions to add or retrieve {\tt
  AttrTable}s as well as individual attributes based on a variable name. In
addition, both the {\tt DASVHMap} object and the {\tt AttrTable} object may
be traversed using a {\tt Pix}\footnote{A {\tt Pix} is a libg++ psuedoindex
  object. See the libg++ documentation for more information.}.

\begin{figure}
\centerline{\psfig{figure=das-class-diagram.ps}}
\caption{A Booch diagram of the {\tt DAS} class hierarchy.} 
\label{fig:das-class-diagram} 
\end{figure}

\subsubsection{The DDS Object}
\label{toolkits:dds}

The DDS is used to store information about the organization of the data set
and its variables. It contains information about the type, shape, units, ...
of variables. The DDS is `lexically scoped' so that two {\tt Structure}
variables may have components with the same names; each component will be
referred to using the {\tt Structure} name and the dot operator (see the \Dap
section on operators\externalref{api:operators}).

While the DDS is similar to the DAS in that it is used to store
information about the data set, it is used quite differently by both the
client-library and server components. The DAS is a stand alone object and is
used solely for the purpose of storing attributes of variables and the
dataset. The DDS, however, is used to store type information about the
data set's variables using instances of those variables. Because the variable
objects have methods that can be used to read values from a data set or
transfer the variable's value over the network, the DDS object is used by
both the DDS filter program and the data filter program.

Figure~\ref{fig:dds-class-diagram} shows the class hierarchy for the {\tt
  DDS} object. Rather than use the Gnu {\tt VHMap} template object, the {\tt
  DDS} class uses a singly-linked list to store multiple instances of
variables in the data set. Each variable is stored in an instance of one of
{\tt BaseType}'s descendent classes. Each DDS contains a {\tt String} object
used to store the name of the dataset and a singly-linked list of constraint
expression clauses.

The {\tt DDS} class contains member functions that can be used to set or
access any of its members. In addition, the class contains a parser and
scanner that can be used to internalize the textual representation of the DDS
and a parser and scanner which can be used to parse a constraint expression.
This parser stores information about the constraint expression in the DDS and
its contained variable objects which can be used later during constraint
expression evaluation.

\begin{figure}
\centerline{\psfig{figure=dds-class-diagram.ps}}
\caption{A diagram of the {\tt DDS} class hierarchy.}
\label{fig:dds-class-diagram}
\end{figure}

\subsection{The Type Hierarchy}

The {\em Type Hierarchy\/} is the implementation of the data model described
in ``\DAP''. It refers to the set of classes that form the hierarchy used to
build objects that contain data. These classes comprise the data model for
DODS. It contains simple programming language types such as integer and
floating point values as well as constructor type like structure and
sequence. The \dap\ paper describes the semantics of these data types.

Unlike the other classes in the DODS toolkit libraries, these are abstract
classes---in order to be used by a program, you must subclass the hierarchy
and create concrete classes to instantiate.
Figures~\ref{fig:cardinal-types}~to~\ref{fig:ctor-types} contain diagrams
which show how this class hierarchy is organized. Note that the entire
hierarchy is split into three figures for clarity's sake; in the toolkit
library it is one hierarchy. 

\subsubsection{Common Ancestor: BaseType}

The root of the type hierarchy is the abstract class {\tt BaseType}. This
class, because it is abstract, is never instantiated itself. It contains
common member functions used by all the other classes. {\tt BaseType} is used
as the base class for all of the different types of variables. For scalar
variables such as {\tt Int32}, only the abstract virtual functions in {\tt
BaseType} need to be added to complete the class definition. However,
constructor types like {\tt Structure} require more. Each constructor data
type contains one or more instances of {\tt BaseType}. These {\em contained
variables\/} are the fundamental types used by the type constructors to build
new types.

\subsubsection{Cardinal Types: Byte, Int32, Float64, Str, Url}

The cardinal data types (see Figure~\ref{fig:cardinal-types}) consist of {\tt
  Byte}, {\tt Int32}, {\tt Float64}, {\tt Str} and {\tt Url}. These data
types match very closely the corresponding types in C or Fortran. Note that
each of these types uses either the C or \Cpp\ representation to hold the
value of the object. The class {\tt BaseType} adds some additional behavior
such as network transmission.

These five abstract classes are direct descendents of {\tt BaseType} which
contain only their definitions for {\tt BaseType}'s abstract member functions
and a single constructor.

\begin{description}

\item [Byte] Variables which store bytes. Equivalent to unsigned
char on most UNIX workstations.

\item [Int32] Variables which store integer values as 32-bit
twos-complement  signed integers. Equivalent to {\tt int} on
most 32-bit UNIX workstations.

\item [Float64] Variables which store floating point data.
Defined as the  IEEE 64-bit floating point data type, equivalent
to a {\tt double} in ANSI  C.

\item [Str] Variables which store string information. A string in the DAP is
{\em not\/} a sequence of characters referenced using a pointer (as it is in
C). In the DAP a string represented using a \Cpp\ object. 

\item [Url] Variables which store references to network
resources. The  semantics of this type is defined in \URL.

\end{description}

\subsubsection{Vector Types: Array, List}

The vector data types (see Figure~\ref{fig:vector-types}) are {\tt Array} and
{\tt List}.  Note that an array is a vector with a list of dimensions. When
creating an array, the dimension sizes (and optionally their names) must be
set.  Regardless of the shape of the array, it is always stored as a vector.
In order to access the element of a multidimensional array it is necessary to
calculate the offset for a given element.

\begin{description}
  
\item[Array] Instances of {\tt Array} have different semantics than arrays in
  C.  They are multi-dimensional data structures which contain {\em N\/}
  instances of some data type. Each instance in the array can be referred to
  by an integer index between {\em 0 and N-1\/}. Multidimensional arrays are
  declared using C's syntax of a sequence of bracketed integer values: {\tt
    Int32 a[10][20]} declares an array of 10 arrays of 20 integers. However,
  unlike C arrays, the {\tt Array} class supports named dimensions. In the
  preceding example, the array could have been declared: {\tt Int32 a[row =
    10][col = 20]} where {\tt row} and {\tt col} are the names of the first
  and second dimension, respectively. It is possible to determine the name
  for the $i^{th}$ dimension using member functions supported by this class.
  
  {\tt Array}, like all of the constructor types, contains a reference to a
  component variable. In the preceding example, the instance of {\tt Array}
  would contain information about the dimensions of the array (10 by 20), but
  not the type of the elements (Int32). The element type information is
  stored in the component variable which the instance of {\tt Array}
  references.
  
\item[List] A List is an ordered collection of elements of unknown length
  (whereas an array's size is always known in advance). Each element of the
  List must be of the same type. When a List type is declared no size
  information is supplied. However, to transmit a {\tt List} object the
  length of that list must be known.  Thus, internally the number of elements
  in the current value {\em is\/} stored.

\end{description}

\subsubsection{Constructor Types: Structure, Sequence, Function, Grid}

The constructor data types are used to build new types as aggregates of other
types, including constructor types. Note that List and Array are constructor
types, also. Structure, Sequence and Function all contain a list (or lists
for Function) of {\tt BaseType} objects. However, they have different
semantics; a Structure is a simple aggregate; nothing other than aggregation
is implied, while Sequence and Function define templates for relational
objects. A Grid combines several {\tt Array} objects so that nonlinear values
may be applied to the indices of an array.

\begin{description}
  
\item[Structure] A Structure is an ordered collection of variables that
  conveys no relational information other than grouping. The variables that
  are members of a Structure may be of different types. In addition to the
  (possible) benefit of added organization, Structure may be used to supply
  information to the system that may be useful in optimizing the access or
  translation operations.

\item[Sequence]
  
  A Sequence {\em S\/} is similar to a Structure in that it consists of an
  ordered collection of {\em N\/} variables which may be of different types.
  However, the sequence {\em S\/} actually describes a template for an
  ordered collection of instances {\em I\/}. Each element {\em i\/} of {\em
    I\/} is an instance {\em s\/} of {\em S\/}. An entire sequence {\em S\/}
  may be described thusly:

\begin{displaymath} 
\begin{array}{ccc}  
        s_{0 0} & \cdots & s_{0 n} \\  
        \vdots & \ddots & \vdots \\  
        s_{i 0} & \cdots & s_{i n}
\end{array} 
\end{displaymath}

\noindent Every instance {\em s\/} of Sequence {\em S\/} has the same number,
order, and class of variables. A Sequence implies that each of the {\em N\/}
variables is related to each other in some logical way. A Sequence is
different from a Structure because its constituent variables have several
instances while a Structure's values have only one set of variables. Because
a Sequence has several values for each of its variables it has an implied
{\em state}, in addition to those values.

\begin{table} 
\caption{Table of relational data.}
\label{toolkit:seq2} 
\begin{center} 
\begin{tabular}{|| c | c | c ||} \hline 
{\sc Name} & {\sc Age} & {\sc Weight} \\ \hline 
James & 32 & 165 \\ 
Charlie & 6 & 65 \\ 
Bob & 10 & 80 \\ \hline
\end{tabular} 
\end{center} 
\end{table}

For example given the the information in Table~\ref{toolkit:seq2}, $s_{0}$ is
{\tt James}, {\tt 32}, {\tt 165}, $s_{1}$ is {\tt Charlie}, {\tt 7}, {\tt
  65.4}, \ldots. The data in the table might have the following Sequence:

\begin{code}{cb}
Sequence {
    Str name;
    Int32 age;
    Float64 weight;
} people;
\end{code}

\item[Functions] A Function is a special case of a Sequence that separates
  its variables into two categories: {\em independent\/} and {\em
    dependent\/} corresponding to the independent and dependent variables in
  an experiment. Variables in a Function may be of differing classes. The
  mathematical description of this functional relation is not specified.
  Instead the Function type is used to indicate that one of the two sets
  constitute the independent variables and the other the dependent variables.
  
\item[Grid] A Grid is an association of an {\em N\/} dimensional Array with
  {\em N\/} named vectors, each of which has the same number of elements as
  the corresponding dimension of the Array. Each vector is used to map
  indices of one of the Array's dimensions to a set of values which are
  normally non-integral (e.g., floating point values). The {\em N\/} (map)
  vectors may be members of different classes. 

\end{description}

\begin{figure}
\centerline{\psfig{figure=cardinal-types-diagram.ps}}
\caption{A diagram of the cardinal data type hierarchy.}
\label{fig:cardinal-types}
\end{figure}

\begin{figure}
\centerline{\psfig{figure=vector-types-diagram.ps}}
\caption{A diagram of the vector data type hierarchy.}
\label{fig:vector-types}
\end{figure}

\begin{figure}
\centerline{\psfig{figure=ctor-types-diagram.ps}}
\caption{A diagram of the constructor data type hierarchy.}
\label{fig:ctor-types}
\end{figure}

\section{Managing Connections Using Netio} 
\label{tk:manage-conns}

The toolkit contains two classes to help managing connections within
surrogate libraries. One class, {\tt Connect} stores information that is used
by the surrogate library when a connection is established. It also provides
some rudimentary support for providing local access without using a DODS data
server.

The class {\tt Connections} manages instances of the {\tt Connect} class. It
provides a mechanism for the surrogate libraries to pass back to user
programs the type of object ({\tt int}, opaque pointer, \ldots) they expect
and then to use one of those objects to access the correct instance of {\tt
  Connect}. {\tt Connections} is a template class. A client-library uses the
{\tt Connections} class to make an array of some sub-class of {\tt Connect}.

\subsection{Connect}

{\tt Connect} manages one connection to either a remote data set, via a DODS
data server, or a local access. For each data set or file that the user
program opens, there must be exactly one instance of {\tt Connect}. Because
information needed for local access is stored in an instance of {\tt
  Connect}, the class is normally sub classed for each supported
API\footnote{As of \today, subclasses {\tt JGConnect} and {\tt NCConnect}
  exist for JGOFS and NetCDF, respectively.}. In the API-specific child of
{\rm Connect\/} additional members can be added to store state information
that the client-library needs to maintain the virtual connection (See
Section~\ref{tk:virtual-connections} for more information on making virtual
connections).

The {\tt Connect} class contains member function which are used to get the
data set's DAS, DDS and data. The Instance of {\tt Connect} (or a subclass of
{\tt Connect}) stores the URL as the user provides it, usually by being
created inside the API's open function call. It then appends information to
the user-supplied URL to make the URL which will return the various objects
from the data set's DODS data server (See
``\URL''~\externalref{urls:user-vs-generated} for a discussion on the
difference between user URLs and those actually sent to a data server; see
Section~\ref{tk:using-constraints} for more information on using constraints
to access data).

The {\tt Connect} class also contains information about all of the
constraints used to access data so far. This information includes all of the
expressions as well as all of the resulting DDSs and values they returned.
This is a powerful form of caching that can be tricky to use. If the user
program, via the API, has previously requested the exact same data (i.e., the
constraint expressions match exactly for any two requests) then you can use
the data already stored by the {\tt Connect} object rather than send for the
data a second time. Within the {\tt Connect} class there are member functions
that can be used to search the previously submitted constraint expressions.

If you are re-implementing an API and must support function calls that modify
how data is accessed (e.g., by creating array slices or by choosing one of a
set of variables), then you will need to translate those requests into a DODS
constraint expression (See ``\DAP''~\externalref{api:constraints} for more
information on constraint expressions, their syntax and semantics). You would
then pass these synthesized constraint expressions to the {\tt
  Connect::request\_data} member function.

\begin{figure}
\centerline{\psfig{figure=connect-diagram.ps}}
\caption{The Connect class; used to store information about a virtual
connection to a data set.}
\label{fig:connect-diagram}
\end{figure}

\subsection{Connections}

The class {\tt Connections} is used to manage a set of instance of the class
{\tt Connect} by providing a means to map an index or opaque pointer to an
instance of {\tt Connect}. 

When a new instance of {\tt Connect} (or a descendent of {\tt Connect}; See
Section~\ref{tk:subclass-netio}) is created, it is added to the {\tt
  Connections} object using the {\tt add\_connect} member function. {\tt
  add\_connect} returns an integer that can be used to access that instance
of Connect at any time. Similarly, when an instance of {\tt Connect} is to be
deleted, the object can be referred to via the {\tt Connections} object and
this index.

\section{Sub-classing the DAP and Net I/O Libraries}
\label{tk:subclassing}

In order to be used by real client-libraries and data servers, many of the
classes in the DODS toolkit must be subclassed. For example, the {\em Type
Hierarchy\/} classes, which represent the data types in the DODS data model,
are all abstract classes. In order to use them in a program they must be
subclassed. Furthermore, the {\tt Connect} class (in {\tt libnetio} must be
subclassed if it to hold additional information about the connection. In this
section sub-classing is discussed in general and some specific caveats are
discussed. 

\subsection{Sub-classing the Type Hierarchy}

In order to link a program with the DAP library, eleven of the DAP's abstract
classes must be subclassed and those subclasses must ensure that all of the
member functions of those classes have valid definitions. This is necessary
because of \Cpp's rules governing abstract classes\cite{stroustrup:cpp}. The
classes are {\tt Byte}, {\tt Int32}, {\tt Float64}, {\tt Str}, {\tt Url},
{\tt Array}, {\tt List}, {\tt Structure}, {\tt Sequence}, {\tt Function} and
{\tt Grid}. The next three sections cover sub-classing the cardinal, vector
and constructor classes, respectively. In addition, a sample set of classes
(called the {\tt Test} classes because they are used by the DAP tests) is
included with the DAP distribution. You can read the source code for those
classes to find out how they were created.

\subsubsection{Sub-classing the Cardinal Types}

The DODS Cardinal types all inherit directly from the class {\tt BaseType}
(see Figure~\ref{fig:cardinal-types}. Each of the cardinal classes must
define two member functions {\tt ptr\_duplicate} and {\tt read}.

The {\tt ptr\_duplicate} member function returns a pointer to a new instance
of the type of object from which it was invoked. This member function exists
so that objects that are referenced through pointers to {\tt BaseType} can
correctly copy themselves. If you were to use the operator {\tt new} to copy
and object referenced through {\tt BaseType}, you would get a {\tt BaseType},
not a new instance of the type of the referenced object! Note that {\tt
  ptr\_duplicate} is a virtual function so an object which is a descendent of
{\tt BaseType} will get the most specific definition of the member function.

The read member function is much more complicated to write that the 
{\tt ptr\_duplication} member function. Note however, that the {\tt read}
member 
function is only used on the server side of the system (i.e., it is only used
in construction of the data server, and not the client-library) and thus can
be implemented as a with a null function body if all you are building is a
client-library. read takes a single argument, the data set name, and must
read from that data set the values specified by the current constraint
expression. The class {\tt BaseType} contains a number of member functions
that facilitate this (See Section~\ref{tk:constraints} for more information).

Once the name of the variable to be read is determined, the value must be
read and stored in the object's internal buffer. To do the later, use the
member function {\tt val2buf}. Note that the toolkit does not provide any
support for reading the data from the data set; that is out of the scope of
the DAP. IF the data server must work with files written using an API (such
as \netcdf\ or \hdf) then use what ever function calls the API provides
to read a variable given its name.

Finally each of the cardinal subclasses must define a constructor and a
virtual destructor. The constructor must take a reference to a single {\tt
  String} (that's the GNU String class, not the DODS {\tt Str} class) as its
only argument.

The {\tt read} member function is used by the constraint expression evaluator
to extract data from a data set during evaluation of the constraint
expression.  This is particularly important to remember because the {\tt
  read} member function for a cardinal will be called when reading an
aggregate type such as Structure.

\subsubsection{Sub-classing the Vector Types}

The vector types (classes {\tt Array} and {\tt List}) require the same
abstract member functions be defined as the cardinal types. The definition
for {\tt ptr\_duplicate} is the same for vector as for cardinal types.
However, the {\tt read} member function is more complicated because arrays of
values are represented in two ways in DODS, depending on the type of
variable. Arrays are stored as C~\cite{kr:c} would store them for the
cardinal types {\tt Byte}, {\tt Int32} and {\tt Float64} and are stored as
arrays of \Cpp\ objects for all other types (with the exception of arrays
themselves, but more on that latter).

When reading a array of Byte values, the {\tt val2buf} member function should
be passed a pointer to values stored in a contiguous piece of memory. For
example, it read is called to read a variable {\tt byte-array}, then it
should determine how much memory to allocate to hold that much information,
use {\tt new} to allocate the memory, use the data set's API calls to read
byte-array into the newly allocated memory and then pass that memory to {\tt
  val2buf}q. This same procedure can be followed for all the cardinal types.

However, when reading an array of {\tt Structures}, for example, the values
must be stored in the DODS {\tt Array} object one at a time using the {\tt
  Array} member function {\em set\_vec}. In this case a C-like chunk of
memory is not allocated.

Arrays in DODS are unlike arrays in C in that an array object may have more
than one dimension. In terms of the way a value is stored, however, and Array
is a single dimensional object. When an Array is declared as having two or
more dimensions, those are mapped onto a single vector. To access the element
$A_ij$ of array {\em A\/}, you must know the size of the first dimension,
{\em I\/}, and use the expression {\em i * I + j\/} to compute the offset
into the vector.

Since the class {\tt List} is a single dimension array without {\em
  declared\/} size (the size of each value of the List object is stored in
the object) the rules for {\tt Array}'s read member function apply.

\subsubsection{Sub-classing the Constructor Types}

In order to subclass the constructor classes you must supply the {\tt
  ptr\_duplicate} and (as of \today) the read member functions. The {\tt
  ptr\_duplicate} member function has the same semantics for constructor
types as it does for vector and cardinal types and it can be implemented in a
similar fashion. The {\tt read} member function simply iterates over the
contained variables calling their {\tt read} member functions. In the future,
this definition will move into the supplied classes (i.e., read will no
longer be a abstract member function).

The two constructor types Sequence and Function are different from all the
other types in the DAP in that they have {\em state}. That is, the value of a
sequence depends on how many values have been read previously.  This is very
different from an array where the $i^{th}$ element has the same value
regardless of what has happened before. When you write implementations for
{\tt read} in the {\tt Sequence} and {\tt Function} classes, you must be sure
to write those member functions so that they can be called repeatedly and
that each call to read return the next value of the Sequence or Function.
This is true because the constraint expression evaluator must be able to
apply certain constraints to values of individual sequence elements and is
actually implemented in the DDS class by first calling the read member
function, evaluating the constraint expression based on the values and, if
they constraint expression is satisfied, calling the serialize member
function.  See the member function {\tt DDS::send} (See
section~\ref{tk:ce-evaluation} for more information on evaluation of
constraint expressions).

If, for some reason, it is not possible to write {\tt read} so that it gets
called once for each sequence value (sometimes called and instance of the
sequence, but that's confusing when we are also talking about instances of
objects \ldots), then you must re implement {\tt DDS::send} so that its
functions are performed. For example, your could implement {\tt
  Sequence::read} so that the entire sequence is read in and overload {\tt
  Sequence::serialize} and {\tt Sequence::deserialize} so that the next set
of values are sent/received. You would then build a send that called the {\tt
  Sequence::read} member function once and extracted each successive value,
evaluated the constraint expression using on that value and used the result
of that evaluation to determine whether to send the value or not.

\subsection{Sub-classing the Network I/O Classes}
\label{tk:subclass-netio}

There are two network I/O classes: {\tt Connect} and {\tt Connections}. Note
that the network I/O library is used only by the client library. If you are
building a data server only, you do not need to use {\tt libnetio}.

The {\tt Connections} class (a template class) is not generally subclassed.

To be effective, the {\tt Connect} class must be subclassed for each API. For
each API that is reimplemented as a DODS client library, some additional
information must be stored with each virtual connection that is made to a
data set. This information can be stored in the subclassed {\tt Connect}.

It is important to note that the {\tt Connect} class contains and instance of
the data set's complete DDS plus {\em N\/} more instances of the DDS
constrained by the constraint expressions used to read information from the
data set. However, to subclass {\tt Connect} you do not need to use any of
this information. What is done when sub-classing Connect is to add additional
fields to the new class and to ensure that the class constructor takes all
the arguments required by the API's open function call. For example, NetCDF's
{\tt ncopen} function call takes two arguments, a string which names the data
file to open and a mode flag similar to UNIX's open system call. The
constructor for {\tt Connect} takes only one argument, a data set name (as a
\Cpp\ {\tt String}). In the reimplementation of NetCDF for DODS, {\tt
  Connect} is subclassed by {\tt NCConnect}. The constructor for {\tt
  NCConnect} takes two arguments, a {\tt String} and a file access mode flag.
The new version of {\tt ncopen} calls {\tt NCConnect}'s constructor, when it
makes a new instance of the class {\tt NCConnect}, using the arguments passed
to ncopen by the user program. the {\tt NCConnect} constructor in turn calls
{\tt Connect}'s constructor to complete the object creation process.

I addition to adding a new constructor, the subclass of {\tt Connect} will
typically require a new destructor which frees any storage allocated by the
new class. This storage might be used by the new client-library to keep state
information. The API's close function call is the best place to delete the
instance of {\tt Connect}'s subclass.

\section{Using the Toolkit to Build Client-Libraries and Data Servers}
\label{tk:using}

This section describes how to use the toolkit software to build new client
libraries and data servers. Both tasks are potentially demanding because
knowledge of the API for which the software is being built is paramount. To
create a client-library that can {\em replace\/} the original API
implementation at {\em link time\/} means that the client-library must
present exactly the same interface as the original library. This includes,
to the extent that they are widely used, any undocumented features of the
original implementation that manifest themselves as symbols that require
link-time resolution. Thus building a client-library requires great
understanding of the existing implementation and current use of the target
API. 

To build a good data server for files or data sets encoded using an API it is
important to understand the data model(s) the API supports and how they
relate to the DODS data models. Each of the various data types that the API
supports must be translated into a DODS data type (i.e., one of the DODS
classes that descend from {\tt BaseType}). However, there is often not a
one-to-one match between the API's types and the DODS types. Thus, the data
server author must decide how to best translate the API's types into DODS's
so as to preserve as much of the data set author's intent. This is
exacerbated by the use of various conventions that (implicitly) bind several
variables together with a data set. When this pattern shows up (as it does
with NetCDF) you must decide whether to lump all variables together that {\em
  appear\/} to use the convention (and thus falsely group some variables) or
to group only those which actually are explicitly grouped using whatever the
API provides. If you choose the later then any data sets which follow the
convention will lose information. When building the data server it is
important to keep such tradeoffs in mind.

In the following subsections the specifics of building a data server and a
client-library are discussed. The existing NetCDF server and client library
are used as examples. At this time they are the most complete software for
the supported APIs and many APIs are very similar in their overall
organization. The source code used for these examples can be found in {\em
  \$(DODS\_ROOT)/src/http/nc-dods/\/}. If you are building a data server or
client-library for a relational system, much of the NetCDF example will still
be relevant. In addition, the software in {\em
  \$(DODS\_ROOT)/src/http/jg-dods/\/} contains both a data server and
client-library for the \JGOFS\ relational data system.

\subsection{Data Servers}

The data server consists of three filter programs written in \Cpp\ and a
dispatch CGI which selects which of the filters to run based on the URL. When
the client-library receives a URL via its open call (which will not actually
be called {\em open\/} but performs the function of opening a file or data
set; e.g., NetCDF's {\tt ncopen}) it passes that URL to the class Connect
member functions {\tt request\_das} and {\tt request\_dds}. These member
functions append the extension {\tt .das} and {\tt .dds} (respectively) on to
the URL and retrieve the resulting URL, which is the DAS and DDS for the data
set (for more information about URLs, and the difference between URLs
supplied by the user and those used by the software, see ``\URL'').

\begin{figure}
\begin{code}{cb}

#!\bin\sh

cgi_dir=`pwd`
cgi_dir2=`echo $cgi_dir | sed 's@\\(.*\\)/.*@\\1@'`

extension=`echo "$PATH_INFO" | sed 's/.*\.\(.*\)/\1/'`

server_pgm="${cgi_dir2}${SCRIPT_NAME}_${extension}"

file_name=`echo "$PATH_TRANSLATED" | sed "s/\(.*\)\.${extension}/\\1/"`

query=`echo "$QUERY_STRING" | sed 's/+/ /g'` 

$server_pgm $file_name $query
$

\end{code}
\caption{The DODS-NetCDF data server dispatch CGI (comments elided).}
\label{fig:cgi}
\end{figure}

\subsubsection{The Dispatch CGI}
\label{tk:cgi}

A sample dispatch CGI is shown in Example~\ref{fig:cgi}. It is a Bourne Shell
script.  The dispatch CGI first puts the current working directory into {\tt
  cgi\_dir2} using pwd and the URL `file type' extension into {\tt
  extension}. Note that the extension is stripped from {\tt cgi\_dir2} using
{\tt sed}. The variable {\tt server\_pgm} contains the name of the data
server filter program to run. It is the catenation of the CGI script name,
and underscore and the URL's extension ({\tt SCRIPT\_NAME}, {\tt \_} and {\tt
  extension}). For example, if the dispatch CGI received the URL {\em
  http://machine/cgi\_bin/nc/data.das\/} the variable {\tt SCRIPT\_NAME}
would contain {\em nc\/} and the variable {\tt extension} would contain {\em
  das\/}; thus {\tt server\_pgm} is assigned {\em nc\_das\/}.  The last line
of the script runs the program named by {\tt server\_pgm} with two arguments.

The {\tt file\_name} and {\tt query} arguments to the filter program are
built by extracting the path and query string from the URL. For example the
URL {\em http://machine/cgi\_bin/nc/data/exp1.nc?var1.das\/} would result in
{\em /data/exp1.nc\/} in {\tt file\_name} and {\em var1\/} in {\tt query}.

\subsubsection{The DAS and DDS filter programs}

The source code for the DAS filter program distributed with the NetCDF server
software is shown in Listing~\ref{fig:das-filter}. The DAS and DDS filters
are both very similar, so only the DAS filter will be discussed here. If
there are important differences between the two, they will be pointed out.

Each of the filter programs must create a \MIME\ document to hold its return
value. The DAS and DDS filters return a text MIME document; they set up the
\MIME\ headers (Content-Length, \ldots) using the utility function {\tt
  set\_mime\_text}. 

The DAS filter program takes only one argument, the data set or file name.
This is passed in to the program in {\tt argv[1]}; the filter program's {\tt
  main()} function calls {\tt read\_variables()} which does all the actual
work of building the DAS structure. The DAS class contains a number of member
functions that enable new variables to be added to the DAS ({\tt
  DAS::add\_table}) as well as member functions to retrieve tables ({\tt
  DAS::get\_table}) and traverse the collection of variables and their
attribute tables. The class AttrTable contains the member functions necessary
to build the attribute table for a single variable ({\tt
  AttrTable::append\_attr}) and read name, type and value of an attribute.

Once the data set has been read and the attribute table built, the DAS
ancillary file is loaded. The example filter looks for a file with the same
root name as the data set and an extension equal to {\tt .das}. If such a
file exists, the it is read in using the DAS member function {\tt
  DAS::parse} and the information it contains is merged with the DAS built
from the data set. 

Finally the DAS member function {\tt print} is used to send the textual
representation of the DAS to the client. Note that the dispatch CGI's
standard input and output are a socket connected to the client process (since
the CGI is running in a child process of {\tt httpd}) because the filter is
invoked by the CGI script, its output also goes to the client. Thus the
DAS::print member function actually writes it output to standard output, but
that happens to be connected to the client process. The {\tt print}
member function contains several options which permit the output to be sent
somewhere other than standard output.

Note that the example filter does not use any caching. It is possible to
build a more sophisticated filter program that saves the generated DAS to a
text file and then uses that (without first interrogating the data set, thus
saving on access) file. It is also possible to write a DAS by hand and {\em
  always\/} use that if the data set does not contain any of the type of
information that the DAS has.

\begin{figure}
\begin{code}{cb}
extern bool read_variables(DAS &das, char *filename);

int 
main(int argc, char * argv[])
{
    set_mime_text();

    if(argc < 2) {
        ErrMsgT("Usage: nc_das <data set>");
        exit(1);
    }

    // read the netCDF file attributes from dataset

    DAS das;

    if(!read_variables(das, argv[1])){
        exit(1);
    }

    // read the ancillary DAS if it exists.

    String ExtFile = (String)argv[1] + ".das";
    FILE *in = fopen(ExtFile, "r");
 
    if (in) {
        int status = das.parse(in);

        if(!status) {
            String Msgt = (String)"Parse error in external file " + ExtFile;
            ErrMsgT(Msgt);
        }
        fclose(in);
    }

    // send the DAS object (as text)

    das.print();

    exit(0);                    // exit success
}
\end{code}
\caption{The DAS filter program.}
\label{fig:das-filter}
\end{figure}

\subsubsection{The Data filter}

The data filter program is structured similarly to both the DAS and DDS
filters except that it returns a binary MIME document rather than text and
that it takes two arguments instead of just one. In addition to the data set
or file name (argument 1) it also takes the DODS constraint expression
(argument 2, which was encoded in the URLs {\em query\/}).

The filter first sets up the binary MIME document headers with the
convenience function {\tt set\_mime\_binary}. Once done the data filter must
get a copy of the data set's DDS. Currently no caching is used by the NetCDF
filter so the data filter must do all of the steps that the DDS filter does:
scan the data set and build the DDS, then look for an ancillary DDS file and
merge it with the DDS just built. It would improve performance considerably
to cache the DDS and not rebuild it via the data set.

Once this is done the data filter must use the data set name and constraint
expression (query) to find the correct file and select information from
within that file. The class DDS supports a new member function called {\tt
  send} that simplifies this process by evaluating the constraint expression
(CE) and creating a new DDS containing only those variables described by the
CE. That DDS is used to read the variable's values, complete any CE
evaluation that remains and send the data if appropriate (see {\tt DDS:send}
for more information).

If for one reason or another you cannot use the send member function of DDS,
then you must ensure that the the {\em read\/}, {\em CE evaluation\/} and the
{\em serialize\/} operations are all carried out in the correct order.
Furthermore, you must ensure that the return value of the data filter is a
binary MIME document with a text prefix (currently, DODS does not use the
multi-part MIME standard); that is a regular binary MIME document with a
section at the start that is text. This text is the DDS generated after
evaluating the projection clauses of the CE. The text part is separated from
the data by the keyword ``Data:'' at the start of the line (and out the scope
of the text DDS so it is possible to have the text {\em Data:\/} in the DDS).

\subsection{Client Libraries}

The goal of building a client-library is to provide a drop-in replacement for
an existing API so that user programs written for that API can switch to the
DODS version and access DODS data. The user programs should not require any
modification to change over to the DODS client-library version of the
API. However, the API will clearly need substantial changes to its current
implementation. In the following subsections several different problem areas
are discussed. 

\subsubsection{Maintaining Virtual Connections}
\label{tk:virtual-connections}

The DODS \Dap\ is a connection-less protocol yet most data access APIs
maintain state information about the files or data sets that the user program
currently has open. Many client-server systems would implement the
network-capable versions of these APIs by having the server maintain state
information for a number of connections. However, since the DAP is a
connections protocol, its servers cannot maintain any state information. To
simulate an API's connection, the client-library for that API must create a
{\em virtual connection\/} using information about the data set it has read
from the server.

The DAP data set attribute structure (DAS) and data set descriptor structure
(DDS) are objects specifically designed to simplify this task. These objects
contain information about the data set's logical organization, variable names
and types and any additional information the data set author thought was
important. 

Most APIs used for storage of science data have one function call used to
open a data set or file (e.g., NetCDF uses the function {\tt ncopen}). For
the remainder of this document this will be referred to as the API's open
call. When building a replacement for this API, the open call must be
completely recoded (see \DDD\ for a description of the tasks the new function
must complete) so that the DAS and DDS objects are requested from the data
set's server and stored locally. There are several ways to store this
information: the class Connect can be subclassed to that additional
information can be stored with each Connect object, the DAS and DDS objects
themselves can be used to store the information, or part of the original
implementation of the API may be used. Of these three, the first option offers
the best balance between simplicity and efficiency. Using the DAS and DDS
objects may not efficient at all since some information may be obtained only
by iterating through all the objects and using large pieces of the original
API may mean using software intended for a very different type of system---it
may not store much of the information that the DAS and DDS objects store,
thus increasing accesses to the server.

Figures~\ref{fig:open}~and~\ref{fig:NCConnect} show sample versions of the
recoded open call in the DODS-NetCDF software and the subclass of Connect
used to store information extracted from the DAS and DDS objects.

The recoded open call of NetCDF has exactly the same type as the original
implementation; it takes the same number and type of arguments and returns
the same type. The first operation performed by the new open call is to
create a {\tt Connect} object (actually an {\tt NCConnect} object \ldots)
using the arguments passed to the open call.

The constructor for {\tt NCConnect} is shown in
Figure~\ref{fig:NCConnect-ctor}. Note that the constructor takes {\em mode\/}
as a second argument, but never uses it. This is an artifact of the original
design for the class; in a future version this may be removed. The
constructor initializes several members and then clears its internal symbol
table with the {\tt init\_list} private member function.

After the open call creates the new {\tt NCConnect} object, it makes sure
that the user is not trying to open the remote data set for writing and, if
they are not, reads the DAS and DDS from the data set. Once read, the DAS and
DDS objects are parsed using {\tt NCConnect}'s member functions {\tt
  parse\_das\_loc} and {\tt parse\_dims}. These two member functions, along
with the additional state variables in NCConnect, effectively create the
virtual connection. Subsequent calls to the client-library for information
about the variables (e.g., their size, names, etc.) will be answered using
information stored in the symbol table in the {\tt NCConnect} object.

\begin{figure}
\begin{code}{cb}
int
ncopen(const char *path, int cmode)
{
    int id;
    NCConnect *c = new NCConnect(path, cmode);

    cdf_routine_name = "ncopen" ;

    if(cmode != NC_NOWRITE) {    
        NCadvise(NC_EPERM, "Error, read-only access over the network");
        delete c;
        return(-1);
    }

    if(!c->request_das()){
        NCadvise(NC_EINVAL, "Error in getting DAS from \"%s\"",path);
        delete c;
        return(-1);
    }

    if(!c->request_dds()){
        NCadvise(NC_EINVAL, "Error in getting DDS from \"%s\"",path);
        delete c;
        return(-1);
    }

    // we can extract dim. and attr. info into tables
    c->parse_das_loc();
    c->parse_dims();

    return(conns.add_connect(c)) ;
}
\end{code}
\caption{The recoded open call of NetCDF.}
\label{fig:open}
\end{figure}

\begin{figure}
\begin{code}{cb}
class NCConnect: public Connect {
private:
    int _ncid;                  // used only if LOCAL is true, otherwise == -1
    int _nvars;                 // Number of variables in this file
    int _ndims;                 // Number of dimensions found
    String _dim_name[MAX_NC_DIMS];
    int _dim_size[MAX_NC_DIMS];
    int _das_loc[MAX_NC_VARS];
    void init_list(int i);
    void parse_array_dims();
    void parse_grid_dims();

public:
    NCConnect(const String &name, const int mode);
    ~NCConnect();

    int &ncid();
    int &ndims();
    int &nvars();
    int &dim_size(const int dimid);
    const String &dim_name(const int dimid);
    int &das_loc(const int varid);
    void parse_das_loc();
    void parse_dims();
};
\end{code}
\caption{The subclass of Connect used with the NetCDF client-library.}
\label{fig:NCConnect}
\end{figure}

\begin{figure}
\begin{code}{cb}
NCConnect::NCConnect(const String &name, const int mode) : Connect(name, API)
{
    if(is_local()){
        _ncid = -1;             // default value
    }
    else {
        _ncid = -1;
        _nvars = -1;
        init_list(MAX_NC_VARS); // clear the look up table
    }

}       
\end{code}
\caption{The constructor for NCConnect.}
\label{fig:NCConnect-ctor}
\end{figure}

\subsubsection{Requesting Data Values}
\label{tk:requesting-data}

When the user program requests data it typically does so be requesting that
one or more variables be read from the data set. This request must be
translated into a URL referencing the data set's variable or variables. This
is done by appending the suffix {\tt .dods} to the data set's URL and then
appending a constraint expression (CE) to the resulting URL. For example,
suppose that a data set has the following URL: {\em
  http://machine/cgi-bin/nc/data/exp1.nc}. To access the variable {\tt
  var1} the client-library would create the URL: {\em
  http://machine/cgi-bin/nc/data/exp1.nc.dods?var1\/}. Note that the Connect
class contains a member function, {\tt request\_data} that performs this
task; it takes three arguments: the CE to append, a flag which is used to
request synchronous or asynchronous behavior and the suffix to use for
requesting data. The last argument defaults to {\tt .dods} which all the DODS
data servers expect, thus for most purposes this member function requires
only two arguments (See Section~\ref{tk:using-constraints} for more
information on using constraint expressions to access data).

When a DODS data server sends a variable (i.e., when it sends the variable's
value) the binary data is prefixed with a block of text that contains a DDS
describing just the returned data (rather than the entire data set). The
client-library should use this DDS object to {\em deserialize\/} the requested
variable(s) and access their values. This is important because some variables
may have been constrained is such a way that their types have changed. It is
important to use the correct variable type when decoding the binary
information. For example, suppose a user program provides a way for a user to
select a two dimensional slice of a three dimensional object. When the API is
used to get such a variable the user program will expect the API to only
return two dimensions, not three. Furthermore, DODS CEs are powerful enough
to limit the data sent over the network to only those dimensions requested
(via the API) by the user program (i.e., you can use the CE mechanism to
limit the request to two of three dimensions, among other things). However,
when the variable is deserialized you must be sure that the type information
used deserializes a two dimensional variable and not a three dimensional
variable. That is why each data access is prefixed by a DDS which corresponds
to the type(s) of variable(s) sent for that access. The {\tt request\_data}
member function returns a DDS which matches the incoming data and can be used
to deserialize it.

Once the {\tt request\_data} member function has returned, the
client-library must still call the deserialize member function (which is part
of the DODS Type Hierarchy) for each variable in the return document. The
client-library should use the variable objects contained in the DDS object
returned by {\tt request\_data} to invoke the {\tt deserialize} member
function. Once that is done, the data values are stored in the internal
buffers of the variable objects in the new DDS. The client-library should
store this new DDS, along with the CE passed to {\tt request\_data} so that
future requests by the user program for the same information can be handled
without accessing the remote data server.

The data values of variables in a DDS are accessed using the {\tt buf2val}
member function for the cardinal and vector types and by accessing the values
of fields for constructor types.

\subsection{Using Constraints}
\label{tk:constraints}

Constraint expressions (CE) are an important part of DODS because they
provide a powerful way to control how data is accessed without forcing the
\Dap\ to support a lot of different messages. CEs are used to select which
variables will be extracted from a data set by both the user and by the
client-library. In the preceding sections, the mechanism by which the
client-library requests a single variable was described. However, it is
possible to formulate complicated CEs in just the same way as it is possible
to make CEs which request only a single variable. The difference, of course,
is that the expression itself must be different; the process of building the
URL, requesting the data it describes and decoding the resultant binary
information is the same.

\subsubsection{How Constraint Expressions are Evaluated}
\label{tk:ce-evaluation}

The server-side CEs are evaluated using a two step process. Every CE has
two parts, the projection and the selection subexpressions (see ``\DAP''). The
projection part of a CE tells which variables to include in any return
document describing the data set (although they don't yet work for the DAS
object) and the selection subexpression limits the returned data to variables
with values that satisfy a set of relational expressions. The projection
subexpression is evaluated when the entire constraint expression is parsed;
at parse-time the server's copy of the data set's DDS is marked with the
variables included in the projection. The selection subexpression, however,
is not evaluated until values are read from the data set. One way to classify
the projection and selection subexpressions is that projections depend solely
on the logical structure of a data set, while selections depend on the values
of particular variables within that data set.

\subsubsection{Different Ways of Using Constraint Expressions}
\label{tk:using-constraints}

There are two different ways that CEs can be used. One is by the
client-library and the other is by the user. When writing a client-library
that has features for selecting variables or parts of variables, try to code
the replacements to those calls so that they build up DODS CEs that will
request only the data the user wants. Then read the data from the returned
DDS and store it in the variable(s) passed to the API call by the user. This
is a much better solution than requesting the entire variable from the data
set and then throwing away parts of it.

Suppose that the user program (via the APIs functional interface) asks for
the data in variable {\em X\/}. The CE that will retrieve {\em X\/} is simply
`{\em X\/}'. Suppose, given the following DDS:

\begin{code}{cb}
Dataset {
    Int32 u[time_a = 16][lat = 17][lon = 21];
    Int32 v[time_a = 16][lat = 17][lon = 21];
    Float64 lat[lat = 17];
    Float64 lon[lon = 21];
    Float64 time[time = 16];
} fnoc1;
\end{code}

\noindent\ that the user program requests the two variables {\em u\/}
and {\em v\/} from the embedded structure. A CE that would project just those
variables would be {\em fnoc1.u, fnoc1.v\/}. To restrict the arrays {\em u\/}
and {\em v\/} to only the first two dimensions (time and lat), the projection
subexpression would be: {\em fnoc1.u[1:16][1:17], fnoc1.v[1:16][1:17]\/}.
Both of these CEs have null selection subexpressions. Note that the comma
operator separates the two clauses of the projection subexpression. Also note
that whitespace is ignored by the CE parser. See the grammar for CEs in the
\Dap\ for more information about CE grammar and the kind of things that can
be done with the projection subexpression.

The user program may have an interface that provides the user with a way to
request only certain values be returned. This is particularly true for APIs
such as JGOFS that support access to relational data sets. Suppose the
following DDS describes a relational data set:

\begin{code}{cb}
Dataset {
    Sequence {
        Int32 id;
        Float64 lat;
        Float64 lon;
        Sequence {
            Float64 depth;
            Float64 temperature;
        } xbt;
    } site;
} cruise;
\end{code}

\noindent\ If the user program requests data with a certain range of latitude
and longitude values, then the selection subexpression would be used like
this: $\&\ lat\ >=\ 10.0\ \&\ lat\ <=\ 20.0\ \&\ long\ >=\ 5.5\ \&\ long\ <=\
7.5$.
Note that each clause of the selection subexpression {\em begins\/} with a
{\em \&\/} and that the clauses are combined using a boolean {\em and}.
Finally, using the previous DDS, if a user requested only depth and
temperature given the above lat and lon range (i.e., the user program
requests that only the depth and temperature values be returned given a
certain lat and lon range) the client-library would use the following CE:
$site.xbt.depth,\ site.xbt.temp\ \&\ lat\ >=\ 10.0\ \&\ lat\ <=\ 20.0\ \&\
long\ >=\ 5.5\ \&\ long\ <=\ 7.5$.

A second way that CEs can be used is that users may specify an initial URL
with a CE already attached. In this case the {\tt request\_data} member
function will append the CE built by the client-library to the one supplied
by the user. In effect the DDS returned by the user-constrained URL will be
the DDS of a virtual data set; one that they have made using an existing data
set and some constraints. Form the standpoint of a client-library (or a data
server, for that matter) there is no difference between a URL supplied with
an initial constraint and one supplied without one.

\section{Linking User Programs}
\label{tk:linking}

When linking a user program to the DODS client-library version of an API
library, several ne libraries must be substituted for the original library.
These libraries are: the reimplemented API (or API for short), {\tt
  libdap++}, {\tt libnetio} and {\tt liberrmsg}. Because the DAP library and
the API have circular dependencies they must each be included on the linker
command line twice. An example of this can be seen in the DODS-NetCDF
Makefile; the value of {\em LIBS\/} is passed to the linker {\tt ld}. Note
that the API library is listed first.

\begin{code}{cb}
LIBS = -lnc-dods -ldap++ -lnc-dods -ldap++ -lnetio -lerrmsg 
\end{code}

You should have users link their programs using {\tt gcc} or {\tt g++} since
the libraries are all built using those tools. In particular, {\tt g++}
includes libstdc++ and libg++ by default when it builds an executable program
from object modules. If you use {\tt gcc} instead of {\tt g++} when you link,
e sure to include these libraries as well after all the libraries listed
above. If you don't use {\tt gcc}, but instead use the linker directly (i.e,
you call {\tt ld} yourself) you are on your own - you can use {\tt gcc -V} to
determine what flags and additional libraries it uses that are specific to your
system and then experiment with those. We cannot tell you how to proceed
since each UNIX variant requires different flags and libraries.

\section{Conclusion}

This document describes how to use the {\tt libdap++} and {\tt libnetio} to
simplify building DODS data servers and client-libraries. Each of the two
libraries contains a collection of \Cpp\ classes which can be used in the
filter programs and client-library to simplify reading data from a server and
returning it to a user program via a pre-existing API definition.

It is possible to build only a data server or client-library and, since all
data servers and client-libraries communicate using the same protocol, the
resulting software will still be very useful. However, once you have gone to
the trouble of sub-classing the DODS {\em Type Hierarchy\/} it is probably
worth the extra effort to build both components.

\newpage

\addcontentsline{toc}{section}{Bibliography}
\bibliographystyle{plain}
\bibliography{../boiler/dods}

\end{document}



