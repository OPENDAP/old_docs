% A guide to installing an OPeNDAP aggregation server.
%
\documentclass{dods-book}
\usepackage{acronym}
\usepackage{xspace}

% TODO: giftext.pl doesn't work on the laptop
%       check versions of hyperlatex.el for the if statement in
%           hyperlatex-forw-node.  This seemed to be missing in my
%           copy of version 2.6 for some reason.
%       check into CVS
%       the hyperlatex index still isn't so great
%
%  other laptop chores: Install Fink vsn of Gimp, check perlmagick

%\renewcommand{\subj}[1]{}

\rcsInfo $Id$
\newcommand{\DOCversion}{Version \rcsInfoRevision}

%%% Tex customizations and command definitions for DODS user
%%% guides, 2 October 1997 - tomfool
%%%
%%% $Id$

%%% $Log: layout.tex,v $
%%% Revision 1.5  1999/05/25 20:49:34  tom
%%% changed version numbers to 3.0
%%%
%%% Revision 1.4  1999/02/04 17:39:02  tom
%%% modified for use with dods-book.cls
%%%
%%% Revision 1.3  1998/03/13 20:43:26  tom
%%% writing of API manual.
%%%
%%% Revision 1.2  1998/02/12 15:47:25  tom
%%% updated for GUI doc
%%%
%%% Revision 1.1  1997/10/02 17:18:14  tom
%%% moved from user guide to boiler, made slightly more general,
%%% for use with other guides.
%%%

%%%%%%%%%%%%%%%%%%%%%%%%%%%%%%%%%%%%%%%%%%%%%%%%%%%%%%%%%%%%%%%%%%%%%
%%% This file is for TeX macros that are equally appropriate for the
%%% hard-copy (LaTeX) and html (hyperlatex) versions of the dods
%%% books.  If it's not appropriate for both, then it probably belongs
%%% in dods-book.cls or dods-book.hlx.

%%%%%%%%%%%%%%%%%%%%%%%%%%%%%%%%%%%%%%%%%%%%%%%%%%%%%%%%%%%%%%%%%%%%%
\NotSpecial{\do\_}% This removes the special meaning of `_', so for
                  % subscripts, there must be an `tex' environment
                  % around any diagram using subscripts, and an entire
                  % alternate html figure using <sub> tags.


%%%%%%%%%%%%%%%%%%%%%%%%%%%%%%%%%%%%%%%%%%%%%%%%%%%%%%%%%%%%%%%%%%%%%
%%% Different kinds of cross references.
%\newcommand{\chapterref}[1]{Chapter~\refl{#1} on page~\pagerefl{#1}}
\newcommand{\chapterref}[1]{\link{Chapter~\ref{#1}}{#1}}
\newcommand{\appref}[1]{\link{Appendix~\ref{#1}}%
  [~on page~\pageref{#1}]{#1}}
\newcommand{\sectionref}[1]{\link{Section~\ref{#1}}%
  [~on page~\pageref{#1}]{#1}} 
\newcommand{\pagexref}[1]{\link*{here}[page~\pageref{#1}]{#1}}
\newcommand{\tableref}[1]{\link{table~\ref{#1}}{#1}}
\newcommand{\Tableref}[1]{\link{Table~\ref{#1}}{#1}}
\newcommand{\figureref}[1]{\link{figure~\ref{#1}}{#1}}
\newcommand{\Figureref}[1]{\link{Figure~\ref{#1}}{#1}}

%%% A Prefatory list of the font conventions:
\newcommand{\listconventions} {
\section{Conventions}
\label{pref,conventions}

The \indn{typographic conventions} shown in
Table~\ref{typo-conventions} are followed in this guide and all the
other DODS documentation.

\begin{table}[htbp]
  \begin{center}
  \caption{Typographic Conventions}
  \label{typo-conventions}
  \begin{tabular}{|c|p{3in}|} \hline
    \lit{Literal text}  &  
         Typed by the computer, or in a code listing.\\ \hline
    \inp{User input}    &  
         Type this precisely as written.\\ \hline
    \var{Variables}     &   
         Used as a place holder for a user-specified or variable
         value. Choose an appropriate value and use that in place.\\
         \hline 
    \but{Button Text}\texonly{\rule{0pt}{2.5ex}}   
        &  Used to indicate text on a GUI button.\\ 
         \hline 
    \pdmenu{Menu Name}    &  This is the name of a GUI menu.\\ \hline 
  \end{tabular}
  \end{center}
\end{table}

When referring to a button in a menu, we will often use the notation:
\but{Menu,Button}. For example, \but{Options,Colors,Foreground} would
indicate the \but{Foreground} button in the \pdmenu{Colors} menu,
selected under the \pdmenu{Options} menu.
 }


%%% Local Variables: 
%%% mode: latex
%%% TeX-master: t
%%% End: 

%
% These are html links which are used often enough in writing about DODS to
% merit an input file.
% jhrg. 4/17/94
%
% File rationalized and updated while writing the DODS User
% Guide. Also includes other useful abbreviations.
% tomfool 3/15/96
%
% Moved to dods-def.tex so I can remove links to documents that no
% longer reflect reality.
% tomfool 2/13/98
%
% $Id$
%
% Make sure to include layout.tex *before* using this file.

%% NOTE NOTE NOTE NOTE NOTE NOTE NOTE NOTE NOTE NOTE NOTE NOTE NOTE NOTE 
%%
%% If this file causes problems when running latex, you may have to edit your
%% texmf.cnf file. Here's a meesage from Tom:
%% > Are these references that use relative addresses (like
%% > ../boiler/blah.tex)?  If they are, you should look for the texmf.cnf
%% > file.  (It's often at /usr/share/texmf/web2c/texmf.cnf, and look for
%% > the openout_any parameter.  Check there anyway; there were some recent
%% > (i.e. in the early '90's) security fixes to TeX.
%%
%%%%%%%%%%%%%%%%%%%%%%%%%%%%%%%%%%%%%%%%%%%%%%%%%%%%%%%%%%%%%%%%%%%%%%%%%%%

%%% These are some DODS-specific convenience commands.
\newcommand{\DODSroot}{\lit{\$DODS\_ROOT}}     
% $

\newcommand{\opendap}{OPeNDAP\xspace}

%%% The OPD books and reference material
\newcommand{\OPDDoc}{http://opendap.org/support/docs.html}
\newcommand{\DODSDoc}{http://opendap.org/support/docs.html}

% \newcommand{\OPDDoc}{http://www.unidata.ucar.edu/packages/dods}
% \newcommand{\DODSDoc}{http://www.unidata.ucar.edu/packages/dods}

\newcommand{\OPDhomeUrl}%
  {http://opendap.org}
\newcommand{\OPDexampleUrl}%
  {BROKEN--FIX ME!}%\OPDDoc/examples}
% \newcommand{\OPDftpUrl}%
%  {ftp://dods.gso.uri.edu/pub/dods}
\newcommand{\OPDftpUrl}%
  {ftp://ftp.unidata.ucar.edu/pub/opendap/}
\newcommand{\OPDuserUrl}%
  {\OPDhomeUrl/user/guide-html/}
\newcommand{\OPDmguiUrl}%
  {\l/user/mgui-html/}
\newcommand{\OPDapiUrl}%
  {\OPDhomeUrl/api/pguide-html/}
\newcommand{\OPDapirefUrl}%
  {\OPDhomeUrl/api/pref/html/}
\newcommand{\OPDffUrl}%
  {\OPDhomeUrl/user/servers/dff-html/}
\newcommand{\OPDquickUrl}%
  {\OPDhomeUrl/user/quick-html/}
\newcommand{\OPDinstallUrl}%
  {\OPDhomeUrl/server/install-html}%
\newcommand{\OPDregexUrl}%
  {\OPDhomeUrl/user/regex-html}%
\newcommand{\OPDjavaUrl}%
  {\OPDhomeUrl/home/swJava1.1/}
\newcommand{\OPDwclientUrl}%
  {\OPDhomeUrl/api/wc-html/}
\newcommand{\OPDwserverUrl}%
  {\OPDhomeUrl/api/ws-html/}
\newcommand{\OPDaggUrl}%
  {\OPDhomeUrl/server/agg-html/}

\newcommand{\OPDhome}{\xlink{OPeNDAP Home page}{\OPDhomeUrl}}
\newcommand{\OPDjava}{\xlink{OPeNDAP Java home page}{\OPDjavaUrl}}
\newcommand{\OPDexamples}{\xlink{OPeNDAP examples page}{\OPDexampleUrl}}
\newcommand{\OPDftp}{\xlink{OPeNDAP ftp site}{\OPDftpUrl}}
%% Book titles do *not* contain the article.
\newcommand{\OPDuser}[1][]{\xlink%
  {\cit{OPeNDAP User Guide}}{\OPDuserUrl{}#1}}
\newcommand{\OPDmgui}{\xlink%
  {\cit{OPeNDAP Matlab GUI}}{\OPDmguiUrl}}
\newcommand{\OPDapi}{\xlink%
  {\cit{OPeNDAP Toolkit Programmer's Guide}}{\OPDapiUrl}}
\newcommand{\OPDapiref}{\xlink%
  {\cit{OPeNDAP Toolkit Reference}}{\OPDapirefUrl}}
\newcommand{\OPDffbook}{\xlink%
  {\cit{OPeNDAP Freeform ND Server Manual}}{\OPDffUrl}}
\newcommand{\OPDquick}{\xlink%
  {\cit{OPeNDAP Quick Start Guide}}{\OPDquickUrl}}
\newcommand{\OPDinstall}{\xlink%
  {\cit{OPeNDAP Server Installation Guide}}{\OPDinstallUrl}}
\newcommand{\OPDregex}{\xlink%
  {\cit{Introduction to Regular Expressions}}{\OPDregexUrl}}
\newcommand{\OPDagg}{\xlink%
  {\cit{OPeNDAP Aggregation Server Guide}}{\OPDaggUrl}}
\newcommand{\OPDwclient}{\xlink%
  {\cit{Writing an OPeNDAP Client}}{\OPDwclientUrl}}
\newcommand{\OPDwserver}{\xlink%
  {\cit{Writing an OPeNDAP Server}}{\OPDwclientUrl}}

\newcommand{\OPDffs}{OPeNDAP Freeform ND Server}

%%% Other DODS links.
\newcommand{\homepage}% For hyperlatex
  {\OPDDoc/}
\newcommand{\OPDsupport}{\xlink{support@unidata.ucar.edu}{mailto:support@unidata.ucar.edu}}
\newcommand{\DODSsupport}{\xlink{support@unidata.ucar.edu}{mailto:support@unidata.ucar.edu}}
\newcommand{\DODS}{\xlink{Distributed Oceanographic Data System}{\OPDhomeUrl}}
\newcommand{\OPD}{\xlink{Open Source Project for a Data Access Protocol}{\OPDhomeUrl}}
\newcommand{\OPDtechList}{\xlink{OPeNDAP Mailing Lists}{\OPDhomeUrl/mailLists/}}

%%% DODS versions
%% This has been removed.  Documents should not have an automatic
%% version number, because then it appears as if they have been
%% updated when they haven't.  Put the relevant version number to
%% whatever software is being described into each document's preface. 

\newcommand{\ifh}{WWW Interface}

% external refs for DODS documents

\newcommand{\CGI}{\xlink{Common Gateway Interface}
  {http://hoohoo.ncsa.uiuc.edu/cgi/overview.html}}

\newcommand{\MIME}{\xlink{Multipurpose Internet Mail Extensions}
  {http://www.cis.ohio-state.edu/htbin/rfc/rfc1590.html}}

\newcommand{\netcdf}{\xlink{NetCDF}
  {http://www.unidata.ucar.edu/packages/netcdf/guide.txn_toc.html}}

\newcommand{\JGOFS}{\xlink{Joint Geophysical Ocean Flux Study}
  {http://www1.whoi.edu/jgofs.html}}

\newcommand{\jgofs}{\xlink{JGOFS}
  {http://www1.whoi.edu/jgofs.html}}

\newcommand{\hdf}{\xlink{HDF}
  {http://www.ncsa.uiuc.edu/SDG/Software/HDF/HDFIntro.html}}

\newcommand{\ffnd}{FreeForm ND}

\newcommand{\Cpp}{\texorhtml
  {{\rm {\small C}\raise.5ex\hbox{\footnotesize ++}}}
  {C\htmlsym{##43}\htmlsym{##43}}}

% Commands

% Use pdflink instead. jhrg 8/4/2006
% \newcommand{\pslink}[1]{\small
% \begin{quote}
%   A \xlink{PDF version}{#1} of this document is available.
% \end{quote}
% \normalsize
% }

% Use the copy of this in dods-paper.hlx/cls or cut and paste this on
% a document-by-document basis. This version conflicts with the
% version in the class. jhrg 8/4/2006
% \newcommand{\pdflink}[1]{\small
% \begin{quote}
%   A \xlink{PDF version}{#1} of this document is available.
% \end{quote}
% \normalsize
% }

%%%%%%%%%%%%%% DODS macros
%
% These are here so that older latex files will compile. Someday remove these
% and fix the files. 04/13/04 jhrg

\newcommand{\DODShomeUrl}%
  {\OPDhomeUrl}
\newcommand{\DODSexampleUrl}%
  {\OPDDoc/examples}
% \newcommand{\OPDftpUrl}%
%  {ftp://dods.gso.uri.edu/pub/dods}
\newcommand{\DODSftpUrl}%
  {ftp://ftp.unidata.ucar.edu/pub/opendap/}
\newcommand{\DODSuserUrl}%
  {\OPDhomeUrl/user/guide-html/}
\newcommand{\DODSmguiUrl}%
  {\OPDhomeUrl/user/mgui-html/}
\newcommand{\DODSapiUrl}%
  {\OPDhomeUrl/api/pguide-html/}
\newcommand{\DODSapirefUrl}%
  {\OPDhomeUrl/api/pref/html/}
\newcommand{\DODSffUrl}%
  {\OPDhomeUrl/user/servers/dff-html/}
\newcommand{\DODSquickUrl}%
  {\OPDhomeUrl/user/quick-html/}
\newcommand{\DODSinstallUrl}%
  {\OPDhomeUrl/server/install-html}%
\newcommand{\DODSregexUrl}%
  {\OPDhomeUrl/user/regex-html}%
\newcommand{\DODSjavaUrl}%
  {\OPDhomeUrl/home/swJava1.1/}
\newcommand{\DODSwclientUrl}%
  {\OPDhomeUrl/api/wc-html/}
\newcommand{\DODSwserverUrl}%
  {\OPDhomeUrl/api/ws-html/}
\newcommand{\DODSaggUrl}%
  {\OPDhomeUrl/server/agg-html/}

\newcommand{\DODShome}{\xlink{OPeNDAP Home page}{\DODShomeUrl}}
\newcommand{\DODSjava}{\xlink{OPeNDAP Java home page}{\DODSjavaUrl}}
\newcommand{\DODSexamples}{\xlink{OPeNDAP examples page}{\DODSexampleUrl}}
\newcommand{\DODSftp}{\xlink{OPeNDAP ftp site}{\DODSftpUrl}}
\newcommand{\DODSuser}[1][]{\xlink%
  {\cit{The OPeNDAP User Guide}}{\DODSuserUrl{}#1}}
\newcommand{\DODSmgui}{\xlink%
  {\cit{The OPeNDAP Matlab GUI}}{\DODSmguiUrl}}
\newcommand{\DODSapi}{\xlink%
  {\cit{The DODS Toolkit Programmer's Guide}}{\DODSapiUrl}}
\newcommand{\DODSapiref}{\xlink%
  {\cit{The DODS Toolkit Reference}}{\DODSapirefUrl}}
\newcommand{\DODSffbook}{\xlink%
  {\cit{The DODS Freeform ND Server Manual}}{\DODSffUrl}}
\newcommand{\DODSquick}{\xlink%
  {\cit{The DODS Quick Start Guide}}{\DODSquickUrl}}
\newcommand{\DODSinstall}{\xlink%
  {\cit{The DODS Server Installation Guide}}{\DODSinstallUrl}}
\newcommand{\DODSregex}{\xlink%
  {\cit{Introduction to Regular Expressions}}{\DODSregexUrl}}
\newcommand{\DODSagg}{\xlink%
  {\cit{OPeNDAP Aggregation Server Guide}}{\DODSaggUrl}}
\newcommand{\DODSwclient}{\xlink%
  {\cit{Writing an OPeNDAP Client}}{\DODSwclientUrl}}
\newcommand{\DODSwserver}{\xlink%
  {\cit{Writing an OPeNDAP Server}}{\DODSwclientUrl}}

\newcommand{\DODSffs}{DODS Freeform ND Server}

% $Log: dods-def.tex,v $
% Revision 1.24  2004/12/21 22:30:04  jimg
% Fixed pslink; Added pdflink.
%
% Revision 1.23  2004/12/14 05:19:17  tomfool
% restored fix to pslink
%
% Revision 1.22  2004/12/09 21:01:58  tomfool
% excised test.dods.org
%
% Revision 1.21  2004/12/09 18:50:21  tomfool
% de-dodsifying
%
% Revision 1.15  2004/04/24 21:37:22  jimg
% I added every directory in preparation for adding everyting. This is
% part of getting the opendap web pages going...
%
% Revision 1.14  2004/02/12 16:05:50  jimg
% Moved the log to the end of the file.
%
% Revision 1.13  2004/01/16 18:05:31  jimg
% Added a note from Tom about setting texmf.cnf to allow \include to process
% files with ../ in their pathnames. You can also change the include to input,
% but I think include may offer some advantages for bigger/complex things like
% the Guides.
%
% Revision 1.12  2003/12/28 21:48:22  tom
% added newer books
%
% Revision 1.11  2003/12/08 19:04:43  tom
% little adjustments for DODS->opendap
%
% Revision 1.10  2003/12/08 18:53:30  tom
% DODS->OPeNDAP
%
% Revision 1.9  2002/07/15 17:49:55  tom
% added \DODSDoc
%
% Revision 1.8  2001/05/04 15:07:45  tom
% fixed pslink to include pdf files
%
% Revision 1.7  2001/02/19 20:39:13  tom
% added links to the new regex intro.
%
% Revision 1.6  2000/03/23 18:27:52  tom
% added abbreviations
%
% Revision 1.5  1999/07/01 16:00:19  tom
% added a couple of web page references
%
% Revision 1.4  1999/05/25 20:49:34  tom
% changed version numbers to 3.0
%
% Revision 1.3  1999/02/04 17:27:08  tom
% adjusted for hyperlatex and dods-book.cls
%
%

%%% Local Variables: 
%%% mode: latex
%%% TeX-master: t
%%% TeX-master: t
%%% End: 

%
% $Id$
%
\htmldirectory{agg-html}
\htmlname{agg}
\htmltitle{DODS Aggregation Server Guide}
\htmladdress{Tom Sgouros, \rcsInfoDate}
\htmlcss{/resources/dods-book.css}
\W\renewcommand{\HlxIcons}[1]{/icons/}

\newcounter{exampleno}
\setcounter{exampleno}{0}
\newcounter{examplerefno}
\setcounter{examplerefno}{0}
\newcommand{\examplelabel}[1]{\refstepcounter{exampleno}\label{#1}%
  \begin{center}Example \theexampleno\end{center}}
\newcommand{\exampleref}[1]{\texorhtml{Example \ref{#1}%
    \refstepcounter{examplerefno}\label{exref\theexamplerefno}%
    % This is a test whether r@exref... is defined.  If not, skip
    % anything with the \pageref macro.
    \expandafter\ifx\csname r@exref\theexamplerefno\endcsname\relax\else%
    \bgroup\count100=\pageref{exref\theexamplerefno}%
    \count101=\pageref{#1}\ifnum\count100=\count101\else~%
    on page~\pageref{#1}\fi\egroup\fi}
  {\link{Example \ref{#1}}{#1}}}

\newcommand{\tag}[1]{\emph{#1}}
\newcommand{\element}[1]{\link{\tag{#1}}{agg,xml,#1}}

\newcommand{\currentelement}{}
\newcommand{\ELEMENT}[1]{\renewcommand{\currentelement}{#1 element}%
  \subsection{#1}\label{agg,xml,#1}\indc{\currentelement}%
  \indc{catalog tag!#1}\indc{aggregation tag!#1}%
  \indc{XML!#1 element}}
\newcommand{\ATTRIBUTE}[1]{\item{#1}\indc{\currentelement!#1}
    \indc{#1 attribute!of \currentelement}%
    \indc{XML!#1 attribute}}

% OPeNDAP is defined in dods-def.tex
\newcommand{\AggServer}{\opendap\ Catalog/Aggregation Server}
\newcommand{\aggser}{Aggregation Server}
\newcommand{\aggversion}{0.6}
\newcommand{\thredds}{\ac{THREDDS}}
\newcommand{\dtd}{\ac{DTD}}
\newcommand{\dds}{\ac{DDS}}
\newcommand{\das}{\ac{DAS}}

% $Log: agg-def.tex,v $
% Revision 1.2  2004/07/07 22:51:23  jimg
% Updated icons and switched to dods-book.css
%
% Revision 1.1  2002/12/02 04:28:25  tom
% added aggregation server documentation to tree
%
%

%%% Local Variables: 
%%% mode: latex
%%% TeX-master: t
%%% TeX-master: t
%%% End: 
  % This contains software name and version number.

\figpath{agg/figs}
\htmldirectory{agg-html}
\htmltitle{OPeNDAP Aggregation Server Guide}

\makeindex

\begin{document}
%------------------------------------front matter
\title{OPeNDAP Aggregation Server Guide\\\DOCversion}
\author{John Caron\\Tom Sgouros}
\date{\rcsInfoDate}
\pagenumbering{arabic}
\maketitle

\copyrightmatter

\W\pslink{http://www.opendap.org/pdf/agg.pdf}

% Preface to the DODS Programmer's Guide.
%
% $Id$
%
% $Log: preface.tex,v $
% Revision 1.5  2004/02/18 06:38:48  jimg
% Various changes, mostly for the DODS --> OPD macros.
%
% Revision 1.4  1999/07/22 18:54:56  tom
% fixed errors
%
% Revision 1.3  1999/02/04 17:46:05  tom
% Modified for dods-book.cls and Hyperlatex
%
% Revision 1.2  1998/12/07 15:51:33  tom
% updated for DODSv2.19
%
% Revision 1.1  1998/03/13 20:50:04  tom
% created API manual from James's Toolkit document
%
%
%

\T\chapter*{Preface}
\T\addcontentsline{toc}{chapter}{Preface}

This document describes how to use the \opendap\ toolkit software to build
\opendap\ data servers, clients and client-libraries. Using the objects and
functions contained in the toolkit, you can create programs which serve data
over the internet as well as programs that can request data from any
\opendap\ server.

This document covers release \DODSversion\ and later of the DODS
software.

\begin{ifhtml}
  \htmlmenu{4}
  \chapter*{Preface}
\end{ifhtml}

%%%%%%%%%%%%%%%%%%%%%%%%%%%%%%%%%%%%%%%%%%%%%%%%%%%%%%%%%%%%%%%%%%%%
\section{Who is this Guide for?}
\label{pref,who}

This guide is for people who wish to use the \opendap\ software to write a
new \opendap\ data server, a new client, or a new client library. Typically,
this will only be those people who wish to serve data in a format that is not
currently supported by the DODS team, or who have an existing application
that uses an idiosyncratic or unusual API for data access. Most people will
be able to use one of the already written servers or client libraries. See
the \OPDuser\ for a list of these.

This documentation assumes that the readers are \Cpp\ programmers, are
familiar with networked applications, and the POSIX programming environment.
The DODS/\opendap project also provides a native Java class library (API)
that parallels the \Cpp\ spftware described here.\footnote{While this manual
describes the \Cpp\ toolkit in detail, all of the concepts and much of the
structure can be directly translated to the Java toolkit.}

Also available are two tutorials, \OPDwclient\ and \OPDwserver\ , which
descrie how to write a client or a server, respectively.

Because the type of information presented in a document like this depends to
a large extent on the needs of its readers we welcome your feedback and
comments. In particular, if you have any questions about individual sections,
email those questions and we'll send back an answer as well as including that
information in the next version of this document. Send queries to:
\DODSsupport.


%%%%%%%%%%%%%%%%%%%%%%%%%%%%%%%%%%%%%%%%%%%%%%%%%%%%%%%%%%%%%%%%%%%%
\section{Organization of this Document}

This Guide is divided into five chapters. 

\begin{description}
  
\item[\chapterref{tk,overview}] provides background information on the
  organization of the toolkit software.

\item[\chapterref{tk,manage-conns}] describes how to use the
Network I/O classes to manage virtual connections.

\item[\chapterref{tk,subclassing}] discusses how to sub-class the toolkit
\Cpp\ classes so that they are specialized for your specific use.

\item[\chapterref{tk,using}] describes in detail how to write certain
sections of both the data server and the client-library for a new API.

\item[\chapterref{tk,linking}] describes how to link user programs
with the new client-library implementation of an API. 

\texonly{\item[\chapterref{tk,classref}] contains complete
  descriptions of all the DODS classes.}

\end{description}

\htmlonly{Note that this is {\em not\/} a reference volume. See
  \OPDapiref\ for a concise listing of the member functions in each
  of the toolkit's classes.}

%%%%%%%%%%%%%%%%%%%%%%%%%%%%%%%%%%%%%%%%%%%%%%%%%%%%%%%%%%%%%%%%%%%%

\listconventions

%%% Local Variables: 
%%% mode: latex
%%% TeX-master: "../pguide.tex"
%%% End: 


\tableofcontents
\listoffigures
%\listoftables

\clearemptydoublepage
%------------------------------------book body

%  Outline of this book:
%
%
%  How to install an OPeNDAP server.
%
%    What an OPeNDAP server is.
%      httpd server equipped with a special set of CGI scripts and
%      access to data to be served.
%    How to set one up.
%      Start by putting the scripts and perl modules in the cgi

\chapter{Theory}

The \AggServer\ is an \opendap\ server that creates virtual datasets
from collections of other \opendap\ datasets or from NetCDF files. It can
also serve single NetCDF files as (non-aggregated) \opendap\ datasets. It
uses \thredds\ catalogs to specify what datasets it serves, and how to
aggregate them.  The specification is formatted as an XML document.

\subj{\opendap\ data is accessed by files, but that's not always the
  way users want it.}  The \acf{OPeNDAP} is organized around the
concept that the relevant unit of data storage is a disk file.  In
order to get or retrieve data, you must know what file the desired
data is in.  To allow users to identify specific files, data providers
create lists of files, or catalogs, that may be browsed or searched.
This approach makes certain aspects of data management simpler, and is
a natural consequence of several popular data storage APIs which work
the same way.  However, many users may perceive it as a burden to have
to keep track of file names, which are not always chosen to be easy to
remember.  What's worse, it becomes quite awkward if the slice of data
you happen to be interested in spans many files.  Looking for a time
series in an archive of satellite data, for example, might require
thousands of requests to individual data files.

The \AggServer\ is designed to accommodate these problems, providing a
single point of entry to collections of many files.  Conceptually, it
consists of two components: a catalog of data files, and the smarts
necessary to determine how to use those data files to satisfy user
requests.  The files in question may be either local files, existing
on the same computer as the server, or they may be remote files,
specified with an \opendap\ URL.  The catalog is a file of XML
declarations which identify how individual data files can be aggregated
to look like single larger files.

\subj{To the end-user, aggregated data looks the same as other data.}
In operation, the \aggser\ accepts a query from a user, and determines
how to use the data files listed in its catalog to fulfill that query.
After making this determination, the \aggser\ makes the subsidiary
queries necessary to satisfy the original request, aggregates this
data into the form expected by the user who initiated the request, and
returns the result to that user.  The user need not even know that the
result is an aggregation.  Though a complex aggregation will obviously
take longer than a simple \opendap\ data request, there should be no
other differences in the user interaction.

\figureplace{JoinNew aggregation}{ht}
{JoinNew,fig}{JoinNew.ps}{JoinNew.gif}{}

\section{Methods of Aggregation}

\subj{Aggregated data is data that should belong together.}
Any data can be included in a given data catalog, but not all data can
be (or should be) aggregated.  Two data files that can (usefully)
appear joined together in an \aggser\ might be records of the same kinds of
data on different dates, or different kinds of data at the same
location, or data at adjoining locations at the same time.  That is,
the data to be aggregated must share some common features to be worth
the trouble to aggregate them.  If you can't imagine a user wanting to
make the same query to two different data files, they probably aren't
worth aggregating.

Even though two data files are not aggregated, they may still be
included as part of the same data catalog.  The catalog can help users
find data, even though it isn't aggregated.

The \AggServer\ can handle three different forms of aggregation.  We
call these three \class{JoinNew}, \class{JoinExisting}, and \class{Union}.
We'll examine each of these in turn.  

\note{As of version \aggversion\ of the \aggser , only Grid and Array
  data types can be aggregated.}


\paragraph{JoinNew}
The \class{JoinNew} form of aggregation is used to join datasets along
a new dimension.  For example, if you have a set of measurements taken
of the same spatial area at different dates, you could arrange these
\subj{JoinNew is used to aggregate data along a data type not in any
file.}in order by time to create a time series of measurements.  If a user
wanted to see the time evolution of a measurement at a subsection of
the larger measured area, the situation might look sort of like what's
pictured in \figureref{JoinNew,fig}.

Here, A, B, and C represent sets of measurements in the X and Y
dimensions.  The three sets are aggregated along the Z axis.  This
means that none of A, B, or C have a Z variable in those sets, but
they all contain (or represent) a single Z value.  In our example
above, X and Y are spatial dimensions, and Z would be time.  

\figureplace{JoinExisting aggregation}{hbt}
{JoinExisting,fig}{JoinExist.ps}{JoinExist.gif}{}

\paragraph{JoinExisting} If you
have several datasets that consist of data in adjoining regions of
space or time, you may be able to aggregate them with the
\class{JoinExisting} form of aggregation.  For example, if you have a
time series that begins right after another one ends, you can combine
\subj{JoinExisting is used to combine two adjoining datasets.}  
these two along the time axis.  Similarly, if you have two spatial
grids, whose edges adjoin, you might be able to join them along the
dimension orthogonal to the common edge.  Something of this nature is
shown in \figureref{JoinExisting,fig}, where three datasets, each of
which contain independent variables X, Y, and Z, are joined on the Z
axis.

\figureplace{Union aggregation}{!hbtp}
{Union,fig}{Union.ps}{Union.gif}{}

\paragraph{Union}  If you need to aggregate datasets that cover
the same space and time areas, but consist of different data types,
you can use the \class{Union} aggregation to join them.  
\subj{Union is used to combine datasets consisting of
  different data types.} 
For example,
you might have a grid of satellite sea surface temperature values, and
another grid of wind speed observations.  You can use the
\class{Union} aggregation to combine the two into one dataset
containing both temperature and wind speed.

Now that you have the idea of what is meant by aggregation, the next
section will show how to specify the method and parameters for
aggegration using the necessary XML syntax.  The following chapter
will explain how to install the server.  \chapterref{aggser,configure}
describes how to configure the aggregation server so it will show your
data in the way you want.


\section{A little more detail}

\indc{\acs{DDS} (\acl{DDS})}\indc{\acs{DAS} (\acl{DAS})}
Here are some worked-out examples of aggregation types.  If you don't
know what the elements of an \opendap\ \dds\ mean, or aren't familiar
with the basic \opendap\ data types, please refer to \DODSuser .

\subsection{JoinNew Aggregation}
\label{agg,joinnew}


\indc{JoinNew!aggregation type}\indc{aggregation type!JoinNew}
\indc{adding a dimension!JoinNew}\indc{dimension!adding}
\indc{variable!adding}\indc{time, adding |see{JoinNew}}
The \tag{JoinNew} aggregation type joins variables along a new
dimension. The dimension and a coordinate variable are created and
values for the coordinates are specified in the \element{aggregation}
element: 

\begin{vcode}{sib}
<aggregation serviceName="ISCCP" aggType="JoinNew" 
             varName="time">
  <fileAccess urlPath="cldfrc/isccp.8501.bin" coord="Jan"/>
  <fileAccess urlPath="cldfrc/isccp.8502.bin" coord="Feb"/>
  <variable name="cldfrc"/>
</aggregation>
\end{vcode}

This XML fragment will combine these two datasets:

\begin{vcode}{sib}
Dataset {
  Float32 cldfrc[lat = 180][lon = 360];
} isccp_c2;

Dataset {
  Float32 cldfrc[lat = 180][lon = 360];
} isccp_c2;
\end{vcode}

into this one:

\begin{vcode}{sib}
Dataset {
  String time[time = 2];
  Float32 cldfrc[time = 2][lat = 180][lon = 360];
} ISCCP/cldfrc;
\end{vcode}

The coordinate variable \lit{time} has been assigned the values
``Jan'' for dataset 8501 and ``Feb'' for dataset 8502, as specified in
the \element{aggregation} tag.

\note{\class{JoinNew} aggregations will automatically join all
  variables of type \class{Grid}.  Variables of type \class{Array} may
  also be joined, but must be explicitly listed in a
  \element{variable} element. In the example here, the
  \element{variable} element specified that all arrays called
  \lit{cldfrc} were to be joined.}


There are several options for assigning values to the coordinate
variable of the joined dimension of \class{JoinNew} aggregations; see
a description of the \tag{varType} and \tag{varUnit} attributes of the
\element{aggregation} element. The most common use of \class{JoinNew}
aggregations is to add a time dimension. The \tag{dateFormat}
attribute is used to convert date/time strings into other units. In
the following example:

\begin{vcode}{sib}
<aggregation aggType="JoinNew" varName="time" varType="int" 
    varUnit="days since 0000-01-01 00:00:00" 
    dateFormat="yy/M/d:HH:mm:ss z">
<fileAccess 
    urlPath="20020101.dat" coord="2002/1/1:00:00:00 GMT"/>
<fileAccess 
    urlPath="20020102.dat" coord="2002/1/2:00:00:00 GMT"/>
<fileAccess 
    urlPath="20020103.dat" coord="2002/1/3:00:00:00 GMT"/>
</aggregation>
\end{vcode}

\ind{The} time data strings are read in using the \class{SimpleDateFormat}
string \lit{yyyy/M/d:HH:mm:ss z}, then converted to the unit specified
with the \tag{varUnit} attribute:

\begin{vcode}{sib}
  days since 0000-01-01 00:00:00
\end{vcode}

using the \class{ucar.units} package and stored in the dataset as an
\lit{int32}, so the time coordinate values will look like:

\begin{vcode}{sib}
time[3]
0, 1, 2
\end{vcode}


\subsection{JoinExisting Aggregation}
\label{agg,joinexist}

\indc{JoinExisting!aggregation type}\indc{aggregation type!JoinExisting}
\indc{extending a dimension!JoinExisting}\indc{dimension!extending}
\indc{variable!extending}\indc{time, extending |see{JoinExisting}}
The \class{JoinExisting} aggregation type joins variables along an
existing dimension. The dimension must have a coordinate variable and
is named in the \element{aggregation} element:

\begin{vcode}{sib}
<aggregation serviceName="local" varName="time" 
    aggType="JoinExisting">
  <fileAccess urlPath="cdc/air.1948.nc"/>
  <fileAccess urlPath="cdc/air.1949.nc"/>
</aggregation>
\end{vcode}

The above aggregation declaration joins these two datasets:

\begin{vcode}{sib}
Dataset {
  Float32 level[level = 17];
  Float32 lat[lat = 73];
  Float32 lon[lon = 144];
  Float64 time[time = 366];
  Grid {
    ARRAY:
      Int16 air[time = 366][level = 17][lat = 73][lon = 144];
    MAPS:
      Float64 time[time = 366];
      Float32 level[level = 17];
      Float32 lat[lat = 73];
      Float32 lon[lon = 144];
  } air;
} cdc/air.1948.nc;

Dataset {
  Float32 level[level = 17];
  Float32 lat[lat = 73];
  Float32 lon[lon = 144];
  Float64 time[time = 365];
  Grid {
    ARRAY:
      Int16 air[time = 365][level = 17][lat = 73][lon = 144];
    MAPS:
      Float64 time[time = 365];
      Float32 level[level = 17];
      Float32 lat[lat = 73];
      Float32 lon[lon = 144];
  } air;
} cdc/air.1949.nc;
\end{vcode}

The resulting \dds\ is shown below in \exampleref{ex10}.  Notice the
\lit{time} dimension of the \lit{air} grid.  In this case, all
variables of type \class{Array} or \class{Grid }with the specified
outermost dimension are joined into a single variable in the
aggregated dataset.  All other non-joined variables, and all
attributes are taken from the first of the joined datasets.

\examplelabel{ex10}
\begin{vcode}{sib}
Dataset {
  Float32 level[level = 17];
  Float32 lat[lat = 73];
  Float32 lon[lon = 144];
  Float64 time[time = 731];
  Grid {
    ARRAY:
      Int16 air[time = 731][level = 17][lat = 73][lon = 144];
    MAPS:
      Float64 time[time = 731];
      Float32 level[level = 17];
      Float32 lat[lat = 73];
      Float32 lon[lon = 144];
  } air;
} local/MeanAir;
\end{vcode}

\subsection{Union Aggregation}
\label{agg,union}

\indc{Union!aggregation type}\indc{aggregation type!Union}
\indc{collection, variables |see{Union}}
A \class{Union} aggregation type creates the union of all the
variables in the component datasets.  Assume we have two datasets,
with the following two \dds 's:

\begin{vcode}{sib}
Dataset {
  Float32 lat[lat = 21];
  Float32 lon[lon = 360];
  Grid {
    ARRAY:
      Int16 cldc[lat = 21][lon = 360];
    MAPS:
      Float32 lat[lat = 21];
      Float32 lon[lon = 360];
  } cldc;
} coads/cldc.mean.nc;

Dataset {
  Float32 lat[lat = 21];
  Float32 lon[lon = 360];
  Grid {
    ARRAY:
      Int16 lflx[lat = 21][lon = 360];
    MAPS:
      Float32 lat[lat = 21];
      Float32 lon[lon = 360];
  } lflx;
} coads/lflx.mean.nc;
\end{vcode}

Using the following XML fragment to specify the aggregation creates a
\dds\ like the one in \exampleref{ex9}.

\begin{vcode}{sib}
<aggregation serviceName="COADS" aggType="Union">
 <fileAccess urlPath="cldc.mean.nc"/>
 <fileAccess urlPath="lflx.mean.nc"/>
</aggregation>
\end{vcode}

The resulting \dds\ (and the accompanying \das ) is the union of all the
individual files' \dds . The first time a variable is encountered it is
added to the combined \dds , subsequent variables with the same name are
ignored. In the above example, the top level variables \lit{lat}
and \lit{lon} are therefore taken from the
\lit{cldc.mean.nc} dataset.  See \figureref{Union,fig}.

\examplelabel{ex9}
\begin{vcode}{sib}
Dataset {
  Float32 lat[lat = 21];
  Float32 lon[lon = 360];
  Grid {
    ARRAY:
      Int16 cldc[lat = 21][lon = 360];
    MAPS:
      Float32 lat[lat = 21];
      Float32 lon[lon = 360];
   } cldc;
  Grid {
    ARRAY:
      Int16 lflx[lat = 21][lon = 360];
    MAPS:
      Float32 lat[lat = 21];
      Float32 lon[lon = 360];
  } lflx;
} local/coads;
\end{vcode}

\chapter{Practice I: Installing the \AggServer}
\label{aggser,install}

\indc{\aggser!installing}\indc{installing!\aggser}
\indc{Java!required version}\indc{system requirements!Java}
\indc{servlet!\aggser}
The \aggser\ is a Java servlet that implements an \opendap\ server both
for  netCDF files and for ``aggregation'' datasets.  Version 0.6 and
after requires Java 1.4.  

\section{Install Jakarta-Tomcat Server}
\label{aggser,installinstruct}

\indc{Tomcat servlet engine!installing}\indc{installing!Tomcat servlet
  engine}\indc{tomcat.conf}\indc{server.xml}\indc{XML configuration!Tomcat}

To begin installation, you must install the Tomcat servlet engine.
The \aggser\ will work under either Tomcat 3.3 or 4.0.  Install Tomcat
in standalone mode.  This is the default configuration, using
port 8080 and the default tomcat.conf and server.xml files.
\subj{The \aggser\ requires the Tomcat servlet engine.}

Here is an example installation, assuming that the Aggregation Server
name is dodsC, so that all URLS start with \emph{server}\lit{/dodsC},
where \emph{server} is whatever the name of your machine is.


\begin{enumerate}
\item Edit \lit{\$TOMCAT\_HOME/bin/startup.sh} and make sure it
  contains the equivalent of the following:

\begin{vcode}{sib}
export TOMCAT_HOME=/usr/local/jakarta-tomcat
export JAVA_HOME=/usr/local/jdk1.4
\end{vcode}

Of course you may use different pathnames, depending on where you've
installed Tomcat or Java.

\item For expected heavy loads, you might want to increase the memory
  given to the server. If so, edit \lit{\$TOMCAT\_HOME/bin/tomcat.sh}
  and set the desired amount of memory, for example:

\begin{vcode}{sib}
TOMCAT_OPTS = "-Xmx512m"
\end{vcode}

\item Edit \lit{\$TOMCAT\_HOME/conf/server.xml} and add the following:

\begin{vcode}{sib}
<Context path="/dodsC"
         docBase="webapps/dodsC"
         crossContext="false"
         debug="0"
         reloadable="false" >
</Context>
\end{vcode}


\item Create the directory \lit{\$TOMCAT\_HOME/webapps/dodsC}, and copy
\xlink{dodsC.war}[ (find it at ftp://ftp.unidata.ucar.edu/pub/thredds/dodsC.war)]{ftp://ftp.unidata.ucar.edu/pub/thredds/dodsC.war}
into it. 

\item Unpack dodsC.war:

\begin{vcode}{sib}
cd $TOMCAT_HOME/webapps/dodsC
jar -xf dodsC.war
\end{vcode}
%$

\indc{web.xml}\indc{XML configuration!web.xml}
\item Default server values are in the web.xml file, located in 
  \lit{\$TOMCAT\_HOME/webapps/dodsC/WEB-INF/}. You might want
  to change one or more of the \lit{init-param} values, but it should
  work correctly right out of the box. In particular, runtime debug
  flags can be set with an init-param name of \lit{DebugOn}, and a
  value of (space delimited) debug flag names. Currently only the
  \lit{showRequests} flag is useful to the user. This will log each
  request received by Tomcat to the server console.

\indc{web.xml!example}
Here's an excerpt from an example \lit{web.xml} file:

\label{agg,webxmlex}
\begin{vcode}{xib}
<web-app>
  <display-name> OPeNDAP Catalog/Aggregation Server</display-name>
  <description>
    OPeNDAP Catalog Aggregation Server - for netCDF and 
    Aggregation datasets.
  </description>
  <servlet>
    <servlet-name> dodsC </servlet-name>
    <servlet-class> 
      dods.servers.agg.CatalogServlet 
    </servlet-class>
    <init-param>
      <param-name>displayName</param-name>
      <param-value>OPeNDAP Aggregation Server</param-value>
    </init-param>
    <init-param>
      <param-name>serverConfig</param-name>
      <param-value>
        file:///../webapps/dodsC/AggServerConfig.xml
      </param-value>
    </init-param>
    <init-param>
      <param-name>maxDODSDatasetsCached</param-name>
      <param-value>0</param-value>
    </init-param>
    <init-param>
      <param-name>maxNetcdfFilesCached</param-name>
      <param-value>100</param-value>
    </init-param>
    <load-on-startup/>
  </servlet>
  <servlet-mapping>
    <servlet-name>dodsC</servlet-name>
    <url-pattern>/</url-pattern>
  </servlet-mapping>
  <session-config>
    <session-timeout>30</session-timeout>
    <!-- 30 minutes -->
  </session-config>
</web-app>
\end{vcode}

\end{enumerate}

\section{Locating the Configuration File}

\indc{XML configuration!locating the catalog}\indc{catalog!where to
  put it}\indc{configuration file!installing}

If you've successfully completed the steps above, the aggregation
server is now installed.  However, though you've successfully
configured the Tomcat servlet engine to run the \aggser , you haven't
yet configured the \aggser\ itself.  \subj{Now you have an
  \aggser .  But it won't work until you configure it.} To do that,
you need to create 
and install a configuration file.  There is an example configuration
file, \lit{AggServerConfig.xml}, provided in the \aggser\ 
distribution.  Look for it in
\lit{\$TOMCAT\_HOME/webapps/dodsC/WEB-INF/} (it comes in the
\lit{dodsC.war} archive file).  Copy this file up to
\lit{\$TOMCAT\_HOME/webapps/dodsC} then modify it for your datasets.
\chapterref{aggser,configure} contains instructions for comfiguring
your server.

\indc{web.xml!points to \aggser\ catalog}
If you want to keep the configuration file somewhere else (in a
different directory or on a different machine, you must
modify Tomcat's \lit{web.xml} file.  The \aggser\ configuration file
location is specified by the \lit{serverConfig} parameter.  For
example, the following XML code specifies that the configuration file
may be found at \lit{http://www.opendap.net/aggser.xml}

\begin{vcode}{xib}
<init-param>
  <param-name>serverConfig</param-name>
  <param-value>http://www.opendap.net/aggser.xml</param-value>
</init-param>
\end{vcode}

You can also use an absolute path to locate the configuration file on
your local machine.  Write it like this for a Win machine:

\begin{vcode}{xib}
<init-param>
  <param-name>serverConfig</param-name>
  <param-value>file://E:/dev/dodsAS/aggser.xml</param-value>
</init-param>
\end{vcode}

Or like this for a Unix machine:

\begin{vcode}{xib}
<init-param>
  <param-name>serverConfig</param-name>
  <param-value>file:///usr/local/dodsAS/aggser.xml</param-value>
</init-param>
\end{vcode}


\section{Setting Cache Sizes}

% Doesn't this belong in the configuration chapter?  And precisely how
% are the cache sizes set?

\indc{cache size!choosing}\indc{configuration!cache size}
\indc{performance!\aggser}\indc{tuning!\aggser}\indc{\aggser!tuning}
\indc{\aggser!performance}
The \AggServer\ uses two independent caches.  One is for serving local
netCDF files, and the other is used in the process of aggregating
remote files.

If your \aggser\ serves a small number of files, it will work fine
with the default settings.  If you plan on serving large number of files,
you must deal with this issue.

There are two independent caches, one for local netCDF files, and one
for remote \opendap\ files that you are aggregating. Generally,
\opendap\ files take up memory, but no system resources. The default
is to set this size to zero, which means there is no limit to the
cache.  A netCDF file uses a system file handle, so you are limited by
maximum number of open files possible on your system. The default is
100.

Set the cache size by adjusting parameters in Tomcat's
\lit{web.xml} configuration file (see \link{the web.xml example}[on
\pageref{agg,webxmlex}]{agg,webxmlex}).  The file contains a couple of
parameter initializations like this:

\begin{vcode}{sib}
<init-param>
  <param-name>maxDODSDatasetsCached</param-name>
  <param-value>0</param-value>
</init-param>
<init-param>
  <param-name>maxNetcdfFilesCached</param-name>
  <param-value>100</param-value>
</init-param>
\end{vcode}

Here are some tips for selecting an appropriate size for your cache:

\begin{itemize}
  
\item The larger the cache the better for performance. Resources are
  only used if needed.

\item Cache size must be at least as large as the largest number of
  datasets in a \class{JoinExisting} aggregation.
  
\item Setting the cache size at least as large as the largest number
  of datasets in a \class{JoinNew} or \class{Union} aggregation helps
  performance, but is not necessary.
\end{itemize}


\chapter{Practice II: Configuring the \AggServer}
\label{aggser,configure}

Now that you've set up your \aggser , you need to
configure it to handle your data.  All configuration is done with the
XML catalog file described in \chapterref{aggser,install}.  This
chapter describes what goes into that file.

Before reading this chapter, you should be familiar with basic XML
concepts, such as elements, attributes, and \dtd 's.  
%There are
%hundreds of books and resources out there.  Go find one.

\section{How it works}

The \AggServer\ is configured by adding \element{metadata} elements of
type \element{aggregation} to catalog \element{dataset} elements.  In
the example catalog fragment below (\exampleref{ex1}), a dataset named
``NCEP-RUC Model Output'' is defined. It will have a relative URL of
``NCEP/RUC''.  That is, if the \aggser\ at \lit{http://opendap.org}
has a catalog with these entries in it, the aggregated data
represented by the \element{dataset} element will be at
\lit{http://opendap.org/NCEP/RUC}.  The data a client sees there will
be created by joining three NetCDF files along the existing
\lit{valtime} dimension, specified in the \tag{variable} attribute of
the \element{aggregation} element. The NetCDF files will be found at
\lit{E:/data/070516.nc} and so on.
\subj{Specify datasets to be aggregated with XML aggregation elements.}
\indc{JoinExisting!example}\indc{configuration!JoinExisting example}

\examplelabel{ex1}
\begin{vcode}{sib}
<dataset name="NCEP RUC Model Output" urlPath="NCEP/RUC">
  <service name="localGrids" serviceType="NetCDF" 
    base="file:///E:/data/"/>
  <metadata metadataType="Aggregation">
    <aggregation serviceName="localGrids" 
        variable="valtime" type="JoinExisting">
      <fileAccess urlPath="070516.nc"/>
      <fileAccess urlPath="070517.nc"/>
      <fileAccess urlPath="070518.nc"/>
    </aggregation>
  </metadata>
</dataset>
\end{vcode}

The \aggser\ can also aggregate files from other \opendap\ servers. In 
the following example, the aggregation dataset called ``Example
JoinNew'' is created from files that live on the ``GSO'' \opendap\ 
server. The datasets are joined by creating a new dimension called
\lit{time}. The data values of that dimension are specified in the
\tag{fileAccess} elements. The \opendap\ datasets out of which this
aggregate dataset is constructed have URLs like this:
\subj{The aggregated datasets can be remote ones, too.}
\lit{http://opendap.org/cgi/nph-dsp/htnsst/k2828.htn}.
\indc{JoinNew!example}\indc{configuration JoinNew example}

\examplelabel{ex2}
\begin{vcode}{xib}
<dataset name="Example JoinNew" urlPath="htn_sst_decloud">
  <service name="GSO" serviceType="DODS" 
      base="http://opendap.org/cgi/nph-dsp/"/>
  <metadata metadataType="Aggregation">
    <aggregation serviceName="GSO" variable="time" 
      type="JoinNew" dateFormat="yyyy/M/d:HH:mm:ss z">
      <fileAccess urlPath="htnsst/k2828.htn" coord="1985/1/1:18:28:28 GMT"/>
      <fileAccess urlPath="htnsst/k1750.htn" coord="1985/1/2:18:17:50 GMT"/>
      <fileAccess urlPath="htnsst/k0718.htn" coord="1985/1/3:18:07:18 GMT"/>
    </aggregation>
  </metadata>
</dataset>
\end{vcode}

\indc{netCDF!serving un-aggregated}\indc{DODS server!using \aggser\ as}
The \aggser\ is also an \opendap\ server for non-aggregated NetCDF
files. The following shows a collection of \opendap\ datasets. Since
they don't have an aggregation \element{metadata} element
(\lit{metadataType=''Aggregation''}, they refer to
NetCDF files that are served as \opendap\ datasets by this \aggser .
\subj{You can use the \aggser\ to serve simple netCDF files, too.}
There is more detail about this in \sectionref{agg,simplenetcdf}.

\examplelabel{ex3}
\begin{vcode}{sib}
<dataset name="Non-Aggregated Netcdf Files" serviceName="this">
   <property name="internalService" value="local"/>
   <dataset name="RUC/01070516" urlPath="01070516_ruc.nc"/>
   <dataset name="RUC/01070517" urlPath="01070517_ruc.nc"/>
   <dataset name="RUC/01070518" urlPath="01070518_ruc.nc"/>
 </dataset>
\end{vcode}

Note here that the \lit{this} service needs no \element{service}
element to define it.

\section{Dataset URLs}
\label{agg,dataseturls}

The \new{Dataset URL}is the URL that clients use to access an \aggser\ 
dataset.  This would be \lit{http://opendap.org:8080/RUC/01070516} in
\exampleref{ex3} (assuming that Tomcat is listening to port 8080).
For those who are familiar with the operation of other \opendap\ 
servers, note that in the \aggser\ catalog, you do not use the usual
\opendap\ suffixes (\lit{.dds}, \lit{.das}, \lit{.info}, \lit{.html},
and so on). These are automatically added by the \aggser\ when
constructing html pages, and by \thredds\ clients when they read the
\lit{catalog.xml} file directly.

In general, \thredds\ catalogs can specify Datasets that may come from
multiple \opendap\ (and even non-\opendap ) servers. For \aggser\ 
catalogs, typically we want to specify just Datasets that are served
by \lit{this} \aggser . The following fragment shows how a typical
\aggser\ catalog begins:

\examplelabel{ex4}
\begin{vcode}{sib}
<catalog name="Example Agg Server Catalog" version="0.6" >
<dataset name="Top level" dataType="Grid" serviceName="this">
  <service name="this" serviceType="DODS" base=""/>
     ...
    <dataset name="NCEP RUC Model Output" urlPath="NCEP/RUC">
    ...
    </dataset>
</dataset>
</catalog>
\end{vcode}

The first line shows the root \element{catalog} element. A
\element{catalog} element always has exactly one dataset element,
sometimes known as the "top-level" dataset, shown on the second line.
The third line defines the \opendap\ \element{service} that is the
\aggser\ itself.  By convention, the \aggser\ in its own catalog is
called \lit{this}. The top-level dataset element makes \lit{this}
service the default service for all datasets in the catalog. Notice
that \lit{this} service has a base consisting of an empty string.

One thing about \aggser\ catalogs that may be slightly confusing is
that a \element{dataset} element may define an \opendap\ dataset, or
it may just be a container for other dataset elements. In
\exampleref{ex4}, the dataset element called ``Top level'' is just a
container, because it doesn't have a \tag{urlPath} attribute or
contained \tag{access} elements. The dataset element called "NCEP RUC
Model Output" does define an \opendap\ dataset, because it has a
\tag{urlPath} of \lit{NCEP/RUC}. In this document, we use ``Dataset'' to
mean an \opendap\ dataset, and ``dataset element'' to mean an element in
the XML document.

\indc{Dataset URL}\indc{URL!Dataset}\indc{finding!your data's URL}
The Dataset URL is constructed by concatenating the service \new{base} and 
the dataset \tag{urlPath}:

\begin{vcode}{i}
Dataset URL = service.base + dataset.urlPath
\end{vcode}

Dataset \tag{urlPaths} should thus be relative to the service, and
typically  \emph{not} be absolute URLs. By leaving the \aggser\
\lit{this} service \tag{base} empty, the Dataset URLs become relative to
the catalog XML doc itself. The \aggser\ relies on this, so it's a
very good idea for you to follow this convention.  The important
things to remember are: 

\begin{enumerate}
\item Dataset \tag{urlPaths} should be relative, 
\item dataset \tag{urlPaths} must be unique within an \aggser\
  catalog, and
\item dataset \tag{urlPaths} for aggregated datasets are arbitrary,
  since the \aggser\ is mapping these to the actual files specified in
  the \tag{aggregation} element. You should use meaningful names
  however, since the \tag{urlPath} shows up as the dataset name in the
  \dds .
\end{enumerate}


\section{Mapping Datasets to Internal Files}
\label{agg,mapping}

\indc{Dataset URL}\indc{URL!Dataset}
The Dataset URL specifies the external name of an \opendap\ dataset.
These are mapped to internal datasets specified in the
\element{fileAccess} elements, which may be other \opendap\ datasets, or
files internally accessible to the \aggser . (For \aggversion\ these must be
NetCDF files, but the \aggser\ could be extended to handle other types
of files.) We will call these internal datasets and files
\new{internal files}, (even though they are not always files)
to avoid overloading the word ``datasets'' any further. The service that the
\aggser\ uses to access these internal files is called the
\new{internal service}, to distinguish from the external
service, which is the \aggser\ itself.

The internal file location is specified by the concatenation of the
internal service base and the \tag{fileAccess} \tag{urlPath}:

\begin{vcode}{sib}
internal file URL = internalService.base + fileAccess.urlPath
\end{vcode}

The internal service is specified by the \tag{serviceName} attribute
of the \tag{aggregation} or \tag{fileAccess} element, as in
\exampleref{ex5}.  The service elements can be placed in any parent
dataset, including the top-level dataset:

\examplelabel{ex5}
\begin{vcode}{xib}
<dataset name="JoinNew Example" urlPath="htnsst">
  <service name="GSO" serviceType="DODS" 
      base="http://opendap.org/cgi-bin/nph-dsp/"/>
  <service name="GSO-derived" serviceType="NetCDF" 
      base="file:///E:/data/grids/GSO/"/>
  <metadata metadataType="Aggregation">
    <aggregation serviceName="GSO" variable="time" type="JoinNew" 
        dateFormat="yyyy/M/d:HH:mm:ss z">
      <fileAccess urlPath="htnsst/1985/k2828.htn" 
          coord="1985/1/1:18:28:28 GMT"/>
      <fileAccess urlPath="htnsst/1985/k1750.htn_d.Z" 
          coord="1985/1/2:18:17:50 GMT"/>
      <fileAccess serviceName="GSO-derived" 
          urlPath="1985/k85003180718.nc" 
          coord="1985/1/3:18:07:18 GMT"/>
    </aggregation>
  </metadata>
</dataset>
\end{vcode}

In \exampleref{ex5}, the first two internal files of the aggregation
come from the ``GSO'' \opendap\ service, and the third from the
``GSO-derived'' NetCDF service.

\section{Using the \aggser\ as a NetCDF server}
\label{agg,simplenetcdf}

\indc{netCDF!serving un-aggregated}\indc{DODS server!using \aggser\
  as}\indc{server!netCDF files (un-aggregated)}
The \aggser\ can also serve NetCDF files as \opendap\ datasets without
aggregating them.  In this case, specifying the internal service is
harder because we don't have an \tag{aggregation} or \tag{fileAccess}
element to specify the \tag{serviceName}, and putting a
\tag{serviceName} attribute on the dataset element would change the
Dataset URL, not the internal file location.  To specify a local
NetCDF file, use a \element{property} element whose
name is \lit{internalService} and whose value is the name of the internal
service to use, as in \exampleref{ex6}.

\examplelabel{ex6}
\begin{vcode}{sib}
...
<service name="local" serviceType="NetCDF" 
    base="file:///E:/data/grids/GSO-derived/"/>
<service name="loco" serviceType="NetCDF" 
    base="file:///E:/data/grids/UFO-derived/"/>
<dataset name="Non-Aggregated Netcdf Files" serviceName="this">
  <property name="internalService" value="local"/>
  <dataset name="RUC/516" urlPath="516_ruc.nc"/>
  <dataset name="RUC/517" urlPath="517_ruc.nc"/>
  <dataset name="RUC/518" urlPath="518_ruc.nc">
    <property name="internalService" value="loco"/>
  </dataset>
</dataset>
\end{vcode}

In \exampleref{ex6}, the first \element{property} element sets up a
default value for the datasets, while the second \element{property}
tag specifies an exception to the default.  The result is that the
first two internal files of the dataset come from the ``local''
NetCDF service, and the third from the ``loco'' NetCDF service.  The
only real difference is the base path in which the \aggser\ looks for
the files.  Setting up a default in this manner is not essential, but
it can save you some typing.

\note{When the \aggser\ is used to serve non-aggregated netCDF files,
  the \tag{urlPath} of the dataset is used in both the external
  Dataset URL \emph{and} the internal File URL.  Therefore
  \emph{dataset URLs for NetCDF files are not arbitrary}, since they
  must map to real files accessible to the \aggser . However the
  \tag{urlPath} must still be unique within the catalog.}

\examplelabel{ex7}
\begin{vcode}{sib}
<catalog name="Example OPeNDAP Aggregation Server Catalog" version="0.6" >
  <dataset name="Top level dataset" dataType="Grid" serviceName="this">
    <service name="local" serviceType="NetCDF" base="file:///MyDataPath"/>
    <property name="internalService" value="local"/>
    ...
  </dataset>
</catalog>
\end{vcode}

\section{Dataset Aliases}

If you want to refer to the same dataset in more than one place in the
catalog, use an \tag{alias} attribute, so that you don't have to
maintain the aggregation element in more than one place, as in
\exampleref{ex8}: 

\examplelabel{ex8}
\begin{vcode}{sib}
<dataset name="NCEP RUC Model Output" urlPath="NCEP/RUC" ID="NCEP-RUC">
  <service name="localGrids" serviceType="NetCDF" 
    base="file:///E:/data/grids/"/>
  <metadata metadataType="Aggregation">
    <aggregation serviceName="localGrids" varName="valtime" 
        aggType="JoinExisting">
      <fileAccess urlPath="01070516_ruc.nc"/>
      <fileAccess urlPath="01070517_ruc.nc"/>
      <fileAccess urlPath="01070518_ruc.nc"/>
    </aggregation>
  </metadata>
</dataset>      
...
<dataset name="Another way to sort the datasets>
  <dataset name="NCEP RUC Model Output" alias="NCEP-RUC"/>      
  ..
</dataset>
\end{vcode}

\section{Multiple Aggregation elements}

\indc{DDS!aggregating}\indc{DAS!aggregating}
A dataset can contain multiple \element{aggregation} elements. The \aggser\ 
creates a single \dds\ and \das\ by adding the elements of the individual
aggregation's \dds\ and \das , in the order that they are specified. If a
variable or attribute with the same name already exists, then the
duplicate is not added. While this does not allow general dataset
restructuring, it is useful for simple cases of combining datasets.
In the following example:

\begin{vcode}{sib}
<dataset name="Combine Type 2 and Type 3:" urlPath="Type2and3testCombine">
<metadata metadataType="Aggregation">
<aggregation serviceName="local" varName="time" aggType="JoinExisting">
<fileAccess urlPath="cdc/air.1948.nc"/>
<fileAccess urlPath="cdc/air.1949.nc"/>
<fileAccess urlPath="cdc/air.1950.nc"/>
</aggregation>
<aggregation serviceName="local" aggType="Union">
<fileAccess urlPath="cdc/sst.mnmean.nc"/>
</aggregation>
</metadata>
</dataset>
\end{vcode}

A \class{JoinExisting} aggregation that has this \dds :

\begin{vcode}{xib}
Dataset {
  Float32 level[level = 17];
  Float32 lat[lat = 73];
  Float32 lon[lon = 144];
  Float64 time[time = 1096];
  Grid {
    ARRAY:
      Int16 air[time = 1096][level = 17][lat = 73][lon = 144];
    MAPS:
      Float64 time[time = 1096];
      Float32 level[level = 17];
      Float32 lat[lat = 73];
      Float32 lon[lon = 144];
  } air;
} local/MeanAir;
\end{vcode}

and another dataset with this \dds :

\begin{vcode}{xib}
Dataset {
  Float32 lat[lat = 180];
  Float32 lon[lon = 360];
  Float64 time[time = 233];
  Grid {
    ARRAY:
      Int16 sst[time = 233][lat = 180][lon = 360];
    MAPS:
      Float64 time[time = 233];
      Float32 lat[lat = 180];
      Float32 lon[lon = 360];
  } sst;
} SST;
\end{vcode}

are joined to create the following \dds :

\begin{vcode}{xib}
Dataset {
  Float32 level[level = 17];
  Float32 lat[lat = 73];
  Float32 lon[lon = 144];
  Float64 time[time = 1096];
  Grid {
    ARRAY:
      Int16 air[time = 1096][level = 17][lat = 73][lon = 144];
    MAPS:
      Float64 time[time = 1096];
      Float32 level[level = 17];
      Float32 lat[lat = 73];
      Float32 lon[lon = 144];
  } air;
  Grid {
    ARRAY:
      Int16 sst[time = 233][lat = 180][lon = 360];
    MAPS:
      Float64 time[time = 233];
      Float32 lat[lat = 180];
      Float32 lon[lon = 360];
  } sst;
} Type2and3testCombine;
\end{vcode}

Note that the sst \class{Grid} (and maps) are added to the combined
dataset, but the top-level variables \lit{lat}, \lit{lon}, and
\lit{time} are all taken only from the \class{JoinNew} dataset, since
it was specified first.

\chapter{Configuration Elements and Attributes Specification}

The \aggser\ \dtd\ is a specialization of the \thredds\ catalog \dtd .
This chapter contains a a brief summary of the \aggser\ \dtd\ 
followed by a description of the \thredds\ catalog \dtd .

\section{\aggser\ Configuration Elements}
\label{agg,xml-elements}

The \aggser\ XML \dtd\ adds four elements to the \thredds\ \dtd : an
\element{aggregation} element, and the \element{fileAccess}, \element{variable},
and \element{fileScan} elements it depends on.

These two lines appear at the top of the \aggser\ \dtd\ to include
the \thredds\ \dtd .

\begin{vcode}{sib}
<!ENTITY % catalogDTD SYSTEM "InvCatalog.0.6.dtd">
%catalogDTD;
\end{vcode}

This is what makes the \aggser\ a specialization of the \thredds\
catalog server.


\ELEMENT{aggregation}

\begin{vcode}{sib}
<!ENTITY % AggregationType "JoinNew | Union | JoinExisting">
<!ENTITY % VariableType "byte | short | int | float | double | String">

<!ELEMENT aggregation (fileAccess+, variable*, fileScan?)>
<!ATTLIST aggregation
    aggType (%AggregationType;) #REQUIRED
    serviceName CDATA #IMPLIED
    varName CDATA #IMPLIED
    varType (%VariableType;) #IMPLIED
    varUnit CDATA #IMPLIED
    dateFormat CDATA #IMPLIED
>
\end{vcode}

Use the \element{aggregation} element to define the collections of files
to be aggregated, and the manner in which it is to be done.

An \element{aggregation} element contains one or more \element{fileAccess}
elements, followed by 0 or more \element{variable} elements, followed by
an optional \element{fileScan} element.  (As of \aggser\ version
\aggversion , you should not use the \element{fileScan} element.)

Unlike \class{Union} and \class{JoinExist}, \class{JoinNew} type
aggregations must create a new coordinate variable. The variable name
is specified with the \tag{varName} attribute to the
\element{aggregation}, and the variable's values must be specified in the
\tag{coord} attribute of the \element{fileAccess} elements contained in
this \element{aggregation} element.  The \tag{varType} attribute specifies
the type of the new coordinate variable, while the \tag{varUnit}
attribute specifies the unit string which is added as its attribute.

For example, consider three data files containing satellite
measurements in 1440x720 element arrays.  Aggregating them with the
following catalog entry:

\begin{vcode}{xib}
<aggregation serviceName="GSO" aggType="JoinNew" varName="time" 
    varType="int" varUnit="secs since 0000-01-01 00:00:00" 
    dateFormat="yyyy/M/d:HH:mm:ss z">
  <fileAccess urlPath="qscat/01.dat" coord="0000/1/1:00:00:00 GMT"/>
  <fileAccess urlPath="qscat/02.dat" coord="0000/1/2:00:00:00 GMT"/>
  <fileAccess urlPath="qscat/03.dat" coord="0000/1/3:00:00:00 GMT"/>
</aggregation>
\end{vcode}

will result in this DDS and DAS:

\begin{vcode}{sib}
Dataset {
  Int32 time[time = 3];
  Byte binarydata[time = 3][latitude = 720][longitude = 1440];
} qscat/bmaps;

Attributes {
  time {
    String units "secs since 0000-01-01 00:00:00";
  }
}
\end{vcode}


and a query on the time variable returns:

\begin{vcode}{sib}
time[3]
0, 86400, 172800
\end{vcode}

An \element{aggregation} element appears inside \element{metadata}
elements, which allow content type \lit{ANY}.

These are the possible attributes for an \element{aggregation} element.

\begin{description}

\ATTRIBUTE{aggType}

Required.  One of \class{JoinNew}, \class{JoinExisting}, or
\class{Union}. 

\ATTRIBUTE{serviceName}

The \tag{serviceName} specifies the internal data service to use.  The
name is given by the \tag{name} attribute of the \element{service} element.
It may be overridden by one of the \element{fileAccess} elements (or
supplied, if it is omitted here).  If it is not present in either this
element or in the \element{fileAccess} element, the server will issue
an error.

\ATTRIBUTE{varName}

The \tag{varName} specifies the existing (\class{JoinExisting}) or new
(\class{JoinNew}) coordinate variable to join the files on. It is not
used for \class{Union} type aggregations.

\ATTRIBUTE{varType}

The coordinate values specified in the \tag{coord} attribute of the
\element{fileAccess} elements are converted to the type specified by
\tag{varType}, which must be one of \class{byte}, \class{short},
\class{int}, \class{float}, \class{double} or \class{String}. If
\tag{varType} is not specified, then the coordinates are added as
\class{Strings}.  This attribute is only used in \class{JoinNew}
aggregations.

\ATTRIBUTE{varUnit}

A unit string of the added variable (only for \class{JoinNew}).  This
is added to the DAS of the retrieved data, if possible.

\ATTRIBUTE{dateFormat}

Date-valued coordinates are handled in a special way if the
\tag{dateFormat} attribute is specified. In this case, the
\tag{dateFormat} is the format of the date
coordinate values. This format is defined by
\class{java.text.SimpleDateFormat}.  The coordinate values are first
parsed by \class{SimpleDateFormat} according to the \tag{dateFormat}.
This gives a \lit{long} value in units of ``msecs since Jan 1, 1970''. If
\tag{varUnit} is specified, this value is converted to it using the
\lit{ucar.units} package, and the \tag{varUnit} must therefore be
convertible with "msecs since Jan 1, 1970". If \tag{varUnit} is not
specified, the value is converted into "secs since Jan 1, 1970". If
\tag{varType} is specified, the value is converted to that type. If
not, the value is converted to a double.


\end{description}

\ELEMENT{fileAccess}

\begin{vcode}{sib}
<!ELEMENT fileAccess EMPTY>
<!ATTLIST fileAccess
    urlPath CDATA #REQUIRED
    serviceName CDATA #IMPLIED
    coord CDATA #IMPLIED
>
\end{vcode}

The \element{fileAccess} element specifies a file to be used in this
aggregation.  The \tag{urlPath} must be specified, and is used with
the service to create the internal file's URL.  The service is
specified through the \tag{serviceName} here or in the parent
aggregation element. See \sectionref{agg,mapping} for more
information.

The \element{fileAccess} element is analogous to the \element{access} elements
of the \thredds\ catalogs, except that it specifies netCDF files or
\opendap\ datasets that are used only internally by the \aggser .


The \tag{coord} is used only by \class{JoinNew} aggregations, in order
to specify the coordinate value that this \element{fileAccess} corresponds
to.  The variable type, units, and name are specified with the
\element{aggregation} element.

\begin{description}

\ATTRIBUTE{urlPath}

Relative to the base URL given in the \element{service} element, this is
the path to the data file in question.

\ATTRIBUTE{serviceName}

Use this attribute to nominate the service which is to supply this
file.  If this is omitted, use the \tag{serviceName} nominated by the
\element{aggregation} element.  If that one is missing, use \lit{this}.

\ATTRIBUTE{coord}

The \emph{value} of the coordinate variable defined in the
\element{aggregation} element.  This is only relevant for \class{JoinNew}
aggregations. 

\end{description}


\ELEMENT{variable}

\begin{vcode}{sib}
<!ELEMENT variable EMPTY>
<!ATTLIST variable
        name CDATA #REQUIRED
>
\end{vcode}

In a \class{JoinNew} aggregation, all \class{Grids} will be joined,
automatically.  Any variables of type \class{Array} will be joined
only if they are specifically nominated by a \element{variable} element.

\begin{description}

\ATTRIBUTE{name}

The name of the \class{Array} variable to be joined.  See
\sectionref{agg,joinnew} for an example.

\end{description}

\ELEMENT{fileScan}

\note{Do not use this element.}

\begin{vcode}{sib}

<!ELEMENT fileScan EMPTY>
<!ATTLIST fileScan
    urlPath CDATA #REQUIRED
    scanMin CDATA #IMPLIED
>
\end{vcode}


\section{\thredds\ Catalog Configuration Elements}
\label{agg,thredds-elements}

These XML elements are part of the \thredds\ catalog server \dtd .
They are inherited by the \opendap\ \aggser\ \dtd , which is based on
the \thredds\ server.


\ELEMENT{access}

\begin{vcode}{sib}
<!ELEMENT access EMPTY>
<!ATTLIST access
    urlPath CDATA #REQUIRED
    serviceName CDATA #IMPLIED
    serviceType (%ServiceType;) #IMPLIED
>
\end{vcode}

An \element{access} element specifies how a dataset can be accessed
through a data service. It is typically used when there is more than
one service available for a dataset.

Typically a \tag{serviceName} is specified, which is the name of a
\element{service} element in a parent element of the same catalog.
Note it may not refer to a \element{service} in another catalog
referred to by a \element{catalogRef} element. The dataset URL is then
formed from the service \tag{base} and the access \tag{urlPath}, and
optionally the service \tag{suffix} (see \sectionref{agg,dataseturls}).


If a \tag{serviceName} is not specified, a \tag{serviceType} must be
specified, which creates an "anonymous service" of that type. In this
case the \tag{urlPath} must be absolute.


\ELEMENT{catalog}

\begin{vcode}{sib}
<!ELEMENT catalog (dataset) >
<!ATTLIST catalog
    name CDATA #REQUIRED
    version CDATA #REQUIRED
    xmlns:xlink CDATA #FIXED "http://www.w3.org/1999/xlink"
    xmlns CDATA #FIXED "http://www.unidata.ucar.edu/thredds"
>
\end{vcode}

This is the top-level element. A \element{catalog} element contains
exactly one top-level \element{dataset}. The name of the catalog
should be displayed to the user when selecting among catalogs. The
\tag{version} allows DTD migration and should be set to \lit{0.6}.

The XLink and default namespaces are declared here, so technically
they do not have to be declared in the catalog XML itself. However
Internet Explorer cannot deal with namespaces declared in the DTD, so
you should add the same two namespace declarations in the catalog
element in the XML document itself (see \xlink*{this
  example}[http://www.unidata.ucar.edu/projects/THREDDS/xml/InvCatalog.0.6d.xml]{http://www.unidata.ucar.edu/projects/THREDDS/xml/InvCatalog.0.6d.xml}
\htmlonly{This allows you to view the catalog in the IE browser. Netscape
Navigator cannot yet view XML files (as of version 6.2.1).}

 
\ELEMENT{catalogRef}

\begin{vcode}{sib}
<!ELEMENT catalogRef EMPTY>
<!ATTLIST catalogRef
    xlink:type (simple) #FIXED "simple"
    xlink:href CDATA #REQUIRED
    xlink:title CDATA #REQUIRED
>
\end{vcode}

A \element{catalogRef} element refers to another catalog that becomes
a \element{dataset} inside this catalog. This is used to seperately
maintain catalogs and to break up large catalogs. The referenced
catalog should not be read until the user explicitly requests it, so
that very large dataset collections can be represented with
\element{catalogRef} elements without large delays in presenting them
to the user. The referenced catalog is not textually substituted into
the containing catalog, but remains a self-contained object.  The
referenced catalog must be a valid THREDDS catalog, but it does not
have to match versions with the containing catalog.

The value of \tag{xlink:href} is the URL of the referenced catalog.
The value of \tag{xlink:title} is displayed as the name of the dataset
that the user can click on to follow the XLink. Note that the XLink
has a fixed type of "simple" that is part of the DTD, so does not have
to be specified in the catalog XML.

The dataset chooser software should seamlessly present a
\element{catalogRef} to the user, for example by eliminating the
referenced catalog's top-level dataset in its presentation of the
catalog when its name matches the title of the catalogRef title
attribute.

 
\ELEMENT{dataset}

\begin{vcode}{sib}
<!ENTITY % DataType "Grid | Image | Station">

<!ELEMENT dataset (service*, (documentation | metadata | property)*, access*, (dataset | catalogRef)*)>
<!ATTLIST dataset
    name CDATA #REQUIRED
    dataType (%DataType;) #IMPLIED
    authority CDATA #IMPLIED
    ID ID #IMPLIED
    alias IDREF #IMPLIED
    serviceName CDATA #IMPLIED
    urlPath CDATA #IMPLIED
>
\end{vcode}

A \element{dataset} element represents a logical set of data at a
level of granularity appropriate for presentation to a user. A dataset
is \emph{selectable} if it contains at least one access path,
otherwise it is just a container for nested datasets. If selectable,
upon selection, an event is sent to the client software.

A \element{dataset} element contains 0 or more \element{service}
elements followed by 0 or more \element{documentation},
\element{metadata}, or \element{property} elements in any order,
followed by 0 or more \element{access} elements, followed by 0 or more
nested \element{dataset} or \element{catalogRef} elements.  The data
represented by a nested \element{dataset} element should be a subset,
a specialization or in some other sense "contained" within the data
represented by its parent \element{dataset} element.

A dataset must have one or more access paths, specified implicitly
through a \tag{urlPath} attribute, or explicitly in contained
\element{access} elements.  An access path should be thought of as a
URL, but its actually information from which a protocol-aware layer
can construct URLs.  When there is only one URL, this is typically
specified in the \element{dataset} element itself. When there are
multiple URLs, these may be specified in the \element{dataset} element
and/or in contained \element{access} elements.  Multiple URLs specify
different services for accessing the dataset.  Choices among these
different services should be filterered by client software or
presented to the user for selection.  A URL specified in the dataset
element itself is the default URL, which should be the preferred URL
when no filtering or user choice is possible. Also see forming URLs.

A dataset may have a \tag{dataType}, specified within itself or in a
containing \element{aggregation}, whose value comes from a controlled
vocabulary.

If a \element{dataset} has an \tag{alias} attribute, the value of the
attribute must be an ID of another \element{dataset} within the same
catalog. Note it may not refer to a \element{dataset} in another
catalog referred to by a \element{catalogRef} element. In this case,
any other properties of the dataset are ignored, and the dataset to
which the alias refers is used in its place.

A dataset may have a \tag{authority} specified within itself or in a
containing \element{aggregation}.  If a dataset has an \tag{ID} and a
\tag{authority} attribute, then the combination of the two should be
globally unique for all time. If the same dataset is specified in
multiple catalogs, then its \tag{authority} - \tag{ID} should be
identical if possible.

Many of the properties of a dataset become the default for contained
\element{dataset} elements.  This includes \element{property} elements, and
\tag{dataType}, \tag{authority} and \tag{serviceName} attributes. Any
\element{documentation} elements are displayed at the dataset itself
when presenting the catalog to the user. Any \element{metadata}
elements apply to all contained datasets.

 
\ELEMENT{documentation}

\begin{vcode}{sib}
<!ELEMENT documentation (#PCDATA)>
<!ATTLIST documentation
    xlink:type (simple) #FIXED "simple"
    xlink:href CDATA #IMPLIED
    xlink:title CDATA #IMPLIED
    xlink:show (new | replace | embed) "new"
>
\end{vcode}

A \element{documentation} element contains or refers to content that
should be displayed to an end-user when making selections from the
catalog. The content may be HTML or plain text. We call this kind of
content "human readable" information.

The \element{documentation} element may contain arbitrary plain text
content, which should be displayed inline at the position of the
\element{aggregation} or the \element{dataset} element that contains
it.

The \element{documentation} element may also contain an XLink to an
HTML or plain text web page. This text should be either shown inline
or displayed when the user activates the XLink, depending on the value
of the \tag{xlink:show} attribute, whose default is \lit{new}. If the
value of \tag{xlink:show} is \lit{new}, then the content of the XLink
should be displayed in a new window when the user selects it. If the
value of \tag{xlink:show} is \lit{embed}, then the context should be
displayed inline, as if it was text content in the documentation
element.  If the value of \tag{xlink:show} is \lit{replace}, the
content should replace the existing window. The value of
\tag{xlink:title} is used for \lit{show} and \lit{replace}, and should
be the displayed as the name that the user can click on to follow the
XLink. The value of \tag{xlink:show} and \tag{xlink:title} are
heuristics for the dataset choosing widget, which may not be able to
fully implement them. These heuristics are intended to follow the
\xlink{XLink specification}[(see
http://www.w3.org/TR/xlink/)]{http://www.w3.org/TR/xlink/} as closely
as possible. Note that the XLink has a fixed type of \lit{simple} that
is part of the DTD, so does not have to be specified in the XML.

 
\ELEMENT{metadata}

\begin{vcode}{sib}
<!ENTITY % MetadataType "THREDDS | ADN | Aggregation | DublinCore |
         % DIF | FGDC | LAS | Other">

<!ELEMENT metadata ANY>
<!ATTLIST metadata
    xlink:type (simple) #FIXED "simple"
    xlink:href CDATA #IMPLIED
    metadataType (%MetadataType;) #REQUIRED
>
\end{vcode}

A \element{metadata} element contains or refers to structured
information about datasets, which is used by client programs to
properly display or search for the dataset.  Typically, metadata is
not displayed to an end-user when making selections from the catalog,
although it may be useful to make it optionally available. We call
this kind of content "machine readable" information.

The \element{metadata} element must contain a \tag{metadataType}
attribute whose value comes from a controlled vocabulary. The types
and formats of the metadata are still being developed, and the current
list should be considered experimental. Most are currently not
operational.

\begin{description}
\item[\thredds] a/k/a ``Dataset Description''
\item[ADN] Alexandria / DLESE format
\item[Aggregation] DODS/\opendap\ Aggregation Server
\item[DublinCore] Dublin Core
\item[DIF] NASA's Global Change Master Directory (GCMD) format
\item[FGDC] Federal Geographic Data Committee
\item[LAS] Live Access Server
\end{description}

The metadata content may be placed in the \element{metadata} element
itself, or it may be pointed to through an XLink, but it may not have
both. Generally when the metadata is referenced by an XLink, the
information is not read until explicitly requested.


\ELEMENT{property}

\begin{vcode}{sib}
<!ELEMENT property EMPTY>
<!ATTLIST property
   name CDATA #REQUIRED
   value CDATA #REQUIRED
>
\end{vcode}

Property elements are arbitrary name/value pairs to associate with a
dataset, collection or service elements. They will be used to create
extended semantics, and should be available to client applications,
but not typically displayed during dataset selection. Currently they
have no specified semantics.

\ELEMENT{service}

\begin{vcode}{sib}
<!ENTITY % ServiceType "DODS | ADDE | NetCDF | Catalog | FTP | WMS |
         % WFS | WCS | WSDL | Compound | Other">

<!ELEMENT service (property*, service*)>
<!ATTLIST service
    name CDATA #REQUIRED
    serviceType (%ServiceType;) #REQUIRED
    base CDATA #REQUIRED
    suffix CDATA #IMPLIED
>
\end{vcode}


A \element{service} element represents a data service. It must contain
a \tag{name} and a \tag{serviceType} attribute whose value comes from
a controlled vocabulary. It must contain a \tag{name} unique within
the catalog (note that catalogs referenced by a \element{catalogRef}
contain their own \tag{ID} namespaces). It must have a \tag{base}
attribute and may have an optional \tag{suffix} atribute which are
used to construct the dataset URL (see constructing URLS). A
\tag{service} element may contain 0 or more \element{property}
elements. These property elements are made available to the
application when a dataset is selected, but are not otherwise used.

The scope of a \element{service} element is its sibling elements and
their descendents, excluding catalogs referenced by
\element{catalogRef} elements.  The service \tag{name} should be
unique within its scope.

A \element{service} element with \tag{serviceType} equal to
\lit{Compound} must have nested service elements, and services with
type other than \lit{Compound} may not have nested \element{service}
elements. Nested \element{service} elements may be used directly by
\element{dataset} or \element{access} elements. They are at the same
scoping level as their parent \element{service}.

Each \element{dataset} element must refer to one or more
\element{service} elements that appear in a parent collection. Since
typically there will be only a few \element{service} elements in a
catalog but many \element{dataset} elements, a \element{service}
element factors out the common properties of the data service for
efficient representation within the catalog.


\begin{comment}

\section{Validation}

<ul>
<li>
<b>Dataset (1) <datasetName>: has unknown service named <serviceName></b></li>

<ul>
<li>
<datasetName> declares a service <serviceName> that cannot be found.
(FATAL)</li>
</ul>

<li>
<b>Dataset (2) <datasetName>: is selectable but no data type declared
in it or in a parent element</b></li>

<ul>
<li>
no dataType attribute was declared in a selectable dataset element or in
any parent (WARN)</li>
</ul>

<li>
<b>Dataset (3) <datasetName>: is not selectable and does not have nested
datasets</b></li>

<ul>
<li>
this dataset has no use (WARN)</li>
</ul>

<li>
<b>Dataset Access (1) <datasetName>: has unknown service named <ServiceName></b></li>

<ul>
<li>
cannot find a service of the given name within a parent element (FATAL)</li>
</ul>

<li>
<b>Dataset Access (2) <datasetName>: cannot declare service <ServiceName>
and serviceType <ServiceTypeName></b></li>

<ul>
<li>
the ServiceType will be ignored (WARN)</li>
</ul>

<li>
<b>Dataset Access (3) <datasetName>: urlPath bad syntax <urlPath></b></li>

<ul>
<li>
urlPath could not be parsed as a URI reference (FATAL)</li>
</ul>

<li>
<b>Dataset Access (4) <datasetName>: urlPath must be absolute <urlPath></b></li>

<ul>
<li>
urlPath must be absolute when you create an "anonymous" service by specifying
a serviceType but no serviceName (FATAL)</li>
</ul>

<li>
<b>Dataset Access (5) <datasetName>: has access <urlPath> with no
valid service</b></li>

<ul>
<li>
no valid service is declared in access element or parent dataset.</li>
</ul>

<li>
<b>InvCatalogFactory.readXML (1) MalformedURLException on URL <urlPath>
<message></b></li>

<ul>
<li>
malformed URL</li>
</ul>

<li>
<b>InvCatalogFactory.readXML (2) cant open catalog; response = <http
response code> <message></b></li>

<ul>
<li>
URL does not exist</li>
</ul>

<li>
<b>InvCatalogFactory.readXML (3) IOException on catalog <message></b></li>

<ul>
<li>
probably a network or web server error</li>
</ul>

<li>
<b>InvCatalogFactory.readXML (4) cant find 'version' attribute in catalog</b></li>

<ul>
<li>
catalog element must have a "version" attribute</li>
</ul>

<li>
<b>InvCatalogFactory.readXML (5) No factory for version <version></b></li>

<ul>
<li>
library cannot read that version of catalog docs</li>
</ul>

<li>
<b>InvCatalogFactory6 catalog DTD is <DTD URL> must be <http://www.unidata.ucar.edu/projects/THREDDS/xml/InvCatalog.0.6.dtd></b></li>

<ul>
<li>
version 6 catalogs must use the named standard DTD</li>
</ul>

<li>
<b>Metadata (1)  href = <XLink URL>: MalformedURLException <message></b></li>

<ul>
<li>
The specified URL has incorrect syntax</li>
</ul>

<li>
<b>Metadata (2)  href = <XLink URL>: IOException <message></b></li>

<ul>
<li>
Error reading the specified URL.</li>
</ul>

<li>
<b>Service (1) <ServiceName> type COMPOUND must have a nested service</b></li>

<ul>
<li>
a compound service cannot be used without nested services.</li>
</ul>

<li>
<b>Service (2) <ServiceName> type <ServiceTypeName> may not have
nested services</b></li>

<ul>
<li>
non-compound service cannot have nested services.</li>
</ul>
</ul>

\end{comment}

%%% Local Variables: 
%%% mode: latex
%%% TeX-master: t
%%% End: 


\appendix
\chapter{Acronyms}

These acronyms are used frequently in this manual.

\begin{acronym}
  \acro{DAS}{Data Attribute Structure} A description of the
  ``metadata'' attributes for an \opendap\ dataset.  See DODS.
  \acro{DDS}{Data Descriptor Structure} A description of the data
  types and sizes of data in an \opendap\ dataset.  See DODS.
  \acro{DODS}{Distributed Oceanographic Data System} The ancestor to
  \opendap .  See
  \xlink{unidata.ucar.edu/packages/dods}{http://unidata.ucar.edu/packages/dods}
  for information.
  \acro{DTD}{Document Type Definition} This is the set of definitions
  that make up an XML specification.
  \acro{OPeNDAP}{Open Source Project for Network Data Access Protocol}
  A network protocol for transmitting data across the internet.
  Though appropriate for many kinds of data, the system was designed
  with scientific data in mind.  See
  \xlink{www.nvods.org}{http://www.nvods.org} for information about
  the group that has developed \opendap\ and
  \xlink{unidata.ucar.edu/packages/dods}{http://unidata.ucar.edu/packages/dods}
  for information about \opendap 's progenitor, DODS.
  \acro{THREDDS}{Thematic Realtime Environmental Data Distributed
  Services}  A project of Unidata to create ways to establish useful
  collections of earth science data.  See
  \xlink{unidata.ucar.edu/projects/THREDDS}{http://unidata.ucar.edu/projects/THREDDS}
  for more information.
\end{acronym}

\printindex

\end{document}

%% $Log: agg.tex,v $
%% Revision 1.5  2004/07/07 22:50:46  jimg
%% Switched to dods-book, rcsInfoDate and updated PDF link.
%%
%% Revision 1.4  2004/04/24 21:37:23  jimg
%% I added every directory in preparation for adding everyting. This is
%% part of getting the opendap web pages going...
%%
%% Revision 1.3  2003/12/28 21:27:02  tom
%% added hlx material
%%
%% Revision 1.2  2003/12/28 21:23:43  tom
%% added log
%%

%%% Local Variables: 
%%% mode: latex
%%% TeX-master: t
%%% End: 
