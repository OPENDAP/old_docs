\chapter{Configuration Elements and Attributes Specification}

The \aggser\ \dtd\ is a specialization of the \thredds\ catalog \dtd .
This chapter contains a a brief summary of the \aggser\ \dtd\ 
followed by a description of the \thredds\ catalog \dtd .

\section{\aggser\ Configuration Elements}
\label{agg,xml-elements}

The \aggser\ XML \dtd\ adds four elements to the \thredds\ \dtd : an
\element{aggregation} element, and the \element{fileAccess}, \element{variable},
and \element{fileScan} elements it depends on.

These two lines appear at the top of the \aggser\ \dtd\ to include
the \thredds\ \dtd .

\begin{vcode}{sib}
<!ENTITY % catalogDTD SYSTEM "InvCatalog.0.6.dtd">
%catalogDTD;
\end{vcode}

This is what makes the \aggser\ a specialization of the \thredds\
catalog server.


\ELEMENT{aggregation}

\begin{vcode}{sib}
<!ENTITY % AggregationType "JoinNew | Union | JoinExisting">
<!ENTITY % VariableType "byte | short | int | float | double | String">

<!ELEMENT aggregation (fileAccess+, variable*, fileScan?)>
<!ATTLIST aggregation
    aggType (%AggregationType;) #REQUIRED
    serviceName CDATA #IMPLIED
    varName CDATA #IMPLIED
    varType (%VariableType;) #IMPLIED
    varUnit CDATA #IMPLIED
    dateFormat CDATA #IMPLIED
>
\end{vcode}

Use the \element{aggregation} element to define the collections of files
to be aggregated, and the manner in which it is to be done.

An \element{aggregation} element contains one or more \element{fileAccess}
elements, followed by 0 or more \element{variable} elements, followed by
an optional \element{fileScan} element.  (As of \aggser\ version
\aggversion , you should not use the \element{fileScan} element.)

Unlike \class{Union} and \class{JoinExist}, \class{JoinNew} type
aggregations must create a new coordinate variable. The variable name
is specified with the \tag{varName} attribute to the
\element{aggregation}, and the variable's values must be specified in the
\tag{coord} attribute of the \element{fileAccess} elements contained in
this \element{aggregation} element.  The \tag{varType} attribute specifies
the type of the new coordinate variable, while the \tag{varUnit}
attribute specifies the unit string which is added as its attribute.

For example, consider three data files containing satellite
measurements in 1440x720 element arrays.  Aggregating them with the
following catalog entry:

\begin{vcode}{xib}
<aggregation serviceName="GSO" aggType="JoinNew" varName="time" 
    varType="int" varUnit="secs since 0000-01-01 00:00:00" 
    dateFormat="yyyy/M/d:HH:mm:ss z">
  <fileAccess urlPath="qscat/01.dat" coord="0000/1/1:00:00:00 GMT"/>
  <fileAccess urlPath="qscat/02.dat" coord="0000/1/2:00:00:00 GMT"/>
  <fileAccess urlPath="qscat/03.dat" coord="0000/1/3:00:00:00 GMT"/>
</aggregation>
\end{vcode}

will result in this DDS and DAS:

\begin{vcode}{sib}
Dataset {
  Int32 time[time = 3];
  Byte binarydata[time = 3][latitude = 720][longitude = 1440];
} qscat/bmaps;

Attributes {
  time {
    String units "secs since 0000-01-01 00:00:00";
  }
}
\end{vcode}


and a query on the time variable returns:

\begin{vcode}{sib}
time[3]
0, 86400, 172800
\end{vcode}

An \element{aggregation} element appears inside \element{metadata}
elements, which allow content type \lit{ANY}.

These are the possible attributes for an \element{aggregation} element.

\begin{description}

\ATTRIBUTE{aggType}

Required.  One of \class{JoinNew}, \class{JoinExisting}, or
\class{Union}. 

\ATTRIBUTE{serviceName}

The \tag{serviceName} specifies the internal data service to use.  The
name is given by the \tag{name} attribute of the \element{service} element.
It may be overridden by one of the \element{fileAccess} elements (or
supplied, if it is omitted here).  If it is not present in either this
element or in the \element{fileAccess} element, the server will issue
an error.

\ATTRIBUTE{varName}

The \tag{varName} specifies the existing (\class{JoinExisting}) or new
(\class{JoinNew}) coordinate variable to join the files on. It is not
used for \class{Union} type aggregations.

\ATTRIBUTE{varType}

The coordinate values specified in the \tag{coord} attribute of the
\element{fileAccess} elements are converted to the type specified by
\tag{varType}, which must be one of \class{byte}, \class{short},
\class{int}, \class{float}, \class{double} or \class{String}. If
\tag{varType} is not specified, then the coordinates are added as
\class{Strings}.  This attribute is only used in \class{JoinNew}
aggregations.

\ATTRIBUTE{varUnit}

A unit string of the added variable (only for \class{JoinNew}).  This
is added to the DAS of the retrieved data, if possible.

\ATTRIBUTE{dateFormat}

Date-valued coordinates are handled in a special way if the
\tag{dateFormat} attribute is specified. In this case, the
\tag{dateFormat} is the format of the date
coordinate values. This format is defined by
\class{java.text.SimpleDateFormat}.  The coordinate values are first
parsed by \class{SimpleDateFormat} according to the \tag{dateFormat}.
This gives a \lit{long} value in units of ``msecs since Jan 1, 1970''. If
\tag{varUnit} is specified, this value is converted to it using the
\lit{ucar.units} package, and the \tag{varUnit} must therefore be
convertible with "msecs since Jan 1, 1970". If \tag{varUnit} is not
specified, the value is converted into "secs since Jan 1, 1970". If
\tag{varType} is specified, the value is converted to that type. If
not, the value is converted to a double.


\end{description}

\ELEMENT{fileAccess}

\begin{vcode}{sib}
<!ELEMENT fileAccess EMPTY>
<!ATTLIST fileAccess
    urlPath CDATA #REQUIRED
    serviceName CDATA #IMPLIED
    coord CDATA #IMPLIED
>
\end{vcode}

The \element{fileAccess} element specifies a file to be used in this
aggregation.  The \tag{urlPath} must be specified, and is used with
the service to create the internal file's URL.  The service is
specified through the \tag{serviceName} here or in the parent
aggregation element. See \sectionref{agg,mapping} for more
information.

The \element{fileAccess} element is analogous to the \element{access} elements
of the \thredds\ catalogs, except that it specifies netCDF files or
\opendap\ datasets that are used only internally by the \aggser .


The \tag{coord} is used only by \class{JoinNew} aggregations, in order
to specify the coordinate value that this \element{fileAccess} corresponds
to.  The variable type, units, and name are specified with the
\element{aggregation} element.

\begin{description}

\ATTRIBUTE{urlPath}

Relative to the base URL given in the \element{service} element, this is
the path to the data file in question.

\ATTRIBUTE{serviceName}

Use this attribute to nominate the service which is to supply this
file.  If this is omitted, use the \tag{serviceName} nominated by the
\element{aggregation} element.  If that one is missing, use \lit{this}.

\ATTRIBUTE{coord}

The \emph{value} of the coordinate variable defined in the
\element{aggregation} element.  This is only relevant for \class{JoinNew}
aggregations. 

\end{description}


\ELEMENT{variable}

\begin{vcode}{sib}
<!ELEMENT variable EMPTY>
<!ATTLIST variable
        name CDATA #REQUIRED
>
\end{vcode}

In a \class{JoinNew} aggregation, all \class{Grids} will be joined,
automatically.  Any variables of type \class{Array} will be joined
only if they are specifically nominated by a \element{variable} element.

\begin{description}

\ATTRIBUTE{name}

The name of the \class{Array} variable to be joined.  See
\sectionref{agg,joinnew} for an example.

\end{description}

\ELEMENT{fileScan}

\note{Do not use this element.}

\begin{vcode}{sib}

<!ELEMENT fileScan EMPTY>
<!ATTLIST fileScan
    urlPath CDATA #REQUIRED
    scanMin CDATA #IMPLIED
>
\end{vcode}


\section{\thredds\ Catalog Configuration Elements}
\label{agg,thredds-elements}

These XML elements are part of the \thredds\ catalog server \dtd .
They are inherited by the \opendap\ \aggser\ \dtd , which is based on
the \thredds\ server.


\ELEMENT{access}

\begin{vcode}{sib}
<!ELEMENT access EMPTY>
<!ATTLIST access
    urlPath CDATA #REQUIRED
    serviceName CDATA #IMPLIED
    serviceType (%ServiceType;) #IMPLIED
>
\end{vcode}

An \element{access} element specifies how a dataset can be accessed
through a data service. It is typically used when there is more than
one service available for a dataset.

Typically a \tag{serviceName} is specified, which is the name of a
\element{service} element in a parent element of the same catalog.
Note it may not refer to a \element{service} in another catalog
referred to by a \element{catalogRef} element. The dataset URL is then
formed from the service \tag{base} and the access \tag{urlPath}, and
optionally the service \tag{suffix} (see \sectionref{agg,dataseturls}).


If a \tag{serviceName} is not specified, a \tag{serviceType} must be
specified, which creates an "anonymous service" of that type. In this
case the \tag{urlPath} must be absolute.


\ELEMENT{catalog}

\begin{vcode}{sib}
<!ELEMENT catalog (dataset) >
<!ATTLIST catalog
    name CDATA #REQUIRED
    version CDATA #REQUIRED
    xmlns:xlink CDATA #FIXED "http://www.w3.org/1999/xlink"
    xmlns CDATA #FIXED "http://www.unidata.ucar.edu/thredds"
>
\end{vcode}

This is the top-level element. A \element{catalog} element contains
exactly one top-level \element{dataset}. The name of the catalog
should be displayed to the user when selecting among catalogs. The
\tag{version} allows DTD migration and should be set to \lit{0.6}.

The XLink and default namespaces are declared here, so technically
they do not have to be declared in the catalog XML itself. However
Internet Explorer cannot deal with namespaces declared in the DTD, so
you should add the same two namespace declarations in the catalog
element in the XML document itself (see \xlink*{this
  example}[http://www.unidata.ucar.edu/projects/THREDDS/xml/InvCatalog.0.6d.xml]{http://www.unidata.ucar.edu/projects/THREDDS/xml/InvCatalog.0.6d.xml}
\htmlonly{This allows you to view the catalog in the IE browser. Netscape
Navigator cannot yet view XML files (as of version 6.2.1).}

 
\ELEMENT{catalogRef}

\begin{vcode}{sib}
<!ELEMENT catalogRef EMPTY>
<!ATTLIST catalogRef
    xlink:type (simple) #FIXED "simple"
    xlink:href CDATA #REQUIRED
    xlink:title CDATA #REQUIRED
>
\end{vcode}

A \element{catalogRef} element refers to another catalog that becomes
a \element{dataset} inside this catalog. This is used to seperately
maintain catalogs and to break up large catalogs. The referenced
catalog should not be read until the user explicitly requests it, so
that very large dataset collections can be represented with
\element{catalogRef} elements without large delays in presenting them
to the user. The referenced catalog is not textually substituted into
the containing catalog, but remains a self-contained object.  The
referenced catalog must be a valid THREDDS catalog, but it does not
have to match versions with the containing catalog.

The value of \tag{xlink:href} is the URL of the referenced catalog.
The value of \tag{xlink:title} is displayed as the name of the dataset
that the user can click on to follow the XLink. Note that the XLink
has a fixed type of "simple" that is part of the DTD, so does not have
to be specified in the catalog XML.

The dataset chooser software should seamlessly present a
\element{catalogRef} to the user, for example by eliminating the
referenced catalog's top-level dataset in its presentation of the
catalog when its name matches the title of the catalogRef title
attribute.

 
\ELEMENT{dataset}

\begin{vcode}{sib}
<!ENTITY % DataType "Grid | Image | Station">

<!ELEMENT dataset (service*, (documentation | metadata | property)*, access*, (dataset | catalogRef)*)>
<!ATTLIST dataset
    name CDATA #REQUIRED
    dataType (%DataType;) #IMPLIED
    authority CDATA #IMPLIED
    ID ID #IMPLIED
    alias IDREF #IMPLIED
    serviceName CDATA #IMPLIED
    urlPath CDATA #IMPLIED
>
\end{vcode}

A \element{dataset} element represents a logical set of data at a
level of granularity appropriate for presentation to a user. A dataset
is \emph{selectable} if it contains at least one access path,
otherwise it is just a container for nested datasets. If selectable,
upon selection, an event is sent to the client software.

A \element{dataset} element contains 0 or more \element{service}
elements followed by 0 or more \element{documentation},
\element{metadata}, or \element{property} elements in any order,
followed by 0 or more \element{access} elements, followed by 0 or more
nested \element{dataset} or \element{catalogRef} elements.  The data
represented by a nested \element{dataset} element should be a subset,
a specialization or in some other sense "contained" within the data
represented by its parent \element{dataset} element.

A dataset must have one or more access paths, specified implicitly
through a \tag{urlPath} attribute, or explicitly in contained
\element{access} elements.  An access path should be thought of as a
URL, but its actually information from which a protocol-aware layer
can construct URLs.  When there is only one URL, this is typically
specified in the \element{dataset} element itself. When there are
multiple URLs, these may be specified in the \element{dataset} element
and/or in contained \element{access} elements.  Multiple URLs specify
different services for accessing the dataset.  Choices among these
different services should be filterered by client software or
presented to the user for selection.  A URL specified in the dataset
element itself is the default URL, which should be the preferred URL
when no filtering or user choice is possible. Also see forming URLs.

A dataset may have a \tag{dataType}, specified within itself or in a
containing \element{aggregation}, whose value comes from a controlled
vocabulary.

If a \element{dataset} has an \tag{alias} attribute, the value of the
attribute must be an ID of another \element{dataset} within the same
catalog. Note it may not refer to a \element{dataset} in another
catalog referred to by a \element{catalogRef} element. In this case,
any other properties of the dataset are ignored, and the dataset to
which the alias refers is used in its place.

A dataset may have a \tag{authority} specified within itself or in a
containing \element{aggregation}.  If a dataset has an \tag{ID} and a
\tag{authority} attribute, then the combination of the two should be
globally unique for all time. If the same dataset is specified in
multiple catalogs, then its \tag{authority} - \tag{ID} should be
identical if possible.

Many of the properties of a dataset become the default for contained
\element{dataset} elements.  This includes \element{property} elements, and
\tag{dataType}, \tag{authority} and \tag{serviceName} attributes. Any
\element{documentation} elements are displayed at the dataset itself
when presenting the catalog to the user. Any \element{metadata}
elements apply to all contained datasets.

 
\ELEMENT{documentation}

\begin{vcode}{sib}
<!ELEMENT documentation (#PCDATA)>
<!ATTLIST documentation
    xlink:type (simple) #FIXED "simple"
    xlink:href CDATA #IMPLIED
    xlink:title CDATA #IMPLIED
    xlink:show (new | replace | embed) "new"
>
\end{vcode}

A \element{documentation} element contains or refers to content that
should be displayed to an end-user when making selections from the
catalog. The content may be HTML or plain text. We call this kind of
content "human readable" information.

The \element{documentation} element may contain arbitrary plain text
content, which should be displayed inline at the position of the
\element{aggregation} or the \element{dataset} element that contains
it.

The \element{documentation} element may also contain an XLink to an
HTML or plain text web page. This text should be either shown inline
or displayed when the user activates the XLink, depending on the value
of the \tag{xlink:show} attribute, whose default is \lit{new}. If the
value of \tag{xlink:show} is \lit{new}, then the content of the XLink
should be displayed in a new window when the user selects it. If the
value of \tag{xlink:show} is \lit{embed}, then the context should be
displayed inline, as if it was text content in the documentation
element.  If the value of \tag{xlink:show} is \lit{replace}, the
content should replace the existing window. The value of
\tag{xlink:title} is used for \lit{show} and \lit{replace}, and should
be the displayed as the name that the user can click on to follow the
XLink. The value of \tag{xlink:show} and \tag{xlink:title} are
heuristics for the dataset choosing widget, which may not be able to
fully implement them. These heuristics are intended to follow the
\xlink{XLink specification}[(see
http://www.w3.org/TR/xlink/)]{http://www.w3.org/TR/xlink/} as closely
as possible. Note that the XLink has a fixed type of \lit{simple} that
is part of the DTD, so does not have to be specified in the XML.

 
\ELEMENT{metadata}

\begin{vcode}{sib}
<!ENTITY % MetadataType "THREDDS | ADN | Aggregation | DublinCore |
         % DIF | FGDC | LAS | Other">

<!ELEMENT metadata ANY>
<!ATTLIST metadata
    xlink:type (simple) #FIXED "simple"
    xlink:href CDATA #IMPLIED
    metadataType (%MetadataType;) #REQUIRED
>
\end{vcode}

A \element{metadata} element contains or refers to structured
information about datasets, which is used by client programs to
properly display or search for the dataset.  Typically, metadata is
not displayed to an end-user when making selections from the catalog,
although it may be useful to make it optionally available. We call
this kind of content "machine readable" information.

The \element{metadata} element must contain a \tag{metadataType}
attribute whose value comes from a controlled vocabulary. The types
and formats of the metadata are still being developed, and the current
list should be considered experimental. Most are currently not
operational.

\begin{description}
\item[\thredds] a/k/a ``Dataset Description''
\item[ADN] Alexandria / DLESE format
\item[Aggregation] DODS/\opendap\ Aggregation Server
\item[DublinCore] Dublin Core
\item[DIF] NASA's Global Change Master Directory (GCMD) format
\item[FGDC] Federal Geographic Data Committee
\item[LAS] Live Access Server
\end{description}

The metadata content may be placed in the \element{metadata} element
itself, or it may be pointed to through an XLink, but it may not have
both. Generally when the metadata is referenced by an XLink, the
information is not read until explicitly requested.


\ELEMENT{property}

\begin{vcode}{sib}
<!ELEMENT property EMPTY>
<!ATTLIST property
   name CDATA #REQUIRED
   value CDATA #REQUIRED
>
\end{vcode}

Property elements are arbitrary name/value pairs to associate with a
dataset, collection or service elements. They will be used to create
extended semantics, and should be available to client applications,
but not typically displayed during dataset selection. Currently they
have no specified semantics.

\ELEMENT{service}

\begin{vcode}{sib}
<!ENTITY % ServiceType "DODS | ADDE | NetCDF | Catalog | FTP | WMS |
         % WFS | WCS | WSDL | Compound | Other">

<!ELEMENT service (property*, service*)>
<!ATTLIST service
    name CDATA #REQUIRED
    serviceType (%ServiceType;) #REQUIRED
    base CDATA #REQUIRED
    suffix CDATA #IMPLIED
>
\end{vcode}


A \element{service} element represents a data service. It must contain
a \tag{name} and a \tag{serviceType} attribute whose value comes from
a controlled vocabulary. It must contain a \tag{name} unique within
the catalog (note that catalogs referenced by a \element{catalogRef}
contain their own \tag{ID} namespaces). It must have a \tag{base}
attribute and may have an optional \tag{suffix} atribute which are
used to construct the dataset URL (see constructing URLS). A
\tag{service} element may contain 0 or more \element{property}
elements. These property elements are made available to the
application when a dataset is selected, but are not otherwise used.

The scope of a \element{service} element is its sibling elements and
their descendents, excluding catalogs referenced by
\element{catalogRef} elements.  The service \tag{name} should be
unique within its scope.

A \element{service} element with \tag{serviceType} equal to
\lit{Compound} must have nested service elements, and services with
type other than \lit{Compound} may not have nested \element{service}
elements. Nested \element{service} elements may be used directly by
\element{dataset} or \element{access} elements. They are at the same
scoping level as their parent \element{service}.

Each \element{dataset} element must refer to one or more
\element{service} elements that appear in a parent collection. Since
typically there will be only a few \element{service} elements in a
catalog but many \element{dataset} elements, a \element{service}
element factors out the common properties of the data service for
efficient representation within the catalog.


\begin{comment}

\section{Validation}

<ul>
<li>
<b>Dataset (1) <datasetName>: has unknown service named <serviceName></b></li>

<ul>
<li>
<datasetName> declares a service <serviceName> that cannot be found.
(FATAL)</li>
</ul>

<li>
<b>Dataset (2) <datasetName>: is selectable but no data type declared
in it or in a parent element</b></li>

<ul>
<li>
no dataType attribute was declared in a selectable dataset element or in
any parent (WARN)</li>
</ul>

<li>
<b>Dataset (3) <datasetName>: is not selectable and does not have nested
datasets</b></li>

<ul>
<li>
this dataset has no use (WARN)</li>
</ul>

<li>
<b>Dataset Access (1) <datasetName>: has unknown service named <ServiceName></b></li>

<ul>
<li>
cannot find a service of the given name within a parent element (FATAL)</li>
</ul>

<li>
<b>Dataset Access (2) <datasetName>: cannot declare service <ServiceName>
and serviceType <ServiceTypeName></b></li>

<ul>
<li>
the ServiceType will be ignored (WARN)</li>
</ul>

<li>
<b>Dataset Access (3) <datasetName>: urlPath bad syntax <urlPath></b></li>

<ul>
<li>
urlPath could not be parsed as a URI reference (FATAL)</li>
</ul>

<li>
<b>Dataset Access (4) <datasetName>: urlPath must be absolute <urlPath></b></li>

<ul>
<li>
urlPath must be absolute when you create an "anonymous" service by specifying
a serviceType but no serviceName (FATAL)</li>
</ul>

<li>
<b>Dataset Access (5) <datasetName>: has access <urlPath> with no
valid service</b></li>

<ul>
<li>
no valid service is declared in access element or parent dataset.</li>
</ul>

<li>
<b>InvCatalogFactory.readXML (1) MalformedURLException on URL <urlPath>
<message></b></li>

<ul>
<li>
malformed URL</li>
</ul>

<li>
<b>InvCatalogFactory.readXML (2) cant open catalog; response = <http
response code> <message></b></li>

<ul>
<li>
URL does not exist</li>
</ul>

<li>
<b>InvCatalogFactory.readXML (3) IOException on catalog <message></b></li>

<ul>
<li>
probably a network or web server error</li>
</ul>

<li>
<b>InvCatalogFactory.readXML (4) cant find 'version' attribute in catalog</b></li>

<ul>
<li>
catalog element must have a "version" attribute</li>
</ul>

<li>
<b>InvCatalogFactory.readXML (5) No factory for version <version></b></li>

<ul>
<li>
library cannot read that version of catalog docs</li>
</ul>

<li>
<b>InvCatalogFactory6 catalog DTD is <DTD URL> must be <http://www.unidata.ucar.edu/projects/THREDDS/xml/InvCatalog.0.6.dtd></b></li>

<ul>
<li>
version 6 catalogs must use the named standard DTD</li>
</ul>

<li>
<b>Metadata (1)  href = <XLink URL>: MalformedURLException <message></b></li>

<ul>
<li>
The specified URL has incorrect syntax</li>
</ul>

<li>
<b>Metadata (2)  href = <XLink URL>: IOException <message></b></li>

<ul>
<li>
Error reading the specified URL.</li>
</ul>

<li>
<b>Service (1) <ServiceName> type COMPOUND must have a nested service</b></li>

<ul>
<li>
a compound service cannot be used without nested services.</li>
</ul>

<li>
<b>Service (2) <ServiceName> type <ServiceTypeName> may not have
nested services</b></li>

<ul>
<li>
non-compound service cannot have nested services.</li>
</ul>
</ul>

\end{comment}

%%% Local Variables: 
%%% mode: latex
%%% TeX-master: t
%%% End: 
