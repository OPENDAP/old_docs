% $Id$

\chapter{The Catalog Server}
\label{dods-server,catalog}

\indc{server!file}\indc{server!catalog} Some data servers are
responsible for serving data from thousands of data files. Trying to
assert some kind of order to a dataset like this can be a challenge.
You can create a catalog of these data files, which will allow a user
to select among the data files according to variables or variable
ranges.

\subj{A catalog server is just a server that serves URLs.} A
\new{catalog server}, or \new{file server}, is just a collection of
DAP URL's, compiled into a big list, along with some data a user might
use to select among these data files. Typically the \new{selection
  data} is data not included in the files themselves, but is data
\emph{about} the file. For example, the URI AVHRR archive records the
time each file was recorded next to the file's name.

The good thing about catalog servers is that they are just vanilla
data servers, but used to serve information about data, rather than
the data themselves. The only requirement is that they serve Sequence
data, and that one of the fields be called \lit{DODS\_URL}. This means
that the procedures you've used to get your data served are exactly
the same procedures you will use to get your data catalog served. The
steps that are already done will not need to be done again.

Here are the steps in the process.

\section{Step One: Install the FreeForm Data Handler}

The catalog server typically uses the FreeForm handler, though it
could use the JGOFS handler too, or any handler that can serve
Sequence data. This guide shows how to use the FreeForm handler to
make a catalog server, since that's what most sites choose to do.

\subj{Any handler that serves Sequences can be a catalog server.} If
you haven't done so already, download and install the FreeForm handler
(or whatever server you intend to use). If you're using the FreeForm
handler, move on to the next section. If not, look at the file server
chapter in \DODSquick\ and make your catalog output look like that.

\section{Step Two: Make Your Catalog and Format File}

\subj{Make a catalog text file.}
Once you've got the FreeForm handler installed and working (check it
out with one of the sample data and format files that come with it),
compile a list of all the URLs of all the files you need to serve.
Put this in a text file, along with the selection data you intend to
use, one file per line.

Here's a part of the AVHRR archive catalog file used at URI:

\begin{vcode}{xib}
1979/04/11:18:53:59  http://dods.gso.uri.edu/cgi-bin/nph-dods/avhrr/1979/4/t79101185359.pvu.Z
1979/04/15:19:48:52  http://dods.gso.uri.edu/cgi-bin/nph-dods/avhrr/1979/4/t79105194852.pvu.Z
1979/04/16:19:40:24  http://dods.gso.uri.edu/cgi-bin/nph-dods/avhrr/1979/4/t79106194024.pvu.Z
1979/04/17:08:00:11  http://dods.gso.uri.edu/cgi-bin/nph-dods/avhrr/1979/4/t79107080011.pvu.Z
1979/04/17:09:49:59  http://dods.gso.uri.edu/cgi-bin/nph-dods/avhrr/1979/4/t79107094959.pvu.Z
1979/04/17:19:28:33  http://dods.gso.uri.edu/cgi-bin/nph-dods/avhrr/1979/4/t79107192833.pvu.Z
1979/04/18:07:50:34  http://dods.gso.uri.edu/cgi-bin/nph-dods/avhrr/1979/4/t79108075034.pvu.Z
1979/04/18:20:57:43  http://dods.gso.uri.edu/cgi-bin/nph-dods/avhrr/1979/4/t79108205743.pvu.Z
\end{vcode}

Now you need to create a \new{format file} that describes the layout
of the data file to the FreeForm handler.  The first row of the format
section should read \lit{ASCII\_input\_data}, and be followed by the
name of the Sequence.  (You choose it, it doesn't really matter what
it is.)

\subj{Make a format file to match.}
After the first line, each row of the format file describes another
data value, and the columns that contain it.  The indexes start at
one, and are inclusive.  Here is the format file for the AVHRR archive
shown above:

\begin{vcode}{sib}
ASCII_input_data "URI_Avhrr"
year     1 4    int32 0
month    6 7    int32 0
day      9 10   int32 0
hours    12 13  int32 0
minutes  15 16  int32 0
seconds  18 19  int32 0
DODS_URL 21 92  text  0
\end{vcode}

The \lit{year} variable is in columns one through four, the
\lit{month} variable in columns six and seven and so on.  The data
type is also indicated in the format table.  The last column in the
format table is for scaling.  All numeric data are scaled by the given
power of ten.

\note{The FreeForm format is quite flexible, and can handle many more
  cases than the ones shown.  \DODSffbook\ contains examples and
  complete descriptions of all the FreeForm capabilities, as well as
  tools you can use to check your format files.  Refer to that book
  for more information about the server or writing format files.}

\subj{Install them.  You're done.}
Once you've created a file containing your data files, and created a
format file describing them, you need only find a place to park them.
Give the catalog file some name ending in \lit{.dat}, and the format
file the \emph{same name}, but ending in \lit{.fmt}.  Put them both in
the same directory, somewhere in your document root directory tree,
and test them by requesting a list of URLs.

The URI archive shown above is sampled in the \DODSquick .  That book
explains how to construct a \new{constraint expression}, and gives
examples of them for use with a catalog server.

% $Log: cat-chap.tex,v $
% Revision 1.1  2000/11/30 15:25:12  tom
% added the catalog server chapter as a separate document
%

%%% Local Variables: 
%%% mode: latex
%%% TeX-master: t
%%% End: 
