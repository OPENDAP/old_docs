%CHAPTER 2       
%
% $Id$
%

\chapter{Quick Tour of FreeForm ND}
\label{ff,quick}

This chapter provides you a quick introduction to writing format
descriptions and using several FreeForm ND programs. You will look at
a format description file, convert data from one format to another,
read the data in a binary file, and create summary files. This chapter
introduces only tabular format data (arranged in rows and columns).
See \chapterref{ff,arrayfmt} for information on handling data in
non-tabular formats.

\section{Writing Format Descriptions}

\indc{format descriptions}\indc{descriptions!format}
You can easily create FreeForm ND format description files that
describe the formats of input and output data and headers. FreeForm
ND-based programs then use these files to correctly access and
manipulate data in various formats. An example format description file
is shown and described below.

For complete information about writing format descriptions, see the
next two chapters.

Here is a format description file, included in the sample file set,
called \lit{latlon.fmt}:

\begin{vcode}{sib}
/ This is the format description file for data files latlon.bin 
/ and latlon.dat. Each record in both files contains two fields, 
/ latitude and longitude.

binary_data "binary format"
latitude 1 8 double 6
longitude 9 16 double 6

ASCII_data "ASCII format"
latitude 1 10 double 6
longitude 12 22 double 6 
\end{vcode}

\note{You can display \lit{latlon.fmt} on your screen by changing to
  the directory containing the FreeForm ND example files and using the
  appropriate command (\lit{cat} or \lit{more}).  }

This format description file contains two format descriptions. The
first describes data in the binary data file \lit{latlon.bin} and the
second describes data in the ASCII data file \lit{latlon.dat}
(contents shown below).

The binary and ASCII variables both have the same names. The binary
variables are defined to occupy 8 bytes each (positions 1-8 and 9-16).
The ASCII variable latitude occupies 10 bytes (positions 1 to 10) and
longitude occupies 11 bytes (positions 12-22). Both the binary and
ASCII variables are stored as doubles because they have more than
seven digits and include a decimal point (see the \lit{latlon.dat}
listing below). The precision of 6 for all the variables indicates
that there are six digits to the right of the decimal point.

Here is \lit{latlon.dat}.  Note that the first two lines are included
as aids, to show what appears in which column of the file.  These two
lines are not part of the file.

\begin{vcode}{ib}
         1         2         3         4         5         6
12345678901234567890123456789012345678901234567890123456789012345
-47.303545 -176.161101
 -0.928001    0.777265
-28.286662   35.591879
 12.588231  149.408117
-83.223548   55.319598
 54.118314 -136.940570
 38.818812   91.411330
-34.577065   30.172129
 27.331551 -155.233735
 11.624981 -113.660611
 77.652742  -79.177679
 77.883119  -77.505502
-65.864879  -55.441896
-63.211962  134.124014
 35.130219 -153.543091
 29.918847  144.804390
-69.273601   38.875778
-63.002874   36.356024
 35.086084  -21.643402
-12.966961   62.152266
\end{vcode}

\section{Changing Formats}

\indc{changing formats!newform@\lit{newform}}
\indc{commands!newform@\lit{newform}}
\indc{FreeForm ND commands!newform@\lit{newform}}
\indc{newform@\lit{newform}!changing formats}
\indc{format conversion}\indc{conversion!format!quick intro}
The FreeForm ND program \lit{newform} is used to convert data from one
format to another. Format descriptions for all the data (input and
output) involved in the conversion must be included in a format
description file.

In this example, you will use \lit{newform} to convert ASCII data in
the input file \lit{latlon.dat} to binary data in the output file
\lit{latlon2.bin}. First you need to create a format description file
like the following that describes the data in these two files.

Here is the file \lit{latlon2.fmt}:

\begin{vcode}{ib}
/ This is the format description file for data files latlon.dat
/ and latlon2.bin. Each record in both files contains two fields, 
/ latitude and longitude.

ASCII_data "ASCII format"
latitude 1 10 double 6
longitude 12 22 double 6

binary_data "binary format"
latitude 1 4 long 6
longitude 5 8 long 6
\end{vcode}

The ASCII and binary variables both have the same names. The ASCII
variable latitude occupies 10 bytes (positions 1-10) and longitude
occupies 11 bytes (positions 12-22). The ASCII variables are defined
to be of type double because they have more than seven digits and
include a decimal point. (See the \lit{latlon.dat} listing above.) The
binary variables are defined to occupy four bytes each (positions 1-4
and 5-8) and to be of type long. The precision for all is 6.  To
convert the ASCII data in \lit{latlon.dat} to binary data, you would:

\begin{enumerate}
\item Change to the directory that contains the FreeForm ND files. 

\item Enter the following command: 

\begin{example}
newform latlon.dat -f latlon2.fmt -o latlon2.bin 
\end{example}
\end{enumerate}

This command creates a new binary data file called \lit{latlon2.bin}
with the 20 latitude and longitude values in \lit{latlon.dat} stored
as binary longs.

For complete information about using \lit{newform}, see
\chapterref{ff,fmtconv}. 

\section{Viewing Binary Data Files}

\indc{binary files!viewing}
\indc{binary files!readfile@\lit{readfile}}
\indc{commands!readfile@\lit{readfile}}
\indc{FreeForm ND commands!readfile@\lit{readfile}}
\indc{readfile@\lit{readfile}!binary file}
The FreeForm ND Data Access System includes an interactive utility
program, \lit{readfile}, for reading binary files. You can use
\lit{readfile} to read the binary file \lit{latlon2.bin} and check
that the data are correct.

To read \lit{latlon2.bin}, you would: 

\begin{enumerate}
\item Change to the directory that contains the FreeForm ND files. 

\item On the command line, enter readfile \lit{latlon2.bin} 

\item The data are stored as longs, so enter l to view the first value. 
  
  The number -47303545, corresponding to the first number in
  \lit{latlon.dat} (but with implied precision, i.e., without a
  decimal point), should appear.

\item To check additional numbers, continue to enter l or press Return. 
  
  The numbers should correspond to those in \lit{latlon.dat}.

\item When you want to quit \lit{readfile}, enter q. 
\end{enumerate}

For complete information about using \lit{readfile}, see
\chapterref{ff,fmtconv}. 

\section{Creating Summary Files}

\indc{creating!summary files}\indc{summary files!creating}
The FreeForm ND-based utility program \lit{checkvar} creates a summary
file for each variable in a data file, a list of maximum and minimum
values, and a summary of processing activity. A variable summary file
(also called a histogram data file) contains histogram information
that shows the variable's distribution in the data file. In this
example, you will use \lit{checkvar} to create a processing summary
file and variable summary files for the two variables latitude and
longitude in the file \lit{latlon2.bin}.

\section{Generating the Summaries}

To create summary files for \lit{latlon2.bin}, you would: 

\begin{enumerate}
\item Change to the directory that contains the FreeForm ND files. 

\item Enter the following command: 

checkvar latlon2.bin -o checkvar.out 

A summary of processing information and the maximum and minimum for each variable are displayed on the screen. The following three files are created: 

\begin{description}
\item[\lit{checkvar.out}] recaps processing activity, maximums and
  minimums
  
\item[\lit{latitude.lst}] shows distribution of the latitude values in
  \lit{latlon2.bin}
  
\item[\lit{longitude.lst}] shows distribution of the longitude values
  in \lit{latlon2.bin}
\end{description}

\item To view the files, use the appropriate command, i.e., \lit{cat}
  or \lit{more.}
\end{enumerate}

\section{Interpreting the Summaries}

The three files output by \lit{checkvar} are shown and discussed
below. To remind yourself of the input values, refer to
\lit{latlon.dat} since it contains the same values as
\lit{latlon2.bin} in ASCII representation.

Here is \lit{checkvar.out}:

\begin{vcode}{ib}
Input file: latlon2.bin
No requested precision, Approximate number of sorting bins = 100

Input data format       (latlon2.fmt)
binary_input_data       "binary format"
The format contains 2 variables; length is 8.

Output data format       (latlon2.fmt)
ASCII_output_data       "ASCII format"
The format contains 2 variables; length is 24.

Histogram data precision: 5, Number of sorting bins: 20
latitude: 20 values read
minimum: -83.223548 found at record 5
maximum:  77.883119 found at record 12
Summary file: latitude.lst

Histogram data precision: 5, Number of sorting bins: 20
longitude: 20 values read
minimum: -176.161101 found at record 1
maximum:  149.408117 found at record 4
Summary file: longitude.lst
\end{vcode}

The processing summary file \lit{checkvar.out} first shows the name of
the input data file (\lit{latlon2.bin}).  Since precision and a
maximum number of bins were not specified on the command line, No
requested precision and the default value for sorting bins of 100 are
shown.

A summary of each format shows the type of format (in this case, Input
data format and Output data format) and the name of the format file
containing the format descriptions (\lit{latlon2.fmt} describes both
the input and output formats; note that \lit{checkvar} ignores output
formats). Next, you see the format descriptor as resolved by FreeForm
ND (e.g., \lit{binary\_input\_data}) and the format title (e.g.,
``binary format''). Then the number of variables in a record and total
record length are given; for ASCII, record length includes the
end-of-line character.

A section for each variable processed by \lit{checkvar} indicates the
histogram precision and actual number of sorting bins. Under some
circumstances, the precision of values in the histogram file may be
different than the precision you specified on the command line. No
precision was specified on the command line in this case, so the
default maximum precision of 5 is used. The second line shows the name
of the variable (latitude and longitude) and the number of values in
the data file for the variable (20 for both latitude and longitude).

The minimum and maximum values for the variable are shown next
(-83.223548 is the minimum and 77.883119 is the maximum value for
latitude). The maximum and minimum values are given here with a
precision of 6, which is the precision specified in the relevant
format description file. The locations of the maximum and minimum
values in the input file are indicated. (-83.223548 is the fifth
latitude value in \lit{latlon2.bin} and 77.883119 is the twelfth).

Finally, the name of the histogram data (or variable summary) file
generated for each variable is given. The two example histogram files,
\lit{latitude.lst} and \lit{longitude.lst}, are shown next.


\begin{center}
\begin{tabular}{p{2in}p{2in}}
\lit{latitude.lst} 

\begin{vcode}{ts}
-83.22355       1
-69.27361       1
-65.86488       1
-63.21197       1
-63.00288       1
-47.30355       1
-34.57707       1
-28.28667       1
-12.96697       1
 -0.92801       1
 11.62498       1
 12.58823       1
 27.33155       1
 29.91884       1
 35.08608       1
 35.13021       1
 38.81881       1
 54.11831       1
 77.65274       1
 77.88311       1 
\end{vcode}

& 
\lit{longitude.lst}
\begin{vcode}{ts}
-176.16111       1
-155.23374       1
-153.54310       1
-136.94057       1
-113.66062       1
 -79.17768       1
 -77.50551       1
 -55.44190       1
 -21.64341       1
   0.77726       1
  30.17212       1
  35.59187       1
  36.35602       1
  38.87577       1
  55.31959       1
  62.15226       1
  91.41133       1
 134.12401       1
 144.80439       1
 149.40811       1
\end{vcode}

\end{tabular}
\end{center}
 
The histogram files consist of two columns. The first indicates
boundary values for data bins and the second gives the number of data
points in each bin. The boundary values are determined dynamically by
\lit{checkvar} and often do not correspond exactly to data values in
the input file, even if the \lit{checkvar} and data file precisions
are the same.

The first data bin in \lit{latitude.lst} contains data points in the
range -83.22355 (inclusive) to -69.27361 (exclusive). The first bin
has one data point, -83.223548 (refer to \lit{latlon.dat} on page 7).
The fifth data bin contains latitude values from -63.00288 (inclusive)
to -47.30355 (exclusive); the data point in the fourth bin is
-63.002874.

For complete information about using \lit{checkvar}, see
\chapterref{ff,datachk}. 

%%% Local Variables: 
%%% mode: latex
%%% TeX-master: t
%%% End: 
