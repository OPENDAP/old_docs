
% This document describes the DODS FreeForm ND Server

% $Id$
%
% $Log: ff-server.tex,v $
% Revision 1.1  2000/11/28 21:43:41  tom
% moved the freeform books to the DODS-doc/server/freeform directory
%
% Revision 1.1  1999/08/30 15:50:09  tom
% added parallel book for DODS Freeform server.  See ff/README for details.
%
% Revision 1.2  1999/03/22 21:26:22  tom
% converted original book source to LaTeX, put into CVS
%
% Revision 1.1  1998/11/16 20:32:20  tom
% First draft checked in.
%
%

\chapter{The \ffs}
\label{ff,ff-server}

The \ffs\ is a OPeNDAP server add-on that uses \ffnd\ to convert and serve data
in formats that are not directly supported by DODS servers.  Bringing
\ffnd 's data conversion capacity into the OPeNDAP world widens data
access for DAP2 clients, since any format that can be described in
\ffnd\ can now be served by the OPeNDAP data server.

Like all DAP2 servers, the \ffs\ responds to client requests for data
by returning either data values or information about the data.  It
differs from other DAP2 servers because it invokes \ffnd\ to read the
data from disk before serving it to the client.

The following sequence of steps illustrates how the \ffs\ works:

\begin{enumerate}
\item A DAP2 client sends a request for data to a \ffs .  The request
  must include the name of the file that contains the data, and may
  include a constraint expression to sample the data.
\item The \ffs\ looks in its path for two files: a data file with the
  name sent by the client, and a format definition file to use with
  the data file.  The format definition file contains a description of
  the data format, constructed according to the \ffnd\ syntax.
\item The server uses both files in invoking the \ffnd\ engine.  The
  \ffnd\ engine reads the data file and the format file, using the
  instructions in the latter to convert the former into data which is
  then passed back to the \ffs .
\item On receiving the converted data, the \ffs\ converts the data
  into the DAP2 transmission format.  The conversion may involve some
  adjustment of data types; these are listed in
  \sectionref{sec,convert}.  The server also applies any constraint
  expressions the client sent along with the URL.
\item The server then constructs DDS and DAS files based on the format
  of the converted data.  If the server has access to DDS and DAS
  files that describe the data, it applies those definition before
  sending them back to the client.
\item Finally, the server sends the DDS, DAS, and converted data back
  to the client.
\end{enumerate}

For information about how to write a \ffnd\ data description, refer to
the \chapterref{ff,tblfmt} for sequence data and
\chapterref{ff,arrayfmt} for array data. 

For an introduction to DAP2 and to the OPeNDAP project, please refer
to \DODSuser .

%\section{About \ffnd\ }
%\label{sec,about}

%\ffnd\ is a data description and conversion toolkit, originally
%developed by NOAA's NGDC, that has been adapted for use with DODS.

%Rather than define its conversions internally, \ffnd\ uses an external
%format definition file, which describes the format of the input data
%and the desired format of the output data. \ffnd\ users write the format
%definition files, so the data conversion capabilities are limited only
%by the syntax of the format definition, and a user's ability to
%describe the format of the data.  The format definition file can
%describe data files that are arranged in either tabular form (rows and
%columns) or array form (elements of variable size). \ffnd\ can read data
%in ASCII, binary, and dBASE file formats.

%\ffnd\ includes utilities for performing and verifying file-to-file
%conversions, and libraries for writing applications that make use of
%its conversion capabilities.  See the \chapterref{a} for a complete
%description of the format definition language, and instructions for
%preparing a format definition file.


\section{Differences between \ffnd\ and the \ffs}
\label{sec,advanced}

The \ffs\ is based on the same libraries used to make the \ffnd\
utilities.  However, there are some important differences in the
resulting software:

\begin{itemize}
\item The \ffs\ is a \ffnd\ application that converts data \emph{on
    receiving a client request for that data}, and not before.  Data
  served by the \ffs\ remains in its original format.
\item The \ffs\ does not produce an output file containing the
  converted data, but serves it directly over the network to the DAP2
  client.  Therefore, the \ffs\ ignores the output section of the
  format definition file.
\item To sample a data file, you do not write format definitions that
  cause the \ffnd\ engine to sample the data file.  Instead, you add a 
  DAP2 ``constraint expression'' to the URL that the client sends to
  the \ffs .
\item The \ffs\ performs data conversion on the fly.  Conversion only
  takes place when the client sends a URL requesting data from the
  \ffs . 
\item Unlike \ffnd , there is no static file created by the conversion.
  (If you wish to create or work with such a file, use the \ffnd\
  utilities, such as \lit{newform}.) 
\end{itemize}


\section{Data Type Conversions}
\label{sec,convert}

The \ffs\ performs data conversions, based on the data it receives
from the \ffnd\ engine.  Note that OPeNDAP does not recommend the use of
\lit{int64} and \lit{uint64} in the format definition file.

\begin{table}[htb]
  \caption{DAP2 Data Type Conversions}
  \begin{center}
    \begin{tabular}{|l|l|}\hline
      \textbf{\ffnd}  & \textbf{DAP2} \\ \hline \hline
      \lit{text} & \lit{String} \\ \hline
      \lit{int8}, \lit{uint8} & \lit{Byte}  \\ \hline
      \lit{int16} & \lit{Int16}  \\ \hline
      \lit{int32}, \lit{int64} & \lit{Int32}  \\ \hline
      \lit{uint16} & \lit{UInt16}  \\ \hline
      \lit{uint32}, \lit{uint64} & \lit{UInt32}  \\ \hline
      \lit{float32} & \lit{Float32} \\ \hline
      \lit{float64}, \lit{enote} & \lit{Float64} \\ \hline
    \end{tabular}
    \caption{Basic Data Type Conversions}
    \label{tab,dods-convert}
  \end{center}
\end{table}


\subsection{Conversion Examples}
\label{sec,examples}

The examples show how the \ffs\ treats data received from the \ffnd\
engine.  Please see the \ffs\ distribution for more test data and
format definition files, and the \chapterref{a} for more information on
writing format definitions.


\subsubsection{Arrays}

If you define a variable as an array in the \ffnd\ format definition
file, the \ffs\ produces an array of variables with matching types.

For exmple, this entry in the format definition file:

\begin{vcode}{ib}
binary_input_data "array"
fvar1 1 4 ARRAY["records" 1 to 101] of int32 0
\end{vcode}

in converted by the \ffs\ to:

\begin{vcode}{ib}
Int32 fvar1[records = 101]
\end{vcode}


\subsubsection{Collections of Variables}

If you define several variables in the format definition file, the
\ffs\ produces a Sequence of variables with matching types.

For example, this entry in the format definition file:

\begin{vcode}{ib}
ASCII_input_data "ASCII_data"
fvar1   1 10  int32 2
svar1  13 18  int16 0
usvar1 21 26 uint16 1
lvar1  29 39  int32 0
ulvar1 42 52 uint32 4
\end{vcode}

is converted by the \ffs\ to:

\begin{vcode}{ib}
Sequence {
  Int32 fvar1;
  Int32 svar1;
  ...
} ASCII_data;
\end{vcode}


\subsubsection{Multiple Arrays}

If you define a collection of arrays in the format definition file, as
you would expect, the \ffs\ produces a dataset containing multiple
arrays.

For example, this entry in the format definition file:

\begin{vcode}{ib}
binary_input_data "arrays"
fvar1 1 4 ARRAY["records" 1 to 101] of int32 0
fvar2 1 4 ARRAY["records" 1 to 101] of int32 0
\end{vcode}

is converted by the \ffs\ to:

\begin{vcode}{ib}
Dataset {
  Int32 fvar1[records=101]
  Int32 fvar2[records=101]
};
\end{vcode}



\tbd{Correct service name in readme file (not nph-nc, but nph-ff)}
\tbd{Can reuse same format def file, if you give it the naem ext.fmt,
  where ext is the extension used for each of the data files you want
  to convert.  e.g. for one file you would have Apple.dat and the
  corres. Apple.fmt.  But for a set you would have Empire.apple,
  Cortland.apple, Rome.apple, and the format def file would be
  apple.fmt.}





%%% Local Variables: 
%%% mode: latex
%%% TeX-master: t
%%% End: 
