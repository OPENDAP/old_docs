%CHAPTER 2       
%
% $Id$
%

\chapter{Quick Tour of the \ffs}
\label{ff,dquick}

This chapter provides you a quick introduction to the \ffs , including
writing format descriptions and serving test datasets.

\section{Getting Started Serving Data}
\label{ff,startup}

To get going with the \ffs , follow these steps:

\begin{enumerate}
\item Obtain and install the OPeNDAP data server distribution and the
  \ffs\ format handler distribution. (You can get these files from the
  \DODShome .)
  
\item Install the \ffs . (See the instructions in the INSTALL file
  accompanying the server.)  

\item Examine the structure of the data file(s) you intend to serve,
  and construct a \ffnd\ format definition file that describes the
  layout of data in the files.  (Refer to the \chapterref{ff,tblfmt}
  for instructions about sequence data and \chapterref{ff,arrayfmt}
  for array data.  Consult \DODSuser\ if you do not know the
  difference between the two data types.)
  
\item If you wish, you may include an output definition format within
  this file, to allow you to test that your input description is
  accurate.  You can use the \ffnd\ utilities, such as \lit{newform},
  to validate the conversion.  The \chapterref{ff,fmtconv} contains a
  detailed description of \lit{newform}.  This step is optional, since
  the \ffs\ ignores the output definition section of the format
  definition file.

\item Although the \ffs\ can generate default DDS and DAS files, you
  may want to write these files yourself, to override the default data
  descriptions, or to add attribute data.  The default descriptions
  are based on the format of the data the the \ffs\ receives from the
  \ffnd\ engine.

\item Place the data files, and a corresponding format file for each
  data file, in a place where the \ffs\ can find them.  This is
  generally in the server's document root, or in a subdirectory
  there.  See \DODSuser\ for detailed instructions on installing a
  server if the document root is not a familiar concept.

\end{enumerate}

Your data is now available to anyone who knows about your server.
The next section contains examples of writing format descriptions.

\section{Examples}

\indc{format descriptions}\indc{descriptions!format} 
You can easily create FreeForm ND format description files that
describe the formats of input and output data and headers. The \ffs\ 
and other \ffnd -based programs then use these files to correctly
access and manipulate data in various formats. An example format
description file is shown and described below.

For complete information about writing format descriptions, see
\chapterref{ff,tblfmt} and \chapterref{ff,arrayfmt}.

\subsection{Sequence Data}

Here is a data file, containing a sequence of four data types.  (This
data file and several of the other examples in this chapter are
available \xlink*{here}[under ``Documentation'' on the OPeNDAP web page
(\DODShomeUrl)]{\DODShomeUrl/examples}.)

Here is the data file, called
\xlink{\lit{ffsimple.dat}}{\DODShomeUrl/examples/ffsimple.dat}:

\begin{vcode}{xib}
Latitude and Longitude:   -63.223548 54.118314  -176.161101 149.408117
-47.303545 -176.161101 11.7125 34.4634
-25.928001   -0.777265 20.7288 35.8953
-28.286662   35.591879 23.6377 35.3314
 12.588231  149.408117 28.6583 34.5260
-63.223548   55.319598  0.4503 33.8830
 54.118314 -136.940570 10.4085 32.0661
-38.818812   91.411330 13.9978 35.0173
-34.577065   30.172129 20.9096 35.4705
 27.331551 -155.233735 23.0917 35.2694
 11.624981 -113.660611 27.5036 33.7004
\end{vcode}

The file consists of a single header line, followed by a sequence of
records, each of which contains a latitude, longitude, temperature,
and salinity.

Here is a format file you can use to read \lit{ffsimple.dat}.  It is
called \xlink{\lit{ffsimple.fmt}}{\DODShomeUrl/examples/ffsimple.fmt}:

\begin{vcode}{sib}
ASCII_file_header "Latitude/Longitude Limits"
minmax_title 1 24 char 0
latitude_min 25 36 double 6
latitude_max 37 46 double 6
longitude_min 47 59 double 6
longitude_max 60 70 double 6

ASCII_data "lat/lon"
latitude 1 10 double 6
longitude 12 22 double 6
temp 24 30 double 4
salt 32 38 double 4

ASCII_output_data "output"
latitude 1 10 double 3
longitude_deg 11 15 short 0
longitude_min 16 19 short 0
longitude_sec 20 23 short 0
salt 31 40 double 2
temp 41 50 double 2
\end{vcode}

The format file consists of three sections.  The first shows \ffnd\
how to parse the file header.  The second section describes the
contents of the data file.  The third part describes how to write the
data to another file.  This part is not important for the \ffs , but
is useful for debugging the input descriptions.

Download the \lit{ffsimple} files described above, and type:

\begin{example}
> newform ffsimple.dat
\end{example}

You should see results like this:

\begin{vcode}{xib}
Welcome to Newform release 4.2.3 -- an NGDC FreeForm ND application

(ffsimple.fmt) ASCII_input_file_header  "Latitude/Longitude Limits"
File ffsimple.dat contains 1 header record (71 bytes)
Each record contains 6 fields and is 71 characters long.

(ffsimple.fmt) ASCII_input_data "lat/lon"
File ffsimple.dat contains 10 data records (390 bytes)
Each record contains 5 fields and is 39 characters long.

(ffsimple.fmt) ASCII_output_data        "output"
Program memory contains 10 data records (510 bytes)
Each record contains 7 fields and is 51 characters long.

   -47.304 -176   9  40            34.46     11.71
   -25.928    0 -46  38            35.90     20.73
   -28.287   35  35  31            35.33     23.64
    12.588  149  24  29            34.53     28.66
   -63.224   55  19  11            33.88      0.45
    54.118 -136  56  26            32.07     10.41
   -38.819   91  24  41            35.02     14.00
   -34.577   30  10  20            35.47     20.91
    27.332 -155  14   1            35.27     23.09
    11.625 -113  39  38            33.70     27.50
100% processed
\end{vcode}

Now take both the \lit{ffsimple} files and put them into a directory
in your web server's document root directory.  (Refer to the
\DODSuser\ for some tips on figuring out where that is.)

Here's an example, on a computer on which the web server document root
is \lit{/export/home/http/htdocs}:

\begin{example}
> mkdir /export/home/http/htdocs/data
> cp ffsimple.* /export/home/http/htdocs/data
\end{example}

Now, using a common web browser such as Netscape Navigator, enter the
following URL (substitute your machine name and CGI directory for the
ones in the example):

\begin{vcode}{sib}
http://test.opendap.org/opendap/nph-dods/data/ff/ffsimple.dat.asc
\end{vcode}

You should get something like the following in your web browser's
window:

\begin{vcode}{ib}
latitude, longitude, temp, salt
-47.3035, -176.161, 11.7125, 34.4634
-25.928, -0.777265, 20.7288, 35.8953
-28.2867, 35.5919, 23.6377, 35.3314
12.5882, 149.408, 28.6583, 34.526
-63.2235, 55.3196, 0.4503, 33.883
54.1183, -136.941, 10.4085, 32.0661
-38.8188, 91.4113, 13.9978, 35.0173
-34.5771, 30.1721, 20.9096, 35.4705
27.3316, -155.234, 23.0917, 35.2694
11.625, -113.661, 27.5036, 33.7004
\end{vcode}

Try this URL:

\begin{vcode}{sib}
http://test.opendap.org/opendap/nph-dods/data/ffsimple.dat.dds
\end{vcode}

This will show a description of the dataset structure (See \DODSuser\ 
for a detailed description of the DAP2 ``Dataset Description
Structure,'' or DDS.):

\begin{vcode}{ib}
Dataset {
    Sequence {
        Float64 latitude;
        Float64 longitude;
        Float64 temp;
        Float64 salt;
    } lat/lon;
} ffsimple;
\end{vcode}

\subsection{Array Data}

If your data more naturally comes in arrays, you can still use the
\ffs\ to serve your data.  The \ffnd\ format for sequence data is
somewhat simpler than the format for array data, so you may find it
easier to begin with the example in the previous section.


\subsubsection{One-dimensional Arrays}

Here is a data file, called
\xlink{\lit{ffarr1.dat}}{\DODShomeUrl/examples/ffarr1.dat}, containing
four ten-element vectors:

\begin{vcode}{ib}
123456789012345678901234567
 1.00  50.00 0.1000  1.1000
 2.00  61.00 0.3162  0.0953
 3.00  72.00 0.5623 -2.3506
 4.00  83.00 0.7499  0.8547
 5.00  94.00 0.8660 -0.1570
 6.00 105.00 0.9306 -1.8513
 7.00 116.00 0.9647  0.6159
 8.00 127.00 0.9822 -0.4847
 9.00 138.00 0.9910 -0.7243
10.00 149.00 0.9955 -0.3226
\end{vcode}

Here is a format file to read this data
(\xlink{\lit{ffarr1.fmt}}{\DODShomeUrl/examples/ffarr1.fmt}):

\begin{vcode}{ib}
ASCII_input_data "simple array format"
index 1 5 ARRAY["line" 1 to 10 sb 23] OF float 1
data1 6 12 ARRAY["line" 1 to 10 sb 21] OF float 1
data2 13 19 ARRAY["line" 1 to 10 sb 21] OF float 1
data3 20 27 ARRAY["line" 1 to 10 sb 20] OF float 1

ASCII_output_data "simple array output"
index 1 7 ARRAY["line" 1 to 10] OF float 0
/data1 6 12 ARRAY["line" 1 to 10 sb 21] OF float 1
/data2 13 19 ARRAY["line" 1 to 10 sb 21] OF float 4
/data3 20 27 ARRAY["line" 1 to 10 sb 20] OF float 4
\end{vcode}

The output section is not essential for the \ffs , but is included so
you can check out the data with the \lit{newform} command.

Download the files from the OPeNDAP web site, and try typing:

\begin{example}
> newform ffarr1.dat
\end{example}

You should see the \lit{index} array printed out.  Uncomment different
lines in the output section of the example file to see different data
vectors. 

Now look a little closer at the input section of the file:

\begin{vcode}{ib}
index 1 5 ARRAY["line" 1 to 10 sb 23] OF float 1
\end{vcode}

This line says that the array in question---called ``index''---starts
in column one of the first line, and each element takes up five bytes.
The first element starts in column one and goes into column five.  The
array has one dimension, ``line'', and is composed of floating point
data.  The remaining elements of this array are found by skipping the
next 23 bytes (the newline counts as a character), reading the
following five bytes, skipping the next 23 bytes, and so on.

Of course, the 23 bytes skipped in between the \lit{index} array
elements also contain data from other arrays.  The second array,
\lit{data1}, starts in column 6 of line one, and has 21 bytes between
values.  The third array starts in column 13 of the first line, and
the fourth starts in column 20.

Move the \lit{ffarr1.*} files into your data directory:

\begin{example}
> cp ffarr1.* /export/home/http/htdocs/data
\end{example}

Now you can look at this data the same way you looked at the sequence
data.  Request the DDS for the dataset with a URL like this one:

\begin{vcode}{sib}
http://test.opendap.org/opendap/nph-dods/data/ffarr1.dat.dds
\end{vcode}

You can see that the dataset is a collection of one-dimensional
vectors.  You can see the individual vectors with a URL like this:

\begin{vcode}{sib}
http://test.opendap.org/opendap/nph-dods/data/ffarr1.dat.asc?index
\end{vcode}

\subsubsection{Multi-dimensional Arrays}

Here's another example, with a two-dimensional array.
(\xlink{\lit{ffarr2.dat}}{\DODShomeUrl/examples/ffarr2.dat}):

\begin{vcode}{ib}
         1         2         3         4
1234567890123456789012345678901234567890
  1.00  2.00  3.00  4.00  5.00  6.00
  7.00  8.00  9.00 10.00 11.00 12.00
 13.00 14.00 15.00 16.00 17.00 18.00
 19.00 20.00 21.00 22.00 23.00 24.00
 25.00 26.00 27.00 28.00 29.00 30.00
\end{vcode}

There are no spaces between the data columns within an array row, but in
order to skip reading the newline character, we have to skip one
character at the end of each row.  Here is a format file to read this
data (\xlink{\lit{ffarr2.fmt}}{\DODShomeUrl/examples/ffarr2.fmt}):

\begin{vcode}{ib}
ASCII_input_data "one"
data 1 6 ARRAY["y" 1 to 5 sb 1]["x" 1 to 6] OF float 1

ASCII_output_data "two"
data 1 4 ARRAY["x" 1 to 6 sb 2]["y" 1 to 5] OF float 1
\end{vcode}

Again, the output section is only for using with the \lit{newform}
tool.  Put these data files into your \lit{htdocs} directory, and look
at the DDS as you did with the previous example.


\subsubsection{A Little More Complicated}

You can use the \ffs\ to serve data with multi-dimensional arrays and
one-dimensional vectors interspersed among one another.  Here's a file
containing this kind of data 
(\xlink{\lit{ffarr3.dat}}{\DODShomeUrl/examples/ffarr3.dat}): 

\begin{vcode}{ib}
         1         2         3         4
1234567890123456789012345678901234567890123
XXXX  1.00  2.00  3.00  4.00  5.00  6.00YY
XXXX  7.00  8.00  9.00 10.00 11.00 12.00YY
XXXX 13.00 14.00 15.00 16.00 17.00 18.00YY
XXXX 19.00 20.00 21.00 22.00 23.00 24.00YY
XXXX 25.00 26.00 27.00 28.00 29.00 30.00YY
\end{vcode}

In order to read this file successfully, we define three vectors to
read the ``XXXX'', the ``YY'', and the newline.  Here is a format file
that does this 
(\xlink{\lit{ffarr3.fmt}}{\DODShomeUrl/examples/ffarr3.fmt}): 

\begin{vcode}{ib}
dBASE_input_data "one"
headers 1 4 ARRAY["line" 1 to 5 sb 39] OF text 0
data 5 10 ARRAY["y" 1 to 5 sb 7]["x" 1 to 6] OF float 1
trailers 41 42 ARRAY["line" 1 to 5 sb 41] OF text 0
newline 43 43 ARRAY["line" 1 to 5 sb 42] OF text 0

ASCII_output_data "two"
data 1 4 ARRAY["x" 1 to 6 sb 2]["y" 1 to 5] OF float 0
/headers 1 6 ARRAY["line" 1 to 5] OF text 0
/trailers 1 4 ARRAY["line" 1 to 5] OF text 0
/newline 1 4 ARRAY["line" 1 to 5] OF text 0
\end{vcode}

The following chapters offer more detailed information about how
exactly to create a format description file.

%\section{Adding Dataset Attributes}
%\label{ff,quick,das}





%%% Local Variables: 
%%% mode: latex
%%% TeX-master: t
%%% End: 
