%CHAPTER 1       
%
% $Id$
%


\chapter{Introduction}
\label{ff,intro}

The FreeForm ND Data Access System is a flexible system for specifying
data formats to facilitate data access, management, and use. How many
data sets have you not examined or used because they were not in the
correct format for your applications? How many others have foregone
analysis of your data for the same reason? FreeForm ND can save you
countless hours of changing the formats of data sets prior to
analyzing them.

The large variety of data formats is a primary obstacle in the way of
creating flexible data management and analysis software. FreeForm ND
was conceived, developed, and implemented at the National Geophysical
Data Center (NGDC) to alleviate the problems that occur when you need
to use data sets with varying native formats or to write
format-independent applications.

The most recent version of FreeForm, FreeForm ND, is now supported
only for use with the Distributed Oceanographic Data System (DODS).

\section{The Format Problem}

Programmers can readily describe a format for a specific data set, but
a compiled application cannot be used with other data sets until
either the data or the program is modified. Two possible methods for
handling data in a variety of formats are to reformat all the data
into a standard format or to develop programs that can read data in
many different formats.

\subsection{Standard Formats}

A number of standard formats have been proposed over the years and the
specifications for these formats have generally improved. However,
standard formats do not enjoy widespread use, which will probably
continue to be the case.

Many scientists have large amounts of data on hand in non-standard
formats. Converting to standard formats is cumbersome and
time-consuming. In addition, there are so many standard formats that
format-independent applications are required even if only standard
formats are used.

\subsection{Smart Programs}

Software developers can create programs that use data in many
different formats. This approach has several advantages:

\begin{itemize}
\item The programs are flexible enough to allow the introduction of
  new data formats.
  
\item The scientist collecting the data is not forced to conform to
  any single data format.
  
\item The information contained in the original data is not lost
  through reformatting.
\end{itemize}

\subsection{The FreeForm ND Solution}

FreeForm ND uses a variation on the smart program approach. With
FreeForm ND, you specify formats outside application programs by
writing text files that describe the formats of your data sets. The
applications then use these format files as they process data.
FreeForm ND-based applications are in effect format-independent and
you do not need to modify the data or the applications.

FreeForm ND provides a mechanism for data description that is flexible
and easy to use. A set of ready-to-use programs for manipulating a
wide variety of data in standard and non-standard formats is also
provided. FreeForm ND lets you concentrate on your specialty rather
than trying to figure out how to access and manipulate data in
multiple formats. Additionally, the application programmer can use
FreeForm ND libraries and data constructs to develop
format-independent applications.

\section{The FreeForm ND System}

The FreeForm ND Data Access System comprises a format description
mechanism, a library of C functions, object-oriented constructs for
data structures, and a set of programs (built using the FreeForm ND
library and data objects) for manipulating data.

There are two types of FreeForm ND users. Data users and providers
create format description files and run FreeForm ND programs such as
\lit{newform}. Programmers use the FreeForm ND library and data
objects to write data management and analysis applications.

FreeForm ND includes the following programs for accessing and
manipulating data in various formats:

\begin{description}

\item[\lit{newform}] reformats data.

\item[\lit{chkform}] checks a file format description against a file.

\item[\lit{readfile}] reads binary files.

\item[\lit{checkvar}] creates variable summaries.
\end{description}

The FreeForm ND data objects provide an interface between application
and data files.

Programmers use the FreeForm ND library routines to develop
applications.

Data users and providers write format description files that FreeForm
ND-based programs use to correctly access data.

\section{FreeForm ND Files}

The FreeForm ND file set is available only through the DODS
distribution system. You can download the FreeForm ND files from the
DODS website (http://dods.gso.uri.edu).

\section{About this Guide}

\indc{about this guide}
This guide provides instructions for writing format descriptions,
using FreeForm ND programs, and writing your own FreeForm ND-based
applications. The content of each chapter is outlined below.

This chapter introduces the FreeForm ND Data Access System and
summarizes typographic conventions and the contents of this guide.

\begin{description}
\item[\chapterref{ff,quick}] provides a brief
  introduction to writing format descriptions and using several of the
  FreeForm ND programs.
  
\item[\chapterref{ff,tblfmt}] provides
  detailed information about writing format descriptions to facilitate
  access to data in tabular formats.
  
\item[\chapterref{ff,arrayfmt}] provides
  detailed information about writing format descriptions to facilitate
  access to data in non-tabular (array) formats.
  
\item[\chapterref{ff,convs}] presents FreeForm ND
  file name conventions, the search rules for locating format files,
  and standard command line arguments for FreeForm ND programs.
  
\item[\chapterref{ff,fmtconv}] shows you how to use the
  FreeForm ND program \lit{newform} to convert data from one format to
  another and also how to read the data in a binary file.
  
\item[\chapterref{ff,convvars}] discusses FreeForm ND
  conversion variables, which let you translate between a number of
  representations of space and time values.
  
\item[\chapterref{ff,hdrfmt}] tells you how to work with
  header formats.
  
\item[\chapterref{ff,datachk}] discusses the FreeForm ND
  program \lit{checkvar}, which you can use to check data distribution
  and quality.
  
\item[\chapterref{ff,develop}] summarizes
  how to use the FreeForm ND Data Access System to build FreeForm
  ND-based programs.

\item[\chapterref{ff,hdf}] provides explanations for a small selection
  of tools that will be useful for programmers working with the HDF
  file format.
  
\item[\chapterref{ff,varname}] lists the conversion
  variable names that FreeForm ND recognizes.
  
\item[\chapterref{ff,errors}] presents a list of common
  FreeForm ND error messages.
  
\item[\chapterref{ff,query}] lists the operators, symbols, and
  functions you can use to construct queries.
\end{description}

\listconventions

A position box is often used in this book to indicate column position
of field values in data files. It is shown at the beginning of a data
list in the documentation, but does not appear in the data file
itself.  It looks something like this:

\begin{vcode}{sib}
1         2         3         4         5         6
012345678901234567890123456789012345678901234567890
\end{vcode}


%%% Local Variables: 
%%% mode: latex
%%% TeX-master: t
%%% TeX-master: t
%%% TeX-master: t
%%% End: 
