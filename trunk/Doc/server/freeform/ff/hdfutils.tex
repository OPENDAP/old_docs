%CHAPTER 10       
%
% $Id$
%

\chapter{HDF Utilities}
\label{ff,hdf}

FreeForm ND includes three utilities for use with HDF (hierarchical
data format) files: \lit{makehdf}, \lit{splitdat}, and \lit{pntshow}.
These programs were built using both the FreeForm library and the HDF
library, which was developed at the National Center for Supercomputer
Applications (NCSA).

The \lit{makehdf} program converts binary and ASCII data files to HDF
files and converts multiplexed band interleaved by pixel image files
into a series of single parameter files. The \lit{splitdat} program is
used to separate and reformat data files containing headers and data
into separate header and data files, or to translate them into HDF
files. The \lit{pntshow} program extracts point data from HDF files
into binary or ASCII format.

It is assumed in this chapter that you have a working familiarity with
HDF terminology and conventions. See the HDF user documentation for
detailed information.

\note{Do not try the examples in this chapter. The example file set is
  incomplete.  }

\section{\lit{makehdf}}
\label{ff,hdf,makehdf}

Using \lit{makehdf} you can convert data files with formats described
in a FreeForm format file into HDF files. You should follow FreeForm
naming conventions for the data and format files. For details about
FreeForm conventions, see \chapterref{ff,convs}.

\note{A dBASE input file must be converted to ASCII or binary using
  \lit{newform} before you can run \lit{makehdf} on it.  }

The HDF file resulting from a conversion consists either of a group of
scientific datasets (SDS's), one for each variable in the input data
file, or of a \new{vgroup} containing all the variables as one
\new{vdata.} If you are working with grid data, you will want SDS's
(the default) in the output HDF file. A vdata (\lit{-vd} option) is the
appropriate choice for point data.

The \lit{makehdf} command has the following form:   

\newcommand{\hdfr}{\hbox{\lit{-r} \var{rows}}}
\newcommand{\hdfc}{\hbox{\lit{-c} \var{columns}}}
\newcommand{\hdfv}{\hbox{\lit{-v} \var{var_file}}}
\newcommand{\hdfd}{\hbox{\lit{-d} \var{HDF_description_file}}}
\newcommand{\hdfxy}{\hbox{\lit{-xl} \var{x_label} \lit{-yl} \var{y_label}}}
\newcommand{\hdfxuyu}{\hbox{\lit{-xu} \var{x_units} \lit{-yu} \var{y_units}}}
\newcommand{\hdfxfyf}{\hbox{\lit{-xf} \var{x_format} \lit{-yf} \var{y\_format}}}
\newcommand{\hdfid}{\hbox{\lit{-id} \var{file\_id}}}
\newcommand{\hdfvd}{\hbox{\lit{-vd} [\var{vdata\_file}]}}
\newcommand{\hdfdmx}{\hbox{\lit{-dmx} [\lit{-sep}]}}
\newcommand{\hdfdf}{\hbox{\lit{-df}}}
\newcommand{\hdfmd}{\hbox{\lit{-md} \var{missing\_data\_file}}}
\newcommand{\hdfdof}{\hbox{\lit{-dof} \var{HDF\_file}}}
  
\proto{\lit{makehdf} \var{input_file} [\hdfr] [\hdfc] [\hdfv] [\hdfd]
  [\hdfxy] [\hdfxuyu] [\hdfxfyf] [\hdfid] [\hdfvd] [\hdfdmx] [\hdfdf]
  [\hdfmd] [\hdfdof] }

\begin{description}

\item[\var{input\_file}]
  
  Name of the input data file. Following FreeForm naming conventions,
  the standard extensions for data files are \lit{.dat} for ASCII
  format and \lit{.bin} for binary.

\item[\hdfr]\indc{r@\lit{-r}}
  
  Option flag followed by the number of rows in each resulting
  scientific dataset. The number of rows must be specified through
  this option on the command line, or in an equivalence table, or in a
  header (\lit{.hdr}) file defined according to FreeForm standards.

\item[\hdfc]\indc{c@\lit{-c}}
  
  Option flag followed by the number of columns in each resulting
  scientific dataset. The number of columns must be specified through
  this option on the command line, or in an equivalence table, or in a
  header (\lit{.hdr}) file defined according to FreeForm standards.
  
  For information about equivalence tables, see the GeoVu Tools
  Reference Guide.

\item[\hdfv]\indc{v@\lit{-v}}
  
  Option flag followed by the name of the variable file. The file
  contains names of the variables in the input data file to be
  processed by \lit{makehdf}. Variable names in \var{var\_file} can be
  separated by one or more spaces or each name can be on a separate
  line.

\item[\hdfd]\indc{d@\lit{-d}}
  
  Option flag followed by the name of the file containing a
  description of the input file. The description will be stored as a
  file annotation in the resulting HDF file.

\item[\hdfxy]
\indc{xl@\lit{-xl} \var{x_label}}
\indc{yl@\lit{-yl} \var{y_label}}
  
  Option flags followed by strings (labels) describing the x and y
  axes; labels must be in quotes (\lit{" "}) if more than one word.

\item[\hdfxuyu]
\indc{xu@\lit{-xu} \var{x_units}}
\indc{yu@\lit{-yu} \var{y_units}}
  
  Option flags followed by strings indicating the measurement units
  for the x and y axes; strings must be in quotes (\lit{" "}) if more than
  one word.

\item[\hdfxfyf]
\indc{xf@\lit{-xf} \var{x_format}}
\indc{yf@\lit{-yf} \var{y_format}}
  
  Option flags followed by strings indicating the formats to be used
  in displaying scale for the x and y dimensions; strings must be in
  quotes (\lit{" "}) if more than one word.

\item[\hdfid]\indc{id@\lit{-id}}
  
  Option flag followed by a string that will be stored as the ID of
  the resulting HDF file.

\item[\hdfvd]\indc{vd@\lit{-vd}}
  
  Option flag indicating that the output HDF file should contain a
  vdata. The optional file name specifies the name of the output HDF
  file; the default is \lit{input\_file.HDF}.

\item[\hdfdmx]\indc{dmx@\lit{-dmx}}
  
  The option flag \lit{-dmx} indicates that input data should be
  demultiplexed from band interleaved by pixel to band sequential form
  in \lit{input\_file.dmx}. If \lit{-dmx} is followed by \lit{-sep},
  the input data are demultiplexed into separate variable files called
  \lit{data\_file.1} \ldots \lit{data\_file.n}

\item[\hdfdf]\indc{df@\lit{-df}}
  
  To use this option, the input file (\lit{data\_file.ext}) must be a
  binary demultiplexed (band sequential) file.  For each input
  variable in the applicable FreeForm format description file, there
  is a corresponding demultiplexed section in the output HDF file.

\item[\hdfmd]\indc{md@\lit{-md}}
  
  Option flag followed by the name of the file defining missing data
  (data you want to exclude). Use this option only along with the
  vdata (-vd) option. Each line in the missing data file has the form:

\begin{vcode}{ib}
variable_name lower_limit upper_limit 
\end{vcode}

The precision of the upper and lower limits matches the precision of
the input data.

\item[\hdfdof]\indc{dof@\lit{-dof}}
  
  Option flag followed by the name of the output HDF file. If you do
  not use the \lit{-dof} option, the default output file name is
  \lit{input\_file.HDF}.
\end{description}

\subsection{Example}

You will use \lit{makehdf} to store \lit{latlon.dat} as an HDF file.
The HDF file will consist of two SDS's, one each for the two variables
latitude and longitude. Each SDS will have four rows and five columns.

To convert \lit{latlon.dat} to an HDF file, enter the following
command:

\begin{example}
makehdf latlon.dat -r 4 -c 5 
\end{example}

As \lit{makehdf} translates \lit{latlon.dat} into HDF, processing
information is displayed on the screen:

\begin{vcode}{sib}
1   Caches (1150 bytes) Processed: 800 bytes written to latlon.dmx 

Writing latlon.HDF and calculating maxima and minima ...

Variable latitude:
Minimum: -86.432712  Maximum 89.170904
Variable longitude:
Minimum: -176.161101  Maximum 165.066193 
\end{vcode}

The output from \lit{makehdf} is an HDF file named \lit{latlon.HDF}
(by default). It contains the minimum and maximum values for the two
variables as well as the two SDS's.

A temporary file named \lit{latlon.dmx} was also created. It contains
the data from latlon.dat in demultiplexed form . The data was
converted from its original multiplexed form to enable \lit{makehdf}
to write sections of data to SDS's.

If you start with a demultiplexed file such as \lit{latlon.dmx}, the
translation process is much quicker, particularly for large data
files. As an illustration, try this. Rename \lit{latlon.dmx} to
\lit{latlon.bin} (renaming is necessary for \lit{makehdf} to find the
format description file \lit{latlon.fmt} by default). Enter the
following command:

\begin{example}
makehdf latlon.bin -df -r 4 -c 5 
\end{example}

The output file again is \lit{latlon.HDF}, but notice that no
demultiplexing was done.

\section{\lit{splitdat}}
\label{ff,hdf,splitdat}

The \lit{splitdat} program translates files with headers and data into
separate header and data files or into HDF files. If the translation
is to separate header and data files, the header file can include
indexing information.

The combination of header and data records in a file is often used for
point data sets that include a number of observations made at one or
more stations or locations in space. The header records contain
information about the stations or locations of the measurements. The
data records hold the observational data. A station record usually
indicates how many data records follow it. The structure of such a
file is similar to the following:

\begin{vcode}{ib}
Header for Station 1
Observation 1 for Station 1
Observation 2 for Station 1
       .
       .
       .
Observation N for Station 1

Header for Station 2
Observation 1 for Station 2
Observation 2 for Station 2
       .
       .
       .
Observation N for Station 2

Header for Station 3
       .
       .
       . 
\end{vcode}

Many applications have difficulty reading this sort of heterogeneous
data file. One solution is to split the data into two homogeneous
files, one containing the headers, the other containing the data. With
\lit{splitdat}, you can easily create the separate data and header
files. To use \lit{splitdat} for this purpose, the input and output
formats for the record headers and the data must be described in a
FreeForm format description file. To use \lit{splitdat} for
translating files to HDF, the input format must be described in a
FreeForm format description file. You should follow FreeForm naming
conventions for the data and format files. For details about FreeForm
conventions, see \chapterref{ff,convs}.

The \lit{splitdat} command has the following form:   

\proto{\lit{splitdat} \var{input_file} [\var{output_data_file} > \var{output_header_file}]}

\begin{description}
\item[\var{input\_file} ]
  
  Name of the file to be processed. Following FreeForm naming
  conventions, the standard extensions for data files are \lit{.dat}
  for ASCII format and \lit{.bin} for binary.

\item[\var{output\_data\_file} ]
  
  Name of the output file into which data are transferred with the
  format specified in the applicable FreeForm format description file.
  The standard extensions are the same as for input files. If an
  output file name is not specified, the default is standard output.

\item[\var{output\_header\_file} ]
  
  Name of the output file into which headers from the input file are
  transferred with the format specified in the applicable FreeForm
  format description file. If an output header file name is not
  specified, the default is standard output.
\end{description}

\subsection{Index Creation}

You can use the two variables begin and extent (described below) in
the format description for the output record headers to indicate the
location and size of the data block associated with each record
header. If you then use \lit{splitdat}, the header file that results
can be used as an index to the data file.

\begin{description}

\item[\lit{begin}]
  
  Indicates the offset to the beginning of the data associated with a
  particular header. If the data is being translated to HDF, the units
  are records; if not, the units are bytes.

\item[\lit{extent}]
  
  Indicates the number of records (HDF) or bytes (non-HDF) associated
  with each header record.
\end{description}

\subsubsection{Example}

You will use \lit{splitdat} to extract the headers and data from a
rawinsonde (a device for gathering meteorological data) ASCII data
file named \lit{hara.dat} (HARA = Historic Arctic Rawinsonde Archive)
and create two output files-\lit{23338.dat} containing the ASCII data
and \lit{23338hdr.dat} containing the ASCII headers. The format
description file \lit{hara.fmt} should contain the necessary format
descriptions.

Here is \lit{hara.fmt}:
\nopagebreak
\begin{vcode}{xib}
ASCII_input_record_header "ASCII Location Record input format"
WMO_station_ID_number 1 5 char 0
latitude 6 10 long 2
longitude_east 11 15 long 2
year 17 18 uchar 0
month 19 20 uchar 0
day 21 22 uchar 0
hour 23 24 uchar 0
flag_processing_1 28 28 char 0
flag_processing_2 29 29 char 0
flag_processing_3 30 30 char 0
station_type 31 31 char 0
sea_level_elev 32 36 long 0
instrument_type 37 38 uchar 0
number_of_observations 40 42 ushort 0
identification_code 44 44 char 0

ASCII_input_data "Historical Arctic Rawinsonde Archive input format"
atmospheric_pressure 1 5 long 1
geopotential_height 7 11 long 0
temperature_deg 13 16 short 0
dewpoint_depression 18 20 short 0
wind_direction 22 24 short 0
wind_speed_m/s 26 28 short 0
flag_qg 30 30 char 0
flag_qg1 31 31 char 0
flag_qt 33 33 char 0
flag_qt1 34 34 char 0
flag_qd 36 36 char 0
flag_qd1 37 37 char 0
flag_qw 39 39 char 0
flag_qw1 40 40 char 0
flag_qp 42 42 char 0
flag_levck 43 43 char 0

ASCII_output_record_header "ASCII Location Record output format"
       .
       .
       .

ASCII_output_data "Historical Arctic Rawinsonde Archive output format"
       .
       .
       .
\end{vcode}

To ``split'' \lit{hara.dat}, enter the following command: 

\begin{example}
splitdat hara.dat 23338.dat > 23338hdr.dat 
\end{example}

The data values from \lit{hara.dat} are stored in \lit{23338.dat} and
the headers in \lit{23338hdr.dat}.

Because the variables begin and extent were used in the header output
format in \lit{hara.fmt} to indicate data offset and number of
records, \lit{23338hdr.dat} has two columns of data showing offset and
extent. Thus, it can serve as an index into \lit{23338.dat}.

\subsection{HDF Translation}

If output files are not specified on the \lit{splitdat} command line,
a file named \lit{input\_file.HDF} is created. It is hierarchically
named and organized as follows:

\begin{vcode}{ib}
           vgroup
      input_file_name
          /      \
         /        \
     vdata1       vdata2
"PointIndex"      "input_file_name" 
\end{vcode}

\begin{itemize}
\item \lit{vdata1} contains the record headers
\item \lit{vdata2} contains the data
\item If writing to a Vset (represented by a vgroup), both output
  formats are converted to binary, if not binary already.
\end{itemize}

\subsubsection{Example}

To create the file \lit{hara.HDF} from \lit{hara.dat}, enter the
following abbreviated command:

\begin{example}
splitdat hara.dat 
\end{example}

The output formats in \lit{hara.fmt} are automatically converted to
binary, and subsequently the ASCII data in \lit{hara.dat} are also
converted to binary for HDF storage.

\section{\lit{pntshow}}
\label{ff,hdf,pntshow}

The \lit{pntshow} program is a versatile tool for extracting point
data  from HDF files containing scientific datasets and Vsets. The
extraction can be done into any binary or ASCII format described in a
FreeForm format description file. Before using \lit{pntshow} on an HDF
file, you should pack the file using the NCSA-developed HDF utility
hdfpack.

You can use \lit{pntshow} to extract headers and data from an HDF file
into separate files or to extract just the data. It's a good idea to
define GeoVu keywords in an equivalence table to facilitate access to
HDF objects. For information about equivalence tables, see the GeoVu
Tools Reference Guide. The input and output formats must be described
in a FreeForm format description file. You should follow FreeForm
naming conventions for the data and format files. For details about
FreeForm conventions, see \chapterref{ff,convs}.

If a format description file is not specified on the command line, the
output format is taken by default from the FreeForm output format
annotation stored in the HDF file. If there is no annotation, a
default ASCII output format is used.

\note{An equivalence table takes precedence over
everything. (vdata=1963, SDS=702)}

If you have not specified an HDF object in an equivalence table,
\lit{pntshow} uses the following sequence to determine the appropriate
source for output:

\begin{enumerate}
\item Output the first vdata with class name Data.

\item Output the largest vdata. 

\item Output the first SDS. 
\end{enumerate}

If no vdatas exist in the file, but an SDS is found, it is extracted
and a default ASCII output format is used.

\subsection{Extracting Headers and Data}

The \lit{pntshow} command takes the following form when you want to
extract headers and data from HDF files into separate files.

\newcommand{\hdfh}{\hbox{\lit{-h} [\var{output\_header\_file}]}}
\newcommand{\hdfhof}{\hbox{\lit{-hof} \var{output\_header\_format\_file}}}
\renewcommand{\hdfd}{\hbox{\lit{-d} [\var{output\_data\_file}]}}
\renewcommand{\hdfdof}{\hbox{\lit{-dof} \var{output\_data\_format\_file}}}
\proto{\lit{pntshow} \var{input\_HDF\_file} [\hdfh] [\hdfhof]
[\hdfhof] [\hdfd] [\hdfdof]}

\begin{description}

\item[\var{input\_HDF\_file}]
  
  Name of the input HDF file, which has been packed using
  \lit{hdfpack}.

\item[\hdfh]\indc{h@\lit{-h}}
  
  Option flag followed optionally by the name of the file designated
  to contain the record headers currently stored in a vdata with a
  class name of Index. If an output header file name is not specified,
  the default is standard output.

\item[\hdfhof]\indc{of@\lit{-of}}
  
  Option flag followed by the name of the FreeForm format file that
  describes the format for the headers extracted to standard output or
  output\_header\_file.

\item[\hdfd]\indc{d@\lit{-d}}
  
  Option flag followed optionally by the name of the file designated
  to contain the data currently stored in a vdata with a class name of
  Data. If an output file name is not specified, the default is
  standard output.

\item[\hdfdof]\indc{dof@\lit{-dof}}
  
  Option flag followed by the name of the FreeForm format file that
  describes the format for data extracted to standard output or
  \var{output\_data\_file}.
\end{description}

\subsubsection{Example}

You will extract data and headers from \lit{hara.HDF} (created by
\lit{splitdat} in a previous example). This file contains two vdatas:
one has the class name Data and the other has the class name Index.
Because this file is extremely small, no appending links were created
in the file, so there is no need to pack the file before using
\lit{pntshow}, though you can if you wish.

To extract data and headers from \lit{hara.HDF}, enter the following
command:

\begin{example}
pntshow hara.HDF -d haradata.dat -h harahdrs.dat 
\end{example}

The data from the vdata designated as Data in \lit{hara.HDF} are now
stored in \lit{haradata.dat}. The data are in their original format
because the original output format was stored by \lit{splitdat} in the
HDF file. The header data from the vdata designated as Index in
\lit{hara.HDF} are now stored in \lit{harahdrs.dat}. In addition to
the original header data, the variables begin and extent have also
been extracted to \lit{harahdrs.dat}.

\subsection{Extracting Data Only}

The \lit{pntshow} command takes the following form when you want to
extract just the data from an HDF file:

\newcommand{\hdfof}{\hbox{\lit{-of} \var{default\_output\_format\_file}}}
\proto{\lit{pntshow} \var{input\_HDF\_file} [\hdfof] 
  \hbox{[> \var{output\_file}]}}

\begin{description}
\item[\var{input\_HDF\_file}]
  
  Name of the input HDF file, which has been packed using \lit{hdfpack}.

\item[\hdfof]\indc{of@\lit{-of}}
  
  Option flag followed by the name of the FreeForm format file that
  describes the format for data extracted to standard output or
  \var{output\_file.}

\item[\var{output\_file}]
  
  Name of the output file into which data is transferred. If an output
  file name is not specified, the default is standard output.
\end{description}

\subsubsection{Examples}

You can use \lit{pntshow} to extract designated variables from an HDF
file. In this example, you will extract temperature and pressure
values from \lit{hara.HDF} to an ASCII format. First, the following
format description file must exist.

Here is \lit{haradata.fmt}:

\begin{vcode}{ib}
ASCII\_output_data "ASCII format for pressure, temp"
atmospheric_pressure 1 10 long 1
temperature_deg 15 25 float 1
\end{vcode}

To create a file named \lit{temppres.dat} containing only the
temperature and pressure variables, enter either of the following
commands:

\begin{example}
pntshow hara.HDF -of haradata.fmt > temppres.dat 
\end{example}

or 

\begin{example}
pntshow hara.HDF -d temppres.dat -dof haradata.fmt 
\end{example}

If you use the first command, \lit{pntshow} searches \lit{hara.HDF}
for a vdata named Data. Since \lit{hara.HDF} contains only one vdata
named \lit{Data}, this vdata is extracted by default with the format
specified in \lit{haradata.fmt}.

The results are the same if you use the second command. Now, try
running \lit{pntshow} on the previously created file \lit{latlon.HDF},
which contains two SDS's. Use the following command:

\begin{example}
pntshow latlon.HDF > latlon.SDS 
\end{example}

The \lit{latlon.SDS} file now contains the latitude and longitude
values extracted from \lit{latlon.HDF}. They have the default ASCII
output format. You could have used the -of option to specify an output
format included in a FreeForm format description file.

%%% Local Variables: 
%%% mode: latex
%%% TeX-master: t
%%% End: 
