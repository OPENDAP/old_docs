% Preface to the DODS FF Server Manual
%
% $Id$
%

\T\chapter*{Preface}
\T\addcontentsline{toc}{chapter}{Preface}

This document describes the DODS FreeForm ND Server.  It is not a
complete description of the \ffnd\ software.  For that, please refer
to the \ffnd\ manual.

This document contains much material originally written at the
National Oceanic and Atmospheric Administration's National
Environmental Satellite, Data, and Information Service, which is part
of the National Geophysical Data Center in Boulder, Colorado.

This document has been updated to include information on FreeForm ND,
the last release of FreeForm. FreeForm is now supported only for use
with the Distributed Oceanographic Data System; see the 
\DODShome\ for more information.

We are interested in your comments about the DODS software, and the
FreeForm ND software and this document.  Send them to:
\xlink{support@unidata.ucar.edu}{mailto:support@unidata.ucar.edu}.

\begin{ifhtml}
  \htmlmenu{4}
  \chapter*{Preface}
\end{ifhtml}

Using \ffnd\ with OPeNDAP, a researcher can easily make his or her data
available to the wider community of OPeNDAP users without having to
convert that data into another data file format.  This document
presents the \ffnd software, and shows how to use it with the OPeNDAP server.

%%%%%%%%%%%%%%%%%%%%%%%%%%%%%%%%%%%%%%%%%%%%%%%%%%%%%%%%%%%%%%%%%%%%
\section{Tasks Illustrated in this Guide}
\label{pref,tasks}

For a quick start to getting, installing, and using the \ffnd\
software, see the list below of tasks described in this document.

\begin{itemize}

\item Quick start. (\pagexref{a})

\item Getting and installing the \ffnd\ software. 
(\pagexref{a})

\item Installing the rest of the OPeNDAP Server software. 
(\pagexref{a})

\item Troubleshooting. (\pagexref{a})

\end{itemize}


%%%%%%%%%%%%%%%%%%%%%%%%%%%%%%%%%%%%%%%%%%%%%%%%%%%%%%%%%%%%%%%%%%%%
\section{Who is this Guide for?}
\label{pref,who}

This guide is for people who wish to use \ffnd\ to serve scientific
datasets using the OPeNDAP software.  Scientists who wish to share their
data with colleagues may also find this a useful system, since it is a
relatively simple matter to set up a server that can allow remote
access to your data.

This documentation assumes that the readers are familiar with
computers and the internet, but are not necessarily programmers. More
than a passing familiarity with different data file formats will be
useful, as will an understanding of elementary internet concepts, such
as URLs and http.

This manual also assumes some familiarity with the OPeNDAP software.  If
you are starting from scratch, knowing nothing at all about OPeNDAP, we
strongly encourage you to browse the \DODSuser\ before reading too far
here.  

%%%%%%%%%%%%%%%%%%%%%%%%%%%%%%%%%%%%%%%%%%%%%%%%%%%%%%%%%%%%%%%%%%%%
\section{Organization of this Document}

This book contains both introductory and reference material. There is
also a description of the installation procedure.

\begin{description}

\item[\chapterref{ff,dintro}] contains an overview of the \ffs\
  software, including how to get it and install it.

\item[\chapterref{ff,dquick}] provides a brief introduction to writing
  format descriptions and using the \ffs .
  
\item[\chapterref{ff,tblfmt}]provides detailed information about
  writing format descriptions to facilitate access to data in tabular
  formats.
  
\item[\chapterref{ff,arrayfmt}] provides detailed information about
  writing format descriptions to facilitate access to data in
  non-tabular (array) formats.
  
\item[\chapterref{ff,hdrfmt}] tells you how to work with header
  formats.

\item[\chapterref{ff,ff-server}] describes the operation of the \ffs ,
  with tips for writing format files.
  
\item[\chapterref{ff,convs}] presents FreeForm ND file name
  conventions, the search rules for locating format files, and
  standard command line arguments for FreeForm ND programs.
  
\item[\chapterref{ff,fmtconv}] shows you how to use the FreeForm ND
  program \lit{newform} to convert data from one format to another and
  also how to read the data in a binary file.
 
\item[\chapterref{ff,datachk}] discusses the FreeForm ND program
  \lit{checkvar}, which you can use to check data distribution and
  quality.

\item[\appref{ff,hdf}] provides explanations for a small selection
  of tools that will be useful for programmers working with the HDF
  file format.
  
\item[\appref{ff,errors}] presents a list of common FreeForm ND error
  messages.  These are the error messages that may be issued by the
  \ffnd\ utilities, such as \lit{newform}, not the \ffs .

\end{description}

%%%%%%%%%%%%%%%%%%%%%%%%%%%%%%%%%%%%%%%%%%%%%%%%%%%%%%%%%%%%%%%%%%%%

%\newcommand{\DODSNoButtons}{1}
\listconventions

A position box is often used in this book to indicate column position
of field values in data files. It is shown at the beginning of a data
list in the documentation, but does not appear in the data file
itself.  It looks something like this:

\begin{vcode}{sib}
1         2         3         4         5         6
012345678901234567890123456789012345678901234567890
\end{vcode}


\W\section{Notices}





%%% Local Variables: 
%%% mode: latex
%%% TeX-master: t
%%% TeX-master: t
%%% End: 
