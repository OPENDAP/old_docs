%CHAPTER 3       
%
% $Id$
%

\chapter{Format Descriptions for Tabular Data}
\label{ff,tblfmt}

\indc{format descriptions}\indc{descriptions!format}
Format descriptions define the formats of input and output data and
headers. FreeForm ND provides an easy-to-use mechanism for describing
data. FreeForm ND programs and FreeForm ND-based applications that you
develop use these format descriptions to correctly access data. Any
data file used by FreeForm ND programs must be described in a format
description file.

This chapter explains how to write format descriptions for data
arranged in tabular format---rows and columns---only. For data in
non-tabular formats, see the next chapter.

\section{FreeForm ND Variable Types}
\label{ff,tblfmt,vartypes}

The data sets you produce and use may contain a variety of variable
types. The characteristics of the types that FreeForm ND supports are
summarized in \tableref{ff,tab,datatypes}, which is followed by a
description of each type.

\begin{ifclear}{ffs}
\begin{table}[htb]
\caption{\ffnd\ Data Types}
\label{ff,tab,datatypes}
\begin{center}
\begin{tabular}{|p{0.75in}|p{1.0in}|p{1.0in}|p{0.6in}|p{0.65in}|} \hline
\tblhd{Name} & \tblhd{Minimum Value} & \tblhd{Maximum Value} & 
\tblhd{\texonly{\raggedright} Size in Bytes} & \tblhd{Precision}* \\ \hline 
\lit{char} &  &  & ** & \\ \hline 
\lit{uchar} & 0 & 255 & 1 & \\ \hline 
\lit{short} & -32,767 & 32,767 & 2 & \\ \hline 
\lit{ushort} & 0 & 65,535 & 2 & \\ \hline 
\lit{long} & -2,147,483,647 & 2,147,483,647 & 4 & \\ \hline 
\lit{ulong} & 0 & 4,294,967,295 & 4 & \\ \hline 
\lit{float} & $10^{-37}$ & $10^{38}$ & 4 & 6***\\ \hline 
\lit{double} & $10^{-307}$ & $10^{308}$ & 8 & 15***\\ \hline 
\lit{constant} &  &  & ** & \\ \hline 
\lit{initial} &  &  & record length & \\ \hline 
\lit{convert} &  &  & ** & \\ \hline 
\multicolumn{5}{l}{* Expressed as the number of significant digits} \\
\multicolumn{5}{l}{** User-specified} \\
\multicolumn{5}{l}{*** Can vary depending on environment}
\end{tabular}
\end{center}
\end{table}
\end{ifclear}
\begin{ifset}{ffs}
\begin{table}[htb]
\caption{\ffnd\ Data Types}
\label{ff,tab,datatypes}
\begin{center}
\begin{tabular}{|p{0.75in}|p{1.0in}|p{1.0in}|p{0.6in}|p{0.65in}|} \hline
\tblhd{Name} & \tblhd{Minimum Value} & \tblhd{Maximum Value} & 
\tblhd{\texonly{\raggedright} Size in Bytes} & \tblhd{Precision}* \\ \hline 
\lit{char} &  &  & ** & \\ \hline 
\lit{uchar} & 0 & 255 & 1 & \\ \hline 
\lit{short} & -32,767 & 32,767 & 2 & \\ \hline 
\lit{ushort} & 0 & 65,535 & 2 & \\ \hline 
\lit{long} & -2,147,483,647 & 2,147,483,647 & 4 & \\ \hline 
\lit{ulong} & 0 & 4,294,967,295 & 4 & \\ \hline 
\lit{float} & $10^{-37}$ & $10^{38}$ & 4 & 6***\\ \hline 
\lit{double} & $10^{-307}$ & $10^{308}$ & 8 & 15***\\ \hline 
%\lit{constant} &  &  & ** & \\ \hline 
%\lit{initial} &  &  & record length & \\ \hline 
%\lit{convert} &  &  & ** & \\ \hline 
\multicolumn{5}{l}{* Expressed as the number of significant digits} \\
\multicolumn{5}{l}{** User-specified} \\
\multicolumn{5}{l}{*** Can vary depending on environment}
\end{tabular}
\end{center}
\end{table}
\end{ifset}


\note{The sizes in \tableref{ff,tab,datatypes} are machine-dependent.
  Those given are for most Unix workstations.  }

\begin{description}
\item[\lit{char}]
  
  The \lit{char} variable type is used for character strings.
  Variables of this type, including numerals, are interpreted as
  characters, not as numbers.

\item[\lit{uchar}]
  
  The \lit{uchar} (unsigned character) variable type can be used for
  integers between 0 and 255 (28- 1). Variables that can be
  represented by the \lit{uchar} type (for example: month, day, hour,
  minute) occur in many data sets. An advantage of using the \lit{uchar}
  type in binary formats is that only one byte is used for each
  variable. Variables of this type are interpreted as numbers, not
  characters.

\item[\lit{short}]
  
  A \lit{short} variable can hold integers between -32,767 and 32,767
  ($2^{15}- 1$). This type can be used for signed integers with less
  than 5 digits, or for real numbers with a total of 4 or fewer digits
  on both sides of the decimal point (-99 to 99 with a precision of 2,
  -999 to 999 with a precision of 1, and so on).

\item[\lit{ushort}]
  
  A \lit{ushort} (unsigned short) variable can hold integers between 0
  and 65,535 ($2^{16} - 1$).

\item[\lit{long}]
  
  A \lit{long} variable can hold integers between -2,147,483,647 and
  +2,147,483,647 ($2^{31} - 1$). This variable type is commonly used
  to represent floating point data as integers, which may be more
  portable. It can be used for numbers with 9 or fewer digits with up
  to 9 digits of precision, for example, latitude or longitude
  (-180.000000 to 180.000000).

\item[\lit{ulong}]
  
  The \lit{ulong} (unsigned long) variable type can be used for
  integers between 0 and 4,294,967,295 ($2^{32} - 1$).

\item[\lit{float, double}]
  
  Numbers that include explicit decimal points are either \lit{float}
  or \lit{double} depending on the desired number of digits. A
  \lit{float} has a maximum of 6 significant digits, a \lit{double}
  has 15 maximum. The extra digits of a \lit{double} are useful, for
  example, for precisely specifying time of day within a month as
  decimal days. One second of time is approximately 0.00001 day. The
  number specifying day (maximum = 31) can occupy up to 2 digits. A
  \lit{float} can therefore only specify decimal days to a whole
  second (31.00001 occupies seven digits). A \lit{double} can,
  however, be used to track decimal parts of a second (for example,
  31.000001).

\begin{ifclear}{ffs}
\item[\lit{constant}]
  
  FreeForm ND has two variable types, \lit{constant} and
  \lit{initial}, for sequences of characters (or bytes) that are the
  same for all records in a file. A \lit{constant} variable is placed
  into the output buffer on initialization. The \lit{constant} value
  is the same as the name of the variable. For example, given the
  variable description below:

\begin{example}
NGDCDATA 1 8 constant 0 
\end{example}

the string NGDCDATA, which is both the variable name and value, is
placed in characters 1-8 of each output record.

FreeForm ND recognizes the special \lit{constant} type \lit{NEWLINE}
as an end-of-line character, which is used with multi-line records.
The variable descriptions shown next are for a data record that
includes several variables, the end-of-line character, then several
more variables.

\begin{example}
year 1 2 short 0
       .
       .   (more variables)
       .
latitude_sec 75 80 float 2
EOL 81 82 NEWLINE 0
longitude_deg 83 89 float 3
longitude_min 72 78 float 2
       .
       .   (remaining variables)
       . 
\end{example}

The variable \lit{longitude\_deg} starts a new line in the data file. 

\item[\lit{initial}]
  
  The variable type \lit{initial} can be used when you want to set
  more than one constant value at a time. It provides an
  initialization template for the output record. This template is read
  from a file with the same name as the \lit{initial} variable. For
  example, suppose you have the following variable description:

\begin{example}
seattle.ini 1 80 initial 0 
\end{example}

The initial variable is named \lit{seattle.ini}, so the initialization
template file \lit{seattle.ini} is read and used to initialize the
output records. Assume the Seattle template contains the following
values, which are written to an earthquake record:

\begin{example}
SEA19                                   SEA          
\end{example}

The other values in the output record are written over this template
resulting in a record that looks like the following:

\begin{example}
SEA19  5  -146.34172   -47.39710   1011 SEA   910802 
\end{example}

\note{The length of the template file must equal the length of the
  record in the output format. The file name and extension are of your
  choosing. }

\item[\lit{convert}]
  
  The \lit{convert} variable type allows you to access an extensive
  set of functions for constructing output variables that do not exist
  in input files, but can be computed from variables which do.
  FreeForm ND can transparently identify and call conversion functions
  during the data access process if you use properly named input and
  output variables in variable descriptions.
  
  See \chapterref{ff,convvars} for examples and
  \chapterref{ff,varname} for a complete list of names for conversion
  variables.

\end{ifclear}

\item[\lit{header}]
  
  Older versions of FreeForm ND included header variables. You can now
  specify header formats in format description files.

  For details, see \sectionref{ff,tblfmt,formatdesc} and also
  \chapterref{ff,hdrfmt}.

\end{description}

\section{FreeForm ND File Types}

\indc{binary!file type}\indc{file type!binary}
\indc{ASCII!file type}\indc{file type!ASCII}
\indc{dBASE!file type}\indc{file type!dBASE}
FreeForm ND supports binary, ASCII, and dBASE file types. Binary data
are stored in a fixed amount of space with a fixed range of values.
This is a very efficient way to store data, but the files are
machine-readable rather than human-readable. Binary numbers can be
integers or floating point numbers.

Numbers and character strings are stored as text strings in ASCII. The
amount of space used to store a string is variable, with each
character occupying one byte.

The dBASE file type, used by the dBASE product, is ASCII text without
end-of-line markers.

\section{Format Description Files}

\indc{fmt@\lit{.fmt} file}
\indc{format descriptions!example}\indc{descriptions!format}
Format description files accompany data files. A format description
file can contain descriptions for one or more formats. You include
descriptions for header, input, and output formats as appropriate.
Format descriptions for more than one file may be included in a single
format description file.

An example format description file is shown next. The sections that
follow describe each element of a format description file.

\begin{vcode}{nib}
/ This format description file is for
/ data files latlon.bin and latlon.dat. 

binary_data "Default binary format"
latitude 1 4 long 6
longitude 5 8 long 6

ASCII_data "Default ASCII format"
latitude 1 10 double 6
longitude 12 22 double 6
\end{vcode}

Lines 1 and 2 are comment lines.  Lines 4 and 8 give the format type
and title, as described in \sectionref{ff,tblfmt,format}. Lines 5, 6,
9, and 10 contain variable descriptions, described in
\sectionref{ff,tblfmt,var}.  Blank lines signify the end of a format
description

You can include blank lines between format descriptions and comments
in a format description file as necessary.  Comment lines begin with a
slash (/). FreeForm ND ignores comments.

\section{Format Descriptions}
\label{ff,tblfmt,formatdesc}

A format description file comprises one or more format descriptions. A
format description consists of a line specifying the format type and
title followed by one or more variable descriptions, as in the
following example: 

\begin{vcode}{ib}
binary_data "Default binary format"
latitude 1 4 long 6
longitude 5 8 long 6
\end{vcode}

\subsection{Format Type and Title}
\label{ff,tblfmt,format}

\indc{format!title}\indc{title!format}
\indc{format!type}\indc{type!format}
A line specifying the format type and title begins a format
description. A \new{format descriptor}, for example, \lit{binary\_data}, is
used to indicate format type to FreeForm ND. The \new{format title}, for
example, ``Default binary format'', briefly describes the format. It
must be surrounded by quotes and follow the format descriptor on the
same line. The maximum number of characters for the format title is 80
including the quotes.

\subsubsection{Format Descriptors}

\indc{format!descriptor}\indc{descriptor!format}
Format descriptors indicate (in the order given) file type, read/write
type, and file section. Possible values for each descriptor component
are shown in the following table.

\begin{table}[htb]
\caption{Format Descriptor Components}
\begin{center}
\begin{tabular}[t]{|l|p{1.2in}|l|} \hline
\tblhd{File Type} & \tblhd{Read/Write Type
 (optional)} & \tblhd{File Section} \\ \hline 
\begin{tabular}{l}
\lit{ASCII} \\
\lit{binary} \\
\lit{dBASE} \\
\end{tabular}
& 
\begin{tabular}{l}
\lit{input}\\
\lit{output} \\
\end{tabular}
& 
\begin{tabular}{l}
\lit{data} \\
\lit{file\_header} \\
\lit{record\_header} \\
\lit{file\_header\_separate}* \\
\lit{record\_header\_separate}* \\
\end{tabular}
 \\ \hline 
\multicolumn{3}{p{4.5in}}{* The qualifier \lit{separate} indicates
  there is a header file separate from the data file.}
\end{tabular}
\end{center}
\end{table}

The components of a format descriptor are separated by underscores
(\_). For example, \lit{ASCII\_output\_data} indicates that the format
description is for ASCII data in an output file. The order of
descriptors in a format description should reflect the order of format
types in the file. For instance, the descriptor
\lit{ASCII\_file\_header} would be listed in the format description
file before \lit{ASCII\_data}. The format descriptors you can use in
FreeForm ND are listed in the next table, where \lit{XXX} stands for
\lit{ASCII}, \lit{binary}, or \lit{dBASE}.  (Example: \lit{XXX\_data}
= \lit{ASCII\_data}, \lit{binary\_data}, or \lit{dBASE\_data}.)

\begin{table}[htb]
\caption{Format Descriptors}
\begin{center}
\begin{tabular}[t]{|l|l|l|} \hline
\tblhd{Data} & \tblhd{Header} & \tblhd{Special} \\ \hline 
\begin{tabular}{l}
\lit{XXX\_data} \\
\lit{XXX\_input\_data} \\
\lit{XXX\_output\_data} \\
\end{tabular}
& 
\begin{tabular}{l} 
\lit{XXX\_file\_header} \\
\lit{XXX\_file\_header\_separate} \\
\lit{XXX\_record\_header} \\
\lit{XXX\_record\_header\_separate} \\
\lit{XXX\_input\_file\_header} \\
\lit{XXX\_input\_file\_header\_separate} \\
\lit{XXX\_input\_record\_header} \\
\lit{XXX\_input\_record\_header\_separate} \\
\lit{XXX\_output\_file\_header} \\
\lit{XXX\_output\_file\_header\_separate} \\
\lit{XXX\_output\_record\_header} \\
\lit{XXX\_output\_record\_header\_separate} \\
\end{tabular}
& 
\begin{tabular}{l}
\lit{RETURN}* \\
\lit{EOL}** \\
\end{tabular}
\\ \hline 

\multicolumn{3}{p{4.5in}}{* The RETURN descriptor lets FreeForm ND skip
    over end-of-line characters in the data.} \\

\multicolumn{3}{p{4.5in}}{** The EOL descriptor is a constant
    indicating an end-of-line character should be inserted in a
    multi-line record.} \\
\end{tabular}
\end{center}
\end{table}

For more information about header formats, see \chapterref{ff,hdrfmt}
on page 89.





\subsection{Variable Descriptions}
\label{ff,tblfmt,var}

\indc{variable descriptions}\indc{descriptions!variable}
A variable description defines the name, start and end column
position, type, and precision for each variable. The fields in a
variable description are separated by white space. Two variable
descriptions are shown below with the fields indicated. Each field is
then described.

Here are two example variable descriptions.  Each one consists of a
name, a start position, and end position, a type, and a precision.

\begin{vcode}{ib}
latitude    1  10  double  6
longitude  12  22  double  6
\end{vcode}

\begin{description}
\item[Name]
  
  The variable name is case-sensitive, up to 63 characters long with
  no blanks. The variable names in the example are latitude and
  longitude. If the same variable is included in more than one format
  description within a format description file, its name must be the
  same in each format description.

\item[Start Position]
  
  The column position where the first character (ASCII) or byte
  (binary) of a variable value is placed. The first position is 1, not
  0. In the example, the variable latitude is defined to start at
  position 1 and longitude at 12.

\item[End Position]
  
  The column position where the last character (ASCII) or byte
  (binary) of a variable value is placed. In the example, the variable
  latitude is defined to end at position 10 and longitude at 22.

\item[Type]
  
  The variable type can be a standard type such as char, float,
  double, or a special FreeForm ND type. The type for both variables
  in the example is double. See \sectionref{ff,tblfmt,vartypes} for
  descriptions of supported types.

\item[Precision]
  
  Precision defines the number of digits to the right of the decimal
  point. For float or double variables, precision only controls the
  number of digits printed or displayed to the right of the decimal
  point in an ASCII representation. The precision for both variables
  in the example is 6.
\end{description}


%%% Local Variables: 
%%% mode: latex
%%% TeX-master: t
%%% End: 
