
% $Id$

\chapter{Making Your Data Available}
\labl{register}

%\section{Why Register Your DODS Data Server}
\labl{register:service}

The DODS data registration scheme is currently undergoing revision. Once
revised the new scheme will be documented here.

%In order for the DODS data locator to work, data providers must
%register their data archives in a place where the locator can find
%it. Of course, there is no mandated central authority over DODS data
%sets. This registration is done entirely at the data provider's
%discretion. In this respect, the DODS hierarchy mirrors the structure
%of the Web itself.

%However, also like the Web itself, if DODS data are not registered in
%a central location, data users may never find them except by accident.
%The DODS Dataset Registration Service allows data set providers to
%document and register their data with the Global Change Master
%Directory, GCMD, one component of the DODS Locator Service. The
%registration service includes a user interface that aids data set
%providers in the creation of a Directory Interchange Format (DIF) data
%structure, the input format used to load directory level information
%into the GCMD.

%Because DODS is a system which will support many distributed data
%resources it is necessary to provide a data location service for users
%that allows them to determine quickly what data sets are available
%from DODS servers. The DODS Data Location Service uses the NASA Global
%Change Master Directory to provide directory level information on DODS
%data sets as well as the URL of their DODS servers.  The GCMD is an
%on-line directory of Earth science data sets that supports
%multi-parameter searching on its relational database management system.
%Data providers who have installed DODS servers and intend to allow
%others access to their on-line data sets via DODS should register
%their data sets with the GCMD. The DODS Registration Service is
%intended to aid data providers in providing the data set and server
%information required to create a reference in the GCMD.

%\subsection{DIF Data Structure}

%The standard input data structure for the GCMD is called the Directory
%Interchange Format or DIF.  A DIF is a complex data structure that
%contains information used to support both relational and text based
%searching within the GCMD.  The types of information contained in a
%DIF include: data set attributes; archive and data provider
%descriptions; and general summary information.  The documentation on
%the DIF standard is available at {\tt
%http://gcmd.gsfc.nasa.gov/docs.html}.

%The DODS registration service generates a data set DIF based on
%information supplied interactively by the data provider and
%information extracted automatically from DODS servers of the data sets
%being registered.  The data provider reviews, modifies and submits the
%registration service DIF controlling when the DIF is registered and
%its content.

%\subsection{Registration Service Component Architecture}

%The Registration service components and their architecture are
%depicted in Figure~\refl{register:fig:service}.  The components are
%the Data Provider Input form, a DIF generation CGI, a DIF review and
%modification form and a DIF uploader CGI that submits the completed
%DIF to the GCMD.  Each is described below.

%\begin{figure}[htbp]
%\centerline{\psfig{figure=DRS_fig1.ps}}
%\caption{Registration Service Components}
%\label{register:fig:service}
%\end{figure}

%\subsubsection{Data Provider Input Form}

%When a data provider wants to register a data set he or she pulls up
%the Data Provider Input form, shown in
%Figure~\refl{register:fig:input}.  
%This form is stored at:

%\begin{center}
%{\tt \$DODS\_ROOT/etc/dataset\_registration.html}
%\end{center}

%Use a web browser like Netscape to open the form.  (For Netscape, use the
%Open File option on the File menu.)
%This form takes as its input the
%name of the data set being registered, the address of the DODS server
%that accesses the data set, the names and location of the DODS CGIs
%that describe that data set and the email address of the data provider.
%In addition the data provider can submit the name of a file containing
%descriptive information on the data set.  Once the information has been
%filled in the data provider hits the ``Create DIF'' button to post the
%form to the DIF generation CGI.  The following input fields and the
%information required are in the Data Provider Input form:

%\begin{figure}[tbp]
%\centerline{\psfig{figure=DRS_fig2gs.ps}}
%\caption{Registration Input Form}
%\label{register:fig:input}
%\end{figure}

%\begin{description}

%\item{\bf DODS Server Address:}  This field requires the URL address of the
%  machine where the DODS server is installed, such as {\tt
%  http://dods.gso.uri.edu}.

%\item{\bf DDS CGI Name and Location:} This is the specification of the DDS CGI
%  that access the data set being registered.  The location is relative
%  to the httpd home directory defined at the time the httpd
%  server is configured (e.g., ScriptAlias - /cgi-bin/ =
%  /usr/local/etc/httpd/cgi-bin/).  An example of a DDS specification,
%  /cgi-bin/DODS-cgi/ctd$\_$DDS.
 
%\item{\bf DAS CGI Name and Location:} Equivalent to the DDS only for the DAS
%  CGI.
  
%\item{\bf Data Set name} The name of the data set.  This name can be a
%  discrete filename or a reference name.  The actual value will depend on the
%  CGI filters that access the data set.  For example, in the JGOFS system
%  data set names actually reference JGOFS objects.  The DODS CGIs use a JGOFS
%  object lookup table to reference the actual JGOFS data file and its access
%  methods.  For a NetCDF implementation the data set name must include the
%  directory location of the data file being accessed.  The directory location
%  can be either relative to the document root of the httpd server (such as
%  {\tt /usr/local/spool/data/DODS/fnoc1.nc}) or an alias that is defined when
%  the httpd server is configured.
  
%\item{\bf Contact EMAIL} The email address of the person registering the data
%  set is requested.

%\item{\bf Other Dataset Description Information}  The name of a file that
%  contains summary information.  This function is not implemented at
%  present.  In the future it will allow the data provider to specify a
%  text file to also be used to help generate the DIF valids for the
%  data set.

%\end{description}

%\subsubsection{DIF Generation CGI}

%This CGI reads the input data from the Data Provider Input form.
%Using the DODS server URL, the CGI names and the data set name it next
%accesses the CGIs of the data set being registered and reads in its DDS
%and DAS.  The DDS and DAS structures are parsed into words and each
%word is submitted to a DIF thesaurus which matches input words with
%words found in the complete set of DIF valids.  The DIF thesaurus
%outputs a set of DIF valids based on the words in the DAS and DDS that
%are separated DIF attribute categories, Sensor, Source, Parameter,
%Discipline and Location. Figure~\refl{register:fig:dif} is a data flow
%diagram for the DIF generator CGI.  The output from the DIF thesaurus
%along with the information from the Data Provider Input form are then
%used to create the DIF Review and Modification form.

%\begin{figure}[htbp]
%\centerline{\psfig{figure=DRS_fig3.ps}}
%\caption{DIF Generator Data Flow}
%\label{register:fig:dif}
%\end{figure}

%\subsubsection{DIF Review and Modification Form}

%This html form (Figures~\refl{register:fig:review} and
%\refl{register:fig:modify} allows the data provider an opportunity to review
%and modify the DIF valids that describe the data set being registered.  The
%most critical information is that in the DIF Valids Generated from Dataset
%Descriptor Files section of this form.  The data provider must select all
%appropriate DIF valids from the lists provided under each of the DIF
%categories in the form (Sensor, Source, Parameter, Discipline and Location).
%The form permits the specification of multiple values for any of the
%categories by simply clicking on the value.  In cases where no DIF valids
%were generated from the DIF thesaurus for a particular DIF category, a
%default list of all valid DIF values for that category is provided.

%Once the data provider has reviewed the form he or she may then submit
%the completed DIF to the GCMD for uploading by hitting the Submit DIF button.

%\begin{figure}[htbp]
%\centerline{\psfig{figure=DRS_fig4a.ps}}
%\caption{DIF Review Form}
%\label{register:fig:review}
%\end{figure}

%\begin{figure}[htbp]
%\centerline{\psfig{figure=DRS_fig4b.ps}}
%\caption{DIF Modify Form}
%\label{register:fig:modify}
%\end{figure}

%\subsubsection{DIF Upload CGI}

%This CGI accepts input from the DIF Review and Modification Form and constructs
%a valid GCMD DIF for uploading.  The DIF is then evaluated for 
%completeness and either accepted or rejected for uploading into the GCMD.
%In either case an acknowledgment and copy of the submitted DIF is returned
%to the data provider along with an indication of whether the DIF was accepted
%or rejected.
%\tbd{How is it returned?}

%\section{Creating the Data Catalog}
%\labl{register:catalog}

%\tbd{Where does the data go?  What does it look like?} The data catalog is
%simply a text (or html) file, generally called \emph{API}{\tt
%  \_catalog\_data} (where API is the name of the DODS server, for example
%{\tt jg\_catalog\_data}), that contains the catalog of data served by this
%host.  The file must be created with any text editor, and is stored in the
%hypertext document root directory.

%A user may examine the data catalog for some server with a standard web browser
%like Netscape. There is a catalog at the DODS home server, that may be accessed
%with the following URL:

%{\tt http://dods.gso.uri.edu/cgi-bin/jg/catalog}

%\tbd{Finish this section. Finish the software that it documents. Are they using
%the usage service thing?  What about constraint expressions?}

%\section{The Boolean Service}
%\indc{Boolean Service} \indc{Service!Boolean}

%The Boolean Service CGI filter program takes a single dataset name and the
%optional constraint expression as its input.  The output indicates the status
%of the dataset.  It returns a result that indicates one of two possible
%conditions:

%Note that this result is meant to be read by a program, not a person.  The
%Boolean Service is designed to be part of a URL in another URL's constraint
%expression, for example.  No output that may be displayed by a standard web
%browser will be output from this service.
