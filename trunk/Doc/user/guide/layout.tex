%%% Tex customizations and command definitions for the DODS user
%%% guide, 15 March 1996 - tomfool
%%%
%%% $Id$

%%%%%%%%%%%%%%%%%%%%%%%%%%%%%%%%%%%%%%%%%%%%%%%%%%%%%%%%%%%%%%%%%%%%%
%%% The following are commands for the draft version. They should be
%%% changed or expunged (or disabled) from the book before the final
%%% version. 

\newcommand{\version}{Version 1.0}
\newcommand{\showtime}{\the\time}

\ifthenelse{\value{draft-document}=1}{%
  \newcommand{\tbd}[1]{\marginpar{\scriptsize{{\bf TBD}: #1}}}
  %% The \ind command puts an index token in the line and a margin
  %% note, and prints the argument where it lies. It is meant to save
  %% the retyping, as in complicated concept\index{complicated concept}.
  %% The \indc command does not print the argument in the line. When
  %% implemented, the optional argument to \ind will be used to control
  %% the index token type, and possibly to indicate the concept should
  %% be permuted.
  \newcommand{\ind}[2][0]{#2\index{#2}%
    \marginpar{\scriptsize{\ensuremath{\cal\ I}}: #2}}
  \newcommand{\indc}[1]{\index{#1}%
    \marginpar{\scriptsize{\ensuremath{\cal\ I}}: #1}}
  \newcommand{\labl}[1]{\label{#1}%
    \marginpar{\scriptsize{\ensuremath{\cal\ L}}: #1}}
  \newcommand{\refl}[1]{\ref{#1}%
    \marginpar{\scriptsize{\ensuremath{\cal\ R}}: #1}}
  \newcommand{\refb}[1]{\marginpar{\scriptsize{\ensuremath{\cal\ R}}: #1}}
  \newcommand{\citel}[1]{\cite{#1}%
    \marginpar{\scriptsize{\ensuremath{\cal\ C}}: #1}}%
}{%
  \newcommand{\tbd}[1]{ }
  \newcommand{\ind}[2][0]{#2\index{#2}}
  \newcommand{\indc}[1]{\index{#1}}
  \newcommand{\labl}[1]{\label{#1}}
  \newcommand{\refl}[1]{\ref{#1}}
  \newcommand{\refb}[1]{ }
  \newcommand{\citel}[1]{\cite{#1}}%
}

%%%%%%%%%%%%%%%%%%%%%%%%%%%%%%%%%%%%%%%%%%%%%%%%%%%%%%%%%%%%%%%%%%%%%
%%% The following commands are for the real version of the book.
%%%
%%% Makes a ``note'' or ``warning'' or ``caution.'' This needs a
%%% little help to keep the ``note'' from hanging into the text of the
%%% note. Good enough for now.
\newcommand{\note}[1]{%
  \begin{list}{\bfseries\ NOTE:}{\rightmargin\leftmargin}%
  \item #1%
  \end{list}}

%%% These commands are for modifying the headers and footers (with
%%% the fancyheadings package)
\setlength\headheight{15pt}
\addtolength{\headwidth}{\marginparsep}
\addtolength{\headwidth}{\marginparwidth}
\renewcommand{\chaptermark}[1]{\markboth{#1}{}}
\renewcommand{\sectionmark}[1]{\markright{\thesection\ #1}}
\lhead[\fancyplain{}{\bfseries\thepage}]
   {\fancyplain{}{\bfseries\rightmark}}
\rhead[\fancyplain{}{\bfseries\leftmark}]
   {\fancyplain{}{\bfseries\thepage}}
%%%                  center footer should be empty for the final draft
\ifthenelse{\value{draft-document}=1}{\cfoot{\today---\version---\showtime}}
{\cfoot{}}

%%% miscellaneous
\newcommand{\clearemptydoublepage}%
   {\newpage{\pagestyle{empty}\cleardoublepage}}
%\renewcommand{\thepart}         {\arabic{part}}

%%% Font change commands:
\newcommand{\new}[1]{{\em \ind{#1}}} % Marks the first occurrence of a term
\newcommand{\var}[1]{{\em #1}}     % Font for variables
\newcommand{\lit}[1]{{\tt #1}}     % Font for literals
\newcommand{\inp}[1]{{\tt #1}}  % Font for user typed stuff

%%% Rearrange paragraph look.
\setlength\parskip{0.7em}
\setlength\parindent{0.0em}

%%% Allow page lengths to vary slightly
\raggedbottom

%%% There should be a set of commands for the list environments and
%%% any others that need to use the paranormal environment. 

%%% Compress table of contents...
\newenvironment{paranormal}{%
  \setlength\parskip{0.0em}%
  \setlength\parindent{0.0em}}%
 {\setlength\parskip{0.7em}%
  \setlength\parindent{0.0em}}
 
  