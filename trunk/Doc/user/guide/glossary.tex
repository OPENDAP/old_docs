% Glossary for the DODS User Guide
%
% $Id$
%
% $Log: glossary.tex,v $
% Revision 1.4  1999/02/04 17:42:13  tom
% modified to use dods-book.cls and Hyperlatex
%

\chapter*{Glossary}
\addcontentsline{toc}{chapter}{Glossary}

\texorhtml
{\newcommand{\glosshead}[1]{%
  \vspace{8pt}\par\noindent\textbf{#1}\normalfont\vspace{2pt}}}%
{\newcommand{\glosshead}[1]{\html{h4}#1\html{/h4}}}

\glosshead{alias}

A synonym.  DODS uses aliases in its attribute-naming scheme.  An
attribute can have an alias, or second name, by which a user can
identify it.  An alias can have aliases of its own, but at this point,
it becomes difficult to keep track of what points to whom, and we do
not recommend this.

\glosshead{Application Program Interface (API)}

An API is simply a collection of functions and data types a program
uses to access some service. The data types and functions defined in
\lit{stdio.h}, such as \lit{printf()}, \lit{fseek()}, \lit{FILE}, and
\lit{fputc()}, constitute a commonly used API for C program I/O.  The
advantage of an API is that it insulates the user from the
implentation details of the program.

\glosshead{Array}

An array is an ordered set of variables. The members of a DODS array
must all be of the same data type. Array members may be accessed with
the \lit{[]} operator. That is, \lit{array[4]} specifies the fifth
member of \lit{array}.  Note that the index of the first array member
is zero. Arrays with multiple dimensions are defined as
single-dimensional arrays of arrays. For example, a two-dimensional
array is an array of arrays.

\glosshead{attribute}

A quality of a data variable.  This could be the method used to
measure the variable's value, the name of the scientist who measured
it, the color of the sky at the time, or whatever might be relevant.

%\glosshead{Catalog Service}

%The Catalog Service provides a user with a list of the data sets
%served by a specific DODS server.

\glosshead{Common Gateway Interface (CGI)}

A CGI program is a program that is executed by a httpd server upon
receiving an appropriately configured URL. The DODS server is a CGI
program. CGI is an Internet standard.  DODS uses version 1.1.1 of the
CGI standard.

\glosshead{compound data type}

A compound data type is one that is constructed from other data
types. A compound type can be built from simple base types or from
other compound data types. \class{Arrays}, \class{Lists},
\class{Grids}, \class{Sequences}, and \class{Structures} are all
compound types, because they are all defined as aggregates of other
data types.

\glosshead{constraint expression}

A constraint expression is appended to a DODS URL to select data from
the data set identified by the URL.

\glosshead{constructor type}

See \emph{Compound Data Type}


\glosshead{container attribute}

A container attribute contains other attributes.  The analogy is with
a compound data variable, such as a \class{Sequence} or
\class{Structure}, that contains other variables.

\glosshead{daemon}

A daemon is simply a process that runs unattended on a UNIX computer.
An \lit{httpd} server is generally run as a daemon. It's not clear how
the peculiar spelling came into use.

\glosshead{Data Access Protocol (DAP)}

The DAP is the method a DODS client uses to retrieve data from a DODS
server.  The DAP consists of an {\em intermediate data
  representation}, an {\em ancillary data format}, a {\em procedure}
for requesting the data from a server, and an {\em API} with which to
execute the protocol.

\glosshead{Data Attribute Structure (DAS)}

This is a DODS construct, showing a list of variables, and attributes
associated with those variables, for a given data set. A variable's
attributes may include such things as the instrument that recorded it,
quality control information and so on.  The response to a \lit{\_das}
request to a DODS server is a DAS.

\glosshead{Data Description Structure (DDS)}

This is a DODS construct, showing a textual representation of a data
set's data model. The response to a {\tt \_dds} request to a DODS
server is a DDS.

\glosshead{data dictionary}

This is a JGOFS construct. The JGOFS API uses data set names instead
of file names to refer to data sets. The API uses the data dictionary
to look up the data set names, where it finds the file names or URLs
to which the name refers.

\glosshead{data model}

The data model of a particular data set can be defined as the set of
relationships between the variables that make up that data set. The
important thing to remember is that it is this relationship that
provides meaning to each of the numbers recorded in that data set. For
example, without the relationship of the adjacent location
measurement, a temperature measurement is just a number with no
meaning.

\glosshead{dataset}

A quantity of data, considered as a unit.  A dataset may occupy one
computer file, or several.  A DODS dataset is a dataset that is served
through a DODS server.

\glosshead{DODS}

Distributed Oceanographic Data System. They wrote a book about it 
once.

\glosshead{global attribute}

A data attribute that applies to an entire dataset.

\glosshead{Grid}

The Grid data type consists of an array with named dimensions, and a
one dimensional array corresponding to each dimension. It is used to
define data grids with irregular spacing.

\glosshead{GUI manager}

The DODS core software can create a client-side sub-process with which
to manage the user's screen display. Most clients adapted to using DODS
will not be able to display intermediate results of a data query, nor will
they be able to make sense of network error messages. The GUI manager
creates a path whereby messages can travel from the DODS server to the
DODS client core software to the user without returning to the client
application. The DODS core software can also use the GUI manager by
itself, without messages from the server.

\glosshead{HTML}

Hyper-text Markup Language. This is the text formatting language in
which web pages are written.


\glosshead{httpd}

The httpd server is the web server. Web clients, such as browsers like
Netscape or Mosaic, send messages to httpd servers on the machine
identified in a URL. The return messages from these servers are the
data that is displayed to the user.

\glosshead{info service}

The \lit{info} service provides information about the usage of a
particular server.  This is meant to include such information as any
functions defined by the server for use in constraint expressions,
error messages, the revision of the server software, and a list of any
data model translations defined for that server.

\glosshead{lazy evaluation}

A method of evaluating a logical expression where evaluation halts
after further evaluation could produce no change in the result. For
example, when evaluating a string of sub-expressions linked by a
logical {\em AND}, a lazy evaluator would halt after the first false
sub-expression, because evaluation of subsequent sub-expressions would
not change the result.

\glosshead{List}

A List is an unordered list of variables. DODS list members must be of
only one data type, but the type may be any of the base or constructor
types.

\glosshead{Sequence}

A Sequence is a data type similar to a structure, but multi-valued. It
is possible to think of a Sequence as an array of Structure values.
Unlike an Array, however, only one value of the Sequence,
corresponding to the Sequence state is available at any one time.

\glosshead{state}

See Sequence. A sequence can be thought of as a multi-values
structure. Unlike an array, where all the variable's values are
available at once, the values of a sequence's members are only
available for the current state. When the state advances, new
variables become available.

\glosshead{stride}

A stride value is used to select a hyperslab from an array. If \lit{d}
is a 10 by 10 array, then \lit{d[0:2:9][0:2:9]} is a hyperslab
consisting of every second point in both dimensions.

\glosshead{Structure}

The Structure data type is a set of variables, similar to a structure
in the language.

\glosshead{Uniform Resource Locator (URL)}

A URL is a name that is unique across the Internet. It is analogous to
a file name on a single machine in that it identifies some resource
that might be data or a program.

\glosshead{Usage Service}

See info service.







%%% Local Variables: 
%%% mode: latex
%%% TeX-master: t
%%% End: 
