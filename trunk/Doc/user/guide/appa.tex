%
%  $Id$
%

\chapter{Installing the \opendap Software}
\label{install}

The current version of the Distributed Oceanographic Data System core
software is \OPDversion .  Note that this number applies only to the
core software.  The individual servers, clients, and client libraries
have their own version numbers.

Full information about the latest versions of the \opendap software is
available from the \opendap web site: \OPDhome .

This version of \opendap uses CGI standard 1.1.1.

\section{Acquiring the \opendap Software}



\subsubsection{The \opendap Web Site}

We recommend that you use the \OPDhome\ to obtain the most current
version of the software.  The ``Software'' page includes a form for
selecting \opendap components.  Completing the form automatically creates
a custom compressed archive of all the software components you
selected, which you can then download to your own machine.

Whenever possible, you should use the provided binary software rather
than trying to compile and link \opendap yourself.  Compiling will work,
but the \opendap software is large, and it takes a long time to compile.

Don't forget to select the core software as well as the specific data
access API or server you need.  For an \opendap client, you will also need
\lit{Tcl/Tk} and \lit{gzip}.


\subsubsection{By Anonymous FTP}

The \opendap software may be down-loaded by anonymous \lit{\ind{ftp}} from
the \OPDftp .

Don't forget to down-load the core software \lit{tar} file as well as
the \lit{tar} file corresponding to whatever data access API you need.

The \opendap project provides a small number of archive files containing
linked libraries and executables for some computing platforms.  You
should use these files whenever possible, to avoid the hassle of
compiling the software yourself.  The \opendap software is a substantial
chunk of code, and it takes a long time to compile and link.


\section{Installing the Software}
\label{sec,installing}

To install the \opendap software, choose a directory to be the \opendap root
directory.  This directory must be identified with the
\lit{\ind{DODS\_ROOT}} environment variable for the \opendap software to
run.  

First, set the working directory to the \lit{DODS\_ROOT} directory.For
example,

\begin{vcode}{ib}
export DODS_ROOT=/usr/local/DODS
cd $DODS_ROOT
\end{vcode}
%$

Next, expand the archives with \lit{gzip} and unpack the expanded
files with \lit{tar}:

\begin{vcode}{ib}
gzip -d DODS-88772.tar.gz
tar -xvf DODS-88772.tar
\end{vcode}

If you got the files via anonymous ftp, you would have to repeat the
process for each down-loaded archive file, specifying the name of each
component file.  For example:

\begin{vcode}{ib}
gzip -d DODS-core-2.19.tar.gz
tar -xvf DODS-core-2.19.tar
\end{vcode}

\subsubsection{Building the \opendap Core}

Unpacking the core software archive will create a configure script in
the \lit{DODS\_ROOT} directory.  The core software may then be built
with:

\begin{vcode}{ib}
cd $DODS_ROOT
./configure
make
\end{vcode}
%$

\noindent
If you downloaded binary files, you can skip the ``building'' steps.

\subsubsection{Building an \opendap Client or Server}

Unpacking the client library archives in the \lit{DODS\_ROOT}
directory will produce directories such as \lit{jg-dods} for the JGOFS
software and \lit{nc-dods} for the netCDF software.  These will appear
in the \lit{src} subdirectory under \lit{DODS\_ROOT}.  Each of these
directories will also be equipped with a configure script to create a
makefile, so the build procedure for them is the same as for the core
software, for example:

\begin{vcode}{ib}
cd $DODS_ROOT/src/nc-dods
./configure
make
\end{vcode}
%$

\subsubsection{Getting the Software Used by DODS}

Several pieces of software are required to run \opendap.  These are all
free software, and are descibed in \appref{req-software}.

\subsection{Installing the \opendap Libraries}
\label{install,lib-install}

In order to link and run client programs, the only \opendap software that
must be installed are the \opendap libraries. The following libraries,
described in \sectionref{opd-client,link} must be installed in the
\lit{/usr/lib} directory, or somewhere else where the linker will find
them. Simply copy them into that directory from the \opendap distribution.

\begin{itemize}

\item \lit{libdap++.a}

\item The \opendap version of the API your software uses.

\end{itemize}

In addition to the \opendap libraries, the following software is also
required to link an \opendap client.  You can obtain them from the
\OPDhome\ or \OPDftp .  (Look under your machine
type, then under \lit{DODS/packages/lib}.)  The first two libraries
should be part of the Tcl/Tk distribution.  The other three libraries
are GNU software.

\begin{itemize}

\item \lit{libtcl.a}

\item \lit{libexpect.a}

\item \lit{libstdc++.a}
  
\item \lit{librx.a} (Regular expression software.  Part of the
  \lit{regex} package.)

\item \lit{libz.a} (Compression software.  Part of the \lit{gzip}
  distribution.)

\end{itemize}

To run an \opendap client, you should have the \lit{wish} and \lit{gzip}
programs in the \lit{PATH}, and the following Tcl programs must be
either in the \lit{DODS\_ROOT/etc} directory:

\begin{itemize}

\item \lit{dod\_gui.tcl}

\item \lit{progress.tcl}

\item \lit{error.tcl}

\end{itemize}

An \opendap client will still execute without these programs, but important
functionality, including the GUI manager and the data compression will
be absent.  Refer to the \lit{INSTALL} file included in each
distribution for specific information about setting up the client.


\section{The \opendap Client Initialization File (\lit{.dodsrc})}

The following section refers only to \opendap clients from release 3.2 and
after. 

\indc{initialization file}\indc{client initialization}
\indc{client!initialization}\indc{initialization!client} \indc{dodsrc file@\lit{.dodsrc} file}\indc{startup file} 
When an \opendap client starts up, it checks to
see whether the user has an initialization file available to control
the setting of a number of parameters having to do with caching, proxy
servers, and other http issues.  If this file is not found, one is
created in the user's home directory, using default
values.\footnote{The default values enable a maximum cache size of 20
  megabytes, a cache expiration time of 24 hours, and no proxy servers.}

The client initialization file is usually called \lit{.dodsrc}, and is
usually located in the user's home directory.  You can change this by
creating an environment variable called \lit{DODS\_CACHE\_INIT} and
setting it to the full pathname of the configuration file.  As of \opendap
version 3.4, you should use \lit{DODS\_CONF} instead.
\lit{DODS\_CACHE\_INIT} is deprecated, and will disappear in future
releases.

Here is a sample configuration file:
%  (the line numbers are not part of the
%file; they've been added to make the description clearer):

\begin{vcode}{inb}
# Sample configuration file
USE_CACHE=1
MAX_CACHE_SIZE=20
MAX_CACHED_OBJ=5
IGNORE_EXPIRES=0
CACHE_ROOT=/home/user/.dods_cache/
DEFAULT_EXPIRES=86400
PROXY_SERVER=http,http://dcz.dods.org/
PROXY_FOR=http://dax.dods.org/.*,http://dods.org/
NO_PROXY_FOR=http://dcz.dods.org
DEFLATE=1
ALWAYS_VALIDATE=0
\end{vcode}    

\subsubsection{Comments}

Starting a line with a \lit{\#} makes that line a comment. 

\subsubsection{Caching}
\indc{caching!client}\indc{client!caching}\indc{data requests!caching}

The parameters on lines 2 through 7 control caching.  The \opendap client
can store data you've requested on your local computer.  If you repeat
a request, the data can be retrieved from this local cache, saving the
expense of a network connection.  Most web browsers operate the same
way.  You can control the caching behavior with the following
configuration file parameters.

\begin{description}
\item[\lit{CACHE\_ROOT}] This parameter contains the pathname to the
  cache's top directory.  If two or more users want to share a cache,
  then they must both have read and write permissions to the cache
  root.

\item[\lit{MAX\_CACHE\_SIZE}] The value of \lit{MAX\_CACHE\_SIZE} sets
  the maximum size of the cache in megabytes.  Once the cache reaches
  this size, caching more objects will cause cache garbage collection.
  \opendap will first purge the cache of any stale entries and then remove
  remaining entries starting with those that have the lowest hit
  count. Garbage collection stops when 90\% of the cache has been
  purged.
  
\item[\lit{MAX\_CACHED\_OBJ}] This parameter sets the maximum size of
  any individual object in the cache in megabytes. Objects received
  from a server larger than this value will not be cached even if
  there's room for them without purging other objects.
  
\end{description}

\indc{expiration date}\indc{date!expiration}
Many web documents, and \opendap data, are delivered with an expiration
date in their header information.  Generally, this is done for
time-sensitive information that may not be valid after the expiration
date.  You can control the behavior of the \opendap client with respect to
expiration dates with two configuration file parameters.

\begin{description}
\item[\lit{IGNORE\_EXPIRES}] If the value of this parameter is 1
  (one), then expiration dates will be ignored, and the caching
  behavior will be ruled by the \lit{DEFAULT\_EXPIRES} parameter.

\item[\lit{DEFAULT\_EXPIRES}] Any data received without an expiration
  time will expire in the number of seconds given by this parameter.
  If \lit{IGNORE\_EXPIRES} is zero, this will apply to all data,
  whether or not it comes with an expiration date.  The value is given
  in seconds.  The configuration in the example is set for 24 hours.

\end{description}

\subsubsection{Proxy Servers}

\indc{proxy server}\indc{server!proxy}\indc{configuration!proxy server}
An \opendap client can negotiate proxy servers, with help from directions
derived from its configuration file.  There are three parameters that
control proxy behavior.  There can be more than one of each of these
declarations.

\begin{description}
\item[\lit{PROXY\_SERVER}] This identifies a proxy server to use for
  all \opendap requests, except for requests specifically modified by the
  other two proxy behavior directives.  The format is:

  \begin{example}
PROXY_SERVER=\var{protocol},\var{proxy_URL}
  \end{example}
  
  Where \var{protocol} is the name of an internet protocol, and
  \var{proxy\_URL} must be a full URL to the host running the proxy
  server.  HTTP is the only internet protocol supported by DODS, so
  \var{protocol} will always read \lit{http}.  There can be more than
  one proxy declaration, in which case, the \opendap client will use the
  first proxy server on the list that responds.
  
\item[\lit{PROXY\_FOR}] The \lit{PROXY\_FOR} parameter provides a way
  to specify that URLs which match a regular expression should be
  accessed using a particular proxy server. The syntax for PROXY_FOR
  is:

  \begin{example}
PROXY_FOR=\var{regular expression},\var{proxy_URL}[,\var{flags}]
  \end{example}

  Where \var{regular expression} is an expression which matches the
  URL or group of URLs. For example `http://dax.dods.org/.*\.hdf'
  would match a URL ending in `.hdf' at dax.dods.org. The regular
  expression uses the POSIX basic syntax.\var{proxy\_URL} is the
  same as above.

  The \var{flags} parameter is an optional integer that configures the
  regular expression matcher. A value of zero sets the default. The
  four flag values and their meanings are:

  \begin{description}
  \item[\lit{REG\_EXTENDED}] If set, use the POSIX extended syntax
    regular expressions.  To set this, add 1 to the value of \var{flags}.
  \item[\lit{REG\_NEWLINE}] If set, then \lit{.} and \lit{[\^{ }...]}
    don't match newline.  Also, the regular expression matcher will
    try a match beginning after every newline. Set this by adding 2 to
    \var{flags}.
  \item[\lit{REG\_ICASE}] If set, then we consider upper- and
    lowercase versions of letters to be equivalent when matching. Set
    by adding 4 to \var{flags}.
  \item[\lit{REG\_NOSUB}] If set, then when \lit{PREG} is passed to regexec, that
     routine will report only success or failure, and nothing about the
     registers. Add 8 to \var{flags} to set this.
  \end{description}

  You can find a brief tutorial to regular expressions in the \opendap
  bookshelf\texorhtml{. See the \opendap documentation page at 
  \OPDhome}{at \OPDregex}.
  
\item[\lit{NO\_PROXY}] Use this parameter to say that access to a
  certain host should never go through a proxy without using the more
  complicated regular expression syntax. The syntax of \lit{NO\_PROXY}
  is:

  \begin{example}
NO_PROXY=\var{protocol},\var{hostname}
  \end{example}
  
  Where \var{protocol} is as for \lit{PROXY\_SERVER}, \var{hostname}
  is the name of the host, not a url.  

\subsubsection{Compression}

Many \opendap servers support compression of the returned data.  This can
save network bandwidth, and transmission time.  You can tell your
client to ask the server for compressed data by including a line in
your \lit{.dodsrc} file like this:

\begin{example}
DEFLATE=1
\end{example}

Using a value of zero will make the client request only uncompressed
data.  If the directive is omitted, the default value is zero.

\subsubsection{Validation}

Caching data locally can be risky if the data in question change
often.  The \lit{ALWAYS\_VALIDATE} option controls whether the \opendap
client checks the validity of the cached data.  The validity check in
this case is simply a comparison of the date the data was cached with
the ``Last-modified'' date of the remote data.  If the configuration
value is set to 1, the client will always validate the cached data.
If set to 0, the data will not be validated (but may expire, according
to the cache policy set by the \lit{DEFAULT\_EXPIRES} configuration
directive).

If the directive is omitted, the default value is zero.  That is, the
default behavior is not to validate the data.

\end{description}

%
%  $Log: appa.tex,v $
%  Revision 1.12  2004/08/24 22:51:32  jimg
%  Fixed broken label/refs.
%
%  Revision 1.11  2003/12/28 20:00:20  tom
%  accepted conflicts
%
%  Revision 1.10  2003/09/04 19:42:06  tom
%  DODS->OPeNDAP
%
%

%%% Local Variables: 
%%% mode: latex
%%% TeX-master: t
%%% End: 
