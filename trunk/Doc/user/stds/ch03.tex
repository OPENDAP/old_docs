% Chapter three to the DODS Data Standards Guide
%
%
% You have a server running, how do you add data attributes?

% $Id$
%
% $Log: ch03.tex,v $
% Revision 1.2  1999/10/18 21:37:43  tom
% updated for newest consensus
%
% Revision 1.1  1999/07/01 16:05:27  tom
% added example chapter
%
%
%

\chapter{Adding Data Attributes}
\label{adding-attr}

One of the original design goals of DODS was to permit data providers
to use DODS with as little work as possible.  A corollary to this
requirement is that if ancillary information about a dataset is
required, it must be very easy to provide.  The DODS architecture
allows a data provider to add attribute information to a dataset
without modifying the data files at all.  All that is required is a
separate file, stored in the same directory as the data.

This chapter shows how you can easily add attribute information to an
existing dataset to comply with the DODS data standard.  We start by
creating a small dataset and serving it with \DODSffs .  A similar
example is presented in \DODSffbook , highlighting different aspects.

All the files used in this chapter are available on the \DODSexamples .

\section{A Worked Example}

Consider the following table of numbers:

\begin{vcode}{ib}
-47.303545 -176.161101  1.17125
-25.928001   -0.777265  2.07288
-28.286662   35.591879  2.36377
 12.588231  149.408117 -100.000
-63.223548   55.319598  0.04503
 54.118314 -136.940570  1.04085
-38.818812   91.411330  1.39978
-34.577065   30.172129  2.09096
 27.331551 -155.233735  2.30917
 11.624981 -113.660611  2.75036
\end{vcode}

This table represents a series of temperature measurements at ten
different geographical locations.  The latitudes and longitudes are
given in decimal degrees, and the temperatures are given in units of
ten degrees.  That is, the temperature as recorded, times ten, equals
the temperature in Celsius.  There is one temperature measurement
missing, and its place is marked with a value of \lit{-100.0}.

(This dataset, and the ancillary files discussed in this chapter, are
in the DODS examples directory, and is available at the \DODSexamples ,
under \lit{dasex.*}.)

The dataset shown here can be served as is, with the \DODSffs , using
the following format file (\lit{dasex.fmt}):\footnote{See the
  \DODSffbook\ for more information about that software.}

\begin{vcode}{ib}
ASCII_data "lat/lon"
latitude 1 10 double 6
longitude 12 22 double 6
t 24 31 double 4
\end{vcode}

Put both of these files in your \lit{htdocs} directory (refer to the
documentation for your web server to locate this directory).  After
installing the \DODSffs , you can issue data requests on this
dataset.  Use a web browser like netscape, and enter the following
URL (Use your own server's name instead of \lit{machine.edu}.):

\begin{vcode}{ib}
http://machine.edu/cgi-bin/nph-ff/dasex.dat.dds
\end{vcode}

You should see something like this:

\begin{vcode}{ib}
Dataset {
    Sequence {
        Float64 latitude;
        Float64 longitude;
        Float64 t;
    } lat/lon;
} dasex;
\end{vcode}

The is the DDS (Data Description Structure) of the dasex dataset.  (See
\DODSuser\ for more information about the DDS and what it means, as
well as a description of what a Sequence is.)  Instead of \lit{.dds} at
the end of the URL, you can use \lit{.asc} to see the data itself.
Try that.  

To see data attributes, try the same URL, ending with \lit{.das}:

\begin{vcode}{ib}
http://machine.edu/cgi-bin/nph-ff/dasex.dat.das
\end{vcode}

You should see something like this:

\begin{vcode}{ib}
Attributes {
 FF_GLOBAL {
   String Server "DODS FFND release 4.2.3";
   String Native_file " ...";
    }
}
\end{vcode}

The DAS is a list of the dataset's attributes.  This dataset has two,
and they are global ones.  That is, they describe the entire dataset,
and don't correspond to any of the data variables listed in the DDS.
They are also in a container called \lit{FF_GLOBAL}, which is another
hint that they are global attributes.  The \lit{Server} string
describes the actual DODS server that sent the data, and the
\lit{Native_file} string is a long one, containing information about
the original file, and what it contained.

This is all very well, but what if someone wanted to know what units
the \lit{t} variable was in, or where the data came from, or what that
value of -100 is supposed to imply, or even whether the variable
represents temperature or time.  The data themselves are silent on
those issues, so either we have to provide additional documentation,
or we have to introduce attributes for the data variables, to make the
dataset self-documenting.  We do that with an ancillary file of data
attributes.  Call this file \lit{dasex.das}\footnote{It is called
  dasex1.das in the \DODSexamples .  You will have to rename this file
  to make it work.}:

\begin{vcode}{ib}
Attributes {
  t {
    String short_name "Temperature";
    String units "DegreesC";
    Float64 missing_data -100.0;
    Float64 scale_factor 10.0;
  }
  latitude {
    String short_name "Latitude";
    String units "degree_north";
  }
  longitude {
    String short_name "Longitude";
    String units "degree_east";
  }
}
\end{vcode}

Now the DAS that is returned from the dataset looks like this:

\begin{vcode}{ib}
Attributes {
    FF_GLOBAL {
        String Server "DODS FFND release 4.2.3";
        String Native_file "...";
    }
    temp {
        String short_name "Temperature";
        String units "DegreesC";
        Float64 missing_data -100.0;
        Float64 scale_factor 10.0;
    }
    latitude {
        String short_name "Latitude";
        String units "degree_north";
    }
    longitude {
        String short_name "Longitude";
        String units "degree_east";
    }
}
\end{vcode}

Now you can see that \lit{t} stands for ``Temperature'' and that the
\lit{-100} is missing data, and that all the measurements have to
multiplied by 10 before they are in degrees Celsius.

Now that the data have these additional attributes, the dataset is
compliant with level 1 of the DODS data standard.  This can also be
noted in the DAS, with the addition of another global attribute
container called \lit{DODS}.  A file called \lit{dasex2.das} contains
the following:

\begin{vcode}{ib}
DODS {
  String Conventions "DODS";
  String Acknowledge "Example dataset from the DODS documentation.";
  String history "Created for DODS standards book.";
}
\end{vcode}

Add this information to the bottom of the \lit{dasex.das} file
(\emph{before} the last closing bracket), and now the dataset is
compliant with level 2 of the DODS data standard.



%%% Local Variables: 
%%% mode: latex
%%% TeX-master: t
%%% End: 
