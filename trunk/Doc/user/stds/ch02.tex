% Chapter two to the DODS Data Standards Guide
%
% $Id$
%
% $Log: ch02.tex,v $
% Revision 1.3  1999/10/18 21:37:42  tom
% updated for newest consensus
%
% Revision 1.2  1999/07/01 16:05:27  tom
% added example chapter
%
% Revision 1.1  1999/06/07 05:09:32  tom
% added to CVS
%
%
%

\newcommand{\iso}[3]{This corresponds to the ISO Z39.50 GeoProfile
  element \emph{#1}, value \emph{#2}, SGML tag \lit{#3}.}

\chapter{The Emerging DODS Data Standards}
\label{dods-standard}

Using DODS, a scientist or data center can make their data readily
available to anyone who knows about it.  But there is a difference
between ``readily available'' and ``readily usable.''  Just because
DODS can easily retrieve data doesn't mean that it is easy to use.

The DODS project has found that the single biggest obstacle to ease of
use of a dataset's data is incompatibility of metadata, or data
attributes.\footnote{Please see the \DODSuser\ for a discussion of
  DODS treatment of attributes, or metadata.}  This incompatibility
can range from the seemingly simple, such as different data names, to
the far less tractable, such as incompatible time representations.

To address this problem, the DODS project has created the data
attribute standard described in this chapter.  This standard is
\emph{entirely optional}.  A server can still serve data from a DODS
dataset that does not conform to this standard, and a DODS client can
still read that data.  However, a dataset that conforms to this
standard will be more easily read and processed by another user.

\note{The DODS project has an original emphasis on oceanography.
  Therefore, the DODS standard described here has been defined, where
  possible, to be compatible with existing geo-spatial metadata
  conventions, including the COARDS and the ISO Z39.50 GeoProfile
  conventions.  It should be noted that the DODS standard is a small
  subset of these metadata definitions, and may well be applicable to
  data from other scientific (or other) disciplines.  In addition, the
  architecture of the DODS software does not preclude the adoption of
  alternate data standards that may be more appropriate to those other
  fields.}

In a similar effort, attempting to assert some order over incompatible
data attributes among netCDF files, the COARDS project
(\xlink{``Cooperative Ocean/Atmosphere Research Data Service''}
{http://ferret.wrc.noaa.gov/noaa_coop/coop_cdf_profile.html}) has
specified some standard attributes that shall be defined in a dataset.
Note that this has little or nothing to do with netCDF itself.  You
can have a netCDF dataset that doesn't comply with the COARDS
standard.  But it will be more useful to others if it does.

Note that the DODS architecture allows data attributes to be added to
any dataset \emph{without} modifying the data files themselves.
This makes the task of conforming to the data standard fairly simple,
in most cases consisting of adding a few files to a directory.

Finally, this is not an exhaustive list of attributes possible.
Beyond syntax-checking, the DODS software does no parsing of attribute
structures for a dataset.  You are free to add whatever attributes you
choose to any dataset served by a DODS server.  If you want to add
enough data attributes to make your metadata 100\% compliant with the
ISO Z39.50 GeoProfile, there is no reason not to do so.\footnote{We
  would suggest that if you do add GeoProfile attributes not mentioned
  in this document to your dataset, you should use the SGML attribute
  tags as attribute names.  In the event that future DODS servers
  support a part of the GeoProfile protocol, this will allow your
  datasets to be merged seamlessly with that standard.}

\section{Essential and Optional Attributes}

One of the motivating principles of the DODS project is to keep the
burden on the data provider as light as possible.  To further decrease
the burden of making data available, the DODS data standard attributes
are separated into two groups:

\begin{itemize}
\item Those that are essential to minimal data attribute
  interoperability, and
\item Those that are very useful, but not essential.
\end{itemize}

The second group can be further split up among the more and less
useful attributes.

The DODS project recognizes these as different ``classes'' of
compatibility.  For example, a class 0 DODS dataset is one that is
served by DODS, but does not comply with the DODS data standards,
except in the most minimal way.  A class 1 dataset conforms, but only
contains the essential attributes.  A class 2 datasets conforms and
contains the essential and the class 2 attributes, and so on.  The
attributes described in the following sections are identified by their
classes in this way.


\subsection{Global and Variable Attributes}

DODS attributes consist of sets of name-value
pairs.\footnote{Actually, there is type information in there, too, so
  it's more accurate to say name-type-value triples, but ``triples''
  is more awkward than ``pair.''}  To say that a data variable has
attributes is the same as saying that there is a set of attributes (in
the dataset's DAS) with the same name as this variable.  The set can
contain one or more name-value pairs, such as ``units'' with
``degreesC'' and ``scale\_factor'' with ``2.35''.

Consider the dataset with a DDS like this:

\begin{vcode}{ib}
Dataset {
  Int16 temp[time = 16][lat = 17][lon = 21];
  Float32 lat[lat = 17];
  Float32 lon[lon = 21];
  Float32 time[time = 16];
} test1;
\end{vcode}

The DAS for this DODS-compliant dataset might look like this:

\begin{vcode}{xib}
Attributes {
  temp {
      String units "degreesC";
      String long_name "Surface Temperature";
      String missing_value "-32767";
      String scale_factor "0.005";
      String short_name "Temp";
  }
  lat {
      String units "degree North";
      String short_name "Latitude";
  }
  lon {
      String units "degree East";
      String short_name "Longitude";
  }
  time {
      String units "hours from base_time";
      String short_name "Time";
  }
  DODS {
      String conventions "DODS", "COARDS";
  }
}
\end{vcode}

One of the attribute containers in this DAS does not correspond to any
data variables.  The attributes in this container are said to be
``global'' attributes, and modify the entire dataset.  There is
nothing special about the names; an attribute container can be called
anything at all.  However, the container called ``DODS'', if it
exists, should contain the global attributes described in the DODS
data standard, such as \lit{conventions}, \lit{Acknowledge}, and
\lit{history}.


\section{Utterly Essential Attributes (Class 0)}

The following attributes must appear in any DODS dataset for it to be
considered class 0 compliant.  These attributes must be defined for
\emph{each} data variable in the dataset.

\begin{description}

\item[\lit{long\_name}] A long descriptive name (title). This could be
  used for labelling plots, for example. If a variable has no
  \lit{long\_name} attribute assigned, the variable name should be
  used as a default.  This corresponds to the ``Detailed Variable''
  field of the GCMD variable naming hierarchy. 
  (\iso{Attribute\_Label}{3507}{attrlabl}) 
  
\item[\lit{units}] A character array that specifies the units used for
  the variable's data. The units attribute should be formatted as per
  the recommendations in the Unidata
  \xlink{\lit{udunits}}{http://www.unidata.ucar.edu/packages/udunits/}
  package. (\iso{Attribute\_Units\_of\_Measurement}{3522}{attrunit}).
  
\item[\lit{scale\_factor}] If present for a variable, the data are to
  be multiplied by this factor after the data are read by the
  application that accesses the data. One or both of \lit{scale\_factor} or
  \lit{add\_offset} must be present if the data are not stored in the
  specified units, unless there is also a \lit{transform} specified.
  
\item[\lit{add\_offset}] If present for a variable, this number is to
  be added to the data after it is read by the application that
  accesses the data.  One or both of \lit{scale\_factor} or
  \lit{add\_offset} must be present if the data are not stored in the
  specified units, unless there is also a \lit{transform} specified.
  If both \lit{scale\_factor} and \lit{add\_offset} attributes are
  present, the data are first scaled before the offset is added.
  
\item[\lit{transform}] This is a string containing a transformation
  function used to convert the raw data into the units specifed in the
  \lit{units} attribute.\footnote{The format of the function
    specification (i.e. Java, Perl, Tcl, Lisp, whatever) is not
    currently specified.}

\end{description}

There are some special cases, outlined in the next two sections.
However, if your dataset contains no null values, and no data stored
as DODS Arrays, the above list is complete.  

\note{A DODS Array is a different data type than a Grid, and contains
  less information about its independent variables.  For more
  information, see below, or refer to the \DODSuser .}


\subsection{Missing Data}

If a dataset contains missing data flagged with special values, those
values must be specified in the attribute list of the variable.  That
is, if you have a data sequence called \lit{ralph}, and it contains missing
values flagged with -999, the DAS for the dataset should have an
attribute container like this:

\begin{vcode}{ib}
Attributes {
  ...
  ralph {
     Float32 missing_value -999.0;
     ...
  }
  ...
}
\end{vcode}

The DODS metadata standard provides three categories of missing data.
These attribute values are essential only in the sense that they must
be in the data variable's attribute container \emph{if you use them}.
If your data doesn't use these values---that is, if there are no
missing values flagged with special numeric values--these attributes
need not be specified.

\begin{description}
\item[\lit{missing\_value}]  This is a conventional name for a
  missing value that will not be treated in any special way by the
  client application.  This attribute is part of the COARDS standard.
  
\item[\lit{null\_value}] A null value differs from a missing value in
  that it describes data that isn't there, but shouldn't have been,
  either.  That is, where a missing value might be used to fill in for
  a sensor malfunction, a null value is used to indicate that no data
  was taken.  A dataset that contained random data interpolated onto a
  grid might use a null value on those grid points too distant from
  data values to make an accurate estimate.\footnote{The COARDS
    standard does not make this distinction, and would use
    \lit{missing\_value} for these points.}

\item[\lit{default\_value}]  A default value is yet another sort of
  missing data.  In this case, data would never have been at those
  points.  Land points in gridded sea-surface temperature data would
  be default values, as would the end of profile data vectors filled
  to uniform length.  This differs some from the semantics of the
  COARDS \lit{\_FillValue}, but perhaps not so far as to prevent the
  DODS project from adopting the COARDS name as a synonym.

%\item[\lit{\_FillValue}] If a scalar attribute with this name is
%  defined for a variable, it will be subsequently used as the fill
%  value for that variable. The purpose of this attribute is to save
%  the applications programmer the work of prefilling the data and also
%  to eliminate the duplicate writes that result from netCDF filling in
%  missing data with its default fill value, only to be immediately
%  overwritten by the programmer's preferred value. This value is
%  considered to be a special value that indicates missing data, and is
%  returned when reading values that were not written. The missing
%  value should be outside the range specified by valid_range (if used)
%  for a variable.  It is not necessary to define your own _FillValue
%  attribute for a variable if the default fill value for the type of
%  the variable is adequate.
  
\end{description}


\subsection{Array Data}

A dataset containing only an array may be missing some important
information about the dataset's independent variables.  To make the
dataset conpliant with the Class 0 of the DODS standard, this
information must be included in the attribute list.

The information missing from an Array variable is the location of that
array's corners---the \lit{min} and \lit{max} for each dimension---and
other information about the array dimensions.  The requirements,
therefore, are that an Array's dimensions be named, and that attribute
containers with those same names be contained in the dataset DAS.
That is, for a dataset with the following array data:

\begin{vcode}{ib}
Dataset {
    Array {
      Byte dsp_band_1[lat = 1024][lon = 1024];
    } dsp_band_1;
}
\end{vcode}

The DAS should look something like the following:

\begin{vcode}{ib}
Attributes {
  dsp_band_1 {
    String long_name "AVHRR sea surface temperature data";
    String units "DegreesC";
    Float32 scale_factor 0.15625;
    Float32 add_offset 5.0;
    Float32 missing_value -999.0;
  }
  lat {
    String long_name "Latitude";
    String units "Degrees North";
    Float32 min 0.0;
    Float32 max 70.0;
  }
  lon {
    String long_name "Longitude";
    String units "Degrees East";
    Float32 min -100.0;
    Float32 max 0.0;
  }
}
\end{vcode}

If using the name of the dimension would cause a name collision with
some other variable in the dataset, you can use the name of the
dimension prefixed by the name of the array.  That is, in the above
example, the attribute containers for the array dimensions would be
\lit{dsp_band_1.lat} and \lit{dsp_band_1.lon}.
  

\section{Less Essential, But Still Useful Attributes (Class 1)}

The optional attributes are divided into two different classes.  This
again allows a data provider some latitude in the class of compliance
with the standard.

\begin{description}
\item[\lit{short\_name}] This is the name of the variable, taken from
  a list of DODS names (see \appref{std-names}).  Including this
  attribute is a way to ensure that two datasets can be usefully
  compared with one another on a variable-by-variable basis.

\item[\lit{Convention}] This global attribute is recommended to
  identify the dataset as conforming to the DODS data standard,
  identified here.  The attribute should be a string with the value
  ``DODS''.   Note that this is a global attribute, and should appear
  in the ``DODS'' attribute container, like this:

\begin{vcode}{ib}
DODS {
  String Convention "DODS"; }
\end{vcode}

Note that the presence of the ``DODS'' attribute container is itself
a clue to whether the dataset is DODS-compliant or not.

\iso{Metadata\_Standard\_Name}{3705}{metstdn} To be compliant with the
GeoProfile, one element in the Convention vector should read
``\lit{FGDC Content Standards for Digital Geospatial Metadata}''.

\end{description}

\section{Very Useful Attributes, But Not Essential (Class 2)}

This is a set of attributes that has been found to be quite useful in
the use of DODS datasets.  A dataset can be considered to be class 2
compliant if it contains more than one of these attributes.

\begin{description}
  
\item[\lit{Acknowledge}] This string should contain an acknowledgement
  paragraph that can appear in papers that use the data from this
  dataset.  For example: ``The principal investigators in the
  production of these data are J.D. Elms, S.D. Woodruff, and
  S. Worley (http://www.cdc.noaa.gov/coads/participants.html).'' and
  so on.  This is a global attribute, and should appear in the
  ``DODS'' attribute container. 
  (\iso{Citation\_Information}{3101}{citeinfo}) 
  
\item[\lit{history}] The \lit{history} attribute is recommended to
  record the evolution of the data contained within a DODS data file.
  Applications which process this data can append their information to
  this attribute.  This is a global attribute, and should appear in
  the ``DODS'' attribute container.

\item[\lit{Data_Use_Policy}] If there are any restrictions on the use
  of the data, they should be noted in the string with this name.
  This is a global attribute, and should appear in the ``DODS''
  attribute container. 
  (\iso{Distribution\_Information}{3600}{distinfo}) 
  
\item[\lit{Theme}] A string containing one or more of the GCMD parameter
  valid names. See \xlink{the ``GCMD Parameters'' list maintained at GCMD.}
  {http://gcmd.gsfc.nasa.gov/cgi-bin/md/valids_display.pl} 
  (\iso{Theme}{3122}{theme})

\end{description}

%\section{More Great Ideas (Class >2)}
%
% A DODS server is not currently compatible with the ISO GeoProfile
% vision of a data server.  However, this is a possible future expansion
% of DODS. 



%%% Local Variables: 
%%% mode: latex
%%% TeX-master: t
%%% TeX-master: t
%%% End: 
