% This is the dods quick start guide for data users
%

\documentclass{dods-book}

\rcsInfo $Id$
\newcommand{\DOCversion}{Version \rcsInfoRevision}

%%% Tex customizations and command definitions for DODS user
%%% guides, 2 October 1997 - tomfool
%%%
%%% $Id$

%%% $Log: layout.tex,v $
%%% Revision 1.5  1999/05/25 20:49:34  tom
%%% changed version numbers to 3.0
%%%
%%% Revision 1.4  1999/02/04 17:39:02  tom
%%% modified for use with dods-book.cls
%%%
%%% Revision 1.3  1998/03/13 20:43:26  tom
%%% writing of API manual.
%%%
%%% Revision 1.2  1998/02/12 15:47:25  tom
%%% updated for GUI doc
%%%
%%% Revision 1.1  1997/10/02 17:18:14  tom
%%% moved from user guide to boiler, made slightly more general,
%%% for use with other guides.
%%%

%%%%%%%%%%%%%%%%%%%%%%%%%%%%%%%%%%%%%%%%%%%%%%%%%%%%%%%%%%%%%%%%%%%%%
%%% This file is for TeX macros that are equally appropriate for the
%%% hard-copy (LaTeX) and html (hyperlatex) versions of the dods
%%% books.  If it's not appropriate for both, then it probably belongs
%%% in dods-book.cls or dods-book.hlx.

%%%%%%%%%%%%%%%%%%%%%%%%%%%%%%%%%%%%%%%%%%%%%%%%%%%%%%%%%%%%%%%%%%%%%
\NotSpecial{\do\_}% This removes the special meaning of `_', so for
                  % subscripts, there must be an `tex' environment
                  % around any diagram using subscripts, and an entire
                  % alternate html figure using <sub> tags.


%%%%%%%%%%%%%%%%%%%%%%%%%%%%%%%%%%%%%%%%%%%%%%%%%%%%%%%%%%%%%%%%%%%%%
%%% Different kinds of cross references.
%\newcommand{\chapterref}[1]{Chapter~\refl{#1} on page~\pagerefl{#1}}
\newcommand{\chapterref}[1]{\link{Chapter~\ref{#1}}{#1}}
\newcommand{\appref}[1]{\link{Appendix~\ref{#1}}%
  [~on page~\pageref{#1}]{#1}}
\newcommand{\sectionref}[1]{\link{Section~\ref{#1}}%
  [~on page~\pageref{#1}]{#1}} 
\newcommand{\pagexref}[1]{\link*{here}[page~\pageref{#1}]{#1}}
\newcommand{\tableref}[1]{\link{table~\ref{#1}}{#1}}
\newcommand{\Tableref}[1]{\link{Table~\ref{#1}}{#1}}
\newcommand{\figureref}[1]{\link{figure~\ref{#1}}{#1}}
\newcommand{\Figureref}[1]{\link{Figure~\ref{#1}}{#1}}

%%% A Prefatory list of the font conventions:
\newcommand{\listconventions} {
\section{Conventions}
\label{pref,conventions}

The \indn{typographic conventions} shown in
Table~\ref{typo-conventions} are followed in this guide and all the
other DODS documentation.

\begin{table}[htbp]
  \begin{center}
  \caption{Typographic Conventions}
  \label{typo-conventions}
  \begin{tabular}{|c|p{3in}|} \hline
    \lit{Literal text}  &  
         Typed by the computer, or in a code listing.\\ \hline
    \inp{User input}    &  
         Type this precisely as written.\\ \hline
    \var{Variables}     &   
         Used as a place holder for a user-specified or variable
         value. Choose an appropriate value and use that in place.\\
         \hline 
    \but{Button Text}\texonly{\rule{0pt}{2.5ex}}   
        &  Used to indicate text on a GUI button.\\ 
         \hline 
    \pdmenu{Menu Name}    &  This is the name of a GUI menu.\\ \hline 
  \end{tabular}
  \end{center}
\end{table}

When referring to a button in a menu, we will often use the notation:
\but{Menu,Button}. For example, \but{Options,Colors,Foreground} would
indicate the \but{Foreground} button in the \pdmenu{Colors} menu,
selected under the \pdmenu{Options} menu.
 }


%%% Local Variables: 
%%% mode: latex
%%% TeX-master: t
%%% End: 

%
% These are html links which are used often enough in writing about DODS to
% merit an input file.
% jhrg. 4/17/94
%
% File rationalized and updated while writing the DODS User
% Guide. Also includes other useful abbreviations.
% tomfool 3/15/96
%
% Moved to dods-def.tex so I can remove links to documents that no
% longer reflect reality.
% tomfool 2/13/98
%
% $Id$
%
% Make sure to include layout.tex *before* using this file.

%% NOTE NOTE NOTE NOTE NOTE NOTE NOTE NOTE NOTE NOTE NOTE NOTE NOTE NOTE 
%%
%% If this file causes problems when running latex, you may have to edit your
%% texmf.cnf file. Here's a meesage from Tom:
%% > Are these references that use relative addresses (like
%% > ../boiler/blah.tex)?  If they are, you should look for the texmf.cnf
%% > file.  (It's often at /usr/share/texmf/web2c/texmf.cnf, and look for
%% > the openout_any parameter.  Check there anyway; there were some recent
%% > (i.e. in the early '90's) security fixes to TeX.
%%
%%%%%%%%%%%%%%%%%%%%%%%%%%%%%%%%%%%%%%%%%%%%%%%%%%%%%%%%%%%%%%%%%%%%%%%%%%%

%%% These are some DODS-specific convenience commands.
\newcommand{\DODSroot}{\lit{\$DODS\_ROOT}}     
% $

\newcommand{\opendap}{OPeNDAP\xspace}

%%% The OPD books and reference material
\newcommand{\OPDDoc}{http://opendap.org/support/docs.html}
\newcommand{\DODSDoc}{http://opendap.org/support/docs.html}

% \newcommand{\OPDDoc}{http://www.unidata.ucar.edu/packages/dods}
% \newcommand{\DODSDoc}{http://www.unidata.ucar.edu/packages/dods}

\newcommand{\OPDhomeUrl}%
  {http://opendap.org}
\newcommand{\OPDexampleUrl}%
  {BROKEN--FIX ME!}%\OPDDoc/examples}
% \newcommand{\OPDftpUrl}%
%  {ftp://dods.gso.uri.edu/pub/dods}
\newcommand{\OPDftpUrl}%
  {ftp://ftp.unidata.ucar.edu/pub/opendap/}
\newcommand{\OPDuserUrl}%
  {\OPDhomeUrl/user/guide-html/}
\newcommand{\OPDmguiUrl}%
  {\l/user/mgui-html/}
\newcommand{\OPDapiUrl}%
  {\OPDhomeUrl/api/pguide-html/}
\newcommand{\OPDapirefUrl}%
  {\OPDhomeUrl/api/pref/html/}
\newcommand{\OPDffUrl}%
  {\OPDhomeUrl/user/servers/dff-html/}
\newcommand{\OPDquickUrl}%
  {\OPDhomeUrl/user/quick-html/}
\newcommand{\OPDinstallUrl}%
  {\OPDhomeUrl/server/install-html}%
\newcommand{\OPDregexUrl}%
  {\OPDhomeUrl/user/regex-html}%
\newcommand{\OPDjavaUrl}%
  {\OPDhomeUrl/home/swJava1.1/}
\newcommand{\OPDwclientUrl}%
  {\OPDhomeUrl/api/wc-html/}
\newcommand{\OPDwserverUrl}%
  {\OPDhomeUrl/api/ws-html/}
\newcommand{\OPDaggUrl}%
  {\OPDhomeUrl/server/agg-html/}

\newcommand{\OPDhome}{\xlink{OPeNDAP Home page}{\OPDhomeUrl}}
\newcommand{\OPDjava}{\xlink{OPeNDAP Java home page}{\OPDjavaUrl}}
\newcommand{\OPDexamples}{\xlink{OPeNDAP examples page}{\OPDexampleUrl}}
\newcommand{\OPDftp}{\xlink{OPeNDAP ftp site}{\OPDftpUrl}}
%% Book titles do *not* contain the article.
\newcommand{\OPDuser}[1][]{\xlink%
  {\cit{OPeNDAP User Guide}}{\OPDuserUrl{}#1}}
\newcommand{\OPDmgui}{\xlink%
  {\cit{OPeNDAP Matlab GUI}}{\OPDmguiUrl}}
\newcommand{\OPDapi}{\xlink%
  {\cit{OPeNDAP Toolkit Programmer's Guide}}{\OPDapiUrl}}
\newcommand{\OPDapiref}{\xlink%
  {\cit{OPeNDAP Toolkit Reference}}{\OPDapirefUrl}}
\newcommand{\OPDffbook}{\xlink%
  {\cit{OPeNDAP Freeform ND Server Manual}}{\OPDffUrl}}
\newcommand{\OPDquick}{\xlink%
  {\cit{OPeNDAP Quick Start Guide}}{\OPDquickUrl}}
\newcommand{\OPDinstall}{\xlink%
  {\cit{OPeNDAP Server Installation Guide}}{\OPDinstallUrl}}
\newcommand{\OPDregex}{\xlink%
  {\cit{Introduction to Regular Expressions}}{\OPDregexUrl}}
\newcommand{\OPDagg}{\xlink%
  {\cit{OPeNDAP Aggregation Server Guide}}{\OPDaggUrl}}
\newcommand{\OPDwclient}{\xlink%
  {\cit{Writing an OPeNDAP Client}}{\OPDwclientUrl}}
\newcommand{\OPDwserver}{\xlink%
  {\cit{Writing an OPeNDAP Server}}{\OPDwclientUrl}}

\newcommand{\OPDffs}{OPeNDAP Freeform ND Server}

%%% Other DODS links.
\newcommand{\homepage}% For hyperlatex
  {\OPDDoc/}
\newcommand{\OPDsupport}{\xlink{support@unidata.ucar.edu}{mailto:support@unidata.ucar.edu}}
\newcommand{\DODSsupport}{\xlink{support@unidata.ucar.edu}{mailto:support@unidata.ucar.edu}}
\newcommand{\DODS}{\xlink{Distributed Oceanographic Data System}{\OPDhomeUrl}}
\newcommand{\OPD}{\xlink{Open Source Project for a Data Access Protocol}{\OPDhomeUrl}}
\newcommand{\OPDtechList}{\xlink{OPeNDAP Mailing Lists}{\OPDhomeUrl/mailLists/}}

%%% DODS versions
%% This has been removed.  Documents should not have an automatic
%% version number, because then it appears as if they have been
%% updated when they haven't.  Put the relevant version number to
%% whatever software is being described into each document's preface. 

\newcommand{\ifh}{WWW Interface}

% external refs for DODS documents

\newcommand{\CGI}{\xlink{Common Gateway Interface}
  {http://hoohoo.ncsa.uiuc.edu/cgi/overview.html}}

\newcommand{\MIME}{\xlink{Multipurpose Internet Mail Extensions}
  {http://www.cis.ohio-state.edu/htbin/rfc/rfc1590.html}}

\newcommand{\netcdf}{\xlink{NetCDF}
  {http://www.unidata.ucar.edu/packages/netcdf/guide.txn_toc.html}}

\newcommand{\JGOFS}{\xlink{Joint Geophysical Ocean Flux Study}
  {http://www1.whoi.edu/jgofs.html}}

\newcommand{\jgofs}{\xlink{JGOFS}
  {http://www1.whoi.edu/jgofs.html}}

\newcommand{\hdf}{\xlink{HDF}
  {http://www.ncsa.uiuc.edu/SDG/Software/HDF/HDFIntro.html}}

\newcommand{\ffnd}{FreeForm ND}

\newcommand{\Cpp}{\texorhtml
  {{\rm {\small C}\raise.5ex\hbox{\footnotesize ++}}}
  {C\htmlsym{##43}\htmlsym{##43}}}

% Commands

% Use pdflink instead. jhrg 8/4/2006
% \newcommand{\pslink}[1]{\small
% \begin{quote}
%   A \xlink{PDF version}{#1} of this document is available.
% \end{quote}
% \normalsize
% }

% Use the copy of this in dods-paper.hlx/cls or cut and paste this on
% a document-by-document basis. This version conflicts with the
% version in the class. jhrg 8/4/2006
% \newcommand{\pdflink}[1]{\small
% \begin{quote}
%   A \xlink{PDF version}{#1} of this document is available.
% \end{quote}
% \normalsize
% }

%%%%%%%%%%%%%% DODS macros
%
% These are here so that older latex files will compile. Someday remove these
% and fix the files. 04/13/04 jhrg

\newcommand{\DODShomeUrl}%
  {\OPDhomeUrl}
\newcommand{\DODSexampleUrl}%
  {\OPDDoc/examples}
% \newcommand{\OPDftpUrl}%
%  {ftp://dods.gso.uri.edu/pub/dods}
\newcommand{\DODSftpUrl}%
  {ftp://ftp.unidata.ucar.edu/pub/opendap/}
\newcommand{\DODSuserUrl}%
  {\OPDhomeUrl/user/guide-html/}
\newcommand{\DODSmguiUrl}%
  {\OPDhomeUrl/user/mgui-html/}
\newcommand{\DODSapiUrl}%
  {\OPDhomeUrl/api/pguide-html/}
\newcommand{\DODSapirefUrl}%
  {\OPDhomeUrl/api/pref/html/}
\newcommand{\DODSffUrl}%
  {\OPDhomeUrl/user/servers/dff-html/}
\newcommand{\DODSquickUrl}%
  {\OPDhomeUrl/user/quick-html/}
\newcommand{\DODSinstallUrl}%
  {\OPDhomeUrl/server/install-html}%
\newcommand{\DODSregexUrl}%
  {\OPDhomeUrl/user/regex-html}%
\newcommand{\DODSjavaUrl}%
  {\OPDhomeUrl/home/swJava1.1/}
\newcommand{\DODSwclientUrl}%
  {\OPDhomeUrl/api/wc-html/}
\newcommand{\DODSwserverUrl}%
  {\OPDhomeUrl/api/ws-html/}
\newcommand{\DODSaggUrl}%
  {\OPDhomeUrl/server/agg-html/}

\newcommand{\DODShome}{\xlink{OPeNDAP Home page}{\DODShomeUrl}}
\newcommand{\DODSjava}{\xlink{OPeNDAP Java home page}{\DODSjavaUrl}}
\newcommand{\DODSexamples}{\xlink{OPeNDAP examples page}{\DODSexampleUrl}}
\newcommand{\DODSftp}{\xlink{OPeNDAP ftp site}{\DODSftpUrl}}
\newcommand{\DODSuser}[1][]{\xlink%
  {\cit{The OPeNDAP User Guide}}{\DODSuserUrl{}#1}}
\newcommand{\DODSmgui}{\xlink%
  {\cit{The OPeNDAP Matlab GUI}}{\DODSmguiUrl}}
\newcommand{\DODSapi}{\xlink%
  {\cit{The DODS Toolkit Programmer's Guide}}{\DODSapiUrl}}
\newcommand{\DODSapiref}{\xlink%
  {\cit{The DODS Toolkit Reference}}{\DODSapirefUrl}}
\newcommand{\DODSffbook}{\xlink%
  {\cit{The DODS Freeform ND Server Manual}}{\DODSffUrl}}
\newcommand{\DODSquick}{\xlink%
  {\cit{The DODS Quick Start Guide}}{\DODSquickUrl}}
\newcommand{\DODSinstall}{\xlink%
  {\cit{The DODS Server Installation Guide}}{\DODSinstallUrl}}
\newcommand{\DODSregex}{\xlink%
  {\cit{Introduction to Regular Expressions}}{\DODSregexUrl}}
\newcommand{\DODSagg}{\xlink%
  {\cit{OPeNDAP Aggregation Server Guide}}{\DODSaggUrl}}
\newcommand{\DODSwclient}{\xlink%
  {\cit{Writing an OPeNDAP Client}}{\DODSwclientUrl}}
\newcommand{\DODSwserver}{\xlink%
  {\cit{Writing an OPeNDAP Server}}{\DODSwclientUrl}}

\newcommand{\DODSffs}{DODS Freeform ND Server}

% $Log: dods-def.tex,v $
% Revision 1.24  2004/12/21 22:30:04  jimg
% Fixed pslink; Added pdflink.
%
% Revision 1.23  2004/12/14 05:19:17  tomfool
% restored fix to pslink
%
% Revision 1.22  2004/12/09 21:01:58  tomfool
% excised test.dods.org
%
% Revision 1.21  2004/12/09 18:50:21  tomfool
% de-dodsifying
%
% Revision 1.15  2004/04/24 21:37:22  jimg
% I added every directory in preparation for adding everyting. This is
% part of getting the opendap web pages going...
%
% Revision 1.14  2004/02/12 16:05:50  jimg
% Moved the log to the end of the file.
%
% Revision 1.13  2004/01/16 18:05:31  jimg
% Added a note from Tom about setting texmf.cnf to allow \include to process
% files with ../ in their pathnames. You can also change the include to input,
% but I think include may offer some advantages for bigger/complex things like
% the Guides.
%
% Revision 1.12  2003/12/28 21:48:22  tom
% added newer books
%
% Revision 1.11  2003/12/08 19:04:43  tom
% little adjustments for DODS->opendap
%
% Revision 1.10  2003/12/08 18:53:30  tom
% DODS->OPeNDAP
%
% Revision 1.9  2002/07/15 17:49:55  tom
% added \DODSDoc
%
% Revision 1.8  2001/05/04 15:07:45  tom
% fixed pslink to include pdf files
%
% Revision 1.7  2001/02/19 20:39:13  tom
% added links to the new regex intro.
%
% Revision 1.6  2000/03/23 18:27:52  tom
% added abbreviations
%
% Revision 1.5  1999/07/01 16:00:19  tom
% added a couple of web page references
%
% Revision 1.4  1999/05/25 20:49:34  tom
% changed version numbers to 3.0
%
% Revision 1.3  1999/02/04 17:27:08  tom
% adjusted for hyperlatex and dods-book.cls
%
%

%%% Local Variables: 
%%% mode: latex
%%% TeX-master: t
%%% TeX-master: t
%%% End: 

% 
% These are some definitions for the Quick start guide.  Some of these
% should find their way into the dods-book class.
%
% $Id$
%

\htmldirectory{quick-html}
\htmlname{quick}
\htmltitle{DODS Quick Start Guide}
\htmladdress{Tom Sgouros, \rcsInfoDate}
%\htmlcss{/css/quickstyle.css}
%\W\renewcommand{\HlxIcons}[1]{http://top.gso.uri.edu/icons.n/}

\htmlcss{/resources/dods-book.css}
\W\renewcommand{\HlxIcons}[1]{/icons/}

% $Log: quick-def.tex,v $
% Revision 1.4  2004/07/07 22:19:15  jimg
% Added icons and dods-book.css
%
% Revision 1.3  2004/04/24 21:37:25  jimg
% I added every directory in preparation for adding everyting. This is
% part of getting the opendap web pages going...
%
% Revision 1.2  2000/10/30 20:13:37  tom
% pointed icons and css to unidata.
%
% Revision 1.1  2000/10/12 17:24:36  tom
% added quick.tex to repository

%%% Local Variables: 
%%% mode: latex
%%% TeX-master: t
%%% TeX-master: t
%%% End: 


\figpath{quick/figs}

% This is a kluge to use brackets in the optional argument 
% to vcode environments.
\newcommand{\brk}[1]{[#1]}

\makeindex

\begin{document}
%------------------------------------front matter
\title{An \opendap Quick Start Guide\\\DOCversion}
\author{Tom Sgouros}
\date{\rcsInfoDate}
\pagenumbering{roman}
\maketitle

\copyrightmatter

\W\pslink{http://www.opendap.org/pdf/quick.pdf}

%% Preface to the DODS Programmer's Guide.
%
% $Id$
%
% $Log: preface.tex,v $
% Revision 1.5  2004/02/18 06:38:48  jimg
% Various changes, mostly for the DODS --> OPD macros.
%
% Revision 1.4  1999/07/22 18:54:56  tom
% fixed errors
%
% Revision 1.3  1999/02/04 17:46:05  tom
% Modified for dods-book.cls and Hyperlatex
%
% Revision 1.2  1998/12/07 15:51:33  tom
% updated for DODSv2.19
%
% Revision 1.1  1998/03/13 20:50:04  tom
% created API manual from James's Toolkit document
%
%
%

\T\chapter*{Preface}
\T\addcontentsline{toc}{chapter}{Preface}

This document describes how to use the \opendap\ toolkit software to build
\opendap\ data servers, clients and client-libraries. Using the objects and
functions contained in the toolkit, you can create programs which serve data
over the internet as well as programs that can request data from any
\opendap\ server.

This document covers release \DODSversion\ and later of the DODS
software.

\begin{ifhtml}
  \htmlmenu{4}
  \chapter*{Preface}
\end{ifhtml}

%%%%%%%%%%%%%%%%%%%%%%%%%%%%%%%%%%%%%%%%%%%%%%%%%%%%%%%%%%%%%%%%%%%%
\section{Who is this Guide for?}
\label{pref,who}

This guide is for people who wish to use the \opendap\ software to write a
new \opendap\ data server, a new client, or a new client library. Typically,
this will only be those people who wish to serve data in a format that is not
currently supported by the DODS team, or who have an existing application
that uses an idiosyncratic or unusual API for data access. Most people will
be able to use one of the already written servers or client libraries. See
the \OPDuser\ for a list of these.

This documentation assumes that the readers are \Cpp\ programmers, are
familiar with networked applications, and the POSIX programming environment.
The DODS/\opendap project also provides a native Java class library (API)
that parallels the \Cpp\ spftware described here.\footnote{While this manual
describes the \Cpp\ toolkit in detail, all of the concepts and much of the
structure can be directly translated to the Java toolkit.}

Also available are two tutorials, \OPDwclient\ and \OPDwserver\ , which
descrie how to write a client or a server, respectively.

Because the type of information presented in a document like this depends to
a large extent on the needs of its readers we welcome your feedback and
comments. In particular, if you have any questions about individual sections,
email those questions and we'll send back an answer as well as including that
information in the next version of this document. Send queries to:
\DODSsupport.


%%%%%%%%%%%%%%%%%%%%%%%%%%%%%%%%%%%%%%%%%%%%%%%%%%%%%%%%%%%%%%%%%%%%
\section{Organization of this Document}

This Guide is divided into five chapters. 

\begin{description}
  
\item[\chapterref{tk,overview}] provides background information on the
  organization of the toolkit software.

\item[\chapterref{tk,manage-conns}] describes how to use the
Network I/O classes to manage virtual connections.

\item[\chapterref{tk,subclassing}] discusses how to sub-class the toolkit
\Cpp\ classes so that they are specialized for your specific use.

\item[\chapterref{tk,using}] describes in detail how to write certain
sections of both the data server and the client-library for a new API.

\item[\chapterref{tk,linking}] describes how to link user programs
with the new client-library implementation of an API. 

\texonly{\item[\chapterref{tk,classref}] contains complete
  descriptions of all the DODS classes.}

\end{description}

\htmlonly{Note that this is {\em not\/} a reference volume. See
  \OPDapiref\ for a concise listing of the member functions in each
  of the toolkit's classes.}

%%%%%%%%%%%%%%%%%%%%%%%%%%%%%%%%%%%%%%%%%%%%%%%%%%%%%%%%%%%%%%%%%%%%

\listconventions

%%% Local Variables: 
%%% mode: latex
%%% TeX-master: "../pguide.tex"
%%% End: 


\tableofcontents
\listoffigures
%\listoftables

\clearemptydoublepage
%------------------------------------book body

\pagenumbering{arabic}

%% Chapter one to the DODS Data Standards Guide
%
% $Id$
%
% $Log: ch01.tex,v $
% Revision 1.1  1999/06/07 05:09:32  tom
% added to CVS
%
%
%

\chapter{DODS and Data Standards}
\label{intro}

Once upon a time there was DODS.  And it was good.  DODS solved many
problems of dataset incompatibility by making remote datasets pretend
to be in a format they were not in.  This was tricky, and it was
largely accomplished by hiding from the user the actual data storage
format, and using many different converters to provide data in a
standard transmission format.

However, data storage format isn't everything.  Though the DODS
solution is quite a useful one, it has its limitations.  Two datasets
can be incompatible even though they share an identical data storage
format.  For example, consider two sea surface temperature datasets
stored in netCDF format files.  One can store its temperature in
degrees Celsius, and another can store temperature as raw data, in
unscaled satellite counts, that have to be converted into temperature
values meaningful to a human.


\section{Levels of Compatibility}
\label{sec:compat}

In discussions about whether two different datasets are compatible, we
must distinguish between different levels of compatibility.  What
level is important depends to some extent on how you intend to use the
data.  To a human, for example, two datasets could be considered
compatible if they are about the same data variable.  Two temperature
datasets that cover the same area might be considered quite compatible,
if all one is doing is looking at two different color contour maps
lying on a table, or side-by-side on a computer screen.

However if one is seeking datasets on which to execute various
analyses and computations, a higher level of compatibility is
required.

The issues of data compatibility can be separated into four
categories:

\begin{itemize}
\item Data storage format

\item Data attributes 

\item Data organization

\item Server interface

\end{itemize}


\subsection{Data Storage Format}

Two datasets whose data are stored and retrieved with different data
format software are incompatible.  A dataset stored with netCDF
software cannot be read by a program to read HDF files, and vice
versa. 

Data storage format was the first compatibility issue
that DODS addressed, and it does so fairly successfully.  Whatever
storage format data are kept in, the DODS server translates them into
a standard transmission format before sending them to the DODS client
requesting that data.  This translation happens file by file, however,
so datasets that span multiple files may remain somewhat problematic,
even though the file format itself is not an issue.


\subsection{Data Attributes}

Having solved the problem of data storage compatibility, DODS was
faced with the problem of incompatible data attributes.  That is,
considering our two temperature datasets, one dataset could refer to
its temperature data in an array called ``T'', while another could
refer to an array called ``Temp''.  To a human comparing the two
files, the equation between the two might be obvious, but to a
computer it might not.  (Even to a human, the facts might not be as
obvious as they seem.  For example, ``T'' could also refer to
``Time'')

Other data attributes that are important to be able to compute with a
set of data include flags for missing data and unit and unit
conversion information.  The data standard outlined in
\chapterref{dods-standard} of this document is an attempt to address
this issue.



\subsection{Data Organization}

The organization of a dataset can create another impediment to
compatibility.  Consider, for example, two datasets containing
salinity measurements of the ocean at various different depths, and at
different times.  Assuming that there is some reason not to put all
the information in a single file, each dataset must be divided among
several files, and this is where certain incompatibilitles can sneak
in. 

One dataset could store its data in several files, each of which
contains a three-dimensional array---corresponding to latitude,
longitude, and depth---of salinity values, corresponding to a
particular time.  But a similar dataset might be arranged as arrays of
latitude, longitude, and time, with several different files
corrsponding to different depths.  Both arrangements are, in fact,
common among on-line datasets.

The DODS Catalog Server, under development (as of \today), is intended
to address the issue of data organization within a dataset, allowing a
user to query differently organized datasets with the same syntax
query. 


\subsection{Server Interface}

The DODS software does create solutions to the compatibility issues
outlined above.  However, in doing so, it introduces a fourth
compatibility issue, which is that of the server interface itself.  A
DODS server is programmed to respond to a set of server commands
issued by a DODS client.  These commands come in the form of URLs sent
to the server.

Because the DODS project is a widely distributed project, and because
the datasets available through DODS are---by design---not under
central control, it is not always certain that two given DODS servers
will be running software from the same software release.  Similarly,
since there are provisions for making local modifications to server
behavior, it is also not certain that server functions available on
one server are available on another, even when they are both running
at the same software release.

Addressing these issues is an ongoing preoccupation of the DODS
project and its (growing) user base.  The DODS servers also contain
provisions for making this information available to clients. See
\DODSuser\ for more information.


















%%% Local Variables: 
%%% mode: latex
%%% TeX-master: t
%%% End: 



%  Outline of this book:
%
%
%  A Quick Start: What to do with a URL

\T\chapter{What To Do With An \opendap URL}

The \OPD\ is a system that allows you to access data over the internet,
from programs that weren't originally designed for that purpose, as
well as some that were.

With \opendap, you access data using a URL, just like a URL you would use
to access a web page.  However, before you request any data, you need
to know how to request it in a form your browser can handle.  \opendap
data is stored in binary form, and by default, it is transmitted that
way, too.\indc{DODS!URL}\indc{data!accessed by URL}\indc{URL!data accessed by}

The other problem with an \opendap URL is that a single URL might point to
an archive containing 50 megabytes of data.  You rarely want to
request the whole thing without knowing a little about it.  \opendap
provides sophisticated sub-sampling capabilities, but you need to know
a little bit about the data in order to use them.

\texorhtml{}{So here's what to do if someone gives you a raw URL, and
  says there's some \opendap data on the other end.

\htmlmenu{4}
\chapter{What To Do With An \opendap URL}}
\label{reynolds,chapter}

Suppose someone gives you a hot tip that there's a lot of good data
at:

\begin{vcode}{sib}
http://www.cdc.noaa.gov/cgi-bin/nph-nc/Datasets/reynolds_sst/sst.mnmean.nc
\end{vcode}

This URL points to monthly means of sea surface temperature,
worldwide, compiled by Richard \ind{Reynolds} at the Climate Modeling branch
of NOAA, but pretend you don't know that yet.\indc{Climate
  Modeling!NOAA}\indc{NOAA!Climate Modeling} 

The simplest thing you can do with this URL is to download the data it
points to.  You could feed it to an \opendap-enabled data analysis package
like Ferret, or you could append \lit{.asc}, and feed the URL to a
regular web browser like Netscape.  This will work, but you don't
really want to do it because in binary form, there are about 28
megabytes of data at that URL.

\note{An \opendap server will work with many different clients, some of
  which are supported by the \opendap team, and some of which are
  supported by others.  The operation of any individual package is
  beyond the scope of this manual.  This guide explains how to use a
  typical web browser such as Netscape Navigator to discover
  information about the data that will be useful when analyzing data
  in \emph{any} package.}

\subj{You need to sample the data}
A better strategy is to find out some information about the data.
\opendap has sophisticated methods for subsampling data at a remote site,
but you need some information about the data first.  First, we'll try
looking at the data's \new{Dataset Descriptor Structure} (\ind{DDS}).  This
provides a description of the ``shape'' of the data, using a vaguely
C-like syntax.  You get a dataset's DDS by appending \lit{.dds} to the
\xlinkn{URL}{http://www.cdc.noaa.gov/cgi-bin/nph-nc/Datasets/reynolds_sst/sst.mnmean.nc.dds}.\indc{dataset!shape}\indc{shape of dataset}

\figureplace{An \opendap DDS (\lit{sst.mnmean.nc.dds})}{htb}
{reynolds,dds}{reynolds-dds.ps}{reynolds-dds.gif}{http://www.cdc.noaa.gov/cgi-bin/nph-nc/Datasets/reynolds_sst/sst.mnmean.nc.dds}

From the DDS shown, you can see that the dataset consists of five
pieces: 

\subj{Find out what's in the data}
\begin{itemize}
\item A 180-element vector called ``lat'',
\item A 360-element vector called ``lon'',
\item A 226-element vector called ``time'',
\item A ``Grid'' containing a three-dimensional array of integer
  values (\lit{Int16}) called \lit{sst}, and three ``Map'' vectors,
  which may look familiar, and
\item Another Grid called \lit{mask}.
\end{itemize}

The \new{Grid} is a special \opendap data type that includes a
multidimensional array, and \new{map vectors} that indicate the
independent variable values.  That is, you can use a Grid to store an
array where the rows are not at regular intervals.  \texorhtml{There's
  a simple grid in figure~\ref{grid,diagram}.}{Here's a simple grid:}
\indc{array!irregular|see{Grid}}\indc{array!map vector}\indc{vector!map}

\figureplace{A Grid}{h}{grid,diagram}{gridpts.ps}{gridpts.gif}{}

The array part of the grid would contain the data points measured at
each one of the squares, the X map vector would contain the positions
of the columns, and the Y map vector would contain the positions of
the rows.

Of course you can also use a Grid to store arrays where the columns
and rows are at regular intervals, and you'll often see \opendap data that
way. 

(The other special \opendap data type worth worrying about is the
\emph{Sequence}.  You'll see more about them in
section~\ref{quick,sequences}.  There are also \new{Structures} and
\new{Lists}, but they exist largely for internal uses, and you don't
often see these used in real datasets.)

You can see from the DDS that the Reynolds data is in a 180x360x226
element grid, and the dimensions of the Grid are called ``lat'',
``lon'', and ``time''.  This is suggestive, but not as helpful as one
could wish.  To find out more about what the data \emph{is}, you can
look at the other important \opendap structure: the \ind{DAS}, or
\new{Data Attribute Structure}.  This is somewhat similar to the DDS,
but contains information about the data, such as units and the name of
the variable.  Part of the DAS for the Reynolds data we saw above is
\texorhtml{shown in ~\ref{reynolds,das}.}{shown in the figure below.
  Click \xlinkn{here}{http://www.cdc.noaa.gov/cgi-bin/nph-nc/Datasets/reynolds_sst/sst.mnmean.nc.das}
  or on the figure to see the rest of it.}

\figureplace{An \opendap DAS (\lit{sst.mnmean.nc.das})}{h}
{reynolds,das}{reynolds-das.ps}{reynolds-das.gif}
{http://www.cdc.noaa.gov/cgi-bin/nph-nc/Datasets/reynolds_sst/sst.mnmean.nc.das}

\note{The DAS is populated at the data provider's discretion.  Because
  of this, the quality of the data in it (the \new{metadata}) varies
  widely.  The data in the Reynolds dataset used in this example are
  \ind{COARDS} compliant.  Other metadata standards you may encounter with
  \opendap data are \ind{HDF-EOS}, \ind{EPIC}, \ind{FGDC}, or no metadata at all.}

\subj{Find out more about the data variables}
Now we can tell something more about the data.  Apparently the
\lit{lat} vector contains latitude, in degrees north, and the range is
from 89.5 to -89.5.  Since this is a global grid, the latitude values
probably go in order.  We can check this by asking for just the
latitude vector, like \xlinkn{this}%
{http://www.cdc.noaa.gov/cgi-bin/nph-nc/Datasets/reynolds_sst/sst.mnmean.nc.asc?lat}:

\begin{vcode}[.]{sib}
http://www.cdc.noaa.gov/cgi-bin/nph-nc/Datasets/reynolds_sst/sst.mnmean.nc.asc?lat
\end{vcode}

What we've done here is to append a \new{constraint expression}
 to the \opendap URL, to indicate how to
constrain our request for data.  Constraint expressions can take many
forms.  This guide will only describe a few of them.  (You can refer
to the \OPDuser\ for more complete information about constraint
expressions.)  Try requesting the
\xlinkn{time}%
{http://www.cdc.noaa.gov/cgi-bin/nph-nc/Datasets/reynolds_sst/sst.mnmean.nc.asc?time}
and \xlinkn{longitude}
{http://www.cdc.noaa.gov/cgi-bin/nph-nc/Datasets/reynolds_sst/sst.mnmean.nc.asc?lon}
\subj{The info service also provides the DAS and DDS information.}
vectors to see how this works.
\indc{expression!constraint|see{constraint expression}}

According to the DAS, time is kept in ``days since 1-1-1 00:00:00'' in
this dataset.  You can also learn from the DAS the actual time period
recorded in the data which, because of your familiarity with the
\ind{Julian calendar}, you instantly recognize as beginning in November,
1981.  You might also notice that the \lit{mask} array is used to
indicate land and sea, and has only the values 0 and
1.\indc{calendar!Julian} 

\indc{service!help}\indc{help service}\indc{service!info} 
\opendap provides an \new{info service} that returns all the information
we've seen so far in a single 
request.  The returned information is also formatted differently (some
would say ``nicer''), and you can occasionally find server-specific
documentation here, as well.  Some will find this the easiest way to
read the attribute and structure information.  You can see what
information is available by appending \lit{.info} to a URL, like
\xlinkn{this}
{http://www.cdc.noaa.gov/cgi-bin/nph-nc/Datasets/reynolds_sst/sst.mnmean.nc.info}:

\begin{vcode}[.]{sib}
http://www.cdc.noaa.gov/cgi-bin/nph-nc/Datasets/reynolds_sst/sst.mnmean.nc.info
\end{vcode}

\section{Peeking at Data}

Now that we know a little about the shape of the data, and the data
attributes, let's look at some of the data.

\subj{Use subscripts to sample a Grid.}
You can request a piece of an array with \ind{subscripts},
just like in a C program or in Matlab or many other computer
languages.  Use a colon to indicate a subscript range.
\indc{Grid!sampling}\indc{sampling!Grid}\indc{array!subscripts}
\indc{array!sampling}\indc{sampling!array}\indc{Grid!subscripts}
\indc{peeking at data}\indc{data!peeking at}

\begin{vcode}[http://www.cdc.noaa.gov/cgi-bin/nph-nc/Datasets/reynolds_sst/sst.mnmean.nc.asc?time\[0:6\]]{sib}
...sst/mnmean.nc.asc?time[0:6]
\end{vcode}

This \xlinkn{URL}{http://www.cdc.noaa.gov/cgi-bin/nph-nc/Datasets/reynolds_sst/sst.mnmean.nc.asc?time[0:6]} will produce \texorhtml{figure~\ref{reynolds,timevec}}{the following:}

\figureplace{Part of a vector.}{h}{reynolds,timevec}{timevec.ps}{timevec.gif}{}

You can do the 
\xlinkn{same}
{http://www.cdc.noaa.gov/cgi-bin/nph-nc/Datasets/reynolds_sst/sst.mnmean.nc.asc?mask[28:30][206:209]}
for one of the grids:

\begin{vcode}[http://www.cdc.noaa.gov/cgi-bin/nph-nc/Datasets/reynolds_sst/sst.mnmean.nc.asc?mask\[28:30\]\[206:209\]]{sib}
...sst/mnmean.nc.asc?mask[28:30][206:209]
\end{vcode}

\subj{Sampling a Grid produces part of the Grid, including the map vectors.}
Which produces a portion of the land mask somewhere near Alaska's
Kenai peninsula\texorhtml{, shown in figure~\ref{reynolds,mask}}{:}

\figureplace{Part of an \opendap Grid.}{h}{reynolds,mask}{mask.ps}{mask.gif}{http://www.cdc.noaa.gov/cgi-bin/nph-nc/Datasets/reynolds_sst/sst.mnmean.nc.asc?mask[28:30][206:209]}

Notice that when you ask for part of an \opendap Grid, you get the array
part along with the corresponding parts of the map vectors.

If you are interested in the Reynolds dataset, you are probably more
interested in the sea surface temperature data than the land mask.
The temperature data is a three-dimensional grid.  To sample the 
\xlinkn{\lit{sst}}
{http://www.cdc.noaa.gov/cgi-bin/nph-nc/Datasets/reynolds_sst/sst.mnmean.nc.asc?sst[12:13][28:30][206:209]}
Grid, you
just add a dimension for time:

\begin{vcode}[http://www.cdc.noaa.gov/cgi-bin/nph-nc/Datasets/reynolds_sst/sst.mnmean.nc.asc?sst\[12:13\]\[28:30\]\[206:209\]]{sib}
...sst/mnmean.nc.asc?sst[12:13][28:30][206:209]
\end{vcode}

This produces something like\texorhtml{ the figure shown in
  figure~\ref{reynolds,sst}}{this:}

\figureplace{Part of the Reynolds SST data}{h}{reynolds,sst}{sst.ps}{sst.gif}{http://www.cdc.noaa.gov/cgi-bin/nph-nc/Datasets/reynolds_sst/sst.mnmean.nc.asc?sst[12:13][28:30][206:209]}

\indc{units!scaling}
Note that the sst values are in celsius degrees multiplied by 100, as
indicated by the \lit{\ind{scale_factor}} attribute of the \xlinkn{DAS}
{http://www.cdc.noaa.gov/cgi-bin/nph-nc/Datasets/reynolds_sst/sst.mnmean.nc.das}.
Further, it's important to remember with this dataset, that the data
were obtained by calculating spatial and temporal means.
Consequently, the data points in the \lit{sst} array should be ignored
when the corresponding entry in the \lit{mask} array indicates they
are over land.

%************THIS SECTION COMMENTED OUT BECAUSE SAMPLING GRIDS BY
%************VALUE DOESN'T CURRENTLY WORK.  WHEN IT'S FIXED, 
%************UNCOMMENT THIS SECTION.

%\subsection{Sampling Grids by Value}

%It is not always the easiest thing to figure out the array dimensions
%you need.  OPeNDAP servers provide a way to sample a Grid using the
%values of the dependent variables (the map vectors).  A function
%called \lit{grid} is used to examine a Grid and its map vectors, and
%translate relational clauses into array subscripts.

%The Reynolds data we've been looking at is a 226x180x360 element
%array.  What's more important now is that the \lit{time} dimension
%varies from 723,486 to 730,334 (days since January 1, 0001), the
%\lit{lat} dimension varies from 89.5 to -89.5, and the \lit{lon}
%dimension varies from 0.5 to 359.5.  To request a series of worldwide
%data arrays, covering the span of time between 730,000 and the
%present, you can formulate a constraint expression like 
%\xlinkn{this}
%{http://www.cdc.noaa.gov/cgi-bin/nph-nc/Datasets/reynolds_sst/sst.mnmean.nc.asc?grid(sst, "time>730000")}:

%\begin{vcode}{sib}
%...sst.mnmean.nc.asc?grid(sst, "time>730000")
%\end{vcode}

%There is no limit to the number of clauses that can be included in the
%parentheses (although all bets are off if you include clauses that
%conflict with one another).  So you can be more precise with a
%constraint expression like \xlinkn{this}
%{http://www.cdc.noaa.gov/cgi-bin/nph-nc/Datasets/reynolds_sst/sst.mnmean.nc.asc?grid(sst, "time > 730000", "lat>0", "lat<55", "lon<100")}:

%\begin{vcode}{sib}
%...sst.mnmean.nc.asc?grid(sst, "time>730000", "lat>0", "lat<55", "lon<100")
%\end{vcode}

%\note{As of \today\ this form of sampling is broken, and does not work.}


\section{Sequence Data}
\label{quick,sequences}

Gridded data works well for satellite images, model data, and data
compilations such as the Reynolds data we've just looked at.  Other
data, such as data measured at a specific site, is not so readily
stored in that form.  \opendap provides a data type called a \new{Sequence}
to store this kind of data.

\subj{A Sequence is a relational table.}
A Sequence can be thought of as a \ind{relational data table}, with each
column representing a different data value, and each row representing
a different data ``instance.''  For example, an ocean \ind{temperature
profile} can be stored as a Sequence of pressure and temperature pairs,
and a weather station's data can be stored as a Sequence with time in
one column, and each weather variable occupying another column.
\indc{data!relational table}\indc{RDBMS}\indc{profile!temperature}

Let's look at a couple of Sequences.  The first one is a collection of
\ind{CTD data} (\ind{hydrographic data}, including temperature, pressure,
salinity, and so on):\indc{data!hydrographic}\indc{station data}
\indc{data!station}\indc{data!time series}\indc{time series data}

\begin{vcode}{sib}
http://dods.gso.uri.edu/cgi-bin/nph-jg/rlctd
\end{vcode}

The \xlinkn{DAS}{http://dods.gso.uri.edu/cgi-bin/nph-jg/rlctd.das}
(append \lit{.das} to the URL) for this data is pretty uninformative,
telling us only that all the data are stored as strings\texorhtml{.
  You can see this in figure~\ref{rlctd,das}.}{:}

\figureplace{A DAS for Sequence data.}{h}{rlctd,das}{rlctd-das.ps}%
{rlctd-das.gif}{http://dods.gso.uri.edu/cgi-bin/nph-jg/rlctd.das}

\subj{You can sometimes find data attributes among the data.}
On the other hand, a lot of the information we would get from the DAS
is actually encoded in the data itself, which you can see by looking
at the data's
\xlinkn{DDS}{http://dods.gso.uri.edu/cgi-bin/nph-jg/rlctd.dds} (append
\lit{.dds} to the URL)\texorhtml{, shown in figure~\ref{rlctd,dds}.}{:}
\indc{metadata!among the data}\indc{data!metadata mixed in}

\figureplace{A DDS for Sequence data.}{h}{rlctd,dds}{rlctd-dds.ps}%
{rlctd-dds.gif}{http://dods.gso.uri.edu/cgi-bin/nph-jg/rlctd.dds}

We can get some idea of the data coverage by asking for some of the
time and location data, with a URL like this:

\begin{vcode}[http://dods.gso.uri.edu/cgi-bin/nph-jg/rlctd.asc?cruiseid,station,year_s,month_s,day_s,lat_s,lon_s]{sib}
...rlctd.asc?cruiseid,station,year_s,month_s,day_s,lat_s,lon_s
\end{vcode}

This produces a response shown \texorhtml{in
  figure~\ref{rlctd,coverage}.}{\xlinkn{here}{http://dods.gso.uri.edu/cgi-bin/nph-jg/rlctd.asc?cruiseid,station,year_s,month_s,day_s,lat_s,lon_s}.}

% This has changed slightly, and probably should be regenerated.
\figureplace{The \lit{rlctd} dates and locations}{h}{rlctd,coverage}
{rlctd-cov.ps}{rlctd-cov.gif}{http://dods.gso.uri.edu/cgi-bin/nph-jg/rlctd.asc?cruiseid,station,year_s,month_s,day_s,lat_s,lon_s}


\subj{Use a selection clause to select Sequence rows.}  
After reviewing the data in the last request, perhaps we decide we
only want to see data from one of the cruises listed, or maybe only
data from the month of May.  We can add a \new{selection clause} to
the constraint expression to select only that data.  For example:
\indc{constraint expression!selection clause}

\begin{vcode}[http://dods.gso.uri.edu/cgi-bin/nph-jg/rlctd.asc?cruiseid,station,year_s,month_s,day_s,lat_s,lon_s&month_s=5]{sib}
...rlctd.asc?cruiseid,station,year_s,month_s,day_s,lat_s,lon_s&month_s=5
\end{vcode}

This produces a table containing all the rows from
\texorhtml{figure~\ref{rlctd,coverage}}{the last example} where the
month datum is May.  \texorhtml{Try entering the new URL in your browser
  and see what you get.}{Click \xlinkn{here}%
{http://dods.gso.uri.edu/cgi-bin/nph-jg/rlctd.asc?cruiseid,station,year_s,month_s,day_s,lat_s,lon_s\&month_s=5}
to see that table.}

Selection clauses can be stacked endlessly against a URL, allowing all
the flexibility most people need to sample data files.  Here's an
example of a
\xlinkn{URL}{http://dods.gso.uri.edu/cgi-bin/nph-jg/rlctd.asc?o2\&month_s=5\&pres>50\&pres<100}
that requests all the oxygen data in the file taken in May at a
specific depth range:

\begin{vcode}[http://dods.gso.uri.edu/cgi-bin/nph-jg/rlctd.asc?o2&month_s=5&pres>50&pres<100]{sib}
...rlctd.asc?o2&month_s=5&pres>50&pres<100
\end{vcode}

The first clause in a constraint expression has a name, too.  It is
the \new{projection clause}.  This is the list of variables that you
wish to have returned, subject to the constraint of the selection
clause.  In the previous example, the projection clause consiste only
of the \lit{o2} variable.  In the one before that, the list was
longer, containing 7 variables.\indc{constraint expression!projection
  clause} 



\tbd{There is a get_row() method for Sequences now, so that you can
  select a sequence row by its ordinal number.  When this makes it
into the server releases, document it.}



\section{An Easier Way}

\subj{The \opendap query form is an easier way to sample data.}
\opendap also includes a way to sample data that makes writing a
constraint expression somewhat easier.  Append \lit{.html} to the URL,
and you get a form that directs you to add information to sample the
data at a \xlinkn{URL}
{http://www.cdc.noaa.gov/cgi-bin/nph-nc/Datasets/reynolds_sst/sst.mnmean.nc.html}:
\indc{constraint expression!query form}
\indc{constraint expression!building aids}
\indc{query form}\indc{form!query}
\indc{web interface}\indc{ifh}\indc{html interface}
\indc{Data Access Form}

\begin{vcode}[http://www.cdc.noaa.gov/cgi-bin/nph-nc/Datasets/reynolds_sst/sst.mnmean.nc.html]{sib}
...sst.mnmean.nc.html
\end{vcode}

Sending a URL ending in \lit{.html} returns a form like this:

\figureplace{The \opendap Dataset Access Form}{h}{reynolds,ifh}{ifh.ps}{ifh.gif}{http://www.cdc.noaa.gov/cgi-bin/nph-nc/Datasets/reynolds_sst/sst.mnmean.nc.html}

It's useful to have a browser window open with one of these query forms in it
while you read this section.  \texorhtml{}{Click
  \xlinkn{here}
  {http://www.cdc.noaa.gov/cgi-bin/nph-nc/Datasets/reynolds_sst/sst.mnmean.nc.html}
  to bring up a copy of the form to use while you read.}

Near the top of the page, you'll see a box entitled ``Data URL''.  At
this point, if you've been following along, it should look pretty
familiar.  If you're just jumping in, it's the \opendap URL connected to
the data we're interested in, but unsampled.

Moving down the page, there is a list of ``Global Attributes'', which
is really just for your perusal.  At this point, there's not much to
be done with this, but it is often helpful information.

\subj{Select variables by clicking on a checkbox.}  The important part
of the page is the ``Variables'' section.  For each variable in the
dataset, you'll see the data description (e.g. ``Array of 32 bit Reals
[lat = 0..179]''), a checkbox, a text input box, and a list of the
variable's attributes.  If you click on the checkbox, you'll see the
variable's array bounds appear in the text box, and you'll see that
variable appear in a constraint expression appended to the Data URL at
the top of the page.  If you edit the array bounds in the text box,
hitting ``enter'' will place your edits in the Data URL box.

In the oh-so-unlikely event you dare try all this without your
documentation \new{vade mecum} along, there's a \but{Show Help}
button up near the top of the page.  Clicking there will show you
instructions about how to proceed.

\note{You'll see a ``stride'' mentioned.  This is another way to
  subsample an \opendap array or Grid.  Asking for \lit{lat[0:4]} gets you the first
  five members of the \lit{lat} array.  Adding a stride value allows
  you to skip array values.  Asking for \lit{lat[0:2:10]} gets you
  every second array value between 0 and
  10: 0, 2, 4, 6, 8, 10.\indc{array!stride}\indc{stride}} 
\indc{Grid!sampling}\indc{sampling!Grid}\indc{array!subscripts}
\indc{array!sampling}\indc{sampling!array}\indc{Grid!stride}
\indc{Grid!subscripts}

Move on down the variable list, editing your request, and experiment
with adding and changing variable requests.

When you have a request you'd like to make, look at the buttons at the
top of the page.  

\figureplace{Dataset Access Form Detail}{h}{reynolds,ifh-buttons}
{ifh-buttons.ps}{ifh-buttons.gif}{}

You can click on \but{Get ASCII}, and the data
request will appear in a browser window, in comma-separated form.  The
\but{Get Binary} button will save a binary data file on your local
disk.  (The \but{Send to Program} will send the URL directly to an \opendap
client.  However, it requires a suitable \opendap client to be running on
your computer, and also requires you to install a helper application
for your browser.  There are instructions for doing this at the \opendap
home page.)

\subj{The web interface works for Sequence data, too.}
The \opendap Data Access Form interface works for Sequence data as well as
Grids.  However, since Sequence constraint expressions look different
than Grid expressions, the form looks slightly different, too.  You
can see \texorhtml{from figure~\ref{rlctd,ifh-seq}}{below} that the
variable selection boxes allow you to enter relational expressions for
each variable.  Beside that, however, the function is exactly the same.
\indc{constraint expression!query form}
\indc{constraint expression!building aids}
\indc{query form!Sequence data}\indc{form!query}
\indc{web interface!Sequence data}\indc{html
  interface!Sequence data}
\indc{Data Access Form!Sequence Data}

\figureplace{Dataset Access Form for Sequence Data (detail)}{h}
{rlctd,ifh-seq}{ifh-seq.ps}{ifh-seq.gif}{http://dods.gso.uri.edu/cgi-bin/nph-jg/rlctd.html}

\texorhtml{}{Click \xlinkn{here}
  {http://dods.gso.uri.edu/cgi-bin/nph-jg/rlctd.html} to see a copy of
  a Sequence form.}

\note{Not all \opendap servers support all the \opendap functionality.  There
  are a few non-standard \opendap servers out there in the world that only
  support the bare minimum required.  That minimum is to respond to
  queries for the DDS, DAS, and (binary) data.  The ASCII data and the
  web access form are optional add-ons that are not required for the
  basic \opendap function.}\indc{minimum server configuration}\indc{server!minimum configuration}


\chapter{Finding More \opendap URLs}

The \opendap package was developed to improve ways to share data among
scientists.  Many times, data comes in the form of a URL enclosed in
an email message.  But there are several other ways to find data served
by \opendap servers.

\section{GCMD}

\subj{The GCMD now catalogs \opendap URLs!}  The \xlink{Global Change
  Master Directory}{http://gcmd.gsfc.nasa.gov} is a source of a huge
amount of earth science data.  They now catalog \opendap URLs for the
datasets that have them.  You can search on ``\opendap'' right from the
main page to find many of these datasets.  Try that search, then click
on one of the data set names that returns, and look at the bottom of
the resulting Set Description'' page, under the heading ``Related
URL.''\indc{GCMD}\indc{Global Change Master Directory}
\indc{NASA!Global Change Master Directory}
\indc{URL!catalogued at GCMD}

If you make that search, check the list for the Reynolds data from
chapter~\ref{reynolds,chapter}; it should be there.

\section{\opendap Dataset List}

The \OPDhome\ has a list of available \opendap datasets.  Click on
\xlink{\but{Datasets}}
{http://www.unidata.ucar.edu/cgi-bin/dods/datasets/datasets.cgi?xmlfilename=datasets.xml}
\subj{The \opendap project supports an ad hoc list of data URLs.}
in the table of contents\texorhtml{}{ or right here}.  You can find a URL and a
brief description for several hundred different datasets from that
\xlinkn{list}{http://unidata.ucar.edu/packages/dods/home/data.shtml}.
\indc{dataset!list, \opendap}\indc{list!\opendap datasets}

\section{Web Interface}

This is a little bit sneaky.  Many sites that serve one \opendap dataset
serve several others as well.  The \opendap web interface (if it's enabled
by the site) allows you to check the directory structure for other
datasets.  For example, let's look at the \xlinkn{Reynolds data}
{http://www.cdc.noaa.gov/cgi-bin/nph-nc/Datasets/reynolds_sst/sst.mnmean.nc.html}
we saw in chapter~\ref{reynolds,chapter}:
\subj{The web interface allows browsing data directories.}

\begin{vcode}[.]{sib}
http://www.cdc.noaa.gov/cgi-bin/nph-nc/Datasets/reynolds_sst/sst.mnmean.nc.html
\end{vcode}

\indc{ifh}\indc{html interface!finding more data}
\indc{web interface!finding more data}
\indc{Data Access Form!using to find data}
If we use the same URL, but without the file at the end, we can browse
the directory of data:

\begin{vcode}[.]{sib}
http://www.cdc.noaa.gov/cgi-bin/nph-nc/Datasets/reynolds_sst/
\end{vcode}

The \opendap server checks to see whether the URL is a directory, and if
so, it generates a directory listing, like \texorhtml{in figure~\ref{reynolds,ifh-dir}.}{\xlinkn{this}{http://www.cdc.noaa.gov/cgi-bin/nph-nc/Datasets/reynolds_sst/}:}

\figureplace{Web Interface Index Listing}{h}{reynolds,ifh-dir}%
{ifh-dir.ps}{ifh-dir.gif}%
{http://www.cdc.noaa.gov/cgi-bin/nph-nc/Datasets/reynolds_sst/}

You can see from the directory listing that the monthly mean dataset
we've been looking at is accompanied by a weekly mean set, and a daily
set.  You can click on those datasets for more information about them,
and proceed to examine and use them just as we've done with the other
examples in chapter~\ref{reynolds,chapter}.

\note{This list is produced by an \opendap server.  It only really understands
\opendap data files.  If the directory you're looking at has other files
in it, clicking on them will probably produce an error.}

\section{File Servers}

Some datasets you'll find are actually lists of other datasets.  There
are a few of these \emph{file servers} in the \xlinkn{\opendap Dataset
  List}{http://unidata.ucar.edu/packages/dods/home/data.shtml} on the
\OPDhome .  A file server is itself an \opendap dataset, organized as a
Sequence, containing URLs with some other identifying data (often time).  You
can request the entire dataset, or subsample it just like any other
\opendap dataset.
\subj{A file server is a list of other datasets, but it's a dataset, too.}

There is a file server for GSO/URI's archive of AVHRR sea surface
temperature data:
\indc{query form!file server}\indc{form!query}
\indc{web interface!file server}\indc{file server}\indc{server!file}
\indc{Data Access Form!file server}
\indc{sampling!file server}\indc{Sequence!file server}
\indc{dataset!of datasets}

\begin{vcode}{sib}
http://maewest.gso.uri.edu/cgi-bin/nph-ff/catalog/avhrr.catalog
\end{vcode}

Look at this server's \xlinkn{DDS}{http://maewest.gso.uri.edu/cgi-bin/nph-ff/catalog/avhrr.catalog.dds}, and the \xlinkn{web interface}{http://maewest.gso.uri.edu/cgi-bin/nph-ff/catalog/avhrr.catalog.html}, and then try asking
for some data like \xlinkn{this}{http://maewest.gso.uri.edu/cgi-bin/nph-ff/catalog/avhrr.catalog.asc?DODS_URL\&year=2000\&month=1}:

\begin{vcode}[http://maewest.gso.uri.edu/cgi-bin/nph-ff/catalog/avhrr.catalog.asc?DODS_URL&year=2000&month=1]{sib}
  .../catalog/avhrr.catalog.asc?DODS_URL&year=2000&month=1
\end{vcode}

This produces a list of all the data URLs corresponding to
measurements taken in the month of January, 2000.

\section{Matlab GUI}

The \opendap Matlab GUI browser contains its own frequently updated list of
available datasets.  Using that software, you can select datasets with
\subj{The Matlab GUI has its own list of available data.}
a mouse from a large selection of the available URLs.  For more
information, please refer to the \OPDmgui\ manual.

\chapter{Further analysis}

This guide is about forming an \opendap URL.  After you have figured out
how to request the data, there are a variety of things you can do with
it.  (\opendap software mentioned here is available from the \OPDhome .)

\tbd{Add links to the following list.}

\begin{itemize}
\item Use a generic web client like \lit{geturl} (a standard part of
  the \opendap package), the free programs
  \xlinkn{\lit{wget}}{http://www.gnu.org/manual/wget-1.5.3/html_mono/wget.html}
  or \xlinkn{\lit{lynx}}{http://lynx.browser.org}, or even a browser
  like \lit{Netscape Navigator} or \lit{Internet Explorer} to download
  data into a local data file.  To be able to use the data further,
  you will probably have to download the ASCII version by using the
  \indc{geturl!\opendap utility web client}\indc{wget!web client}
  \indc{lynx!web browser}\indc{Netscape}\indc{Internet Explorer}
  \lit{.asc} suffix on the URL, as in the examples shown.
\subj{Use a generic web client or an \opendap client to get the data you've chosen.}
\item There are pre-packaged \opendap clients available that can download
  binary \opendap data from the web into a useful form.  As of \today ,
  command line clients (\lit{loaddods}) are available for the Matlab
  and IDL data analysis environments, with which you can download \opendap
  data directly into IDL or Matlab objects.  \indc{loaddods!Matlab or
    IDL client}\indc{Matlab!loaddods client}\indc{IDL!loaddods client}
\item The \xlink{\ind{Ferret}}{http://ferret.wrc.noaa.gov/Ferret} and \ind{GrADS}
  free data analysis packages both support \opendap.  You can use these
  for downloading \opendap data, and for examining it afterwards.  (There
  are limitations.  As of \today , Ferret can not read datasets served
  as Sequence data.)
\item The Matlab analysis package also supports an \opendap client attached
  to a graphical user interface.  You can use the GUI to create a
  constrained \opendap URL, and download the data directly into Matlab.
  The \OPDmgui\ contains more information about the Matlab GUI
  client.\indc{Matlab!GUI \opendap browser}
\item If you have a data analysis program or package that you like,
  you can look into the possibility of linking that package to the
  \opendap toolkit library, in effect making your program into a
  web-capable \opendap client. \indc{DODS!linking to your
    software}\indc{linking!\opendap to your programs} \ind{DODS!libraries}
  \indc{libraries!\opendap} exist to mimic the behavior of the
  \netcdf and \jgofs\ data access APIs.  If your program already uses
  one of these APIs, getting it to run with \opendap may be as simple as
  changing the libraries to which you link it.  The \OPDuser\ 
  describes how to do this, and the \OPDapi\ describes how you can
  use the \opendap toolkit directly to create a new application that
  doesn't use one of the established data access
  APIs.\indc{toolkit!\opendap}\indc{DODS!toolkit}
\indc{JGOFS!data access API}\indc{data access API!JGOFS}
\indc{netCDF!data access API}\indc{data access API!netCDF}

\end{itemize}

The use of these clients, like the ways in which you can analyze the
data you find, is beyond the scope of this (or any) book.  Enjoy.

\printindex

\end{document}



% $Log: quick.tex,v $
% Revision 1.6  2004/11/09 14:41:10  tomfool
% converted DODS->OPeNDAP
%
% Revision 1.5  2004/07/07 03:35:30  jimg
% Added rcs Info Date.
%
% Revision 1.4  2001/12/05 15:15:11  tom
% changed DODS home page dataset URL
%
% Revision 1.3  2001/06/29 15:54:15  tom
% added PDF file generation
%
% Revision 1.2  2000/10/30 20:11:46  tom
% indexed, incorporated first round of comments.
%
% Revision 1.1  2000/10/12 17:24:36  tom
% added quick.tex to repository
%
