
% Architecture for the DODS data delivery mechanism. 
%
% $Id$

\documentstyle[12pt,psfig,html,margins]{article}

\psfigurepath{reg-fig}
%
% These are html links which are used often enough in writing about DODS to
% merit an input file.
% jhrg. 4/17/94
%
% File rationalized and updated while writing the DODS User
% Guide. Also includes other useful abbreviations.
% tomfool 3/15/96
%
% $Id$
%
% HTML references to DODS documents
%
% For most of the DODS documentation there are three html references defined:
% 1) The upper case \newcommand produces the title of the paper, an
%    html link in the online documentation and a footnote in the paper
%    version.  
% 2) A capitalized version produces a capitalized description of the
%    paper and a link in the online version (but no footnote in the
%    paper version). 
% 3) A lower case version which produces a lower case description of
%    the paper and a html link, but no footnote.

% External references to documents:
%
% In order for external references to work (via latex2html) the labels used
% throughout the documents must be unique. The text labels are all prefixed
% with a document identifier (e.g., ddd) with a colon separater. The
% \externalrefs below (way below...) includes the perl files html.sty uses to
% make the cross refs.

\newcommand{\DODS}{\htmladdnormallink{Distributed Oceanographic Data System}
  {http://www.unidata.ucar.edu/packages/dods/}}

\newcommand{\Dods}{\htmladdnormallink{DODS}{http://www.unidata.ucar.edu/packages/dods/}}

\newcommand{\wrkshp}{\htmladdnormallink{Report on the First Workshop for the
    Distributed Oceanographic Data System, Proposed System Architectures}
  {http://www.unidata.ucar.edu/packages/dods/archive/reports/workshop1/section3.13.html}}

% not DODS URLs, but useful all the same...

\newcommand{\www}{\htmladdnormallink{World Wide Web} 
  {http://www.w3.org/hypertext/WWW/TheProject.html}}
\newcommand{\url}{\htmladdnormallink{Uniform Resource Locator}
  {http://www.w3.org/hypertext/WWW/Addressing/URL/url-spec.html}}

\newcommand{\uri}{\htmladdnormallinkfoot{Uniform Resource Identifiers}
  {http://www.w3.org/hypertext/WWW/Addressing/URL/uri-spec.html}}

\newcommand{\urn}{\htmladdnormallinkfoot{Uniform Resource Name}
  {http://www.acl.lanl.gov/URI/archive/uri-archive.messages/1143.html}}

\newcommand{\HTTPD}{\htmladdnormallinkfoot{HTTPD}
  {http://hoohoo.ncsa.uiuc.edu/docs/Overview.html}}

\newcommand{\HTTP}{\htmladdnormallinkfoot{HyperText Transfer Protocol}
  {http://www.w3.org/hypertext/WWW/Protocols/HTTP/HTTP2.html}}

%\newcommand{\HTML}{\htmladdnormallinkfoot{HyperText Markup Language}
%  {http://www.w3.org/hypertext/WWW/MarkUp/MarkUp.html}}

% CERN used to be `Conseil Europeen pour la Recherche Nucleaire'
\newcommand{\CERN}
  {\htmladdnormallink{European Laboratory for Particle Physics}
  {http://www.cern.ch/}}

\newcommand{\NCSA}
  {\htmladdnormallink{National Center for Supercomputing Applications}
  {http://www.ncsa.uiuc.edu/}}

\newcommand{\WWWC}{\htmladdnormallink{World Widw Web Consortium}
  {http://www.w3.org/}}

\newcommand{\CGI}{\htmladdnormallinkfoot{Common Gateway Interface}
  {http://hoohoo.ncsa.uiuc.edu/cgi/overview.html}}

\newcommand{\MIME}{\htmladdnormallink{Multipurpose Internet Mail Extensions}
  {http://www.cis.ohio-state.edu/htbin/rfc/rfc1590.html}}

\newcommand{\netcdf}{\htmladdnormallink{NetCDF}
  {http://www.unidata.ucar.edu/packages/netcdf/guide.txn_toc.html}}

\newcommand{\JGOFS}{\htmladdnormallink{Joint Geophysical Ocean Flux Study}
  {http://www1.whoi.edu/jgofs.html}}

\newcommand{\jgofs}{\htmladdnormallink{JGOFS}
  {http://www1.whoi.edu/jgofs.html}}

\newcommand{\hdf}{\htmladdnormallink{HDF}
  {http://www.ncsa.uiuc.edu/SDG/Software/HDF/HDFIntro.html}}

\newcommand{\Cpp}{{\rm {\small C}\raise.5ex\hbox{\footnotesize ++}}}

% Commands

\newcommand{\pslink}[1]{\small
\begin{quote}
  A \htmladdnormallink{postscript}{#1} version of this document is
  available.  You may also use \htmladdnormallink{anonymous
  ftp}{ftp://ftp.unidata.ucar.edu/pub/dods/ps-docs/} to access postscript files
  of all of the DODS documentation.
\end{quote}
\normalsize
}

\newcommand{\declaration}[1]{\small
{\tt {#1}}
\normalsize
}

% All the links from here down are for the white papers. I'm not sure that any
% of these are still valid given the reorganization of the web site. 5/12/98
% jhrg.

\newcommand{\DDA}{\htmladdnormallinkfoot{DODS---Data Delivery Architecture}
  {http://www.unidata.ucar.edu/packages/dods/archive/design/data-delivery-arch/data-delivery-arch.html}}
\newcommand{\Dda}{\htmladdnormallink{Data Delivery Architecture}
  {http://www.unidata.ucar.edu/packages/dods/archive/design/data-delivery-arch/data-delivery-arch.html}}
\newcommand{\dda}{\htmladdnormallink{data delivery architecture}
  {http://www.unidata.ucar.edu/packages/dods/archive/design/data-delivery-arch/data-delivery-arch.html}}

\newcommand{\DDD}{\htmladdnormallinkfoot{DODS---Data Delivery Design}
  {http://www.unidata.ucar.edu/packages/dods/archive/design/data-delivery-design/data-delivery-design.html}}
\newcommand{\Ddd}{\htmladdnormallink{Data Delivery Design}
  {http://www.unidata.ucar.edu/packages/dods/archive/design/data-delivery-design/data-delivery-design.html}}
\newcommand{\ddd}{\htmladdnormallink{data delivery design}
  {http://www.unidata.ucar.edu/packages/dods/archive/design/data-delivery-design/data-delivery-design.html}}

\newcommand{\URL}{\htmladdnormallinkfoot{DODS---Uniform Resource Locator}
  {http://www.unidata.ucar.edu/packages/dods/archive/design/urls/urls.html}}
\newcommand{\Url}{\htmladdnormallink{Uniform Resource Locator}
  {http://www.unidata.ucar.edu/packages/dods/archive/design/urls/urls.html}}
\newcommand{\dodsurl}{\htmladdnormallink{uniform resource locator}
  {http://www.unidata.ucar.edu/packages/dods/archive/design/urls/urls.html}}

\newcommand{\DAP}{\htmladdnormallinkfoot{DODS---Data Access Protocol}
  {http://www.unidata.ucar.edu/packages/dods/archive/design/api/api.html}}
\newcommand{\Dap}{\htmladdnormallink{Data Access Protocol}
  {http://www.unidata.ucar.edu/packages/dods/archive/design/api/api.html}}
\newcommand{\dap}{\htmladdnormallink{data access protocol}
  {http://www.unidata.ucar.edu/packages/dods/archive/design/api/api.html}}

\newcommand{\SOFT}
        {\htmladdnormallinkfoot{DODS---Software Development Environment}
  {http://www.unidata.ucar.edu/packages/dods/archive/managment/software/software.html}}
\newcommand{\Soft}{\htmladdnormallink{Software Development Environment}
  {http://www.unidata.ucar.edu/packages/dods/archive/managment/software/software.html}}
\newcommand{\soft}{\htmladdnormallinkfoot{software development environment}
  {http://www.unidata.ucar.edu/packages/dods/archive/managment/software/software.html}}

% I used to call the DAP the API, then for a few days I called it the
% DTP. These def's were used then. Rather than find everywhere they are used
% (not that hard, but...) I just hacked the macros. NB: the source file for
% the DAP document is still called `api.tex'

\newcommand{\API}{\htmladdnormallinkfoot{DODS---Data Access Protocol}
  {http://www.unidata.ucar.edu/packages/dods/archive/design/api/api.html}}
\newcommand{\api}{\htmladdnormallink{data access protocol}
  {http://www.unidata.ucar.edu/packages/dods/archive/design/api/api.html}}

\newcommand{\DTP}{\htmladdnormallinkfoot{DODS---Data Access Protocol}
  {http://www.unidata.ucar.edu/packages/dods/archive/design/api/api.html}}
\newcommand{\dtp}{\htmladdnormallink{data access protocol}
  {http://www.unidata.ucar.edu/packages/dods/archive/design/api/api.html}}

\newcommand{\TOOLKIT}{\htmladdnormallinkfoot{DODS---Client and Server Toolkit}
  {http://www.unidata.ucar.edu/packages/dods/archive/implementation/toolkits/toolkits.html}}
\newcommand{\Toolkit}{\htmladdnormallink{Client and Server Toolkit}
  {http://www.unidata.ucar.edu/packages/dods/archive/implementation/toolkits/toolkits.html}}
\newcommand{\toolkit}{\htmladdnormallink{client and server toolkit}
  {http://www.unidata.ucar.edu/packages/dods/archive/implementation/toolkits/toolkits.html}}

\newcommand{\TKR}{\htmladdnormallinkfoot{DODS---DAP Toolkit Reference}
  {http://www.unidata.ucar.edu/packages/dods/archive/implementation/toolkits/toolkits.html}}
\newcommand{\Tkr}{\htmladdnormallink{DAP Toolkit Reference}
  {http://www.unidata.ucar.edu/packages/dods/archive/implementation/toolkits/toolkits.html}}
\newcommand{\tkr}{\htmladdnormallink{DAP toolkit reference}
  {http://www.unidata.ucar.edu/packages/dods/archive/implementation/toolkits/toolkits.html}}

% I know that these are wrong. The papers have been moved to the gso web
% site, but I'm not sure exactly where. 5/12/98 jhrg

\newcommand{\DM}{\htmladdnormallinkfoot{DODS---Data Model}
  {http://lake.mit.edu/dods-dir/dodsdm2.html}}
\newcommand{\Dm}{\htmladdnormallink{Data Model}
  {http://lake.mit.edu/dods-dir/dodsdm2.html}}
\newcommand{\dm}{\htmladdnormallink{data model}
  {http://lake.mit.edu/dods-dir/dodsdm2.html}}

\newcommand{\DD}{\htmladdnormallinkfoot{DODS---Data Delivery}
  {http://lake.mit.edu/dods-dir/dods-dd.html}}

% external refs for DODS documents

\externallabels{http://www.unidata.ucar.edu/packages/dods/design/api}
  {/home/www/packages/dods/archive/design/api/labels.pl}
\externallabels{http://www.unidata.ucar.edu/packages/dods/design/data-delivery-arch}
  {/home/www/packages/dods/archive/design/data-delivery-arch/labels.pl}
\externallabels{http://www.unidata.ucar.edu/packages/dods/design/data-delivery-design}
  {/home/www/packages/dods/archive/design/data-delivery-design/labels.pl}
\externallabels{http://www.unidata.ucar.edu/packages/dods/implementation/toolkits}
  {/home/www/packages/dods/archive/implementation/toolkits/labels.pl}

%\renewcommand{\externalref}[2]{#2}

%%% Local Variables: 
%%% mode: latex
%%% TeX-master: t
%%% End: 


\begin{document}

\title{DODS---Dataset Registration Service}
\author{George Milkowski}
\date{\today}

\maketitle

\begin{abstract}

  This paper describes the DODS Dataset Registration Service. 

\end{abstract}


% Biolerplate text warning that anything can (and almost certainly will)
% change. 
%
% jhrg 3/25/94
%
% $Id$

\small

\centerline{{\bf Note}} 

\begin{quotation}

\begin{htmlonly}
  This is a working document made available to solicit comments. To receive
  e-mail notice of new or significantly changed documents, send e-mail
  to \htmladdnormallink{majordomo@unidata.ucar.edu}
  {mailto:majordomo@unidata.ucar.edu} with 'subscribe dods' in the
  body of the message.
\end{htmlonly}

\begin{latexonly}
  This is a working document made available to solicit comments. To receive
  e-mail notice of new or significantly changed documents, send e-mail
  to majordomo@unidata.ucar.edu with 'subscribe dods' in the body of
  the message. 

  Hypertext references in the hardcopy version of this document are
  represented using footnotes that contain a Universal Resource Locator (URL)
  to the referenced document.  To read the online documentation, open the URL
  `http://www.unidata.ucar.edu/' with a World Wide Web browser.
\end{latexonly}

\end{quotation}

\normalsize

%
% Biolerplate text expaining that this document is intended for programmers.
%
% jhrg 4/17/94
%
% $Id$

\small

\begin{quotation}
  This document is written for people interested in implementing software
  that will be part of, or work in parallel with, the Distributed
  Oceanographic Data System. 
\end{quotation}

\normalsize

\begin{htmlonly}
\pslink{file://dods.gso.uri.edu/pub/DODS/dataset-registration.ps}
\end{htmlonly}

\clearpage

\tableofcontents

\clearpage

\section{Introduction}

This paper describes the Dataset Registration Service of the Distributed
Oceanographic Data Systems, DODS.  The registration service allows
dataset providers to document and register their data with the Global Change
Master Directory, GCMD, one component of the DODS Locator Service.
The registration service includes a user interface that aids dataset
providers in the creation of a Directory Interchange Format (DIF) data
structure, the input format used to load directory level information into the
GCMD.  

\section{Overview}

Because DODS is a system which will support many distributed data resources
it is necessary to provide a data location service for users which allows
them to quickly determine what datasets are available from DODS servers.  The
DODS Data Location Service will use the NASA Global Change Master Directory
to provide directory level information on DODS datasets as well as the URL of
their DODS servers.  The GCMD is an on-line directory of Earth science
datasets that supports multi-parameter searching on its relation database
management system.  Data providers who have installed DODS servers and intend
to allow others access to their on-line datasets via DODS will need to
register their datasets with the GCMD.  The DODS Registration Service is
intended to aid data providers in providing the dataset and server
information required to create a reference in the GCMD.

\subsection{DIF Data Structure}

The standard input data structure for the GCMD is called the Directory
Interchange Format or DIF.  A DIF is a complex data structure that contains
information used to support both relational and text based searching within
the GCMD.  The types of information contained in a DIF include; dataset
attributes, archive and data provider descriptions as well as general summary
information.  The documentation on the DIF standard is available at {\tt
  http://gcmd.gsfc.nasa.gov/docs.html}.

The DODS registration service generates a dataset DIF based on information
supplied interactively by the data provider and information extracted
automatically from DODS servers of the datasets being registered.  The data
provider reviews, modifies and submits the registration service DIF
controlling when the DIF is registered and its content.  


\subsection{Registration Service Component Architecture}

The Registration service components and their architecture are depicted in
Figure 1.  The components are the Data Provider Input form, a DIF generation
cgi, a DIF review and modification form and a DIF uploader cgi that submits
the completed DIF to the GCMD.  Each is describe below.


\subsubsection{Data Provider Input Form}

When a data provider wants to register a dataset he or she 
pulls up the Data Provider Input form, Figure 2.  This form takes as its
input the name of the dataset being registered, the address of the DODS
server that accesses the dataset, the names and location of the DODS cgis
that describe that dataset and the email address of the data provider.  In
addition the data provider can submit the name of a file containing
descriptive information on the dataset.  Once the information has been filled
in the data provider hits the ``Create DIF'' button to post the form to the
DIF generation cgi.  The following input fields and the information required
are in the Data Provider Input form:

\begin{description}

\item{DODS Server Address:}  This field requires the URL address of the
  machine where the DODS server is installed (e.g.; http://dods.gso.uri.edu ).

\item{DDS cgi name and location:} This is the specification of the dds cgi
  that access the dataset being registered.  The location is relative to the
  httpd ScriptAlias directory defined at the time the httpd server is
  configured (e.g., ScriptAlias -> /cgi-bin/ =
  /usr/local/etc/httpd/cgi-bin/).  An example of a DDS specification,
  /cgi-bin/DODS-cgi/ctd$\_$DDS.
 
\item{DAS cgi name and location:} Equivalent to the DDS only for the DAS cgi.

\item{Dataset name} The name of the dataset.  This name can be a discrete
  filename or a reference name.  The actual value will depend on the cgi
  filters that access the dataset.  For example, in the JGOFS system dataset
  names actually reference JGOFS objects.  The DODS cgis use a JGOFS object
  lookup table to reference the actual JGOFS data file and its access
  methods.  For a NetCDF implementation the dataset name must include the
  directory location of the data file being accessed.  The directory location
  can be either relative to the DocumentRoot of the httpd server (e.g.,
  DocumentRoot/data/DODS/fnoc1.nc) or an alias that is defined when the httpd
  server is configured (e.g. /icons/ -> /usr/local/spool/icons/).

\item{Contact EMAIL} The email address of the person registering the dataset
  is requested.

\item{Other Dataset Description Information}  The name of a file that
  contains summary information.  This function is not implemented at present.
  In the future it will allow the data provider to specify a text file to
  also be used to help generate the DIF valids for the dataset. 
\end{description}

\subsubsection{DIF Generation cgi}

This cgi reads the input data from the Data Provider Input form.  Using the
DODS server URL, the cgi names and the dataset name it next accesses the cgis
of the dataset being registered and reads in its DDS and DAS.  The DDS and
DAS structures are parsed into words and each word is submitted to a DIF
thesaurus which matches input words with words found in the complete set of
DIF valids.  The DIF thesaurus outputs a set of DIF valids based on the words
in the DAS and DDS that are separated DIF attribute categories, Sensor,
Source, Parameter, Discipline and Location. Figure 3 is a data flow diagram
for the DIF generator cgi.  The output from the DIF thesaurus
along with the information from the Data Provider Input form are then used
to create the DIF Review and Modification form.

\subsubsection{DIF Review and Modification Form}

This html form (Figures 4a \& 4b) allows the data provider an opportunity to
review and modify the DIF valids that describe the dataset being registered.
The most critical information is that in the DIF Valids Generated from
Dataset Descriptor Files section of this form.  The data provider must select
all appropriate DIF valids from the lists provided under each of the DIF
categories in the form (Sensor, Source, Parameter, Discipline and Location).
The form permits the specification of multiple values for any of the
categories by simply clicking on the value.  In cases where no DIF valids
were generated from the DIF thesaurus for a particular DIF category, a
default list of all valid DIF values for that category is provided.

Once the data provider has reviewed the form he or she may then submit the
completed DIF to the GCMD for uploading by hit the Submit DIF button. 

\subsubsection{DIF Upload cgi}

This cgi accepts input from the DIF Review and Modification Form and
constructs a valid GCMD DIF for uploading.   The DIF is then evaluated for
completeness and either accepted or rejected for uploading into the GCMD.  In
either case an acknowledgement and copy of the DIF submitted along with
whether the DIF was accepted or rejected is returned to the data provider.

\end{document}



