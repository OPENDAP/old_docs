% Appendix to the Matlab GUI User Guide
%
% $Id$
%
% $Log: appb.tex,v $
% Revision 1.5  1999/05/27 15:58:39  tom
% expunged remaining references to "dods.gso.uri.edu"
%
% Revision 1.4  1999/05/25 20:47:46  tom
% modifications for DODS release 3.0
%
% Revision 1.3  1999/01/20 15:18:38  tom
% updated, and changed for dods-book.cls and hyperlatex
%
% Revision 1.2  1998/12/07 15:41:15  tom
% updated for DODS v2.19 and GUIv0.7
%
% Revision 1.1  1998/02/12 18:10:41  tom
% added to CVS archive
%
%

\chapter{Troubleshooting}
\label{gui,trouble}

%\renewcommand{\paragraph}{\@startsection{paragraph}{4}{0cm}%
%                                    {1.25cm}%
%                                    {-1cm}%
%                                    {\normalfont\normalsize\bfseries}}


If you have a problem using the \GUI, check this list
to see if there is a solution to your problem here. If not, send a
message describing the problem to \DODSsupport . If you
can, please include a copy of the any error messages printed by the
software along with information so that we can repeat the problem. The
latter helps immeasurably.

An updated version of this list is provided with the DODS source code,
in \DODSroot\lit{/src/matlab-GUI/README}.

\section{\GUI\ problems}

\problem The \pdmenu{Variables} or \pdmenu{Datasets} menus are not
persistent; they disappear as soon as I ask for them.

\fix The windows have probably not disappeared. They have been covered
up by the browser window. While not a useful feature, this is
apparently a result of some interaction between Matlab and the window
manager software you use that we are unable to address. Just move the
browser window aside, and you will see the missing menus.

\problem  I select a dataset, and then a variable, and then my dataset
gets unselected.

\fix  The \GUI\ is set up so that a user can select a variable of
interest first, or a dataset of interest.  A byproduct of that
convenience is that the behavior of the \pdmenu{Variables} and
\pdmenu{Datasets} menus is somewhat confusing.  When you select a
dataset, all the variables not in that dataset are deleted from the
\pdmenu{Variables} menu.  Then, when you select a variable, all the
datasets that contain that variable are displayed in the
\pdmenu{Datasets} menu.  Sometimes this may require you to select the
dataset again.  If your selected dataset gets ``un-selected'' it does
not contain the variable of interest, or is not valid in the time and
geographic ranges you've picked.

\problem  I can't select vector data, such as vector winds.

\fix  The \GUI\ presents vector winds by individual component.  This
means that you must select \lit{U\_Wind} and \lit{V\_Wind}.


if a vector is stored at the server as X and Y components,
it may appear in the \GUI\ in a different manner than a dataset stored
as speed and direction pairs.

\problem  The colors I see in this browser window are wierd, and they
change terribly when I move my mouse around.

\fix What you are seeing is a collision between Matlab and some other
program hogging space on the X terminal color table. Netscape
Navigator is often the culprit. Matlab uses a polite color allocation
scheme that doesn't mess up other applications, but the downside of
this is that its graphics can look pretty awful when another
application isn't as ``polite.'' The solution is to exit Matlab, close
the other application, and start up Matlab again.

A more permanent solution is to ask your system manager to increase
the color depth of your X display.  A color depth (the number of bits
per pixel) of 16 is usually sufficient to prevent this problem from
occurring. 

\problem The browser complains about about not finding `loaddods'.

\fix Make sure that the directory containing the \lit{loaddods}
program is on your \lit{\$MATLABPATH} See \sectionref{install,install}. 
Also, note that this program actually consists of two files:
\lit{loaddods.m} and \lit{loaddods.mex***}. (The \lit{***} indicates an
architecture-dependent extension to the name of a Matlab MEX-file. For
example, on a DEC Alpha, this file is called
\lit{loaddods.mexaxp}). \emph{Both} of these files must be present for
the browser to function.

If this does not resolve the problem, please contact \xlink{DODS user
  support: {\small \DODSsupport}}{mailto:\DODSsupport}.  There may be
a problem with the distributed binary.

\problem The browser complains that is cannot find
\lit{writeval} or \lit{geturl}.

\fix Make sure that your \lit{\$PATH} environment variable
points towards the correct directory (and for Bourne Shell users, that
it is exported). See \sectionref{install,install}.

\problem The dataset I want to use has disappeared from
the dataset list.

\fix This probably means you have selected a time, depth or 
geographic range that does not overlap with the defined boundaries of
the dataset you want to use. (See \pagexref{gui,ref,Datasets}) To
be sure you are not eliminating datasets you might like to see, push
the \but{Clear Selections} button at the bottom of the browse window.
Now select the desired dataset, and the depth and geographic ranges
will be set to those of the dataset.  Zoom in on the time axis to
select an appropriate time period within the dataset range, or click
\but{View Text} and type in an appropriate time range (use decimal
years).

To view the ranges for a variety of datasets,
simply click \but{Reset List} at the bottom of the dataset menu.  Each
dataset may then be selected in turn, and the ranges for that set will
appear on the browse window.  Once you have found a dataset you would
actually like to use, click on the \but{Clear Selections} button (to clear
selected ranges) and then re-select the desired dataset.  Your depth
and geographic selection ranges will automatically be set to those of
the dataset.  Zoom in on the time axis to select an appropriate time
period within the dataset range, or click \but{View Text} and type in an
appropriate time range (use decimal years).

If you still can't make your dataset appear, try clicking the
\but{Update Dataset List} button in the \pdmenu{Options} menu, and
issuing whatever command will update the dataset having the
problem. 

\problem  I've selected a dataset.  I click on \but{Get
Data} or \but{Get Details} and nothing happens.

\fix  Look for messages in the main workspace.  Most
likely, one or more of your selection ranges is not set, or possibly
they are all set but are not within the valid ranges for that dataset.
In the former case, the browser will send a message prompting you to
set all the ranges.  In the latter case, it will send a message that
the dataset is out of range, and the dataset selection will be
removed. 

For very large datasets, it is possible to select so much data that
the server cannot process it all before the request times out. If you
suspect that your request is for a colossal amount of data, set a
smaller time or space range and try again. This is a limitation of the
\lit{httpd} servers installed at a particular location, and is not
under the control of the DODS group.

\problem  I've selected a dataset.    I click on \but{Get
Data} or \but{Get Details} and something happens, but it's an error
like this:

\begin{vcode}{ib}
??? Undefined function or variable lon.

Error in ==> /usr/users/tomfool/matlab/DATASETS/getrectangular.m
On line 494  ==>          TempLongitude = eval(LongitudeName);
...
\end{vcode}

\fix  This is a bug that causes the \GUI\ to fail gracelessly when it 
encounters a network error such as the server being down, or a
timeout. A way to confirm that the network is the problem is to type 

\begin{vcode}{ib}
>> loaddods{http://whatever.edu/cgi/nph-nc/...}
\end{vcode}

at the Matlab prompt, substituting a valid URL.  You can also cut
the URL you used and paste it into the URL window of a standard web
browser like Netscape Navigator. Append ``.dds'' to the URL, and
submit it:

\begin{vcode}{ib}
http://saci.gso.uri.edu/cgi-bin/nph-nc/weekly.nc.dds
\end{vcode}

If the network is working, you will receive a terse description of the
dataset in the browser window. If it's not working, you will get an
error. 

The fix is just to wait until the error is repaired, or the server
goes back on line.

\problem Plotting the dataset results in a box with some 
vertical lines running up through the middle of the range box.

\fix  For a full-global dataset, this is because the 
dataset range given in the 
metadata is less than the full 360 degrees.  For example, a range of
[0.5 359.5].  When this is displayed in longitude coordinates of [-180
180], vertical lines at -0.5 and 0.5 will result---the browser has no
way to determine if this gap is `real' (a gap in the dataset $>$ dx), or
just a side effect of wraparound.  Data providers should list
truly global datasets as having a 360 degree longitude range, and 
write the supplementary software to handle such requests.

\problem{I can't find the data threshold menu.}

\fix{Try making the browser window larger.  If the window is too
  small, the threshold menu window (which is not a true window)
  sometimes gets cut off.\indc{threshold window!missing}}

\section{Matlab problems}

The following problems are due to known Matlab bugs as of \today .

\subsection{Bugs in Matlab Version 4}

\problem An error when starting the browser that looks like this:

\begin{vcode}{ib}
???          OKCallback
             |
 Undefined function or improper matrix reference.
\end{vcode}

\fix It appears some distributions of \mfour\ do not 
contain the script \lit{inputdlg.m}.  You are missing that function.
We have included it in this distribution, as \lit{inputdlg.tpl}
(renamed to avoid conflict with systems that \emph{do} have the
script).  Simply copy the provided script to the correct name (\lit{mv
inputdlg.tpl inputdlg.m}). You can leave it right in the browser directory. 
Once the file is in place, quit Matlab, and start again.

\problem Zooming in on a displayed image causes it to disappear.

\fix This is a fundamental bug in the \lit{image} command 
of \mfour.  The Mathworks has no plans to patch it now that \mfive\ has
been released.  The only fixes are: 

\begin{enumerate}

\item 
Plot each dataset separately in its own window, and then set \lit{zoom
on} in these windows.  In this case, you may encounter the following
instability in \mfour:

\begin{vcode}{ib}
Image color indexing error in InterpolateImageWarped
Image color indexing error in InterpolateImageWarped
Image color indexing error in InterpolateImageWarped
Image color indexing error in InterpolateImageWarped
\end{vcode}

repeatedly scrolling up the screen.  Hit Control-C to stop this.

\item
You can use the \lit{pcolor} command instead of \lit{image}. 

\item
Upgrade to \mfive.
\end{enumerate}

\subsection{Bugs in Matlab Version 5}

\problem Repeated ``error'' messages of the nature:

\begin{vcode}{ib}
  Warning: Future versions will return empty 
       for empty == scalar comparisons
  In dispmenu.m at line 23
  In dispmenu.m at line 458
  In browse.m at line 997
  In browse.m at line 975
\end{vcode}

or

\begin{vcode}{ib}
  Warning: One or more output arguments not assigned 
       during call to 'browse'.
\end{vcode}

\fix There are no real problems here. The MathWorks made several
changes in suggested Matlab syntax and use between versions 4 and
5. The \GUI\ runs in both versions, so occasionally Matlab version 5
helpfully finds future potential fault with a valid version 4
expression.  If you find the messages less than enlightening, you
could type \lit{warning off} at the Matlab prompt.

\problem The slider bar on the dataset and variables menus
(only visible when there are more than 20 items on the list)
``jumps,'' not scrolling the list smoothly.

\fix This is a bug in the Unix/Linux release of \mfive.
The MathWorks has promised a patch.  Contact
\lit{tech-support@mathworks.com} or (508) 653-1415 to find out if the 
patch is available yet.

\problem Zooming in causes vertical lines (for example, 
the vertical parts of dataset range boxes or the selection range
boxes) to disappear.

\fix  This is a bug in the Linux release of \mfive.  
The Mathworks has been notified.  Contact
\lit{tech-support@mathworks.com} or (508) 653-1415 to find 
out if a patch is available yet.

\problem Zooming in on an image causes the image pixels 
to be located in the wrong place or to appear larger than actual
(bleeding effect).

\fix This is an unavoidable side-effect of the \lit{image}
command. Use \lit{pcolor} instead.


%%% Local Variables: 
%%% mode: latex
%%% TeX-master: t
%%% End: 
