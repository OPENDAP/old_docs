% Chapter to the Matlab GUI User Guide
%
% $Id$
%
%

\chapter{Adding Your Own Datasets to the \GUI}
\label{gui,adding}


There are a few reasons you may need to add a dataset to the \GUI.  

\begin{itemize}
\item You may be interested in an \opendap data source that hasn't
  been put into the \GUI\ already, 
\item You may want to add to or modify some of the metadata supplied
  for one of the \GUI\ datasets, or
\item You may have some of your own data you want to make available
  via \opendap.
\end{itemize}


It is possible to add datasets to the ones that come with the \GUI,
and to edit or add to the metadata for the datasets that do come with
it.  This will require some minor programming in Matlab, but much of
the time it's fairly simple.

The \GUI\ is designed for the retrieval and display of data served by
an \opendap server. This means that in order for the \GUI\ to read
your data, there must be an \opendap server, on a machine connected to
yours \subj{Start with getting your data served by an OPeNDAP
  server.}  somehow, with access to that data.  The machine serving
the data could be connected to yours on a local network or by the
internet, or it could even be on your own computer.  Either way, the
data must come from an \opendap server.

Making your data available by an \opendap server is a fairly
straightforward affair for datasets stored in one of the several
supported formats.  The server itself is a typical web server
(\lit{httpd}), adapted with a set of \opendap CGI programs. To set up a
server for your data, please refer to \OPDuser .  You can find it and
all the other \opendap documentation at \OPDhome.

Once a \opendap server is set up to serve your data, the dataset may be
added to the \GUI\ by providing the following pieces of information.
Many datasets can use minimally modified versions of one or more of
these pieces from existing datasets. Each of these are detailed in the
sections that follow.



\begin{itemize}
  
\item An ``archive'' M-file, to describe the data in the dataset.

\item A \class{Get} function that accepts a data range and retrieves
  data from the remote dataset. This is also called the \lit{getxxx}
  function. Most often it's possible to use (or adapt) one of the
  existing \lit{getxxx} functions.

\end{itemize}


\section{The Archive File}
\label{gui,adding,archive.m}

The archive M-file is a Matlab program that sets a number of
variables describing various aspects of some dataset. In order for the
browser to display the geographic and temporal bounds of a dataset,
they must be specified in this file.

Many archive M-files exist in the DATASETS directory. If your dataset
resembles (in data structure) one of the datasets already in the \GUI\ 
dataset collection, it may be \subj{The archive M-file describes a
  dataset to the GUI.}  simplest to copy the corresponding archive
M-file and modify it (and its corresponding \lit{getxxx} function) to
suit the new dataset. For example, the Reynolds Weekly Sea Surface
Temperature data are stored in a single file, containing a global
one-degree grid of temperature for each week covered. If your data are
stored as a similar grid, it will likely be easier to copy the
Reynolds M-files and adapt them than to write new ones from scratch.

\subsection{An example}

The following archive M-file describes the Levitus ocean atlas
dataset.  (You can find it in the DATASETS directory, called
\lit{woa94mon.m}.)  Many of the M-files follow the same basic
structure.  Most of the variables set here are necessary for
the operation of the browser. A few of them, however, are set only for
the convenience of the \lit{getxxx} function (called \lit{getrectg}
here). If you have to write a new \lit{getxxx} function, you may not
need them.

\begin{vcode}{xib}
%  World Ocean Atlas Monthly Climatology

% FUNCTIONS -- REQUIRED
GetFunctionName = 'getrectg';

% VARIABLES -- OPTIONAL
Server = ['http://ferret.wrc.noaa.gov/cgi-bin/nph-nc',...
   '/data/Ocean_atlas_monthly.nc'];
axes_order = [4 3 2 1];
% force time axis to correct day [Y M D H M S]
Time_Offset = [0 0 -1 0 0 0];

% VARIABLES -- REQUIRED
LonRange = [20.000 380.000];
LatRange = [-90.000 90.000];
TimeRange = [1800 str2num(datestr(date,10))];
DepthRange = [0.0 1000.0];
DepthUnits = 'Meters';
Resolution = 111.0;

DataName = 'Water Column - World Ocean Atlas 1994 Monthly - PMEL';
SelectableVariables = str2mat('Sea_Temp', 'Salinity');
DodsName = str2mat('TEMP','SALT');

Acknowledge = sprintf('%s\n%s\n%s\n%s\n%s', ...
Comments1 = sprintf('%s\n', ...
\end{vcode}

The following variables are for use by the browser.

\begin{description}
\item[\lit{GetFunctionName}] This is the name of the \lit{getxxx}
  function that will return both the catalog information as well as
  the actual data.  See \sectionref{gui,adding,get}.  The
  \lit{getrectg} function is a common one used to return gridded
  multi-dimensional data.  The \lit{getxxx} function must be available
  in your \lit{MATLABPATH}.
  
\item[\lit{LonRange}, \lit{LatRange}, \lit{TimeRange},
  \lit{DepthRange}] These four two-element vectors describe the
  spatial bounds of the dataset's data. This dataset is stored on a
  grid; the first elements of these range vectors indicate the
  positions of the first row (column) in the dataset. The latitude and
  longitude are given in decimal degrees.  Times are given in decimal
  years. In the example here, the data is a climatology, an average
  over many years This means that the entire time range isn't
  particularly meaningful (which is why it's set to the maximum),
  though being able to select by time is important, so you can select
  the appropriate month.

\item[\lit{DepthUnits}] This is a character string naming the units in
  which the data stores depth information.  The \GUI\ has no knowledge
  of units, so it's really just an axis label.
  
\item[\lit{Resolution}] This is a value, in kilometers, of the
  distance between adjacent horizontal values in the grid.  The \GUI\ 
  is only set up for grids with the same resolution in the X and Y
  dimensions.  For georeferenced grids, the resolution refers to the
  distance between adjacent X values at the equator.
  
\item[\lit{DataName}] This is the name of the dataset, to be used in
  the bookmark list. 
  
\item[\lit{DataRange}] An array of data pairs, giving an upper and
  lower bound to the data.  The entries in the array correspond to the
  entries in the \lit{DataNull} array.  This matrix is used by the
  browser plot functions to create a consistent color scale for the
  imaging of multiple returning data slices within one dataset.  It is
  \emph{only} used for \lit{image} and \lit{pcolor} plots.  An entry
  of \lit{[NaN NaN]} indicates the total range of that variable is
  not known and the browser should scale each image using the local
  minimum and maximum.  The temptation to write in non-NaN scale
  entries for longitude should be avoided, since the browser
  converts longitude to match the current window view upon return.
  Not required.

\item[\lit{Comments}] A detailed description of the dataset.  To avoid
  Matlab limitations on the length of a command, data descriptions may
  be broken up into several input lines, which are then concatenated:

\begin{vcode}{xib}
Comments1 = sprintf('%s\n', ...
    '        Levitus (1982) Climatological Atlas of the World Ocean', ...
    '    ', ...
    'Annual climatologies of salinity and temperature on 1x1 degree ');

Comments2 = sprintf('%s\n', ...
    '    ', ...
    'Levitus, S., Climatological Atlas of the World, NOAA/ERL GFDL', ...
    'Professional Paper 13,  Princeton, N.J.,  (NTIS PB83-184093),', ...
    '1982.');
Comments3 = sprintf('%s\n', ...
    '    Time_Period_of_Content = ', ...
    '         Beginning_Date:1935 ', ...
    '         Ending_Date:1978 ');

Comments = [Comments1 Comments2 Comments3];
    
\end{vcode}


\item[\lit{Acknowledge}] A text string indicating how uses of this
  data should acknowledge its originator.

\item[\lit{Data\_Use\_Policy}] A text string indicating what
  limitations are set on the use of this data.

\end{description}

\subsection{Getrectg function variables}

A few variables aren't used by the \GUI, but are used by
\lit{getrectg}, the \lit{getxxx} function used here.  They are
documented here because the \lit{getrectg} function is flexible enough
to be used for many different gridded datasets.  See
section~\ref{gui,adding,get} for more information about the
\lit{getxxx} API.

\begin{description}
  
\item[\lit{Server}] The server from which the data in question will
  come.
  
\item[\lit{axes\_order}] The dataset may store its variables in a
  different order than the \GUI\ expects them.  You can use this
  vector to rearrange the variables before \lit{getrectg} returns
  them.
    
\item[\lit{Time\_Offset}] Data providers often disagree whether a time
  value marks the beginning of the interval, or the middle, and
  whether the first unit is number one or number zero.  You can use
  this six-element vector to provide a constant with which to adjust
  the time values from the dataset.  The elements are adjustments to
  the year, month, day, hour, minute, and second, in that order.  For
  example, the offset vector in the example instructs \lit{getrectg}
  to subtract one day from the day values in the dataset before
  returning them to the \GUI.

\item[\lit{LongitudeName}, \lit{LatitudeName}, \lit{TimeName},
    \lit{DepthName}] These identify the names of the spatial
  quantities as used in the dataset.  Data requests to the \opendap server
  will be made with these names.  The variables may contain empty
  strings if no meaningful information is returned by the server about
  any of these variables.
  
\item[\lit{SelectableVariables}] The names of the variables in the
  dataset. These are the names that will be used in displaying the
  data to a user.  Wherever possible, you should use a name that is
  already in use in the browser.  For example, `Salinity' is already
  in use.  If you are adding a dataset containing salinity
  measurements, don't call it `Salt', unless there is some reason you
  don't want to see your dataset in a list of datasets containing
  salinity measurements.
  
\item[\lit{DodsName}] This is the name of the selectable variable as
  used by the dataset server.  Data requests to the dataset server
  will be made using these names.  For example, though the \opendap
  standard is to use `Salinity' (see above), the dataset server itself
  uses `SALT', and requests to that server must be made with that
  name.  Each entry on this list must exactly correspond to an entry
  on the \lit{SelectableVariables} list.
  
\item[\lit{DataNull}] This is a vector of null values, indicating
  the ``no data'' value for each of the spatial variables (in order:
  Time, Longitude, Latitude, and Depth), and each of the selectable
  variables (in the same order as the \lit{SelectableVariables} list.
  As data from the remote dataset is interned into Matlab, any values
  matching the appropriate element of this array will be changed into
  the Matlab NaN (``not a number'') value.
  
\item[\lit{DataScale}] A 2-column array, to be used in a linear
  conversion from the data as it is received from the remote dataset.
  The first column is the intercept and the second the slope.  The
  rows the array correspond to the entries in the \lit{DataNull}
  array.  Entries of \lit{[NaN NaN]} or \lit{[0 1]} do no scaling.
  Within the \GUI\ an effort is made to present the data to the user
  in SI units.  For example, density would be presented in $kg/m^3$.
  Not required.
  
  \note{While technically not required, \lit{DataScale} and
    \lit{DataNull} are in practice almost essential for the return of
    intelligible results to the user.  In addition, for datasets that
    use extreme values as a bad or missing data flag, (e.g., -9e42) ,
    no meaningful plotting will be able to be done within the browser
    unless \lit{DataNull} is declared.}

\end{description}
  
  

\subsection{Another example}

In the Levitus Ocean Atlas example above, the time axis is faked using
the start time, the stop time and the interval.  Some datasets consist
of many similar files, corresponding to different times.  These
datasets can benefit from using a catalog server to indicate the time
at which data was measured.

A catalog server is an \opendap server, serving a file that associates
time data or other data values with \opendap data URLs.  (See the
\OPDinstall for more information about setting up servers.)

For example, the ASCII reply (see \OPDuser) of the catalog server
corresponding to the Total Ozone Mapping Spectrometer (TOMS) project
looks like this:

\begin{vcode}{ib}
TOMS.type, TOMS.year, TOMS.month, TOMS.day, TOMS.DODS_URL
"ep", 2000, 1, 1, "http://daac.gsfc.nasa.gov/..."
"ep", 2000, 1, 2, "http://daac.gsfc.nasa.gov/..."
"ep", 2000, 1, 3, "http://daac.gsfc.nasa.gov/..."
"ep", 2000, 1, 4, "http://daac.gsfc.nasa.gov/..."
\end{vcode}

The \GUI\ (the \lit{getrectg} function, actually) contains a provision
for accessing this kind of catalog, but nominating a function to deal
with this data.

The archive M-file for the TOMS project looks like this:


\begin{vcode}{xib}
%      Total Ozone Mapping Spectrometer Daily Dataset

% FUNCTIONS -- REQUIRED
GetFunctionName = 'getrectg';

% FUNCTIONS -- OPTIONAL
Cat_m_File = 'ffcsquery';
CatalogServer = [ 'http://daac.gsfc.nasa.gov/daac-bin/nph-ff/DODS/catalog/', ...
                'toms-cat.dat'];

% VARIABLES -- REQUIRED
LonRange = [-180 180];
LatRange = [90 -90];
TimeRange = [1978.83287671233,2000.5];
DepthRange = [0 0];
DepthUnits = '';
Resolution = 138.75;

DataName = 'TOMS Global Level 3 Daily Gridded - GSFC';
LongitudeName = 'LONGITUDE';
LatitudeName = 'LATITUDE';
TimeName = '';
DepthName = '';
SelectableVariables = str2mat('Total_Ozone','Reflectivity');
DodsName = str2mat('TOTAL_OZONE','REFLECTIVITY');

DataNull = [nan nan nan nan 0 999];

Data_Use_Policy = '';

Acknowledge = '';

Comments1 = sprintf('%s\n', ...
\end{vcode}

The \lit{ffcsquery} function is packaged with the \GUI\ and is called
by \lit{getrectg}, using the \lit{CatalogServer} identified.  To adapt
this, you could create a catalog server, and use \lit{ffcsquery}, or
it might be simpler to write a catalog function (\lit{Cat_m_File})
that cheats and has the URLs and time values encoded within.

\section{The ``Get'' Function}
\label{gui,adding,get}

The \lit{getxxx} function, another Matlab program, serves as the connection
between the \opendap browser and individual datasets, hiding their
idiosyncracies and providing a uniform interface between the browser
and the data. Some get functions may serve for more than one dataset,
such as \lit{getrectg.m}, a function that works for a number of
gridded datasets.  The \lit{getxxx} function can be named anything, but
\subj{The browser uses the getxxx function both to get data, and
  information about data.}  convention dictates that the name be of
the form \lit{get}\var{xxx}\lit{.m}, where \var{xxx} is some sequence
of characters that indicates the dataset (or class of datasets) for
which the \lit{getxxx} function is used.  Note that several \lit{getxxx}
functions are provided with the \GUI ; be careful to avoid those
names.

A \lit{getxxx} function has 3 modes of operation (or 3 subroutines).
Each mode receives the same input arguments, although for some
arguments and some datasets, not all the incoming information will be
used.  The modes differ in their action, \emph{and in the data they
  return to the calling function}.  The input arguments to the
\lit{getxxx} function are as follows:

\proto{\lit{function [arg1,...] = }\lit{get}\var{xxx}
  \hbox{\lit{(}\var{mode},} \hbox{\var{ranges},} \hbox{\var{dataset},}
  \hbox{\var{vars},} \hbox{\var{stride},} \hbox{\var{num\_urls},}
  \hbox{\var{georange},} \hbox{\var{variables},} \hbox{\var{archive},}
  \hbox{\var{whichurl},}
  \hbox{\var{url\_info}\lit{)}}}


\begin{description}
  
\item[\var{mode}] A string argument which may be \lit{cat} (short for
  `catalog'), \lit{datasize}, or \lit{get}. The return values of this
  function differ depending on the \var{mode} argument.  They are
  outlined in the following sections.
  
\item[\var{ranges}] A 4x2 matrix containing the current user-selected
  longitude, latitude, depth and time ranges. The first column is the
  minimum and the second the maximum.
  
\item[\var{dataset}] An index number into an ordered list of datasets
  contained within the browser.  You can use this to check whether the
  dataset is the same one as was specified in the previous request.

\item[\var{vars}] A vector of indexes into an ordered list of
  user-selectable variables contained within the browser.  You can use
  this to check whether the request has changed since the last request
  made, but use the variable list for the names of the requested
  variables. 
  
\item[\var{stride}] A user-selected subsampling interval for gridded
  datasets.  A stride value of two requests every other data value,
  three gets you every third, and so on.

\item[\var{num\_urls}] The number of URLs in a data request.
  
\item[\var{georange}] A two-element vector indicating the geographic
  orientation and ranges of the browser display --- typically this
  will be \lit{[-180 180]} or \lit{[0 360]} but can be set to other
  values.
  
\item[\var{variablelist}] A character array of the variable names
  selected by the user.
  
\item[\var{archive}] The name of the \var{archive}\lit{.m} file for
the selected dataset.

\item[\var{whichurl}] -- For a ``catalog'' mode request, this is a
  structure containing information about the data to be found at the
  URL.  For a ``get'' mode request, this is an index into a list of
  URLs that comprise a data request.  The list will have been returned
  by a \lit{cat} mode request to the \lit{getxxx} function.

\end{description}

In some modes, a particular input argument will not contain valid
information.  For example, the input number of URLs is not valid for a
catalog request, since part of the function of the catalog mode is
to determine the number of URLs needed to satisfy a request.  However,
all arguments are passed in each time for the sake of consistency.

\subsection{\lit{getxxx} Function Modes}

The \lit{getxxx} function operates in three different modes.  One mode
returns URLs corresponding to a dataset, another returns estimates of
the request size, and the third actually retrieves data.  The
following sections describe the three modes.

\subsubsection{Catalog mode}

In \new{catalog mode} (or \lit{cat} mode) the browser is requesting
information about location in time and space of the indicated dataset.
\subj{Catalog mode is for determining the location of a dataset in
  space and time.}
The function takes the location of an indicated geo-spatial region
(using the \var{ranges} matrix), and returns a list of the points
where the queried dataset contains data.

The return arguments from such a request are:

\proto{\lit{[} {\var{x}}, \var{y}, \var{z}, \var{t}, \var{n},
  \var{URL} \lit{] = getxxx( ... )}}

\begin{description}
  
\item[\var{x, y, z, t}] Are one-dimensional vectors containing the
  locations and times corresponding to individually addressable URLs
  available to the browser user.  (Latitude and longitude in decimal
  degrees, depth in meters from the surface, time in decimal years.)
  These vectors will be used to plot tick marks on the browser
  display, and the user will be able to select one or more of these
  tick marks to get data.  All of the arguments must be returned,
  although those that are not appropriate for a given dataset may be
  returned empty.
  
  For example, a dataset containing buoy data might return a series of
  $(x,y)$ pairs, a vector of depths indicating instrument depths, and
  an empty time vector.  The browser will display the buoy locations,
  and the depth levels, allowing the user to select specific buoys or
  specific depths.  A dataset containing satellite images taken daily
  might return only a time vector, indicating the times of each image.

\item[\var{n}] is the total number of URLs identified by the catalog
  function.  The browser will call the \lit{getxxx} function (in
  \lit{get} mode) this many times to retrieve all the data referenced
  in the \lit{cat} mode request.
  
\item[\var{URL}] A URL directed at the dataset, or at a catalog
  server, if one was used to make the \lit{cat} mode request.  This is
  displayed in case you would like to make this request (or a related
  one) independent of the \GUI , say with \lit{loaddods}.
\end{description}

Sometimes the only way to provide the data requested in a catalog
request is to contact the dataset's \opendap server for data, or contact a
catalog server for information about the dataset (see
\sectionref{gui,adding,tools}).  For some datasets, such as monthly
climatologies, an internet request does not need to be made to satisfy
the request --- instead a Matlab function can be written which
examines the user-selected time and geographic ranges and determines
how many of the climatologies fall within these ranges.  For other
kinds of datasets, such as satellite datasets stored along a
groundtrack, which repeat in a predictable but irregular pattern in
time and space, reference satellite groundtrack locations and data
densities can be stored in a local \lit{.mat} file. 

During a catalog request, the \var{dataset}, \var{vars} and
\var{ranges} values should be stored internally (i.e., in a global
variable).  This is so that for future \lit{get} and \lit{datasize}
mode requests, the \lit{getxxx} function will be able to determine
whether or not the existing catalog is up-to-date.  If a \lit{get} or
\lit{datasize} request is made where the data parameters have
changed from the last catalog request, the \lit{getxxx} function should
make a new catalog request internally, before satisfying the new
request. 

The \lit{getxxx} function should also internally store
an ordered list of URLs based on this catalog request, or enough
information to recreate those URLs when requested by the browser.


\subsubsection{Datasize Mode}

The function of the \new{datasize mode} is to provide the browser with
an estimate of the size of a formulated data request as it will appear
in the Matlab workspace (as a double-precision float).  When a data
\subj{The browser uses datasize mode to estimate the size of a
  data request.}  request is issued, the browser checks to see if the
estimated datasize is below a threshold level (a level which defaults
to 1Mb but can be set by the user).  If above that level, the browser
requires the user to affirm the data transfer.  This is to prevent
users from unintenionally requesting very large data transfers. (See
\sectionref{gui,mref,Preferences}.)

The previous values (from the most recent catalog request) of the
\var{dataset}, \var{variable} and \var{ranges} variables should first
be compared with the new values, to see if the catalog request is
current.  If any of these values has changed, the \lit{getxxx} function
should first call itself in \lit{cat} mode to get the number of data
points or URLs or whatever the pertinent operative may be that causes
the size of a data request to change.

The return arguments from a \lit{datasize} request are:

\proto{\lit{[} {\var{datasize}}, \var{num\_urls} \lit{] = getxxx [ ... ]}}

\begin{description}
  
\item[\var{datasize}] The approximate size (in megabytes) of the data
  that will be returned by the specified data request.
\item[\var{num\_urls}] The number of URLs in the request.  (This is in
  case a new catalog request was required during the \lit{datasize}
  request.  If no new catalog request was required, the input
  \var{num\_urls} argument can simply be returned.)
\end{description}


\subsubsection{Get Mode}

The \lit{get} mode request (also ``data request'') queries the remote
dataset for data.  The browser makes a \lit{datasize} request before
\subj{And the get mode is for getting data.}  each data request,
and a \lit{cat} request before each \lit{datasize} request.  This
series is to be repeated if the user changes any of the search
parameters, like the geographic ranges or times.

The data request therefore need only operate on the ordered list of
URLs generated by the most recent \lit{cat} request --- the browser
simply increments the \var{url\_index\_number} from one to the total
number of URLs and waits for each result.  However because the input
information from the browser (such as the text list of selected
variables) is passed in each time, the \lit{getxxx} function does not
actually have to construct a URL until this point\footnote{This is a
time-saver.  It is not necessary to construct URLs that won't be
used.}.

Sometimes it is appropriate to return a count of one URL to the
browser while actually using more than one, internally.  This can
occur when, for example, the user requests all data from longitudes
40W to 40E from a gridded dataset with global coverage but the grids
are stored with a split at 20E (that is, the longitude in the dataset
ranges from 20E to 380E).  The \lit{getxxx} function in this case must
either transfer the whole dataset, then subset and reformat it, or
dereference two URLs, one from 40W to 20E and one from 20E to 40E, and
combine the result.  In other words, the number of derefernced URLs
returned by the \lit{getxxx} function should match what the browser
expects based on the result of the catalog mode request.\footnote{If,
on the other hand, the catalog function returned a count of two URLs
for this particular request, those URLs should be dereferenced and the
results returned to the browser separately.}

The data should be scaled according to the \var{DataScale} information
in the \var{archive}\lit{.m} file before being returned. 

\smallbreak

The return arguments from the \lit{get} mode are as follows:

\nopagebreak

\proto{\lit{[} \var{data}, \var{sizes}, \var{names}, \var{index},
    \var{URL} \lit{] = getxxx[ ... ]}}

\begin{description}
  
\item[\var{data}] is a $M\times1$ vector of data, where $M$ is the
  total number of data measurements returned.  That is, all the data
  are reshaped to a single column, with their original sizes recorded
  in the 'sizes' argument.
  
\item[\var{sizes}] is an $N\times2$ array of array dimensions,
  indicating the sizes of the returned arrays.  This requirement must
  be met: \lit{sum(sizes(:,1).*sizes(:,2)) = M}.
  
\item[\var{names}] is a list of character strings with the same number
  of rows as \var{sizes} --- each variable as it is unpacked will be
  given the corresponding name.  The current convention is to rename
  returning variables with the common names used in the browser.  If
  `Longitude' and `Latitude' (exactly so) are not present, the browser
  will not allow plotting on the browse window.  Some of the
  \lit{getxxx} functions now written wll create longitude and latitude
  vectors for gridded datasets using the information in the archive
  M-files.  This is an occasionally useful tactic for troublesome
  datasets. 
  
  \note{For the benefit of \mfour\ users, variable names should be no
    longer than 19 characters.}

\item[\var{index}] The index number is the
index into this list:

\begin{example}
Time
Longitude
Latitude
Depth
SelectableVariables[1]
.
.
.
SelectableVariables[\var{n}]
\end{example}

  The browser uses this index into the \lit{DataRange} array in the
  \var{archive}\lit{.m} file for mapping multiple images within the
  same dataset to a unified color scale, and into the \lit{DataScale}
  array for scaling the returned data.

\item[\var{URL}] This variable should contain the URL dereferenced to
  retrieve the data provided as a result of the \lit{get} request.

  This variable is to be a single string --- if several
  URLs have actually been dereferenced to satisfy what the browser has
  treated as a single URL, they should all be included, concatenated
  end-to-end, each separated by a space, i.e.:

\end{description}
                
  If \var{TimeName} in the \var{archive}\lit{.m} file is empty because
  the server doesn't actually return a time, you can insert a time
  made up from the catalog request.  However this is not required;
  every \lit{getxxx} function should be able to handle empty variables
  correctly.

  If any of the variable names resulting from dereferencing a URL is
  longer than 19 characters, the data contained in that variable must
  be requested separately for Matlab 4 users and the result mapped to
  the first 19 characters of that variable.  That is, you must first
  do:

\begin{vcode}{ib}
loaddods(URL)
\end{vcode}
%\noindent

then

\begin{vcode}{ib}
[longname(1:19)] = loaddods(URL);
\end{vcode}

This is a workaround for a bug concerning long variable names interned
into Matlab v4.

\subsection{\lit{getxxx} Function Tools}
\label{gui,adding,tools}

\opendap provides several tools with which to construct your \lit{getxxx}
function.  The tools fall into two broad classes: 

\begin{itemize}
\item \lit{loaddods}, the function used to dereference an \opendap URL, and

\item a set of Matlab utility functions, useful for the sorts of
operations likely to be needed by a \lit{getxxx} function.

\end{itemize}

\subsubsection{The \lit{loaddods} Function}

The loaddods\footnote{loaddods is a Matlab MEX-file, and appears in
your directories as \lit{loaddods.mex}\var{xxx}, where \var{xxx} is a
suffix defined by Matlab for a particular machine architecture.  For
example, \lit{loaddods.mexglx} would be for a GNU/Linux machine.  There is
also a \lit{loaddods.m} function, which contains a help message for
\lit{loaddods}.} function is the \opendap client for the Matlab browser.
\subj{loaddods is at the heart of every getxxx function.}
The function accepts a URL from the user, sends it out over the
internet, and creates Matlab scalars, vectors, and matrices to hold
the data that is returned from that request.  The \lit{loaddods.m}
help file contains several examples of its use. (Type \kbd{help loaddods}
at the Matlab command prompt.)

The function is called like this:

\proto{\lit{function \var{values} = loaddods([\var{switches},]
\var{URL} [per-url switches] [, \var{URL} [per-url switches]])} }

\begin{description}

\item[\var{switches}]  These options are all off by default except for
\lit{-v} which is on by default. 


\begin{description}

\item[\lit{-v}:] Verbose output. (This option is on by default; use
  \lit{+v} to disable it).

\item[\lit{-g}:] Use the Tcl/Tk progress and error reporting GUI.
  (This displays a bar graph showing transmission activity, and is not
  to be confused with the \GUI .  See \OPDuser\ for more information
  about this feature.). 

\item[\lit{-k}:] This switch causes \lit{loaddods} to concatenate two
  or more variables with same name.  When reading from multi-file
  datasets, use this option to concatenate the values read from
  several accesses into a single Matlab variable. This is intended for
  \lit{getxxx} functions which build up lists of URLs and pass them to
  \lit{loaddods} for evaluation. 

  You \emph{must} supply the URLs whose variables are to be
  concatenated in a single invocation of \lit{loaddods}. The \lit{-k}
  switch will not work across two or more calls to loaddods. That is,

\begin{vcode}{ib}
loaddods('-k', '<URL1> <URL2') 
\end{vcode}

will work, while calling \lit{loaddods} twice with \lit{-k} (once for
URL1 and once for URL2) won't work. 

\item[\lit{-n}:] This command switch causes Matlab variable names to
  be built using the complete name of the variable, including all the
  constructor types which contain it. Without \lit{-n}, variables are
  named using the name of the leaf variable.  This means, for example,
  that two separate \class{Structures} who each happen to have a
  member with the same name will create a name conflict.

\item[\lit{-s}:] Return a vector of variable names along with the
  variables themselves.  The \lit{-s} option can only be used when
  assigning the values accessed by \lit{loaddods} to a vector. If
  \lit{-s} is used in this situation, then the first variable in the
  vector will be set to a vector of strings which name the successive
  variables.

\item[\lit{-V}:] Print version information for loaddods and any
  programs it uses. 

\item[\lit{-t[abclmpstu]}:] Enable HTTP tracing (only for the hardcore...).

\item[\lit{-F}:] Force all string variables to be translated to float.
  The \lit{-F} option is needed when reading from datasets that return
  as string values that should be interned as numbers.

  \note[CAUTION]{This is a change from previous versions of
    \lit{loaddods} and introduces potential incompatibilities with
    version 1.5 and prior. In other words you'll need to change your
    software if it used version 1.5 or earlier.}

\item[\lit{-p}:] Recognize the DDMMSS strings and convert them (takes
  precedence over \lit{-F}).
  
\item[\lit{-A}:] The \lit{-A} option provides a way to access the DAS
  object of a dataset.  Given with a URL, \lit{-A} will cause all
  attributes to be interned in the Matlab workspace. Given with
  \lit{-c} only the contents of named attribute
  \emph{container}\footnote{Note the distinction between attributes
    and attribute containers.  A variable called ``v'' in some dataset
    will correspond to an attribute container called ``v'' in the
    dataset's DAS.  That container may contain attributes such as
    ``precision'' or ``scale\_factor'' or ``units.''} in a dataset
  will be interned. The latter is much safer since many DAS objects
  contain long names and names that repeat in several containers.
  Choosing an attribute container known to have `Matlab safe'
  identifiers is a good idea. (See \OPDuser\ for more information
  about the DAS structure, and attributes in general.)  Note that the
  attributes are interned into the Matlab workspace under their own
  names, not the name of the container\footnote{This is due to the
    variable name length limitations of Matlab.}.

  \note{While DAS objects are separate from the DDS (data) objects in
    \opendap, once loaddods reads the information from the writeval client
    program they are treated no differently than data objects. Each
    attribute container is interned as if it was a Structure variable
    in the \linebreak DDS object. }

\end{description}

\item[\var{per-url switches}]  These options are all off by default.

\begin{description}

\item[\lit{-r \var{var}:\var{new name}}.:] Rename a variable. 
  The \lit{-r} option provides a way to rename variables so that
  variables read will not overwrite ones already present in a Matlab
  session.

\item[\lit{-c \var{expr}}:] Supply a constraint without using the
  \lit{?} notation. 

\end{description}


\item[\var{URL}] The \lit{loaddods} function should be called with a
  \opendap URL which references data, in which case the command will
  access the dataset and create Matlab data structures to hold those
  data.  If the \var{URL} argument is a \lit{*} or \lit{?},
  \lit{loaddods} will cause Netscape to start up and open the default
  \opendap URL-builder.  If the \lit{*} or \lit{?} is followed by a URL,
  then that URL will be used instead of the URL builder. Finally, when
  called without an argument, the command will wait for a URL to be
  passed to it by Netscape or another program\footnote{See the \opendap
    \lit{urlqueue} program for information about this feature.}.

\end{description}

\subsubsection{loaddap}

\indc{loaddap}\indc{structure!preserving}\indc{data structure!preserving}
The \lit{loaddods} function is at the core of the \GUI.  However, for
backwards compatibility, it was originally written not to take
advantage of the advanced data structures now available in Matlab.
The \lit{loaddap} function is a newer version, that makes use of
Matlab data structures in an attempt to preserve the structure of
datasets. 

For example, if the target dataset contains three arrays in a
structure, the \lit{loaddods} function will bring them in as three
arrays.  The \lit{loaddap} function will import the data as three
arrays in a structure, corresponding to the original arrangement.

The use of \lit{loaddap} is identical in all other respects to
\lit{loaddods}.

If you find that the \GUI\ is more work than is necessary for your
application, and you use the command-line client, you might consider
using \lit{loaddap}.


\subsubsection{Utility Functions}

Several functions used in the \GUI\ are general enough to be of use to
people who have to write new \lit{getxxx} functions\footnote{The \GUI\ 
  is divided into two directories: the main directory, typically
  called something like \lit{ml-toolbox}, and the \lit{DATASETS}
  subdirectory contained in it.  Roughly speaking, the operational
  division between these two directories is between the generally
  useful functions and dataset-specific functions.  If a function
  written for a specific dataset appears to be useful for other
  datasets, it will be ``elevated'' to the main directory.}.  This
section contains a short list of some of the functions likely to be
useful.  The list is not complete, since the whole collection is
\subj{These functions have been useful in writing the getxxx
  functions provided with the GUI.  They may be useful to you, too.}
fairly fluid, as new datasets are constantly being added to the
archive.  We encourage function writers to browse the \GUI\ 
directories before starting in on what may seem like a common problem.

Many of the utility functions are listed below.  Use the Matlab
\lit{help} command for details of their invocation.

\begin{description}
    
\item[\lit{day2year}] Converts an integer day into a decimal year.

\item[\lit{dods\_dbk}] Removes trailing blanks from a string, and
  translates http escape sequences (e.g. \lit{\%20}) into underscores,
  so that \lit{Surface\%20Temperature} gets translated into
  \lit{Surface\_Temperature}.
  
\item[\lit{dods\_ddt}] (``de-dot'') Removes the parent structure name
  from compound names. For example, \lit{Cast.Temperature} is changed
  to \lit{Temperature}.  Actually, it removes all of a string up to
  and including the right-most period.

\item[\lit{findnew}] This function returns the position of
  non-duplicate data in a vector.  It is useful for pruning duplicate
  information from a string or vector.

\item[\lit{isleap}] Accepts an integer year AD, and returns 1 if it is
  a leap year according to Pope Gregory's newfangled calendar.

\item[\lit{mminmax}] Finds the minimum and maximum of a vector or a
  matrix, and returns them in a two-element vector.  The function is
  robust with respect to NaN.
  
\item[\lit{strntok}] Given a string and a token separator, quickly
  extracts tokens from short input strings.

\item[\lit{year2day}] Converts a decimal year into an integer day.
  (Note that \lit{1985.0} will return day 0.  To convert to the more
  common calendrical representation, where days are numbered from one,
  you must add one to the returned day integer.)

\end{description}


\section{Adding a New Dataset}
\label{gui,adding,manifest.lcl}

\indc{updating!local dataset list}\indc{dataset list!updating local}
\indc{local dataset list!updating}
Here's an example of how to add a gridded dataset to the \GUI.

We start with just a URL:

\begin{vcode}{xib}
http://ferret.pmel.noaa.gov/cgi-bin/dods/nph-dods/data/
  PMEL/large_trenberth_moncl.nc';
\end{vcode}

We will do this iteratively, improving our archive M-file as we go.
For the first step, we'll use the following.  (We've shortened the URL
here for formatting reasons.  Use the whole URL shown above.)

\begin{vcode}{sib}
GetFunctionName = 'getrectg';
Server = 'http://ferret.pmel.noaa.gov/...
          
LonRange = [-180 180];
LatRange = [-90 90];
TimeRange = [1800.0 2000.0];
DepthRange = [0.0 0.0];
Resolution = NaN;
DataName = 'Trenberth Wind Stress - PMEL';

SelectableVariables = str2mat('Wind Stress U', 'Wind Stress V');
\end{vcode}

It's missing the DodsName variable, but we can find that information
using \lit{loaddods}.  At the Matlab command line, type:

\begin{vcode}{sib}
DAS =loaddods('-A','http://ferret.pmel.noaa.gov/... ')
\end{vcode}

You can look at the DAS variable now, and you'll see a \lit{TAUX} and
a \lit{TAUY}, so those must be the names for the stresses.

So we'll add the following to the bottom of our tentative archive
M-file:

\begin{vcode}{sib}
DodsName = str2mat('TAUX', 'TAUY');
\end{vcode}

Now open the bookmarks editor (control-B is a shortcut), select a wind
dataset and copy and paste it to get a copy.  Open the properties
editor for the new entry, and erase the name and archive.  Pick a new
color, too.  Click \but{Apply} to make the changes in the bookmarks,
and \but{Reload} to force the old archive files to be cleared, and to
be reread, along with this new archive.  The dataset name should be
filled in automatically with the information from the new (but still
tentative) archive M-file.

Now click \but{OK} to dismiss the properties dialog.  Once back in the
bookmark editor window, Save (and apply to the GUI) and Close.

Once back in the browser window, select the new dataset.  Select the
entire time range and click \but{Get Details}.  You can examine the
URLlist returned, and you'll see that the URL returned has only twelve
data points in the time dimension.  This would imply that this is a
climatology, containing monthly averages.

\begin{vcode}{sib}
http://ferret.pmel.noaa.gov/...?TAUX[0:1:11][0:1:72][0:1:143]...
\end{vcode}

We still don't know the resolution or actual geographic extent,
but we can correct the original guess for TimeRange.  Substitute the
following line for the original time range guess: 

\begin{vcode}{ib}
TimeRange = [1800 str2num(datestr(date,10))];
\end{vcode}

To update the bookmarks, you can press Ctrl-B to bring up the
bookmarks editor, select the new dataset, Ctrl-I to get the properties
editor, Press \but{Reload/Clear}, then \but{OK} to dismiss this
window, Ctrl-S to save the bookmarks and apply them to the GUI again.

Now select June 1947 and Wind Stress U and V and click \but{Get
  Details}.  You will probbaly need to identify the axes again because
reloading the archive clears the cached metadata.  Check the URL
again, and you'll see the time slice which is selected in the
URL:

\begin{vcode}{sib}
http://ferret.pmel.noaa.gov/...?TAUX[5:1:5][0:1:72][0:1:143]...
\end{vcode}

Month 5 (starting at zero) is June, so the time axis is set
correctly.  Had it not been, we could use the \lit{Time-Offset}
feature of \lit{getrectg} to correct it.

Now we can click \but{Get Data}.  Go with the default plot option, and
you'll see that the extent of the dataset is indeed global.

To figure out the resolution, type this at the Matlab command prompt:

\begin{vcode}{ib}
>> diff(R1_Latitude)
\end{vcode}

This gives a value of 2.5 (so does the same test with Longitude), so
we'll replace the Resolution line in the archive file with this one
(111 km is the distance of one degree of longitude at the equator):

\begin{vcode}{ib}
Resolution = 2.5*111; 
\end{vcode}

To reload the new file, hit Ctrl-B to get the bookmarks editor, Ctrl-I
get the properties, 'Reload/Clear' one more time, dismiss the
properties, Ctrl-S, save and close.
Select the new dataset, reselect 
June 1947, and this time, a Resolution of 555.0 km.  Click
\but{Get Data}, re-identify the axes one last time when prompted,
and change the quiver plot color to 'r' (red).  Select ``overplot'' to
plot on top of the previous ploy, and you'll see that the scaling is
now different.  But you also see that you have every other data point
and they are also clearly one per 5 degrees, so the addition has been
a success.

% $Log: ch03.tex,v $
% Revision 1.7  2004/12/09 15:00:45  tomfool
% touch-up
%
% Revision 1.6  2004/12/09 03:04:14  tomfool
% updated for GUI 6
%
% Revision 1.5  2002/02/10 02:59:47  tom
% updated for GUI version 5 and dods-book2 templates
%
% Revision 1.4  1999/05/25 20:47:46  tom
% modifications for DODS release 3.0
%
% Revision 1.3  1999/01/20 15:18:39  tom
% updated, and changed for dods-book.cls and hyperlatex
%
% Revision 1.2  1998/12/07 15:41:15  tom
% updated for DODS v2.19 and GUIv0.7
%
% Revision 1.1  1998/02/12 18:10:41  tom
% added to CVS archive

%%% Local Variables: 
%%% mode: latex
%%% TeX-master: t
%%% End: 



