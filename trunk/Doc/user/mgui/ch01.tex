% Chapter to the Matlab GUI User Guide
%
% $Id$
%

\chapter{The OPeNDAP Matlab Graphical User Interface}
\label{gui,intro}

The \GUI\ combines OPeNDAP, a powerful data access protocol, with a
convenient set of display functions, and the flexibility and power of
the Matlab mathematics programming environment.  The effect of the
\GUI\ is to turn Matlab into a powerful network browser, capable of
reading specialized data messages from the Internet. For this reason,
we often refer to it as an \OPD\ ``browser.''

This chapter contains an overview of the browser, which will be useful
for most browser users. It starts in \sectionref{gui,intro,intro}.

\section{Quick Start}
\label{gui,quickstart}

If you are already somewhat familiar with Matlab and OPeNDAP, you may
find it simplest to jump right in.  (You might want to consult 
\OPDquick\ before this book, for a brief description of how OPeNDAP
works.)  After the \GUI\ has been installed, \subj{Some find it
  easiest to jump right in.}  and the necessary environment variables
set (See \appref{gui,install}), run Matlab.  

\begin{vcode}{ib}
$ matlab
\end{vcode}
%$

When Matlab has started up, start the browser  by typing:

\begin{vcode}{ib}
>> browse
\end{vcode}

Three windows will pop up.  One big one, for selecting geographic
regions in which to search for data, and for displaying it, and two
smaller ones.  One of these is for selecting datasets to browe or
display, and one for selecting the variables you want to display.

Use the mouse to select a dataset to look at.  (Click once to selet a
dataset.  Click twice to open a dataset folder.)  You'll see that the
lists of available variables changes to list only those variables in
that dataset.  Move to the Variable window, and choose a variable from
the list presented.  Now, switch to the big main window and use the
\but{Set Data Range} button to outline a geographic rectangle and a
time and depth range to look at.  Click \but{Get Details} to get an
indication about where the data is, and some information you can use
to estimate the size of the dataset you're requesting.  Click \but{Get
  Data}, and follow the directions, and you should soon see a map of
your data in the display window.  The browser imports data in a
consistent set of units, so that data from any source in the list will
appear in the same units.

Subsequent requests for data will not overwrite the original ones.
Each set of downloaded data is stored with a unique \matlab variable
name, so that many requests can be made in each \matlab session.

You can also begin by setting a data range to specify a time period or
geographic area of interest, and then selecting a dataset and
variable, or the other way around.  All of these parameters must be
set before making a data request, but they can be set in any order.

You will find a useful reference to the \GUI\ buttons and menus in
\chapterref{gui,ref}.

\section{What is the \GUI ?}
\label{gui,intro,intro}

The \GUI\ display consists of a map, a timeline, a depth
range, a small array of pushbuttons, and several drop-down menus
(\Figureref{gui,intro,mainwin}).

You can use the \GUI\ for four distinct operations:
setting the parameters of a request for data, querying a dataset about
its contents, issuing a request for data, and displaying the results
of a request.

\figureplace{The \GUI\ Main Window}{htb!}{gui,intro,mainwin}
{screen.ps}{screen.gif}{}

You can also see a text version of the window, by clicking on the
\but{View Text} button.  Return to the graphical view by clicking
\but{View Plots}.

\figureplace{The \GUI\ Main Window}{htb!}{gui,intro,mainwin-text}
{screen-text.ps}{screen-text.gif}{}

\subsection{Setting up a data request}

To set up a data request, you must select the variable and dataset you
wish to examine.  Select these from the variable and dataset lists
that appear when you start the browser.  If you've closed those
windows, you can recall them with the \but{Data,Show Data Bookmarks}
button. 

\subj{Choose the dataset and variable you want to see.}  The
\pdmenu{Variable} and \pdmenu{Dataset} menus depend on each other.
Once you've selected something from one menu, the other menu changes
to display the corresponding entries. For example, if you are
interested in temperatures, you can select it in the
\pdmenu{Variables} menu.  After this, the datasets that contain
temperature data will be highlighted (displayed with bold font) in the
\pdmenu{Bookmarks} menu. Once these have been selected, the spatial
and time ranges of the selected dataset are indicated on the display
map. You can use the \but{Zoom} button to examine any of the data
displays more closely.

You can also begin your data selection by selecting the region, time,
or depth you wish to examine. The \pdmenu{Datasets} and \pdmenu{Variable}
menus will be updated to show only datasets that contain data in those
regions.

The \GUI\ uses locally-cached archives of ``metadata'' to
describe datasets which may be stored on any machine connected to the
Internet. This allows the \GUI\ to download its data in a consistent
set of units, allowing comparisons among datasets.
You can use the \but{Data,Update Bookmarks} button to
update the local archives with any new datasets or changes in existing
datasets.

\subsection{Querying a Dataset}

After a dataset, variable, and region (space and time) are chosen, you
can find out roughly how much data (and what kind) you have selected with the
\but{Get Details} button. Clicking this will send a query to the
dataset to find out how much of it fits your chosen ranges; for
example, how many images fall within your time range, or how many XBT
casts fall within your geographic selection box.

For gridded data, the \but{Get Details} button will display a star in
the Time window for each grid available.  That is, if you are looking
at a dataset consisting of a series of complete grids, one per day,
you will see a star drawn for each day in the requested time range.
If you're looking at a dataset containing non-gridded data, you'll see
stars in the geographic and depth windows, too.

\subsection{Issuing a Data Request}

Once the parameters of a data request have been specified, you can
issue the request by clicking the \but{Get Data} button.  

Sometimes, inadvertently or not, you ask for a lot of data.  The
browser has a data size threshold you can set to notify you when you
make a request like this.  The data source you've selected may give 
the browser an estimate of how much data is on the way, and the
browser will ask you for confirmation in the event that the estimated
size of the data requested takes up more memory than the data size
threshold specified in the \but{Preferences,Get Data Threshold} popup
menu. 

%\pagebreak

If you approve the request (or if you weren't asked), the requested
data are retrieved, loaded into Matlab, and displayed according to the
directions (you will be asked for these directions, see
section~\ref{mgui,data-disp}).  The browser displays a message telling
you what has been loaded into Matlab (\figureref{gui,intro,urlmsg}).

\figureplace{Data returned to Matlab}{htb}
{gui,intro,urlmsg}{urlmsg.ps}{urlmsg.gif}{}

Note that along with the data you requested, the URL that pointed to
that data, as well as the text of any credits that apply to \subj{The
  request returns the URL, which you can use later.}  the data are
also returned.  To duplicate the request just made, say in some later
session, you could issue the following command:

\begin{vcode}{xib}
>> loaddods(R1_URL)
\end{vcode}

(This assumes that \lit{R1\_URL} has not changed.  Since the browser
is likely to re-use that name in future sessions, it is good practice
to save it in a different name if you think you might request that
data again.)

Of course, once the data are incorporated into Matlab variables, the
entire power of the Matlab analysis package is available.  For many
users, this is the real goal.  Other users may find the data display
functions of the \GUI\ valuable.


\subsection{Displaying the Data}
\label{mgui,data-disp}

Each time you use the browser to request data, it will ask how the
data are to be displayed.  The first thing that happens is that the
browser makes a guess as the X, Y, and Z coordinates you want to
display.  It confirms the guess by presenting a dialog to you
\subj{Choose the display axes.}  like the one in
\figureref{gui,intro,dispdlg}.  Using this dialog, you can select the
three coordinates to be displayed from among all the variables
returned by the data request.

Using this Display dialog, you can modify your earlier choices, and
you can also select the type of display you want (color image, contour
plot, quiver plot, and so on), and further select the returned data.

You can call this dialog up with the \but{Data, Plot acquired Rxx}
button. 

\figureplace{The Display Dialog}{htb!}{gui,intro,dispdlg}%
{dispdlg.ps}{dispdlg.gif}

In some circumstances, the Z data might be returned as a
three-dimensional array.  To display it, the array must be subsampled.
You can do this by typing a Matlab expression right into the Z-data
text box.  For some datasets, the Matlab matrix transposition operator
(\lit{'}) is important to use here.  Others might require a matrix
sampling. \subj{You can use the Display Dialog to subsample or modify
  data arrays.}

The dialog contains the following fields to modify:

\begin{description}
\item[Plot Number] This is the number of the plot, assigned and
  incremented by the browser.
\item[Acquisition Number] Requests for OPeNDAP data are numbered within a
  \matlab\ session.  Names of data variables from the first request
  are prefixed with \lit{R1\_}.  The second request makes variables
  with \lit{R2\_}, and so on.  This keeps data from being overwritten
  as requests are made after other requests.
\item[Plot Type] A list of available plot types.  These are standard
  \matlab\ types.
\item[Plot Data] Depending on the plot type selected, the display
  dialog will present an assortment of selection lists to set plot
  parameters. 
\item[Display Window] A selection list is presented to allow you to
  specify whether the plotting should be done in a new window, in the
  existing browser window, or in a window to be specified.
\item[X-data, Y-data, Z-data, V-data] The data to be plotted must be
  specified in these fields.  If any of these arrays must be trimmed,
  processed, transposed, or sampled, you can enter valid \matlab\ 
  expressions to do so in the text boxes.  For example, if you have an
  array that needs transposing, you can type the \matlab\ array
  inversion operator (\lit{'}) after the array name.
\end{description}



At the bottom of the display dialog, there are four buttons.  Use the
\but{Cancel} button if you've thought better of the whole thing, and
wish to make another data request.  The data will still be available
inside Matlab, but the browser will skip plotting it.  The \but{Plot
  Selected} button will plot the data selected in the dialog.  If you
want to plot several layers at once, use \but{Plot All}, and all the
data will be plotted at once.  A \but{Next Plot} button will appear on
the browser window to allow you to flip through the plots.

The \but{Overplot} button is a toggle button.  When it's activated,
the browser will not clear the previous plot when drawing a new one.
Click it to activate, and click it again to de-activate this mode.

\figureplace{The Display Dialog, Contour
  plotting}{htb!}{gui,intro,dispcontur}{dispdlg-part.ps}{dispdlg-part.gif}

Several of the plotting modes require different settings.  When you
select the plot mode, drop-down lists and dialogs will appear to allow
these settings.  See \figureref{gui,intro,dispcontur} for an example.


\section{Example Sessions}
\label{gui,intro,examples}

The simplest way to begin to use the \GUI\ is to follow along with a
simple session. Two examples follow: the first uses the \GUI\ window
to display data, and the second uses the browser only as a mechanism
to retrieve data, which is then examined with some other Matlab
functions.

%If you have any trouble following the directions in this section, you
%may find it useful to consult the troubleshooting hints in
%\appref{gui,trouble}.


\subsection{Drawing a Simple Picture}

\subj{Select the dataset and the variable.}  
Start Matlab, and start up the browser. (See 
\pagexref{gui,quickstart}).

\begin{enumerate}

\item The \pdmenu{Bookmarks} and \pdmenu{Variables} menus should
already be visible.  If not, use the \but{Data,Show Data Bookmarks}
button. The \pdmenu{Bookmarks} menu will look roughly like
\figureref{gui,intro,datamenu}.  Look under ``SST,'' then click on the entry
marked ``Reynolds Daily Long Term Mean SST.''

\note{When you click on this dataset, you'll see a little message come
  up labeled ``Policy.''  This describe where you can download this
  data yourself.  Data providers often require messages like this when
  programs like the \GUI\ incorporate their data.}

\figureplace{The Bookmarks Menu}{htb!}
{gui,intro,datamenu}{dsetmenu.ps}{dsetmenu.gif}{}
  
  This dataset contains sea surface temperatures for the entire globe,
  in one degree squares, for each day since about 1800. You will see a
  rectangle in the main window outlining the entire globe.  The time
  window contains a small line starting around 1800, and moving
  forward.

\subj{Zoom in to select the time.}
\item Click the \but{Zoom} button, and click the left mouse button
  near the Reynolds time line a few times, until the time is measured
  in days or months. (If you find the default colors hard to see,
  click on the \but{Preferences,Colors} button to adjust them to your
  liking. Use the \but{Preferences,Save Preferences} button to save
  your colors for the next time.)  The left mouse button zooms in, and
  the right button zooms out.  You can also select a small range by
  clicking and dragging, and that range will be mapped over the
  timeline.\indc{zoom!usage}
  
\item Click on \but{Set Data Range}, and select a time range by
  clicking and dragging in the time window (be patient). Remember that
  this dataset has a global grid recorded every day.  Be careful not
  to select more than you'r willing to wait for.
  
  While you are selecting the range, you might notice the
  \pdmenu{Datasets} or \pdmenu{Variables} menus changing.  These menus
  are dynamically loaded with all the datasets and variables that
  might overlap the data range you select.  However, the changes will
  not affect your selection.

\note{If you select a date or geographic range not covered by the
  dataset you've selected, the browser will unselect it.  Messages for
  such events are printed in the main Matlab window.}
  
\subj{Select the geographic range.}
\item Now select a geographic range by clicking and dragging in the
  main window. By default, the whole world is selected, so if you skip
  this step, you will still get data.  When selecting a time range,
  remember that this dataset is made up of weekly means, so be careful
  you don't ask for too much.
  
\item Click the \but{Get Details} button.  This will put a star at
  each point on the time line where data is available.  You may see
  that you've selected 29 points, which could bring a lot of data
  back.  You can refine your selection at this point to select the few
  data points of interest.
  
\item Click the \but{Get Data} button. You will see a dialog box that
  says: ``Hang on, transferring data,'' and some other stuff, like a
  dialog announcing what number this request is.
  Eventually, the browser will pop up the \pdmenu{Data Display} dialog
  (\figureref{gui,intro,dispmenu}), asking how you wish to display the
  \subj{Get the data.}
  data returned.  Unless you've been careful in
  selecting your time range, it's likely that the temperature array
  returned will have three dimensions.  The browser is designed to
  allow this, so it's not a problem.  If you want to see a plot of
  something besides the first time array, you may have to
  click on the text box for the Z values, and add a Matlab indexing
  expression.  That is, if the box reads \lit{SST}, and you want to
  see the twelfth array in the series, you should change
  the Z value text box to something like \lit{SST(:,:,12)} to get it.

\figureplace{The ``Display Data'' menu}{htb!}
{gui,intro,dispmenu}{dispdlg.ps}{dispdlg.gif}{}
  
\item Watch the data display. A message will announce when the data
  have been returned to the Matlab workspace, and what the new
  variable names are. You can examine and display these variables like
  any other Matlab variable.

\end{enumerate}

\subsubsection{A Short But Edifying Digression About the Internet}
\label{gui,intro,loaddods}

\subj{The GUI is just a specialized form of a web browser like Netscape.}
The \GUI gets its data over the internet, and the way it specifies the
location of the data it wants is with a URL very similar to the URL
you might use in a web browser, such as \lit{Netscape}.  In fact, the
\GUI\ can be thought of as a very specialized web browser, that can
only look at a very particular kind of web page.  When you type a URL
into a web browser, the browser ``de-references'' the URL to produce
the page you can read.  When you enter a URL into an OPeNDAP Matlab
program called \lit{loaddods}, that program de-references the URL to
produce data in the Matlab workspace.
\indc{loaddods@\lit{loaddods}}

It's a lot to type, but try typing the following at the Matlab prompt
(all one line):

\begin{vcode}{xib}
>> loaddods('http://ferret.wrc.noaa.gov/cgi-bin/nph-nc/data/
          COADS_climatology.nc?SST[7:7][0:1:89][0:1:179]') 
\end{vcode}

When \lit{loaddods} dereferences this long URL, it prints something
like this in the terminal window (not in the Matlab window, but in the
terminal you were using when you started Matlab):

\begin{vcode}{xib}
Reading: http://ferret.wrc.noaa.gov/cgi-bin/nph-nc/data/COADS_climatology.nc
  Constraint: SST[7:7][0:1:89][0:1:179]
Server version: dods/2.15
Creating matrix SST (90 by 180) with 16200 elements.
Creating scalar TIME.
Creating vector COADSY with 90 elements.
Creating vector COADSX with 180 elements.
\end{vcode}

and it puts the four arrays (\lit{SST}, \lit{TIME}, \lit{COADSX}, and
\lit{COADSY}) into your Matlab workspace.

\begin{vcode}{ib}
>> whos
  Name               Size         Bytes  Class

  COADSX           180x1           1440  double array
  COADSY            90x1            720  double array
  SST               90x180       129600  double array
  TIME               1x1              8  double array
\end{vcode}

The \lit{loaddods} program is the heart of the \GUI .  The rest of the
\GUI\ is simply a ``front-end'' to the \lit{loaddods} program that you
use to construct URLs and make some housekeeping chores somewhat
easier.

You can use this to your advantage, as is shown in the next example.

Incidentally, when the \GUI\ returns data, one of the returned
variables is the exact URL that was used to issue the request for that
data.  This provides a unique identifier should you want to recreate a
request or expand on it, perhaps in your own Matlab script.

%% loaddap fun facts.
%%
%% loaddap also includes the real name of the variable in a ``real
%% name'' field.

\subsection{Creating a Time Series}
\label{gui,intro,example-time}

It is often the case that the data you wish to see cannot be easily
displayed in the \GUI\ window. \opendap allows much richer access to the
data than the browser can accommodate. This means that you can often
use the \GUI\ to help construct a URL, even though you may not use it
to submit the data request.

\subj{You can also use OPeNDAP from the Matlab command line.}
The following example shows how you might go about constructing and
displaying a time series from the \GUI\ datasets, even though there is
limited browser capacity for displaying this kind of data.  The \GUI\
actually is flexible enough to plot this example, but there are others
for which this technique is useful.  It's also useful to become
familiar with \lit{loaddods} in case you want to include access to
\opendap datasets in your own scripts.

\begin{enumerate}
  
\item Start up the browser and select the ``Surface - COADS Monthly
  Datasets - PMEL'' dataset.

\item Select ``Air Temperature'' from the \pdmenu{Variables} menu. (See
  \figureref{gui,intro,varmenu}.)  If it's not there, you haven't
  selected a dataset containing that variable. 

\figureplace{The Variables Menu}{htb!}{gui,intro,varmenu}
{var-menu.ps}{var-menu.gif}{}
\tbd{This figure does not (now) have ``AIR_TEMP'' highlighted.}  

\item Using the \but{Zoom} and \but{Select Data Range} buttons, select
  a very small geographic range. This dataset is arranged in
  two-degree squares, so pick a two-degree square somewhere. Pick a
  wide time range.
  
\item Click on \but{Get Details}. Select a time at the beginning
  of the range. Click on \but{Get Data}.
  
\item Repeat the process for a time at the end of the time range.
  Click on \but{Get Data}.
  
\item Look at the Matlab session. You will see two messages like this:

\begin{vcode}{xib}
Reading: http://ferret.wrc.noaa.gov/cgi-bin/nph-nc/data/coads_air.nc
  Constraint: AIR[1316:1316][57:1:57][140:1:140]
Server version: dods/2.15
Creating matrix AIR (2 by 2) with 4 elements.
Creating scalar TIME.
Creating vector LAT with 2 elements.
Creating vector LON with 2 elements.
                                  
          ..... data transfer complete
   
This request generated 1 separate URLs,
which are stored in the sets:  R3_   
   
Each individual argument is stored like so:
   
R3_Time R3_Longitude R3_Latitude R3_Air_Temp 
\end{vcode}

The above URL indicates that your request asked for time index 1316 in
the dataset, latitude index 57, and longitude index 140. (This
corresponds to September, 1963, at about 24\degree N and 78\degree W.)

\item Type the following at the Matlab prompt (all on one line):

\begin{vcode}{xib}
>> loaddods('http://ferret.wrc.noaa.gov/cgi-bin/nph-nc/data/
coads_air.nc?AIR[1316:1500][57:57][140:140]')
\end{vcode}

You will see a message like this:

\begin{vcode}{xib}
Reading: http://ferret.wrc.noaa.gov/cgi-bin/nph-nc/data/coads_air.nc
  Constraint: AIR[1316:1500][57:57][140:140]
Server version: dods/2.15
Creating vector AIR with 185 elements.
Creating vector TIME with 185 elements.
Creating scalar LAT.
Creating scalar LON.
\end{vcode}

\item You now have, in the Matlab workspace, a vector of temperatures,
  \lit{AIR}, whose entries correspond to a vector of times,
  \lit{TIME}. The scalars \lit{LAT} and \lit{LON} give the location of
  the data. You can display the data with Matlab functions. Type the
  following at the Matlab prompt:

\begin{vcode}{ib}
>> figure
>> plot(TIME,AIR)
\end{vcode}

\end{enumerate}

\note[Caution]{There are some pitfalls to using \lit{loaddods}
  directly. The \GUI\ does quite a bit of housekeeping for you,
  including some unit conversions.  When you use the browser, you get your
  data back in consistent units, whatever dataset you are looking at.
  When you use \lit{loaddods}, you get the data are returned in the units
  they are stored in.  The COADS time data, used in the above example,
  comes in units of days since 1700.\indc{loaddods!pitfalls}
  \notebreak
  Similarly, different data providers use different values within
  their datasets to indicate ``bad'' or ``missing'' data points.
  Through the browser, these data flags are transformed into Matlab
  \lit{NaN}s. (See Matlab `help nan' if you are not familiar with the
  \lit{NaN} symbol.)  Again, downloading the data directly with
  \lit{loaddods} skips this transformation.}

If you like, you can save your data with the Matlab \lit{save} command
and restore it with the \lit{load} command. See the Matlab
documentation for more information about these commands.  \indc{saving
  your data}\indc{data!saving}

\section{Updating the Dataset List}
\label{gui,intro,update}

\subj{Maintaining your set of dataset bookmarks.}
\indc{updating!dataset list}\indc{dataset list!updating}
The \GUI\ consists of two sets of files: the M-files and associated
data used to run the display and user interface, and a second
collection of Matlab M-files used to describe the datasets available
through the \GUI.  These ``metadata''\indc{metadata} files (also
called the ``\indn{archive M-files}'') contain descriptions of the
datasets, and retrieval and catalog functions tailored
specifically for them.

The metadata files are stored in a sub-directory of the main \GUI\ 
directory, called \lit{DATASETS}. Generally speaking, most users will
be quite unlikely to want to edit the \GUI\ M-files. However, there
are several reasons why you might have to edit the archive M-files:
adding a dataset, adding comments to a dataset, or otherwise editing
the metadata.  \chapterref{gui,adding} contains information about the
structure of the archive M-files and their associated functions.

For example, the archive M-file for the declouded Hatteras to Nova
Scotia 1.1km resolution AVHRR dataset is called \lit{htn\_decld.m}.
You will find this script in the browser subdirectory \lit{DATASETS}.

The dataset descriptions are stored in Matlab M-files, but the way a
browser user selects these M-files is through a ``bookmark'' file,
comparable to the bookmarks used in a web browser like
\lit{Netscape}.  You can edit and rearrange these bookmarks into
folders like in \lit{Netscape}, too.
\indc{bookmarks}
If you want to add your own M-file dataset descriptions to the
bookmark file, you can just add it with the bookmark editor.

\figureplace{GUI Server Dialog}{htb!}{gui,intro,guiserver}
{guiserver.ps}{guiserver.gif}{}

The collection of datasets the browser can access is continually
updated at the \GUI\ central archive.  You can update your browser
with the contents of that archive by clicking the \but{Data,Update
  Datasets} button.  If you know of another archive of GUI dataset
descriptions, you can use the \but{Preferences,Change GUI Server}
option to point the browser at that server instead.

\figureplace{Update Datasets Dialog}{htb!}{gui,intro,updatefig}
{update.ps}{update.gif}{}

When you want to update the dataset descriptions in your copy of the
\GUI, click on \but{Data,Update Datasets}, and the dialog in
\figureref{gui,intro,updatefig} will appear.  If you want to continue,
simply click the \but{Update} button.

The update function downloads a new \lit{MANIFEST} file, and then
retrieves all the new files mentioned in that file, as well as new
versions of already-existing M-files in your DATASETS directory.  The
update process also updates any bookmarks you might have that point to
datasets in the \lit{MANIFEST}.  If there are bookmarks that you've
added (that match no entry in the \lit{MANIFEST}), they are left
untouched.  If there are datasets in the \lit{MANIFEST} that are not
in your bookmarks file, they are added to your bookmarks, in a folder
called ``New Datasets.''

\section{Editing Bookmarks}
\label{gui,intro,editbook}

\indc{bookmarks!editing}\indc{editing!bookmarks}
The \opendap datasets presented to the user in the \GUI\ are described by
a list of ``bookmarks.'' These bookmarks are comparable to the
bookmarks in the \lit{Netscape} WWW browser.

You can edit and augment the bookmark list with the bookmark editor.
Invoke the editor with the \but{Data,Edit Bookmarks} button.

The editor window is shown in \figureref{gui,intro,bookedit}.  It looks
a great deal like the bookmark window (see
\pagexref{gui,intro,datamenu}), but the menu along its top is
different.

\figureplace{Bookmark Editor Window}{htb!}{gui,intro,bookedit}%
{edit-book.ps}{edit-book.gif}{}

To see a bookmark entry, highlight the bookmark entry, and click on
\but{Edit,Properties}.   You'll see an entry like the one in
\figureref{gui,intro,bookprop}. 

\figureplace{Bookmark Properties Window}{htb!}{gui,intro,bookprop}%
{edit-pmenu.ps}{edit-pmenu.gif}{}

The ``Dataname'' at the top of the window is the name given to the
data by the M-file describing that dataset.  The ``Name'' field is the
name that appears in the bookmark window, and the ``Color'' is the
color in which that name is displayer.  The ``Archive'' field names
the M-file that described the dataset in question.  You can edit fields
here and save the changes with the \but{Ok} or \but{Apply} buttons.

If you want to add a dataset, and have the M-file ready (see
\chapterref{gui,adding}), select \but{Edit,Insert New}, and you'll get
a dialog like the properties dialog, but empty.
(\Figureref{gui,intro,booknew}) 

\figureplace{Adding Bookmarks}{htb!}{gui,intro,booknew}%
{edit-newb.ps}{edit-newb.gif}{}

To add the dataset, make sure the M-file is in place in the
\lit{DATASETS} directory, fill out this dialog, and click \but{Apply},
then \but{Reload/Clear}.  When the correct Dataname appears in the top
window, you'll know your dataset has been registered.


% $Log: ch01.tex,v $
% Revision 1.10  2004/12/16 06:52:34  tomfool
% edits from Peter
%
% Revision 1.9  2004/12/09 15:00:45  tomfool
% touch-up
%
% Revision 1.8  2004/12/09 03:04:14  tomfool
% updated for GUI 6
%
% Revision 1.7  2002/02/10 02:59:47  tom
% updated for GUI version 5 and dods-book2 templates
%
% Revision 1.6  2000/10/04 15:01:37  tom
% changed \figureplace definition
%
% Revision 1.5  1999/05/25 20:47:46  tom
% modifications for DODS release 3.0
%
% Revision 1.4  1999/02/04 17:42:13  tom
% modified to use dods-book.cls and Hyperlatex
%
% Revision 1.3  1999/01/20 15:18:38  tom
% updated, and changed for dods-book.cls and hyperlatex
%
% Revision 1.2  1998/12/07 15:41:15  tom
% updated for DODS v2.19 and GUIv0.7
%
% Revision 1.1  1998/02/12 18:10:41  tom
% added to CVS archive
%
%

%%% Local Variables: 
%%% mode: latex
%%% TeX-master: t
%%% End: 
