% Chapter to the Matlab GUI User Guide
%
% $Id$
%
%

\chapter{The DODS GUI Menu Reference}
\label{gui,ref}

This chapter contains a description of each of the menus and
buttons that make up the \GUI. They are arranged by function, and by
containing menu.  The buttons and menus are also referenced in the
index. 

%% Make ``index'' above into a link, make the index file have a
%% permanent name, fix refs below so that the spaces don't appear in
%% the labels.  They don't appear, to, thoguh.  So review the hlx use
%% of labels and refs, because we're confused.

\newcommand{\button}[1]{\item[\but{#1}]\label{gui,bref,#1}%
                        \indc{#1!button}\indc{button!#1}}
\newcommand{\menudesc}[1]{\subsection{#1}\label{gui,mref,#1}%
                          \indc{#1!menu}\indc{menu!#1}}
\newcommand{\submenudesc}[1]{\subsubsection{#1}\label{gui,smref,#1}%
                          \indc{#1!menu}\indc{menu!#1}}

\section{The Buttons}

The following buttons are available on the main screen of the \GUI .
(See \figureref{gui,intro,mainwin}.) They are roughly arranged by
function.

\begin{description}

\button{Set Data Range}

Before a query can be made to a dataset, the dataset and variables
must be selected, and a range must be selected in space and time. If
you are looking at the graphical view, you can click this button, and
then select an area in any of the selection windows (global location,
time, and depth) by clicking and dragging the left mouse button.

You can also enter the boundaries of the selected range by
hand. Simply switch to the text view with the \but{View Text} button,
and enter the values in the entry windows provided. 

\button{Zoom}

Use this button to examine any of the displayed axes more closely.
When the zoom operation is in effect, the left mouse button zooms in
to a closer view, while the right mouse button zooms out to a more
distant view. If you click and drag, the enclosed area will be
enlarged to fill the selection window. (This is the standard Matlab
``zoom'' function.)  \indc{zoom!usage}

\note{Zooming in on the map, time line, or depth line does not select
a region. In order to submit a request for data, you must select a
range with the \but{Set Data Range} button.}

\button{Get Details}

Once you have selected a dataset, variables, and range, you can click
the \but{Get Details} button for information about the volume and
location of the data in the dataset. For example, if you select the
Reynolds Weekly Sea Surface Temperature dataset, \but{Get Details}
will produce an asterisk at each time within the selected time range
where there is data. If you zoom in on these asterisks, you will see
one of them every seven days. If multiple asterisks appear in the
range you have selected, you might choose to refine your range to make
a smaller request before you submit the data request. Of course, you
can request data from several times, in which case there will be
multiple accesses to the remote dataset. Each request will be interned
separately into the Matlab workspace.

\button{Get Data}

Use this button to submit a request for data from the remote server.
If your data range contains several data points, your data may be
returned in several Matlab arrays, or in a multi-dimensional array
that will need to be subsampled to display. A message box will
announce the names under which the data are to be stored.

For very large requests, a warning window may appear to remind you
that the request you are about to make is a large one. Press ``OK'' to
confirm you still want the request to be submitted.  See
\pagexref{gui,bref,Get Data Threshold} for more information.

\button{Next Plot}

This button only appears when you've plotted many variables at once.
You can use this button to flip through the plots, one at a time,
until the end.

\button{Acknowledgements}

Selecting this button displays a text window containing text
describing the source and authors of the dataset currently selected.
You can use this text to cite the source of the data, should you end
up using it in a context where this is appropriate, such as a
publication.  If you make data products available to other scientists
using this data, it is also considered good form to include this
citation with that data in some way.

\button{View Text}

The text version of the main GUI window allows you to type the
boundaries of the data range, and displays the dataset comments, which
contain detailed descriptions of the datasets.

Press this button if you are looking at the graphic version of the
main window and you want to be looking at the text version. 

\button{View Plot}

The graphic version of the main GUI window allows you to select the
search boundaries by pointing and dragging with the mouse. This is
also where you can see the displayed results of a search. 

Press this button if you are looking at the text version of the main
window and you want to be looking at the graphic version. 

\button{Clear Selection}

This button clears all the selections you have made: data range,
variable and dataset. The \pdmenu{Datasets} and \pdmenu{Variables} menu
windows are restored to their initial conditions.

\button{Quit}

This button terminates the browser. Matlab continues running after the
browser finishes; you have to type \kbd{quit} at the Matlab prompt to
end your Matlab session. 

\end{description}

\section{The Menus}

The menu bar over the matlab main window (\figureref{gui,ref,menubar})
contains drop-down menus with several useful functions.
The functions in the \GUI\ menus allow you to select the datasets and
variables to search for and display, to control many aspects of how
the data are displayed, and to adjust the overall appearance of the
main window.

\figureplace{The GUI Menubar}{htb}{gui,ref,menubar}%
{menubar.ps}{menubar.gif}{}

\menudesc{Data}

The \pdmenu{Data} menu (\figureref{gui,ref,datamenu}) contains buttons
that manage aspects of the data bookmarks and downloaded data.

\figureplace{The Data Menu}{htb}{gui,ref,datamenu}
{datamenu.ps}{datamenu.gif}{}

\submenudesc{Show Data Bookmarks}

This button causes a list of the data bookmarks and the variable
browser window to appear.  You must select a dataset from the
bookmarks window, as well as a variable, in order to submit a data
request. 

If you click on one of the variable names in the variable browser
window, the \pdmenu{Datasets} menu window will only display the
datasets that contain that variable.  Similarly, if you have already
made a selection of a dataset or a range, the \pdmenu{Variables} menu
only displays the variables of datasets still on the dataset menu.

The \pdmenu{Variables} menu is persistent; after you have displayed
it, it will remain on your screen until you close it. Sometimes the
menu will seem to disappear; this is because your window manager
software causes the browser window to cover over the new
\pdmenu{Variables} menu.

\submenudesc{Edit Bookmarks}

Clicking this button invokes the bookmark editor.  You can use this to
edit items in the list of dataset bookmarks offered in the bookmark
window, or you can add your own.  See \sectionref{gui,mref,Bookmarks}
for a description of the bookmark editor menus, and see
\sectionref{gui,intro,editbook} for an introduction to the operation
of the editor.

If you want to add a bookmark, you should probably look at
Chapter~\ref{gui,adding}. 

\submenudesc{Open Bookmarks File}

The \GUI\ supports having multiple files of bookmarks.  Use this
button to change the bookmark file currently in use.


\submenudesc{Plot Acquired Rxx}

\indc{window!Data Display}\indc{dialog!Data Display}
If you want to replot data you've already downloaded, use this option,
and you'll see the plot dialog you originally saw when you downloaded
the data in the first place.  See section~\ref{mgui,data-disp} for
more information about plotting.


\submenudesc{Clear Rxx_ from workspace}

Selecting this option causes all Matlab variables with names like
R20\_Pressure that have been created during the current browse session
to be deleted. Effectively, this clears all variables created by
remote data requests. Of course, if an Rxx variable has been copied to
a variable of a different name, the new variable will not be deleted.

\submenudesc{Find}

\indc{text!search}\indc{search!text}
You can use this button to search for a text string.  You can search
the dataset names in the bookmark file, the variables made available
by any of those datasets, the dataset descriptions, or the
acknowledgements attached to each dataset.  The searcher returns the
first result it finds, displaying the associated dataset in the Text
view.  Click \but{Find Next} to see the next match.

\figureplace{The (Text) Search Dialog Menu}{htb}{gui,ref,finddlg}
{finddlg.ps}{finddlg.gif}{}

\submenudesc{Update Bookmarks}

\indc{MANIFEST@\lit{MANIFEST}}
\indc{updating!bookmark list}\indc{bookmark list!updating}
\indc{updating!local bookmark list}\indc{bookmark list!updating local}
\indc{local bookmark list!updating}
\indc{updating!dataset list}\indc{dataset list!updating}
\indc{updating!local dataset list}\indc{dataset list!updating local}
\indc{local dataset list!updating}
The archive M-files that come with the \GUI\ are kept up-to-date in a
central location on the DODS \GUI\ server, currently at
  URI.  You can update your
copies of these files at any time with this button. (See
\sectionref{gui,intro,update}) Clicking this button pops up the
\pdmenu{Update} menu, which gives you further instructions.

\menudesc{Resolution}

This menu allows you to select the ``resolution'' at
which you wish to see a dataset displayed. The resolutions used
indicate (roughly) the distance between adjacent values of latitude,
in kilometers. Of course, this only applies to data lying on a
rectangular grid.  \note{The resolution applies equally to the X and Y
  directions.  However, the browser expresses resolution in units of
  distance, which (for a dataset whose locations are recorded in
  longitude and latitude) only remains constant in the North-South
  direction.}

\figureplace{The Resolution Menu}{htb}{gui,ref,resolution}
{resolution.ps}{resolution.gif}{}

The resolution choices for the Reynolds Weekly mean dataset are shown
in \figureref{gui,ref,resolution}.  This dataset is recorded on a
global one-degree grid, so the bottom choice (111 km) approximately
corresponds to one-degree of latitude.  Through the browser, the
resolutions available are the following multiples of the fundamental
resolution of the dataset: 2,3,4,8,12,16,24, and 36.  (This can be
adjusted by editing the saved preferences file.  See
\sectionref{gui,ref,options}.) 

The resolution corresponds directly to the ``stride'' value you use in
requesting gridded data with DODS.  (Refer to \DODSuser\ for more
information about grids, stride values, and how they fit together.)
Therefore, if the available resolutions are not what you need, you can
use \lit{loaddods} to request a different resolution.  See
\pagexref{gui,intro,loaddods} for more information about that program.

\menudesc{Display}

This menu, shown in \figureref{gui,ref,displaymenu}, contains various
options that allow you control the way data are displayed, if at all.

\figureplace{Display Menu}{htb}{gui,ref,displaymenu}%
{dispmenu.ps}{dispmenu.gif}{}

It contains several buttons:

\begin{description}

\button{Zoom Out}

If you've zoomed in on some part of the world map, this button
restores the map to cover the whole world.  It also restores the time
line and the depth range to their original values.

\button{Clear Display}

This button erases the displayed data from the browser main window.

\button{Reset Display}

Erases the displayed data from the browser main window, and zooms out
to view the whole world, the complete time line, and all depths.

\button{Start Longitude}

This changes the display map longitude boundaries. Start the longitude
at 180W (the international date line) if you want to see the Atlantic,
but start at 0E (the greenwich meridian) if you want to see the
Pacific entire. This is a toggle switch.

\end{description}

\menudesc{Preferences}

The preferences menu is shown in \figureref{gui,ref,prefs}.

\figureplace{Preferences Menu}{htb!}{gui,ref,prefs}%
{prefmenu.ps}{prefmenu.gif}{}

\begin{description}

\button{Time}

The date can be displayed as Year/Month/Day triples, or as the
year and the ordinal number of the day for that year. This is a
toggle switch.

\button{Messages} 

Messages to the user can be directed either to pop-up windows
(``Messages to Pop-up Window'') or to the \matlab\ command-line
interface (``Messages to Workspace'').  This is a toggle switch, click
it to change its value.

\button{Get Data Threshold}

\label{gui,ref,threshold-desc}
This button calls up a dialog box that controls the data request
threshold. (See \figureref{gui,ref,threshold}.)

This threshold marks the maximum size of data file that the Matlab
DODS browser will fetch without prompting the user for confirmation.
For example, if the threshold is set at 1 megabyte of data, the user
will be prompted for confirmation of any data request likely to return
more than 1 megabyte of data. Because the \GUI\ cannot know exactly
how big a dataset is until it has actually downloaded the entire data
request, this threshold works with estimates.

\figureplace[The Data Threshold Dialog]{The Data Threshold Dialog.
  Use this dialog to control the \emph{estimated} size of the data
  request.}{htb!}{gui,ref,threshold}{threshld.ps}{threshld.gif}{}

\note{A further problem with this threshold is that the estimates are
  about how much memory the data will occupy in Matlab memory,
  \emph{not} the size of the the transmitted information. The memory
  occupied by data in Matlab can be much larger than the amount of
  data transmitted.  Any number in Matlab occupies eight bytes of
  storage space. Many DODS data---especially satellite data---are
  transmitted as single byte numbers. Further, the \GUI\ has no way to
  know whether the transmitted data were compressed or not.  This can
  increase the size of this discrepancy.  In other words, this
  threshold is meant to be a guide to an informed user, not a precise
  prediction.}

\button{Fontsize}

This button pops up a submenu you can use to select the size of the font
used to display titles, legends, and so on.

\button{Color}

This button pops up a submenu you can use to select the colors of
various browser features.  See \figureref{gui,ref,colordlg}.

\figureplace{Color Menu}{htb!}{gui,ref,colordlg}%
{clrmenu.ps}{clrmenu.gif}{}

\button{Save}

Save the preferences to the file from which they were loaded.  See
\pagexref{gui,bref,Save As}.

\button{Load Palette}

Opens a dialog allowing you to load a palette for mapping.  See
\figureref{gui,ref,clrpalet}. 

\figureplace{Color Palette Dialog}{htb!}{gui,ref,clrpalet}%
{clrpalet.ps}{clrpalet.gif}{}

The colormap is an ASCII file containing a standard Matlab RGB
colormap.  This is a 3-column matrix with values between 0 and
1. The columns represent the relative strength of the red, green,
and blue components, respectively. Matlab also has a number of built-in
colormaps (type \kbd{help colormap} at the Matlab prompt for more
information). These can be specified from the Matlab command line.

The \lit{avhrrpal.pal} file (the default color palette) consists of
255 lines like this:

\begin{vcode}{ib}
...
0.2000 0.0000 0.3000
0.3700 0.0000 0.5000
0.4516 0.0000 0.7000
0.4548 0.0323 0.8710
0.4548 0.2323 1.0000
...
\end{vcode}

The file represents a $255x3$ array.  You can think of it as a list of
255 colors, each of which is specified as a combination of red, green,
and blue values (from left to right).  Each value represents the
fraction of intensity, with 1.0 equal to full intensity of that color
component. 

\indc{colormap!changing default}
\indc{avhrr.pal@\lit{avhrr.pal}!changing} 
To use one of the colormaps that come with Matlab, you can create a
palette in an external file with the following steps:

\begin{enumerate}
\item Create a color map with the following command:

\begin{example}
>> newpalette = hsv(256);
\end{example}

Any of the following could substitute for the \lit{hsv} in the above
example.  They are all colormaps supplied with Matlab:

\begin{description}
\item[hsv] Hue-saturation-value color map.
\item[hot] Black-red-yellow-white color map.
\item[gray] Linear gray-scale color map.
\item[bone] Gray-scale with tinge of blue color map.
\item[copper] Linear copper-tone color map.
\item[pink] Pastel shades of pink color map.
\item[white] All white color map.
\item[flag] Alternating red, white, blue, and black color map.
\item[lines] Color map with the line colors.
\item[colorcube] Enhanced color-cube color map.
\item[jet] Variant of HSV.
\item[prism] Prism color map.
\item[cool] Shades of cyan and magenta color map.
\item[autumn] Shades of red and yellow color map.
\item[spring] Shades of magenta and yellow color map.
\item[winter] Shades of blue and green color map.
\item[summer] Shades of green and yellow color map.
\end{description}

\item Save the result into an ASCII file:

\begin{vcode}{ib}
>> fid = fopen('newpalette.pal');
>> fprintf(fid,'%g %g %g\n',newpalette');
>> fclose(fid);
\end{vcode}

\end{enumerate}



\button{Set GUI Server}

You can use the dialog summoned by this button (see
\figureref{gui,intro,guiserver}) to update your master list of
datasets from some other source than the DODS central archive.

\button{Save As}
\label{gui,ref,saveopt-desc}

If you have changed preferences such as color, font, threshold or
anything else, you can save these values by selecting this option.
When the \GUI\ is next started, these same values will be read from
the startup file, by default called \lit{browsopt.m}.  You can control
the name with the Save As dialog box.  To reload a preference file,
use the \but{Preferences,Load Prefs File} button.  Browser preference files
are simple Matlab scripts and may be edited by hand.

The browser options saved with this command are described in
\sectionref{gui,ref,options}.

\button{Load Prefs File}

This button allows you to load a set of user preferences saved from a
previous browser session.  Clicking the button brings up the dialog..
The browser will execute the given preferences file, expecting it to
be a file such as are saved by the \but{Save Preferences} command.
Note that the preferences file \emph{must} be somewhere on your Matlab
path.  If you are unsure, type \lit{path} at the Matlab prompt and
examine the result.

The browser preferences loaded with this command are described in
\sectionref{gui,ref,options}.
Saves the current preferences in a file of your choosing.

\button{Use Defaults}

Selecting this option will restore to their default values:

\begin{itemize}

\item
The size and screen position of the browse windows
\item
The resolution factors (multiples of the dataset stride 
available through the \but{Resolution} menu)
\item
The font size
\item
The color scheme for the user display
\item
The time format (this is year/month/day)
\item
The axis limits
\item
The data size threshold (1Mb)
\item
The color limits (0 255)
\item
The color palette.
\item
The limits of the four browser axes  (time, depth, latitude, longitude).
\end{itemize}

Clicking this button also clears the ranges, dataset and variable
selections.  See \sectionref{gui,ref,options} for more information
about the options.

\end{description}

\menudesc{DODS Help}

This menu allows you to see several short hints addressing frequently
asked questions. You can also select \but{DODS GUI User Manual}, which
will start an html browser on your system, reading a hypertext version
of this document. Your Matlab file, \lit{docopt.m} must be properly
configured to use the appropriate browser. See your system
administrator if it is not set correctly. Similarly, you can edit the
\lit{browsopt.m} file (part of the \GUI) to control the URL submitted
to the browser. The default is to read the copy of the documentation
\xlink{on the DODS server}[: \lit{\footnotesize\DODSmguiUrl}]{\DODSmguiUrl}. 

If you have a local copy of the hypertext documentation, you may want
to refer to that copy instead, to make browsing the text faster.



\menudesc{Display Data}

The data display dialog is described in Section~\ref{mgui,data-disp}.


There are two pull-down menus on the data display dialog:

\subsubsection{Edit}

This menu contains three buttons:

\begin{description}
\button{Add New Plot} Create a new, empty, plot window.
\button{Delete Plot} Delete one of the existing plots.
\button{Delete All Plots} Delete all the existing plots.
\end{description}

\submenudesc{Apply to All}

If you have a set of plots and wish to operate on them all
simultaneously, you can change some aspect of the Display Dialog, and
then select the appropriate button on this menu, and that aspect of
all the plots will be updated.

\menudesc{Bookmarks}

The bookmark editor looks like the bookmark selection window, but with
a different menu bar.  The menu bar has three pull-down menus,
\pdmenu{File}, \pdmenu{Edit} and \pdmenu{View}.

\submenudesc{File}

\figureplace{Bookmarks \pdmenu{File} Menu}{htb!}{gui,ref,filemenu}%
{edit-fmenu.ps}{edit-fmenu.gif}{}

The bookmarks \pdmenu{File} menu looks like
\figureref{gui,ref,filemenu}.  Use it to manage the bookmarks file
itself.  It has the following buttons:

\begin{description}

\button{Open Bookmarks File} Select a different bookmarks file to
edit.
\button{Save As} Save this bookmark file under a different name.
\button{Reset to Master List} Restore the current bookmarks to the
``default'' state, which is defined by the current version of the
master list downloaded from the GUI server.  (See
\sectionref{gui,intro,update}.) 
\button{Apply to GUI} Apply the changes to the \GUI, but don't close
the bookmark edit window.
\button{Apply and Close} Apply the changes to the \GUI, and close
the bookmark edit window.
\button{Cancel} Close the bookmark edit window without applying
the bookmark changes to the \GUI.
\end{description}

\submenudesc{Edit}

\figureplace{Bookmarks \pdmenu{Edit} Menu}{htb!}{gui,ref,editmenu}%
{edit-emenu.ps}{edit-emenu.gif}{}

The bookmarks \pdmenu{File} menu looks like
\figureref{gui,ref,editmenu}.  Use it to manage the bookmarks file
itself.  It has the following buttons:


\begin{description}
\button{Undo} Allows you to undo bookmark edits.  You can undo either
the last change you made, or all the changes since you last applied
changes to the \GUI.
\button{Cut} Cut the selected bookmark or range of bookmarks.  Store
them in an invisible buffer, to be pasted if desired.
\button{Copy} Copy the selected bookmark or range of bookmarks to an
invisible buffer, to be pasted if desired.
\button{Paste} Copy any contents of the invisible copy buffer to the
bookmark list.
\button{Select All} Select all the bookmarks in the list.
\button{Properties} Display the properties of the highlighted
bookmark.  See \figureref{gui,intro,bookprop}.
\button{Insert New} Insert a new bookmark or bookmark folder into the
bookmark list.  See \sectionref{gui,intro,editbook}.
\end{description}

\submenudesc{View}

There is only one option under the \pdmenu{View} menu, which is
\but{Master List}.  Selecting this button provides a view of the
current version of the master list, from which entries can be copied
and pasted.

\section{Browser Preferences}
\label{gui,ref,options}

The \GUI\ uses a set of preferences to control several aspects of its
use, including selection ranges, font sizes, colors and so on.  These
preferences can be loaded with the \but{Preferences,Load File}
button, and saved with the \but{Preferences,Save} or \but{Preferences,
  Save As} buttons
(See \pagexref{gui,ref,saveopt-desc}).  These commands load and save a
Matlab script whose function is to define the values of the various
preferences.  You may find it convenient to edit the file by hand; it
is only an ASCII file of Matlab commands.

If the preferences file is called \lit{browsopt.m}, and is stored
somewhere in the Matlab path, it will be loaded automatically when the
browser starts up.

Following is a list of all the options a preference file may contain.

\begin{description}
\item[\lit{timetoggle}] This option (controlled with the
  \but{Display,Time} button) controls whether times are displayed as
  Year/Month/Day or as the Year/Yearday, where Yearday is the ordinal
  number of the day of that year.
\item[\lit{dods\_colors}] This is the set of colors, expressed in RGB
  values in a $10\times3$ array, used to display various aspects of
  the browser window.  The colors are used as follows:
\begin{description}
\item[\lit{dods\_colors[1]}] Default button color.
\item[\lit{dods\_colors[2]}] Axis and window background.
\item[\lit{dods\_colors[3]}] Axis and window foreground.
\item[\lit{dods\_colors[4]}] Non-interactive window and axis text. 
\item[\lit{dods\_colors[5]}] Edit boxes, where users may type.
\item[\lit{dods\_colors[6]}] Edit box text and button labels.
\item[\lit{dods\_colors[7]}] The \but{Get Data!} button.
\item[\lit{dods\_colors[8]}] Text and axis lines that change.
\item[\lit{dods\_colors[9]}] The ``whizzo'' color, used for drawing
  attention to special options.  The \pdmenu{Help} is displayed in
  this color, among others.
\item[\lit{dods\_colors[10]}] The selection range box color.
\end{description}
\item[\lit{fontsize}] The point size of the displayed type.
\item[\lit{figsizes}] The default size and position of the browser
  window, expressed in the pixel locations of the upper left and lower
  right corners.
\item[\lit{axes\_vals}] The starting axis values for the displayed
  axes, expressed four pairs of minimum and maximum values for (in
  this order), longitude, latitude, depth, and time:

\begin{vcode}{xib}
axes_vals=[minLon maxLon minLat maxLat minDepth maxDepth minTime maxTime]    
\end{vcode}

\item[\lit{ranges}] Specifies the default selection ranges, in the
  same manner as the \lit{axes\_vals} preference.
\item[\lit{num_rang}] An eight element vector of ones and zeros,
  indicating which of the selection ranges have been set by the user.
  A one means the corresponding item in the \lit{ranges} variable has
  been set by the user.  If not all the ranges are set, you can't `get
  data' or `get details'.  If a dataset is selected, and the depth or
  geographic ranges have not been set, these are set to those of the
  dataset.
\item[\lit{color\_limits}] A two element vector indicating the data
  values to assign to the top and bottom colors in the data palette.
\item[\lit{palettefile}] The name of a file containing the data
  palette.  This file must be in the Matlab path.
\item[\lit{check\_level}] This is the number of megabytes for the
  ``Get Data'' threshold. (Set with the \but{Preferences,Get Data
    Threshold} button, see \pagexref{gui,ref,threshold-desc} for more
  information.)
\item[\lit{available\_resolutions}] This is a vector containing the
  available resolutions offered with the \pdmenu{Resolutions} menu,
  expressed as multiples of the dataset's basic resolution.  The
  default value is:

\begin{vcode}{sib}
available_resolutions = [32 24 16 12 8 4 3 2 1];
\end{vcode}
  
  This option is not changable through the browser, but you can edit
  the preferences file directly to change it.
\item[\lit{manURL}] The URL of the \GUI\ manual.  By default, this
  points to \xlink{\lit{\footnotesize\DODSmguiUrl}}{\DODSmguiUrl}.  If
  you have a local copy of the manual served, you can substitute that
  URL.

\item[\lit{guiserver}] The URL to use for updates to the dataset
  metadata files.

\item[\lit{userlist_file}] The name and location (on the local file
  system) of the user's bookmarks file.

\end{description}


% $Log: ch02.tex,v $
% Revision 1.8  2004/12/09 03:04:14  tomfool
% updated for GUI 6
%
% Revision 1.7  2002/02/10 02:59:47  tom
% updated for GUI version 5 and dods-book2 templates
%
% Revision 1.6  2000/10/04 15:01:37  tom
% changed \figureplace definition
%
% Revision 1.5  1999/05/25 20:47:46  tom
% modifications for DODS release 3.0
%
% Revision 1.4  1999/02/04 17:42:13  tom
% modified to use dods-book.cls and Hyperlatex
%
% Revision 1.3  1999/01/20 15:18:39  tom
% updated, and changed for dods-book.cls and hyperlatex
%
% Revision 1.2  1998/12/07 15:41:15  tom
% updated for DODS v2.19 and GUIv0.7
%
% Revision 1.1  1998/02/12 18:10:41  tom
% added to CVS archive
%
%%% Local Variables: 
%%% mode: latex
%%% TeX-master: t
%%% End: 
