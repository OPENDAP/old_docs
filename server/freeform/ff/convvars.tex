%CHAPTER 7       
%
% $Id$
%

\chapter{Conversion Variables}
\label{ff,convvars}

\indc{conversion variables!overview}
Space and time values such as latitude and longitude, date, and time
of day can be represented in various ways. For example, latitude and
longitude can be given in degrees and minutes, or as floating point
numbers (among other possibilities). FreeForm ND conversion variables
make it possible to translate between a number of representations of
space and time values. You tell FreeForm ND that a conversion is
needed by including the appropriate standard conversion variable name
in the relevant format description file. When FreeForm ND reads a
format description file and finds a conversion variable, it
automatically performs the indicated conversion.

\section{Accessing Conversion Functions}

\indc{accessing conversion functions}
\indc{conversion functions!accessing} 
\indc{calculated!output variable}\indc{output variable!calculated}
\indc{variable!calculated output}
FreeForm ND's conversion functions are invoked by using standard
conversion variable names in the input and output format descriptions.
FreeForm ND attempts a conversion only if the input and output names
for a variable differ, and both names are included in FreeForm ND's
list of standard conversion variables (see \chapterref{ff,varname}).
If a variable name in an output format does not correspond to a name
in the input format, FreeForm ND searches the input variables for
standard conversion variable names.

For example, assume the following variable is described in the input
format description:

\begin{vcode}{ib}
latitude 1 10 double 6 
\end{vcode}

The output format description includes the following variable
descriptions, but not one for latitude:

\begin{vcode}{ib}
latitude_deg 1 7 short 0
latitude_min 13 15 short 0
latitude_sec 21 23 short 0
\end{vcode}

FreeForm ND will transparently identify and call conversion functions
to construct the specified output values (latitude in units of
degrees, minutes, and seconds) using the input value given by the
variable latitude.

\section{Latitude and Longitude Conversions}
\label{ff,conv,latlon}

\indc{conversion!degrees, minutes, seconds}
\indc{degrees, minutes, seconds!conversion}
\indc{latitude!conversion}\indc{conversion!latitude}
\indc{longitude!conversion}\indc{conversion!longitude}
Space is often delineated by latitude and longitude in geophysical
applications. Latitude and longitude values can be represented most
directly as floating point numbers, but often are not. Data sets
frequently give latitude and longitude in other representations such
as degrees and minutes, or absolute value of degrees, decimal minutes,
and N/S or E/W to designate hemisphere.

FreeForm ND includes a set of functions that perform conversions
between a number of the most common representations of latitude and
longitude. In order to access these conversions, you must use the
standard variable names listed in \tableref{ff,tab,geo}.
\indc{abs@\lit{\_abs}}\indc{deg@\lit{\_deg}}
\indc{min@\lit{\_min}}\indc{sec@\lit{\_sec}}\indc{sign@\lit{\_sign}}
\indc{ns@\lit{\_ns}}\indc{ew@\lit{\_ew}}
\indc{geographic variables!conversion}
\indc{conversion!geographic variables}

\begin{table}[htb]
\caption{Geographic Variable Conversion Options}
\label{ff,tab,geo}
\begin{tabular}{|p{1.3in}|p{2.5in}|p{0.8in}|} \hline
\tblhd{Name} & \tblhd{Description} & \tblhd{Example Value} \\ \hline
\lit{latitude}
\lit{longitude} & 
Signed floating point number that completely describes a latitude or
longitude coordinate value & 
\parbox[t]{0.8in}{-47.583333
\T\\
-176.75} \\ \hline
\lit{latitude\_abs}
\lit{longitude\_abs} & 
Absolute value of a latitude or longitude coordinate (may include
fractions of a degree) & 
47.583333
176.75 \\ \hline
\lit{latitude\_deg}
\lit{longitude\_deg} & 
Degrees component of a latitude or longitude coordinate value (may be
signed) & 
\parbox[t]{0.8in}{-47
\T\\
-176} \\ \hline
\lit{latitude\_deg\_abs}
\lit{longitude\_deg\_abs} & 
Absolute value of the degrees component of a latitude or longitude
coordinate & 
\parbox[t]{0.8in}{47 
\T\\
176} \\ \hline
\lit{latitude\_min}
\lit{longitude\_min} & 
Minutes component of a latitude or longitude coordinate value & 
\parbox[t]{0.8in}{30.5
\T\\
45.0} \\ \hline
\lit{latitude\_sec}
\lit{longitude\_sec} & 
Seconds component of a latitude or longitude coordinate value & 
\parbox[t]{0.8in}{30.0
\T\\
0.0} \\ \hline
\lit{latitude\_sign}
\lit{longitude\_sign} & 
Sign of a latitude or longitude coordinate value 
+ or - (data type is char) & 
- \\ \hline
\lit{latitude\_ns}
\lit{longitude\_ew} & 
Hemisphere: N for north, S for south, E for east, W for west (char) & 
S \\ \hline
\lit{geog\_quad\_code} & 
A geographic quadrant defined by DMA (Defense Mapping Agency), 1 = NE,
2 = NW, 3 = SE, 4 = SW (char) & 
4 \\ \hline 
\end{tabular}
\end{table}


\indc{conventions!geographic}\indc{geographic conventions}
\indc{latitude!north positive}\indc{longitude!east positive}
FreeForm ND uses the convention that northern latitudes and eastern
longitudes are positive.

\subsection{Degrees, Minutes, and Seconds}
\label{ff,convvars,dms}


\indc{conversion!degrees, minutes, seconds!example}
\indc{degrees, minutes, seconds!conversion!example}
\indc{latitude!conversion!example}
\indc{conversion!latitude!example}
\indc{longitude!conversion!example}
\indc{conversion!longitude!example}
In this example you will convert latitude and longitude values in
\lit{latlon2.bin} from long integers (with implied precision) to
latitude and longitude values given in degrees, minutes and seconds.
The binary file \lit{latlon2.bin} was created earlier from the ASCII
file \lit{latlon.dat} (see \chapterref{ff,quick}). The input and
output formats are described in \lit{ll\_d\_m\_s.fmt}. Conversion
variable names are included in the input and output formats.

Here are the contents of the \lit{ll\_d\_m\_s.fmt} file:

\begin{vcode}{ib}
/ This is the format description file for the data files latlon2.bin and
/ ll_d_m_s.dat. Each record of the input binary file latlon2.bin contains 
/ two fields, latitude and longitude. These values are stored as integers.
/ Each record of the output ASCII file ll_d_m_s.dat contains latitude and 
/ longitude given in units of degrees, minutes, and seconds.

binary_data "binary input format"
latitude 1 4 long 6
longitude 5 8 long 6

ASCII_data "ASCII output format"
latitude_deg 1 7 short 0
latitude_min 13 15 short 0
latitude_sec 21 23 short 0
longitude_deg 27 31 short 0
longitude_min 37 39 short 0
longitude_sec 45 47 short 0 
\end{vcode}

To convert the data to the new ASCII format use the following command: 

\begin{vcode}{ib}
newform latlon2.bin -f ll_d_m_s.fmt -o ll_ d_m_s.dat 
\end{vcode}

The ASCII file \lit{ll\_d\_m\_s.dat} is created with the 20 latitude and
longitude values given in degrees, minutes, and seconds. If a degree
value is between 0 and -1, then either the minute or second value is
signed. When you view \lit{ll\_d\_m\_s.dat}, you should see the following
values:

\begin{vcode}{ib}
         1         2         3         4         5
12345678901234567890123456789012345678901234567890
    -47      18      13    -176       9      40
      0     -55      41       0      46      38
    -28      17      12      35      35      31
     12      35      18     149      24      29
    -83      13      25      55      19      11
     54       7       6    -136      56      26
     38      49       8      91      24      41
    -34      34      37      30      10      20
     27      19      54    -155      14       1
                .
                .
                . 
\end{vcode}

You can convert the data file \lit{ll\_d\_m\_s.dat} back to its original ASCII
format (in \lit{latlon.dat}) but the values will be somewhat different than
those in \lit{latlon.dat}. The ASCII format for \lit{ll\_d\_m\_s.dat} uses whole
seconds, which are not precise enough to represent decimal degrees to
six decimal places. Fractional seconds are required to preserve the
values of decimal degrees to six places. If ushort variables with a
precision of 3 were specified in \lit{ll\_m\_d\_s.fmt}, fractional seconds
could be represented.

\subsection{Absolute Degrees and Minutes}

In the following two examples, you will create new ASCII data files
from \lit{latlon2.bin} that give latitude and longitude in absolute degrees
and minutes with hemisphere indicated in the first case and geographic
quadrant in the second case. Conversion variable names are used in the
input and output formats in both examples.

\subsubsection{With Hemisphere}

You will convert the data in \lit{latlon2.bin} to latitude and longitude
values given in absolute degrees and minutes. FreeForm ND converts the
sign (+ or -) of the input data to N for north, S for south, E for
east, or W for west as appropriate. Southern latitudes and western
longitudes are negative. The input and output formats are described in
\lit{degabsm.fmt}.

Here are the contents of \lit{degabsm.fmt} :

\begin{vcode}{ib}
/ This is the format description file for the data files latlon2.bin and
/ degabsm.dat. Each record of the input binary file latlon2.bin contains 
/ two fields, latitude and longitude. These values are stored as integers.

binary_data "binary input format"
latitude 1 4 long 6
longitude 5 8 long 6

/ Each record of the output ASCII file degabsm.dat contains latitude and 
/ longitude given in units of absolute degrees and minutes. The hemisphere
/ is indicated by the variables latitude_ns and longitude_ew. The value can be
/ the character N for north, S for south, E for east, or W for west.

ASCII_data "ASCII output format"
latitude_deg_abs 6 7 short 0
latitude_min_abs 10 15 float 2
latitude_ns 17 17 char 0
longitude_deg_abs 24 26 short 0
longitude_min_abs 28 34 float 3
longitude_ew 36 36 char 0
\end{vcode}

To convert the data to absolute degrees and minutes with hemisphere included, use the following command:

\begin{example}
newform latlon2.bin -f degabsm.fmt -o degabsm.dat 
\end{example}

When you view \lit{degabsm.dat}, you should see the following values: 

\begin{vcode}{ib}
         1         2         3         4
1234567890123456789012345678901234567890
     47  18.21  S      176   9.666 W
      0  55.68  S        0  46.636 E
     28  17.20  N       35  35.513 E
     12  35.29  N      149  24.487 E
     83  13.41  S       55  19.176 E
              .
              .
              . 
\end{vcode}

\subsubsection{With Quadrant}

You will convert the data in \lit{latlon2.bin} to latitude and longitude
values given in absolute degrees and minutes with the geographic
quadrant indicated by a character code. The input and output formats
are described in \lit{degmgeog.fmt}.

Here is \lit{degmgeog.fmt}:

\begin{vcode}{ib}
/ This is the format description file for the data files latlon2.bin and
/ degmgeog.dat. Each record of the input binary file latlon2.bin contains 
/ two fields, latitude and longitude. These values are stored as integers.

binary_data "binary input format"
latitude 1 4 long 6
longitude 5 8 long 6

/ Each record of the output ASCII file degmgeog.dat contains latitude and 
/ longitude given in units of absolute degrees and minutes. The
/ geographic quadrant of the data is indicated by a numeric character code.
/
/ 1 = Northeast
/ 2 = Northwest
/ 3 = Southeast
/ 4 = Southwest
/

ASCII_data "ASCII output format"
latitude_deg_abs 6 7 short 0
latitude_min_abs 10 15 float 2
longitude_deg_abs 21 23 short 0
longitude_min_abs 26 31 float 2
geog_quad_code 40 40 char 0
\end{vcode}

To convert the data to absolute degrees and minutes with quadrant, use
the following command:

\begin{example}
newform latlon2.bin -f degmgeog.fmt -o degmgeog.dat
\end{example}

When you view \lit{degmgeog.dat}, you should see the following values: 

\begin{vcode}{ib}
         1         2         3         4
123456789012345678901234567890123456789012345
     47   18.21     176    9.67        4
      0   55.68       0   46.64        3
     28   17.20      35   35.51        1
     12   35.29     149   24.49        1
     83   13.41      55   19.18        3
              .
              .
              . 
\end{vcode}

\section{Date and Time Conversions}
\label{ff,convvars,date}

\indc{conversion!time}\indc{time!conversion}
\indc{conversion!date}\indc{date!conversion}
Time is a variable found in many scientific data sets and it can have
various representations. In ASCII formats meant to be read by
application users, time is often represented with six variables: year,
month, day, hour, minute, second. In formats meant to be read by
computers, it makes sense to represent time as a floating point number
in days and decimal fractions of a day, or perhaps seconds and
fractions of a second.

FreeForm ND can perform conversions between various representations of
dates when standard conversion variable names are included in the
format descriptions. Several examples are given below.

\subsection{Year, Month, Day}

In this example you will convert a date string in the form of
month/day/year to a date string in the form of year, month, day with
no separators. The format description file \lit{yymmdd.fmt} describes the
input and output formats and the input data is stored in \lit{mdy.dat}.
Notice that this is a conversion from one ASCII data format to
another.

Here is \lit{yymmdd.fmt}:

\begin{vcode}{ib}
/ This is the format description file for the data files mdy.dat and
/ yymmdd.dat.

ASCII_input_data "ASCII input format"
date_mm/dd/yy 1 10 char 0

ASCII_output_data "ASCII output format"
date_yymmdd 1 12 char 0
\end{vcode}

Try this on the example file \lit{mdy.dat}:

\begin{vcode}{ib}
 1/26/20
 7/25/78
11/19/99
     .
     .
     . 
\end{vcode}

To convert the data in \lit{mdy.dat} from m/d/y format to yymmdd format, use
the following command:

\begin{example}
newform mdy.dat -f yymmdd.fmt -o yymmdd.dat 
\end{example}

The resulting file \lit{yymmdd.dat} will contain the following values: 

\begin{vcode}{ib}
200126
780725
991119
  .
  .
  . 
\end{vcode}

\subsection{Serial Dates}

\indc{conversion!date!serial}\indc{serial date!conversion}
\indc{date!conversion!serial}
If you have time data in an ASCII format and the data will be read
primarily by an application, you may want to convert it to a binary
format. FreeForm ND supports a binary representation of time as a
serial day starting at \ind{January 1, 1980}.

FreeForm ND conversion functions let you convert from an ASCII
representation to the binary serial date representation. As an
example, you will convert the ASCII data in \lit{time.dat}, which contains
10 random times from this century, to a binary serial date format in
\lit{serial.bin}. The format description file \lit{serial.fmt} describes the input
and output formats for \lit{time.dat} and \lit{serial.bin}. It also contains a
format description for \lit{serial.dat}, which will contain the data in
\lit{serial.bin} in an ASCII format.

Here is the file \lit{serial.fmt}:

\begin{vcode}{ib}
/ This is the format description file for the data files time.dat, serial.bin, 
/ and serial.dat. Each record of the ASCII file time.dat contains six 
/ fields: year, month, day, hour, minute, second.

ASCII_data "ASCII ymdhms date"
year 2 5 ushort 0
month 10 11 uchar 0
day 19 20 uchar 0
hour 28 29 uchar 0
minute 37 38 uchar 0
second 43 47 float 2

/ Each record of the binary file serial.bin contains one field, 
/ serial date, defined as a double that occupies 8 bytes and has 
/ 8 places of precision.

binary_data "binary serial date"
serial_day_1980 1 8 double 8
/ Each record of the ASCII file serial.dat contains one field, 
/ serial date.

ASCII_data "ASCII serial date"
serial_day_1980 1 16 double 8
\end{vcode}

Here is an example file, \lit{time.dat}:

\begin{vcode}{ib}
1920     1       26       11       26    49.79
1978     7       25        1       36    14.89
1999    11       19       14        4     4.78
                    .
                    .
                    . 
\end{vcode}

To convert the dates from the ASCII format in \lit{time.dat} to the binary
serial date format, use the following command:

\begin{example}
newform time.dat -f serial.fmt -ift "ASCII ymdhms date"  -o serial.bin 
\end{example}

Then view the binary file \lit{serial.bin} with either of the following
commands:

\begin{example}
newform serial.bin -oft "ASCII serial date" -o serial.dat 
\end{example}

or 

\begin{example}
readfile serial.bin 
\end{example}

You should see the following values: 

\begin{vcode}{ib}
-21889.52303484
  -524.93316100
  7262.58616644
-20525.28111250
  5046.80073889
  .
  .
  .
\end{vcode}
 
\section{Calculated Input Variables}
\label{ff,convvars,calcinp}

\indc{input variable!calculated}\indc{variable!calculated input}
\indc{calculated!input variable}
Calculated input variables are very similar to calculated output
variables; both perform arithmetic on the data being processed.
However, calculated input variables are processed immediately after
data is read, and calculated output variables are processed after
input data has been converted.  Often this distinction is unimportant,
and calculated output variables should be used whenever possible.  Yet
there are times you must use, or would prefer to use, calculated input
variables.  Here are two examples.

\subsection{Deriving Record Count}

A data file contains record headers, but the record headers contain
the record number of the current and the next record header. FreeForm
ND requires that the number of data records after each record header
be defined (or derived).  This format file shows how a calculated
input variable is used to derive the number of data records that
follow each record header.

\begin{vcode}{ib}
binary_input_file_header "File Header"
stuff 1 24 text 0

/ equation to determine count:
/ next-head-r# - this-head-r# - (2160 / 24) 
/ note: 2160 / 24 is # records occupied by record header
/ next-head-r# - this-head-r# - 90

input_eqv
begin constant
this_header_record_number_eqn text [next_header_record_number]-[this_header_record_number]-90
end constant

begin name_equiv
count this_header_record_number
end name_equiv

binary_input_record_header "Record Header"
this_header_record_number_eqn 1    4 long 0
next_header_record_number     5    8 long 0
extra_stuff                   9 2160 text 0

binary_input_data "Inventory Points"
stuff 1 24 text 0

binary_output_data "Inventory Points"
stuff 1 24 text 0
\end{vcode}

In this example, the file header, record header, and data are read,
but only the data is output.  When calculating the number of data
records we subtract the number of records in the record header (the
record header is 2160 bytes = 90 records $\times$ 24 bytes/record).
The \lit{\_eqn} suffix on the variable named
\lit{this\_header\_record\_number} indicates that this is a calculated
variable (just as with calculated output variables).  The equation for
this variable is defined in an input equivalence section. Furthermore,
a name equivalence between count and
\lit{this\_header\_record\_number} is required.  Alternatively, we
could rename \lit{this\_header\_record\_number\_eqn} to
\lit{count\_eqn}.

Note that the calculated value for \lit{this\_header\_record\_number}
overwrites (replaces) the value read from the file for
\lit{this\_header\_record\_number}.  In this case, we are not writing
record headers so it doesn't matter. However, if we were writing
record headers and wanted to preserve the original meaning of this
variable, we could ``undo'' the calculation with a calculated output
variable.

\subsection{A Year 2000 Solution}

\indc{y2k!date conversion}\indc{date conversion!y2k}
A calculated input variable can be used to make a two-digit date field
appear as a four-digit date field (or more digits, up to about 15,
which is the maximum allowed by a double-precision float).  Here is a
format file from Year 2000 Compliant Data, rewritten to use calculated
input variables:

\begin{vcode}{xib}
input_eqv
begin constant
year_eqn char 1900 + [year]
end constant

ASCII_input_data        "Year without Century, but seems as if it did"
year_eqn           1  2   uint8 0
month              3  4   uint8 0
day                5  6   uint8 0
hour               8  9   uint8 0
minute            10 11   uint8 0
second            12 17 float32 2
latitude_deg_abs  18 20   uint8 0
latitude_ns       21 21    text 0
latitude_min      22 26 float32 2
longitude_deg_abs 27 30   uint8 0
longitude_ew      31 31    text 0
longitude_min     32 36 float32 2
depth             37 43 float32 2
magnitude_local   44 50 float32 2
spaces            51 79    text 0

ASCII_output_data       "no calculated output variables here"
year               1  4  uint16 0
month              5  6   uint8 0
day                7  8   uint8 0
hour              10 11   uint8 0
minute            12 13   uint8 0
second            14 19 float32 2
latitude_deg_abs  20 22   uint8 0
latitude_ns       23 23    text 0
latitude_min      24 28 float32 2
longitude_deg_abs 29 32   uint8 0
longitude_ew      33 33    text 0
longitude_min     34 38 float32 2
depth             39 45 float32 2
magnitude_local   46 52 float32 2
spaces            53 81    text 0 
\end{vcode}

Almost immediately after the data is read the year field is
transmogrified from an uint8 into a float64 field, and each year value
has 1900 added to it. The difference between using calculated input
variables and calculated output variables to solve the Year 2000
problem for data is like the difference between applications which
read data and the data itself.  In the above example, the data is not
Y2K compliant, but thanks to the above format file, any FreeForm ND
application will think it is.  This avoids mass conversion of data
(not a trivial task), but the data remains in noncompliance with
respect to any software except FreeForm ND.




%%% Local Variables: 
%%% mode: latex
%%% TeX-master: t
%%% End: 
