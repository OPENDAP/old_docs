%CHAPTER 11       
%
% $Id$
%

 
\chapter{Conversion Variable Names}
\label{ff,varname}

\indc{conversion variables!names}
FreeForm ND can automatically perform conversions between various
representations of space and time values. When FreeForm ND encounters
standard conversion variable names in a format description file, it
performs the appropriate conversion.

This appendix lists the conversion variable names that FreeForm ND
recognizes. For conversions that occur in one direction only, the
conversion variable names are listed in columns titled Input and
Output. Conversion names that you can use for either input or output
variables are in untitled columns.

\section{General}

By adding the appropriate suffix to a variable name, you can perform
several general conversions. To convert between meters
and feet, add \lit{\_m} and \lit{\_ft} to the relevant variable
names.  That is, you should add \lit{\_m} to the input variable name
and \lit{\_ft} to the output name or vice versa.

\indc{m@\lit{\_m}}\indc{meters!conversion variable}
\indc{ft@\lit{\_ft}}\indc{feet!conversion variable}
\indc{abs@\lit{\_abd}}\indc{absolute value!conversion variable}
\indc{sign@\lit{\_sign}}\indc{sign!conversion variable}
\indc{enote@\lit{\_enote}}\indc{exponential notation!reading}
\indc{scientific notation!reading}
For absolute and signed values, add \lit{\_abs} and \lit{\_sign}. The
sign value is a single \emph{character}, either `+' or `-'.  For
exponential notation, you can use \lit{\_enote}.  The range of the
exponent is E+/-999. Be sure the output field is large enough for the
converted number.

%\begin{table}[h]
%\caption{General Variable Name Conversions}
\begin{center}
\begin{tabular}{ll}
$\begin{array}{l} 
\var{name}\lit{\_m}
\end{array}$
& \var{name}\lit{\_ft} \\
$\left. \begin{array}{l} 
\var{name}\lit{\_abs} \\
\var{name}\lit{\_sign} \\
\end{array}
\right\} $
& \var{name} \\
$\left. \begin{array}{l}
\var{name}\lit{\_enote}
\end{array}
\right\} $
& \var{name}\\ 
\end{tabular}
\end{center}
%\end{table}

Where \var{name} = any character string without blanks.  Note,
however, that using \lit{\_ft} and \lit{\_m} with special names, such
as \lit{latitude} or \lit{year}, may create unintended consequences or
may not work.

\section{Latitude/Longitude}

FreeForm ND supports conversions between a number of representations
of latitude and longitude values with the use of the correct
conversion variable names.

\subsection{General Lat/Lon}

By using the appropriate suffixes, you can perform conversions between
a number of different representations of latitudes and longitudes.

\indc{deg@\lit{\_deg}}
\indc{min@\lit{\_min}}
\indc{sec@\lit{\_sec}}
\indc{abs@\lit{\_abs}}
\indc{ns@\lit{\_ns}}
\indc{ew@\lit{\_ew}}
\ffnd\ supports a variety of expressions of longitude and latitude.
You can use these suffixes to convert back and forth between any of
them.  For example, you can convert from decimal degrees, with a sign
(\var{name}) to degrees, minutes and seconds (\var{name}\lit{\_deg},
\var{name}\lit{\_min} and \var{name}\lit{\_sec}).  The
conversion would happen automatically when one variable appears in the
input format description, and the other three in the output format
description.  The conversions also work in the opposite direction, and
you can input degrees, minutes, and seconds, and write decimal degrees
to output.

In the following, \var{name} can be \lit{latitude} or \lit{longitude}.

\begin{tabular}{lp{3.0in}} 
\begin{tabular}{l}
\var{name}
\end{tabular}
& The location in decimal degrees, with a sign
  indicating the quadrant (positive is North and East).
\\ 
\begin{tabular}[b]{l}
  \var{name}\lit{\_deg}\\
  \var{name}\lit{\_min}\\
  \var{name}\lit{\_sec}\\
\end{tabular}
& These are integers representing degrees,
  minutes, and seconds.  The sign of \var{name}\lit{\_deg} indicates
  the quadrant, but if it is between zero and one, the sign of the
  minutes or seconds variable indicates the quadrant.
\\ 
\begin{tabular}[b]{l} 
  \var{name}\lit{\_abs} \\
  \lit{latitude\_ns}\\
  \lit{longitude\_ew}\\ 
\end{tabular}
& This representation is in decimal degrees, but
  without a sign.  The quadrant is indicated by a single character
  (\lit{N}, \lit{S}, \lit{E}, or \lit{W}) contained in
  \lit{longitude\_ew} or \lit{latitude\_ns}
\\ 
\begin{tabular}[b]{l} 
  \var{name}\lit{\_deg\_abs}\\
  \var{name}\lit{\_min\_abs}\\
  \var{name}\lit{\_sec\_abs}
\end{tabular}
& These are integers representing degrees,
  minutes, and seconds.  The quadrant can be derived from
  \lit{longitude\_ew} or \lit{latitude\_ns}
\\ 
\end{tabular}

\subsubsection{Geographic Quadrants}

\indc{geogquadcode@\lit{geog\_quad\_code}}
You can use the following variables to convert from several different
representations of latitude and longitude to a geographic quadrant
defined by DMA (Defense Mapping Agency) for their gravity data. \ffnd\  
stores this code as a single character (\lit{char}).

DMA defines four geographic quad codes as follows: 

\begin{tabular}{|c|l|} \hline
\lit{1} & Northeast \\ \hline
\lit{2} & Northwest \\ \hline
\lit{3} & Southeast \\ \hline
\lit{4} & Southwest \\ \hline
\end{tabular}

You can produce this code with any of the following pairs of input
data:

\begin{itemize}

\item \lit{latitude} and \lit{longitude}

\item \lit{latitude\_ns} and \lit{longitude\_ew}

\item \lit{latitude\_sign} and \lit{longitude\_sign}
\end{itemize}

Note that this conversion is not reversible.  That is, if you have 
\lit{geog\_quad\_code} on input, you cannot produce
\lit{latitude\_sign} and \lit{longitude\_sign}. 

\subsubsection{Longitude East}

\indc{longitudeeast@\lit{longitude\_east}}
\indc{east@\lit{\_east}}
\ffnd\ defines (\lit{longitude\_east}) as longitude where the values
range from 0 to 360, instead of -180 to 180 for \lit{longitude}.
\ffnd\ can freely convert one form to the other, including the other 

Either form can stand as input or output, and you can convert freely
with the other longitude formats, such as the degrees, minutes,
seconds, and so on.

\subsubsection{Quadrant, Sign}

The \var{name}\lit{\_sign} form for longitude and latitude converts
readily back and forth into the corresponding \lit{\_ns} and
\lit{\_ew} variables.  Note that the sign variable is a single
character (`+' or `-'), as are the quadrant variables (`N', `S', `E',
and `W').

\section{Earthquake Magnitude}

FreeForm ND includes a conversion function that lets you extract one
of the three magnitudes out of the variable \lit{longmag}, or create
\lit{longmag} from one, two, or more of the individual magnitudes.

\indc{longmag@\lit{longmag}}
\indc{ms1@\lit{ms1}}\indc{ms2@\lit{ms2}}\indc{mb@\lit{mb}}
\indc{earthquake magnitude}\indc{magnitude!earthquake}
The variable \lit{longmag} is a long which contains three magnitudes: 

\begin{description}
\item[\lit{ms2}] has a precision of 2 and is multiplied by 10,000,000
\item[\lit{ms1}] has a precision of 2 and is multiplied by 10,000
\item[\lit{mb}]  has a precision of 1 and is multiplied by 10
\end{description}

The variable names are as follows:

\begin{center}
\begin{tabular}{ll}
\lit{longmag} &
$\left\{ \begin{array}{l} 
\lit{magnitude\_mb} \\
\lit{magnitude\_ms1} \\
\lit{magnitude\_ms2} \\
\lit{magnitude\_ml} \\
\lit{magnitude\_local} \\
\end{array}
\right. $ \\
\end{tabular}
\end{center}

\section{Date and Time}

\indc{date!conversion}\indc{time!conversion}
\indc{yeardecimal@\lit{year\_decimal}}\indc{decimal years!conversion}
\indc{serial day}\indc{day!serial}\indc{date!serial day}
\indc{ipedate@\lit{ipe\_date}}\indc{minutes!AD}\indc{date!in minutes}
\indc{Institute of Physics of the Earth!date format}
FreeForm ND includes conversion functions that let you convert between
various representations of date and time including decimal year;
serial date with January 1, 1980 as 0; any combination of year, month,
day, hour, minute, second; and IPE (Institute of Physics of the Earth)
date in minutes AD.

The following three date representations may be freely converted among
each other:

\begin{tabular}{lll}
\begin{tabular}{l}
\lit{year} \\
\lit{month} \\
\lit{day} \\
\lit{hour} \\
\lit{minute} \\
\lit{second} \\ 
\end{tabular}
&
\begin{tabular}{l}
\lit{year\_decimal} \\
\end{tabular}
&
\begin{tabular}{l}
\lit{serial\_day\_1980} \\
\end{tabular}
\end{tabular}

In addition, \lit{serial\_day\_1980} and \lit{year}, \lit{month},
\lit{day}, \lit{hour}, \lit{minute}, \lit{second} may be converted
\emph{into} \lit{ipe\_date} (though not vice versa).

You can also convert back and forth between different string
representations of time and date: \lit{date\_dd/mm/yy} and 
\lit{date\_ddmmyy}, and \lit{time\_hhmmss} and 
\lit{time\_hh:mm:ss}.


%%% Local Variables: 
%%% mode: latex
%%% TeX-master: t
%%% End: 
