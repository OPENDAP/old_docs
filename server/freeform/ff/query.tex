%CHAPTER 13     
%
% $Id$
%

\chapter{Query Syntax}
\label{ff,query}

This appendix lists the operators, symbols, and functions you can use
to construct queries. The lists are followed by definitions, rules for
combining elements to form equations and queries, and brief usage
explanations.

\section{Symbols and Operators}

Operators cause the indicated operation to be performed on two values
(string or numeric) with a third value resulting. The format for using
operators is as follows:

\begin{example}
\var{value1 operator value2}
\end{example}

\subsection{Arithmetic Operators}

Arithmetic operators cannot be used with string variables or string
constants.

\begin{center}
\begin{tabular}{|l|l|p{2in}|} \hline
\tblhd{Symbol} & \tblhd{Meaning} &
\tblhd{Explanation} \\ \hline   
\circflex & exponentiation & Raises \var{value1} to the \var{value2}
power. Generates a domain error if \var{value1} is negative and
\var{value2} is not an integer.\\ \hline  
\% & modulus & Returns the remainder when \var{value1} is divided by
\var{value2.} If \var{value1} = \var{value2} * \var{a} + \var{R},
where \var{a} is an integer and \var{R} is less than \var{value2}, the
modulus operator returns \var{R}. Generates a domain error if
\var{value2} is 0.\\ \hline  
* & multiplication & Multiplies \var{value1} by \var{value2}.\\ \hline  
/ & division & Divides \var{value1} by \var{value2}. Generates a
domain error if \var{value2} is 0.\\ \hline  
+ & addition & Adds \var{value1} and \var{value2}.\\ \hline 
- & subtraction & Subtracts \var{value2} from \var{value1}.\\ \hline 
\end{tabular}
\end{center}

\subsection{Logical Operators}

Inputs to the logical operators are evaluated to FALSE if they are
equal to 0. Any value other than 0 evaluates to TRUE. Outputs of the
logical operators are 0 for FALSE, 1 for TRUE.

Some of the logical operators have a ``word'' which is synonymous with
their symbol (C language compatible), or multiple acceptable symbols.
There is no advantage in speed to using any one of these alternate
names, but the equation may be more human-readable in one form than in
another. For example, the following equations, which evaluate to TRUE
(i.e., 1) if the variables x and y are TRUE, are all equivalent:

\begin{example}
[x] & [y]
[x] && [y]
[x] and [y]
\end{example}

\begin{longtable}{|p{0.5in}|p{1in}|p{3in}|} 
\caption{Functions}
\T\\ \hline
\tblhd{Symbol} & \tblhd{Meaning} &
\tblhd{Explanation}\\ \hline   
\endfirsthead
\& & logical AND & TRUE if \var{value1} and \var{value2} are both TRUE\\  
\&\& & & \\
and & & \\ 
& & Logical AND accepts only
    numeric arguments. Here is the truth table for logical AND:
\begin{center}
\begin{tabular}{ccc} 
\var{value1} & \var{value2} & \var{output} \\ 
FALSE &  FALSE &  FALSE \\
FALSE &  TRUE  &  FALSE \\
TRUE  &  FALSE &  FALSE \\
TRUE  &  TRUE  &  TRUE \\
\end{tabular} \end{center}
\\ \hline

$|$ & logical OR & TRUE if \var{value1} or \var{value2} are TRUE\\ 
$||$ & & \\
or & & \\

& & Logical OR accepts only numeric arguments. 

Truth table for logical OR: 

\begin{center}
\begin{tabular}{ccc}
\var{value1} & \var{value2} & \var{output} \\
FALSE &  FALSE &  FALSE \\
FALSE &  TRUE  &  TRUE \\
TRUE  &  FALSE &  TRUE \\
TRUE  &  TRUE  &  TRUE \\ 
\end{tabular} 
\end{center}
\\ \hline

x$|$ & logical exclusive or (XOR) & TRUE if \var{value1} or \var{value2} are TRUE, but not both\\  
xor & & \\

& & Logical XOR accepts only numeric arguments 

Truth table for logical XOR: 

\begin{center}
\begin{tabular}{ccc}
\var{value1} & \var{value2} & output \\
FALSE &  FALSE &  FALSE \\
FALSE &  TRUE  &  TRUE \\
TRUE  &  FALSE &  TRUE \\
TRUE  & TRUE  &  FALSE \\
\end{tabular}
\end{center}
\\ \hline  

= & equal to  & TRUE if \var{value1} is equal to \var{value2} \\ 
== & & \\

& & This operator can be used with both numeric and string values as long
as \var{value1} and \var{value2} are both of the same type.\\ \hline  

$<$ & less than & TRUE if \var{value1} is less than \var{value2}

This operator can be used with both numeric and string values, as long
as \var{value1} and \var{value2} are both of the same type.\\ \hline 

$>$ & greater than & TRUE if \var{value1} is greater than \var{value2}

This operator can be used with both numeric and string values, as long
as \var{value1} and \var{value2} are both of the same type.\\ \hline 

!= & not equal to & TRUE if \var{value1} is not equal to
\var{value2}\\ 

$<>$ & & \\

$><$ & & \\

& &This operator my be used with both numeric and string values, as long
as \var{value1}  and \var{value2} are both of the same type.\\ \hline  

$<$= & less than or equal to & TRUE if \var{value1} is less than or
equal to \var{value2} 

This operator can be used with both numeric and string values, as long
as \var{value1} and \var{value2} are both of the same type.\\ \hline  

$>$= & greater than or equal to & This operator can be used with both
numeric and string values, as long as \var{value1} and \var{value2}
are both of the same type.\\ \hline   

! & logical NOT & TRUE if \var{value} is FALSE, FALSE if \var{value}
is TRUE\\ 

not & & \\

& & Logical NOT accepts only numeric arguments. 

The logical NOT operator, unlike all other logical operators, takes
only 1 argument. Thus, the format for a logical NOT statement is one
of the following:

\begin{example}
! \var{value}
not \var{value}
\end{example}\\  \hline  
\end{longtable}

\subsection{Special Symbols}

\begin{center}
\begin{tabular}{|l|l|} \hline
\tblhd{Rep.} & \tblhd{Meaning}\\ \hline 
\texorhtml{~}{\htmlsym{##126}} & negative sign\\ \hline 
( ) & indicate order in which expressions are evaluated\\ \hline 
[ ] & enclose variables\\ \hline 
"" & enclose string constants\\ \hline 
\end{tabular}
\end{center}

\section{Function Definitions}

The functions take only a single argument, in the following manner:

\begin{example}
name(\var{value}) 
\end{example}

The parentheses shown above are not necessary
unless the function is evaluating a complex argument. In the
definitions given below, the \var{value} is represented as x. Function
definitions which require functions themselves are given in a manner
compliant with the equation evaluator.

\begin{longtable}{|l|p{1.3in}|p{2.5in}|} 
\caption{Functions}
\\ \hline
\tblhd{Name} & \tblhd{Meaning} & \tblhd{Explanation} \\ \hline 
\endfirsthead
\caption{Functions (continued)}
\\ \hline
\endhead
acosh & inverse hyperbolic cosine & $\ln{(x + \sqrt{x^{2} - 1})}$ 
Domain error if $x < 1$.\\ \hline 
asinh & inverse hyperbolic sine & $\ln{(x + \sqrt{x^{2} + 1})}$\\ \hline 
atanh & inverse hyperbolic tangent & $\frac{\ln{(\frac{1 + x}{1 - x})}}{2}$ 
Domain error if $x \geq 1$ or $x \leq -1$ \\ \hline 
asech & inverse hyperbolic secant & $\ln{(\frac{1 + \sqrt{1 - x^{2}}}{x})}$
Domain error if $x \leq 0$ or $x > 1$ \\ \hline 
acsch & inverse hyperbolic cosecant & 
$\ln{(\frac{\frac{1}{x} + \sqrt{1 + x^{2}}}{|x|})}$ 
Domain error if $x = 0$\\ \hline 
acoth & inverse hyperbolic cotangent & $\frac{\ln{(\frac{x + 1}{x -1})}}{2}$ 
Domain error if $-1 \leq x \leq 1$ \\ \hline 
acos & inverse cosine (radians) & Domain error if $x < -1$ or $x > 1$\\ \hline 
asin & inverse sine (radians) & Domain error if $x < -1$ or $x > 1$\\ \hline 
atan & inverse tangent (radians) & \\ \hline 
asec & inverse secant (radians) & Domain error if $-1 < x < 1$ \\ \hline 
acsc & inverse cosecant (radians) & Domain error if $-1 < x < 1$ \\ \hline 
acot & inverse cotangent (radians) & Domain error if $x = 0$ \\ \hline 
cosh & hyperbolic cosine & \\ \hline 
sinh & hyperbolic sine & \\ \hline 
tanh & hyperbolic tangent & \\ \hline 
sech & hyperbolic secant & \\ \hline 
csch & hyperbolic cosecant & \\ \hline 
coth & hyperbolic cotangent & \\ \hline 
sqrt & square root & Domain error if $x < 0$ \\ \hline 
sign & sign of argument & Evaluates to 1 if $x > 0$, 0 if $x = 0$, 
and -1 if $x < 0$ \\ \hline 
cos & cosine (radians) & \\ \hline 
sin & sine (radians) & \\ \hline 
tan & tangent (radians) & \\ \hline 
sec & secant (radians) & \\ \hline 
csc & cosecant (radians) & \\ \hline 
cot & cotangent (radians) & \\ \hline 
abs & absolute value & \\ \hline 
exp & e to the power &  $e^x$\\ \hline 
ln & logarithm base e & Domain error if $x \leq 0$\\ \hline 
log & logarithm base 10 & Domain error if $x \leq 0$\\ \hline 
fac & factorial & Domain error if $x <= 0$ 
x is rounded to nearest smaller integer before factorial is calculated.\\ \hline 
deg & radians to degrees & $180 x /\pi$ \\ \hline 
rad & degrees to radians & $x\times\pi /180$ \\ \hline 
rup & round to nearest larger integer & \\ \hline 
rdn & round to nearest smaller integer & \\ \hline 
rnd & round to nearest integer & \\ \hline 
sqr & square & $x^2$\\ \hline 
ten & ten to the power & $10^x$\\ \hline 
not & logical not & This is the same as the logical NOT operator, but
is included here because of its function-like behavior. Evaluates to 1
if $x = 0$, 0 otherwise.\\ \hline 
\end{longtable}

%\subsection{Arithmetic Operators}

%\begin{center}
%\begin{tabular}{|l|l|} \hline
%\tblhd{Rep.} & \tblhd{Meaning} \\ \hline 
%\circflex & exponentiation\\ \hline 
%\% & modulus\\ \hline 
%* & multiplication\\ \hline 
%/ & division\\ \hline 
%+ & addition\\ \hline 
%- & subtraction\\ \hline 
%\end{tabular}
%\end{center}

%\subsection{Logical Operators}

%\begin{center}
%\begin{tabular}{|l|l|} \hline
%\tblhd{Rep.} & \tblhd{Meaning}\\ \hline 
%! & logical not (takes only 1 argument)\\ \hline 
%not & logical not (takes only 1 argument)\\ \hline 
%\& & logical and\\ \hline 
%\&\& & logical and\\ \hline 
%and & logical and\\ \hline 
%| & logical or\\ \hline 
%|| & logical or\\ \hline 
%or & logical or\\ \hline 
%x| & logical exclusive or\\ \hline 
%xor & logical exclusive or\\ \hline 
%= & equal to\\ \hline 
%== & equal to\\ \hline 
%< & less than\\ \hline 
%> & greater than\\ \hline 
%!= & not equal to\\ \hline 
%< > & not equal to\\ \hline 
%> < & not equal to\\ \hline 
%< = & less than or equal to\\ \hline 
%> = & greater than or equal to\\ \hline 
%\end{tabular}
%\end{center}

%\subsection{Special Symbols}

%\begin{center}
%\begin{tabular}{|l|l|} \hline
%\tblhd{Rep.} & \tblhd{Meaning}\\ \hline 
%\texorhtml{~}{\htmlsym{##126}} & negative sign\\ \hline 
%( ) & indicate order in which expressions are evaluated\\ \hline 
%[ ] & enclose variables\\ \hline 
%"" & enclose string constants\\ \hline 
%\end{tabular}
%\end{center}

%\section{Functions}

%\begin{longtable}{|p{0.5in}|p{2in}|}
%\caption{Functions}
%\T\\ \hline
%\tblhd{Name} & \tblhd{Meaning}\\ \hline 
%\endfirsthead
%\caption{Functions (continued)}
%\\ \hline
%\tblhd{Name} & \tblhd{Meaning}\\ \hline 
%\endhead
%acosh & inverse hyperbolic cosine\\ \hline 
%asinh & inverse hyperbolic sine\\ \hline 
%atanh & inverse hyperbolic tangent\\ \hline 
%asech & inverse hyperbolic secant\\ \hline 
%acsch & inverse hyperbolic cosecant\\ \hline 
%acoth & inverse hyperbolic cotangent\\ \hline 
%acos & inverse cosine (radians)\\ \hline 
%asin & inverse sine (radians)\\ \hline 
%atan & inverse tangent (radians)\\ \hline 
%asec & inverse secant (radians)\\ \hline 
%acsc & inverse cosecant (radians)\\ \hline 
%acot & inverse cotangent (radians)\\ \hline 
%cosh & hyperbolic cosine\\ \hline 
%sinh & hyperbolic sine\\ \hline 
%tanh & hyperbolic tangent\\ \hline 
%sech & hyperbolic secant\\ \hline 
%csch & hyperbolic cosecant\\ \hline 
%coth & hyperbolic cotangent\\ \hline 
%sqrt & square root\\ \hline 
%sign & sign of argument (1 if pos, -1 if neg, 0 if 0)\\ \hline 
%cos & cosine (radians)\\ \hline 
%sin & sine (radians)\\ \hline 
%tan & tangent (radians)\\ \hline 
%sec & secant (radians)\\ \hline 
%csc & cosecant (radians)\\ \hline 
%cot & cotangent (radians)\\ \hline 
%abs & absolute value\\ \hline 
%exp & e to the power\\ \hline 
%log & logarithm base 10\\ \hline 
%fac & factorial\\ \hline 
%deg & radians to degrees\\ \hline 
%rad & degrees to radians\\ \hline 
%rup & round to nearest larger integer\\ \hline 
%rdn & round to nearest smaller integer\\ \hline 
%rnd & round to nearest integer\\ \hline 
%sqr & square\\ \hline 
%ten & ten to the power\\ \hline 
%not & logical not\\ \hline 
%ln & logarithm base e\\ \hline 
%\end{longtable}

\section{Definitions of Terms}
\label{glossary}

\begin{description}
\item[Constant]
  
  A number whose value is explicitly stated in the query.

\item[Predefined Constant]
  
  A number whose value is known explicitly by the equation
  interpreter, but is not stated in the equation. The two predefined
  constants, with names preceded by a colon, are :e (2.71828182846)
  and :pi (3.14159265359).

\item[String Constant]
  
  A character string whose value is explicitly stated in the equation.

\item[Variable]
  
  A number which is referenced by a unique name in the equation, but
  whose value is not stated in the equation.

\item[String Variable]
  
  A character string which is referenced by a unique name in the
  equation, but whose value is not stated in the equation.

\item[Domain Error]
  
  A problem which arises when a function or operation is undefined for
  certain input values, such as division by 0. If a domain error is
  generated, the equation is evaluated to 0.
\end{description}

\section{Rules}

Equations or queries are formed by combining variables and constant
values with operators or functions in a meaningful way. The following
set of rules applies to creating an equation.

\begin{itemize}
\item Variable names must be enclosed in [ ] (square brackets).
  
  For instance, finding the sum of a variable named height and another
  variable named altitude is expressed as:

\begin{example}
[height] + [altitude] 
\end{example}

\item String constants must be surrounded by " " (quotation marks).
  
  For instance, a query to see if a string variable latitude is equal
  to the string north is expressed as:

\begin{example}
[latitude] = "north" 
\end{example}

\item Constants with negative values must be proceeded by ~ (tilde).
  
  For instance, multiplying the variable altitude by a negative four
  is expressed as:

\begin{example}
[altitude] * ~4 
\end{example}

\item At least one variable (numeric or string) must be referenced in
  the equation.
  
\item Variable names cannot contain \lit{"} (quotation marks), or [ ]
  (square brackets).
  
\item Spaces are ignored in equations (except inside string constants
  and variable names), so you can use spaces to separate the parts of
  your equation and make it more readable.
\end{itemize}

\section{Pre-defined Constants}

The names of the two predefined constants e and pi are preceded by a
colon to differentiate them from a function name or variable name.

The following equation:

\begin{example}
[degrees] * :e + :pi 
\end{example}

multiplies the variable \lit{degrees} by \emph{e} (2.718) and adds $\pi$
(3.141).

\section{Order of Operations}

An equation is evaluated in the following order: 

\begin{enumerate}
\item anything inside parentheses (left to right, sub-parentheses
  given precedence)
  
\item functions (no explicit order to function evaluation)
  
\item exponentiation (left to right)

\item multiplication, division, and modulus (left to right) 

\item addition and subtraction (left to right) 
  
\item logical operators (no explicit order to logical operator
  evaluation)
\end{enumerate}

For instance, the equation

\begin{example}
4 + (cos[x] - [y] \circflex\ 3) * (([y] + 4) / 7) + ([x] > 1)
\end{example}

is evaluated as follows: ([x] = 3.14159265359, [y] = 3, bold is
changed value)

\begin{example}
4 + (\textbf{~1} - [y] \circflex\ 3) * (([y] + 4) / 7) + ([x] > 1)
4 + (~1 - \textbf{27}) * (([y] + 4) / 7) + ([x] > 1)
4 + \textbf{~28} * (([y] + 4) / 7) + ([x] > 1)
4 + ~28 * (\textbf{7} / 7) + ([x] > 1)
4 + ~28 * \textbf{1} + ([x] > 1)
4 + ~28 * 1 + \textbf{1}
4 + \textbf{~28} + 1
\textbf{~24} + 1
\textbf{~23}
\end{example}

The similar equation 

\begin{example}
4 + (cos[x] - [y] \circflex\ 3) * (([y] + 4) / 7) + [x] > 1 
\end{example}

is evaluated as follows: (with the same values for [x] and [y]) 

\begin{example}
4 + (\textbf{~1} - [y] \circflex\ 3) * (([y] + 4) / 7) + [x] > 1
4 + (~1 - \textbf{27}) * (([y] + 4) / 7) + [x] > 1
4 + \textbf{~28} * (([y] + 4) / 7) + [x] > 1
4 + ~28 * (\textbf{7} / 7) + [x] > 1
4 + ~28 * \textbf{1} + [x] > 1
4 + \textbf{~28} + [x] > 1
\textbf{~24} + [x] > 1
\textbf{~20.8584073464} > 1
\textbf{0} 
\end{example}

\section{General Suggestions}

It is best to use the supported operators and functions in order to
reduce the number of items to be evaluated. This reduces the time it
takes to evaluate the equation, perhaps negligible for each
evaluation, but with repeated evaluation, the time savings can be
substantial. For instance, the equation:

\begin{example}
ln(((1 / [x]) + (sqrt(1 + ([x] \circflex\ 2))) / abs([x]))) 
\end{example}

is equivalent to: 

\begin{example}
acsch[x] 
\end{example}

but the first equation takes much longer to evaluate than the second.
There are a few cases where there are various ways to state the same
equation, such as:

\begin{example}
[x] * [x]
[x] \circflex\ 2
sqr[x] 
\end{example}

All three equations above square the variable x. All three require
only one evaluation each, and therefore require almost exactly the
same amount of time to evaluate. The equations below are also
completely equivalent:

\begin{example}
[x] * [x] * [x]
[x] \circflex\ 3 
\end{example}

In this case, the first equation requires two evaluations ([x] * [x],
and then the result of that * [x]), while the second equation requires
only one evaluation. The evaluation of the second equation will
therefore be approximately 2 times faster.

Avoid causing unnecessary evaluations. For instance, the following
equation:

\begin{example}
(1 / 2) * [base] * [height] 
\end{example}

is faster if expressed as: 

\begin{example}
[base] * [height] / 2 
\end{example}

\subsection{Examples}

\subsubsection{Absolute Latitude}

The following equation takes the variable \lit{abs\_latitude} and
multiplies it by -1 if the value of \lit{latitude\_n\_or\_s} is S,
\lit{abs\_latitude} is multiplied by 1 otherwise.

\begin{vcode}{xib}
[abs_latitude] * (([latitude_n_or_s] != "S") + ~1 * ([latitude_n_or_s] = "S")) 
\end{vcode}

\subsubsection{Distance Between Points}

The following equation computes the distance between the points given
by x1, y1, x2 and y2 (assuming all points lie on a plane).

\begin{vcode}{ib}
sqrt(sqr([x2] - [x1]) + sqr([y2] - [y1])) 
\end{vcode}

\subsubsection{Quadratic Solution}

The following equation computes one of the solutions to the quadratic
formula:

\begin{vcode}{sib}
(~1 * [b] + sqrt([b] \circflex\ 2 - 4 * [a] * [c])) / (2 * [a]) 
\end{vcode}

\subsubsection{Sine of Angle}

The following 3 equations are all roughly equivalent, finding the sine
of angle deg (which is measured in degrees). The first equation is not
recommended, because the value used for pi is not as accurate as the
value given by the pre-defined constant :pi. The second equation is
better, but requires more time to be evaluated than the third.

\begin{example}
sin(3.141592 * [deg] / 180)
sin(:pi * [deg] / 180)
sin(rad[deg]) 
\end{example}

\subsubsection{Volume of Sphere}

The following equation finds the volume of a sphere: 

\begin{example}
4 * :pi * [radius] \circflex\ 3 / 3 
\end{example}

%%% Local Variables: 
%%% mode: latex
%%% TeX-master: t
%%% End: 
