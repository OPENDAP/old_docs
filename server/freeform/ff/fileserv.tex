\chapter{File Servers}
\label{fs}

\note{The DODS and OPenDAP projects have used the \ffs\ to present a catalog
of data files to the world as a single dataset. In many ways this was
a very successful system, providing catalogs for multi-granule
datasets that could be searched by date and time. However, the OPeNDAP
project has decided (winter 2006) to adopt the THREDDS xml-based
catalog system developed at Unidata, Inc. The remainder of this
chapter describes the 'file servers' that can be built using the
FreeForm data handler. Even though we feel it's best to adopt the
THREDDS catalogs, there are good reasons to keep existing catalog
servers running and to build new catalogs as a stop-gap measure to
support existing client software.} 

Normally, in the OPeNDAP
argot, a ``dataset'' is contained in a single file on a disk. However,
this paradigm is often broken by large datasets that may contain many
thousands or tens of thousands of data files. The OPeNDAP file server is
a way to make these dicrete datasets appear to be a single large
dataset.

The OPeNDAP file server is an OPeNDAP server that returns a URL or set of
URLs in response to a query containing selection variables.  For
example, a dataset organized by date and geographic location might
provide a file server that allowed you to query the dataset with a
range of dates and longitudes.  This fileserver would return a list of
one or more URLs corresponding to files within that dataset that fell
within the given range.


\section{The Problem}
\label{fs,problem}

Consider the following (imaginary) list of files:

\begin{vcode}{ib}
1997360.nc  1998001.nc  1998007.nc  1998013.nc ...
1997361.nc  1998002.nc  1998008.nc  1998014.nc
1997362.nc  1998003.nc  1998009.nc  1998015.nc
1997363.nc  1998004.nc  1998010.nc  1998016.nc
1997364.nc  1998005.nc  1998011.nc  1998017.nc
1997365.nc  1998006.nc  1998012.nc  1998018.nc
\end{vcode}

These appear to be a set of netCDF files, arranged by
date\footnote{A serial date, with a year and the day of the
  year, expressed in an ordinal number from 1 to 365 or 366.}

If you want data from the first week of January, 1998, it is fairly
clear which files to request.  However, the OPeNDAP server provides
no way to request data from more than one file, so your request would
have to be split into 7 different requests, from \lit{1998001.nc} to
\lit{1998007.nc}.  This could be represented as a set of seven DODS
URLs: 

\begin{vcode}{ib}
http://dods/cgi-bin/nph-nc/1998001.nc
http://dods/cgi-bin/nph-nc/1998002.nc
http://dods/cgi-bin/nph-nc/1998003.nc
http://dods/cgi-bin/nph-nc/1998004.nc
http://dods/cgi-bin/nph-nc/1998005.nc
http://dods/cgi-bin/nph-nc/1998006.nc
http://dods/cgi-bin/nph-nc/1998007.nc
\end{vcode}

But what if you then uncover another similar dataset whose data you
want to compare to the first?  Or what if you want to expand the
inquiry to cover the entire year?  Keeping track of this many URLs
will quickly become burdensome.

What's more, another similar dataset could be arranged in two
different directories, \lit{1997} and \lit{1998}, each with files:

\begin{vcode}{ib}
001.nc
002.nc
003.nc
...
\end{vcode}

and so on.  Now you have to keep track of two large sets of URLs, in
two different forms.  But you could also imagine files called:

\begin{vcode}{ib}
0011998.nc
0021998.nc
0031998.nc
\end{vcode}

or 

\begin{vcode}{ib}
00198.nc
00298.nc
00398.nc
\end{vcode}

or 

\begin{vcode}{ib}
1Jan98.nc
2Jan98.nc
3Jan98.nc
\end{vcode}

That is, the number of possible sensible arrangements may not, in
fact, be infinite, but it may seem that way to a scientist who is
simply trying to find data.


\section{The OPeNDAP File Server Solution}
\label{fs,solution}

To create a system that allows data providers to assert a degree of
uniformity over wildly variable dataset organizations, OPeNDAP
provides for the installation of an OPeNDAP \new{file server}. The
file server is a server that provides access to a special dataset,
containing associations between the names of files within a dataset
and some ``selectable'' data values.


\subsection{Selectable Data}
\label{fs,selectable}

The concept of \new{selectable data} requires some explanation.  This
is used to indicate the data variables you might ordinarily use to
narrow your search for data in the first pass at a dataset.

For geophysical data, the selectable data is often the time and
location of the data, since typical searches for data often begin by
specifying a part of the globe that bears examining, or a date of some
event.  For other types of data, other data variables will seem more
appropriate.  Model data, for example, which has no real location or
time, might be arranged by the parameters that varied between runs.

A comprehensive definition of selectable data has so far eluded the
OPeNDAP group, but there are some guidelines, albeit fairly vague ones:

\begin{itemize}
\item The selectable data is generally \emph{not} recorded within each
  data file.  However, the selectable data may often include a
  \emph{range} summarizing some of the data within each file.
\item The selectable data should help a user decide whether a
  particular data file in a dataset is useful.  A temperature range
  might not be as useful as a time range, since data searches more
  often start with time.  (Both would presumably be still more useful,
  but there is a trade-off between the utility of the file server and
  the time spent maintaining it.)
\end{itemize}

\subsection{What It Looks Like}
\label{fs,look}

Consider again the set of data files shown in \sectionref{fs,problem}.
We could associate each one of these files with a date, and this would
provide the rudiments of a file server if we then serve that data with
an OPeNDAP server such as the \ffs .


\begin{vcode}{ib}
1997/360 http://dods/cgi-bin/nph-nc/1997360.nc
1997/361 http://dods/cgi-bin/nph-nc/1997361.nc
1997/362 http://dods/cgi-bin/nph-nc/1997362.nc
1997/363 http://dods/cgi-bin/nph-nc/1997363.nc
1997/364 http://dods/cgi-bin/nph-nc/1997364.nc
1997/365 http://dods/cgi-bin/nph-nc/1997365.nc
1998/001 http://dods/cgi-bin/nph-nc/1998001.nc
1998/002 http://dods/cgi-bin/nph-nc/1998002.nc
1998/003 http://dods/cgi-bin/nph-nc/1998003.nc
1998/004 http://dods/cgi-bin/nph-nc/1998004.nc
1998/005 http://dods/cgi-bin/nph-nc/1998005.nc
1998/006 http://dods/cgi-bin/nph-nc/1998006.nc
1998/007 http://dods/cgi-bin/nph-nc/1998007.nc
1998/008 http://dods/cgi-bin/nph-nc/1998008.nc
1998/009 http://dods/cgi-bin/nph-nc/1998009.nc
1998/010 http://dods/cgi-bin/nph-nc/1998010.nc
\end{vcode}

This list represents a set of DAP URLs, each identified by a date,
given as a year and a serial day.  The files appear to be netCDF
format files, served by an OPeNDAP netCDF server, but that is not
important for this discussion.

To use the \ffs\ for your file server, you could use a format
description file with an input section like this:

\begin{vcode}{ib}
ASCII_input_data "File Server Example Input"
year 1 4 short 0
serial_day 6 8 short 0
DODS_Url 10 46 char 0
\end{vcode}

\begin{comment}
\section{File Server Functions}
\label{ss-fns}




Yes, but it may take a while to get there. Right now the grid_select function
*is* available to every server, while the DODS_Date, DODS_Time,
DODS_DateTime, date, time and datetime function are only in the FF server.
I'm not sure when those will mve into the general server population.





 >"Paul" == Paul Hemenway <paul@koko.gso.uri.edu> writes:

 > Hi, Dan,
 >>> loaddods('http://koko.gso.uri.edu/cgi-bin/nph-ff/servers/nscat/fsu/fsutestfs.dat?DODS_URL,DODS_Date()&date("1996/366","1997/006")')          

 > Reading: http://koko.gso.uri.edu/cgi-bin/nph-ff/servers/nscat/fsu/fsutestfs.dat
 >   Constraint: DODS_URL,DODS_Date()&date("1996/366","1997/006")
 > Server version: dods/2.21
 > Creating string vector `DODS_Date'.
 > Creating string vector `DODS_URL'.

 > DODS_URL                                           

 > DODS_URL =

Oh, Bad docs on my part. Add the name of the constructor variable that should
contain the DODS_Date variable. In the following dataset:

    [dcz:/usr/local/DODS/src/dap-2.23] ./geturl -d http://koko.gso.uri.edu/cgi-bin/nph-ff/servers/nscat/fsu/fsutestfs.dat
    Dataset {
        Sequence {
            Int32 year;
            Int32 day;
            String DODS_URL;
        } FSU_nscat1;
    } fsutestfs;

The ctor variable that should contain DODS_Date is `FSU_nscat1'. How did I
know that? Because the variable year and day are contained by the ctor
variable. If you look at the DAS you can see that this is the case. But I
can't since the server is older and does not have the ancillary DAS fix in it.

However, something else is wrong, maybe in the ancillary DAS. Here's some
tests: 

This works OK:

    [dcz:/usr/local/DODS/src/dap-2.23] ./geturl -D http://koko.gso.uri.edu/cgi-bin/nph-ff/servers/nscat/fsu/fsutestfs.dat -c "DODS_URL&year=1997&day=170"The data:
        Sequence {
            String DODS_URL;
        } FSU_nscat1;
     = { { "http://www.coaps.fsu.edu/cgi-bin/nph-nc/nscat/data/wt_W25/NSCAT.wt0.5.1997170.nc" }};

But not this:

    [dcz:/usr/local/DODS/src/dap-2.23] ./geturl -D http://koko.gso.uri.edu/cgi-bin/nph-ff/servers/nscat/fsu/fsutestfs.dat -c "DODS_URL,DODS_Date(FSU_nscat1)&year=1997&day=170"
    The data:
        Sequence {
            String DODS_URL;
            String DODS_Date;
        } FSU_nscat1;
     = { { "", "" }};

Nor this:

    [dcz:/usr/local/DODS/src/dap-2.23] ./geturl -D http://koko.gso.uri.edu/cgi-bin/nph-ff/servers/nscat/fsu/fsutestfs.dat -c"DODS_URL&date(\"1997/170\")"
The data:
        Sequence {
            String DODS_URL;
        } FSU_nscat1;
     = { { "" }};

 > *****************************************

 > I can go ahead with the "year" and "day", but it's much less elegant than
 > with date().  And I'd like to find out how to do it "right" in the end.

 > (I changed the day_variable to year_day_variable in the .das file.)

 > Any ideas?

Send your ancillary DAS to the list and/or get the latest server.




Overview:

To set up a file server for particular dataset you first must create a simple
database to server up with the FreeForm server. Once that is done you will
need to customize some of the constraint expression functions and rebuild
(compile and link) the FreeForm server with those functions. In some future
version of the file server I'll make this something that can be set up in a
(text) config file. However, for now its C++ all the way (just what you
wanted to hear, huh?).

Stepwise:

1) Build a database to server with FreeForm. Use something like `ls -1 -R' or
   find to build up a list of the files. If you do this for the DSP files
   that gives you the date information needed to build a simple ASCII
   flat-file database. I made a test one using `ls -1' and perl. You need to
   put the year, day-number and URL in separate columns. Bag time for right
   now (I'm still working on Time and DateTime). The database should look
   something like:

   98 001 http://.../h98001xxyyzz.dsp.Z

   Of course, you'll start at (19)79 or whatever.

2) Set up a regular FF server. To do that you need to make a FreeForm format
   file. To do that you need the starting and ending column positions for the
   fields. Using the earlier example:

   98 001 http://.../h98001xxyyzz.dsp.Z
   ^^ ^ ^ ^                           ^
   || | | |                           35
   || | | 8
   || | 6
   || 4
   |2
   1

   Note that FreeForm is like Fortran in that counting starts at one. The
   format file would look like:

      ASCII_input_data "AVHRR"
      year 1 2 int32 0
      day 4 6 int32 0
      DODS_URL 8 35 text 0

   There are some trick here: `ASCII_input_data' must be the first word on
   the first non-comment line (comments start with `/'). The name of the
   dataset is the string following ASCII_input_data. FF will let you put
   spaces there, but DODS *won't*. The variable used to hold the URL *cannot*
   be called `URL' (or Url or url) because that is a reserved word in DODS. I
   use DODS_URL, but anything other than a type name will work.

   One more trick: You can add an ASCII_output_data section to the format
   file and use the `newform' program (in the FF server distribution subdir
   FFND) to test out your new dataset. It could be:

      ASCII_output_data
      year 1 2 int32 0
      DODS_URL 4 27 text 0

   The idea is to make a different (smaller) set of output variables so you
   can process things with newform and see it change those data.

3) set up the FF stuff. Store the database in a text file named
   <something>.dat and store the format file *in the same directory* in a
   file named <something>.fmt. The <something>s have to match.

   --> The FF manual and DODS FF server manual (should) explain all this
   --> stuff.

4) Test the FF server. Look at the DDS and make simple queries with the
   dataset. You can run the ff_dods server component on the command line and
   pipe the resulting stuff to asciival:

      ff_dods AVHRR.dat -e 'year,day,DODS_URL&year=82&day=12' | asciival -a -- -

   Note the convention that `--' separates options from the URLs and the `-'
   for a URL means read from stdin. asciival is the DODS_ROOT/etc.

5) Build a custom copy of the ce_functions.cc file for you dataset. In the
   distribution of the FF server there is a file called ce_functions.cc. In
   that file you'll need to customize the function

      DODS_Date_jpl_pathfinder(DDS &dds)          

   so that it does you bidding. Here's the code:

    DODS_Date
    DODS_Date_jpl_pathfinder(DDS &dds)
    {
        BaseType *day_var = dds.var("JPL_Pathfinder.day");
        int day_number;
        int *day_number_p = &day_number;
        day_var->buf2val((void **)&day_number_p);

        BaseType *year_var = dds.var("JPL_Pathfinder.year");
        int year;
        int *year_p = &year;
        year_var->buf2val((void **)&year_p);

        DODS_Date d(year, day_number);

        DBG(cerr << "year: " << year << " day number: " << day_number << endl);

        return d;
    }

   This function reads two variables from the dataset and builds a DODS_Date
   object. Note that I use the fullname of each variable. The DODS_Date class
   has ctors for yyddd and yymmdd dates (look at the file DODS_Date.h for
   reference documentation; run it through doc++ to get Latex & ps).

   You should (for sanity's sake) change the name from DODS_Date_jpl... to
   DODS_Date_gso... Once you've done that, make sure to change the other
   places in ce_functions.cc where this particular function is called. The
   function could be declared static (in fact I guess not doing so is a bug
   on my part). 

6) What is going on here? The comment in ce_functions.cc are pretty lean.
   Here's a run down. To get standard comparisons of dates I hide the actual
   work inside a function that is called in a CE. To do the same sample URL I
   gave above, you could now say:

    ff_dods AVHRR.dat -e 'year,day,DODS_URL&date("1982/12")' | asciival -a -- -

   Note that the URL is in single quotes and the date is in double quotes.
   Dates are passed to the `date()' CE function as strings.

   The other part of those functions in ce_functions.cc provide a way to get
   dates back from the server in a standard form. I added `projection
   functions' to the CE parser to make that possible. By asking for a
   variable with the projection function DODS_JDate() you will get a DODS
   year-day date string back from the server. For example:

    ff_dods AVHRR.dat -e DODS_URL,DODS_JDate(AVHRR)&date("1982/12")' \
      | asciival -a -- -

  Could produce:
  
    "http://..../h98012...dsp.Z" "1998/12"
    .
    .
    .

  The key thing is that the using the function `DODS_JDate(...)' in the
  projection part of the CE causes a new variable to be inserted into the DDS
  and added to the current projection (users don't need to know that, but
  reading through the ce-functions.cc code might be easier knowing it).




DODS and the Fileserver.

Paul Hemenway 20 March 1999

DODS links a user (Client) to a dataset, usually through a Graphical 
User Interface (GUI).   The link is made over the World Wide Web (www) 
through a set of serving files and programs that takes the user's data 
specifications and converts them into a request for specific data from the 
specified data set(s).  Often, but not always, these serving files are 
located in a "Fileserver".

When it exists, THE FILESERVER is 
a) the set of files that is used by DODS to convert the user's 
      data specifications into specific Universal Record Locators (URLs), and 
b  the location of that set of files.
 

The steps before, during and after the call to the file server are:

1) Client  sets data criteria (e.g., date-, lat-, lon-, temp-, ranges).

2) Client    <==>  FILESERVER  (to get data URLs)

3) Client    <==>  Dataset   (specific data are requested and sent.)


   
In the following example, the Client will be represented by the 
MATLAB gui, the fileserver will be represented by three specific
Freeform files which make up the fileserver, and the served dataset
will be represented by a directory containing the given dataset.  The
dataset will be assumed to be served as dsp files.

1)  Client, using the MATLAB-gui, specifies:
                dataset,variables and ranges (including lat,lon,time) 
                of the request.
                     
                         
                   THIS IS THE FreeForm FILESERVER!
                                ^
                                |
                                |
                             fs.dat  
2)      MATLAB-gui <==>    fs.dat.das    (located in:
                             fs.fmt       http://fsmachine/path2fsdirectory)



3)  MATLAB-gui, requests data from: http://datamachine/path2datadirectory
                which is returned over the www.


All machine names, path names, and directories are generic.




Notes on Communication.

I)  In the MATLAB-gui:

The MATLAB-gui must know where the dataset and the fileserver reside.
In the archive.m file (see the MATLAB-gui Manual), two variables are set:

Server = 'http://mymachine/cgi-bin/nph-dsp/path2datadirectory'

CatalogServer = 'http://fsmachine/cgi-bin/nph-ff/path2fsdirectory'

The first points to the dataset.  The second points to the fileserver.


II) Within the Fileserver itself:

The fs.dat file is made up of the dataset URLs, prepended with time
or other information used by the gui to determine which URLs to call.

(e.g.:

1981/01/26:23:09:02 http://dods.gso.uri.edu/cgi-bin/nph-dsp/avhrr/1981/1/n81026230902.pvu.Z )

The fs.dat.das file contains, among other things, a construct that looks like:

  DODS_Date_Time {
                # The DODS_Date_Time prototype is as follows:
                #
                #       yyyy/mm/dd:hh:mm:ss
                #
                # This prototype is purely experimental.

                String year_variable URI.Avhrr.year;
                String month_variable URI_Avhrr.month;
                String day_variable URI_Avhrr.day;
                String hours_variable URI_Avhrr.hours;
                String minutes_variable URI_Avhrr.minutes;
                String seconds_variable URI_Avhrr.seconds;
                String gmt_time true;
        }

The string: URI_Avhrr   

is the same string that appears in the fs.fmt file (e.g.:

ASCII_input_data "URI_Avhrr"
year     1 4    int32 0
month    6 7    int32 0
day      9 10   int32 0
hours    12 13  int32 0
minutes  15 16  int32 0
seconds  18 19  int32 0
DODS_URL 21 92  text  0   )

which determines the data structure within the .dat file.  (NB.  In this
case, the fileserver is a FreeForm server, hence the specific formatting
in the .fmt file, and the existance of the .fmt file itself.)

AN OBSERVATION:
   The fileserver and the dataset do not need to know where each other are,
except that the URLs for the dataset must be in the fs.dat file.  The 
Client (through the MATLAB-gui) first selects the data criteria, second, 
accesses the fileserver to determine which URLs to call, and third, after
the the fileserver has provided the URLs, the client  actually calls for
the URLs from the dataset.  The fileserver and dataset may be in the same
directory on the same machine, but they need not be.







\section{Building a File Server: An Example}
\label{fs,example}

Here is an example of building a file server for a large collection of
data files, served by an OPeNDAP server.
\footnote{All the files mentioned in this section are available in the
  \lit{examples} directory on \DODShome .}

\end{comment}

%%% Local Variables: 
%%% mode: latex
%%% TeX-master: t
%%% End: 
