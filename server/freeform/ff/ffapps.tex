%CHAPTER 10       
%
% $Id$
%

\chapter{Developing FreeForm ND Applications}
\label{ff,develop}

\indc{developing FreeForm applications}
\indc{FreeForm applications!developing}
\indc{programming with FreeForm}\indc{FreeForm!programming}
As applications have become increasingly complex, the concept of
layered application development has gained wide acceptance. A series
of layers, each of which is as self-contained as possible, is used to
interface between user and data. Interactions between layers are kept
as simple as possible. FreeForm ND applications use this model and
also incorporate the object-oriented approach to increase application
power and efficiency while simplifying design and maintenance. As an
application programmer, you can use the FreeForm ND Data Access System
to build your own FreeForm ND-based programs.

\section{FreeForm ND Application Layers}

\indc{layers!FreeForm application}
FreeForm ND applications are composed of the layers shown below.

The FreeForm ND Data Access System comprises the FreeForm ND Library
and Data Objects layers. You, the application programmer, write the
application-specific code and the user interface that sit above and
make use of the FreeForm ND layers. The FreeForm ND Library sits
closest to the data. It includes functions for creating and
interpreting format description files, and for reading, converting,
and writing data.

The Data Object layer above the FreeForm ND Library consists of
several types of objects that provide a simplified interface to the
Library. Many common data access tasks have been implemented as events
that the objects know how to accomplish. These objects are implemented
as structures in the C programming language. The members of a
structure are attributes of the object described by the structure.

\section{Building an Application}

You build a FreeForm ND application using the FreeForm ND library
functions and data objects. To use an object in an application, you
must complete three steps:

\begin{enumerate}
\item Create the object. 

\item Set the object's attributes. 

\item Send events to the object to trigger the desired action. 
\end{enumerate}

You can also include calls to show functions, e.g., \lit{db\_show}, to
determine current characteristics of objects as the application runs.

\subsection{Example Program}

\indc{example!FreeForm application}
\indc{example!programming with FreeForm}
\indc{developing FreeForm applications!example}
\indc{FreeForm applications!developing!example}
The example FreeForm ND application \lit{getll.c} extracts and
converts latitude and longitude values in any data file from their
native format into a signed decimal degree representation. The program
first defines a data bin with the native input format for a data file
that includes latitude and longitude variables in any representation.
Then it defines a compile-time format for just latitude and longitude,
and reformats the latitude and longitude variables from their native
format into the decimal degrees format.

Compile-time formats are used to read data from any hard-coded format
into memory, where the data can then be accessed by applications.
Unlike other formats, a compile-time format is not intended to be
written to a file (although it could be). The example program
\lit{getll.c} demonstrates how to implement a compile-time format in a
FreeForm ND-based application.


\subsubsection{Source Code-\lit{getll.c}}
\indc{getll@\lit{getll.c} source code}
\nopagebreak
\T\bgroup
\T\footnotesize
%\T\def\verbatim@font{\tt\small}

\begin{verbatim}
/*
 * NAME:        getll
 *
 * PURPOSE:     This program reads latitude and longitude in any recognized 
 * format, converting to values in decimal degrees.
 *
 * AUTHOR: Ted Habermann, NGDC, (303) 497-6472, haber@mail.ngdc.noaa.gov
 * Modified (MAO)
 *
 * USAGE:              getll data_file
 *
 * COMMENTS:
 *
 * FreeForm ND applications are designed to run on many different types of
 * computers. One of the differences between these computers is the names
 * of various include files. These differences are taken care of by defining
 * your environment by defining one of the following three preprocessor
 * macros:  1) CCMSC (PC, Microsoft C), 2) SUNCC (Unix workstation, ANSI C),
 * or 3) CCLSC (Macintosh, ANSI C).  
 *
 */

#include <limits.h>

/* The FreeForm ND include file is surrounded by a definition of the
 * constant DEFINE_DATA in the main program so that extern arrays that
\end{verbatim}
\subj{}
\vspace{-2.75\baselineskip}
\begin{verbatim}
 * FreeForm ND uses get initialized.  The DEFINE_DATA constant must not be
 * defined in any other files.
 */
 
#define DEFINE_DATA
#include <FreeForm ND.h>
#undef DEFINE_DATA

/* This include file defines the data objects */
#include <databin.h>

#define ROUTINE_NAME "getll"

/* An error call back routine - it tells make_standard_dbin which events
   are okay if they fail.  getll "dynamically" creates the output data
   format, and throws away any existing output data format, so we don't
   require an output data format in the format file.  This function allows
   make_standard_dbin() to process other events, even if the OUTPUT_FORMAT
   event fails to produce an output format.
*/
#ifdef PROTO
static int mkstdbin_cb(int routine_name)
#else
static int mkstdbin_cb(routine_name)
int routine_name;
#endif
       {
       return(routine_name != OUTPUT_FORMAT);
       }

/*****************************************************************************
 * NAME:  check_for_unused_flags()
 *
 * PURPOSE:  Has user asked for an unimplemented option?
 *
 * USAGE:  check_for_unused_flags(std_args_ptr);
 *
 * RETURNS:  void
 *
 * DESCRIPTION:  All FreeForm ND utilities do not employ all of the "standard"
 * FreeForm ND command line options.  Check if the user has unwittingly asked
 * for any options which this utility will ignore.
 *
 * The following "standard" command line options are not used by this
 * application:
 *
 * -v  variable file
 * -q  query file
 * -p  precision (checkvar only)
 * -md missing data flag (checkvar only)
 * -m maximum number of bins (checkvar only)
 * -mm maximum/minimum processing only (checkvar only)
 *
 * AUTHOR:  Mark Ohrenschall, NGDC
 *
 * SYSTEM DEPENDENT FUNCTIONS:
 *
 * GLOBALS:
 *
 * COMMENTS:
 *
 * KEYWORDS:
 *
 * ERRORS:
 ****************************************************************************/

#ifdef PROTO
static void check_for_unused_flags(FFF_STD_ARGS_PTR std_args)
#else
static void check_for_unused_flags(std_args)
FFF_STD_ARGS_PTR std_args;
#endif
       {
       if (std_args->user.set_var_file)
              {
              err_push(ROUTINE_NAME, ERR_IGNORED_OPTION, 
                       "variable file"
                      );
              }
       
       if (std_args->user.set_query_file)
              {
              err_push(ROUTINE_NAME, ERR_IGNORED_OPTION,
                       "query file"
                      );
              }
       
       if (std_args->user.set_cv_precision)
              {
              err_push(ROUTINE_NAME, ERR_IGNORED_OPTION,
                       "precision (checkvar only)"
                      );
              }
       
       if (std_args->user.set_cv_missing_data)
              {                                        
              err_push(ROUTINE_NAME, ERR_IGNORED_OPTION,
                       "missing data flag (checkvar only)"
                      );
              }
       
       if (std_args->user.set_cv_maxbins)
              {
              err_push(ROUTINE_NAME, ERR_IGNORED_OPTION,
                      "maximum number of histogram bins (checkvar only)"
                      );
              }
       
       if (std_args->user.set_cv_maxmin_only)
              {
              err_push(ROUTINE_NAME, ERR_IGNORED_OPTION,
                       "maximum and minimum processing only (checkvar only)"
                      );
              }
       
       if (err_state())
              {
              err_disp();
              }
       }

#ifdef PROTO
void main(int argc, char *argv[])
#else
void main(argc, argv)
int argc;
char *argv[]
#endif
       {
       int error = 0;    /* to hold error return values */

       char *output_buffer = NULL; /* output data buffer */
       long  output_bytes  = 0;    /* bytes written into output buffer */

       FFF_STD_ARGS  std_args; /* holds command line information */
       DATA_BIN_PTR  input         = NULL; /* the data bin */
       FILE         *pfile         = NULL; /* output file */
            
       if (argc == 1)
              {
              fprintf(stderr, "%s%s",
#ifdef ALPHA
"\nWelcome to getll alpha "FF_LIB_VER" "__DATE__\
" -- an NGDC FreeForm ND example application\n\n",
#elif defined(BETA)
"\nWelcome to getll beta "FF_LIB_VER" "__DATE__\
" -- an NGDC FreeForm ND example application\n\n",
#else
"\nWelcome to getll release "FF_LIB_VER\
" -- an NGDC FreeForm ND example application\n\n",
#endif
"Default extensions: .bin = binary, .dat = ASCII, .dab = dBASE\n\
\t.fmt = format description file\n\
\t.bfm/.afm/.dfm = binary/ASCII/dBASE variable description file\n\n\
getll data_file [-f format_file] [-if input_format_file]\n\
                [-of output_format_file] [-ft \"format title\"]\n\
                [-ift \"input format title\"] [-oft \"output format 
                 title\"]\n\
                [-c count] No. records to process at head(+)/tail(-) of 
                 file\n\
                [-o output_file] default = output to screen\n\n\
See the FreeForm ND User's Guide for detailed information.\n"
                     );
              exit(EXIT_FAILURE);
              }
     
        /* The FreeForm ND system uses a hierarchical error handling system
        which allows each layer of an application to add error messages to
        a queue. err_push is the function which adds messages to the queue.
        It is called by any function which runs into an error. err_disp is
        the function that interactivly displays those errors to the user.
        It is called by the main application program when an error occurs.*/
            
        /* Allocate the output buffer:
        FreeForm ND uses two types of buffers extensively and defines default
        buffer sizes in the include file FreeForm ND.h. The local or scratch
        buffers are used as temporary work space. The cache buffers are
        used for reading large blocks of data.*/
     
       output_buffer = (char *)malloc((size_t)DEFAULT_CACHE_SIZE);
       if (!output_buffer)
              {
              err_push(ROUTINE_NAME, ERR_MEM_LACK, "Output Buffer");
              err_disp();
              exit(EXIT_FAILURE);
              }
     
       /* Collect options entered on the command line, this information will be
          used in the call to make_standard_dbin(), below.
       */
       if (parse_command_line(argc, argv, &std_args))
              {
              free(output_buffer);

              err_disp();
              exit(EXIT_FAILURE);
              }
       check_for_unused_flags(&std_args);

       /* Create and initialize the data bin */
       if (make_standard_dbin(&std_args, &input, mkstdbin_cb))
              {
              free(output_buffer);

              err_disp();
              exit(EXIT_FAILURE);
              }
       
       /* make_standard_dbin may have generated an incidental error, in case 
          the OUTPUT_FORMAT event failed.  mkstdbin_cb downgrades such an error 
          from a terminal error to a warning, but an error message might still 
          have been queued.  If so, clear it.
       */
       if (err_state())
              err_clear();
       
       /* Has user indicated an output file? */
       if (std_args.output_file)
              {
              pfile = fopen(std_args.output_file, "wb");
              if (!pfile)
                     {
                     free(output_buffer);

                     err_push(ROUTINE_NAME, ERR_CREATE_FILE, 
                               std_args.output_file);
                     err_disp();
                     exit(EXIT_FAILURE);
                     }
              }
       else
              {
              /* If not, write to standard output */
              pfile = stdout;
              }
                     
       /* Rather than using an output format contained in a file, create a
          "dynamic" buffer, write an output format description into it, and
          use that to initialize the data bin's output format
       */
     
       sprintf(output_buffer, "\
ASCII_output_data \"hard-coded in getll.c:main()\"\n\
longitude  1 11 double 6\n\
latitude  13 25 double 6\n"
         );
         
  /* Use the FORMAT_BUFFER event to set the output format.  The data bin
     knows that this is an output format because of the format type,
     "ASCII_output_data".
  */
       if (db_set(input,
                   FORMAT_BUFFER, output_buffer, NULL, NULL,
                   END_ARGS
                 )
          )
              {
       /* Error in the output format creation -- this must never happen!
          Ensure that the output buffer is syntactically correct, since it
          is hard-coded into the program!
       */
              free(output_buffer);
              if (std_args.output_file)
                     fclose(pfile);

              err_push(ROUTINE_NAME, ERR_MAKE_FORM, output_buffer);
              err_disp();
              exit(EXIT_FAILURE);
              }
       
       /* Display some information about the data formats */
       db_show(input, FORMAT_LIST, FFF_INFO, END_ARGS);
       /* db_show writes into data bin's working buffer */
       fprintf(stderr, "%s", input->buffer);

       /*
       ** process the data
       */
     
       /* use PROCESS_FORMAT_LIST to fill cache and fill headers */
       while ((error = db_events(input,
                                     PROCESS_FORMAT_LIST, FFF_ALL_TYPES,
                                     END_ARGS
                                 )
               ) == 0
              )
              {
              /* Make sure output buffer is large enough for the cache */
              db_show(input, 
                          DBIN_BYTE_COUNTS, DBIN_OUTPUT_CACHE, &output_bytes, 
                       END_ARGS, END_ARGS
                     );

              if ((unsigned long)output_bytes > (unsigned long)UINT_MAX)
                     {
                     error = 1;
                     err_push(ROUTINE_NAME, ERR_MEM_LACK, 
                               "reallocation size too big");
                     break;
                     }
              if (output_bytes > DEFAULT_CACHE_SIZE)
                     {
                     /* The default cache size was too small for the number of 
                        output bytes needed.  This contigency is coded for, but 
                        is extremely unlikely to happen.  However, it is possible 
                        that the program will error out if it can not resize the 
                        output buffer.
                     */
                     char *cp = NULL;
                     
                     cp = (char *)realloc(output_buffer, (size_t)output_bytes);
                     if (cp)
                            {
                            output_buffer = cp;
                            }
                     else
                            {
                            error = 1;
                            err_push(ROUTINE_NAME, ERR_MEM_LACK,
                                     "reallocation of output_buffer");
                            break;
                            }
                     }
         
              /* Convert the cache into the output_buffer -- this will perform a 
                 binary to ASCII conversion if necessary.  (The example shows 
                 getll running on llmaxmin.dat, an ASCII file, but this program 
                 works equally well on llmaxmin.bin, which is created by running 
                 newform on llmaxmin.dat.)
              */
              error = db_events(input,
                                    DBIN_DATA_TO_NATIVE, NULL, NULL, NULL,
                                    DBIN_CONVERT_CACHE, output_buffer, NULL, 
                                    &output_bytes, 
                                END_ARGS
                               );
              if (error)
                     break;

              /* Write the contents of the output buffer to the file
                 (or screen).
              */
              if (  (long)fwrite(output_buffer, sizeof(char), 
                                 (size_t)output_bytes, pfile)
                  < output_bytes
                 )
                     {
                     err_push(ROUTINE_NAME, ERR_WRITE_FILE,  std_args.output_file
                                                           ? std_args.output_file
                                                           : "to screen"
                             );
                     break;
                     }
              }/* End Processing */

       if (std_args.output_file)
              fclose(pfile);

       free(output_buffer);
       
       /* Deallocate all memory assocated with the data bin */
       db_free(input);

       /* The error stack is checked to see if anything went wrong during 
          processing
       */
       if ((error && error != EOF) || err_state())
              {
              err_push(ROUTINE_NAME, ERR_PROCESS_DATA, NULL);
              err_disp();
              exit(EXIT_FAILURE);
              }

       } /* end main() for program getll */
\end{verbatim}
\T\egroup

\subsubsection{Using getll}

In this example, you will use \lit{getll} to extract latitude and
longitude values from the ASCII data file \lit{latlon.dat}. Their
native format is signed decimal degrees, so no conversion takes place.
Enter the following command:

\begin{example}
getll latlon.dat 
\end{example}

This command prints format summary information and a list of longitude
and latitude values to the screen:

\begin{vcode}{ib}
-176.161101   -47.303545
   0.777265    -0.928001
  35.591879   -28.286662
 149.408117    12.588231
       .
       .
       .
\end{vcode}

As another example, use the following command to extract the latitude
and longitude values from the file \lit{calif.tap}.

\begin{example}
getll calif.tap -f eqtape.fmt 
\end{example}

The latitude and longitude values are extracted and converted from
their native representation as absolute values with quadrant
indicated. You should see the following signed decimal degree values
written to the screen:

\begin{vcode}{ib}
-121.815000     37.852000 
-121.740000     37.737000 
-116.550000     33.517000 
       .
       .
       . 
\end{vcode}

Writing out latitude and longitude values to standard output (the
screen) is not a very impressive feat, but you could create a similar
program to use as the front end for a graphics package. In that case,
you might want to include a third output column which contains the
values of a third variable. For example, with a seismological
application, you might want to include values for magnitude or depth.
You can easily add the third column to the \lit{getll} program by
changing the \lit{sprintf} statement, which creates the compile-time
format, so it is similar to the following:

\begin{vcode}{xib}
sprintf(output_buffer, 
        "longitude 1 8 double 6\nlatitude 9 16 double 6\n %s 17 24 double 2", 
        *(argv+2)); 
\end{vcode}

The getllvar program incorporates a more general version of the
sprintf statement and several other small changes to the getll code;
see \lit{getllvar.c}. Now the second command line argument is the name
of the third variable (only with getllvar; this is at variance with
standard FreeForm ND command line syntax). Try the following command:

\begin{example}
getllvar calif.tap depth -f eqtape.fmt 
\end{example}

You should see the following output with values for depth given in the
third column:

\begin{vcode}{ib}
-121.815000     37.852000          11
-121.740000     37.737000          15
-116.550000     33.517000           6
-125.033000     40.600000           5
-118.840000     37.600000           5
-118.875000     37.609000          24
-118.832000     37.527000          12
-118.820000     37.569000          15
     .
     .
     . 
\end{vcode}

If you enter the following command: 

\begin{example}
getllvar calif.tap year -f eqtape.fmt 
\end{example}

the year will be shown in the third column. 

%%% Local Variables: 
%%% mode: latex
%%% TeX-master: t
%%% End: 
