%CHAPTER 5      
%
% $Id$
%

\chapter{FreeForm ND Conventions}
\label{ff,convs}

\indc{FreeForm conventions}
File name conventions have been defined for FreeForm ND. If you follow
these conventions, FreeForm ND can locate format files through a
default search sequence. Using the file name conventions also lets you
reduce the number of arguments on the command line. In addition to
standard file names, FreeForm ND programs recognize various standard
command line arguments.

\section{File Name Conventions}
\label{ff,convs,filename}

\indc{file name!conventions}\indc{conventions!file name}
Naming conventions have been established for files accessed by
FreeForm ND. Although you are not required to follow these
conventions, using them lets you enter abbreviated commands when you
are using FreeForm ND-based programs. FreeForm ND can then
automatically execute several operations:

\begin{itemize}
\item   Determination of input and output formats when they are not
  explicitly identified in the relevant format descriptions in format
  files
  
\item   Location of format files when they are not specified on the
  command line
\end{itemize}

\section{File Name Extensions}
\label{ff,convs,extension}

\indc{file name!extensions}\indc{extensions!file name}
\indc{dat@\lit{.dat}!file name extension}
\indc{dab@\lit{.dab}!file name extension}
\indc{bin@\lit{.bin}!file name extension}
The expected extensions for data files are as follows: 

\begin{description}

\item[\lit{.dat}]  For ASCII, e.g., \lit{latlon.dat}

\item[\lit{.dab}] For dBASE, e.g., \lit{latlon.dab}

\item[\lit{.bin}] binary or anything that is not \lit{.dat} or \lit{.dab}, e.g., \lit{latlon.bin} 
\end{description}

\indc{fmt@\lit{.fmt}!file name extension}
The expected extension for format description files is \lit{.fmt}, e.g.,
\lit{latlon.fmt}. You should not use mixed case extensions for format
description files if you want to take advantage of FreeForm ND's
default search capabilities. If you explicitly specify the names of
format description files on the command line, you can use mixed case
extensions.

\indc{afm@\lit{.afm}!obsolete file name extension}
\indc{bfm@\lit{.bfm}!obsolete file name extension}
\indc{dfm@\lit{.dfm}!obsolete file name extension}
\indc{file name!extensions!obsolete}
\indc{extensions!file name!obsolete}
\indc{obsolete!file name extensions}
\note{Previous versions of FreeForm ND used variable description files
  (formerly called format specification files) each of which contained
  variable descriptions for one file. Expected extensions for these
  files were \lit{.afm} (ASCII), \lit{.bfm} (binary), and \lit{.dfm}
  (dBASE). Variable descriptions for one or more files can now be
  incorporated into a single format description file. It is
  recommended that you convert and combine (as appropriate) existing
  variable description files into format description files.  }

\section{File Name Relationships}
\label{ff,convs,relation}


FreeForm ND-based programs expect certain relationships between data
file and format description file names as outlined below.

\begin{itemize}
\item   The data file is named datafile.ext where datafile is the file
  name of your choosing and ext is the extension.
  
  Example: \lit{latlon.dat}
  
\item   The corresponding format description file should be named
  datafile.fmt.
  
  Example: \lit{latlon.fmt}
  
\item   If one format description file is used for multiple data
  files, all with the same extension, the format description file
  should be named ext.fmt.
  
  Example: \lit{ll.fmt} is the format description file for \lit{lldat1.ll},
  \lit{lldat2.ll}, and \lit{lldat3.ll}.
\end{itemize}

Again, although not required, it is to your advantage to use these
conventions.

\section{Determining Input and Output Formats}
\label{ff,convs,determine}

\indc{input format!determining}\indc{determining!input format}
\indc{output format!determining}\indc{determining!output format}
You can optionally include the read/write type (``\lit{input}'' or
``\lit{output}'') in format descriptors, e.g.,
\lit{ASCII\_input\_data}. You may not want to specify the read/write
type in some circumstances. For example, you may need to translate
from native ASCII to binary, then back to ASCII. ASCII is the input
format in the first translation and the output format in the second
translation, vice versa for binary. You would need to edit the format
description file before executing the second translation if you
included read/write type in the format descriptors.

\note{If you use the -ft option, you do not need to edit the format
  description file. See \sectionref{ff,convs,specifying} later
  in this chapter.}
 
If you do not specify read/write type, FreeForm ND can nevertheless
determine which format in a format description file is input and which
is output as long as you have adhered to FreeForm ND filenaming
conventions.

\begin{itemize}
\item   If the input format is not specified, and 
  \begin{itemize}
    
  \item the input data filename extension is \lit{.bin}, assume binary
    input.
  \item the input data filename extension is \lit{.dab}, assume
    dBASE input.
  \item the input data filename extension is \lit{.dat}, assume ASCII input.
  \item the input data filename extension is anything else, assume
    binary input.
\end{itemize}

\item   If the output format is not specified, and 
  \begin{itemize}
  \item the input format is binary, the output is ASCII or dBASE,
    whichever is found first.
  \item the input format is dBASE, the output is ASCII or binary,
    whichever is found first.
  \item the input format is ASCII, the output is binary or dBASE,
    whichever is found first.  
\end{itemize}
\end{itemize}

\note{The appropriate format descriptions must be in the format
  description file(s) used by FreeForm ND for a translation. If, for
  example, FreeForm ND determines the input format is binary and the
  output format is ASCII, there must be a format description for each
  type.  }

The checkvar program needs only an input format. 

\section{Locating Format Files}
\label{ff,convs,locating}

\indc{format files!locating}\indc{locating!format files}
\indc{format files!search path}\indc{search path!format files}
FreeForm ND programs use the following search sequence to find a
format file (format or variable description file) for the data file
datafile.ext when the format file name is not explicitly specified on
the command line. In summary, FreeForm ND searches the directory
specified by the GeoVu keyword \lit{format_dir} (defined in a equivalence
table or in the environment), the current or working directory, and
the data file's home directory. The rules are applied in the order
given below until a format file is found or all rules have been
exhausted. If the relevant format file does not follow FreeForm ND
conventions for name or location, it should be explicitly specified on
the command line.

\note{GeoVu is a FreeForm ND-based application for data access and
  visualization. FreeForm ND applications other than GeoVu use GeoVu
  keywords.  }

For information about equivalence tables, see the GeoVu Tools
Reference Guide, available from the NGDC.

\subsection{Search Sequence}

\begin{enumerate}
\item Search the directory given by the GeoVu keyword \lit{format_dir} for a
  format description file named datafile.fmt.
  
\item Search the directory given by the GeoVu keyword \lit{format_dir} for
  variable description files named \lit{datafile.afm}, \lit{datafile.bfm}, and
  \lit{datafile.dfm}.
  
  \note{Step 2 is included to accommodate variable description files
    that were created using previous versions of FreeForm ND. It is
    recommended that you convert existing variable description files
    to format description files.  }
  
\item Search the directory given by the GeoVu keyword \lit{format_dir} for a
  format description file named ext.fmt.
  
  If the GeoVu keyword \lit{format_dir} is not found, FreeForm ND continues
  the search for a format file as follows.
  
\item Search the current (default) directory for a format description
  file named datafile.fmt.
  
\item Search the current directory for variable description files
  named \lit{datafile.afm}, \lit{datafile.bfm}, and \lit{datafile.dfm}. Use the criteria
  in step 2 for determining input and output format files.
  
\item Search the current directory for a format description file named
  \lit{ext.fmt}.
  
  If the data file's home directory is not the same as the current
  directory, FreeForm ND continues the search for a format file with
  steps 7-9. The data file's home directory is given by the directory
  path component of the data file name. If the data file name has no
  directory path component, the home directory search is not done.
  
\item Search the data file's home directory for a format description
  file named \lit{datafile.fmt}.
  
\item Search the data file's home directory for variable description
  files named \lit{datafile.afm}, \lit{datafile.bfm}, and \lit{datafile.dfm}. Use the
  criteria in step 2 for determining input and output format files.
  
\item Search the data file's home directory for a format description
  file named \lit{ext.fmt}.
\end{enumerate}


\subsection{Case Sensitivity}
\label{ff,convs,case}

\indc{case sensitivity!locating format files}
\indc{format files!locating!case sensivity}
\indc{locating!format files!case sensivity}
\indc{format files!search path}
\indc{search path!format files!case sensivity}
\indc{file name!case sensivity}\indc{case sensivity!file name}
FreeForm ND adheres to the following rules for case sensitivity (in
applicable operating systems) when it searches for a format file for
the data file datafile.ext.

\begin{itemize}
\item FreeForm ND preserves the case of datafile, for example, the
  default format file for the data file \lit{LATLON.BIN} is \lit{LATLON.fmt} (or
  \lit{LATLON.bfm}).
  
\item FreeForm ND searches for a format file with a lower case
  extension. That is, the format file must have its extension in lower
  case no matter what the case of datafile. For example, the default
  format file for the data file \lit{LatLon.dat} is \lit{LatLon.fmt} (or
  \lit{LatLon.afm}), and \lit{TIMEDATE.fmt} (or \lit{TIMEDATE.bfm}) is the default
  format file for \lit{TIMEDATE.bin}.
  
\item In searching for a format description file of type \lit{ext.fmt},
  FreeForm ND preserves the case of ext. For example, for data files
  named \lit{lldat1.LL}, \lit{lldat2.LL}, and \lit{latlon3.LL}, the default format
  description file is \lit{LL.fmt}.
\end{itemize}

\section{Command Line Arguments}
\label{ff,convs,argument}

\indc{command line arguments}\indc{arguments!command line}
FreeForm ND programs can take various command line arguments. The most
widely used or standard arguments are discussed in this section. They
are used for several different purposes: identifying input and output
files, identifying format files and titles, changing run-time
operation parameters, and defining data filters.

The only required argument for any FreeForm ND program is the name of
the input file or file to be processed. All other arguments are
optional and can be in any order following the input file name. The
command line of a FreeForm ND program with the standard arguments has
the following form:

\newcommand{\dashf}{\hbox{\lit{-f} \var{format\_file}}}
\newcommand{\dashif}{\hbox{\lit{-if} \var{input\_format\_file}}}
\newcommand{\dashof}{\hbox{\lit{-of} \var{output\_format\_file}}}
\newcommand{\dashft}{\hbox{\lit{-ft "}\var{title}\lit{"}}}
\newcommand{\dashift}{\hbox{\lit{-ift "}\var{title}\lit{"}}}
\newcommand{\dashoft}{\hbox{\lit{-oft "}\var{title}\lit{"}}}
\newcommand{\dashb}{\hbox{\lit{-b} \var{local\_buffer\_size}}}
\newcommand{\dashc}{\hbox{\lit{-c} \var{count}}}
\newcommand{\dashvv}{\hbox{\lit{-v} \var{var\_file}}}
\newcommand{\dashq}{\hbox{\lit{-q} \var{query\_file}}}
\newcommand{\dasho}{\hbox{\lit{-o} \var{output\_file}}}

\proto{application\_name \var{input\_file} [\dashf] 
[\dashif] [\dashof] [\dashft] 
[\dashift] [\dashoft] [\dashb] [\dashc] 
[\dashvv] [\dashq] [\dasho]}

\note{To see a summary of command line usage for a FreeForm ND
  program, enter the program's name on the command line without any
  arguments.  }

\subsection{Specifying Input and Output Files}

\indc{arguments!input file}\indc{input file!specifying}
\indc{specifying!input file}
\indc{arguments!output file}\indc{output file!specifying}
\indc{specifying!output file} 
\begin{description}

\item[\var{input\_file}]
  
  Name of the file to be processed. Following FreeForm ND naming
  conventions, the standard extensions for data files are \lit{.dat} for
  ASCII format, \lit{.bin} for binary, and \lit{.dab} for dBASE.

\item[\dasho]\indc{o@\lit{-o}}
  
  Option flag followed by the name of the output file. The standard
  extensions are the same as for input files.
\end{description}

\subsection{Specifying Format Description Source}
\label{ff,convs,specifying}

\indc{arguments!input format file}
\indc{input format file!specifying}\indc{specifying!input format file}
\indc{arguments!output format file}
\indc{output format file!specifying}\indc{specifying!output format file}
\indc{format file!specifying}
FreeForm ND offers a number of command line options for specifying the
source of the format descriptions that a program must find in order to
process data. The proper option or combination of options to use
depends on how you have constructed your format files.

\begin{description}

\item[\dashf]\indc{f@\lit{}}
  
  Option flag followed by the name of the format description file
  describing both input and output data.

\item[\dashif]\indc{if@\lit{-if}}
  
  Option flag followed by the name of the format description file
  describing the input data. Also use this option for an input
  variable description file written using earlier versions of FreeForm
  ND.

\item[\dashof]\indc{of@\lit{-of}}
  
  Option flag followed by the name of the format description file
  describing the output data. Also use this option for an output
  variable description file written using earlier versions of FreeForm
  ND.

\item[\dashft]\indc{ft@\lit{-ft}}
  
\indc{arguments!input format title}
\indc{input format title!specifying}\indc{specifying!input format title}
\indc{arguments!output format title}
\indc{output format title!specifying}\indc{specifying!output format title}
\indc{format title!specifying}\indc{arguments!format title}
\indc{title!format description}
  Option flag followed by the title (enclosed in quotes) of the format
  to be used for both input and output data, in which case there is no
  reformatting. The title follows format type on the first line of a
  format description in a format description file.

\item[\dashift]\indc{ift@\lit{-ift}}
  
  Option flag followed by the title (enclosed in quotes) of the
  desired input format.

\item[\dashoft]\indc{oft@\lit{-oft}}
  
  Option flag followed by the title (enclosed in quotes) of the
  desired output format.
\end{description}

\note{Previous versions of FreeForm ND used variable description files
  (\lit{.afm}, \lit{.bfm}, \lit{.dfm}). It is recommended that you
  convert and combine (as appropriate) existing variable description
  files into format description files.  }

The various options available for specifying the source of a format
description offer you a great deal of flexibility-in naming files,
setting up format description files, and on the command line. In using
these options, you need to consider the content of your format
description files and how FreeForm ND will interpret the arguments on
the command line.

\subsection{Changing Run-time Parameters}

\indc{arguments!run-time parameters}
\indc{run-time parameters!arguments}
FreeForm ND includes three arguments that let you change run-time
parameters according to your needs. One argument lets you specify
local buffer size, another indicates the number of records to process,
and the third indicates which variables to process.

\begin{description}

\item[\dashb]\indc{b@\lit{-b}}
  
\indc{buffer size}\indc{memory buffer}\indc{allocated memory}
  Option flag followed by the size of the memory buffer used to
  process the data and format files.
  
  Default buffer size is 32,768. You many want to decrease the buffer
  size if you are running with low memory. Keep in mind that too small
  a buffer may result in unexpected behavior.

\item[\dashc]\indc{c@\lit{-c}}

  \indc{records!how many to process}
  Option flag followed by a number that specifies how many data
  records at the head or tail of the file to process.
  
  If $count > 0$, \var{count} records at the beginning of the file are
  processed.
  
  If $count < 0$, \var{count} records at the tail or end of the file
  are processed.

\item[\dashvv]\indc{v@\lit{-v}}
  
  \indc{arguments!variable file}\indc{specifying!variables}
  \indc{variables!specifying}
  Option flag followed by the name of a variable file. The file
  contains names of the variables in the input data file to be
  processed by the FreeForm ND program. Variable names in var_file can
  be separated by one or more spaces or each name can be on a separate
  line.
\end{description}

\subsection{Defining Filters}

\indc{arguments!filters}\indc{filters!arguments}
The query option lets you define data filters via a query file so you
can precisely specify which data to process. The FreeForm ND program
will process only those records meeting the query criteria.

\begin{description}

\item[\dashq]\indc{q@\lit{-q}}
  
  Option flag followed by the name of the file containing query
  criteria. See \chapterref{ff,query} for a complete description of
  the query syntax.
\end{description}

%%% Local Variables: 
%%% mode: latex
%%% TeX-master: t
%%% End: 
