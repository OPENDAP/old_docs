%CHAPTER 8       
%
% $Id$
%

\chapter{Header Formats}
\label{ff,hdrfmt}

\indc{header!format}\indc{format!header}
Headers are one of the most commonly encountered forms of
metadata-data about data. Applications need the information contained
in headers for reading the data that the headers describe. To access
these data, applications must be able to read the headers. Just as
there are many data formats, there are numerous header formats. You
can include header format descriptions, which have exactly the same
form as data format descriptions, in format description files.

\section{Header Treatment in FreeForm ND}

\indc{header!in FreeForm ND}
FreeForm ND is not 100 percent backwards compatible with FreeForm in
the area of header treatment.

Headers have traditionally been handled differently from data in
FreeForm ND. If a header format was not specified as either input or
output, it was taken as both input and output. \lit{newform} did
little in processing headers, and FreeForm ND relied on extraneous
utilities to work with headers.

\subsection{New Behavior}

In FreeForm ND, header formats are treated the same as data formats.
This means that header formats must be identified as either input or
output, explicitly or implicitly. If done explicitly, then either the
input or the output descriptor will form the format type (e.g.,
\lit{ASCII\_input\_header}). If done implicitly, then the same
ambiguity resolution rules that apply to data formats will be applied
to header formats. This means that ASCII header formats will be taken
as input for data files with a \lit{.dat} extension, dBASE header
formats will be taken as input for data files with a \lit{.dab}
extension, and binary header formats will be taken as input for all
other data files.

If an embedded header and the data have different file types, then
either the header format or data format (preferably both) must be
explicitly identified as input or output (for example, an ASCII header
embedded in a binary data file). Obviously, ambiguous formats with
different file types cannot both be resolved as input formats.

The same header format is no longer used as both an input and an
output header format.

In FreeForm ND, \lit{newform} honors output header formats that are
separate (e.g., \lit{ASCII\_output\_header\_separate}). The header is
written to a separate file which, unless otherwise specified, is named
after the output data file with a \lit{.hdr} extension. This requires
that you name the output file using the \lit{-o} option flag; redirected
output cannot be used with separate output headers. The output header
file name and path can be specified using the same keywords that tell
FreeForm ND how to find an input separate header file (i.e.,
\lit{header\_file\_ext}, \lit{header\_file\_name}, and
\lit{header\_file\_path}).

When defining keywords to specify how an output header file is to be
named, you must use a new type of equivalence section,
\lit{input\_eqv}, which must appear in the format file along with
\lit{output\_eqv}.

\section{Header Types}

\indc{header!type}\indc{type!header}
FreeForm ND recognizes two types of headers. File headers describe all
the data in a file whereas record headers describe the data in a
single record or data block. FreeForm ND can read headers included in
the data file or stored in a separate file. Header formats, like data
formats, are described in format description files. For a list of the
header descriptors you can use in format descriptions, see
\chapterref{ff,tblfmt}. 

\subsection{File Headers}

\indc{file header}\indc{header!file}
A file header included in a data file is at the beginning of the file.
Only one file header can be associated with a data file.
Alternatively, a file header can be stored in a file separate from the
data file.

In the following example, a file header is used to store the minimum
and maximum for each variable and the data are converted from ASCII to
binary. There are two variables, latitude and longitude. The file
header format and data formats are described in the format description
file \lit{llmaxmin.fmt}.

Here is \lit{llmaxmin.fmt}:

\begin{vcode}{ib}
ASCII_file_header "Latitude/Longitude Limits"
minmax_title 1 24 char 0
latitude_min 25 36 double 6
latitude_max 37 46 double 6
longitude_min 47 59 double 6
longitude_max 60 70 double 6

ASCII_data "lat/lon"
latitude 1 10 double 6
longitude 12 22 double 6

binary_data "lat/lon"
latitude 1 4 long 6
longitude 5 8 long 6 
\end{vcode}

The example ASCII data file \lit{llmaxmin.dat} contains a file header
and data as described in \lit{llmaxmin.fmt}.

\lit{llmaxmin.dat}:

\begin{vcode}{xib}
         1         2         3         4         5         6         7
1234567890123456789012345678901234567890123456789012345678901234567890

Latitude and Longitude:   -83.223548 54.118314  -176.161101 149.408117
-47.303545 -176.161101
-25.928001    0.777265
-28.286662   35.591879
 12.588231  149.408117
-83.223548   55.319598
 54.118314 -136.940570
 38.818812   91.411330
-34.577065   30.172129
 27.331551 -155.233735
 11.624981 -113.660611 
\end{vcode}

This use of a file header would be appropriate if you were interested
in creating maps from large data files. By including maximums and
minimums in a header, the scale of the axes can be determined without
reading the entire file.

FreeForm ND naming conventions have been followed in this example, so
to convert the ASCII data in the example to binary format, use the
following simple command:

\begin{example}
newform llmaxmin.dat -o llmaxmin.bin 
\end{example}

The file header in the example will be written into the binary file as
ASCII text because the header descriptor in \lit{llmaxmin.fmt}
(\lit{ASCII\_file\_header}) does not specify read/write type, so the
format is used for both the input and output header.

\subsection{Record Headers}

\indc{record header}\indc{header!record}
Record headers occur once for every block of data in a file. They are
interspersed with the data, a configuration sometimes called a format
sandwich. Record headers can also be stored together in a separate
file.

The following format description file specifies a record header and
ASCII and binary data formats for aeromagnetic trackline data.

Here is \lit{aeromag.fmt}:

\begin{vcode}{xib}
ASCII_record_header "Aeromagnetic Record Header Format"
flight_line_number 1 5 long 0
count 6 13 long 0
fiducial_number_corresponding_to_first_logical_record 14 22 long 0
date_MMDDYY_or_julian_day 23 30 long 0
flight_number 31 38 long 0
utm_easting_of_first_record 39 48 float 0
utm_northing_of_first_record 49 58 float 0
utm_easting_of_last_record 59 68 float 0
utm_northing_of_last_record 69 78 float 0
blank_padding 79 104 char 0

ASCII_data "Aeromagnetic ASCII Data Format"
flight_line_number 1 5 long 0
fiducial_number 6 15 long 0
utm_easting_meters 16 25 float 0
utm_northing_meters 26 35 float 0
mag_total_field_intensity_nT 36 45 long 0
mag_residual_field_nT 46 55 long 0
alt_radar_meters 56 65 long 0
alt_barometric_meters 66 75 long 0
blank 76 80 char 0
latitude 81 92 float 6
longitude 93 104 float 6

binary_data "Aeromagnetic Binary Data Format"
flight_line_number 1 4 long 0
fiducial_number 5 8 long 0
utm_easting_meters 9 12 long 0
utm_northing_meters 13 16 long 0
mag_total_field_intensity_nT 17 20 long 0
mag_residual_field_nT 21 24 long 0
alt_radar_meters 25 28 long 0
alt_barometric_meters 29 32 long 0
blank 33 37 char 0
latitude 38 41 long 6
longitude 42 45 long 6 
\end{vcode}

The example ASCII file \lit{aeromag.dat} contains two record headers
followed by a number of data records. The header and data formats are
described in \lit{aeromag.fmt}. The variable count (second variable
defined in the header format description) is used to indicate how many
data records occur after each header.

\lit{aeromag.dat}: 

\begin{vcode}{zib}
         1         2         3         4         5         6         7         8         9         10
123456789012345678901234567890123456789012345678901234567890123456789012345678901234567890123456789012345
  420       5     5272     178       2   413669.  6669740.   333345.  6751355.                       
  420      5272   413669.  6669740.   2715963   2715449      1088      1348        60.157307 -154.555191
  420      5273   413635.  6669773.   2715977   2715464      1088      1350        60.157593 -154.555817
  420      5274   413601.  6669807.   2716024   2715511      1088      1353        60.157894 -154.556442
  420      5275   413567.  6669841.   2716116   2715603      1079      1355        60.158188 -154.557068
  420      5276   413533.  6669875.   2716263   2715750      1079      1358        60.158489 -154.557693
  411      10     8366     178       2   332640.  6749449.   412501.  6668591.                         
  411      8366   332640.  6749449.   2736555   2736538       963      1827        60.846806 -156.080185
  411      8367   332674.  6749415.   2736539   2736522       932      1827        60.846516 -156.079529
  411      8368   332708.  6749381.   2736527   2736510       917      1829        60.846222 -156.078873
  411      8369   332742.  6749347.   2736516   2736499       922      1832        60.845936 -156.078217
  411      8370   332776.  6749313.   2736508   2736491       946      1839        60.845642 -156.077560
  411      8371   332810.  6749279.   2736505   2736488       961      1846        60.845348 -156.076904
  411      8372   332844.  6749245.   2736493   2736476       982      1846        60.845062 -156.076248
  411      8373   332878.  6749211.   2736481   2736463      1015      1846        60.844769 -156.075607
  411      8374   332912.  6749177.   2736470   2736452      1029      1846        60.844479 -156.074951
  411      8375   332946.  6749143.   2736457   2736439      1041      1846        60.844189 -156.074295 
\end{vcode}

This file contains two record headers. The first occurs on the first
line of the file and has a count of 5, so it is followed by 5 data
records. The second record header follows the first 5 data records. It
has a count of 10 and is followed by 10 data records.

The FreeForm ND default naming conventions have been used here so you
could use the following abbreviated command to reformat
\lit{aeromag.dat} to a binary file named \lit{aeromag.bin}:

\begin{example}
newform aeromag.dat -o aeromag.bin 
\end{example}

The ASCII record headers are written into the binary file as ASCII
text.

\subsection{Separate Header Files}

You may need to describe a data set with external headers. An external or separate header file can contain only headers-one file header or multiple record headers. 

\subsubsection{Separate File Header}

\indc{file header!separate}\indc{header!file!separate}
\indc{separate!file header}
Suppose you want the file header used to store the minimum and maximum
values for latitude and longitude (from the llmaxmin example) in a
separate file so that the data file is homogenous, thus easier for
applications to read. Instead of one ASCII file (\lit{llmaxmin.dat}),
you will have an ASCII header file, say it is named \lit{llmxmn.hdr},
and an ASCII data file-call it \lit{llmxmn.dat}.

Here is \lit{llmxmn.hdr}:

\begin{vcode}{ib}
Latitude and Longitude:   -83.223548 54.118314  -176.161101 149.408117 
\end{vcode}

And here is \lit{llmxmn.dat}:

\begin{vcode}{ib}
-47.303545 -176.161101
-25.928001    0.777265
-28.286662   35.591879
 12.588231  149.408117
-83.223548   55.319598
 54.118314 -136.940570
 38.818812   91.411330
-34.577065   30.172129
 27.331551 -155.233735
 11.624981 -113.660611 
\end{vcode}

You will need to make one change to \lit{llmaxmin.fmt}, adding the
qualifier separate to the header descriptor, so that FreeForm ND will
look for the header in a separate file. The first line of
\lit{llmaxmin.fmt} becomes:

\begin{vcode}{ib}
ASCII_file_header_separate "Latitude/Longitude Limits" 
\end{vcode}

Save \lit{llmaxmin.fmt} as \lit{llmxmn.fmt} after you make the change.

To convert the data in \lit{llmxmn.dat} to binary format in
\lit{llmxmn.bin}, use the following command:

\begin{example}
newform llmxmn.dat -o llmxmn.bin 
\end{example}

\note{When you run \lit{newform}, it will write the separate header to
  \lit{llmxmn.bin} along with the data in \lit{llmxmn.dat}.  }

\subsubsection{Separate Record Headers}

\indc{record header!separate}\indc{header!record!separate}
\indc{separate!header record}
Record headers in separate files can act as indexes into data files if
the headers specify the positions of the data in the data file. For
example, if you have a file containing data from 25 observation
stations, you could effectively index the file by including a station
ID and the starting position of the data for that station in each
record header. Then you could use the index to quickly locate the data
for a particular station.

Returning to the \lit{aeromag} example, suppose you want to place the
two record headers in a separate file. Again, the only change you need
to make to the format description file (\lit{aeromag.fmt}) is to add
the qualifier separate to the header descriptor. The first line would
then be:

\begin{vcode}{ib}
ASCII_record_header_separate "Aeromagnetic Record Header Format" 
\end{vcode}

The separate header file would contain the following two lines: 

\begin{vcode}{xib}
420       5     5272     178       2   413669.  6669740.   333345.  6751355.
411      10     8366     178       2   332640.  6749449.   412501.  6668591. 
\end{vcode}

The data file would look like the current \lit{aeromag.dat} with the
first and seventh lines removed.

Assuming the data file is named \lit{aeromag.dat}, the default name
and location of the header file would be \lit{aeromag.hdr} in the same
directory as the data file. Otherwise, the separate header file name
and location need to be defined in an equivalence table. (For
information about equivalence tables, see the GeoVu Tools Reference
Guide.)

\subsection{The dBASEfile Format}

\indc{dBASE!file type}\indc{file type!dBASE}
Headers and data records in dBASE format are represented in ASCII but
are not separated by end-of-line characters. They can be difficult to
read or to use in applications that expect newlines to separate
records. By using \lit{newform}, dBASE data can be reformatted to
include end-of-line characters.

In this example, you will reformat the dBASE data file
\lit{oceantmp.dab} (see below) into the ASCII data file
\lit{oceantmp.dat}. The input file \lit{oceantmp.dab} contains a
record header at the beginning of each line. The header is followed by
data on the same line. When you convert the file to ASCII, the header
will be on one line followed by the data on the number of lines
specified by the variable count. The format description file
\lit{oceantmp.fmt} is used for this reformatting.

Here is \lit{oceantmp.fmt}:

\begin{vcode}{sib}
dbase_record_header "NODC-01 record header format"
WMO_quad 1 1 char 0
latitude_deg_abs 2 3 uchar 0
latitude_min 4 5 uchar 0
longitude_deg_abs 6 8 uchar 0
longitude_min 9 10 uchar 0
date_yymmdd 11 16 long 0
hours 17 19 uchar 1
country_code 20 21 char 0
vessel 22 23 char 0
count 24 26 short 0
data_type_code 27 27 char 0
cruise 28 32 long 0
station 33 36 short 0

dbase_data "IBT input format"
depth_m 1 4 short 0
temperature 5 8 short 2

RETURN "NEW LINE INDICATOR"

ASCII_data "ASCII output format"
depth_m 1 5 short 0
temperature 27 31 float 2 
\end{vcode}

This format description file contains a header format description, a
description for dBASE input data, the special RETURN descriptor, and a
description for ASCII output data. The variable \var{count} (fourth
from the bottom in the header format description) indicates the number
of data records that follow each header. The descriptor RETURN lets
\lit{newform} skip over the end-of-line marker at the end of each data
block in the input file \lit{oceantmp.dab} as it is meaningless to
\lit{newform} here. Because the end-of-line marker appears at the end
of the data records in each input data block, RETURN is placed after
the input data format description in the format description file.

\lit{oceantmp.dab}:

\begin{vcode}{xib}
         1         2         3         4         5         6         7
1234567890123456789012345678901234567890123456789012345678901234567890
11000171108603131109998  4686021000000002767001027670020276700302767
110011751986072005690AM  4686091000000002928001028780020287200302872
11111176458102121909998  4681011000000002728009126890241110005000728
112281795780051918090PI  268101100000000268900402711 
\end{vcode}

Each dBASE header in \lit{oceantmp.dab} is located from position 1 to
36. It is followed by four data records of 8 bytes each. Each record
comprises a depth and temperature reading. The variable count in the
header (positions 24-26) indicates that there are 4 data records each
in the first 3 lines and 2 on the last line. This will all be more
obvious after conversion.

To reformat \lit{oceantmp.dab} to ASCII, use the following command: 

\begin{example}
newform oceantmp.dab -o oceantmp.dat 
\end{example}

The resulting file \lit{oceantmp.dat} is much easier to read. It is
readily apparent that there are 4 data records after the first three
headers and 2 after the last.

Here is \lit{oceantmp.dat}:

\begin{vcode}{ib}
         1         2         3         4
1234567890123456789012345678901234567890
11000171108603131109998  46860210000
    0                     27.67
   10                     27.67
   20                     27.67
   30                     27.67
110011751986072005690AM  46860910000
    0                     29.28
   10                     28.78
   20                     28.72
   30                     28.72
11111176458102121909998  46810110000
    0                     27.28
   91                     26.89
  241                     11.00
  500                     07.28
112281795780051918090PI  26810110000
    0                     26.89
   40                     27.11
\end{vcode}





%%% Local Variables: 
%%% mode: latex
%%% TeX-master: t
%%% TeX-master: t
%%% End: 
