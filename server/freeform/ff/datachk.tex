%CHAPTER 9       
%
% $Id$
%

\chapter{Data Checking}
\label{ff,datachk}

The FreeForm ND-based utility program \lit{checkvar} creates variable
summary files, lists of maximum and minimum values, and summaries of
processing activity. You can use this information to check data
quality and to examine the distribution of the data.

\section{Generating the Summaries}

\indc{checking formats!checkvar@\lit{checkvar}}
\indc{commands!checkvar@\lit{checkvar}}
\indc{FreeForm ND commands!checkvar@\lit{checkvar}}
\indc{checkvar@\lit{checkvar}!checking formats}
\indc{format checking}
\indc{summary file!generating}\indc{file!summary}
\indc{summary file |see{checkvar}}
A variable summary file (or list file), which contains histogram
information showing the variable's distribution in the data file, is
created for each variable (or designated variables) in the specified
data file. You can optionally specify an output file in which a
summary of processing activity is saved.

Variable summaries (list files) can be helpful for performing quality
control checks of data. For example, you could run \lit{checkvar} on
an ASCII file, convert the file to binary, and then run \lit{checkvar}
on the binary file. The output from \lit{checkvar} should be the same
for both the ASCII and binary files. You can also use variable
summaries to look at the data distribution in a data set before
extracting data.

The \lit{checkvar} command has the following form:     


\newcommand{\dashp}{\hbox{\lit{-p} \var{precision}}}
\newcommand{\dashm}{\hbox{\lit{-m} \var{maxbins}}}
\newcommand{\dashmd}{\hbox{\lit{-md} \var{missing\_data\_flag}}}
\newcommand{\dashmm}{\hbox{\lit{-mm}}}
\renewcommand{\dasho}{\hbox{\lit{-o} \var{processing\_summary}}}

\proto{\lit{checkvar} \var{input\_file} [\dashf] [\dashif] [\dashof] 
[\dashft] [\dashift] [\dashoft] [\dashb] [\dashc] [\dashvv] [\dashq] 
[\dashp] [\dashm] [\dashmd] [\dashmm] [\dasho]}

The \lit{checkvar} program needs to find only an input format
description. Output format descriptions will be ignored. If conversion
variables are included in input or output formats, no conversion is
performed when you run \lit{checkvar}, since it ignores output
formats.

For descriptions of the standard arguments (first eleven arguments
above), see \sectionref{ff,convs,argument}.

\begin{description}

\item[\dashp]\indc{p@\lit{-p}}
  
  Option flag followed by the number of decimal places. The number
  represents the power of 10 that data is multiplied by prior to
  binning. A value of 0 bins on one's, 1 on tenth's, and so on. This
  option allows an adjustment of the resolution of the \lit{checkvar}
  output.
  
  The default is 0; maximum is 5.
  
  \note{If you use the \lit{-p} option on the command line, the precision
    set in the relevant format file is overridden. The precision in
    the format file serves as the default.  }

\item[\dashm]\indc{m@\lit{-m}}
  
  Option flag followed by the approximate maximum number of bins
  desired in \lit{checkvar} output. The \lit{checkvar} program keeps
  track of the number of bins filled as the data is processed. The
  smaller the number of bins, the faster \lit{checkvar} runs. By
  keeping the number of bins small, you can check the gross aspects of
  data distribution rather than the details.
  
  The number of bins is adjusted dynamically as \lit{checkvar} runs
  depending on the distribution of data in the input file. If the
  number of filled bins becomes > 1.5 * maxbins, the width of the bins
  is doubled to keep the total number near the desired maximum.
  
  The default is 100 bins; minimum is 6. Must be < 10,000.
  
  \note{The precision (-p) and maxbins (-m) options have no effect on
    character variables. }


\item[\dashmd]\indc{md@\lit{-md}}
  
  Option flag followed by a flag value that \lit{checkvar} should
  ignore across all variables in creating histogram data. Missing data
  flags are used in a data file to indicate missing or meaningless
  data. If you want \lit{checkvar} to ignore more than one value, use
  the query (\lit{-q}) option in conjunction with the variable file
  (\lit{-v}) option.

\item[\dashmm]\indc{mm@\lit{-mm}}
  
  Option flag indicating that only the maximum and minimum values of
  variables are calculated and displayed in the processing summary.
  Variable summary files are not created.

\item[\dasho]\indc{o@\lit{-o}}
  
  Option flag followed by the name of the file in which summary
  information displayed during processing is stored.

\end{description}


\section{Example}

\indc{summary file!generating!example}\indc{file!summary!example}
You will use \lit{checkvar} with a precision of 3 to create a
processing summary file and summary files for the two variables
latitude and longitude in the file \lit{latlon.dat}.

Here is \lit{latlon.dat}:

\begin{vcode}{ib}
-47.303545 -176.161101
 -0.928001    0.777265
-28.286662   35.591879
 12.588231  149.408117
-83.223548   55.319598
 54.118314 -136.940570
 38.818812   91.411330
-34.577065   30.172129
 27.331551 -155.233735
 11.624981 -113.660611
 77.652742  -79.177679
 77.883119  -77.505502
-65.864879  -55.441896
-63.211962  134.124014
 35.130219 -153.543091
 29.918847  144.804390
-69.273601   38.875778
-63.002874   36.356024
 35.086084  -21.643402
-12.966961   62.152266 
\end{vcode}

To create the summary files, enter the following command: 

\begin{example}
checkvar latlon.dat -p 3 -o latlon.sum 
\end{example}

A summary of processing information and the maximum and minimum for
each variable are displayed on the screen. The following three files
are created:

\begin{itemize}
\item \lit{latlon.sum} recaps processing activity, maximums and minimums 
  
\item \lit{latitude.lst} shows distribution of the latitude values in
  \lit{latlon.dat}
  
\item \lit{longitude.lst} shows distribution of the longitude values
  in \lit{latlon.dat}.
\end{itemize}

\section{Interpreting the Summaries}

\indc{summary file!interpreting}\indc{file!summary}
The processing and variable summary files output by \lit{checkvar}
from the example in the previous section are shown and discussed
below.

\subsection{Processing Summary}

\indc{processing summary}
If you specify an output file on the command line, it stores the
information that is displayed on the screen during processing. The
file \lit{latlon.sum} was specified as the output file in the example
above.

Here is \lit{latlon.sum}:

\begin{vcode}{ib}
Input file : latlon.dat
Requested precision = 3, Approximate number of sorting bins = 100

Input data format       (latlon.fmt)
ASCII_input_data       "ASCII format"
The format contains 2 variables; length is 24.

Output data format       (latlon.fmt)
binary_output_data       "binary format"
The format contains 2 variables; length is 16.

Histogram data precision: 3, Number of sorting bins: 20
latitude: 20 values read
minimum: -83.223548 found at record  5
maximum:  77.883119 found at record 12
Summary file: latitude.lst

Histogram data precision: 3, Number of sorting bins: 20
longitude: 20 values read
minimum: -176.161101 found at record 1
maximum:  149.408117 found at record 4
Summary file: longitude.lst. 
\end{vcode}

The processing summary file \lit{latlon.sum} first shows the name of
the input data file (\lit{latlon.dat}). If you specified precision and
a maximum number of bins on the command line, those values are given
as Requested precision, in this case 3, and Approximate number of
sorting bins, in this case the default value of 100. If precision is
not specified, No requested precision is shown.

A summary of each format shows the type of format (in this case, Input
data format and Output data format) and the name of the format file
containing the format descriptions (\lit{latlon.fmt}), whether
specified on the command line or located through the default search
sequence (as detailed in chapter 4). In this case, it was located by
default. Since \lit{checkvar} only needs an input format description,
it ignores output format descriptions. Next, you see the format
descriptor as resolved by FreeForm ND (e.g., \lit{ASCII_input_data})
and the format title (e.g., ``ASCII format''). Then the number of
variables in a record and total record length are given; for ASCII,
record length includes the end-of-line character (1 byte for Unix).

A section for each variable processed by \lit{checkvar} indicates the
histogram precision and actual number of sorting bins. Under some
circumstances, the precision of values in the histogram file may be
different than the precision you specified on the command line. The
default value for precision, if none is specified on the command line,
is the precision specified in the relevant format description file or
5, whichever is smaller. The second line shows the name of the
variable (latitude, longitude) and the number of values in the data
file for the variable (20 for both latitude and longitude).

The minimum and maximum values for the variable are shown next
(-83.223548 is the minimum and 77.883119 is the maximum value for
latitude). The maximum and minimum values are given here with a
precision of 6, which is the precision specified in the format
description file. The locations of the maximum and minimum values in
the input file are indicated. (-83.223548 is the fifth latitude value
in \lit{latlon.dat} and 77.883119 is the twelfth). Finally, the name
of the histogram data (or variable summary) file generated for each
variable is given (\lit{latitude.lst} and \lit{longitude.lst}).

\subsection{Variable Summaries}

\indc{variable!summary}\indc{summary!variable}
The name of each variable summary file (list file) output by
\lit{checkvar} is of the form \lit{variable.lst} for numeric variables
and \lit{variable.cst} for character variables. The data in
*\lit{.lst}, and *\lit{.cst} files can be loaded into histogram plot
programs for graphical representation. (You must be familiar enough
with your program of choice to manipulate the data as necessary in
order to achieve the desired result.) In Unix, there is no need to
abbreviate the base file name.

\note{If you use the -v option, the order of variables in var_file has
  no effect on the numbering of base file names of the variable
  summary files.  }

The two example variable summary files, \lit{latitude.lst} and
\lit{longitude.lst}, are shown next.


\begin{center}
\begin{tabular}{p{2.0in}p{2.0in}}
\lit{latitude.lst}

\begin{vcode}{st}
-83.224  1
-69.274  1
-65.865  1
-63.212  1
-63.003  1
-47.304  1
-34.578  1
-28.287  1
-12.967  1
 -0.929  1
 11.624  1
 12.588  1
 27.331  1
 29.918  1
 35.086  1
 35.130  1
 38.818  1
 54.118  1
 77.652  1
 77.883  1 
\end{vcode}
&
\lit{longitude.lst}

\begin{vcode}{st}
-176.162  1
-155.234  1
-153.544  1
-136.941  1
-113.661  1
 -79.178  1
 -77.506  1
 -55.442  1
 -21.644  1
   0.777  1
  30.172  1
  35.591  1
  36.356  1
  38.875  1
  55.319  1
  62.152  1
  91.411  1
 134.124  1
 144.804  1
 149.408  1
\end{vcode}
\end{tabular}
\end{center}

The variable summary files consist of two columns. The first indicates
boundary values for data bins and the second gives the number of data
points in each bin. Because a precision of 3 was specified in the
example, each boundary value has three decimal places. The boundary
values are determined dynamically by \lit{checkvar} and often do not
correspond to data values in the input file, even if the
\lit{checkvar} and data file precisions are the same.

The first data bin in \lit{latitude.lst} contains data points in the
range -83.224 (inclusive) to -69.274 (exclusive); neither boundary
number exists in \lit{latlon.dat}. The first bin has one data point,
-83.223548. The fourth data bin contains latitude values from -63.212
(inclusive) to -63.003 (exclusive), again with neither boundary value
occurring in the data file. The data point in the fourth bin is
-63.211962.

%%% Local Variables: 
%%% mode: latex
%%% TeX-master: t
%%% End: 
