%CHAPTER 1       
%
% $Id$
%


\chapter{Introduction}
\label{ff,dintro}

The \ffs\ is a DODS server that uses \ffnd\ software to serve data
from files in almost any format.  The FreeForm ND Data Access System
is a flexible system for specifying data formats to facilitate data
access, management, and use.  Since DODS allows data to be translated
over the internet and read by a client regardless of the storage
format of the data, the combination allows several format restrictions
to be overcome.

The large variety of data formats is a primary obstacle in the way of
creating flexible data management and analysis software. FreeForm ND
was conceived, developed, and implemented at the National Geophysical
Data Center (NGDC) to alleviate the problems that occur when you need
to use data sets with varying native formats or to write
format-independent applications.

DODS was originally conceived as a way to move large amounts of
scientific data over the internet.  As a consequence of establishing a
flexible data transmission format, DODS also allows substantial
independence from the storage format of the original data.  Up to now,
however, DODS servers have been limited to data in a few widely used
formats.  Using the \ffs , many more datasets can be made available
through DODS.

%\section{The Format Problem}

%Programmers can readily describe a format for a specific data set, but
%a compiled application cannot be used with other data sets until
%either the data or the program is modified. Two possible methods for
%handling data in a variety of formats are to reformat all the data
%into a standard format or to develop programs that can read data in
%many different formats.

%\subsection{Standard Formats}

%A number of standard formats have been proposed over the years and the
%specifications for these formats have generally improved. However,
%standard formats do not enjoy widespread use, which will probably
%continue to be the case.

%Many scientists have large amounts of data on hand in non-standard
%formats. Converting to standard formats is cumbersome and
%time-consuming. In addition, there are so many standard formats that
%format-independent applications are required even if only standard
%formats are used.

%\subsection{Smart Programs}

%Software developers can create programs that use data in many
%different formats. This approach has several advantages:

%\begin{itemize}
%\item The programs are flexible enough to allow the introduction of
%  new data formats.
  
%\item The scientist collecting the data is not forced to conform to
%  any single data format.
  
%\item The information contained in the original data is not lost
%  through reformatting.
%\end{itemize}

\section{The FreeForm ND Solution}

\ffnd\ uses a \new{format descriptor} file to describe the format
of one or more data files.  This descriptor file is a simple text file
that can be created with a text editor, describing the structure of
your data files.  

A traditional DODS server, illustrated in \figureref{fig,regular},
receives a request for data from a DODS client who may be at some
remote computer\footnote{The request comes via http.  The DODS server is, in
reality, an ordinary http server, equipped with a set of CGI programs
to process requests from DODS clients. See \sectionref{ff,install} and
\DODSuser\ for more information.}.  The data served by this server
must be stored in one of the data formats supported by DODS (such as
netCDF, HDF, or JGOFS), and the server uses specialized software to
read this data from disk.

When it receives a request, the server reads the requested data from its
archive, reformats the data into the DODS transmission format (also
called the DODS Data Access Protocol, or DAP), and sends the data back
to the client.

\figureplace{A Traditional DODS Server}{htb}{fig,regular}
{regular.ps}{regular.gif}{}

The \ffs\ works in a similar fashion to a traditional DODS server, but
before the server reads the data from the archive, it first reads the
data format descriptor to determine how it should read the data.  Only
after it has absorbed the details of the data storage format does it
attempt to read the data, pack it into the transmission format and
send it on its way back to the client.

\figureplace{The \ffs}{htb}{fig,ff}{ff.ps}{ff.gif}{}

\section{The FreeForm ND System}

The \ffs\ comprises a format description mechanism, a set of programs
for manipulating data, and the server itself.  The software was built
using the FreeForm ND library and data objects.  These are documented
in \ffbook .

The \ffs\ includes the following programs:

\begin{description}
\item[\lit{nph-ff}] The \ffs\ \new{dispatch script}.  This Perl script
  (requires Perl version 5 or greater) receives the data request from
  the client, and dispatches it to one of several filter programs.
  For definitions of dispatch script and filter programs, and other
  details of the implementation of a DODS server, see \DODSuser .
  
\item[\lit{ff-dods}] This filter program reads a \ffnd\ format
  description, and then reads data from the specified source, packs it
  into the transmission format, and sends it back to the client.
  
\item[\lit{ff-dds}] In order to process the data stream correctly, the
  client first asks the server for a description of the data it will
  send.  This description, called the Data Description Structure (DDS)
  is created and sent by this filter program.
  
\item[\lit{ff-das}] Other metadata that may be useful to the user of
  this data may be sent to the client in the Data Attribute Structure
  (DAS), created and sent by this program.  \sectionref{ff,das}
  contains information about how to include metadata in this
  structure.
  
\item[\lit{asciival}] This filter program allows data to be sent back
  to the user in ASCII format.  This allows a wide variety of clients
  to access this data, including simple web browsers, such as Netscape
  Navigator.
  
\item[\lit{usage}] This filter provides a way to get information about
  the data and about the server itself to a curious user.  See
  \sectionref{ff,usage} for information about what can be sent and how
  to set it up.
\end{description}

The \ffs\ distribution also includes the following \ffnd\ utilities.
These are quite useful to write and debug format description files.

\begin{description}
\item[\lit{newform}] This program reformats data according to the
  input \emph{and output} specifications in a format description file.

\item[\lit{chkform}] After writing a format description file, you can
  use this program to cross-check the description against a data file.

\item[\lit{readfile}] This program is useful to decode the format used
  by a binary file.  It allows you to try different formats on pieces
  of a binary file, and see what works.
\end{description}


\section{Installing the \ffs}
\label{ff,install-dods}

If you don't have the \ffs , and want it, follow these directions.  If
you have a copy of the \ffs , and want to know how to use it, see
\chapterref{ff,dquick} for quick instructions and examples of its use, and
\chapterref{ff,ff-server} for further information.

You can get the \ffs\ from the \DODShome . Follow the links to
``Download Software'' and then to ``Current Release.''  If your
computer is one of the platforms for which we provide a binary release,
get that, otherwise get the source code.

To get a binary release, go to that page, click on the computer you
use, and click on the ``FreeForm'' button in the ``Servers'' box.
Click the \but{Download} button, and follow the directions.  The
server will make a custom binary file for you, which you then
download.

To install the \ffs , first make sure you have an http server running
on the machine where you plan to serve the data.  If you are unsure
whether one is running, you can use a web client, like Netscape, to
send an http request to your own machine.  \DODSuser\ contains some
hints about installing a web server.

When you are sure a web server is properly installed and running,
unpack the archive file you downloaded from the \DODShome .  You will
find a directory called \lit{DODS}, containing \lit{etc} and
\lit{bin}.

\begin{enumerate}
\item Copy all the files in the \lit{etc} directory into your web
  server's CGI directory (see \DODSuser\ for tips on how to find this
  directory)
  
\item Make sure the permissions of those files will allow the web
  server to execute them.  Typically you might type use a command
  like:

\begin{example}
> chmod 555 cgi-bin/*    
\end{example}

\item While you are here, copy the files in the \lit{bin} directory to
  somewhere in your command \lit{PATH}, or modify your command
  \lit{PATH} to include the DODS \lit{bin} directory.  This is not
  strictly necessary to use the server, but will allow you to use the
  \ffnd\ tools that come with the \ffs\ distribution\footnote{There is
    a utility called \lit{newform} that comes as part of the Solaris
    OS.  This conflicts with the \ffnd\ utility of the same name.
    When installing this package on Solaris, make sure to put the
    \lit{bin} directory before the \lit{/usr/bin} directory in your
    \lit{PATH}, or change the name of the \ffnd\ \lit{newform} program.}.

\end{enumerate}

The \ffs\ is now installed.  You can confirm the installation with a
web client like Netscape.  Enter a URL containing your machine
address, the CGI directory name, and then \lit{nph-ff}, like this:

\begin{vcode}{ib}
http://dods.gso.uri.edu/cgi-bin/nph-ff
\end{vcode}

This is not a complete DODS URL, so you will get some kind of error
message\footnote{The example is from a working DODS server.  Try this
  example and compare it with the result from your own server.}.  If
the message you see mentions DODS, however, you have successfully
installed the server.  If it doesn't, and says something like ``The
requested URL was not found on this server,'' than you need to
re-check your installation.  (Check the permissions first, then make
sure all the files from the \lit{etc} directory were copied, then
check the installation of the web server.  Remember that the
\lit{nph-ff} program requires Perl v5.0 or greater; this is a common
problem.  There is a FAQ list at \DODShome\ that may have useful
suggestions.)

Now that the \ffs is installed, \chapterref{ff,ff-server} will explain
how to use it to serve your data.  It may be a good idea to
familiarize yourself with the information in the intervening chapters,
which will explain how to specify your data's format.

\subsubsection{Compiling the \ffs}

If the computer and operating system combination you use is not one of
the ones we own, you will have to compile the \ffs\ from its source.
Go to the ``Download Source'' page, and click on the ``Core'' button,
the ``FreeForm Server'' button, and the ``Packages'' button.

When you have downloaded the source archive (it may be quite large),
unpack it by typing:

\begin{example}
> tar xf DODS-3993.tar
\end{example}

This archive file is made up of several other compressed archive
files, so after this step you will need to unpack each of those files
individually: 

\begin{example}
> gzip -dc DODS-dap-2.22.tar.gz | tar xf -
> gzip -dc DODS-ff-dods-2.22.tar.gz | tar xf -
\end{example}

and so on.

When all the archive files are unpacked, move to the
\lit{DODS/packages/src} directory, and type:

\begin{example}
> ./configure
> make World
> make install
\end{example}

Each of these packages has a \lit{README} and \lit{INSTALL} file
containing directions in case of any trouble.

After the packages have been installed, go to the top DODS directory
(wherever you have installed it), and type:

\begin{example}
> ./configure
> make World
\end{example}

Then see \sectionref{ff,install-dods} for instructions on the final
installation. 





%%% Local Variables: 
%%% mode: latex
%%% TeX-master: t
%%% End: 
