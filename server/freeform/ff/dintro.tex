%CHAPTER 1       
%
% $Id$
%


\chapter{Introduction}
\label{ff,dintro}

The \ffs\ is a DAP2\footnote{The Data Access Protocol, version 2, was
  originally developed for the DODS project, then used by NVODS. In
  2000 the principals from the DODS project formed OPeNDAP, Inc., a
  not-for-profit corporation, tocontinue the development of the
  protocol and the associated software.}-compliant server that uses
\ffnd\ software to serve data from files in almost any format. The
FreeForm ND Data Access System is a flexible system for specifying
data formats to facilitate data access, management, and use. Since
DAP2 allows data to be translated over the internet and read by a
client regardless of the storage format of the data, the combination
allows several format restrictions to be overcome.

The large variety of data formats is a primary obstacle in the way of
creating flexible data management and analysis software. FreeForm ND
was conceived, developed, and implemented at the National Geophysical
Data Center (NGDC) to alleviate the problems that occur when you need
to use data sets with varying native formats or to write
format-independent applications.

DAP2 was originally conceived as a way to move large amounts of
scientific data over the internet.  As a consequence of establishing a
flexible data transmission format, DAP2 also allows substantial
independence from the storage format of the original data.  Up to now,
however, DAP2 servers have been limited to data in a few widely used
formats.  Using the \ffs , many more datasets can be made available
through DAP2.

%\section{The Format Problem}

%Programmers can readily describe a format for a specific data set, but
%a compiled application cannot be used with other data sets until
%either the data or the program is modified. Two possible methods for
%handling data in a variety of formats are to reformat all the data
%into a standard format or to develop programs that can read data in
%many different formats.

%\subsection{Standard Formats}

%A number of standard formats have been proposed over the years and the
%specifications for these formats have generally improved. However,
%standard formats do not enjoy widespread use, which will probably
%continue to be the case.

%Many scientists have large amounts of data on hand in non-standard
%formats. Converting to standard formats is cumbersome and
%time-consuming. In addition, there are so many standard formats that
%format-independent applications are required even if only standard
%formats are used.

%\subsection{Smart Programs}

%Software developers can create programs that use data in many
%different formats. This approach has several advantages:

%\begin{itemize}
%\item The programs are flexible enough to allow the introduction of
%  new data formats.
  
%\item The scientist collecting the data is not forced to conform to
%  any single data format.
  
%\item The information contained in the original data is not lost
%  through reformatting.
%\end{itemize}

\section{The FreeForm ND Solution}

\ffnd\ uses a \new{format descriptor} file to describe the format
of one or more data files.  This descriptor file is a simple text file
that can be created with a text editor, describing the structure of
your data files.  

A traditional DAP2 server, illustrated in \figureref{fig,regular},
receives a request for data from a DAP2 client who may be at some
remote computer\footnote{The request comes via http. The DAP2 server
  is, in reality, an ordinary http server, equipped with a set of CGI
  programs to process requests from DAP2 clients.
  See \sectionref{ff,install} and \DODSuser\ for more information.}.
The data served by this server must be stored in one of the data
formats supported by the OPeNDAP server (such as netCDF, HDF, or
JGOFS), and the server uses specialized software to read this data
from disk.

When it receives a request, the server reads the requested data from
its archive, reformats the data into the DAP2 transmission format and
sends the data back to the client.

\figureplace{A Traditional DAP2 Server}{htb}{fig,regular}
{regular.ps}{regular.gif}{}

The \ffs\ works in a similar fashion to a traditional DAP2 server, but
before the server reads the data from the archive, it first reads the
data format descriptor to determine how it should read the data.  Only
after it has absorbed the details of the data storage format does it
attempt to read the data, pack it into the transmission format and
send it on its way back to the client.

\figureplace{The \ffs}{htb}{fig,ff}{ff.ps}{ff.gif}{}

\section{The FreeForm ND System}

The \ffs\ comprises a format description mechanism, a set of programs
for manipulating data, and the server itself.  The software was built
using the FreeForm ND library and data objects.  These are documented
in \ffbook .

The \ffs\ includes the following programs:

\begin{description}
\item[\lit{dap_ff_handler}] The \ffs\ \new{data handler}. This program
  is run by the OPeNDAP server to handle the parts of any requests
  that require knowledge specifically about FreeForm. This program is
  run by the main server 'dispatch' software. That software is
  described in the Server Installation Guide (\OPDinstallUrl),
  available on the \DODShome .
\end{description}

The \ffs\ distribution also includes the following \ffnd\ utilities.
These are quite useful to write and debug format description files.

\begin{description}
\item[\lit{newform}] This program reformats data according to the
  input \emph{and output} specifications in a format description file.

\item[\lit{chkform}] After writing a format description file, you can
  use this program to cross-check the description against a data file.

\item[\lit{readfile}] This program is useful to decode the format used
  by a binary file.  It allows you to try different formats on pieces
  of a binary file, and see what works.
\end{description}


\section{Installing the \ffs}
\label{ff,install-dods}

If you don't have the \ffs , and want it, follow these directions.  If
you have a copy of the \ffs , and want to know how to use it, see
\chapterref{ff,dquick} for quick instructions and examples of its use, and
\chapterref{ff,ff-server} for further information.

You can get the \ffs\ from the \DODShome . Follow the links to
``Download Software'' and then to ``Current Release.''  If your
computer is one of the platforms for which we provide a binary release,
get that, otherwise get the source code.

To get a binary release, go to that page, click on the computer you
use, and click on the ``FreeForm'' button in the ``Servers'' box.
Click the \but{Download} button, and follow the directions.  The
server will make a custom binary file for you, which you then
download.

To install the \ffs , first make sure you have an http server running
on the machine where you plan to serve the data.  If you are unsure
whether one is running, you can use a web client, like Netscape, to
send an http request to your own machine.  \DODSuser\ contains some
hints about installing a web server.

When you are sure a web server is properly installed and running,
unpack the archive file you downloaded from the \DODShome . Binaries
are available for Solaris, IRIX, Linux, et cetera. Folow the
instructions that come with each, as they all have their own
idiosyncrasies. After unpacking the server will, by default install
some exectuable programs in /usr/local/bin and some other files in
both /usr/local/share/dap-server and /usr/local/share/dap-server-cgi.

Follow the instructions on configuring the OPeNDAP server. This
generally involves configuring a web server (e.g., Apache) to run the
\lit{nh-dods} CGI script and editing the dap-server.rc configuration
file. Both reside in /usr/local/share/dap-server-cgi (assuming that
you have chosen the default location for installation).

Now that the \ffs is installed, \chapterref{ff,ff-server} will explain
how to use it to serve your data.  It may be a good idea to
familiarize yourself with the information in the intervening chapters,
which will explain how to specify your data's format.

\subsubsection{Compiling the \ffs}

If the computer and operating system combination you use is not one of
the ones we own, you will have to compile the \ffs\ from its source.
Go to the OPeNDAP home page (www.opendap.org) and follow the menu item
to the downloads page. From there you will need the libdap, dap-server
and FreeForm handler software source distributions. Get each of these
and perform the following steps:

\begin{enumerate}
\item Expand the distribution (e.g., \lit{tar -xzf
    libdap-3.5.3.tar.gz})
\item Change to the newly created directory (\lit{cd libdap-3.5.3})
\item Run the configure script (\lit{./configure})
\item Run make (\lit{make})
\item Install the software (\lit{make install} or \lit{sudo make
    install})
\end{enumerate}

Each source distribution contains more detailed build instructions;
see the \lit{README}, \lit{INSTALL} and \lit{NEWS} files for the most
up-to-date information.
dquick
See \sectionref{ff,install-dods} for instructions on the final
installation. 

%%% Local Variables: 
%%% mode: latex
%%% TeX-master: t
%%% End: 
