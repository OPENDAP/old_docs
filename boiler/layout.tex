%%% Tex customizations and command definitions for DODS user
%%% guides, 2 October 1997 - tomfool
%%%
%%% $Id$

%%% $Log: layout.tex,v $
%%% Revision 1.5  1999/05/25 20:49:34  tom
%%% changed version numbers to 3.0
%%%
%%% Revision 1.4  1999/02/04 17:39:02  tom
%%% modified for use with dods-book.cls
%%%
%%% Revision 1.3  1998/03/13 20:43:26  tom
%%% writing of API manual.
%%%
%%% Revision 1.2  1998/02/12 15:47:25  tom
%%% updated for GUI doc
%%%
%%% Revision 1.1  1997/10/02 17:18:14  tom
%%% moved from user guide to boiler, made slightly more general,
%%% for use with other guides.
%%%

%%%%%%%%%%%%%%%%%%%%%%%%%%%%%%%%%%%%%%%%%%%%%%%%%%%%%%%%%%%%%%%%%%%%%
%%% This file is for TeX macros that are equally appropriate for the
%%% hard-copy (LaTeX) and html (hyperlatex) versions of the dods
%%% books.  If it's not appropriate for both, then it probably belongs
%%% in dods-book.cls or dods-book.hlx.

%%%%%%%%%%%%%%%%%%%%%%%%%%%%%%%%%%%%%%%%%%%%%%%%%%%%%%%%%%%%%%%%%%%%%
\NotSpecial{\do\_}% This removes the special meaning of `_', so for
                  % subscripts, there must be an `tex' environment
                  % around any diagram using subscripts, and an entire
                  % alternate html figure using <sub> tags.


%%%%%%%%%%%%%%%%%%%%%%%%%%%%%%%%%%%%%%%%%%%%%%%%%%%%%%%%%%%%%%%%%%%%%
%%% Different kinds of cross references.
%\newcommand{\chapterref}[1]{Chapter~\refl{#1} on page~\pagerefl{#1}}
\newcommand{\chapterref}[1]{\link{Chapter~\ref{#1}}{#1}}
\newcommand{\appref}[1]{\link{Appendix~\ref{#1}}%
  [~on page~\pageref{#1}]{#1}}
\newcommand{\sectionref}[1]{\link{Section~\ref{#1}}%
  [~on page~\pageref{#1}]{#1}} 
\newcommand{\pagexref}[1]{\link*{here}[page~\pageref{#1}]{#1}}
\newcommand{\tableref}[1]{\link{table~\ref{#1}}{#1}}
\newcommand{\Tableref}[1]{\link{Table~\ref{#1}}{#1}}
\newcommand{\figureref}[1]{\link{figure~\ref{#1}}{#1}}
\newcommand{\Figureref}[1]{\link{Figure~\ref{#1}}{#1}}

%%% A Prefatory list of the font conventions:
\newcommand{\listconventions} {
\section{Conventions}
\label{pref,conventions}

The \indn{typographic conventions} shown in
Table~\ref{typo-conventions} are followed in this guide and all the
other DODS documentation.

\begin{table}[htbp]
  \begin{center}
  \caption{Typographic Conventions}
  \label{typo-conventions}
  \begin{tabular}{|c|p{3in}|} \hline
    \lit{Literal text}  &  
         Typed by the computer, or in a code listing.\\ \hline
    \inp{User input}    &  
         Type this precisely as written.\\ \hline
    \var{Variables}     &   
         Used as a place holder for a user-specified or variable
         value. Choose an appropriate value and use that in place.\\
         \hline 
    \but{Button Text}\texonly{\rule{0pt}{2.5ex}}   
        &  Used to indicate text on a GUI button.\\ 
         \hline 
    \pdmenu{Menu Name}    &  This is the name of a GUI menu.\\ \hline 
  \end{tabular}
  \end{center}
\end{table}

When referring to a button in a menu, we will often use the notation:
\but{Menu,Button}. For example, \but{Options,Colors,Foreground} would
indicate the \but{Foreground} button in the \pdmenu{Colors} menu,
selected under the \pdmenu{Options} menu.
 }


%%% Local Variables: 
%%% mode: latex
%%% TeX-master: t
%%% End: 
