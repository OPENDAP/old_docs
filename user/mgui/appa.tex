% Appendix to the Matlab GUI User Guide
%
% $Id$
%
% $Log: appa.tex,v $
% Revision 1.7  2004/12/09 03:04:14  tomfool
% updated for GUI 6
%
% Revision 1.6  1999/05/27 15:58:39  tom
% expunged remaining references to "dods.gso.uri.edu"
%
% Revision 1.5  1999/05/25 20:47:46  tom
% modifications for DODS release 3.0
%
% Revision 1.4  1999/02/04 17:42:13  tom
% modified to use dods-book.cls and Hyperlatex
%
% Revision 1.3  1999/01/20 15:18:38  tom
% updated, and changed for dods-book.cls and hyperlatex
%
% Revision 1.2  1998/12/07 15:41:15  tom
% updated for DODS v2.19 and GUIv0.7
%
% Revision 1.1  1998/02/12 18:10:40  tom
% added to CVS archive
%
%

\chapter{Getting Started with the \GUI\ Software}
\label{gui,install}

This appendix contains instructions for acquiring the \GUI\
software. It also contains instructions for a quick start, in case you
feel like diving in without reading the rest of this document. If you
are very familiar with Matlab, this may be just fine.

The \GUI\ consists of the following pieces:

\begin{description}

\item[Program Files]

These are the M-files and data for the browser itself. These are
usually stored in a single directory (the ``Matlab directory'') that
must be accessible on the Matlab path. These files are the same on all
platforms.

\item[Archive Files]

M-files describing the data archives accessible to the browser.  These
are found in the \lit{DATASETS} subdirectory of the Matlab
directory. You can add your own files to the browser by adding M-files
to this directory. (See page~\pageref{gui,adding,manifest.lcl}) These
files are the same on all platforms.

\item[Helper Programs]

Three helper programs: \lit{loaddods}, \lit{writeval}, and
\lit{geturl}. These programs are part of the DODS ``Matlab Client''
distribution. The directory with loaddods in it must be on the Matlab
path, and the directory with writeval and geturl in it must be on the
shell's \$PATH. These programs differ among platforms. You must either
retrieve and compile the source distribution, or make sure to secure
the binary distribution appropriate for the computer you are using.

\end{description}

In addition to these programs, you will need a copy of the GNU gzip
utility to uncompress the archives. The \OPDhome\ contains links to
an up-to-date version of that utility.

The \GUI\ does not come with Matlab itself. You will have to
secure a copy of Matlab in order to run any of this.

\section{Getting the \GUI\ Software}

All the \opendap software can be downloaded from \OPDhome, which has a
link at the top reading \but{Download}.  Click there and follow the
directions. 


\section{Unpacking the \GUI\ Software}
\label{install,unpack}

Once you have the files downloaded 
they must be uncompressed with the \lit{gzip} program, and un-tarred
with the \lit{tar} program. The following command executes both at the
same time:

\begin{vcode}{ib}
gzip -d -c GUI-stuff.tar.gz | tar -xvf -
\end{vcode}

or, if you use GNU \lit{tar}:

\begin{vcode}{ib}
tar -xvzf GUI-stuff.tar.gz 
\end{vcode}

Move to the directory where you wish to store the \GUI, 
repeat this for each \lit{tar} file you need to install.

If you are installing this software from a binary distribution, you
may move on to the next section, \sectionref{install,install}.

After the \lit{tar} files have been unpacked, set the \DODSroot\
environment variable. The way to do this will differ depending on the
shell you use\footnote{Type \lit{echo \$SHELL} to find out which shell is
active on your system.}. For csh and tcsh, use:

\begin{vcode}{ib}
setenv DODS_ROOT /usr/local/OPD-2.8
\end{vcode}

(Make sure to use the correct version number.)
For ksh and bash, use:

\begin{vcode}{ib}
export DODS_ROOT=/usr/local/OPD-2.8
\end{vcode}

After you have set the environment variable (this might be a good time
to add the variable definition to your \lit{.login}, \lit{.cshrc} or
\lit{.profile} initialization file), finish the configuration with the
following shell commands:

\begin{vcode}{ib}
cd $DODS_ROOT
./configure
make
make install
\end{vcode}
%$

If you already have some \opendap software installed on your system, and
are simply adding the Matlab functions, you need not execute
\lit{configure} and \lit{make} in the \DODSroot\ directory, but 
can change your directory to the \lit{\$DODS_ROOT/src/matlab-GUI}
directory, and run \lit{configure} and \lit{make install} there.

\section{Installing the \GUI\ Software}
\label{install,install}

To run the \GUI, change your working directory to the directory where
the \GUI\ files are kept (\lit{\$DODS\_ROOT/bin/matlab-gui}), and
start Matlab. Then issue the \lit{browse} command.
In order to run the \GUI\ from other directories, you only need to set 
three environment variables:

\begin{itemize}
\item
Be sure that the \DODSroot\ environment variable is set, as
described in \sectionref{install,unpack}.

\item
The \lit{\$PATH} environment variable must include an entry for the 
\DODSroot\lit{/bin} directory containing the \lit{writeval} and
\lit{geturl} programs.

\item The \lit{\$MATLABPATH} environment variable must include entries
  for the directories containing the Matlab client program
  (\lit{loaddods}), the browser programs and data, as well as the
  directory containing the data archive files. These are usually
  \DODSroot\lit{/bin}, \DODSroot\lit{/matlab-GUI} and
  \DODSroot\lit{/matlab-GUI/DATASETS}\footnote{If you prefer, you can
    set the Matlab search path from within Matlab with the
    \lit{path()} or \lit{addpath} commands. You can put this command
    into a \lit{startup.m} file to set the path automatically.}

\end{itemize}

If you didn't do it before, put these variable definitions into your
shell's initialization file (\lit{.login}, \lit{.profile},
\lit{bashrc} or whatever), so you won't have to set them by hand.

You are now ready to run the \GUI. Simply start Matlab, and
type 'browse' at the prompt to see the browser window appear:

\begin{vcode}{ib}
>> browse
\end{vcode}


\section{The Archive Files}
\label{install,archive}

The \GUI\ software is not updated as frequently as are the archive
M-files in the central DODS directory. This means that it is generally
a good idea to issue an ``update all'' command (\but{Get All} on the
\pdmenu{Update} menu) soon after you complete the installation. This
is also a good test of the installation.




%%% Local Variables: 
%%% mode: latex
%%% TeX-master: t
%%% End: 
