% This file contains the beginning of an overview of the entire DODS
% system. It is intended to be a first stab at the introductory
% chapters of the DODS user guide. - tom sgouros
%
% $Id$

\documentstyle[code,12pt,html,psfig,margins]{article}
% removed a4paper style. jhrg 2/22/96
%\setpapersize{Afour}
\psfigurepath{../design/api-figs}

%
% These are html links which are used often enough in writing about DODS to
% merit an input file.
% jhrg. 4/17/94
%
% File rationalized and updated while writing the DODS User
% Guide. Also includes other useful abbreviations.
% tomfool 3/15/96
%
% $Id$
%
% HTML references to DODS documents
%
% For most of the DODS documentation there are three html references defined:
% 1) The upper case \newcommand produces the title of the paper, an
%    html link in the online documentation and a footnote in the paper
%    version.  
% 2) A capitalized version produces a capitalized description of the
%    paper and a link in the online version (but no footnote in the
%    paper version). 
% 3) A lower case version which produces a lower case description of
%    the paper and a html link, but no footnote.

% External references to documents:
%
% In order for external references to work (via latex2html) the labels used
% throughout the documents must be unique. The text labels are all prefixed
% with a document identifier (e.g., ddd) with a colon separater. The
% \externalrefs below (way below...) includes the perl files html.sty uses to
% make the cross refs.

\newcommand{\DODS}{\htmladdnormallink{Distributed Oceanographic Data System}
  {http://www.unidata.ucar.edu/packages/dods/}}

\newcommand{\Dods}{\htmladdnormallink{DODS}{http://www.unidata.ucar.edu/packages/dods/}}

\newcommand{\wrkshp}{\htmladdnormallink{Report on the First Workshop for the
    Distributed Oceanographic Data System, Proposed System Architectures}
  {http://www.unidata.ucar.edu/packages/dods/archive/reports/workshop1/section3.13.html}}

% not DODS URLs, but useful all the same...

\newcommand{\www}{\htmladdnormallink{World Wide Web} 
  {http://www.w3.org/hypertext/WWW/TheProject.html}}
\newcommand{\url}{\htmladdnormallink{Uniform Resource Locator}
  {http://www.w3.org/hypertext/WWW/Addressing/URL/url-spec.html}}

\newcommand{\uri}{\htmladdnormallinkfoot{Uniform Resource Identifiers}
  {http://www.w3.org/hypertext/WWW/Addressing/URL/uri-spec.html}}

\newcommand{\urn}{\htmladdnormallinkfoot{Uniform Resource Name}
  {http://www.acl.lanl.gov/URI/archive/uri-archive.messages/1143.html}}

\newcommand{\HTTPD}{\htmladdnormallinkfoot{HTTPD}
  {http://hoohoo.ncsa.uiuc.edu/docs/Overview.html}}

\newcommand{\HTTP}{\htmladdnormallinkfoot{HyperText Transfer Protocol}
  {http://www.w3.org/hypertext/WWW/Protocols/HTTP/HTTP2.html}}

%\newcommand{\HTML}{\htmladdnormallinkfoot{HyperText Markup Language}
%  {http://www.w3.org/hypertext/WWW/MarkUp/MarkUp.html}}

% CERN used to be `Conseil Europeen pour la Recherche Nucleaire'
\newcommand{\CERN}
  {\htmladdnormallink{European Laboratory for Particle Physics}
  {http://www.cern.ch/}}

\newcommand{\NCSA}
  {\htmladdnormallink{National Center for Supercomputing Applications}
  {http://www.ncsa.uiuc.edu/}}

\newcommand{\WWWC}{\htmladdnormallink{World Widw Web Consortium}
  {http://www.w3.org/}}

\newcommand{\CGI}{\htmladdnormallinkfoot{Common Gateway Interface}
  {http://hoohoo.ncsa.uiuc.edu/cgi/overview.html}}

\newcommand{\MIME}{\htmladdnormallink{Multipurpose Internet Mail Extensions}
  {http://www.cis.ohio-state.edu/htbin/rfc/rfc1590.html}}

\newcommand{\netcdf}{\htmladdnormallink{NetCDF}
  {http://www.unidata.ucar.edu/packages/netcdf/guide.txn_toc.html}}

\newcommand{\JGOFS}{\htmladdnormallink{Joint Geophysical Ocean Flux Study}
  {http://www1.whoi.edu/jgofs.html}}

\newcommand{\jgofs}{\htmladdnormallink{JGOFS}
  {http://www1.whoi.edu/jgofs.html}}

\newcommand{\hdf}{\htmladdnormallink{HDF}
  {http://www.ncsa.uiuc.edu/SDG/Software/HDF/HDFIntro.html}}

\newcommand{\Cpp}{{\rm {\small C}\raise.5ex\hbox{\footnotesize ++}}}

% Commands

\newcommand{\pslink}[1]{\small
\begin{quote}
  A \htmladdnormallink{postscript}{#1} version of this document is
  available.  You may also use \htmladdnormallink{anonymous
  ftp}{ftp://ftp.unidata.ucar.edu/pub/dods/ps-docs/} to access postscript files
  of all of the DODS documentation.
\end{quote}
\normalsize
}

\newcommand{\declaration}[1]{\small
{\tt {#1}}
\normalsize
}

% All the links from here down are for the white papers. I'm not sure that any
% of these are still valid given the reorganization of the web site. 5/12/98
% jhrg.

\newcommand{\DDA}{\htmladdnormallinkfoot{DODS---Data Delivery Architecture}
  {http://www.unidata.ucar.edu/packages/dods/archive/design/data-delivery-arch/data-delivery-arch.html}}
\newcommand{\Dda}{\htmladdnormallink{Data Delivery Architecture}
  {http://www.unidata.ucar.edu/packages/dods/archive/design/data-delivery-arch/data-delivery-arch.html}}
\newcommand{\dda}{\htmladdnormallink{data delivery architecture}
  {http://www.unidata.ucar.edu/packages/dods/archive/design/data-delivery-arch/data-delivery-arch.html}}

\newcommand{\DDD}{\htmladdnormallinkfoot{DODS---Data Delivery Design}
  {http://www.unidata.ucar.edu/packages/dods/archive/design/data-delivery-design/data-delivery-design.html}}
\newcommand{\Ddd}{\htmladdnormallink{Data Delivery Design}
  {http://www.unidata.ucar.edu/packages/dods/archive/design/data-delivery-design/data-delivery-design.html}}
\newcommand{\ddd}{\htmladdnormallink{data delivery design}
  {http://www.unidata.ucar.edu/packages/dods/archive/design/data-delivery-design/data-delivery-design.html}}

\newcommand{\URL}{\htmladdnormallinkfoot{DODS---Uniform Resource Locator}
  {http://www.unidata.ucar.edu/packages/dods/archive/design/urls/urls.html}}
\newcommand{\Url}{\htmladdnormallink{Uniform Resource Locator}
  {http://www.unidata.ucar.edu/packages/dods/archive/design/urls/urls.html}}
\newcommand{\dodsurl}{\htmladdnormallink{uniform resource locator}
  {http://www.unidata.ucar.edu/packages/dods/archive/design/urls/urls.html}}

\newcommand{\DAP}{\htmladdnormallinkfoot{DODS---Data Access Protocol}
  {http://www.unidata.ucar.edu/packages/dods/archive/design/api/api.html}}
\newcommand{\Dap}{\htmladdnormallink{Data Access Protocol}
  {http://www.unidata.ucar.edu/packages/dods/archive/design/api/api.html}}
\newcommand{\dap}{\htmladdnormallink{data access protocol}
  {http://www.unidata.ucar.edu/packages/dods/archive/design/api/api.html}}

\newcommand{\SOFT}
        {\htmladdnormallinkfoot{DODS---Software Development Environment}
  {http://www.unidata.ucar.edu/packages/dods/archive/managment/software/software.html}}
\newcommand{\Soft}{\htmladdnormallink{Software Development Environment}
  {http://www.unidata.ucar.edu/packages/dods/archive/managment/software/software.html}}
\newcommand{\soft}{\htmladdnormallinkfoot{software development environment}
  {http://www.unidata.ucar.edu/packages/dods/archive/managment/software/software.html}}

% I used to call the DAP the API, then for a few days I called it the
% DTP. These def's were used then. Rather than find everywhere they are used
% (not that hard, but...) I just hacked the macros. NB: the source file for
% the DAP document is still called `api.tex'

\newcommand{\API}{\htmladdnormallinkfoot{DODS---Data Access Protocol}
  {http://www.unidata.ucar.edu/packages/dods/archive/design/api/api.html}}
\newcommand{\api}{\htmladdnormallink{data access protocol}
  {http://www.unidata.ucar.edu/packages/dods/archive/design/api/api.html}}

\newcommand{\DTP}{\htmladdnormallinkfoot{DODS---Data Access Protocol}
  {http://www.unidata.ucar.edu/packages/dods/archive/design/api/api.html}}
\newcommand{\dtp}{\htmladdnormallink{data access protocol}
  {http://www.unidata.ucar.edu/packages/dods/archive/design/api/api.html}}

\newcommand{\TOOLKIT}{\htmladdnormallinkfoot{DODS---Client and Server Toolkit}
  {http://www.unidata.ucar.edu/packages/dods/archive/implementation/toolkits/toolkits.html}}
\newcommand{\Toolkit}{\htmladdnormallink{Client and Server Toolkit}
  {http://www.unidata.ucar.edu/packages/dods/archive/implementation/toolkits/toolkits.html}}
\newcommand{\toolkit}{\htmladdnormallink{client and server toolkit}
  {http://www.unidata.ucar.edu/packages/dods/archive/implementation/toolkits/toolkits.html}}

\newcommand{\TKR}{\htmladdnormallinkfoot{DODS---DAP Toolkit Reference}
  {http://www.unidata.ucar.edu/packages/dods/archive/implementation/toolkits/toolkits.html}}
\newcommand{\Tkr}{\htmladdnormallink{DAP Toolkit Reference}
  {http://www.unidata.ucar.edu/packages/dods/archive/implementation/toolkits/toolkits.html}}
\newcommand{\tkr}{\htmladdnormallink{DAP toolkit reference}
  {http://www.unidata.ucar.edu/packages/dods/archive/implementation/toolkits/toolkits.html}}

% I know that these are wrong. The papers have been moved to the gso web
% site, but I'm not sure exactly where. 5/12/98 jhrg

\newcommand{\DM}{\htmladdnormallinkfoot{DODS---Data Model}
  {http://lake.mit.edu/dods-dir/dodsdm2.html}}
\newcommand{\Dm}{\htmladdnormallink{Data Model}
  {http://lake.mit.edu/dods-dir/dodsdm2.html}}
\newcommand{\dm}{\htmladdnormallink{data model}
  {http://lake.mit.edu/dods-dir/dodsdm2.html}}

\newcommand{\DD}{\htmladdnormallinkfoot{DODS---Data Delivery}
  {http://lake.mit.edu/dods-dir/dods-dd.html}}

% external refs for DODS documents

\externallabels{http://www.unidata.ucar.edu/packages/dods/design/api}
  {/home/www/packages/dods/archive/design/api/labels.pl}
\externallabels{http://www.unidata.ucar.edu/packages/dods/design/data-delivery-arch}
  {/home/www/packages/dods/archive/design/data-delivery-arch/labels.pl}
\externallabels{http://www.unidata.ucar.edu/packages/dods/design/data-delivery-design}
  {/home/www/packages/dods/archive/design/data-delivery-design/labels.pl}
\externallabels{http://www.unidata.ucar.edu/packages/dods/implementation/toolkits}
  {/home/www/packages/dods/archive/implementation/toolkits/labels.pl}

%\renewcommand{\externalref}[2]{#2}

%%% Local Variables: 
%%% mode: latex
%%% TeX-master: t
%%% End: 


\begin{document}

%\textheight=9.75in commented-out. jhrg 2/22/96
%\textwidth=7.5in
%\oddsidemargin=.1in

\title{DODS---The Distributed Oceanographic Data System}
\author{}
\date{\today}

\maketitle

\begin{abstract}

  This document describes version 2.0 of the \DODS\ (DODS), a data
  system intended to allow researchers transparent access to
  oceanographic data---stored in any of several different file
  formats---across the internet. Using DODS function libraries, many
  existing data analysis programs can be easily modified to
  accommodate access of distant datasets in a manner identical to the
  access of local datasets. DODS includes a protocol for the
  transmission of data across the internet, and supports selection of
  data using constraint expressions. An overview of the system's use
  is presented, and the rationales for important system design
  decisions are described.

\end{abstract}
%
% Biolerplate text warning that anything can (and almost certainly will)
% change. 
%
% jhrg 3/25/94
%
% $Id$

\small

\centerline{{\bf Note}} 

\begin{quotation}

\begin{htmlonly}
  This is a working document made available to solicit comments. To receive
  e-mail notice of new or significantly changed documents, send e-mail
  to \htmladdnormallink{majordomo@unidata.ucar.edu}
  {mailto:majordomo@unidata.ucar.edu} with 'subscribe dods' in the
  body of the message.
\end{htmlonly}

\begin{latexonly}
  This is a working document made available to solicit comments. To receive
  e-mail notice of new or significantly changed documents, send e-mail
  to majordomo@unidata.ucar.edu with 'subscribe dods' in the body of
  the message. 

  Hypertext references in the hardcopy version of this document are
  represented using footnotes that contain a Universal Resource Locator (URL)
  to the referenced document.  To read the online documentation, open the URL
  `http://www.unidata.ucar.edu/' with a World Wide Web browser.
\end{latexonly}

\end{quotation}

\normalsize

%%
% Biolerplate text expaining that this document is intended for programmers.
%
% jhrg 4/17/94
%
% $Id$

\small

\begin{quotation}
  This document is written for people interested in implementing software
  that will be part of, or work in parallel with, the Distributed
  Oceanographic Data System. 
\end{quotation}

\normalsize

\begin{htmlonly}
\pslink{file://dods.gso.uri.edu/pub/DODS/overview.ps}
\end{htmlonly}

\clearpage

\tableofcontents

\clearpage

\section{Introduction}

The Distributed Oceanographic Data System (DODS) provides a way for
ocean researchers to access oceanographic data anywhere on the
internet from a wide variety of new and existing programs. By
developing network versions of commonly used data access Application
Program Interface (API) libraries, such as netCDF and JGOFS, the DODS
project can capitalize on years of development of data analysis and
display packages, allowing users to continue to use programs with
which they are already familiar. DODS incorporates a powerful data
translation facility, so that data may be stored in data structures
and formats defined by the data provider, but may be accessed by the
user in a manner identical to the access of local data files on the
user's own system. A user may not need to know that data from one set
are stored in a format different from data in another set.

The network communication model used by DODS uses URL addresses and
web servers ({\tt httpd}) to deliver data to the researcher, allowing
the DODS project to take advantage of the tremendous design and
development being applied to the World Wide Web. In addition to
creating network-compatible versions of popular data access APIs, the
DODS project also provides a software \toolkit\ to help other
developers create network-compatible DODS versions of other APIs.

The population of people who may be interested in a system such as
DODS may be divided into data consumers and data providers. Though it
was an important observation to the development of DODS that the two
roles are often assumed by the same scientists, the division is a
useful one for the introduction of the system. The following two
sections examine the tasks of data consumers and data providers.

\section{Data Consumer}

A scientist wishing to examine and sample some dataset will typically
be comfortable using a relatively small number of data analysis and
display packages. Some of these packages will use one of the popular
data access APIs currently available. However, few data access APIs
provide direct access to distributed data\footnote{The phrase {\em
distributed data\/} refers to datasets that reside on different
computers which are linked by a network such as Internet. The
computers may or may not be physically remote from each other. The
main point is that the computers manage their data resources
independently. In this paper the terms {\em remote\/} and {\em
distributed\/} are used to imply independently managed resources.}, so
this access must be made with network tools, such as web browsers or
{\tt ftp}. While relatively straightforward in principle, this process
can nonetheless become time-consuming and somewhat challenging.

DODS can be used with both new and existing software to provide
transparent access to distributed datasets, and provides data
translation, so that datasets stored in different formats can be
accessed by the same program.

The following example illustrates some of the differences between
accessing distributed data with the tools currently in widespread use,
and with DODS.

\subsection{An Example: Using ftp}

The advent of the WWW has made possible simple data browsers that
allow sophisticated interactive sampling of on-line datasets. Using a
web browser and ftp, a user can sample any of several large
oceanographic datasets available on the internet. 

Among the problems that can arise are those seen when a user tries to
use the results of one dataset to search a second dataset. Suppose
that a user wishes to choose a sea-surface temperature image from the
NOAA/NASA Pathfinder AVHRR archive at {\tt
http://podaac.www.jpl.nasa.gov/} using the results of a time-series
generated from the COADS Climatology archive at {\tt
http://ferret.wrc.noaa.gov/}. The steps are straightforward:

\begin{enumerate}

\item Create the time series from the COADS Climatology archive.

\item Import it to the user's local data analysis system. This may
require several steps to convert the data into the appropriate
format. Examine it.

\item Formulate a request to the AVHRR archive.

\item Import that to the user's local data display system. This may
also require several steps. 

\item Think about the displayed data.

\end{enumerate}

Though the procedure is straightforward, and the web servers designed
to make sampling the datasets a simple task, upon close examination,
the combination of the steps may create unforeseen difficulties. For
example, a request to the COADS server will return either an ASCII
file, a spreadsheet, or a netCDF format file. If the user is
fortunate, the returned file will already be in a format compatible
with the desired analysis package. But not all users will be so
fortunate, and often this file must be converted to some other file
format so it may be imported to the user's analysis program. This may
or may not be a simple task.

Even a file format for which a user is properly equipped may be used
in an unfamiliar manner. For example, the independent and dependent
variables might be in a different order, or the ASCII data may use
tabs instead of spaces.

Assuming the import of the COADS data has been accomplished, and
boundaries for the AVHRR search identified, the task of selecting from
the second archive may begin. Unfortunately, the request to the AVHRR
archive will return either a GIF picture, an HDF format file, or a raw
(binary) data file. Again, importing this output into the user's
analysis program may or may not be simple, but it will not be the same
procedure as the one used for the first data request.

Other problems are also apparent. The COADS Climatology sampling
program requests the user supply dates (month and day), whereas the
AVHRR archive asks for the julian day (an integer between 1 and 365 or
366). One server will accept ``S'' and ``W'' to indicate South
latitudes and West longitudes, while the other requires that these be
indicated with negative coordinate values.

Other problems may become apparent. The sampling of the COADS dataset,
while somewhat flexible, may not allow sampling in the manner the user
needs. It cannot, for example, provide a section except along a line
of constant latitude or longitude. If a user wanted to see a section
along a NE-SW line, it might be necessary to request several N-S
sections to piece together enough of a picture to construct a rough
version of the desired section. The net effect would be several more
requests for data---involving increased load on the internet and on
the database server---than is strictly necessary.

Further, it might be desirable to use the two databases to construct a
time series. This could conceivably mean repeating the entire
procedure many times.

\subsection{An Example: Using DODS}

To produce the same data selection using DODS, a user would follow
essentially the same steps. However, the steps themselves would be
performed differently. Once the user's data analysis package has been
converted to a DODS client (Section~\ref{dods client}), the accesses
to the remote datasets are made through the analysis package
itself. Instead of specifying a data file by a pathname, the user
specifies a URL, which may point to a local or a remote dataset.

It is important to note that {\em any} data analysis package that can
handle one of the DODS-supported data access APIs can be converted
into a DODS client program capable of reading data stored by {\em all}
of the DODS-supported data access APIs. Therefore, assuming the user
has some analysis package capable of doing the required sampling and
analysis on local data, all the steps would be performed from within
that package, just as if the user were operating on local files. The
result is a simpler procedure, even though the same steps are
followed.

\begin{enumerate}

\item The user need not learn about any of the archival formats, since
the DODS server and client cooperate to deliver the data in the format
in which the analysis package expects to see it. Whereas the user of
the ftp server has to worry about importing the data into the analysis
program, the DODS client program imports it transparently.

\item The user can sample the distant datasets in any fashion supported
by the local analysis package. Unnecessary data need not be sent over
the internet.

\item A substantial amount of the searching and sampling is performed
on the server machines.

\end{enumerate}

\subsection{The DODS Client}
\label{dods client}

DODS uses a client/server model. As mentioned previously, the DODS
servers are simply {\tt httpd} web servers, equipped to interpret a
DODS URL sent to them. (See Section~\ref{dods server}) The DODS client
program can be any program that uses one of the supported APIs, such
as JGOFS or netCDF.

To make some program into a DODS client, it must only be relinked with
the DODS implementation of the supported API library. This is a simple
process, generally requiring only a few minutes.

As an example, consider some program that uses the netCDF API to
access a datafile. Somewhere in the program, there is a call to the
netCDF-defined function {\tt ncopen()}, with a netCDF filename
(e.g. {\tt xbt29.nc}) specified as the argument. After relinking with
the DODS version of the netCDF library, the {\tt ncopen()} function
will accept a URL (e.g. {\tt
http://dods.gso.uri.edu/cgi-bin/nc/data/fnoc1.nc}) instead of a simple
filename. Aside from this difference, however, the program will seem
to function in the same way it had with the simple netCDF library.

DODS also provides a data translation facility. Data from the original
data file is translated by the DODS server into a DODS data model for
transmission to the client. Upon receiving the data, the client
translates the data into the data model it understands. (See
Section~\ref{dap} for more information about the DODS data model.)
Because the data transmitted from a DODS server to the client travel
in a common format, it doesn't matter which of the supported formats a
dataset is stored with. If the DODS client is designed to read netCDF
format files, the data returned by the DODS-netCDF library will appear
to have been read from a netCDF file, whatever the actual format of
the files from which the data were read\footnote{Note that there is a
limit to what can be translated. An API meant to support
two-dimensional arrays may be able to handle one-dimensional vector
data, but a program designed to process one-dimensional vector data
will not know what to do with a two-dimensional array. The set of data
access APIs supported by DODS contain several such mismatches.}. If
the program expects JGOFS data, the DODS-JGOFS library will return
data that seem to have come from a JGOFS dataset, again, no matter
what the actual input file format.

DODS does not pretend to remove all the overhead of data searches. A
user will still have to keep track of the URLs of interesting data
sets in the same way a user must now keep track of the names of files
containing interesting data. DODS will provide a data locator service,
in cooperation with the Global Change Master Directory ({\tt
http://gcmd.gsfc.nasa.gov/}), a DODS server designed to respond with
the URLs of datasets satisfying the query criteria. A user
can also use some less formal avenue to find the URLs.

\section{Data Provider}

The DODS data provider is the person or organization willing to make
their digital datasets available to the community with a DODS
compliant server.

DODS is a system that recognizes that many of the data users are also
the data providers, and it is built with a recognition that providing
the data should be as simple and as straightforward as possible. In
many cases, once a local web server is equipped to become a DODS
server, a scientist need do only very little beyond what must be done
simply to make the data available locally. (i.e., Put the data into
a file format that can be read by the locally used data analysis and
display programs.)

\subsection{The DODS Server}
\label{dods server}

The DODS data server is simply made up of a regular httpd server
equipped with a CGI program that uses three filters, to respond to
requests for dataset structure, data attributes, and data itself. (See
Section~\ref{dap} for a description of the data returned by these
requests, and see Section~\ref{url:introduction} for a description of
the DODS URL syntax, used to send these requests.) Most of the task of
a data provider consists of configuring this server. While perhaps not
a trivial task, it represents far less effort than packaging a dataset
for submission to some central data archive. Furthermore, modifying a
server's configuration to accommodate new data will be an almost
trivial task, involving the simple editing of a configuration file.

\subsection{Administration and Centralization}

Under DODS, there is no central archive of data; data under DODS is
organized in a manner similar to the internet itself. That is, all one
need do to make one's data available is to start up a properly
configured httpd server that has access to the data on the local
internet node. Each data provider is free to join and to leave the
system when it is convenient, just as any proprietor of a web home
page is free to delete it or add to it according to whimsy.

Of course, as can also be seen on the World Wide Web, there are some
disadvantages to the lack of a central authority. If no one knows
about a web site, no one will visit it. Similarly, listing a dataset
in a central data catalog, such as the Global Change Master Directory
({\tt http://gcmd.gsfc.nasa.gov/}), can make data available to other
researchers in a way that simply configuring a DODS server does not.

The central data catalog may be important to the success of the DODS
project in a related way. Setting up a DODS server and creating DODS
clients to use this data is a process requiring some investment in
time. In order for DODS to become widely used, a substantial number of
users must believe it is worth that investment. The availability of
a large number of datasets through DODS can provide that incentive,
but only if they are known to a large number of researchers.

\section{Data Access Protocol}
\label{dap}

The DODS Data Access Protocol (DAP) is a {\em lingua franca\/} for
reading earth science datasets. It is used as an intermediate representation
for data which are nominally accessed using an established third-party
Application Programmer Interface (API) (e.g., netCDF). Because, in addition
to a method for data access, the DAP defines the content of ancillary
information about datasets, it provides the means to access datasets
through a single protocol using any of the DODS supported APIs. Because the
\dap\ serves as an intermediate representation for several different APIs, it
can be used to translate between any two of those APIs.

Once users have the power to access remotely stored data they realize
that not all data are available using the API that they use. In fact,
many datasets' storage formats are incompatible with third party APIs
altogether because they are stored using conventions independently
developed at the site where the data were first analyzed, collected,
or archived. While those in charge of the data at a remote site may be
very willing to provide remote access, they will probably not be
willing to reformat their data for most off-site users. The problem
the DODS data translation servers address is how do users access
remote data when it is not stored in a format or data model their
programs understand?

The DODS Data Access Protocol provides a means to access datasets using any
one of a set of supported APIs by defining an intermediate representation for
datasets, and components of datasets, that is API and format independent.
Using the DODS DAP as an intermediate representation, software components can
be built which translate a user program's API calls into DAP calls. Data
servers can be built which accept DAP calls and, either by translating
into a third-party API or using dataset-local access methods, read the data.
Thus software which uses any one of a set of APIs can be used to read data so
long as a DODS server for that data exists---even if the data is stored in an
API different from the one used by the display or analysis software.

\subsection{Rationale}

Most data access APIs are targeted at application writers---they provide
powerful tools to read, write, and process one specific type of scientific
data and are typically specialized so that software in one application domain
is easy to write. In order to fulfill this function, APIs are typically
`feature rich' (i.e., they provide many ways to perform similar tasks). This
characteristic causes APIs that maintain this level of flexibility over a
wide range of data types to become very complex. Because simplicity and
ease-of-use are two characteristics that make APIs attractive to application
programmers, most API designers try to limit the scope of an API so that it
can provide the features application programmers want, at the level of detail
they want, without becoming excessively large and thus burdensome to use.
The netCDF and JGOFS APIs are good examples of APIs designed for a specific
type of application.  Each is used to access earth-science datasets, but
netCDF is most suited to gridded model data, and JGOFS has direct support
only for relational data.

However, the netCDF API cannot be directly used to access datasets produced
by the JGOFS API and vice versa. An application using either one of these
APIs would need to be modified if a user wanted to integrate both netCDF and
JGOFS compliant datasets. Modifying a program so that the data access API it
uses matches with the API used to write a dataset is an expensive process
and one which is exacerbated by the fact that more and more users do not
write their own software. Modifying someone else's software can be very
complex and costly at best; at worst it is not possible at all.  Moreover,
when the datasets are remote it may be impossible to convince their
maintainers to replicate them for another API since this, too, is very
costly.

One alternative to rewriting application software or replicating data is to
define an intermediate representation which can be used to access data stored
in any one of a number of APIs.  If the intermediate representation is based
on a data model that encompasses the abstractions of a certain set of APIs,
then the intermediate representation can be used to translate between those
APIs. The implementation of an intermediate representation interface need not
be an application interface itself; instead it can be a set of tools used to
build application-level components. These components can then be combined at
link-time and/or runtime with user-programs to access data.

An interface which is to be used as a translation mechanism cannot
afford to present the rich set of access functions that the individual
APIs, which it subsumes, present. It must control the size and
complexity of its interface by removing features while still providing
the capability to transparently interoperate between several data
models. The DODS DAP is such an interface.  It is designed to be used
in the construction of system components which are configured at
runtime by users, maintainers or other software components.  Thus, the
DODS DAP is not a separate addition to the set of interfaces formed by
JGOFS, netCDF, and others.  Instead it is an intermediate
representation which can be used to translate between those different
APIs.

\subsection{Requirements}

This section presents the requirements that must be met for an intermediary
representation for any of the DODS supported APIs. Currently, two APIs are
supported by DODS: netCDF and JGOFS\@. As additional APIs are incorporated
into the system (e.g., HDF), the DAP may add support for the data types those
handle.

\begin{enumerate}

\item The DODS DAP can be used to map data stored using one API to
  calls in a second API so long as both APIs are supported by DODS\@.

\item It must be possible to add new data types and operators to the DAP
  without changing the external interface it presents.

\item The DAP does not characterize the dataset based on its native
  implementation. The DAP supports access to datasets that consist of a
  single file, several files, a relational database, a set of relational
  databases, or any other reasonable representation.

\item The DAP makes all datasets appear cohesive---they always appear as a
  single store of data regardless of their actual implementation. 
%Not yet satisfied (\today).

\item The DAP supports datasets which consist of one or more variables.

\item When a dataset contains two or more variables, the DAP makes their
  relationship explicit. The relations are precisely described using a syntax
  which other computer systems can analyze.

\item The DAP describes each variable in a dataset by its:

\begin{enumerate}
\item Name
\item Type
\item Shape
\item Units. 
% Not yet satisfied (\today)
\item Type-dependent characteristics
\item \ldots zero or more attribute-value pairs
\end{enumerate}

\item The characteristics of the different variable types are given by the
  \dm.

\item The DAP can perform the following operations on a dataset:

\begin{enumerate}
\item Get dataset structure description
\item Get dataset attribute list
\item Send arguments to a dataset
\item Get values from the dataset
\end{enumerate}

\end{enumerate}

\subsection{Design}
\label{design}

The DODS DAP design contains three important parts: A data model which
describes data types that can be supported by the protocol and how they are
handled, the dataset description and dataset attribute structures which
describes the structure of datasets and the data they contain, and a small
set of messages that are used to access data. Each of these components are
described in the following subsections along with a summary of the runtime
process configuration needed for DODS to translate from a dataset API to a
user program API\@. The discussion of the runtime process configuration
frames each of the three design components and is used to explain why they
are needed.

\subsubsection{Process and Module Configuration}
\label{process}

The set of processes and modules needed to access a dataset is shown in
Figure~\ref{fig:structure}. In the figure two processes are shown. Process 1
contains the user program, a surrogate library implementation of the data
access API used by that program, and translator component which uses the DODS
DAP to request data used to satisfy calls from the user-program API calls.
This process, called the {\em user process\/} communicates with the second
process, the {\em translating server\/} process.  The translating server
contains two modules---one to recast the incoming DODS DAP calls into calls
in the the dataset's API and a second module that implements that API\@. In
the figure all data is transmitted over the network using the DODS DAP
regardless of the API used by the user program or used to store the data,
even if the APIs are the same.

\begin{figure}
\centerline{\psfig{figure=structure.ps}}
\caption{Two processes are used to translate API Y calls from the user
  program to API X calls for the dataset. These same two processes are the
  client and server in a distributed data system. Note that DODS does not
  supply either the complete client or the complete server---rather DODS
  software consists of modules that can be combined with existing software
  (user programs, third party libraries) to build distributed systems. In
  this figure, {\em Module\/} refers to a conceptual unit (e.g., a library),
  not just a compile-time unit.}
\label{fig:structure}
\end{figure}

The DODS DAP's main function is to facilitate translation between two
different data access APIs. In order to read data stored in API X as in
Figure~\ref{fig:structure} a server which provides access using the DAP can
use the native implementation of API X to read information from the dataset.

In addition to the data accessible via the native API, each dataset will
contain ancillary data. These data will be directly accessed by some of the
DAP calls. It will be used by both the local implementation of the DAP to aid
in translating the DAP calls into the dataset's native API calls and by
the remote translation process.

In order to effectively translate the user program API calls into DODS
DAP calls, the translator module must have some knowledge of the
source dataset's structure. This structural information will comprise
part of the ancillary data that is accessed directly by the DODS
DAP\@. Based on this information, which can be accessed using DAP
calls, the translator can choose how to best translate the user
programs data-access calls. In effect, the translator must map one
data access API onto a data model to which it may not be well suited
(either because the implementors of the user program or the dataset
have chosen an inappropriate interface).

Finally, in order to be useful by many user programs, particularly those
written by a third party development team for a group of users, the
translator must have some additional information about the representation of
certain data objects expected by the user program. This information, called
{\em Usage Data\/} in Figure~\ref{fig:structure}, will allow the user to
specify the format of dates and similar information which has many different
common forms but no universally accepted format. It is stored in a file
created by the user or developers of the {\em user program}. This
information is defined by DODS outside of the definition of API Y and will be
used by the translator module when data requests are made by the user program
or when data is returned to the user program in response to one of those
requests.

\subsubsection{The Data Model}

Data models provide a way to organize scientific datasets so that useful
relationships between individual datum are evident. Many data models have
been specifically designed to make using the data in a computer program
simpler. Examples of computationally oriented data models for scientific data
are hierarchical, sequential, and gridded data models. 

Data models are abstract, however, and to be used by a computer program they
must first be implemented by a programmer. Often this implementation takes
the form of an API---a library of functions which can read and write data
using a data model or models as guidance. Thus every data access API can be
viewed as implementing some data model, or in some cases several data models.

Because DODS needs to support several very different data models, it is
important to design it around a core set of concepts that can be applied
equally well to each of those data models. If that can be done, then
translation between data represented in those different models may be
possible.

Currently DODS supports two very different data access APIs: netCDF
and JGOFS\@. The netCDF API is designed for access to gridded data,
but has some limited capabilities to access sequence data (although
not with all of its supported programming language interfaces). The
JGOFS API provides access to relational or sequence data\footnote{In
  the remainder of this document, the phrase {\em sequence data}, or
  just {\em sequence}, will mean an ordered set of elements each of
  which contains one or more sub elements where all of the sub-elements
  of an element are explicitly related to each other.}.  Both APIs
support access in several programming languages (at least C and
Fortran) and both provide extensive support for limiting the amount of
data retrieved.  For example a program accessing a gridded dataset
using netCDF can extract a subsampled portion or {\em hyperslab\/} of
that data.  Likewise, the JGOFS API provides a powerful set of
operators which can be used to specify which type of sequence elements
to extract (e.g., only those corresponding to data captured between
12:01am and 11:59am) as well as masking certain parameters from the
returned elements so that only those parameters needed by the program
are returned.

The DODS DAP uses the concepts of variables and operators as the base for the
data model. Within the data model, a dataset consists of one or more
variables where each variable is described formally by a number of
attributes.  Variables associate names with each component of a dataset, and
those names are used to refer to the components of the dataset. In addition
to their different attributes, it is possible to operate on individual
variables or named collections of variables. The principal operation is {\em
  access}, although this can be modified in a number ways.

\subsubsection{Base-Type Variables}
\label{base-type-variables}

Variables in the DODS DAP have two forms. They are either base types or type
constructors. Base type variables are similar to predefined variables in
procedural programming languages like C or Fortran (e.g., {\tt int} or {\tt
  integer*4}).  While these certainly have an internal structure, it is not
possible to access parts of that structure using the DAP\@. Base type
variables in the DAP have three predefined attributes (or characteristics):
Name, Type, and Unit.  They are defined as follows:

\begin{description}

\item [Name] A unique identifier that can be used to reference the part of
  the dataset associated with the variable.

\item [Type] The data type contained by the variable. This can be one of {\tt
    byte}, {\tt int32}, {\tt float64}, {\tt string}, and {\tt URL}\@.
  Where:

\begin{description}
\item [byte] is the same as unsigned char in ANSI C\@.
\item [int32] is a 32 bit integer---it is synonymous with long in
  ANSI C when that type is implemented as 32 bits.
\item [float64] is the IEEE 64 bit floating point data type.
\item [string] is a sequence of bytes terminated by a null character.
\item [URL] is a string as defined in \URL\@.
\end{description}

\item [Unit] This contains the name of the units of the value contained in
  the variable.  Examples of typical units are degrees Celsius, degrees
  Kelvin, \ldots If a variable is unit less, then this is null.

\end{description}  

\subsubsection{Type Constructor Variables} 

Type constructor variables describe the grouping of one or more variables
within a dataset. These classes are used to describe different types of
relations between the variables that comprise the dataset. This information
can be useful to people who would like to understand more about the dataset
than can be conveyed with implicit relations. It is also designed to be
useful to other programs/processes in the data access chain.  There are six
classes of type constructor variables defined by the DAP: lists, arrays,
structures, sequences, functions, and grids. The type constructor classes
besides structure provide information that is used in the translation of
subsetting operations (hyperslabbing or selections and projections in netCDF
or JGOFS parlance, respectively). The types are defined as:

\begin{description}

\item [List] The {\bf List} type constructor is used to hold lists of 0 or
  more items of one type. Lists of {\tt int32}, \ldots, {\tt grid} are
  specified using the keyword {\tt list} before the variable's class. Access
  to an element of a list is possible using one of the five operators given
  in Table~\ref{api:tab:class-ops}.

\item [Array] An {\bf Array} is a one dimensional indexed data
  structure as defined by ANSI C\@. Multidimensional arrays are
  defined as arrays of arrays. In addition to element access using
  subscripts enclosed in brackets ({\tt []}), an array may be accessed
  using only its name to return the entire array or using a hyperslab
  operator to return a rectangular section of the array. In the later
  case, the hyperslab is defined for each dimension by a starting
  index, and ending index, and a stride value.  Specifying a stride
  $>$ 1 will cause the dimension to be subsampled by the stride value.
  Table~\ref{api:tab:class-ops} shows the syntax for array accesses
  including hyperslabs.

  In addition to its magnitude, every dimension of an array may also
  have a name. It is possible to find the name for any given dimension
  (e.g., the $i^{th}$ dimension) and thus write software which access
  the $3^{rd}$ element of the dimension {\em cast\/}.

\item [Structures] A structure is a class that conveys no relational
  information and may contain several variables of different classes. It is
  used to supply information to other parts of the data access and
  translation system that may be useful in optimizing the access or
  translation operations. The structure type can also be used to group a set
  of unrelated variables together into a single dataset.

\item [Sequences] A sequence is an ordered set of $N$ variables which
  has several instantiations (or values). Variables in a sequence may
  be of differing classes.  Each instance of a sequence is one
  instantiation of the variables. Thus a sequence can be represented
  as:

\begin{displaymath}
\begin{array}{ccc}
  s_{0 0} & \cdots & s_{0 n} \\
  \vdots & \ddots & \vdots \\
  s_{i 0} & \cdots & s_{i n}
\end{array}
\end{displaymath}

\noindent Every instance of sequence $S$ has the same number, order,
and class of variables. A sequence implies that each of the variables
is related to each other in some logical way. A sequence is different
from a structure because its constituent variables have several
instances while a structure's variables have only one instance (or
value). Because a sequence has several values for each of its
variables it has an implied {\em state}, in addition to those values.

\item [Functions] Functions are a subclass of Sequences and are used to
  indicate that one set of variables has a functional relation to a second
  set of variables. Variables in a function may be of differing classes. The
  mathematical description of this functional relation is not
  specified. Instead the function type is used to indicate that one of the
  two sets constitute the independent variables and the other the dependent
  variables.  Typically, the variables defined by a function have more than
  one instance---functions are similar to sequences but have additional
  information about the functional dependency of variables.

\item [Grid] A grid is an association of an $N$ dimensional array with $N$
  named vectors, each of which has the same number of elements as the
  corresponding dimension of the array. Each vector is used to map indices of
  one of the array's dimensions to a set of values which are normally
  non-integral (e.g., floating point values). The $N$ (map) vectors may be
  members of different classes. Grids are similar to arrays of base type
  variables, but add named dimensions and maps for each of those dimensions.
  \label{page:grids} 

\end{description}

\subsubsection{Operators}
\label{api:operators}

The principal operation performed on any variable is to access that variable
and retrieve its value or values.  For a base type variable, access is the
implied operation and is achieved by passing the name of the variable to a
data server using the access protocol (see Section~\ref{entry}). For an
instance of a type constructor class, the variable name will access the
entire object and return it in an array (or C struct)\footnote{If a variable
  contains elements of different types then an instance of it {\em must\/} be
  returned in a {\tt struct} and not an array.  This implies that the
  receiving program must dynamically allocate storage for the variable and
  then correctly access the fields.  This implies sophisticated programming
  (beyond the `user' level) or an interpreter---or both.}. Fields of type
constructors may also be accessed using the dot ({\tt.}) operator or the
virtual file system syntax. If a structure {\tt s} has two fields {\tt time}
and {\tt temperature}, then those fields may be accessed using {\tt s.time}
and {\tt s.temperature} or as {\tt s/time} and {\tt s/temperature}.

All of the classes listed in section~\ref{base-type-variables} are local to
the dataset except the {\tt URL} type. When an object of type URL is
accessed, the server must open the dataset referenced by that URL and read
its values such that the structure of that referenced dataset replaces the
URL in the current dataset. In practice this is most useful in a constraint
expression when the value of a variable is compared with the value of another
variable in a different dataset. If the URL cannot be referenced, then the
type of the URL degrades to {\em string\/} and an access to that element of
the dataset returns that string.

Access to variables can be modified using selection operators. Each
type of variable has its own set of selection and projection operators
which can be used to modify the result of accessing a variable of that
type.  Table~\ref{api:tab:class-ops} summarizes the types and the
operators applicable to them. In the table, operators have the meaning
defined by ANSI C except as follows: the array hyperslab operators are
as defined by netCDF\cite{netcdf} (where {\tt a} is the start index,
{\tt b} is the stride, and {\tt c} is the end index), the string
operators are as defined by AWK\cite{kern:upe}, and the list operators
are as defined by Common Lisp\cite{steele:clisp}.

\begin{table}
\caption{Classes and operators in the DAP\@.}
\label{api:tab:class-ops}
\begin{center}
\begin{tabular}{|| l | l ||} \hline
\multicolumn{1}{|| c}{\sc Class} & \multicolumn{1}{c ||}{\sc Operations} \\
\hline \hline
\multicolumn{2}{|| l ||}{\em Base Type} \\ \hline
{\tt byte}, {\tt int32}, {\tt float64} & {\tt < > = != <= >=} \\ \hline
{\tt string} & {\tt = != } $=\sim$ \\ \hline
{\tt URL} & {\tt *} \\ \hline
\multicolumn{2}{|| l ||}{\em Constructor\/} \\ \hline
{\tt array} & {\tt [start:stop] [start:stride:stop]} \\ \hline
{\tt list} & {\tt length, car, cdr, nth, member} \\ \hline
{\tt structure} & {\tt.} \\ \hline
{\tt sequence} & {\tt.} \\ \hline
{\tt function} & {\tt.} \\ \hline
{\tt grid} & {\tt .} \\ \hline
\end{tabular}
\end{center}
\end{table}

The hierarchy of the type constructor classes in the DAP is shown in
Figure~\ref{fig:class}. 

\begin{figure}
\centerline{\psfig{figure=class-hier.ps}}
\caption{The hierarchy of classes in the DAP\@.}
\label{fig:class}
\end{figure}

\subsubsection{The External Representation of Variables}
\label{api:external-rep}

Each of the base-type and type constructor variables has an external
representation defined by the \dap. This representation is used when an
object of the given type is transferred from one computer to another.
Defining a single external representation simplifies the translation of
variables from one computer to another when those computers use different
internal representations for those variable types\cite{xdr}. 

Constraint expressions, which are an important part of the \dap, do
not affect {\em how\/} a base-type variable is transmitted. Constraint
expressions only determine if, given a particular value for a
variable, that variable should be transmitted. However, for
constructor type variables constrained expressions may be used to
exclude portions of the variable. In these cases, the constraint
expressions can be used to change the way a particular variable is
transmitted. For example, if a constraint expression is used to select
the first three of six fields in a structure, the last three fields of
that structure are {\em not\/} transmitted by the server.

The \dap uses Sun Microsystems' XDR protocol\cite{xdr} for the external
representation of all of the base type variables. Table~\ref{tab:base-xdr}
shows the XDR types used to represent the various base type
variables.

\begin{table}
\caption{The XDR data types used by the \dap as the external representations
  of base-type variables}
\label{tab:base-xdr}
\begin{center}
\begin{tabular}{|| l | l ||} \hline
\multicolumn{1}{|| c}{\sc Base Type} & \multicolumn{1}{c ||}{\sc XDR Type} \\
\hline \hline
{\tt byte} & {\tt xdr byte} \\ \hline
{\tt int32} & {\tt xdr long} \\ \hline
{\tt float64} & {\tt xdr double} \\ \hline
{\tt string} & {\tt xdr string} \\ \hline
{\tt URL} & {\tt xdr string} \\ \hline
\end{tabular}
\end{center}
\end{table}

In order to transmit constructor type variables, the \dap defines how the
various base type variables, which comprise the constructor type variable,
are transmitted. Any constructor type variable may be subject to a constraint
expression which changes the amount of data transmitted for the variable (see
Section~\ref{api:constraints}). For each of the six constructor types these
definitions are:

\begin{description}

\item [Array] An array is sent using the {\tt xdr\_array}
  function. This means that an array of 100 {\tt Int32}s is sent as a
  single block of 100 {\tt xdr long}s, not 100 separate {\tt xdr
  long}s.

\item [List] A list is sent as if it were an array.

\item [Structure] A structure is sent by encoding each field in the order
  those fields are declared in the DDS and transmitting the resulting block
  of bytes.

\item [Sequence] A Sequence is transmitted by encoding each instance as for a
  structure and sending one after the other, in the order of their occurrence
  in the dataset. The entire sequence is sent, subject to the constraint
  expression. In other words, if no constraint expression is supplied then
  the entire sequence is sent. However, if a constraint expression is given
  all the records in the sequence that satisfy the expression are
  sent\footnote{The client process can limit the information received by
    either using a constraint expression or prematurely closing the I/O
    stream. In the later case the server will exit without sending the entire
    sequence.}.

\item [Function] A function is encoded as if it is a Sequence (one component
  after the other, in the order of their declaration).

\item [Grid] A grid is encoded as if it is a Structure (one component
  after the other, in the order of their declaration).

\end{description}

\subsubsection{Dataset Descriptor Structure}
\label{api:dds}

In order to translate from the user program's API to the dataset's API, the
translator process must have some knowledge about the types of the variables,
and their semantics, that comprise the dataset. It must also know something
about the relations of those variables---even those relations which are only
implicit in the dataset's own API\@. This knowledge about the dataset's
structure is contained in a text description of the dataset called the {\em
  Dataset Description Structure}.

The dataset description structure (DDS) does not describe how the
information in the dataset is physically stored, nor does it describe
how the dataset's API is used to access that data. Those pieces of
information are contained in the dataset's API and in the translating
server, respectively.  The translating server uses the DDS to describe
the structure of a particular dataset to a translator---the DDS
contains knowledge about the dataset variables and the interrelations
of those variables.  In addition, the DDS can be used to satisfy some
of the DODS supported APIs dataset description calls. For example,
netCDF has a function which returns the names of all the variables in
a netCDF data file. The DDS can be used to get that information.

The DDS is a textual description of the variables and their classes
that comprise the entire dataset. The dataset descriptor syntax is
based on the variable declaration/definition syntax of C and C++. A
variable that is a member of one of the base type classes is declared
by by writing the class name followed by the variable name. The type
constructor classes are declared using C's brace notation. A grammar
for the syntax is given in Table~\ref{tab:DDS}. Each of the keywords
for the type constructor and base type classes have already been
described in section~\ref{base-type-variables}. The {\tt dataset}
keyword has the same syntactic function as {\tt structure} but is used
for the specific job of enclosing the entire dataset even when it
does not technically need an enclosing element (because at the
outermost level it is a single element such as a structure or
sequence).

\begin{table}
\caption{Dataset Descriptor Structure Syntax}
\label{tab:DDS}
\small
\begin{center}
\begin{tabular}{ll} \hline
%
% This syntax comes straight from the bison source code
% (src/api.dds.y). Let's try *not* to change it, since reformatting is a
% royal pain. jhrg 11/9/94
%
{\em data sets\/} & {\em data set\/} \\
                 & {\em data sets} {\em data set\/} \\

{\em data set\/}  & {\tt dataset} {\tt \{} {\em declarations\/} {\tt \}} 
                   {\em  name\/} {\tt ;} \\

{\em declarations\/} & {\em declaration\/} \\
                 & {\em declarations} {\em  declaration\/} \\
		 & {\em (empty list)} \\
%                 &  $\delta$ \\

{\em declaration\/} &   {\em list} {\em declaration\/}  \\
                 & {\em base-type} {\em var\/} {\tt ;} \\
                 & {\em structure\/}  {\tt \{} {\em declarations\/} {\tt \}} 
                  {\em var\/} {\tt ;} \\
                 & {\em sequence\/} {\tt \{} {\em declarations\/} {\tt \}} 
                  {\em var\/} {\tt ;} \\
                 & {\em function\/} {\tt \{} {\tt independent} {\tt :}
                  {\em declarations} {\tt dependent\/} {\tt :} 
                  {\em declarations\/} {\tt \}} 
                  {\em var\/} {\tt ;} \\
                 & {\em grid\/} {\tt \{} {\tt array} {\tt :} 
                  {\em declaration\/} {\tt maps} {\tt :} 
                  {\em declarations\/} {\tt \}} 
                  {\em var\/} {\tt ;} \\

{\em list\/}    & {\tt list} \\

{\em structure\/} & {\tt structure} \\

{\em sequence\/} & {\tt sequence} \\

{\em function\/} & {\tt function} \\

{\em grid\/}    & {\tt grid} \\

{\em base-type\/} & {\tt byte} \\ 
                  & {\tt int32} \\
                  & {\tt float64} \\
                  & {\tt string} \\
                  & {\tt url} \\

{\em var\/}     & {\tt id} \\
                & {\em var} {\em array-decl\/} \\

{\em array-decl\/} & {\tt $[$} {\tt integer} {\tt $]$} \\
                   & {\tt $[$} {\tt id} {\tt =} {\tt integer} {\tt $]$} \\

{\em name\/}    & {\tt id} \\
\end{tabular}
\end{center}
\normalsize
\end{table}

An example DDS entry is shown in Figure~\ref{fig:dds}. Suppose that
three experimenters have each performed temperature measurements at
different locations and at different times. This information could be
held in a dataset consisting of a sequence of the experimenter's
name, the time and location of each measurement and the list of
measurements themselves, and indicates that there is a relation
between the experimenter, location, time and temperature called
temp\_measurement.

\begin{figure}
\begin{code}{cb}
              dataset {
                  int catalog_number;
                  function {
                    independent:
                      string experimenter;
                      int time;
                      structure {
                          float latitude;
                          float longitude;
                      } location;
                    dependent:
                      sequence {
                          float depth;
                          float temperature;
                      } temperature;
                  } temp_measurement;
              } data;
\end{code}
\caption{Example Dataset Descriptor Entry.}
\label{fig:dds}
\end{figure}

\subsubsection{Dataset Attribute Structure}
\label{api:das}

The Dataset Attribute Structure (DAS) is used to store attributes for
variables in the dataset. An attribute is any piece of information
about a variable that the creator wants to bind with that variable
{\em excluding\/} the characteristics type, shape, and units. The
characteristics type, shape and units are always defined for every
variable; they are data type information about the
variable. Attributes, on the other hand, are intended to store extra
information about the data such as a paragraph describing how it was
collected or processed\footnote{To define attributes for the entire
  dataset, create an entry for a variable with the same name as the
  dataset.}. In principle attributes are not processed by software
other than to be displayed. However, many systems rely on attributes
to store extra information that is necessary to perform certain
manipulations on data. In effect, attributes are used to store
information that is used `by convention' rather than `by design'. DODS
can effectively support these conventions by passing the attributes
from dataset to user program via the DAS\@. Of course, DODS cannot
enforce conventions in datasets where they were not followed in the
first place.

The syntax for attributes is given in Table~\ref{tab:DAS}. Every
attribute of a variable is a triple: attribute name, type and
value. Note that the attributes specified using the DAS are different
from the information contained in the DDS\@. Each attribute is
completely distinct from the name, type and value of its associated
variable. The name of an attribute is an identifier, following the
normal rules for an identifier in a programming language with the
addition that the `/' character may be used. The type of an attribute
may be one of: Byte, Int32, Float64, String or Url. An attribute may
be scalar or vector. In the later case the values of the vector are
separated by commas (,) in the textual representation of the DAS\@.

When the \dap\ is used to read the attributes of a variable and that
variable contains other variables, only the attributes of the named
variable are returned. In other words, while the DDS is a hierarchical
structure, the DAS is {\em not\/}; it is similar to a flat-file
database.

\begin{table}
\caption{Dataset Attribute Structure Syntax}
\label{tab:DAS}
\small
\begin{center}
\begin{tabular}{ll} \hline
{\em attributes\/} & {\em attribute\/} \\
                 & {\em attributes} {\em attribute\/} \\

{\em attribute\/} & {\tt attribute} {\tt \{} {\em var-attr-list\/} {\tt \}} \\

{\em var-attr-list\/} & {\em var-attr\/} \\
                 & {\em var-attr-list} {\em var-attr\/} \\
		 & {\em (empty list)} \\
%                 & $\delta$ \\

{\em var-attr\/} & {\em var-name} 
                {\tt \{} {\em attr-list\/} {\tt \}} \\

{\em attr-list\/} & {\em attr-pair\/} \\
                &  {\em attr-list} {\em attr-pair\/} \\
		 & {\em (empty list)} \\
%                & $\delta$ \\

{\em attr-pair\/} & {\em attr-type} {\em attr-name} {\em attr-val\/} 
                {\tt ;} \\

{\em var-name\/} & {\tt identifier} \\

{\em attr-name\/} & {\tt identifier} \\

{\em attr-val-vec\/} & {\em attr-val\/} \\
                     & {\em attr-val-vec\/} {\tt ,} {\em attr-val\/} \\

{\em attr-val\/} & {\tt value} \\
                 & {\tt identifier} \\
                 & {\tt string} \\

{\em attr-type\/} & {\tt Byte} \\
                  & {\tt Int32} \\
                  & {\tt Float64} \\
                  & {\tt String} \\
                  & {\tt Url} \\

\end{tabular}
\end{center}
\normalsize
\end{table}

\subsubsection{Data Access Protocol Entry Points}
\label{entry}

The DAP is a stateless protocol. Each of the DAP's entry points (i.e.,
the messages a data server will respond to) does a single isolated job
and they can be issued in any order (although in many applications it
will not make sense to get the values for a variable before finding
out the name of the variable \ldots). The stateless nature of the DAP
fits well within the context of the data delivery system described in
\DDA\@.  In that paper a client-server architecture for remote access
is described which relies on the HTTPD/CGI mechanism to build a data
server. One implementation of that architecture is described in \DDD\
uses three CGI modules, one of each of the three DAP entry points.

In this paper we talk about messages to the data server as if it is a
stateful server like {\tt ftpd}. However, that is merely a convenientway to
phrase the discussion of the DAP---in fact the data server is a stateless
machine accessed by getting the value of a URL\@. 

Each DODS data server must respond to three URLs; one for each of the objects
(DAS, DDS and variable) which the server returns. These URLs are formed by
appending a suffix to the base URL which references the dataset. In
addition, individual variables may be accessed using the constraint
expression mechanism described here. The paper \URL\ contains more
information on this scheme.

Two messages are provided to access the dataset descriptor structure
(DDS) and the dataset attribute structure (DAS)\@. The response to
these messages is text formated using the respective grammars in
Tables~\ref{tab:DAS}~and~\ref{tab:DDS}. This text can then be parsed
by the caller to determine the structure of the dataset, types and
sizes of each of its components and their attributes. These structures
are derived both from information contained in the dataset and for
ancillary information supplied by the dataset maintainers in separate
text files (in the \ddd\ the origin of these structures is described
in detail). They provide information that is often referred to as
`metadata'\footnote{We have learned to shy away from this term since
we have found that `metadata' to one person is `data' to another; the
categorization often limits the usefulness of the underlying
information.} and may be cached by the client system.  Future accesses
to the same dataset can then skip the retrieval of these structures.

All variables are read from a dataset individually using a single
`read' message. The message must include the name of the variable to
read, and optionally, may include an expression that describes the
range of values desired. No other information need be sent to the
server. In response to this message the value(s) of the variable(s)
is/are then sent back to the client.  Base type variables return only
a single value when accessed, but other types may return more
values. For example, a {\tt structure} with three {\tt int32} and two
{\tt float64} members will return five values (unless an expression
constrains the access).

One important capability of the DODS API, which it inherits from
JGOFS, is the ability to set constraints on variables (JGOFS calls
this `using selection and projection' operators).  Constraints are
used to control the values and/or members of constructor types that
are returned when a variable is accessed. For some datasets and some
variables, constraints make little or no difference in how the
variables are accessed.  However, for certain types of data,
constraining access can vastly reduce the amount of data the
application needs to process and, in DODS case, transmit over the
network.

\label{api:constraints}
Constraint expressions provide more flexibility in the way data is
accessed.  In a dataset with many interrelated variables, or with
very large variables, these expressions are a way for the user program
to move some of the complex logic needed to search data, in order to
find those data with some desired set of properties, away from the
user program and into the data supply system.  Different types of
constraint expressions make sense for different types of data. The
grammar for constraint expressions is given in
Table~\ref{api:tab:expr}.  In the table {\em Operator\/} is one of the
variable-class dependent operators listed in
Table~\ref{api:tab:class-ops}, {\tt !} is boolean {\em not\/} and {\tt
\&} is boolean {\em and}.

\begin{table}
\caption{Constraint Expression Syntax}
\label{api:tab:expr}
\begin{center}
\begin{tabular}{ll} \hline

{\em expression\/} & {\em projection} {\em selection \/} \\

{\em projection\/} & {\em variable\/} \\
                   & {\em variable} {\tt ,} {\em projection\/} \\
		 & {\em (empty list)} \\
%                   & $\delta$ \\

{\em selection\/}  & {\em variable} {\em operator} {\em value\/} \\
                   & {\em variable} {\em operator} {\em variable\/} \\
		 & {\em (empty list)} \\
%                   & $\delta$ \\

{\em compound-sel\/} & {\em selection} {\em boolean-op} {\em compound-sel\/} \\ 
                     & {\tt !} {\em selection} {\em compound-sel\/} \\
                     & {\em selection\/} \\

{\em boolean-op\/} & {\tt \&} \\
\end{tabular}
\end{center}
\end{table}

The Data Access Protocol is used to access datasets within the DODS data
delivery system. It is not intended to be limited to DODS, but it is
specialized for the system. Among those specializations are the stateless
nature of the protocol. While the DAP could be used in many different
contexts, it is principally designed to be the transmission protocol used by
the data servers in the DODS data delivery system.

\section{Universal Resource Locators}
\label{url:introduction}

In order to access remote data, DODS needs a syntax for names which
describe the network location for different resources. There are many
other systems which must solve a similar problem. Three different
classes of solution to this problem have been implemented in existing
systems: (1) The Network File System incorporates remote resources
directly into the client computer's file system where they can be
referenced as if they were local files, (2) {\tt ftp} prompts the user
for an Internet address and then directs all user actions to that
address until it is told to disconnect and (3) \www\ (WWW) clients use
a string, called a \url\ (URL), which combines the protocol, host and
pathname.  We choose the third approach after dismissing the first two
as, respectively, too complex and too rudimentary.

A URL is a generalization of file naming concepts which is already in
common use on the Internet. A URL extends the filename/pathname
concept used to refer to files on a computer by adding an Internet
address and protocol component. A URL specifies the computer and
access method (protocol) in addition to the filename and thus can be
used to refer to files anywhere on a network. While they are used in
various forms by many different systems, URLs, as such, are most
closely associated with the WWW project.

DODS chose the URL notation in part because it represents an evolving
standard for resource specification on the Internet. Conforming, at least in
spirit with this standard, puts DODS in a place to benefit from ongoing work
to develop more general network resource names with minimal additional work.
Because the existing definitions of resource locators (by both DODS and the
WWW) do not adequately address some important issues, compliance with an
existing {\em de facto\/} standard increases the chance that DODS will be
able to take advantage of emerging solutions to those problems.

In addition, the syntax of DODS URLs must match that used by the WWW
because the DODS uses the WWW \HTTPD\ server as its base-communication
module. While previous designs for the data delivery system outlined
the construction of special DODS-only data servers using RPC
technology, the most recent design uses NCSA's \HTTPD\ for the
communication component of all DODS data servers. Thus, each URL that
refers to a dataset must match the syntactical requirements of that
software (see the documentation on \HTTP).

The Common Gateway Interface (CGI) mechanism of HTTPD is flexible enough to
allow sophisticated specialization of the semantics of WWW URLs without
requiring a change in URL syntax. DODS can further restrict the WWW syntax
for its URLs through conventions supported by the DODS server tool kit and by
forcing all access to take place via the HTTPD's CGI mechanism.

\subsection{Structure and Interpretation of URLs}
\label{url:structure}

A formal grammar for WWW \uri\ (URIs) exists and provides a framework
for extending the meanings of URLs beyond those used by the WWW
project. In this section the syntax for DODS URLs is described (See
Table~\ref{url:tab:url}) and the differences between the WWW and DODS
URLs are also discussed. This syntax is less formal than that
presented in the CERN documentation, however, DODS servers must be
able to parse this syntax. In the case where a DODS server forms an
interface between other DODS components and a foreign system, the DODS
server must first parse this syntax and then translate the parsed
expression into the foreign system's syntax.

\subsection{Syntax of the DODS URL}

\begin{table}
\caption{DODS URL Syntax}
\label{url:tab:url}
\begin{center}
\begin{tabular}{ll} \hline
{\em DODS URL\/} & {\em Access\/} {\tt ://} {\em Host\/} {\tt /} {\em CGI\/}
{\tt /} {\em Args\/} \\
{\em Access\/} & {\tt http} \\
{\em Host\/} & {\em Domain name\/} \\
{\em CGI\/} & {\em CGI-directory\/} {\tt /} {\em String\/} \\
{\em Args\/} & {\em Pathname\/} \\
             & {\em Pathname\/} {\em Constraint} \\
             & {\tt (} {\em DODS URL\/} {\tt )} \\
{\em Constraint\/} & {\tt ?} {\em Expression\/} \\
\end{tabular}
\end{center}
\end{table}

\begin{description}

\item [Access] The access component of a DODS URL is always the
  literal string {\tt http}. The DODS data delivery architecture uses
  the World Wide Web \HTTPD\ server developed by NCSA to transport
  data. Note that just about any WWW server will work for DODS as long
  as it supports the CGI mechanism.

\item [Host] The host component is an Internet-style host name that
  can be resolved to an Internet address.
  
\item [CGI] The CGI directory is the directory where CGI programs are kept.
  This varies from machine to machine but many UNIX computer use {\tt
    cgi-bin}. In addition, it is possible to configure {\tt httpd} to search
  several directories for CGI programs; the DODS dispatch CGI may reside in
  any of these directories. The string is the name of the DODS dispatch
  CGI\@. In the software distributed by URI/MIT this is an abbreviation for
  the API (e.g., the dispatch CGI for NetCDF is {\tt nc}). However, server
  implementors are free to choose any name they want. In addition, data
  providers may rename this script if they want to.
  
\item [Args] The URL components named {\em Args\/} contains a pathname
  used by the CGI to locate the dataset. This may or may not be an
  actual path on the remote machine. In some cases the DODS server may
  be an interface to a DBMS\@. In this case the pathname is used
  (probably in conjunction with other information) by the CGI to
  formulate a query to the DBMS\@. The pathname may have a constraint
  expression appended to it. This is used either by the filter program
  or the CGI to limit the data access beyond the pathname. It provides
  additional flexibility in the way data is accessed through the
  server. The arguments may also be a second URL\@. In this case the
  {\em Access\/} and {\em Host\/} components of the outermost URL
  (called an {\em envelope\/}) specify the target process (server)
  which should receive the embedded DODS URL\@. The source process
  (client) does not interpret this embedded URL at all---only the
  target process attempts to evaluate it.  Embedded URLs may be nested
  to any depth.

\item [Constraint] The constraint is a boolean expression used to
  constrain access to a variable. Different classes of variables can
  have different types of constraints. 

\end{description}

\begin{table}
\caption{Sample URLs which name resources; sometimes there is more than one
  acceptable way to name a single resource.}
\label{url:tab:examples}
\footnotesize
\begin{center}
\begin{tabular}{| l | p{5cm} |} \hline
\multicolumn{1}{| c}{\sc URL} 
& \multicolumn{1}{ c |}{\sc References} \\ \hline \hline

{\tt http://gso.uri.edu/cgi-bin/nc/data/fnoc1.nc} 
& the entire dataset \\ \hline

{\tt http://gso.uri.edu/cgi-bin/nc/data/fnoc1.nc?u} 
& {\tt u} from the dataset \\ \hline

{\tt http://gso.uri.edu/cgi-bin/nc/data/fnoc1.nc/u} 
& {\tt u} using the virtual file system syntax \\ \hline

{\tt http://gso.uri.edu/cgi-bin/nc/data/fnoc1.nc?u<10} 
& all variables given that {\tt u} is $<$ 10 
(Makes sense for sequence data only) \\ \hline

{\tt http://lake.mit.edu/cgi-bin/jg/bloom/level1}
& all variables in {\tt  level1} \\ \hline

{\tt http://lake.mit.edu/cgi-bin/jg/bloom/level1/temp} 
& only {\tt temp} in {\tt level1} \\ \hline
\end{tabular}
\end{center}
\normalsize
\end{table}

\subsubsection{Characteristics of DODS and WWW URLs}
\label{url:differences}

DODS URLs are a proper subset of WWW URLs. URLs defined by the WWW project
cover a wide variety of client-server systems while parts of DODS URLs are
interpreted only to be DODS CGIs. WWW URLs include special cases for servers
which support protocols such as {\tt ftp}, {\tt wais} and {\tt gopher}. In
these cases, the {\em Access\/} component of a WWW URL is used to indicate
which protocol to use. When {\em clients\/} receive URLs, they use this
information to select which protocol should be used to access the data
referenced by the URL (i.e., the {\em Access\/} information is used to select
the type of server with which the client will communicate). DODS URLs are
only meaningful to CGI modules accessible over the network via HTTP servers.
Thus, there will never be any variation in the {\em Access\/} component of a
DODS URL\@. It will always be {\tt http}.

In addition to a protocol restriction, DODS URLs always reference a
CGI module on the server. This is necessary because all DODS data
access takes place via one or more CGI modules. The CGI specification
must follow the {\em Host\/} component of the URL\@.

While WWW URLs are the input to WWW browsers written with knowledge of
URLs, and in many cases, the ability to parse parts of URLs, DODS URLs
are passed to programs which nominally have no knowledge of URL
notation. In \DDA\ a data system architecture is presented in which
user programs are re-linked with new implementations of data-access
APIs. These {\em client libraries\/} understand and can manipulate
URLs (and in particular, understand the particular restrictions DODS
places on URLs) but the user programs do not. Some of the user
programs were written before any of the URL software even existed.

A DODS URL references a dataset, or portion of a dataset. Accessing a DODS
URL returns an experimental-type binary \MIME\ document. Such a document
cannot be displayed by browsers nor does it contain URLs which link the
document to other documents on the Internet. URLs that are part of the WWW
system of HTTP servers reference documents which can be displayed by a large
number of difference browsers. DODS URLs are not browsable at all, that is,
they cannot meaningfully serve as input to one of the standard WWW
browsers\footnote{However, it is possible to add one of several suffixes to a
DODS URL and get text information which a browser can display. See
Section~\ref{urls:user-vs-generated} for more information.}. DODS uses URLs
simply to reference datasets on the Internet and not for the more general
purpose of linking a document within the larger context of all documents
publicly accessible on the Internet.

There are two ways DODS CGI modules can interpret the part of the URL
which the HTTP server passes to it; as a constraint expression or as
an access to a virtual file system. This second method of
interpretation is a specialization of the general behavior of HTTPD
where the text following the CGI name is passed to the CGI module
unprocessed. Accessing a dataset as if it were a virtual file system
means specifying variables within the dataset as if they were files
or files within directories subordinate to the CGI\@. For example in
Table~\ref{url:tab:examples} the file name separator ({\tt /}) is used
to indicate that the within the named dataset (the netCDF dataset
called {\em CDT\/}) only the variable {\tt u1} is to be accessed. If
the dataset contained several levels of variables, as in the examples
for the three level JGOFS dataset, then a specific level or variable
within a level may be specified.  However, the virtual file system
syntax is not capable of specifying arbitrary set of variables from a
dataset unless those variables happen to be the sole members of a
structure of sequence (See \DTP\ for more information on data types
DODS supports). The VFS syntax is a special case of constraint
expressions. It is included in DODS because file systems are familiar
tools to many users, much more so than boolean expressions, and
because they are flexible enough for many user's demands.

The second syntax DODS supports for the portion of the URL past the
CGI name is the constraint expression syntax. Constraint expressions
are a way of passing, along with the URL, a set of restrictions to be
applied to the dataset when the client software (i.e., the surrogate
library) accesses the dataset. These constraints will be used during
access to the dataset to limit the variables and variable values
extracted from it. The constraint expression, or portions of the
constraint expression, will be evaluated {\em by the server\/} at the
time of the data access. The syntax of the expression itself is
dependent on the types of variables within the dataset (See \DTP).

\subsection{Uses of URLs by DODS}

This section briefly describes how URLs are used by the two major components
of DODS: the Data Locator and the Data Delivery Mechanism.

\subsubsection{URLs and the Data Locator}
\label{url:locator}

The Data Locator is used to find specific datasets within DODS\@. Users can
query the Data Locator and expect that it will return a URL which can be used
by other components in DODS to access that dataset. These URLs may be used
as supplied to user programs or they may be edited first. Because the URLs
are ASCII text strings, no special software is required to edit or store
them. In addition, DODS URLs may be obtained using other methods such as
Mosaic, e-mail, or personal communications.

\subsubsection{Data Delivery and URLs}
\label{delivery}
\label{urls:user-vs-generated}

The Data Delivery Mechanism consists of libraries with which users re-link
their programs as well as stand alone utilities built at least partially from
those libraries. URLs are used verbatim by the data delivery components of
DODS\@. The URL is passed to the server or translator using the API {\em
  open\/} call (e.g., {\tt ncopen()}). The client library stub is responsible
for removing the envelope of the URL and sending either the path or embedded
URL to the correct host/server or host/translator.

When a user wants to access data via a DODS server they must give to a client
(i.e., a program linked with one of the DODS reimplemented API libraries) A
URL which references that data. Users may choose to supply a constraint along
with that URL effectively limiting the parts of the dataset that the client
can see. For example, suppose a dataset exists which contains several
arrays:

\begin{figure}
\begin{code}{cb}
    Dataset {
        Int32 u[time_a = 16][lat = 17][lon = 21];
        Int32 v[time_a = 16][lat = 17][lon = 21];
        Float64 lat[lat = 17];
        Float64 lon[lon = 21];
        Float64 time[time = 16];
    } fnoc1;
\end{code}
\caption{The DDS of a dataset.}
\label{url:fig:dds}
\end{figure}

\begin{figure}
\begin{code}{cb}
    Dataset {
        Int32 u[time_a = 1][lat = 17][lon = 21];
    } fnoc1;
\end{code}
\caption{The DDS of a dataset constrained by $u[2:2][1:17][1:21]$.}
\label{url:fig:dds2}
\end{figure}

The user can refer to the entire dataset using a URL without a constraint
expression;
\htmladdnormallink{http://dods.gso.uri.edu/cgi-bin/nc/data/fnoc1.nc}
{http://dods.gso.uri.edu/cgi-bin/nc/data/fnoc1.nc.dds}. If the user were to
specify this, then the client-library will receive from the server a DDS like
the one in Figure~\ref{url:fig:dds}. However, if the user knows that they
only want to work with a small part of the dataset they can supply a
constraint expression along with the URL\@. For example, suppose that the
user only wants the latitude and longitude values for $time_a = 2$, and
furthermore that they are only interested in the array $u$. They could supply
the user program with the following URL:

\begin{code}{cb}
\htmladdnormallink{http://dods.gso.uri.edu/cgi-bin/nc/data/fnoc1.nc?u[2:2][1:17][1:21]}
{http://dods.gso.uri.edu/cgi-bin/nc/data/fnoc1.nc?u[2:2][1:17][1:21].dds}
\end{code}

\noindent The user program will receive DDS in
Figure~\ref{url:fig:dds2}. For program which are designed to read the
entire dataset without user interaction this is a powerful additional
feature.

However, constraint expressions have a second use in DODS\@. They are used by
the reimplemented APIs to extract specific parts of a dataset when that is
requested by the user program. Many APIs provide features which make it
possible to write software which opens a dataset, presents the user program
with a collection of variables and then provides a way for the program to
read one or more of those variables. In an API reimplemented for DODS, those
calls must all be satisfied by information the API receives from a DODS
server.

In order for the API to get information about the dataset the API
must synthesize various URLs using the one given by the user as a
base. For example, to get the DDS of the dataset referenced by 

\begin{code}{cb}
    http://dods.gso.uri.edu/cgi-bin/nc/data/fnoc1.nc
\end{code}

\noindent the reimplemented API must append the suffix {\tt .dds}, as
follows:

\begin{code}{cb}
    http://dods.gso.uri.edu/cgi-bin/nc/data/fnoc1.nc.dds
\end{code}

\noindent Similarly, the DAS of that dataset is obtained by appending
{\tt .das}.

Getting data is a bit more work than getting the DAS or the DDS\@. The
request for a particular variable must be translated into a DODS
constraint expression. This constraint expression is then appended to
the URL given by the user and then the suffix {\tt .dods} is appended
to that. For example, suppose the user program makes an API function
call requesting the value of the array $u$ in the preceding
figures. What the API is supposed to return is the values of the
entire array $u$, but no other values. The reimplemented API would
build the URL {\tt
http://dods.gso.uri.edu/cgi-bin/nc/data/fnoc1.nc?u.dods} where the
$?u.dods$ specify that the variable $u$ is to be the only variable in
the projection and the {\tt .dods} selects the data filter from the
DODS server. For more information on the constraint expression syntax,
see \DAP; for more information on the DODS data servers, see \DDD\@.

\subsection{Deficiencies of URLs}
\label{url:deficiencies}

There are several features which URLs as defined by either DODS or the WWW
lack. These include:

\begin{enumerate}

\item Poor support for version information

\item No support for location-independent naming

\item No formal support for quality rating of datasets

\item No support for binding extent information within the name-space (e.g.,
  for how long will this dataset exist)

\item No support for security information

\end{enumerate}

While these are important issues, many are being investigated by the
Internet Engineering Task Force (IETF) and the evolving \urn\ (URN)
standard. By following the de facto URL standard now, DODS is in a
good position to benefit from the future standardization of
URNs. Further, some of these deficiencies are not as severe as they
might seem at first. For example, it is not clear that security
information (providing access to a limited-access resource) belongs in
a resource name. It might be more secure to design the resource so
that it prompts specially for any security information
required. Similarly, version and quality information may be
incorporated informally to the existing URL syntax.  Finally, a crude
type of location independence may be achieved using domain name
service aliases for important resources. These approaches are already
in widespread use with the WWW\@.

Rather than form our own solutions for the deficiencies of URLs, DODS
will wait until either the IETF or WWW community modifies URLs and/or
adopts URNs or until continued development of DODS is significantly
hampered by the problems.

\section{Status}

As of \today, there is a working version (1.1) of DODS available. This
supports the netCDF and JGOFS data access APIs. The constraint
expressions are being beta-tested, and the translation is being coded,
and will be released this summer. A group at JPL is developing the HDF
to DODS translators and URI/GSO is developing the data locator service
in cooperation with the staff at the GCMD. A DODS-compliant server has
been installed at the National Oceanographic Data Center to serve the
data center's holdings of profile data.

\section{Conclusion}

DODS supports access to datasets stored in a number of established,
third party APIs. In addition, it supports cross-access to those data
sets. Thus data in any one of the supported APIs is accessible using
any one of the supported APIs (at least in principle---some {\em
possible\/} accesses may wind up returning meaningless, or null,
data). The DODS Data Access Protocol facilitates this cross
connections of user programs and datasets by providing a common
access protocol for each API\@. Each data server will provide access
to data using the DAP and each surrogate library will satisfy the
supported API's function calls with information obtained via the
DAP\@.

The combination of the DODS network communication model and the data
translation facility make DODS a powerful tool for the retrieval,
sampling, and display of large distributed datasets.  Though DODS was
developed by oceanographers, its application is not constrained to
oceanographic data. The organizing principles and algorithms may be
applied to many other fields where data is not concentrated in central
repositories.

\clearpage

\addcontentsline{toc}{section}{Bibliography}
\bibliographystyle{plain}
\bibliography{../boiler/dods}

\end{document}



