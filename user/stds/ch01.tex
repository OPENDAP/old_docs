% Chapter one to the DODS Data Standards Guide
%
% $Id$
%
% $Log: ch01.tex,v $
% Revision 1.1  1999/06/07 05:09:32  tom
% added to CVS
%
%
%

\chapter{DODS and Data Standards}
\label{intro}

Once upon a time there was DODS.  And it was good.  DODS solved many
problems of dataset incompatibility by making remote datasets pretend
to be in a format they were not in.  This was tricky, and it was
largely accomplished by hiding from the user the actual data storage
format, and using many different converters to provide data in a
standard transmission format.

However, data storage format isn't everything.  Though the DODS
solution is quite a useful one, it has its limitations.  Two datasets
can be incompatible even though they share an identical data storage
format.  For example, consider two sea surface temperature datasets
stored in netCDF format files.  One can store its temperature in
degrees Celsius, and another can store temperature as raw data, in
unscaled satellite counts, that have to be converted into temperature
values meaningful to a human.


\section{Levels of Compatibility}
\label{sec:compat}

In discussions about whether two different datasets are compatible, we
must distinguish between different levels of compatibility.  What
level is important depends to some extent on how you intend to use the
data.  To a human, for example, two datasets could be considered
compatible if they are about the same data variable.  Two temperature
datasets that cover the same area might be considered quite compatible,
if all one is doing is looking at two different color contour maps
lying on a table, or side-by-side on a computer screen.

However if one is seeking datasets on which to execute various
analyses and computations, a higher level of compatibility is
required.

The issues of data compatibility can be separated into four
categories:

\begin{itemize}
\item Data storage format

\item Data attributes 

\item Data organization

\item Server interface

\end{itemize}


\subsection{Data Storage Format}

Two datasets whose data are stored and retrieved with different data
format software are incompatible.  A dataset stored with netCDF
software cannot be read by a program to read HDF files, and vice
versa. 

Data storage format was the first compatibility issue
that DODS addressed, and it does so fairly successfully.  Whatever
storage format data are kept in, the DODS server translates them into
a standard transmission format before sending them to the DODS client
requesting that data.  This translation happens file by file, however,
so datasets that span multiple files may remain somewhat problematic,
even though the file format itself is not an issue.


\subsection{Data Attributes}

Having solved the problem of data storage compatibility, DODS was
faced with the problem of incompatible data attributes.  That is,
considering our two temperature datasets, one dataset could refer to
its temperature data in an array called ``T'', while another could
refer to an array called ``Temp''.  To a human comparing the two
files, the equation between the two might be obvious, but to a
computer it might not.  (Even to a human, the facts might not be as
obvious as they seem.  For example, ``T'' could also refer to
``Time'')

Other data attributes that are important to be able to compute with a
set of data include flags for missing data and unit and unit
conversion information.  The data standard outlined in
\chapterref{dods-standard} of this document is an attempt to address
this issue.



\subsection{Data Organization}

The organization of a dataset can create another impediment to
compatibility.  Consider, for example, two datasets containing
salinity measurements of the ocean at various different depths, and at
different times.  Assuming that there is some reason not to put all
the information in a single file, each dataset must be divided among
several files, and this is where certain incompatibilitles can sneak
in. 

One dataset could store its data in several files, each of which
contains a three-dimensional array---corresponding to latitude,
longitude, and depth---of salinity values, corresponding to a
particular time.  But a similar dataset might be arranged as arrays of
latitude, longitude, and time, with several different files
corrsponding to different depths.  Both arrangements are, in fact,
common among on-line datasets.

The DODS Catalog Server, under development (as of \today), is intended
to address the issue of data organization within a dataset, allowing a
user to query differently organized datasets with the same syntax
query. 


\subsection{Server Interface}

The DODS software does create solutions to the compatibility issues
outlined above.  However, in doing so, it introduces a fourth
compatibility issue, which is that of the server interface itself.  A
DODS server is programmed to respond to a set of server commands
issued by a DODS client.  These commands come in the form of URLs sent
to the server.

Because the DODS project is a widely distributed project, and because
the datasets available through DODS are---by design---not under
central control, it is not always certain that two given DODS servers
will be running software from the same software release.  Similarly,
since there are provisions for making local modifications to server
behavior, it is also not certain that server functions available on
one server are available on another, even when they are both running
at the same software release.

Addressing these issues is an ongoing preoccupation of the DODS
project and its (growing) user base.  The DODS servers also contain
provisions for making this information available to clients. See
\DODSuser\ for more information.


















%%% Local Variables: 
%%% mode: latex
%%% TeX-master: t
%%% End: 
