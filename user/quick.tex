% This is the dods quick start guide for data users
%

\documentclass{dods-book}

\rcsInfo $Id$
\newcommand{\DOCversion}{Version \rcsInfoRevision}

%%% Tex customizations and command definitions for DODS user
%%% guides, 2 October 1997 - tomfool
%%%
%%% $Id$

%%% $Log: layout.tex,v $
%%% Revision 1.5  1999/05/25 20:49:34  tom
%%% changed version numbers to 3.0
%%%
%%% Revision 1.4  1999/02/04 17:39:02  tom
%%% modified for use with dods-book.cls
%%%
%%% Revision 1.3  1998/03/13 20:43:26  tom
%%% writing of API manual.
%%%
%%% Revision 1.2  1998/02/12 15:47:25  tom
%%% updated for GUI doc
%%%
%%% Revision 1.1  1997/10/02 17:18:14  tom
%%% moved from user guide to boiler, made slightly more general,
%%% for use with other guides.
%%%

%%%%%%%%%%%%%%%%%%%%%%%%%%%%%%%%%%%%%%%%%%%%%%%%%%%%%%%%%%%%%%%%%%%%%
%%% This file is for TeX macros that are equally appropriate for the
%%% hard-copy (LaTeX) and html (hyperlatex) versions of the dods
%%% books.  If it's not appropriate for both, then it probably belongs
%%% in dods-book.cls or dods-book.hlx.

%%%%%%%%%%%%%%%%%%%%%%%%%%%%%%%%%%%%%%%%%%%%%%%%%%%%%%%%%%%%%%%%%%%%%
\NotSpecial{\do\_}% This removes the special meaning of `_', so for
                  % subscripts, there must be an `tex' environment
                  % around any diagram using subscripts, and an entire
                  % alternate html figure using <sub> tags.


%%%%%%%%%%%%%%%%%%%%%%%%%%%%%%%%%%%%%%%%%%%%%%%%%%%%%%%%%%%%%%%%%%%%%
%%% Different kinds of cross references.
%\newcommand{\chapterref}[1]{Chapter~\refl{#1} on page~\pagerefl{#1}}
\newcommand{\chapterref}[1]{\link{Chapter~\ref{#1}}{#1}}
\newcommand{\appref}[1]{\link{Appendix~\ref{#1}}%
  [~on page~\pageref{#1}]{#1}}
\newcommand{\sectionref}[1]{\link{Section~\ref{#1}}%
  [~on page~\pageref{#1}]{#1}} 
\newcommand{\pagexref}[1]{\link*{here}[page~\pageref{#1}]{#1}}
\newcommand{\tableref}[1]{\link{table~\ref{#1}}{#1}}
\newcommand{\Tableref}[1]{\link{Table~\ref{#1}}{#1}}
\newcommand{\figureref}[1]{\link{figure~\ref{#1}}{#1}}
\newcommand{\Figureref}[1]{\link{Figure~\ref{#1}}{#1}}

%%% A Prefatory list of the font conventions:
\newcommand{\listconventions} {
\section{Conventions}
\label{pref,conventions}

The \indn{typographic conventions} shown in
Table~\ref{typo-conventions} are followed in this guide and all the
other DODS documentation.

\begin{table}[htbp]
  \begin{center}
  \caption{Typographic Conventions}
  \label{typo-conventions}
  \begin{tabular}{|c|p{3in}|} \hline
    \lit{Literal text}  &  
         Typed by the computer, or in a code listing.\\ \hline
    \inp{User input}    &  
         Type this precisely as written.\\ \hline
    \var{Variables}     &   
         Used as a place holder for a user-specified or variable
         value. Choose an appropriate value and use that in place.\\
         \hline 
    \but{Button Text}\texonly{\rule{0pt}{2.5ex}}   
        &  Used to indicate text on a GUI button.\\ 
         \hline 
    \pdmenu{Menu Name}    &  This is the name of a GUI menu.\\ \hline 
  \end{tabular}
  \end{center}
\end{table}

When referring to a button in a menu, we will often use the notation:
\but{Menu,Button}. For example, \but{Options,Colors,Foreground} would
indicate the \but{Foreground} button in the \pdmenu{Colors} menu,
selected under the \pdmenu{Options} menu.
 }


%%% Local Variables: 
%%% mode: latex
%%% TeX-master: t
%%% End: 

%
% These are html links which are used often enough in writing about DODS to
% merit an input file.
% jhrg. 4/17/94
%
% File rationalized and updated while writing the DODS User
% Guide. Also includes other useful abbreviations.
% tomfool 3/15/96
%
% Moved to dods-def.tex so I can remove links to documents that no
% longer reflect reality.
% tomfool 2/13/98
%
% $Id$
%
% Make sure to include layout.tex *before* using this file.

%% NOTE NOTE NOTE NOTE NOTE NOTE NOTE NOTE NOTE NOTE NOTE NOTE NOTE NOTE 
%%
%% If this file causes problems when running latex, you may have to edit your
%% texmf.cnf file. Here's a meesage from Tom:
%% > Are these references that use relative addresses (like
%% > ../boiler/blah.tex)?  If they are, you should look for the texmf.cnf
%% > file.  (It's often at /usr/share/texmf/web2c/texmf.cnf, and look for
%% > the openout_any parameter.  Check there anyway; there were some recent
%% > (i.e. in the early '90's) security fixes to TeX.
%%
%%%%%%%%%%%%%%%%%%%%%%%%%%%%%%%%%%%%%%%%%%%%%%%%%%%%%%%%%%%%%%%%%%%%%%%%%%%

%%% These are some DODS-specific convenience commands.
\newcommand{\DODSroot}{\lit{\$DODS\_ROOT}}     
% $

\newcommand{\opendap}{OPeNDAP\xspace}

%%% The OPD books and reference material
\newcommand{\OPDDoc}{http://opendap.org/support/docs.html}
\newcommand{\DODSDoc}{http://opendap.org/support/docs.html}

% \newcommand{\OPDDoc}{http://www.unidata.ucar.edu/packages/dods}
% \newcommand{\DODSDoc}{http://www.unidata.ucar.edu/packages/dods}

\newcommand{\OPDhomeUrl}%
  {http://opendap.org}
\newcommand{\OPDexampleUrl}%
  {BROKEN--FIX ME!}%\OPDDoc/examples}
% \newcommand{\OPDftpUrl}%
%  {ftp://dods.gso.uri.edu/pub/dods}
\newcommand{\OPDftpUrl}%
  {ftp://ftp.opendap.org/pub/dods}
\newcommand{\OPDuserUrl}%
  {\OPDDoc/user/guide-html/}
\newcommand{\OPDmguiUrl}%
  {\OPDDoc/user/mgui-html/}
\newcommand{\OPDapiUrl}%
  {\OPDDoc/api/pguide-html/}
\newcommand{\OPDapirefUrl}%
  {\OPDDoc/api/pref-html/}
\newcommand{\OPDffUrl}%
  {\OPDDoc/user/servers/dff-html/}
\newcommand{\OPDquickUrl}%
  {\OPDDoc/user/quick-html/}
\newcommand{\OPDinstallUrl}%
  {\OPDDoc/server/install-html}%
\newcommand{\OPDregexUrl}%
  {\OPDDoc/user/regex-html}%
\newcommand{\OPDjavaUrl}%
  {\OPDDoc/home/swJava1.1/}
\newcommand{\OPDwclientUrl}%
  {\OPDDoc/api/wc-html/}
\newcommand{\OPDwserverUrl}%
  {\OPDDoc/api/ws-html/}
\newcommand{\OPDaggUrl}%
  {\OPDDoc/server/agg-html/}

\newcommand{\OPDhome}{\xlink{OPeNDAP Home page}{\OPDhomeUrl}}
\newcommand{\OPDjava}{\xlink{OPeNDAP Java home page}{\OPDjavaUrl}}
\newcommand{\OPDexamples}{\xlink{OPeNDAP examples page}{\OPDexampleUrl}}
\newcommand{\OPDftp}{\xlink{OPeNDAP ftp site}{\OPDftpUrl}}
%% Book titles do *not* contain the article.
\newcommand{\OPDuser}[1][]{\xlink%
  {\cit{OPeNDAP User Guide}}{\OPDuserUrl{}#1}}
\newcommand{\OPDmgui}{\xlink%
  {\cit{OPeNDAP Matlab GUI}}{\OPDmguiUrl}}
\newcommand{\OPDapi}{\xlink%
  {\cit{OPeNDAP Toolkit Programmer's Guide}}{\OPDapiUrl}}
\newcommand{\OPDapiref}{\xlink%
  {\cit{OPeNDAP Toolkit Reference}}{\OPDapirefUrl}}
\newcommand{\OPDffbook}{\xlink%
  {\cit{OPeNDAP Freeform ND Server Manual}}{\OPDffUrl}}
\newcommand{\OPDquick}{\xlink%
  {\cit{OPeNDAP Quick Start Guide}}{\OPDquickUrl}}
\newcommand{\OPDinstall}{\xlink%
  {\cit{OPeNDAP Server Installation Guide}}{\OPDinstallUrl}}
\newcommand{\OPDregex}{\xlink%
  {\cit{Introduction to Regular Expressions}}{\OPDregexUrl}}
\newcommand{\OPDagg}{\xlink%
  {\cit{OPeNDAP Aggregation Server Guide}}{\OPDaggUrl}}
\newcommand{\OPDwclient}{\xlink%
  {\cit{Writing an OPeNDAP Client}}{\OPDwclientUrl}}
\newcommand{\OPDwserver}{\xlink%
  {\cit{Writing an OPeNDAP Server}}{\OPDwclientUrl}}

\newcommand{\OPDffs}{OPeNDAP Freeform ND Server}

%%% Other DODS links.
\newcommand{\homepage}% For hyperlatex
  {\OPDDoc/}
\newcommand{\OPDsupport}{\xlink{support@unidata.ucar.edu}{mailto:support@unidata.ucar.edu}}
\newcommand{\DODSsupport}{\xlink{support@unidata.ucar.edu}{mailto:support@unidata.ucar.edu}}
\newcommand{\DODS}{\xlink{Distributed Oceanographic Data System}{\OPDhomeUrl}}
\newcommand{\OPD}{\xlink{Open Source Project for a Data Access Protocol}{\OPDhomeUrl}}
\newcommand{\OPDtechList}{\xlink{OPeNDAP Mailing Lists}{\OPDhomeUrl/mailLists/}}

%%% DODS versions
%% This has been removed.  Documents should not have an automatic
%% version number, because then it appears as if they have been
%% updated when they haven't.  Put the relevant version number to
%% whatever software is being described into each document's preface. 

\newcommand{\ifh}{WWW Interface}

% external refs for DODS documents

\newcommand{\CGI}{\xlink{Common Gateway Interface}
  {http://hoohoo.ncsa.uiuc.edu/cgi/overview.html}}

\newcommand{\MIME}{\xlink{Multipurpose Internet Mail Extensions}
  {http://www.cis.ohio-state.edu/htbin/rfc/rfc1590.html}}

\newcommand{\netcdf}{\xlink{NetCDF}
  {http://www.unidata.ucar.edu/packages/netcdf/guide.txn_toc.html}}

\newcommand{\JGOFS}{\xlink{Joint Geophysical Ocean Flux Study}
  {http://www1.whoi.edu/jgofs.html}}

\newcommand{\jgofs}{\xlink{JGOFS}
  {http://www1.whoi.edu/jgofs.html}}

\newcommand{\hdf}{\xlink{HDF}
  {http://www.ncsa.uiuc.edu/SDG/Software/HDF/HDFIntro.html}}

\newcommand{\ffnd}{FreeForm ND}

\newcommand{\Cpp}{\texorhtml
  {{\rm {\small C}\raise.5ex\hbox{\footnotesize ++}}}
  {C\htmlsym{##43}\htmlsym{##43}}}

% Commands

\newcommand{\pslink}[1]{\small
\begin{quote}
  A \xlink{PDF version}{#1} of this document is available.
\end{quote}
\normalsize
}

\newcommand{\pdflink}[1]{\small
\begin{quote}
  A \xlink{PDF version}{#1} of this document is available.
\end{quote}
\normalsize
}

%%%%%%%%%%%%%% DODS macros
%
% These are here so that older latex files will compile. Someday remove these
% and fix the files. 04/13/04 jhrg

\newcommand{\DODShomeUrl}%
  {\OPDhomeUrl}
\newcommand{\DODSexampleUrl}%
  {\OPDDoc/examples}
% \newcommand{\OPDftpUrl}%
%  {ftp://dods.gso.uri.edu/pub/dods}
\newcommand{\DODSftpUrl}%
  {ftp://ftp.opendap.org/pub/dods}
\newcommand{\DODSuserUrl}%
  {\OPDDoc/user/guide-html/}
\newcommand{\DODSmguiUrl}%
  {\OPDDoc/user/mgui-html/}
\newcommand{\DODSapiUrl}%
  {\OPDDoc/api/pguide-html/}
\newcommand{\DODSapirefUrl}%
  {\OPDDoc/api/pref-html/}
\newcommand{\DODSffUrl}%
  {\OPDDoc/user/servers/dff-html/}
\newcommand{\DODSquickUrl}%
  {\OPDDoc/user/quick-html/}
\newcommand{\DODSinstallUrl}%
  {\OPDDoc/server/install-html}%
\newcommand{\DODSregexUrl}%
  {\OPDDoc/user/regex-html}%
\newcommand{\DODSjavaUrl}%
  {\OPDDoc/home/swJava1.1/}
\newcommand{\DODSwclientUrl}%
  {\OPDDoc/api/wc-html/}
\newcommand{\DODSwserverUrl}%
  {\OPDDoc/api/ws-html/}
\newcommand{\DODSaggUrl}%
  {\OPDDoc/server/agg-html/}

\newcommand{\DODShome}{\xlink{OPeNDAP Home page}{\DODShomeUrl}}
\newcommand{\DODSjava}{\xlink{OPeNDAP Java home page}{\DODSjavaUrl}}
\newcommand{\DODSexamples}{\xlink{OPeNDAP examples page}{\DODSexampleUrl}}
\newcommand{\DODSftp}{\xlink{OPeNDAP ftp site}{\DODSftpUrl}}
\newcommand{\DODSuser}[1][]{\xlink%
  {\cit{The OPeNDAP User Guide}}{\DODSuserUrl{}#1}}
\newcommand{\DODSmgui}{\xlink%
  {\cit{The OPeNDAP Matlab GUI}}{\DODSmguiUrl}}
\newcommand{\DODSapi}{\xlink%
  {\cit{The DODS Toolkit Programmer's Guide}}{\DODSapiUrl}}
\newcommand{\DODSapiref}{\xlink%
  {\cit{The DODS Toolkit Reference}}{\DODSapirefUrl}}
\newcommand{\DODSffbook}{\xlink%
  {\cit{The DODS Freeform ND Server Manual}}{\DODSffUrl}}
\newcommand{\DODSquick}{\xlink%
  {\cit{The DODS Quick Start Guide}}{\DODSquickUrl}}
\newcommand{\DODSinstall}{\xlink%
  {\cit{The DODS Server Installation Guide}}{\DODSinstallUrl}}
\newcommand{\DODSregex}{\xlink%
  {\cit{Introduction to Regular Expressions}}{\DODSregexUrl}}
\newcommand{\DODSagg}{\xlink%
  {\cit{OPeNDAP Aggregation Server Guide}}{\DODSaggUrl}}
\newcommand{\DODSwclient}{\xlink%
  {\cit{Writing an OPeNDAP Client}}{\DODSwclientUrl}}
\newcommand{\DODSwserver}{\xlink%
  {\cit{Writing an OPeNDAP Server}}{\DODSwclientUrl}}

\newcommand{\DODSffs}{DODS Freeform ND Server}

% $Log: dods-def.tex,v $
% Revision 1.24  2004/12/21 22:30:04  jimg
% Fixed pslink; Added pdflink.
%
% Revision 1.23  2004/12/14 05:19:17  tomfool
% restored fix to pslink
%
% Revision 1.22  2004/12/09 21:01:58  tomfool
% excised test.dods.org
%
% Revision 1.21  2004/12/09 18:50:21  tomfool
% de-dodsifying
%
% Revision 1.15  2004/04/24 21:37:22  jimg
% I added every directory in preparation for adding everyting. This is
% part of getting the opendap web pages going...
%
% Revision 1.14  2004/02/12 16:05:50  jimg
% Moved the log to the end of the file.
%
% Revision 1.13  2004/01/16 18:05:31  jimg
% Added a note from Tom about setting texmf.cnf to allow \include to process
% files with ../ in their pathnames. You can also change the include to input,
% but I think include may offer some advantages for bigger/complex things like
% the Guides.
%
% Revision 1.12  2003/12/28 21:48:22  tom
% added newer books
%
% Revision 1.11  2003/12/08 19:04:43  tom
% little adjustments for DODS->opendap
%
% Revision 1.10  2003/12/08 18:53:30  tom
% DODS->OPeNDAP
%
% Revision 1.9  2002/07/15 17:49:55  tom
% added \DODSDoc
%
% Revision 1.8  2001/05/04 15:07:45  tom
% fixed pslink to include pdf files
%
% Revision 1.7  2001/02/19 20:39:13  tom
% added links to the new regex intro.
%
% Revision 1.6  2000/03/23 18:27:52  tom
% added abbreviations
%
% Revision 1.5  1999/07/01 16:00:19  tom
% added a couple of web page references
%
% Revision 1.4  1999/05/25 20:49:34  tom
% changed version numbers to 3.0
%
% Revision 1.3  1999/02/04 17:27:08  tom
% adjusted for hyperlatex and dods-book.cls
%
%

%%% Local Variables: 
%%% mode: latex
%%% TeX-master: t
%%% TeX-master: t
%%% End: 

% 
% These are some definitions for the Quick start guide.  Some of these
% should find their way into the dods-book class.
%
% $Id$
%

\htmldirectory{quick-html}
\htmlname{quick}
\htmltitle{DODS Quick Start Guide}
\htmladdress{Tom Sgouros, \rcsInfoDate}
%\htmlcss{/css/quickstyle.css}
%\W\renewcommand{\HlxIcons}[1]{http://top.gso.uri.edu/icons.n/}

\htmlcss{/resources/dods-book.css}
\W\renewcommand{\HlxIcons}[1]{/icons/}

% $Log: quick-def.tex,v $
% Revision 1.4  2004/07/07 22:19:15  jimg
% Added icons and dods-book.css
%
% Revision 1.3  2004/04/24 21:37:25  jimg
% I added every directory in preparation for adding everyting. This is
% part of getting the opendap web pages going...
%
% Revision 1.2  2000/10/30 20:13:37  tom
% pointed icons and css to unidata.
%
% Revision 1.1  2000/10/12 17:24:36  tom
% added quick.tex to repository

%%% Local Variables: 
%%% mode: latex
%%% TeX-master: t
%%% TeX-master: t
%%% End: 


\figpath{quick/figs}

% This is a kluge to use brackets in the optional argument 
% to vcode environments.
\newcommand{\brk}[1]{[#1]}

\makeindex

\begin{document}
%------------------------------------front matter
\title{An \opendap Quick Start Guide\\\DOCversion}
\author{Tom Sgouros}
\date{\rcsInfoDate}
\pagenumbering{roman}
\maketitle

\copyrightmatter

\W\pslink{http://www.opendap.org/pdf/quick.pdf}

%% Preface to the OPeNDAP User Guide
%
% $Id$
%

\T\chapter*{Preface}
\T\addcontentsline{toc}{chapter}{Preface}

This document describes version \OPDversion\ of the \OPD\ (\opendap), a
data system intended to allow researchers transparent access to
oceanographic data---stored in any of several different file
formats---across the Internet. Using \opendap function libraries, many
existing data analysis programs can be easily modified to accommodate
access of distant datasets in a manner identical to the access of
local datasets. \opendap includes a protocol for the transmission of data
across the Internet, and supports selection of data using constraint
expressions, and translation of data from one format to another.

An overview of the system's use is presented, and specific tasks
illustrated, for data providers as well as for users.

\W\htmlmenu{4}
\W\chapter*{Preface}

%%%%%%%%%%%%%%%%%%%%%%%%%%%%%%%%%%%%%%%%%%%%%%%%%%%%%%%%%%%%%%%%%%%%
\section{Tasks Illustrated in this Guide}
\label{pref,tasks}

For a quick start to getting, installing, and using \opendap software, see the
list below of tasks described in this document.

\begin{itemize}

\item Getting the \opendap software. (\pagexref{install})

\item Installing the \opendap software. (\pagexref{install})

\item Using an \opendap client. (\pagexref{opd-client,using-example})

\item Re-linking a data analysis or display application to become a
  \opendap client. (\pagexref{opd-client,link-example})

  \tbd{Add information about the locator here.}
%\item Using the OPeNDAP data locator to find data. 
%(\pagexref{locator,locator})

\item Creating and installing an \opendap server.

  \begin{itemize}
    
  \item Installing the \opendap server and CGI filters.
    (\pagexref{opd-server,cgi-install})

  \item Starting and configuring the httpd server. Due to the variety
    of available servers, this task is beyond the scope of the manual.
    Please refer to the documentation for the particular server in question
    for more information.

  \item Implementing a new DODS-compliant API. (\OPDapi)

  \end{itemize}

\item Writing an \opendap CGI program. (\OPDapi )

\item Writing the CGI service programs. (\OPDapi )

\item A list of all supported APIs. (\pagexref{opd-client,supported-APIs})

\end{itemize}


%%%%%%%%%%%%%%%%%%%%%%%%%%%%%%%%%%%%%%%%%%%%%%%%%%%%%%%%%%%%%%%%%%%%
\section{Who is this Guide for?}
\label{pref,who}

The user documentation for \opendap covers two groups of users: those who
want to provide access to data vian \opendap and those who want to use
data. In many cases the people will be one and the same since most
providers will also be data users.

This documentation assumes that the readers are familiar with
computers, but are not necessarily programmers. 

This guide also contains technical information that will be of
assistance to programmers who plan to write new DODS-compliant APIs
for as yet unsupported data models. Data providers and data consumers
may find some general questions answered by this material, but it is
not necessary to know any of it in order to use the system.


%%%%%%%%%%%%%%%%%%%%%%%%%%%%%%%%%%%%%%%%%%%%%%%%%%%%%%%%%%%%%%%%%%%%
\section{Organization of this Document}

This book is organized into separate sections for data providers, data
consumers, and technical reference material for programmers.

\begin{description}

\item{\bf Part I} is for everybody who wants to use \opendap.

  \begin{description}
    
  \item{\bf\chapterref{intro,opd}} provides a high-level overview of
    the entire system.

  \end{description}
  
\item{\bf Part II} is for data consumers, that is, the people who want
  to look at data using the \opendap system.

  \begin{description}
    
  \item{\bf\chapterref{opd-client}} shows how to look at data using
    \opendap.  It includes a section about the theoretical and practical
    problems of data model translation.  It also explains how to build
    an \opendap client, which is the program used to look at \opendap data.
    
\tbd{Put stuff about using the CS here.}
%  \item{\bf\chapterref{locator}} introduces and explains the data
%    locator and the other facilities provided with OPeNDAP to hunt for
%    data.  It includes information about the server catalog and the
%    boolean services.

  \end{description}
  
\item{\bf Part III} is for data providers, or people who want to make
  their data available through \opendap servers.

  \begin{description}
    
  \item{\bf\chapterref{opd-server}} shows how to use \opendap to
    make your data available to others.  It explains how to set up a
    \opendap server to provide \opendap data to \opendap clients, and also
    contains information about modifying or writing an \opendap server.

\tbd{Put stuff about setting up a CS here.  Also participating in the
  grand CS.}
%  \item{\bf\chapterref{register}} explains how to advertise one's data.
%    It shows how to enter the location of a data set into the data locator
%    database, how to assemble the data catalog, and discusses the role of the
%    data provider.

  \end{description}
  
\item{\bf Part IV} contains technical information about how \opendap
  works. This information is provided to people who want to write new
  libraries to use \opendap through a currently unsupported API.

  \begin{description}
    
  \item{\bf\chapterref{data}} contains general information about
    Data and Data Models.  This is important information to have for
    people intending to use \opendap to provide data to others.  It covers
    the \opendap data attribute and data descriptor structures.  The
    chapter also contains a section outlining the problems associated
    with Data Model translation.

  \end{description}

\item{\bf Appendices}

  \begin{description}
    
  \item{\bf\appref{install}} contains the instructions for installing
    the \opendap libraries, and software that requires these libraries.

  \item{\bf Glossary} A small but useful collection of terms.

  \end{description}

\end{description}

%%%%%%%%%%%%%%%%%%%%%%%%%%%%%%%%%%%%%%%%%%%%%%%%%%%%%%%%%%%%%%%%%%%%

\listconventions

\tbd{Are there any other conventions that need illustration?}

% $Log: preface.tex,v $
% Revision 1.6  2003/09/04 19:42:06  tom
% DODS->OPeNDAP
%
% Revision 1.5  1999/08/30 14:19:28  tom
% updates prior to DODS release 3.0
%
% Revision 1.4  1999/02/04 17:42:13  tom
% modified to use dods-book.cls and Hyperlatex
%

%%% Local Variables: 
%%% mode: latex
%%% TeX-master: t
%%% End: 


\tableofcontents
\listoffigures
%\listoftables

\clearemptydoublepage
%------------------------------------book body

\pagenumbering{arabic}

%% Chapter one to the OPeNDAP User Guide
%
% $Id$
%

\chapter{What is \opendap?}
\label{intro,opd}

The \opendap provides a way for ocean researchers to
access oceanographic data anywhere on the Internet from a wide variety of new
\emph{and existing} programs. By developing network versions of commonly used
data access Application Program Interface (API) libraries, such as
\xlink{NetCDF}
{http://www.unidata.ucar.edu/packages/netcdf/guide.txn_toc.html}\ ,
\xlink{HDF}
{http://www.ncsa.uiuc.edu/SDG/Software/HDF/HDFIntro.html}\ ,
\xlink{JGOFS} {http://www1.whoi.edu/jgofs.html}\ , and others,
the \opendap project can capitalize on years of development of data analysis and
display packages that use those APIs, allowing users to continue to use
programs with which they are already familiar.

The \opendap \ind{architecture} uses a client/server model, with a {\em
  {\ind{client}}} that sends requests for data out onto the network to some
{\em {\ind{server}}}, that answers with the requested data. This is exactly
the model used by the \xlink{World Wide Web}
{http://www.w3.org/hypertext/WWW/TheProject.html}\ where client programs
called browsers submit requests to web servers for the data that make up web
pages. Of course, \opendap clients can do much more than browse this data.  Using
flexible data types suitable for many uses, including scientific data, the
\opendap servers deliver real data directly to the client program in the format
needed by that client.

In fact, the network communication model used by \opendap uses URL
addresses and web servers ({\tt httpd}) to deliver data to the
researcher.  This is done by using the \opendap software to convert a
researcher's data analysis software into a sophisticated (though
specialized) web browser. In addition to providing network-compatible
versions of popular data access APIs, the \opendap project also
provides a software client and server toolkit to help other developers
create network-compatible \opendap versions of other APIs.

To expand the universe of data available to a user, \opendap incorporates
a powerful data translation facility, so that data may be stored in
data structures and formats defined by the data provider, but may be
accessed by the user in a manner identical to the access of local data
files on the user's own system. Though there are limitations on the
types of data that may be translated (See \sectionref{data,trans}),
the facility is flexible and general enough to handle many of the
possible translation.  There are two important results:

\begin{itemize}

\item A user may not need to know that data
from one set are stored in a format different from data in another
set. Further, it may be possible that {\em neither} data set is
stored in a format readable by the original (i.e. without \opendap)
version of the data analysis and display program he or she uses.

\item No segment of \opendap users will be effectively cut off from 
accessing data because of its storage format. A scientist who wishes
to make his or her data available to other \opendap users may do so while
keeping that data in what may actually be a highly idiosyncratic
storage format. Of course, it doesn't have to be in a highly
idiosyncratic format.  The point is that \opendap can handle a wide variety
of possible cases.

\end{itemize}

The combination of the \opendap network communication model and the data
translation facility make \opendap a powerful tool for the retrieval,
sampling, and display of large distributed datasets. Though \opendap was
developed by oceanographers, its application is not constrained to
oceanographic data. The organizing principles and algorithms may be
applied to many other fields where data can be stored on computers.

The population of people who may be interested in a system such as
\opendap may be divided into data consumers and data providers. Though it
was an important observation to the development of \opendap that the two
roles are often assumed by the same scientists, the division is a
useful one for the introduction of the system. The following two
sections provide a broad introduction to the roles of data consumer
and data provider. The remainder of this guide is organized around
this distinction between classes of users.

\section{Why Use \opendap to Read Data?}
\label{intro,consumer}

A scientist wishing to examine and sample some dataset will typically
be comfortable using a relatively small number of data analysis and
display programs or packages. Some of these packages will use one of
the popular data access APIs currently available. However, few data
access APIs provide direct access to \ind{distributed data}
\indc{remote data}\footnote{The phrase {\em distributed data\/}
refers to datasets that reside on different computers which are linked
by a network such as the Internet. The computers may or may not be
physically remote from each other. The main point is that the
computers manage their data resources independently. In this guide the
terms {\em remote\/} and {\em distributed\/} are used to imply
independently managed resources.}, so this access must be made with
network tools, such as web browsers or {\tt ftp}. While
relatively straightforward in principle, this process can nonetheless
become time-consuming and somewhat challenging in practice.

The following example illustrates some of the differences between
accessing distributed data with the tools currently in widespread use,
and the same operation using \opendap.

\subsection{An Example: Using ftp}
\label{intro,ftp-example}

The advent of the WWW has made possible simple data browsers that
allow sophisticated interactive sampling of on-line datasets. Using a
web browser and {\tt{\ind{ftp}}}, a user can sample any of several large
oceanographic datasets available on the Internet. However, there are
several problems with these data search engines that may only become
apparent when a user actually tries to use the data.

Among the problems that can arise are those that appear when a user
tries to use the results of one dataset to search a second
dataset. Suppose that a user wishes to choose a sea-surface
temperature image from the NOAA/NASA Pathfinder AVHRR archive at:

\begin{vcode}{ib}
http://podaac-www.jpl.nasa.gov/mcsst/mcsst_subset.html
\end{vcode}

using the results of a
time-series generated from the COADS Climatology archive at:

\begin{vcode}{ib}
http://ferret.wrc.noaa.gov/fbin/climate_server   
\end{vcode}

The steps are theoretically straightforward:

\begin{enumerate}

\item Create the time series from the COADS Climatology archive. This
is done by answering the menu of options on the COADS web page.

\item Import the time series from step 1 to the user's local data
analysis system.  Note that this step may itself require several steps:

\begin{enumerate}
\item The data must be down-loaded, using {\tt ftp} or a similar
program.

\item Once down-loaded, the data may have to be converted into a format
that can be read by the data analysis program.
\end{enumerate}

\item Examine the data and formulate a request to the AVHRR archive. This
is again done by answering the menu of option on the AVHRR Web page.  Note
that the COADS and AVHRR pages are not completely compatible in this respect.
For example, the date formats of the two pages are different.

\item Import the result of step 3 to the user's local data display system.
This may also require several steps:

\begin{enumerate}
\item The data must be down-loaded again.

\item And again, once down-loaded, the data may have to be converted
into a format that can be read by the data analysis program.  Note that
the set of available formats on the COADS page are distinct from the available
options from the AVHRR archive.
\end{enumerate}
\item Think about the results.

\end{enumerate}

Though the procedure is straightforward and the web servers designed
to make sampling the datasets a simple task, upon close examination,
the combination of the steps may create unforeseen difficulties. For
example, a request to the COADS server will return either a spreadsheet 
suitable for use on a PC, a netCDF format file, or a file in one
of a selection of simple ASCII formats.
If the user is fortunate, the returned file will already be in a
format compatible with the desired analysis package. But not all users
will be so fortunate.  Often this file must be converted to some
other file format before it can be imported to the user's analysis
program. This may or may not be a simple task.

Even a file format for which a user is properly equipped may be used
in an unfamiliar manner. For example, the independent and dependent
variables might be in a different order or an ASCII data file may use
tabs instead of spaces.

Assuming the import of the COADS data has been accomplished and
boundaries for the AVHRR search identified, the task of selecting from
the second archive may begin. Unfortunately, the request to the AVHRR
archive will return either a GIF picture, an HDF format file, or a raw
(binary) data file. Again, importing this output into the user's
analysis program may or may not be simple, but it will not be the same
procedure as the one used for the first data request.

Other problems are also apparent. The COADS Climatology sampling
program requests the user supply dates (month and day), whereas the
AVHRR archive asks for the ``Julian day'' (an integer between 1 and
365 or 366). One server will accept ``S'' and ``W'' to indicate South
latitudes and West longitudes, while the other requires that these be
indicated with negative coordinate values. The sampling of the COADS
dataset, while flexible, may not allow sampling in the manner the user
needs. It cannot, for example, provide a section except along a line
of constant latitude or longitude. If a user wanted to see a section
along a NE-SW line, it would be a challenging and time-consuming
task to assemble one from many small data requests.

Further, it might be desirable to use the results of sampling these
two databases to construct a time series. This could conceivably mean
repeating the entire procedure many times.

\subsection{An Example: Using \opendap}
\label{intro,opd-example}

To produce the same data selection using \opendap, a user would follow
essentially the same steps. However, the steps themselves would be
performed differently. Once the user's data analysis package has been
converted to an \opendap client\tbd{fix this xref?}
(\sectionref{opd-client,link}), the \tbd{add xref to install GUI
  clients} 
accesses to the remote datasets are made through the analysis package
itself. Instead of specifying a data file by a pathname reference to
some local disk file, the user specifies a URL, which may point to
either a local or a remote dataset.  Here is a re cap of the same operation,
outlined as they would be performed by an \opendap application program:

\begin{enumerate}

\item Create the time series from the COADS Climatology archive. This is
done by using the sampling facilities of whatever data analysis program
a scientist is familiar with.  If desired, \opendap constraint expressions
may be used to reduce the network load, or to provide a sampling scheme
not supported by the data analysis program.

\item The data need not be imported to the user's data analysis program,
since it was down-loaded and converted automatically in step 1.

\item Examine the data and formulate a request to the AVHRR archive. This
is again done through the sampling facilities of whatever data analysis
program the user is using, and \opendap constraint expressions.  Note that,
whatever their actual format, both COADS and AVHRR archives appear to the
\opendap client to be stored in identical formats.

\item The data need not be imported to the user's data analysis program,
since it was down-loaded and converted automatically in step 3.

\item Think about the results.

\end{enumerate}

It is important to note that {\em any} data analysis package that can
handle one of the DODS-supported data access APIs can be converted
into an \opendap client program capable of reading data stored by {\em all}
of the DODS-supported data access APIs. (There are some limitations on
translation. See \sectionref{intro,opd-client} and
\sectionref{data,trans} for more information.) Therefore, assuming
the user has some analysis package capable of doing the required
sampling and analysis on local data, all the steps would be performed
from within that package, just as if the user were operating on local
files. The result is a simpler procedure, even though the same 
essential steps are followed.

The \opendap scenario has, among others, the following advantages:

\begin{itemize}

\item The user need not learn about any of the archival formats, since
the \opendap server and client cooperate to deliver the data in the format
in which the analysis package expects to see it. Whereas the user of
the ftp server has to worry about importing the data into the analysis
program, the \opendap client program imports it transparently.

\item The user can sample the distant datasets in any fashion supported
by his or her own (local) analysis package. Unnecessary data need not
be sent over the Internet.

\item By appending a {\em{\ind{constraint expression}}}
to the URLs given to
the analysis program, the user can sample data using techniques that
their analysis program \emph{cannot} do.\footnote{For example, suppose
a user wishes to access the NODC XBT database using a program that
uses the netCDF API. A program that can process the arrays that netCDF
manipulates are largely unsuitable for XBT station data. However, a
user can define constraint expressions in the URL to sample the data
and deliver it in a form the netCDF API can use. For more information
about constraint expressions, see
Section~\ref{opd-client,constraint}. For more information about data
models and translation, see Chapter~\ref{data}.}\tbd{Use a different
example in the footnote}

\item A substantial amount of the searching and sampling is performed
on the server machines. This reduces Internet traffic, as well as
decreasing the load on the local machine.

\end{itemize}

\subsection{The \opendap Client}
\label{intro,opd-client}

\opendap uses a client/server model. As mentioned, the \opendap
servers are simply {\tt httpd} web servers, equipped to interpret an \opendap URL sent to them. (See \chapterref{opd-server}.) The \opendap client
program can be any program that uses one of the supported APIs, such
as JGOFS or netCDF.\footnote{Or a program specially developed to 
read data from \opendap servers.}

Without \opendap, an application program that uses one of the common data
access APIs such as netCDF will operate as shown in \figureref{intro,fig,unlinked}.  
The user
makes a request for data from the application program.  The program in turn
uses procedures defined by the data access API to access the data,
which is stored locally on the host machine.  Some APIs are somewhat more
sophisticated than this, of course, but their general operation is
similar to this outline.

\figureplace{The Architecture of a Data Analysis Package.}{htbp}
{intro,fig,unlinked}{unlinked.ps}{unlinked.gif}{}

The operation of an \opendap client is illustrated in \figureref{intro,fig,linked}.  
Here, the
\emph{same application program} that was used in \figureref{intro,fig,unlinked} 
has been linked
with an \opendap version of the data access API.  Now, in addition to being
able to use local data as before, the application program is able to access
data from \opendap server anywhere on the Internet in the same manner as the
local data.

To make some program into an \opendap client, it must only be re-linked with
the \opendap implementation of the supported API library. This is a simple
process, generally requiring only a few minutes. The process will
create a program that accepts URLs, specifying a location for the data
somewhere on the Internet, in addition to file pathnames which only
specify a location on the local platform's file system. (See
\sectionref{opd-client,link}.) 

\figureplace{The Architecture of a Data Analysis Package Using \opendap.}{htbp}
{intro,fig,linked}{linked.ps}{linked.gif}{}

\opendap also provides a data translation facility. Data from the original
data file is translated by the \opendap server into an \opendap data model for
transmission to the client. Upon receiving the data, the client
translates the data into the data model it understands. (See
\chapterref{data} for more information about the \opendap data model.)
Because the data transmitted from an \opendap server to the client travel
in the \opendap format, the data set's original storage format is completely
irrelevant to the user of an \opendap client. If the client was originally
designed to read netCDF format files, the data returned by the
\opendap-netCDF library will appear to have been read from a netCDF file,
whatever the actual format of the files from which the data were
read\footnote{Note that there is a limit to what can be translated. An
API meant to support two-dimensional arrays may be able to handle
one-dimensional vector data, but a program designed to process
one-dimensional vector data will not know what to do with a
two-dimensional array. The set of data access APIs supported by \opendap
contain several such mismatches. See
Section~\ref{data,trans} for more information.}. If the
program expects JGOFS data, the DODS-JGOFS library will return data
that seem to have come from a JGOFS dataset, again, no matter what the
actual input file format.

\opendap does not pretend to remove all the overhead of data searches. A
user will still have to keep track of the URLs of interesting data
sets in the same way a user must now keep track of the names of files
containing interesting data.  an \opendap \new{catalog service} is in the
process of being constructed that will help users scan the available
datasets.  

\section{Providing Data with \opendap}

The \opendap data provider is the person or organization willing to make
their digital datasets available to the community with an \opendap server.

The designers of \opendap recognized that many of the data users are also
the data providers, and \opendap was built with a recognition that
providing the data should be as simple and as straightforward as
possible. In many cases, once a local web server is equipped to become
an \opendap server, a scientist need do very little beyond what must
be done simply to make the data available locally. (i.e., Put the data
into a file format that can be read by the locally used data analysis
and display programs.) The tasks of a data provider can be separated
into three parts:

\begin{itemize}
\item Install and configure the \opendap server.
  (\sectionref{opd-server,install}.)

\item Create whatever ancillary data files are needed by the data
set (if any). (\sectionref{intro,ancillary}.)

\tbd{Information here about registering the data set/server.}
%\item Register the data set with the master directory (optional).
%(\chapterref{register}.) 

\tbd{Information here about the data catalog.}
%\item Create the data catalog.

\end{itemize}

\subsection{The \opendap Server}
\label{intro,opd-server}

The \opendap data server is simply made up of a regular \lit{httpd} server
equipped with CGI programs (or filters) that will respond to requests
for dataset structure, data attributes, and data itself. (See
\sectionref{data,dap} for a description of the data returned by these
requests and see \sectionref{opd-client,url} for a description of the
\opendap URL syntax used to send these requests.)  Most of the task of a
data provider consists of configuring this server.  While perhaps not
a trivial task, it potentially represents far less effort than
packaging a dataset for submission to some central data archive.
Furthermore, modifying a server's configuration to accommodate new
data will be an almost trivial task, involving the simple editing of a
configuration file.

\subsection{Ancillary Data}
\label{intro,ancillary}

In order for an \opendap client to accept data from an \opendap server, it must be able
to allocate the data structures and arrange internal labels to organize the
incoming data.  The information the client library needs to do this
organizing is called the ancillary data\footnote{It is also referred to as
  the Data Descriptor Structure and the Data Attribute Structure.  See
  Chapter~\ref{data} for more details about these structures.}.  For many
APIs, the ancillary data is inherent in the data files themselves, and the
\opendap server can glean that information by scanning the data files.  For large
data archives, where scanning the data files is impractical, and that might
not change often, \opendap can cache the ancillary data to speed access times.
When a client requests the ancillary data, the \opendap server can check this
data cache first before scanning the data files.

This feature is useful in other cases because not all data file formats
are self-describing.  For example, a data set might contain several files of 
time vs. temperature data; the header information describing which numbers
are temperature and which time may be in a different file or may simply
be understood by the user of the local data analysis program equipped
to look at this data.  As an example, data accessed by \opendap servers using
the FreeForm data access API require provider-created ancillary data files.

\subsection{Administration and Centralization of Data}
\label{intro,admin}

Under \opendap, there is no central archive of data.  Data under \opendap
is organized in a manner similar to the World Wide Web itself.  That
is, all one need do to make one's data available is to start up a
properly configured {\tt httpd} server on an Internet node that has
access to the data to be served.  Each data provider is free to join
and to leave the system when it is convenient, just as any proprietor
of a web page is free to delete it or add to it as whimsy demands.

Of course, as can also be seen on the World Wide Web, there are some
disadvantages to the lack of central authority.  If no one knows about
a web site, no one will visit it.  Similarly, listing a dataset in a
central data catalog, such as the Global Change Master Directory
(\xlink{\lit{http://gcmd.gsfc.nasa.gov/}}{http://gcmd.gsfc.nasa.gov}),
can make data available to other researchers in a way that simply
configuring an \opendap server does not.  \opendap provided a facility for
registering a data set with the GCMD catalog, which makes the data set
known to the \opendap data location service.

\tbd{Information here about the catalog server and data location.}
%OPeNDAP provides a data locator component to be a ``search engine'' to 
%help researchers find an use the available data sets.  A data provider
%must register a data set with the data locator service in order to make
%it available to the public.  See Chapters~\refl{locator} and \refl{register} 
%for more information about this tool.

%The data catalog\footnote{Don't confuse the central data catalog
%at the Global Change Master Directory with the data catalog maintained
%by an OPeNDAP provider.  The first is a list of OPeNDAP data servers, available
%in a single central location, whereas the second is the list of data 
%available from each server.} used by the data locator may be important to
%the ultimate success of the OPeNDAP project.  Setting up an OPeNDAP server and
%creating OPeNDAP clients to use these data is a process requiring some 
%investment in time.  In order for OPeNDAP to become widely used, a substantial
%number of users must believe it is worth that investment.  The availability
%of a large number of datasets through OPeNDAP can provide that incentive,
%but only if they are known to a large number of researchers.

The remainder of this book will be divided into three major sections:
instructions on the building and operating of \opendap clients; a tutorial
and reference on running \opendap servers and making data available to \opendap
clients; and technical documentation describing the implementation details
(and the motivation behind many of the design decisions) of the \opendap
software.



\tbd{Move Preface ``getting started with \opendap'' section to here?}


% $Log: ch01.tex,v $
% Revision 1.10  2004/11/09 14:43:58  tomfool
% forgot a DODS reference
%
% Revision 1.9  2004/08/24 22:51:32  jimg
% Fixed broken label/refs.
%
% Revision 1.8  2003/09/04 19:42:06  tom
% DODS->OPeNDAP
%
% Revision 1.7  2000/10/04 15:02:13  tom
% changed \figureplace definition, misc other cleaning
%
% Revision 1.6  2000/03/23 18:26:14  tom
% misc. updates
%
% Revision 1.5  1999/08/30 14:19:27  tom
% updates prior to DODS release 3.0
%
% Revision 1.4  1999/02/04 17:42:13  tom
% modified to use dods-book.cls and Hyperlatex
%
%
%


%%% Local Variables: 
%%% mode: latex
%%% TeX-master: t
%%% End: 



%  Outline of this book:
%
%
%  A Quick Start: What to do with a URL

\T\chapter{What To Do With An \opendap URL}

The \OPD\ is a system that allows you to access data over the internet,
from programs that weren't originally designed for that purpose, as
well as some that were.

With \opendap, you access data using a URL, just like a URL you would use
to access a web page.  However, before you request any data, you need
to know how to request it in a form your browser can handle.  \opendap
data is stored in binary form, and by default, it is transmitted that
way, too.\indc{DODS!URL}\indc{data!accessed by URL}\indc{URL!data accessed by}

The other problem with an \opendap URL is that a single URL might point to
an archive containing 50 megabytes of data.  You rarely want to
request the whole thing without knowing a little about it.  \opendap
provides sophisticated sub-sampling capabilities, but you need to know
a little bit about the data in order to use them.

\texorhtml{}{So here's what to do if someone gives you a raw URL, and
  says there's some \opendap data on the other end.

\htmlmenu{4}
\chapter{What To Do With An \opendap URL}}
\label{reynolds,chapter}

Suppose someone gives you a hot tip that there's a lot of good data
at:

\begin{vcode}{sib}
http://www.cdc.noaa.gov/cgi-bin/nph-nc/Datasets/reynolds_sst/sst.mnmean.nc
\end{vcode}

This URL points to monthly means of sea surface temperature,
worldwide, compiled by Richard \ind{Reynolds} at the Climate Modeling branch
of NOAA, but pretend you don't know that yet.\indc{Climate
  Modeling!NOAA}\indc{NOAA!Climate Modeling} 

The simplest thing you can do with this URL is to download the data it
points to.  You could feed it to an \opendap-enabled data analysis package
like Ferret, or you could append \lit{.asc}, and feed the URL to a
regular web browser like Netscape.  This will work, but you don't
really want to do it because in binary form, there are about 28
megabytes of data at that URL.

\note{An \opendap server will work with many different clients, some of
  which are supported by the \opendap team, and some of which are
  supported by others.  The operation of any individual package is
  beyond the scope of this manual.  This guide explains how to use a
  typical web browser such as Netscape Navigator to discover
  information about the data that will be useful when analyzing data
  in \emph{any} package.}

\subj{You need to sample the data}
A better strategy is to find out some information about the data.
\opendap has sophisticated methods for subsampling data at a remote site,
but you need some information about the data first.  First, we'll try
looking at the data's \new{Dataset Descriptor Structure} (\ind{DDS}).  This
provides a description of the ``shape'' of the data, using a vaguely
C-like syntax.  You get a dataset's DDS by appending \lit{.dds} to the
\xlinkn{URL}{http://www.cdc.noaa.gov/cgi-bin/nph-nc/Datasets/reynolds_sst/sst.mnmean.nc.dds}.\indc{dataset!shape}\indc{shape of dataset}

\figureplace{An \opendap DDS (\lit{sst.mnmean.nc.dds})}{htb}
{reynolds,dds}{reynolds-dds.ps}{reynolds-dds.gif}{http://www.cdc.noaa.gov/cgi-bin/nph-nc/Datasets/reynolds_sst/sst.mnmean.nc.dds}

From the DDS shown, you can see that the dataset consists of five
pieces: 

\subj{Find out what's in the data}
\begin{itemize}
\item A 180-element vector called ``lat'',
\item A 360-element vector called ``lon'',
\item A 226-element vector called ``time'',
\item A ``Grid'' containing a three-dimensional array of integer
  values (\lit{Int16}) called \lit{sst}, and three ``Map'' vectors,
  which may look familiar, and
\item Another Grid called \lit{mask}.
\end{itemize}

The \new{Grid} is a special \opendap data type that includes a
multidimensional array, and \new{map vectors} that indicate the
independent variable values.  That is, you can use a Grid to store an
array where the rows are not at regular intervals.  \texorhtml{There's
  a simple grid in figure~\ref{grid,diagram}.}{Here's a simple grid:}
\indc{array!irregular|see{Grid}}\indc{array!map vector}\indc{vector!map}

\figureplace{A Grid}{h}{grid,diagram}{gridpts.ps}{gridpts.gif}{}

The array part of the grid would contain the data points measured at
each one of the squares, the X map vector would contain the positions
of the columns, and the Y map vector would contain the positions of
the rows.

Of course you can also use a Grid to store arrays where the columns
and rows are at regular intervals, and you'll often see \opendap data that
way. 

(The other special \opendap data type worth worrying about is the
\emph{Sequence}.  You'll see more about them in
section~\ref{quick,sequences}.  There are also \new{Structures} and
\new{Lists}, but they exist largely for internal uses, and you don't
often see these used in real datasets.)

You can see from the DDS that the Reynolds data is in a 180x360x226
element grid, and the dimensions of the Grid are called ``lat'',
``lon'', and ``time''.  This is suggestive, but not as helpful as one
could wish.  To find out more about what the data \emph{is}, you can
look at the other important \opendap structure: the \ind{DAS}, or
\new{Data Attribute Structure}.  This is somewhat similar to the DDS,
but contains information about the data, such as units and the name of
the variable.  Part of the DAS for the Reynolds data we saw above is
\texorhtml{shown in ~\ref{reynolds,das}.}{shown in the figure below.
  Click \xlinkn{here}{http://www.cdc.noaa.gov/cgi-bin/nph-nc/Datasets/reynolds_sst/sst.mnmean.nc.das}
  or on the figure to see the rest of it.}

\figureplace{An \opendap DAS (\lit{sst.mnmean.nc.das})}{h}
{reynolds,das}{reynolds-das.ps}{reynolds-das.gif}
{http://www.cdc.noaa.gov/cgi-bin/nph-nc/Datasets/reynolds_sst/sst.mnmean.nc.das}

\note{The DAS is populated at the data provider's discretion.  Because
  of this, the quality of the data in it (the \new{metadata}) varies
  widely.  The data in the Reynolds dataset used in this example are
  \ind{COARDS} compliant.  Other metadata standards you may encounter with
  \opendap data are \ind{HDF-EOS}, \ind{EPIC}, \ind{FGDC}, or no metadata at all.}

\subj{Find out more about the data variables}
Now we can tell something more about the data.  Apparently the
\lit{lat} vector contains latitude, in degrees north, and the range is
from 89.5 to -89.5.  Since this is a global grid, the latitude values
probably go in order.  We can check this by asking for just the
latitude vector, like \xlinkn{this}%
{http://www.cdc.noaa.gov/cgi-bin/nph-nc/Datasets/reynolds_sst/sst.mnmean.nc.asc?lat}:

\begin{vcode}[.]{sib}
http://www.cdc.noaa.gov/cgi-bin/nph-nc/Datasets/reynolds_sst/sst.mnmean.nc.asc?lat
\end{vcode}

What we've done here is to append a \new{constraint expression}
 to the \opendap URL, to indicate how to
constrain our request for data.  Constraint expressions can take many
forms.  This guide will only describe a few of them.  (You can refer
to the \OPDuser\ for more complete information about constraint
expressions.)  Try requesting the
\xlinkn{time}%
{http://www.cdc.noaa.gov/cgi-bin/nph-nc/Datasets/reynolds_sst/sst.mnmean.nc.asc?time}
and \xlinkn{longitude}
{http://www.cdc.noaa.gov/cgi-bin/nph-nc/Datasets/reynolds_sst/sst.mnmean.nc.asc?lon}
\subj{The info service also provides the DAS and DDS information.}
vectors to see how this works.
\indc{expression!constraint|see{constraint expression}}

According to the DAS, time is kept in ``days since 1-1-1 00:00:00'' in
this dataset.  You can also learn from the DAS the actual time period
recorded in the data which, because of your familiarity with the
\ind{Julian calendar}, you instantly recognize as beginning in November,
1981.  You might also notice that the \lit{mask} array is used to
indicate land and sea, and has only the values 0 and
1.\indc{calendar!Julian} 

\indc{service!help}\indc{help service}\indc{service!info} 
\opendap provides an \new{info service} that returns all the information
we've seen so far in a single 
request.  The returned information is also formatted differently (some
would say ``nicer''), and you can occasionally find server-specific
documentation here, as well.  Some will find this the easiest way to
read the attribute and structure information.  You can see what
information is available by appending \lit{.info} to a URL, like
\xlinkn{this}
{http://www.cdc.noaa.gov/cgi-bin/nph-nc/Datasets/reynolds_sst/sst.mnmean.nc.info}:

\begin{vcode}[.]{sib}
http://www.cdc.noaa.gov/cgi-bin/nph-nc/Datasets/reynolds_sst/sst.mnmean.nc.info
\end{vcode}

\section{Peeking at Data}

Now that we know a little about the shape of the data, and the data
attributes, let's look at some of the data.

\subj{Use subscripts to sample a Grid.}
You can request a piece of an array with \ind{subscripts},
just like in a C program or in Matlab or many other computer
languages.  Use a colon to indicate a subscript range.
\indc{Grid!sampling}\indc{sampling!Grid}\indc{array!subscripts}
\indc{array!sampling}\indc{sampling!array}\indc{Grid!subscripts}
\indc{peeking at data}\indc{data!peeking at}

\begin{vcode}[http://www.cdc.noaa.gov/cgi-bin/nph-nc/Datasets/reynolds_sst/sst.mnmean.nc.asc?time\[0:6\]]{sib}
...sst/mnmean.nc.asc?time[0:6]
\end{vcode}

This \xlinkn{URL}{http://www.cdc.noaa.gov/cgi-bin/nph-nc/Datasets/reynolds_sst/sst.mnmean.nc.asc?time[0:6]} will produce \texorhtml{figure~\ref{reynolds,timevec}}{the following:}

\figureplace{Part of a vector.}{h}{reynolds,timevec}{timevec.ps}{timevec.gif}{}

You can do the 
\xlinkn{same}
{http://www.cdc.noaa.gov/cgi-bin/nph-nc/Datasets/reynolds_sst/sst.mnmean.nc.asc?mask[28:30][206:209]}
for one of the grids:

\begin{vcode}[http://www.cdc.noaa.gov/cgi-bin/nph-nc/Datasets/reynolds_sst/sst.mnmean.nc.asc?mask\[28:30\]\[206:209\]]{sib}
...sst/mnmean.nc.asc?mask[28:30][206:209]
\end{vcode}

\subj{Sampling a Grid produces part of the Grid, including the map vectors.}
Which produces a portion of the land mask somewhere near Alaska's
Kenai peninsula\texorhtml{, shown in figure~\ref{reynolds,mask}}{:}

\figureplace{Part of an \opendap Grid.}{h}{reynolds,mask}{mask.ps}{mask.gif}{http://www.cdc.noaa.gov/cgi-bin/nph-nc/Datasets/reynolds_sst/sst.mnmean.nc.asc?mask[28:30][206:209]}

Notice that when you ask for part of an \opendap Grid, you get the array
part along with the corresponding parts of the map vectors.

If you are interested in the Reynolds dataset, you are probably more
interested in the sea surface temperature data than the land mask.
The temperature data is a three-dimensional grid.  To sample the 
\xlinkn{\lit{sst}}
{http://www.cdc.noaa.gov/cgi-bin/nph-nc/Datasets/reynolds_sst/sst.mnmean.nc.asc?sst[12:13][28:30][206:209]}
Grid, you
just add a dimension for time:

\begin{vcode}[http://www.cdc.noaa.gov/cgi-bin/nph-nc/Datasets/reynolds_sst/sst.mnmean.nc.asc?sst\[12:13\]\[28:30\]\[206:209\]]{sib}
...sst/mnmean.nc.asc?sst[12:13][28:30][206:209]
\end{vcode}

This produces something like\texorhtml{ the figure shown in
  figure~\ref{reynolds,sst}}{this:}

\figureplace{Part of the Reynolds SST data}{h}{reynolds,sst}{sst.ps}{sst.gif}{http://www.cdc.noaa.gov/cgi-bin/nph-nc/Datasets/reynolds_sst/sst.mnmean.nc.asc?sst[12:13][28:30][206:209]}

\indc{units!scaling}
Note that the sst values are in celsius degrees multiplied by 100, as
indicated by the \lit{\ind{scale_factor}} attribute of the \xlinkn{DAS}
{http://www.cdc.noaa.gov/cgi-bin/nph-nc/Datasets/reynolds_sst/sst.mnmean.nc.das}.
Further, it's important to remember with this dataset, that the data
were obtained by calculating spatial and temporal means.
Consequently, the data points in the \lit{sst} array should be ignored
when the corresponding entry in the \lit{mask} array indicates they
are over land.

%************THIS SECTION COMMENTED OUT BECAUSE SAMPLING GRIDS BY
%************VALUE DOESN'T CURRENTLY WORK.  WHEN IT'S FIXED, 
%************UNCOMMENT THIS SECTION.

%\subsection{Sampling Grids by Value}

%It is not always the easiest thing to figure out the array dimensions
%you need.  OPeNDAP servers provide a way to sample a Grid using the
%values of the dependent variables (the map vectors).  A function
%called \lit{grid} is used to examine a Grid and its map vectors, and
%translate relational clauses into array subscripts.

%The Reynolds data we've been looking at is a 226x180x360 element
%array.  What's more important now is that the \lit{time} dimension
%varies from 723,486 to 730,334 (days since January 1, 0001), the
%\lit{lat} dimension varies from 89.5 to -89.5, and the \lit{lon}
%dimension varies from 0.5 to 359.5.  To request a series of worldwide
%data arrays, covering the span of time between 730,000 and the
%present, you can formulate a constraint expression like 
%\xlinkn{this}
%{http://www.cdc.noaa.gov/cgi-bin/nph-nc/Datasets/reynolds_sst/sst.mnmean.nc.asc?grid(sst, "time>730000")}:

%\begin{vcode}{sib}
%...sst.mnmean.nc.asc?grid(sst, "time>730000")
%\end{vcode}

%There is no limit to the number of clauses that can be included in the
%parentheses (although all bets are off if you include clauses that
%conflict with one another).  So you can be more precise with a
%constraint expression like \xlinkn{this}
%{http://www.cdc.noaa.gov/cgi-bin/nph-nc/Datasets/reynolds_sst/sst.mnmean.nc.asc?grid(sst, "time > 730000", "lat>0", "lat<55", "lon<100")}:

%\begin{vcode}{sib}
%...sst.mnmean.nc.asc?grid(sst, "time>730000", "lat>0", "lat<55", "lon<100")
%\end{vcode}

%\note{As of \today\ this form of sampling is broken, and does not work.}


\section{Sequence Data}
\label{quick,sequences}

Gridded data works well for satellite images, model data, and data
compilations such as the Reynolds data we've just looked at.  Other
data, such as data measured at a specific site, is not so readily
stored in that form.  \opendap provides a data type called a \new{Sequence}
to store this kind of data.

\subj{A Sequence is a relational table.}
A Sequence can be thought of as a \ind{relational data table}, with each
column representing a different data value, and each row representing
a different data ``instance.''  For example, an ocean \ind{temperature
profile} can be stored as a Sequence of pressure and temperature pairs,
and a weather station's data can be stored as a Sequence with time in
one column, and each weather variable occupying another column.
\indc{data!relational table}\indc{RDBMS}\indc{profile!temperature}

Let's look at a couple of Sequences.  The first one is a collection of
\ind{CTD data} (\ind{hydrographic data}, including temperature, pressure,
salinity, and so on):\indc{data!hydrographic}\indc{station data}
\indc{data!station}\indc{data!time series}\indc{time series data}

\begin{vcode}{sib}
http://dods.gso.uri.edu/cgi-bin/nph-jg/rlctd
\end{vcode}

The \xlinkn{DAS}{http://dods.gso.uri.edu/cgi-bin/nph-jg/rlctd.das}
(append \lit{.das} to the URL) for this data is pretty uninformative,
telling us only that all the data are stored as strings\texorhtml{.
  You can see this in figure~\ref{rlctd,das}.}{:}

\figureplace{A DAS for Sequence data.}{h}{rlctd,das}{rlctd-das.ps}%
{rlctd-das.gif}{http://dods.gso.uri.edu/cgi-bin/nph-jg/rlctd.das}

\subj{You can sometimes find data attributes among the data.}
On the other hand, a lot of the information we would get from the DAS
is actually encoded in the data itself, which you can see by looking
at the data's
\xlinkn{DDS}{http://dods.gso.uri.edu/cgi-bin/nph-jg/rlctd.dds} (append
\lit{.dds} to the URL)\texorhtml{, shown in figure~\ref{rlctd,dds}.}{:}
\indc{metadata!among the data}\indc{data!metadata mixed in}

\figureplace{A DDS for Sequence data.}{h}{rlctd,dds}{rlctd-dds.ps}%
{rlctd-dds.gif}{http://dods.gso.uri.edu/cgi-bin/nph-jg/rlctd.dds}

We can get some idea of the data coverage by asking for some of the
time and location data, with a URL like this:

\begin{vcode}[http://dods.gso.uri.edu/cgi-bin/nph-jg/rlctd.asc?cruiseid,station,year_s,month_s,day_s,lat_s,lon_s]{sib}
...rlctd.asc?cruiseid,station,year_s,month_s,day_s,lat_s,lon_s
\end{vcode}

This produces a response shown \texorhtml{in
  figure~\ref{rlctd,coverage}.}{\xlinkn{here}{http://dods.gso.uri.edu/cgi-bin/nph-jg/rlctd.asc?cruiseid,station,year_s,month_s,day_s,lat_s,lon_s}.}

% This has changed slightly, and probably should be regenerated.
\figureplace{The \lit{rlctd} dates and locations}{h}{rlctd,coverage}
{rlctd-cov.ps}{rlctd-cov.gif}{http://dods.gso.uri.edu/cgi-bin/nph-jg/rlctd.asc?cruiseid,station,year_s,month_s,day_s,lat_s,lon_s}


\subj{Use a selection clause to select Sequence rows.}  
After reviewing the data in the last request, perhaps we decide we
only want to see data from one of the cruises listed, or maybe only
data from the month of May.  We can add a \new{selection clause} to
the constraint expression to select only that data.  For example:
\indc{constraint expression!selection clause}

\begin{vcode}[http://dods.gso.uri.edu/cgi-bin/nph-jg/rlctd.asc?cruiseid,station,year_s,month_s,day_s,lat_s,lon_s&month_s=5]{sib}
...rlctd.asc?cruiseid,station,year_s,month_s,day_s,lat_s,lon_s&month_s=5
\end{vcode}

This produces a table containing all the rows from
\texorhtml{figure~\ref{rlctd,coverage}}{the last example} where the
month datum is May.  \texorhtml{Try entering the new URL in your browser
  and see what you get.}{Click \xlinkn{here}%
{http://dods.gso.uri.edu/cgi-bin/nph-jg/rlctd.asc?cruiseid,station,year_s,month_s,day_s,lat_s,lon_s\&month_s=5}
to see that table.}

Selection clauses can be stacked endlessly against a URL, allowing all
the flexibility most people need to sample data files.  Here's an
example of a
\xlinkn{URL}{http://dods.gso.uri.edu/cgi-bin/nph-jg/rlctd.asc?o2\&month_s=5\&pres>50\&pres<100}
that requests all the oxygen data in the file taken in May at a
specific depth range:

\begin{vcode}[http://dods.gso.uri.edu/cgi-bin/nph-jg/rlctd.asc?o2&month_s=5&pres>50&pres<100]{sib}
...rlctd.asc?o2&month_s=5&pres>50&pres<100
\end{vcode}

The first clause in a constraint expression has a name, too.  It is
the \new{projection clause}.  This is the list of variables that you
wish to have returned, subject to the constraint of the selection
clause.  In the previous example, the projection clause consiste only
of the \lit{o2} variable.  In the one before that, the list was
longer, containing 7 variables.\indc{constraint expression!projection
  clause} 



\tbd{There is a get_row() method for Sequences now, so that you can
  select a sequence row by its ordinal number.  When this makes it
into the server releases, document it.}



\section{An Easier Way}

\subj{The \opendap query form is an easier way to sample data.}
\opendap also includes a way to sample data that makes writing a
constraint expression somewhat easier.  Append \lit{.html} to the URL,
and you get a form that directs you to add information to sample the
data at a \xlinkn{URL}
{http://www.cdc.noaa.gov/cgi-bin/nph-nc/Datasets/reynolds_sst/sst.mnmean.nc.html}:
\indc{constraint expression!query form}
\indc{constraint expression!building aids}
\indc{query form}\indc{form!query}
\indc{web interface}\indc{ifh}\indc{html interface}
\indc{Data Access Form}

\begin{vcode}[http://www.cdc.noaa.gov/cgi-bin/nph-nc/Datasets/reynolds_sst/sst.mnmean.nc.html]{sib}
...sst.mnmean.nc.html
\end{vcode}

Sending a URL ending in \lit{.html} returns a form like this:

\figureplace{The \opendap Dataset Access Form}{h}{reynolds,ifh}{ifh.ps}{ifh.gif}{http://www.cdc.noaa.gov/cgi-bin/nph-nc/Datasets/reynolds_sst/sst.mnmean.nc.html}

It's useful to have a browser window open with one of these query forms in it
while you read this section.  \texorhtml{}{Click
  \xlinkn{here}
  {http://www.cdc.noaa.gov/cgi-bin/nph-nc/Datasets/reynolds_sst/sst.mnmean.nc.html}
  to bring up a copy of the form to use while you read.}

Near the top of the page, you'll see a box entitled ``Data URL''.  At
this point, if you've been following along, it should look pretty
familiar.  If you're just jumping in, it's the \opendap URL connected to
the data we're interested in, but unsampled.

Moving down the page, there is a list of ``Global Attributes'', which
is really just for your perusal.  At this point, there's not much to
be done with this, but it is often helpful information.

\subj{Select variables by clicking on a checkbox.}  The important part
of the page is the ``Variables'' section.  For each variable in the
dataset, you'll see the data description (e.g. ``Array of 32 bit Reals
[lat = 0..179]''), a checkbox, a text input box, and a list of the
variable's attributes.  If you click on the checkbox, you'll see the
variable's array bounds appear in the text box, and you'll see that
variable appear in a constraint expression appended to the Data URL at
the top of the page.  If you edit the array bounds in the text box,
hitting ``enter'' will place your edits in the Data URL box.

In the oh-so-unlikely event you dare try all this without your
documentation \new{vade mecum} along, there's a \but{Show Help}
button up near the top of the page.  Clicking there will show you
instructions about how to proceed.

\note{You'll see a ``stride'' mentioned.  This is another way to
  subsample an \opendap array or Grid.  Asking for \lit{lat[0:4]} gets you the first
  five members of the \lit{lat} array.  Adding a stride value allows
  you to skip array values.  Asking for \lit{lat[0:2:10]} gets you
  every second array value between 0 and
  10: 0, 2, 4, 6, 8, 10.\indc{array!stride}\indc{stride}} 
\indc{Grid!sampling}\indc{sampling!Grid}\indc{array!subscripts}
\indc{array!sampling}\indc{sampling!array}\indc{Grid!stride}
\indc{Grid!subscripts}

Move on down the variable list, editing your request, and experiment
with adding and changing variable requests.

When you have a request you'd like to make, look at the buttons at the
top of the page.  

\figureplace{Dataset Access Form Detail}{h}{reynolds,ifh-buttons}
{ifh-buttons.ps}{ifh-buttons.gif}{}

You can click on \but{Get ASCII}, and the data
request will appear in a browser window, in comma-separated form.  The
\but{Get Binary} button will save a binary data file on your local
disk.  (The \but{Send to Program} will send the URL directly to an \opendap
client.  However, it requires a suitable \opendap client to be running on
your computer, and also requires you to install a helper application
for your browser.  There are instructions for doing this at the \opendap
home page.)

\subj{The web interface works for Sequence data, too.}
The \opendap Data Access Form interface works for Sequence data as well as
Grids.  However, since Sequence constraint expressions look different
than Grid expressions, the form looks slightly different, too.  You
can see \texorhtml{from figure~\ref{rlctd,ifh-seq}}{below} that the
variable selection boxes allow you to enter relational expressions for
each variable.  Beside that, however, the function is exactly the same.
\indc{constraint expression!query form}
\indc{constraint expression!building aids}
\indc{query form!Sequence data}\indc{form!query}
\indc{web interface!Sequence data}\indc{html
  interface!Sequence data}
\indc{Data Access Form!Sequence Data}

\figureplace{Dataset Access Form for Sequence Data (detail)}{h}
{rlctd,ifh-seq}{ifh-seq.ps}{ifh-seq.gif}{http://dods.gso.uri.edu/cgi-bin/nph-jg/rlctd.html}

\texorhtml{}{Click \xlinkn{here}
  {http://dods.gso.uri.edu/cgi-bin/nph-jg/rlctd.html} to see a copy of
  a Sequence form.}

\note{Not all \opendap servers support all the \opendap functionality.  There
  are a few non-standard \opendap servers out there in the world that only
  support the bare minimum required.  That minimum is to respond to
  queries for the DDS, DAS, and (binary) data.  The ASCII data and the
  web access form are optional add-ons that are not required for the
  basic \opendap function.}\indc{minimum server configuration}\indc{server!minimum configuration}


\chapter{Finding More \opendap URLs}

The \opendap package was developed to improve ways to share data among
scientists.  Many times, data comes in the form of a URL enclosed in
an email message.  But there are several other ways to find data served
by \opendap servers.

\section{GCMD}

\subj{The GCMD now catalogs \opendap URLs!}  The \xlink{Global Change
  Master Directory}{http://gcmd.gsfc.nasa.gov} is a source of a huge
amount of earth science data.  They now catalog \opendap URLs for the
datasets that have them.  You can search on ``\opendap'' right from the
main page to find many of these datasets.  Try that search, then click
on one of the data set names that returns, and look at the bottom of
the resulting Set Description'' page, under the heading ``Related
URL.''\indc{GCMD}\indc{Global Change Master Directory}
\indc{NASA!Global Change Master Directory}
\indc{URL!catalogued at GCMD}

If you make that search, check the list for the Reynolds data from
chapter~\ref{reynolds,chapter}; it should be there.

\section{\opendap Dataset List}

The \OPDhome\ has a list of available \opendap datasets.  Click on
\xlink{\but{Datasets}}
{http://www.unidata.ucar.edu/cgi-bin/dods/datasets/datasets.cgi?xmlfilename=datasets.xml}
\subj{The \opendap project supports an ad hoc list of data URLs.}
in the table of contents\texorhtml{}{ or right here}.  You can find a URL and a
brief description for several hundred different datasets from that
\xlinkn{list}{http://unidata.ucar.edu/packages/dods/home/data.shtml}.
\indc{dataset!list, \opendap}\indc{list!\opendap datasets}

\section{Web Interface}

This is a little bit sneaky.  Many sites that serve one \opendap dataset
serve several others as well.  The \opendap web interface (if it's enabled
by the site) allows you to check the directory structure for other
datasets.  For example, let's look at the \xlinkn{Reynolds data}
{http://www.cdc.noaa.gov/cgi-bin/nph-nc/Datasets/reynolds_sst/sst.mnmean.nc.html}
we saw in chapter~\ref{reynolds,chapter}:
\subj{The web interface allows browsing data directories.}

\begin{vcode}[.]{sib}
http://www.cdc.noaa.gov/cgi-bin/nph-nc/Datasets/reynolds_sst/sst.mnmean.nc.html
\end{vcode}

\indc{ifh}\indc{html interface!finding more data}
\indc{web interface!finding more data}
\indc{Data Access Form!using to find data}
If we use the same URL, but without the file at the end, we can browse
the directory of data:

\begin{vcode}[.]{sib}
http://www.cdc.noaa.gov/cgi-bin/nph-nc/Datasets/reynolds_sst/
\end{vcode}

The \opendap server checks to see whether the URL is a directory, and if
so, it generates a directory listing, like \texorhtml{in figure~\ref{reynolds,ifh-dir}.}{\xlinkn{this}{http://www.cdc.noaa.gov/cgi-bin/nph-nc/Datasets/reynolds_sst/}:}

\figureplace{Web Interface Index Listing}{h}{reynolds,ifh-dir}%
{ifh-dir.ps}{ifh-dir.gif}%
{http://www.cdc.noaa.gov/cgi-bin/nph-nc/Datasets/reynolds_sst/}

You can see from the directory listing that the monthly mean dataset
we've been looking at is accompanied by a weekly mean set, and a daily
set.  You can click on those datasets for more information about them,
and proceed to examine and use them just as we've done with the other
examples in chapter~\ref{reynolds,chapter}.

\note{This list is produced by an \opendap server.  It only really understands
\opendap data files.  If the directory you're looking at has other files
in it, clicking on them will probably produce an error.}

\section{File Servers}

Some datasets you'll find are actually lists of other datasets.  There
are a few of these \emph{file servers} in the \xlinkn{\opendap Dataset
  List}{http://unidata.ucar.edu/packages/dods/home/data.shtml} on the
\OPDhome .  A file server is itself an \opendap dataset, organized as a
Sequence, containing URLs with some other identifying data (often time).  You
can request the entire dataset, or subsample it just like any other
\opendap dataset.
\subj{A file server is a list of other datasets, but it's a dataset, too.}

There is a file server for GSO/URI's archive of AVHRR sea surface
temperature data:
\indc{query form!file server}\indc{form!query}
\indc{web interface!file server}\indc{file server}\indc{server!file}
\indc{Data Access Form!file server}
\indc{sampling!file server}\indc{Sequence!file server}
\indc{dataset!of datasets}

\begin{vcode}{sib}
http://maewest.gso.uri.edu/cgi-bin/nph-ff/catalog/avhrr.catalog
\end{vcode}

Look at this server's \xlinkn{DDS}{http://maewest.gso.uri.edu/cgi-bin/nph-ff/catalog/avhrr.catalog.dds}, and the \xlinkn{web interface}{http://maewest.gso.uri.edu/cgi-bin/nph-ff/catalog/avhrr.catalog.html}, and then try asking
for some data like \xlinkn{this}{http://maewest.gso.uri.edu/cgi-bin/nph-ff/catalog/avhrr.catalog.asc?DODS_URL\&year=2000\&month=1}:

\begin{vcode}[http://maewest.gso.uri.edu/cgi-bin/nph-ff/catalog/avhrr.catalog.asc?DODS_URL&year=2000&month=1]{sib}
  .../catalog/avhrr.catalog.asc?DODS_URL&year=2000&month=1
\end{vcode}

This produces a list of all the data URLs corresponding to
measurements taken in the month of January, 2000.

\section{Matlab GUI}

The \opendap Matlab GUI browser contains its own frequently updated list of
available datasets.  Using that software, you can select datasets with
\subj{The Matlab GUI has its own list of available data.}
a mouse from a large selection of the available URLs.  For more
information, please refer to the \OPDmgui\ manual.

\chapter{Further analysis}

This guide is about forming an \opendap URL.  After you have figured out
how to request the data, there are a variety of things you can do with
it.  (\opendap software mentioned here is available from the \OPDhome .)

\tbd{Add links to the following list.}

\begin{itemize}
\item Use a generic web client like \lit{geturl} (a standard part of
  the \opendap package), the free programs
  \xlinkn{\lit{wget}}{http://www.gnu.org/manual/wget-1.5.3/html_mono/wget.html}
  or \xlinkn{\lit{lynx}}{http://lynx.browser.org}, or even a browser
  like \lit{Netscape Navigator} or \lit{Internet Explorer} to download
  data into a local data file.  To be able to use the data further,
  you will probably have to download the ASCII version by using the
  \indc{geturl!\opendap utility web client}\indc{wget!web client}
  \indc{lynx!web browser}\indc{Netscape}\indc{Internet Explorer}
  \lit{.asc} suffix on the URL, as in the examples shown.
\subj{Use a generic web client or an \opendap client to get the data you've chosen.}
\item There are pre-packaged \opendap clients available that can download
  binary \opendap data from the web into a useful form.  As of \today ,
  command line clients (\lit{loaddods}) are available for the Matlab
  and IDL data analysis environments, with which you can download \opendap
  data directly into IDL or Matlab objects.  \indc{loaddods!Matlab or
    IDL client}\indc{Matlab!loaddods client}\indc{IDL!loaddods client}
\item The \xlink{\ind{Ferret}}{http://ferret.wrc.noaa.gov/Ferret} and \ind{GrADS}
  free data analysis packages both support \opendap.  You can use these
  for downloading \opendap data, and for examining it afterwards.  (There
  are limitations.  As of \today , Ferret can not read datasets served
  as Sequence data.)
\item The Matlab analysis package also supports an \opendap client attached
  to a graphical user interface.  You can use the GUI to create a
  constrained \opendap URL, and download the data directly into Matlab.
  The \OPDmgui\ contains more information about the Matlab GUI
  client.\indc{Matlab!GUI \opendap browser}
\item If you have a data analysis program or package that you like,
  you can look into the possibility of linking that package to the
  \opendap toolkit library, in effect making your program into a
  web-capable \opendap client. \indc{DODS!linking to your
    software}\indc{linking!\opendap to your programs} \ind{DODS!libraries}
  \indc{libraries!\opendap} exist to mimic the behavior of the
  \netcdf and \jgofs\ data access APIs.  If your program already uses
  one of these APIs, getting it to run with \opendap may be as simple as
  changing the libraries to which you link it.  The \OPDuser\ 
  describes how to do this, and the \OPDapi\ describes how you can
  use the \opendap toolkit directly to create a new application that
  doesn't use one of the established data access
  APIs.\indc{toolkit!\opendap}\indc{DODS!toolkit}
\indc{JGOFS!data access API}\indc{data access API!JGOFS}
\indc{netCDF!data access API}\indc{data access API!netCDF}

\end{itemize}

The use of these clients, like the ways in which you can analyze the
data you find, is beyond the scope of this (or any) book.  Enjoy.

\printindex

\end{document}



% $Log: quick.tex,v $
% Revision 1.6  2004/11/09 14:41:10  tomfool
% converted DODS->OPeNDAP
%
% Revision 1.5  2004/07/07 03:35:30  jimg
% Added rcs Info Date.
%
% Revision 1.4  2001/12/05 15:15:11  tom
% changed DODS home page dataset URL
%
% Revision 1.3  2001/06/29 15:54:15  tom
% added PDF file generation
%
% Revision 1.2  2000/10/30 20:11:46  tom
% indexed, incorporated first round of comments.
%
% Revision 1.1  2000/10/12 17:24:36  tom
% added quick.tex to repository
%
