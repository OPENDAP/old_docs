% Data analysis with OPeNDAP
%
% $Id$
%

\xname{constraint}
\chapter{Data Analysis with \opendap}

The \opendap software is not only a data transport mechanism.  Using \opendap,
you can subsample the data you are looking at.  That is, you can
request an entire data file, or just a small piece of it.

\section{Selecting Data: Using Constraint Expressions}
\label{opd-client,constraint}

The URL such as this one:

\begin{vcode}{ib}
http://dods.gso.uri.edu/cgi-bin/nph-nc/data/buoys.nc
\end{vcode}

refers to the entire
dataset contained in the buoys.nc file. A user may, however, choose
to sample the dataset simply by modifying the submitted URL. The
\new{constraint expression} attached to the URL directs that the data
set specified by the first part of the URL be sampled to select only
the data of interest from a dataset even for programs that do not
have a built-in way to accomplish such selections.  This can vastly
reduce the amount of data a program needs to process, and reduce the
network load of transmitting that data to the client.

\subsection{Constraint Expression Syntax}
\label{opd-client,constraint-syntax}

A constraint expression is appended to the target URL following a
question mark, as in the following examples:
\indc{constraint expression!syntax}\indc{syntax!constraint expression}

\begin{vcode}{ib}
 http://oceans.univ.edu/cgi-bin/nc/expl/buoys.nc?temp

 http://oceans.univ.edu/cgi-bin/nc/expl/buoys.nc?temp[1,100,5]

 http://oceans.univ.edu/cgi-bin/nc/expl/buoys.nc?u&lat>15.0

 http://oceans.univ.edu/cgi-bin/nc/expl/buoys.nc?cast.02<15.0
 
 http://oceans.univ.edu/cgi-bin/nc/expl/buoys.nc
                                   ?station&station.temp<15.0
\end{vcode}


A constraint expression consists of two parts: a \new{projection}
\indc{constraint expression!projection} and a \new{selection},
\indc{constraint expression!selection} separated by an ampersand
(\lit{\&}).  Either part may contain several sub-expressions.  Either
part may be present, or both.

\begin{center}
  $proj_{1},proj_{2},\ldots,proj_{n}\&sel_{1}\&sel_{2}\&\ldots\&sel_{m}$
\end{center}

A \ind{projection} is simply a comma-separated list of the variables
that are to be returned to the client. If an array is to be
subsampled, the projection specifies the manner in which the sampling
is to be done.  If the selection is omitted, all the variables in the
projection list are returned. If the projection is omitted, the entire
dataset is returned, subject to the evaluation of the selection
expression.  The projection can also include functional expressions of
the form:

\begin{center}
  $function(arg_{1},arg_{2},\ldots,arg_{n})$
\end{center}

\noindent
where the arguments are variables from the dataset, scalar values, or
other functions.

A simple selection expression is a boolean expression of the form 

\begin{center}
{\em variable operator variable}\\
or\\ 
{\em variable operator value}\\
or\\
{\em function($arg_{1},arg_{2},\ldots,arg_{n}$)}\\
\end{center}

Where 

\begin{description}
\item[\var{operator}] can be one of the relational operators listed in
  \tableref{opd-client,tab,cons-ops} on
  \pagexref{opd-client,tab,cons-ops};
\item[\var{variable}] can be any variable recorded in the dataset;
\item[\var{value}] can be any scalar, string, function, or list of
  numbers (Lists are denoted by comma-separated items enclosed in
  curly braces ,for example, \{3,11,4.5\}.); and
\item[\var{function}] is a function defined by the server to operate
  on variables or values, and to return a boolean value (See
  \sectionref{opd-client,function}). 
\end{description}

Each selection clause begins with an ampersand (\lit{\&}) representing
the ``AND'' boolean operation\footnote{The ``OR'' function may be
  implemented with a list.  For example, to say that {\em i} must
  equal 3 OR 11 you would write {\em i} = \{3,11\}The clause evaluates
  to true when \var{i} equals any one of the elements.}.

\note{The \lit{\&} is actually a \emph{prefix} operator, not an infix
  operator.  That is, it must appear at the beginning of each
  selection clause, no matter what.  This means that a constraint
  expression that contains no projection clause must still have an
  \lit{\&} in front of the first selection clause.}

There is no limit on the number of selection clauses that can be
combined to create a compound constraint expression.  Data that
produces a true (non-zero) value for the entire selection expression
will be included in the data returned to the client by the server. If
only a part of some data structure, such as a \class{Sequence},
satisfies the selection criteria, then only that part will be
returned.

\note{Due to the differences in data model paradigms, selection is not
  implemented for the \opendap array data types, such as \class{Grid} or
  \class{Array}.  However, many \opendap servers implement selection
  functions you can use for the same effect.  You can query the server
  for the functions it implements with the usage service outlined in
  \sectionref{opd-client,function}.}

\subsubsection{Simple Constraint Expression Examples}

\indc{constraint expression!examples}
Consider the data descriptor in \figureref{opd-client,fig,dds}.  The
figure is an example of the Data Descriptor Structure \indc{Data Descriptor
  Structure!example} \indc{example!DDS}, one of the messages returned by an \opendap server in response to a query about some dataset. The full syntax
description for this structure is given in \sectionref{data,ancillary}. For
the moment, it is only important that it is the description of a dataset
containing station data including temperature, oxygen, and salinity. Each
station also contains 20 oxygen data points, taken at 20 fixed depths, used
for calibration of the data.

The following URL will return only the pressure and temperature pairs
of this dataset. (Note that the constraint expression parser removes
all spaces, tabs, and newline characters before the expression is
parsed.) There is only a projection clause, without a selection, in
this constraint expression\footnote{For the sake of clarity, this and
  several of the following constraint expression examples span
  multiple lines.  While the constraint expression evaluator ignores
  newline characters, program limitations of the \opendap client will
  likely prevent a user from typing a newline in a constraint
  expression.}.

\begin{figure}[htbf]
\begin{vcode}{ib}
Dataset {
   Sequence{
      Int32 day;
      Int32 month;
      Int32 year;
      Float64 lat;
      Float64 lon;
      Float64 O2cal[20];
      Sequence{
         Float64 press;
         Float64 temp;
         Float64 O2;
         Float64 salt;
      } cast;
      String comments;
   } station;
} arabian-sea;
\end{vcode}
\label{opd-client,fig,dds}
\caption{Sample Data Descriptor}
\end{figure}

\begin{vcode}{ib}

http://oceans.edu/cgi/nph-jg/exp1O2/cruise?station.cast.press,
                                           station.cast.temp
\end{vcode}


Incidentally, we have assumed that the dataset was stored in the
JGOFS format\footnote{Because it contains an array, the dataset
  pictured in Figure~\ref{opd-client,fig,dds} is technically not a
  valid JGOFS dataset. We have included the array for pedagogical
  purposes, and hope that the JGOFS purists will forgive us.}  on the
remote host \lit{oceans.edu}, in a file called \lit{explO2/cruise}.
For the sake of brevity, from here on we will omit the first part of
the URL, to concentrate on the constraint expression alone.

If we only want to see pressure and temperature pairs below 500 meters
deep, we can modify the constraint expression by adding a selection
clause.

\begin{vcode}{ib}
?station.cast.press,station.cast.temp&station.cast.press>500.0
\end{vcode}

In order to retrieve all of each cast that has any temperature reading
greater than 22 degrees, use the following:

\begin{vcode}{ib}
?station.cast&station.cast.temp>22.0
\end{vcode}

Simple constraint expressions may be combined into compound
expressions with logical AND (\lit{\ind{\&}}). To retrieve all
stations west of 60 degrees West and north of the equator:
\indc{constraint expression!combining clauses} \indc{constraint
  expression!boolean functions} \indc{boolean functions}

\begin{vcode}{ib}
?station&station.lat>0.0&station.lon<-60.0
\end{vcode}

As was mentioned, the logical \ind{OR} can be implemented using a list
of scalars. The following expression will select only stations taken
north of the equator in April, May, June, or July.

\begin{vcode}{ib}
?station&station.lat>0.0&station.month={4,5,6,7}
\end{vcode}

If our dataset contained a field called \lit{monsoon-month},
indicating the month in which monsoons happened that year, we could
modify the last example search to include those months as follows:

\begin{vcode}{ib}
?station&station.lat>O.O
        &station.month={4,5,6,7,station.monsoon-month}
\end{vcode}

In other words, a list can contain both values and other variables. If
monsoon-month was itself a list of months, a search could be written
as:

\begin{vcode}{ib}
?station&station.lat>0.0&station.month=station.monsoon-month
\end{vcode}

\indc{selection!arrays} \indc{constraint expression!arrays}
\indc{array!constraint expression} \indc{array!selection} For arrays
and grids, there is a special way to select data within the projection
clause.  Suppose we want to see only the first five oxygen calibration
points for each station. The constraint expression for this would be:
\label{opd-client,array-op}

\begin{vcode}{ib}
?station.02cal[0:4]
\end{vcode}
\label{opd-client,CE,array-sel}

By specifying a \new{stride} value, we can also select a
\new{hyperslab} of the oxygen calibration array:
\indc{selection!hyperslab} \indc{constraint expression!hyperslab}
\indc{selection!stride value} \indc{constraint expression!stride value}

\begin{vcode}{ib}
?station.02cal[0:5:19]
\end{vcode}

This expression will return every fifth member of the \lit{02cal}
array. In other words, the result will be a four-element array
containing only the first, sixth, eleventh, and sixteenth members of
the \lit{02cal} array. Each dimension of a multi-dimensional arrays
may be subsampled in an analogous way. The return value is an array of
the same number of dimensions as the sampled array, with each
dimension size equal to the number of elements selected from it.

\subsection{Operators, Special Functions, and Data Types}

The data types accessible through the \opendap software are listed and
described in \sectionref{data,types}. It is advisable to be familiar
with these types before trying to construct complex constraint
expressions.

The constraint expression syntax defines a number of operators for each
data type. These operators are listed in \tableref{opd-client,tab,cons-ops}

\indc{\lit{*}} Except for the $*$ operation defined on the URL data
type, all the operators defined for the scalar base types are boolean
operators whose result depends on the specified comparison between its
arguments. Refer to \sectionref{opd-client,CE,url} for a description
of the URL data type and its operator.

The \math[\~{}=]{\sim =} operator returns true when the character string
on the left of the operator matches the regular expression on the
right. See \sectionref{opd-client,CE,regex} for a discussion of
regular expressions.

The \class{Structure}, \class{Sequence}, and \class{Grid} data types
are each composed of a collection of simpler data types. The \ind{.}
and operators allow a user to refer to the subsidiary variables within
these compound types. For example, \lit{station.year} indicates the
value of the \lit{year} member of the \lit{station} sequence.

The array operator \ind{$[]$} is used to subsample the given array.
See \pagexref{opd-client,array-op} for an explanation and example of
its use.

\begin{table}[htbp]
\caption{Constraint Expression Operators\@.}
\label{opd-client,tab,cons-ops}
\begin{center}
\begin{tabular}{|p{0.75in}|p{2in}|} \hline
\tblhd{Class} & \tblhd{Operators} \\ 
\hline \hline
\multicolumn{2}{|c|}{\em Simple Types\/} \\  \hline
\class{Byte}, \class{Int32}, \class{UInt32}, \class{Float64} & \lit{< > = != <= >=} \\  \hline
\class{String} & \lit{= != } \math[\lit{\~{}=}]{\sim =} \\  \hline
\class{URL} & \lit{*} \\  \hline
\multicolumn{2}{|c|}{\em Compound Types\/} \\  \hline
\class{Array} & \lit{[start:stop] [start:stride:stop]} \\  \hline
\class{List} & \lit{length({\em list}), nth({\em list,n}), member({\em list,elem})} \\  \hline
\class{Structure} & \lit{.} \\  \hline
\class{Sequence} & \lit{.} \\  \hline
\class{Grid} & \lit{[start:stop] [start:stride:stop] .} \\  \hline
\end{tabular}
\end{center}
\end{table}

There are three special functions defined to operate on the
\class{List} data type.  The \lit{length()} function returns the
number of elements in the given list, the {\tt{\ind{nth()}}} function
returns the list element indicated by the input index, and the
{\tt{\ind{member()}}} function, which returns true if the given value
equals any member of the list. Note that the behavior of the
\lit{nth()} function is undefined for indices beyond the range of the
list.

\subsection{Using Functions in a Constraint Expression}
\label{opd-client,function}

\indc{functions} \indc{constraint expressions!functions} \indc{\opendap
  server!functions} An \opendap data server may define its own set of
functions that may be used in a constraint expression. For example,
the data server containing the example data from
\figureref{opd-client,fig,dds} might define a \lit{sigma1()} function
to return the density of the water at the given temperature, salinity
and pressure. A query like the following would return all the stations
containing water samples whose density exceeded 1.0275{\em{$g/cm^3$}}.

\begin{vcode}{ib}
?station.cast&sigma1(station.cast.temp,
                     station.cast.salt,
                     station.cast.press)>27.5
\end{vcode}

\indc{usage service} \indc{\opendap server!usage}
\indc{function!usage} \indc{constraint expression!function} 
\indc{Service!Usage} 
\indc{info service@\lit{info} service} \indc{\opendap server!\lit{info}}
\indc{function!\lit{info} service}
Functions like this one are not a standard part of the \opendap
architecture, and may vary from one server to another.  A user may
query a server for a list of such functions by sending a URL ending with
``.info''. For example, you can query the data server installed on the
\opendap home site with the following URL:

\begin{vcode}{ib}
  http://dods.gso.uri.edu/cgi-bin/nph-nc/fnoc1.nc.info
\end{vcode}
\label{opd-client,usage}

The data returned will be an HTML message, readable with a standard
web browser, containing documentation of the server running on the
given site, and the data named in the URL.  In this case, you will
learn that the specified server defines two functions that can be used
in a constraint expression: 

\begin{description}
\item[geolocate(\var{variable}, \var{lat1}, \var{lat2}, \var{lon1},
  \var{lon2})]
 Returns the elements of \var{variable} that fall
  within the box created by (\var{lat1},\var{lon1}) and
  (\var{lat2},\var{lon2}).
\item[time(\var{variable}, \var{start\_time}, \var{stop\_time})]
  Returns the elements of \var{variable} that fall within the time
  interval \var{start\_time} and \var{stop\_time}.
\end{description}

\subsection{Using URLs in a Constraint Expression}
\label{opd-client,CE,url}

\indc{URL!in constraint expression} \indc{constraint expression!URL}
The \opendap data access protocol defines a special data type to handle
distributed data: \class{URL}. This is a scalar data type, much like
the \class{String} type, intended to hold one \opendap URL.  It generally
points at some remote dataset or data value. Using this data type, a
constraint expression may make the data returned from one \opendap data
server dependent on data held at an entirely different site.

In order to accommodate this data type, \opendap defines a special
``dereference'' \indc{dereference} operator \ind{\lit{*}}. Similar to
its function with pointers in C, applying this operator to a URL
returns the data specified by that URL. The \class{URL} data type
itself contains only a character string. It must be dereferenced to
produce a reference to the data named by the URL.

\subsubsection{Examples}

The following example will return all the stations containing oxygen
values greater than fifteen:

\begin{vcode}{ib}
?station&station.cast.O2>15.0
\end{vcode}

Similarly, the following constraint expression will yield all the
stations in the dataset whose value is greater than that of the
oxygen value indicated by the URL:

\begin{vcode}{ib}
?station&station.cast.O2>*''http://ocean.edu/etc/nc/data?O2MAX''
\end{vcode}

Finally, suppose that the dataset itself contained a variable of type
\class{URL}, and that this URL contained the address of oxygen data
stored at some other site. The data descriptor for the dataset might
look like the following:


\begin{vcode}{ib}
Dataset {
   Sequence{
       .
       .
       .
      URL O2cal;
       .
       .
       .
   } station;
} arabian-sea;
\end{vcode}


We can now write the previous constraint as:

\begin{vcode}{ib}
?station&station.cast.O2>*O2cal
\end{vcode}

URLs stored in remote datasets may also be used in the projection
clause of the constraint expression. Imagine a dataset that consists
only of a list of URLs for each square degree of latitude and
longitude. A user could query this dataset for the actual list of
URLs, or, by using the \lit{*} operator, could construct a constraint
expression that would return the actual data indicated by the URLs in
the target dataset.

\subsection{Pattern Matching with Constraint Expressions}
\label{opd-client,CE,regex}

There are three operators defined to compare one \class{String} data
type to another. The \lit{=} operator returns TRUE if its two input
character strings are identical, and the \lit{!=} operator returns
TRUE if the \class{Strings} do not match. A third operator,
\math[\~{}=]{\sim =} is provided that returns TRUE if the \class{String}
to the left of the operator matches the \ind{regular expression} in
the \class{String} on the right.

\indc{matching character strings} \indc{character strings!matching}
\indc{regex, see regular expression} A regular expression is simply a
character string containing wildcard characters that allow it to match
patterns within a longer string. For example, the following constraint
expression might return all the stations on the sample cruise at which
a shark was sighted:

\begin{vcode}{ib}
?station&station.comment~=``.*shark.*''
\end{vcode}

\indc{shark!sighting} \indc{sighting!shark} Most characters in a
regular expression match themselves. That is, an ``f'' in a regular
expression matches an "f" in the target string. There are several
special characters, however, that provide more sophisticated
pattern-matching capabilities.  \indc{regular expression!syntax}
\indc{syntax!regular expression}

\begin{description}

\item{\lit{.}}

The period matches any single character except a newline.  

\item{\lit{*} \lit{+} \lit{?}}
  
  These are postfix operators, which indicate to try to match the
  preceding regular expression repetitively (as many times as
  possible). Thus, \lit{o*} matches any number of \lit{o}'s. The
  operators differ in that \lit{o*} also matches zero \lit{o}'s,
  \lit{o+} matches only a series of one or more \lit{o}'s, and
  \lit{o?}  matches only zero or one \lit{o}.

\item{`[ ... ]'}
  
  Define a ``character set,'' which begins with \lit{[} and is
  terminated by \lit{]}.  In the simplest case, the characters between
  the two brackets are what this set can match. The expression
  \lit{[Ss]} matches either an upper or lower case \lit{s}. Brackets
  can also contain character ranges, so \lit{[0-9]} matches all the
  numerals. If the first character within the brackets is a caret
  (\lit{\^{ }}), the expression will only match characters that do not
  appear in the brackets. For example, \lit{[\^{ }0-9]*} only matches
  character strings that contain no numerals.

\item{\lit{\^{ }} \lit{\$}}
  
  These are special characters that match the empty string at the
  beginning or end of a line.

\item{\lit{$\backslash|$}}
  
  These two characters define a logical OR between the largest
  possible expression on either side of the operator.  So, for
  example, the string \lit{Endeavor$\backslash|$Oceanus} matches
  either \lit{Endeavor} or \lit{Oceanus}. The scope of the OR can be
  contained with the grouping operators, \lit{$\backslash$(} and
  \lit{$\backslash$)}.


\item{\lit{$\backslash$(} \lit{$\backslash$)}}
  
  These are used to group a series of characters into an expression,
  or for the OR function. So, for example,
  \lit{$\backslash$(abc$\backslash$)*} matches zero or more
  repetitions of the string \lit{abc2}.

\end{description}

\indc{special character|see{regular expression}}
\indc{character!special|see{regular expression}}
There are several more special characters and several other features
of the characters described here, but they are beyond the scope of
this guide. The \opendap regular expression syntax is the same as that
used in the \ind{Emacs} editor. See the documentation for
Emacs~\citel{emacs} for a complete description of all the pattern-
matching capabilities of regular expressions.

\subsubsection{Examples}

In the above example, a user might wonder whether the shark comments
had been spelled with upper or lower case letters. The following
constraint expression will return any station that mentions a shark in
upper or lower case.

\begin{vcode}{ib}
?station&station.comment~=``.*\(SHARK\|shark\).*''
\end{vcode}

Of course, this would miss \lit{Shark} and \lit{sHark} and so on. The
constraint could be written this way to catch all odd permutations of
upper and lower case:

\begin{vcode}{ib}
?station&station.comment~=``.*[Ss][Hh][Aa][Rr][Kk].*''
\end{vcode}

\subsection{Optimizing the Query}

\indc{optimization} \indc{query optimization} 
\indc{constraint expression!optimization}
\indc{search optimization} \indc{data search!optimization}
Using the tools provided by \opendap, a user can build quite elaborate and
sophisticated constraint expressions that will return precisely the
data he or she wishes to examine. However, as the complexity of the
constraint expression increases, so does the time necessary to process
that expression. There are some techniques a user may user to optimize
the evaluation of a constraint that will ease the load on the server,
and provide faster replies to \opendap dataset queries.

The \opendap constraint expression evaluator uses a "\ind{lazy
  evaluation}" algorithm. \indc{evaluation!lazy} This means that the
sub-clauses of the selection clause are evaluated in order, and
parsing halts when any sub-clause returns FALSE. Consider a constraint
expression that looks like this: \indc{constraint expression!parse
  order} \indc{parse order!constraint expression} \indc{precedence}

\begin{vcode}{ib}
?station&station.cast.O2>15.0&station.cast.temp>22.0
\end{vcode}

If the server encounters a station with no oxygen values over 15.0, it
does not bother to look at the temperature records at all. The first sub-
clause evaluates FALSE, so the second clause is never even parsed.

A careful user may use this feature to his or her advantage. In the
above example, the order of the clauses does not really matter; there
are the same number of temperature and oxygen measurements at each
station.  However, consider the following expression:


\begin{vcode}{ib}
?station&station.cast.O2>15.0&station.month={3,4,5}
\end{vcode}

For each station there is only one month value, while there are many
oxygen values. Passing a constraint expression like this one will
force the server to sort through all the oxygen data for each station
(which could be in the thousands of points), only to throw the data
away when it finds that the month requested does not match the month
value stored in the station data. This would be far better done with
the clauses reversed:

\begin{vcode}{ib}
?station&station.month={3,4,5}&station.cast.O2>15.0
\end{vcode}

This expression will evaluate much more quickly because unwanted
stations may be quickly discarded by the first sub-clause of the
selection. The server will only examine each oxygen value in the
station if it already knows that the station might be worth keeping.
\indc{URL!optimization}

This sort of optimization becomes even more important when one of the
clauses contains a URL. In general, any selection sub-clause
containing a URL should be left to the end of the selection. This way,
the \opendap server will only be forced to go to the network for data if
absolutely necessary to evaluate the constraint expression.  \tbd{Are
  there other optimization issues besides order?}

\section{A Word About Data Translation}
\label{opd-client,trans}

\indc{translation!introduction} \indc{data
  translation|see{translation}} Once a researcher is freed from the
confines of using only local data, he or she will soon discover that
there is a wealth of data available on the Internet, and nearly all of
it is stored in formats incompatible with her own. Worse, the data
formats are often mutually incompatible, rendering the confusion
complete.  \opendap provides a solution applicable to a great many such
problems.

\indc{data format!incompatible} \indc{format!data} When an \opendap server
retrieves data from some distant machine, that data may be in any of
several file formats supported by \opendap. The server translates the
data, however, into an intermediate format for transmission. Upon
receipt of the messages containing data, the \opendap client software
unpacks the data into the form expected by the calling client program
and returns it to that program.  Because all data must be translated
into the same intermediate format, \opendap becomes a powerful format
translator for datasets. In effect, this means that a program
designed to read and display JGOFS data can look at the \opendap data
catalog and see everything as JGOFS datasets. A netCDF program can
look at those same datasets, from that same catalog, and think they
are all in netCDF format. This system of translation allows a
researcher to ignore the question of formats and concentrate on the
data alone.

Of course, there are some translations that cannot be done
transparently, if they can be done at all. Consider a two-dimensional
array of satellite sea-surface temperature measurements. Assume the
data is stored in netCDF format on some machine called
\lit{satt.uri.edu}. The data might be uniquely specified by some URL,
say \lit{http://satt.uri.edu/sst/010694.nc}. However, were a user to
feed that URL to a JGOFS-originated \opendap client designed to draw
property vs. depth graphs of station data, no translation facility
would be able to map the original data into a form accommodated by the
client program.

The issues of data models and data translation are important ones to
the data provider. These issues are discussed in detail in
\sectionref{data,trans}


% $Log: ch02b.tex,v $
% Revision 1.3  2003/09/04 19:42:06  tom
% DODS->OPeNDAP
%
% Revision 1.2  2000/10/04 15:02:14  tom
% changed \figureplace definition, misc other cleaning
%
% Revision 1.1  1999/10/18 21:39:44  tom
% updated for new services, fixed server test section, add ifh intro.
%
%


%%% Local Variables: 
%%% mode: latex
%%% TeX-master: t
%%% End: 
