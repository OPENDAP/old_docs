%
%  $Id$
%

\chapter{Software you will need for DODS}
\label{req-software}


To do anything with DODS, you'll need to be able to unpack the archive
files you can download from the \opendap site.  To save space and
transmission time, the archive files are compressed with the
\lit{gzip} program.  You will have to have a copy of that program to
unpack the \opendap software.

Most of the software you need for \opendap is avaliable from the GNU
archives.  Refer to \xlink{http://www.gnu.org}{http://www.gnu.org} for
instructions.  Look at
\xlink{http://www.gnu.org/order/ftp.html}{http://www.gnu.org/order/ftp.html}
for a list of mirrors of that archive.  Use the mirror closest
to you, the transmission will be faster.

\begin{description}
\item[gzip] This is the GNU compression and de-compression program.
  You will need to install it before you can unpack any of the other
  software described here.  This package is \emph{not}
  available in the \opendap distribution, since it is used to unpack the
  distribution archive files.
\end{description}

Follow the instructions to install each of the following software
packages.  Typically, you would install a package called \lit{foo} as
follows: 

\begin{vcode}{ib}
gzip -dc foo.tar.gz | tar xvf -
cd foo
./configure
make
make install
\end{vcode}

This is simply a guide, of course, and the installation instructions
for each software package should be followed carefully.


\section{Running an \opendap Server}

If you use one of the platforms for which \opendap supplies a binary
distribution, you only need the following software to run an \opendap
server. 

\begin{description}
\item[Perl] Perl is used for the server dispatch script.  (See
  \sectionref{opd-server,arch}.) This is the main CGI program
  constituting the \opendap server.  You must have Perl version 5 or
  later.  (Alternatively, you can also rewrite the dispatch script to
  use another scripting language, such as your shell.  However, we
  think installing Perl is generally a simpler task.)  You can get
  Perl from the GNU archives, or from
  \xlink{http://www.perl.com}{http://www.perl.com}.
\end{description}

\section{Running an \opendap Client}

If you use one of the pre-compiled, out-of-the-box, \opendap clients, you
will need no additional software to run \opendap.  However, you can use
the ``GUI'' feature of the \opendap client\footnote{This is not to be
  confused with the \opendap Matlab or IDL GUIs, which are clients of
  their own.  This is simply a client feature that can display
  transmission and error information to the user.} by installing the
following software.  We recommend this, as it provides useful
information about the progress of data transmission or error conditions.

\begin{description}
\item[Tcl/Tk] The Tcl language and Tk libraries are available from
  \xlink{http://www.scriptics.com}{http://www.scriptics.com}. You
  should install the entire package, including the \lit{wish}
  interpreter program\footnote{You can also use a safe Tcl
    interpreter.  Refer to the Tcl documentation for information.} and
  the \lit{expect} package.  The \lit{wish} interpreter is
  part of the Tcl/Tk core distribution package.  This package is also
  available in the \opendap distribution, but the one available from the
  Tcl site may be more current.
\end{description}

\section{Building DODS}

If you need to build the \opendap software, or link it to existing
libraries, you will need the following GNU software.
Refer to \xlink{http://www.gnu.org}{http://www.gnu.org} for
instructions.  Look at
\xlink{http://www.gnu.org/order/ftp.html}{http://www.gnu.org/order/ftp.html}
for a list of mirrors of that archive.  Use the mirror closest
to you, the transmission will be faster.

\begin{description}
\item[GNU \Cpp\ Compiler] \opendap needs \lit{g++}, the GNU \Cpp\ compiler
  to compile.
\item[binutils] The GNU linker is part of this package.
\item[libstdc++] The standard \Cpp\ library.\indc{libstdc++}\indc{stdc++}
\item[GNU Make] GNU Make is not essential, but will make like easier.
\item[flex] The GNU lexical-analyzer generator
\item[bison] The GNU parser generator.
\end{description}

%
%  $Log: appb.tex,v $
%  Revision 1.3  2004/08/24 22:51:32  jimg
%  Fixed broken label/refs.
%
%  Revision 1.2  2003/09/04 19:42:06  tom
%  DODS->OPeNDAP
%
%

%%% Local Variables: 
%%% mode: latex
%%% TeX-master: t
%%% End: 
