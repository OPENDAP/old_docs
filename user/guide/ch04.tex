% Chapter two to the OPeNDAP User Guide
%
% $Id$
%


\chapter{The \opendap Server}
\label{opd-server}

See the \opendap Server installation guide for most of this information.


\tbd{Erase the rest of this chapter to avoid redundancy.}


There are two separate pieces to the task of installing an \opendap data
server: installing and configuring the server itself, and telling the
universe of possible users about it.\footnote{The second step is, of
  course, optional.} Only the first will be considered in this
chapter.  \opendap provides avenues for doing the second, including a
Catalog Service indexing \opendap datasets, and cooperation with the
Global Change Master Directory, but these are still under
construction.

An \opendap server is nothing more than a World Wide Web server
(\lit{httpd}) equipped with Common Gateway Interface (CGI) programs
that enable it to respond to requests for data from \opendap client
programs.  Web servers and CGI programs are standard parts of the Web,
and the details of their operation and installation are beyond the
scope of this guide. For further information about these, consult one
of the many World Wide Web references now available. For the purposes
of understanding the \opendap architecture, a user need only understand
the following:

\begin{itemize}

\item A Web server is a process that runs on a computer (the host
machine) connected to the Internet. When it receives a URL from some
Web client, such as a user somewhere operating Netscape or Mosaic, it
packages and returns the data specified by the URL to that client. The
data can be text, as in a web page, but it may also be images, sounds,
a program to be executed on the client machine, or some other data.

\item A properly specified URL can cause a Web server to invoke a 
CGI program on its host machine, accepting the input that would have
gone to the httpd, and  returning the output of that program
to the client who sent the URL in the first place. The CGI is executed
on the server. The \opendap server relies on this facility. 

\end{itemize}

\section{Server Architecture}
\label{opd-server,arch}

\indc{architecture!server} \indc{server!architecture} A request for
data made to the client \opendap library will result in three different
requests for data to an \opendap server. The user simply enters a single
URL, as described in \sectionref{opd-client,url}. The core \opendap
software then modifies the URL into three slightly different forms,
and makes three requests for data to the server. The first request is
for the ``shape'' of the data, and consists of the dataset descriptor
structure, described on \pagexref{data,dds}. The second request is for
the attributes of the data types described in the DDS. This structure
is described on \pagexref{data,das}. The last request is actually for
the data.

The response to the DDS and DAS request URLs is text formatted using
the grammars in \tableref{data,tab,DAS} and \tableref{data,tab,DDS}.
This text can then be parsed by the caller to determine the structure
of the dataset, types and sizes of each of its components and their
attributes. Depending on the data access API used to access the data,
these structures may be derived either from information contained in
the dataset or from ancillary information supplied by the dataset
maintainers in separate text files, or both.  The data in these
structures (which can be thought of as data about the real data) may
be cached by the client system.

The \opendap DAP is a stateless protocol. The protocol \new{entry points}
may be thought of as the different messages to which an \opendap server
will respond. (A message is just a URL specifying a request.)  Each of
the protocol entry points does a single isolated job and they can be
issued in any order. Of course, it may not make sense to the user to
ask for the data before asking for the data description structure, but
that is not the server's problem. This separability allows the user to
cache data locally if need be, so that future accesses to the same
dataset can skip the retrieval of these structures.

To understand the operation of the \opendap server, it is useful to follow
the actions taken to reply to a data request. The diagram in
\figureref{opd-server,fig,server-design} lays out the relationship
between the various entities. Consider an \opendap URL such as the
following:

\begin{vcode}{sib}
http://dods.gso.uri.edu/cgi-bin/nph-nc/data/fnoc43.nc
\end{vcode}

When this URL is submitted to an \opendap client, it will contact the Web
server (httpd) running on the platform, \lit{dods.gso.uri.edu}. When
the connection has been established, the client will forward to the
server the remaining parts of the URL:
\lit{/cgi-bin/nph-nc/data/fnoc43.nc}. As the server parses this
string, it will notice that \lit{cgi-bin} corresponds to the name of
the directory where it keeps its CGI programs. (The actual directory
name is specific to the particular web server used, and the details of
its installation.  Typically, the web server documnetation might call
it the \lit{ScriptAlias} directory, and it might refer to something
like \lit{/usr/local/etc/httpd/cgi-bin}.) It looks in that directory
to see whether there exists a CGI program called \lit{nph-nc}, which
is the name of the netCDF \opendap server packaged with \opendap.  Finally,
the server executes that program, specifying the rest of the URL
(\lit{data/fnoc43.nc} in this case) for an argument.  The standard
output of the CGI program is redirected to the output of the
\lit{httpd}, so the client will receive the program output as the
reply to its request.

\figureplace{The Architecture of an \opendap Data Server.}{htbp}
{opd-server,fig,server-design}{arch.ps}{arch.gif}{}

For APIs that are designed to read and write files, such as netCDF,
the CGI program will be executed with the working directory specified
by the \lit{httpd} configuration. On the \lit{dods.gso.uri.edu}
server, for example, all CGI programs are executed native to the
directory \lit{/usr/local/spool/http}. The last section of the URL,
then, specifies the file \lit{fnoc43.nc} in the directory:

\begin{vcode}{ib}
/usr/local/spool/http/data.
\end{vcode}

Several existing data APIs, such as JGOFS, are not designed with file
access as their fundamental paradigm. The JGOFS system, for example,
uses an arrangement of ``dictionaries'' that define the location and
method of access for specified data ``objects.'' A URL addressing a
JGOFS object may appear to represent a file, like the netCDF URL
above. 

\begin{vcode}{ib}
http://dods.gso.uri.edu/cgi-bin/nph-jg/station43
\end{vcode}

\indc{data dictionary!JGOFS} \indc{JGOFS!data dictionary}
However, the identifier (\lit{station43}) after the CGI program name
(\lit{nph-jg}) represents, not a file, but an entry in the JGOFS data
dictionary. The entry will, in turn, identify a file or a database
index entry (possibly on yet another system) and a method to access
the data indicated. (The \lit{httpd} server must be a valid JGOFS user to
have access to the dictionary.)

Note that the name and location of the \lit{cgi-bin} directory, as
well as the name and location of the working directory used by the CGI
programs, are local configuration details of the particular web server
in use. The location of the JGOFS data dictionary is a configuration
issue of the JGOFS installation. That is to say these details will
probably be different on different machines.

\subsection{Service Programs}
\label{opd-server,service}

\indc{server!services}\indc{\opendap services}
\indc{services!helper programs} 
At this point, the request for data, encoded in a URL, has caused the
\lit{httpd} server to execute the CGI program that represents the \opendap
server. The \opendap server, in turn, executes one of several different
\ind{service} programs, and returns the result of that execution to
the client.  Though there may be others available on a given machine,
five of the services constitute the core functionality of the \opendap
server:

\begin{itemize}
\item Data Attribute
\item Data Description
\item Data
\item ASCII Data
\item Information
\end{itemize}

\note{There are other important \opendap services.  For a description of
  all the \opendap services, see \sectionref{opd-client,services}.}

The \opendap server is structured as a dispatch function, invoking
ancillary helper programs to provide its services.  Installing an \opendap
server involves making sure that each of the required helper programs
is available to the server software.  Here is a table of the helper
programs required for each of the \opendap services for the netCDF 
server.  For another \opendap server, the names of some of the helper
programs would have a different root (e.g. \lit{ff\_} for the FreeForm
server, \lit{jg\_} for JGOFS, etc.).

%
% Updates to this list should also be reflected in ch02.tex.
\begin{table}[htbp]
\caption{\opendap Services, with their suffixes and helper programs\@.}
\label{opd-server,tab,suffixes}
\begin{center}
\begin{tabular}{|p{0.75in}|p{0.75in}|p{2in}|} \hline
\tblhd{Service} & \tblhd{Suffix} & \tblhd{Helper Program} \\ \hline \hline
Data Attribute & \lit{.das} & \lit{nc\_das} \\ \hline
Data Descriptor & \lit{.dds} & \lit{nc\_dds} \\ \hline
\opendap Data & \lit{.dods} & \lit{nc\_dods} \\ \hline
ASCII Data & \lit{.asc} or \lit{.ascii} & \lit{asciival} \\ \hline
Information & \lit{.info} & \lit{usage}, see
\sectionref{sec,document-data} for configuration information.
 \\ \hline
\ifh & \lit{.html} & None \\ \hline
Version & \lit{.ver} & None \\ \hline
Help & Anything else & None \\ \hline
\end{tabular}
\end{center}
\end{table}


The service programs are started by the CGI depending on the extension
given with the URL. If the URL ends with `.das' then the DAS service
program is started. Similarly, the extension `.dds' will cause the DDS
service to run and so on. The CGI program (the ``dispatch'' script),
which serves to dispatch the request to one of the three service
programs, can be very simple.  In the servers distributed with \opendap,
the CGI is simply a shell script that takes its own name and catenates
the enclosed URL suffix. The services, being more complex programs,
will generally be written in C or \Cpp .

On the client side, the user may never see the `.das,'
`.dds,' or `.dds' URL extensions. Nor will the user necessarily be
aware that each URL given to the \opendap client produces three different
requests for information. These manipulations happen within the client
library, and the user need never be aware of them.  \tbd{(Refer to
  Section ?? for more information about how and where this
  substitution takes place)}

There may be more than five service programs for a given server
implementation.\footnote{A couple of services, such as the version and
  help services, are built into the server software, and need no
  configuration.} A server may provide other ``services,'' such as the
catalog service, or a service specific to a particular data
implementation. The three data services, however, constitute the
minimum configuration for a functional server. All three services are
involved in data requests, as the client program will use the output
from the \lit{\_dds} and \lit{\_dds} services to allocate memory and
define parameters for the output of the \lit{\_dods} service, which is
the actual data requested.  The remaining two services, the ASCII and
information services, are primarily intended for interactive use, as
they make dataset and service information directly available to a
browser client, such as Netscape.

\section{Installing an \opendap Server}
\label{opd-server,install}

\indc{server!installing}\indc{installing!server}
Most of the task of installing an \opendap server consists of getting the
required Web server installed and running. The intricacies of this
task, and the variety of available Web servers make this task beyond
the scope of this guide. Proceed with the following steps only after 
the Web server itself is operational.

%See \appref{web-server} for hints about installing an \lit{http}
%daemon (web server).

Installing the \opendap CGI programs and the data to be served is a
\label{opd-server,cgi-install} relatively simple operation. After
installing the \opendap source tree and building the software, (See
\appref{install}), the user need only copy the CGI program from the
\lit{etc} directory in the \opendap source tree (\lit{\$(DODS\_ROOT)/etc})
to one of the directories where the Web server expects to find its CGI
programs. The exact name of this directory is an implementation detail
of the Web server itself.

The service programs used by the CGI are generally kept in the same
directory as the CGI itself, although this can be changed by modifying
the \opendap CGI dispatch script. 

\note{The server programs come with release notes and installation
  notes, in files \lit{README} and \lit{INSTALL}, among others.  These
  will be found in the distribution directories for the particular
  server.  For example, the documentation for the JGOFS server will be
  found in \lit{\$DODS\_ROOT/src/http/jg-dods}.  See \OPDapi\ for
  additional information about server documentation.}

After installing the CGI program and the services, the data to be
provided must be put in some location where it may be served to
clients. Again, the location of the data depends on the configuration
of the Web server and the API used by the CGI services. Most often,
data that is served by a Web server is kept in the \lit{htdocs}
directory, the exact pathname of which is specified in the
\lit{httpd}.configuration file. A server may also be enabled to search
a user's home directory tree or may follow links from the \lit{htdocs}
directory (if the server is enabled to follow symbolic links). There
may be yet other options provided by the specific server used in a
particular installation, so there is really no way to avoid consulting
the configuration instructions of the Web server.

As noted, the location of the data depends not only on the
configuration of the Web server, but also on the API used to access
the data requested.  For example, the netCDF server simply
stores data in a path relative to the working directory of the CGI
program, \lit{htdocs}, while the JGOFS server uses its data
dictionary to specify the location of its data. Refer to the specific
installation notes for each API for more information about the
location of the data.

\subsection{Configuring the Server}

\indc{server!configuration}\indc{configuration!server}
The issues of server configuration depend to a large extent on the
particular server in question.  The \opendap server for JGOFS data is
configured differently than the \opendap server for netCDF data.  Each
server comes with its own installation and configuration instructions.
These can be found in a file called \lit{INSTALL} in the distribution
directory for the server.  The server distribution directories are in
\lit{\$DODS\_ROOT/src}.  Here is a checklist of items that need to be
attended in order to install any \opendap server: \tbd{Is this list
  complete?}  \indc{server!configuration} \indc{configuration!server}
\indc{installation!server}

\begin{itemize}
\item Is the \lit{httpd} server configured to execute CGI programs?
  
\item Are the main CGI and subsidiary CGI programs installed in the
  server's CGI directory?  For the netCDF API, these will be called
  \lit{nph-nc}, and \lit{nc\_das}, \lit{nc\_dds}, and so on.  The
  server CGI's for other API's will have comparable names.
  
\item Is the \lit{gzip} program installed in the \lit{PATH} of the
  \lit{httpd} server?  This is used to compress data messages returned
  to the client.

\end{itemize}

\subsection{Constructing the URL}

After a dataset has been installed, and the server programs installed,
you need to know what its address is.  \sectionref{opd-client,url}
contains an explanation of the various parts of the \opendap URL,
including a diagram in \figureref{opd-client,fig,url-parts}.  Refer
to this section, with a copy of the Web server configuration data
readily available.  Using the configuration data, you should be able
to determine the appropriate URL for the data you are serving.

Remember that the web server will have its own definition of the root
directory for data, and another definition for CGI programs, depending
on the configuration.  


\subsection{Documenting Your Data}
\label{sec,document-data}

\indc{documenting data}\indc{data!documenting}
\indc{service!info}\indc{info!service}
\opendap contains provisions for supplying documentation to users about a
server, and also about the data that server provides.  When a server
receives an information request (through the \lit{info} service that
invokes the \lit{usage} program), it
returns to the client an HTML document created from the DAS and DDS of
the referenced data.  It may also return information about the server,
and more detail about the dataset.

If you would like to provide more information about a dataset than is
contained in the DAS and DDS, simply create an HTML document (without
the \lit{<html>} and \lit{<body>} tags, which are supplied by the
\lit{info} service), and store it in the same directory as the
dataset, with a name corresponding to the dataset filename.  For
example, the datasets \lit{fnoc1.nc}, \lit{fnoc2.nc}, and
\lit{fnoc3.nc} might be documented with a file called \lit{fnoc.html}.

You may prefer to override this method of creating documentation and
simply provide a single, complete HTML document that contains general
information for the server or for a group of datasets.  For example,
to force the info server to return a particular HTML document for all
its datasets, you would create a complete HTML document and give it
the name \var{dataset}\lit{.ovr}, where \var{dataset} is the dataset
name.

More information about providing user information, including sample
HTML files, and a complete description of the search procedure for
finding the dataset documentation, is to be found in \OPDapi .

\subsection{Testing the Installation}

\indc{server!testing}\indc{testing an \opendap server}
It is possible to test the \opendap server to see whether an installation
has been properly done.  The easiest way to test the installation is
with a simple Web client like Netscape or Mosaic. (A simple Web client
called \lit{\ind{geturl}} is provided in the \opendap core software which
can retrieve text from Web servers.  Look for it in the
\lit{\$(DODS\_ROOT)/etc} directory.)

The simplest test is simply to ask for the version of the server, or
the help message.  The \opendap server uses helper programs to return the
DAS, DDS, and data.  If you want to test the server itself, and not
the configuration of the helper programs, the version, help, or info
services will suffice.  Issuing a URL with \lit{.ver} on the end will
return the version information for this server, appending \lit{.info}
will return the info message, and issuing a URL with a nonsense suffix
or \lit{.help} will return a help message:

\begin{vcode}{ib}
> geturl http://dods.gso.uri.edu/cgi-bin/nph-nc/data/test.nc.ver
> geturl http://dods.gso.uri.edu/cgi-bin/nph-nc/data/test.nc.info
> geturl http://dods.gso.uri.edu/cgi-bin/nph-nc/data/test.nc.help
\end{vcode}

To return the data attribute structure of a dataset, use a URL such as
the following:\footnote{The \lit{geturl} program knows about the \opendap
  protocols, so you can also omit the \lit{.das} suffix, and use the
  \lit{-a} option to the \lit{geturl} command.  This tells
  \lit{geturl} to append \lit{.das} for you.}

\begin{vcode}{ib}
> geturl http://dods.gso.uri.edu/cgi-bin/nph-nc/data/test.nc.das
\end{vcode}

Refer to \sectionref{data,das} for a description of a data attribute
structure. You can compare the description against what is returned by
the above URL to test the operation of the \opendap server.

You can use your web client to test the \opendap server by using it to
submit URLs that address specific services of the client.  See
\sectionref{opd-client,services} for information about how to request
individual services.  If any of the services fail, you can check the
list of helper programs in \sectionref{opd-server,service} to find
out which is missing.  From the web browser, you can access all the
\opendap services, except the (binary) data service.  However, if all the
others work, you can be relatively assured that one will, too.

Using the \lit{.html} suffix produces the \ind{\ifh}, providing a
forms-based interface with which a user can query the dataset using a
simple web browser.  There's more about the \ifh\ in
\chapterref{opd-client}. 

\section{Displaying Information to the \opendap User}
\label{opd-server,display}

\indc{GUI|see{graphic user interface}} \opendap contains a system that
allows an \opendap server a degree of control over the user's graphic user
interface (GUI). \indc{graphic user interface} This system runs the
system progress indicator, that displays to the user the status of a
pending data request. However, a server may also use the GUI interface
to display messages to the user, such as error messages, and even to
query the user for information.

\subsection{GUI Architecture}
\label{opd-server,gui}

\indc{graphic user interface!architecture} \indc{progress indicator}
Since \opendap is built inside a data access API, and since the
application program that has become the \opendap client was presumably not
built with network I/O in mind, an \opendap client will typically not do
any processing at all while it awaits a return message from a data
request. Any communication that must happen between the \opendap software
and the user must occur without the involvement of the application
program that has invoked the \opendap software. To avoid this limitation,
\opendap starts up a \new{GUI manager} sub-process. This sub-process can
receive data from the \opendap core software, and can operate the user's
graphical user interface.  \indc{manager!GUI} \indc{Tcl!interpreter
  subprocess} \indc{wish inter@\lit{wish} interpreter subprocess}

The operation of the GUI manager is illustrated in
\figureref{opd-server,fig,gui}.  As seen in the figure, the client
application can usually control the user's screen, but during a data
request, this communication is suspended. Until the request returns
control to the client application, messages returned from the \opendap
server can be displayed to the user by passing them to the GUI manager
sub-process, who can display them in a window to the user.

\figureplace{The Architecture of an \opendap Client GUI.}{htbp}
{opd-server,fig,gui}{wish.ps}{wish.gif}{}

The GUI manager in \OPDversion uses a Tcl/Tk interpreter (the
\lit{wish} program is the default) to interpret messages from the
server. These messages usually contain Tcl programs to display
information to the user. However, the \lit{wish} interpreter can also
be sent programs to query the user for more information, or draw
little rabbits on the screen or any other graphic function the server
needs to have displayed to the user. See Tcl and the Tk
Toolkit~\citel{osterhout:tcl} for more information about Tcl.

By default, the GUI manager initializes by running the Tcl programs in
the files \lit{dods\_gui.tcl}, \lit{error.tcl} and \lit{progress.tcl}.
(These are stored in \lit{\$DODS\_ROOT/etc}.) Server commands to the
GUI manager can use the functions defined in these files. Note also
that the user may be using a ``safe'' Tcl interpreter, with a
restricted subset of the usual array of Tcl commands available to it.
The user can control these features of the operation of the GUI by
changing several environment variables.  These are described in
\sectionref{opd-client,environment}. 

\indc{graphic user interface!error system} \indc{error system}
\indc{message!error}

A server will use the features of the GUI manager to display error
messages to the user. A server may also use the GUI to query a user to
correct whatever condition caused the error. For example, if a user has
misspelled some part of a constraint expression in a URL submitted to a
server, the server can send the constraint expression back to the user in
an edit window, with instructions to fix it. The user can edit the
expression, and send it back, allowing the server to proceed without
submitting a new request. Consult the client and server toolkit manual
for more information about the \class{Error} object on this subject.

\section{Building \opendap Data Servers}
\label{opd-server,building}

Though servers are included in the \opendap core software, some
users may wish to write their own \opendap data servers. The architecture
of the \lit{httpd} server and the \opendap core software make this a
relatively simple task.  

A user may wish to write his or her own \opendap server for any or all of
the following reasons:

\begin{itemize}
  
\item The data to be served may be stored in a format not compatible
  with one of the existing \opendap servers.
  
\item The data may be arranged in a fashion that allows a user to
  optimize the access of those data by rewriting the service programs.
  
\item The user may wish to provide ancillary data to \opendap clients not
  anticipated by the writers of the servers available.

\end{itemize}

The design of the \opendap library make the task a relatively simple one
for a programmer already familiar with the data access API to be used.
Also, though the servers provided with the \opendap core software are
written in C++, they may be written in any language from which the
\opendap libraries may be called.

Once it is invoked, a CGI program scoops up whatever input is going to
the standard input stream of the Web server (\lit{httpd}) that invoked it.
Further, the standard output of the CGI is piped directly to the WWW
library, which sends it directly back to the requesting client. This means
that the CGI program itself need only read its input from standard input
and write its output to standard output.

Most of the task of writing a server, then, consists of reading the data
with the data access API and loading it into the \opendap classes. Method
functions defined for each class make it simple to output the data so that
it may be sent back to the requesting client.

Refer to \OPDapi\ for specific information about the classes and the
facilities of the \opendap core software, and instructions about how to
write a new server.




% $Log: ch04.tex,v $
% Revision 1.11  2003/09/04 19:42:06  tom
% DODS->OPeNDAP
%
% Revision 1.10  2000/12/18 18:53:32  tom
% anticipate creation of server installation guide
%
% Revision 1.9  2000/10/04 15:02:14  tom
% changed \figureplace definition, misc other cleaning
%
% Revision 1.8  2000/03/23 18:26:14  tom
% misc. updates
%
% Revision 1.7  1999/10/18 21:39:44  tom
% updated for new services, fixed server test section, add ifh intro.
%
% Revision 1.6  1999/08/30 14:19:28  tom
% updates prior to DODS release 3.0
%
% Revision 1.5  1999/02/04 17:42:13  tom
% modified to use dods-book.cls and Hyperlatex
%
%
%




%%% Local Variables: 
%%% mode: latex
%%% TeX-master: "~/DODS/doc/user/guide"
%%% End: 
