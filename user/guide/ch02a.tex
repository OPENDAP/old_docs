% Building an OPeNDAP client.
%
% $Id$
%

\chapter{The \opendap Client}
\label{opd-client}

There are many different data analysis packages in use. Some packages, such
as MATLAB and IDL, are commercially available, but many more are written for
a specialized need or application. Many of these use one of the widely
available sets of scientific data access functions (called an {\em
  Application Program Interface}, or \ind{API})\indc{Application Program
  Interface|see{API}} such as NetCDF, JGOFS, or HDF. There is great variety
among all these programs, but one feature they share is that they all access
data through files containing that data\footnote{This is not true of some
  APIs, such as JGOFS.  That API, however, uses a data dictionary to allow
  the user to think that the data access is through files.}.  That is to say
that each program begins by identifying a file containing the data the user
wishes to examine or analyze.

An \ind{\opendap client} \indc{client!\opendap} is simply a data
analysis application linked with the \opendap libraries instead of the
standard data access API. Using this program, a user can look at files
containing data in the same way as was possible without the \opendap
libraries.  However, by using these libraries, a user can also use a
URL (URL)\indc{Uniform Resource Locator|see{URL}} \indc{URL}, instead
of a simple file name, to specify data located anywhere on the
Internet.  \Figureref{intro,fig,unlinked} and
\figureref{intro,fig,linked} illustrate the operation of an
application program linked with a standard data access API, and the
same program linked with the \opendap version of that API.

An \opendap client is then a data analysis application program
modified to become a web browser, somewhat like any other \ind{web
  browser} (NCSA Mosaic\indc{Mosaic}, \ind{Netscape Navigator}) with
which you may be familiar. A web browser can only display the data it
receives, however. What makes an \opendap client different from
another web browser is that, unlike Netscape, once the data has been
received from an \opendap server, the \opendap client application can
compute with it.

Like a web browser, an \opendap client accepts a URL from a user, and
parses it to come up with a protocol, an address, and a message. (See
\sectionref{opd-client,url} for more information about URLs.) The
browser then sends a message to the address, directed to the server
who can service the desired protocol, asking for the information
specified in the remainder of the URL. Unlike a typical web browser, an \opendap client will not know what to do with data returned for a web page
containing text and pictures, but an \opendap server will return scientific
data that an \opendap client can understand and process.

Here is a simple example, using the \lit{ncview} program. This program
simply prints out the contents of a netCDF formatted data file,
specified on the command line, like this:

\begin{vcode}{ib}
> ncview fnocl.nc
\end{vcode}

Using \opendap, this same function may be executed from any computer connected to
the Internet by substituting a URL\indc{URL!instead of file name} for the
filename\indc{file name!use URL instead} above:

\begin{vcode}{ib}
> dncview http://dods.gso.uri.edu/cgi-bin/nc/data/fnocl.nc
\end{vcode}
\label{opd-client,eg}

(See \figureref{opd-client,fig,url-parts} Aside from the fact that
the data is remote, and must be specified with a URL, the program will
seem to function in the same way it had with the simple netCDF library
(albeit somewhat more slowly due to having to make network connections
instead of local file operations). You can find \lit{dncview} (the
\lit{ncview} program linked with the \opendap library) in the

\begin{vcode}{ib}
$DODS_ROOT/src/nc-dods/ncview
\end{vcode}
%$

directory\label{opd-client,using-example}. Running the above command will produce the following output:

\tbd{RECAPTURE THIS OUTPUT}

\begin{vcode}{ib}
netcdf fnocl {
dimensions:
    time_a = 16
    lat = 17 ;
    lon = 21 ;
    time = 16 ;

variables:
    long u(time_a, lat, ion) ; 
        u:units = ``meter per second'' ; 
        u:long_name = ``Vector wind eastward component'' ; 
        u:missing_value = ``-32767'' ; 
        u:scale_factor = ``0.005'' ; 
    long v(time_a, lat, ion) ; 
        v:units = ``meter per second'' ;
        v:long_name = ``Vector wind northward component'' ;
        v:missing_value = ``-32767'' ;
        v:scale_factor = ``0.005'' ; 
    double lat(lat) ;
        lat:units = ``degree North'' ;
    double lon(lon) ;
        lon:units = ``degree East'' ; 
    double time(time) ;
        time:units = ``hours from base_time'' ;

// global attributes: 
        :base_time = ``88- 10-00:00:00'' ; 
        :title = ``FNOC UV wind components 
                           from 1988- 10 to 1988- 13.'' ;
data:
 u =
  -1728, -2449, -3099, -3585, -3254, -2406, -1252,
    662, 2483, 2910, 2819, 2946, 2745, 2734,
  2931, 2601, 2139, 1845, 1754, 1897, 1854, -1686,
...
\end{vcode}

Although there are packaged \opendap browsing programs that a user can use
to look at data, the user can also construct his or her own.  Linking
an \opendap API with an already existing program allows a user to create a
customized web browser that can access data available from any \opendap
server connected to the Internet.

The \opendap APIs are designed to accurately mimic the behavior of several
different commonly used scientific data APIs.  As of this writing
(\today), the \opendap API set includes:
\indc{API!list of \opendap supported}

\begin{longtable}{|p{0.5in}p{3in}p{1in}|}
\caption{Supported APIs\label{opd-client,supported-APIs}} 
\\ \hline
\textbf{API} & \textbf{Description} & \textbf{Components} \\ \hline
\endfirsthead
\caption[]{Supported APIs} 
\\ \hline
\textbf{API} & \textbf{Description} & \textbf{Components} \\ \hline
\endhead
\hline
\endfoot
netCDF & 
Support for gridded data, such as satellite data,
  interpolated ship station data, or current meter data. & 
Server and client. \\ \hline
JGOFS & 
Support for relational data, such as \class{Sequences}.  
  Created by the Joint Globar Ocean Flux Study (JGOFS) project for use 
  with oceanographic station data. & 
Server and client. \\ \hline
HDF & 
Support for gridded data.  Commonly used for astronomical
  data and model data. & % \rule[-3pt]{0pt}{1pt}&
Server only. \\ \hline
DSP & 
Oceanographic and geophysical satellite data.  Provides
  support for image processing.  Developed at the University of
  Miami/RSMAS.  Primarily used for AVHRR and CZCS data. &
Server only. \\ \hline
GRIB & 
Support for gridded binary data.  GRIB is the World
  Meteorological Organization (WMO) format for the storage of weather
  information and the exchange of weather product messages. &
Server only, due in early 1999. \\ \hline
BUFR & 
  The WMO's standard set of codes for the transmission and
  storage of meteorological data, using a compressed format with each
  data value occupying the least number of bits necessary to contain 
  its range of values.  Suitable for meteorological observations made
  from a single point or set of points. & 
Server only, due in early 1999. \\ \hline
Free\-Form & 
  On-the-fly conversion of arbitrarily formatted data,  including
  relational data and gridded data.  May be used for sequence data,
  satellite data, model data, or any other data format  that can be
  described in the flexible FreeForm format definition 
  language.  This server can be used to serve data stored in almost
  all home-grown data formats. &
Server only; no client required. \\ \hline
native \opendap & 
  The \opendap class library may be used directly by a client program.  It
  supports relational data, array data, gridded data, and
  a flexible assortment of data types that can be combined to
c  accommodate most data models. &
Client. \\ \hline
\end{longtable}

The API set is extensible, meaning that developers can use the \opendap
software toolkit to write \opendap-compliant versions of new APIs.  See
\OPDapi\ for more information.

The most important result of this architecture is that, just as the
use of the \lit{dncview} program above is identical to the original
\lit{ncview}, a user can use remote \opendap data {\em and} continue to
use the same data analysis and display programs with which he or she
is familiar. Any program that uses one of the \opendap-supported APIs may
be re-linked to use the \opendap version of that API.  This creates an \opendap
client. That and a connection to the Internet, are all that a
researcher requires to gain access to the available \opendap data.

\section{Configuring Programs to Use \opendap}
\label{opd-client,link}

\indc{linking~client programs with \opendap} 
\indc{configuration!user programs(relinking)} \indc{user programs!configuring}
\indc{DODS!configuring} \indc{re-link|see{linking}} \indc{converting to \opendap}
\indc{DODS!conversion} 

Relinking an existing program with the \opendap implementation of some
data API is a simple procedure.  Find the directory that contains the
source/object code of the program you want to re-link and modify the
makefile (typically called \lit{Makefile}) for the program so that the
\opendap-compliant API library is used in place of the standard API
library.  (If you can't find the libraries on your system, see
\appref{install}, or ask the system administrator.) These
libraries are: 
\indc{libraries!necessary for \opendap}

\begin{description}
\item[\lit{libdap++.a}] Software common to all of the \opendap-supported
  APIs.
\end{description}

\opendap also uses facilities from some standard libraries, and these must
also be included in the link to resolve all the symbols.

\begin{description}
\item[\lit{libwww.a}] The World Wide Web library. \indc{World Wide
    Web!library} This contains the functions used to communicate
  between the \opendap client and server.

\item[\lit{libexpect.a}] Functions from the \lit{expect}
library are used to communicate between
\opendap client processes. \indc{expect!library}

\item[\lit{libtcl.a}] Contains definitions necessary for the
  \lit{expect} library.  The use of this library in the link is not
  related to the use of Tcl by \opendap clients.  \indc{Tcl!library}
  
\item[\lit{libstdc++.a}] 
  The GNU C++ class library (This is not necessary if using \lit{g++}
  to re-link.) \indc{stdc++}\indc{libstdc++}

\end{description}

You will also need to include the library containing the
\opendap-compliant version of the API. The name of this library of course
depends on the API, but it is generally in the form

\begin{example}
\lit{lib{\em API}-dods.a} 
\end{example}

Where {\em API} is an abbreviation indicating the API emulated by the
specified library.  For example, the \opendap-compliant netCDF library is
called \lit{libnc-dods.a} and the JGOFS version is \lit{libjg-dods.a}.

\subsection{An Example Using netCDF}
\label{opd-client,link-example}

The \lit{ncview} program is a simple utility that prints the contents
of a netCDF-format file to standard output.  This section outlines the
process used to modify the \lit{ncview} makefile to link that program
with the \opendap netCDF API, thereby turning \lit{ncview} into a
network-ready \opendap client. The process of linking any other program
with the corresponding \opendap library is entirely analogous to this one
and only requires the substitution of the program name and the
appropriate library.

First the link flags were modified so that the library search path
would include the likely places to find the \opendap libraries:

\begin{vcode}{ib}
LDFLAGS = -g -L$(DODS_ROOT)/lib
\end{vcode}
%$

\indc{distribution directory|see{\lit{DODS\_ROOT}}}%
\indc{master directory|see{\lit{DODS\_ROOT}}}%
\indc{environment variable!DODS\_ROOT}%
\indc{variable!environment}%

\lit{DODS\_ROOT} is an environment variable that indicates the root
directory of the \opendap installation, and in this manual is used as
shorthand for this directory.  It is typically called something like
\lit{/usr/local/DODS}. If you cannot find these directories on your
system, consult your system administrator, or refer to
\appref{install} for information about acquiring and installing
the \opendap software.

After the link flags were modified, the \opendap libraries were added to the list
of libraries used. The order in which the libraries are listed is important.

\begin{vcode}{ib}
LIBS = -lnc-dods -ldap++ -lnc-dods -ldap++ -lwww -ltcl 
       -lexpect -lz -lrx
\end{vcode}
\tbd{Check this.}

\note{Because \opendap is implemented as a core set of classes contained in one
library (\lit{libdap++.a}) and a set of specializations of those classes in a
second library (\lit{libnc-dods.a}), and because there is a circular
dependence between those two libraries, they must be included twice in the
linker command.}

Finally, \lit{g++} was substituted for the link command.\footnote{It
  is possible to use \lit{\ind{gcc}} instead of \lit{g++}, but in that
  case, \lit{-lg++} must be added to the end of the library list.}


\subsection{Potential Problems}

\indc{linking!problems} \indc{problems!linking} \indc{troubleshooting!linking}
When a user links an existing a program to the \opendap libraries, there are
several possible conditions that may cause problems.

\begin{itemize}
\item Some programs use more than one API.

\item Some programs access data using both API and UNIX system calls.

\item Some programs use undocumented features of the APIs.

\end{itemize}

If this is the case for a given program, there is generally no good solution
beside rewriting the software to conform to a strict usage of the data
reading parts of the given API. Of course if the problem is that the
program uses more than one API, you can try linking the program with an \opendap-compliant version of the second API as well.

\begin{itemize}
\item Re-linked programs can be very large.
\end{itemize}

\indc{size!of \opendap programs} \indc{problems!size} \indc{troubleshooting!size
  of executable} 
The \opendap libraries are large, and the \lit{g++}, \lit{www},
\lit{expect}, and \lit{tcl} libraries on which they are built are even
larger. This means that the executable version of a re-linked \opendap
client can seem unreasonably obese. Much of the disk space is occupied
by symbol tables, which can be removed from the executable file with
the \lit{strip} utility.  In many cases, a user can recover a
substantial amount of disk space this way.

\note[CAUTION]{Without familiarity with the \opendap software, it is best
  only to strip the executable files. Stripping object files or
  libraries might leave them in a useless condition for the linker.
  Furthermore, stripping an executable file removes symbol names,
  which may make diagnosing problems more difficult.}

The \opendap libraries only affect the data \emph{reading} functionality
of the specified API. There are no \opendap replacements for functions
like netCDF's \lit{ncputrec()}, that \emph{write} data to a disk file.
These functions are included in the \opendap-compliant API library, but
they operate in a manner identical to the original (non-\opendap)
versions, that is, they work on local files only, attempting to write
``over the network'' will result in an error.  \indc{API!data output
  functions} \indc{data!output functions}

\section{Writing New \opendap Programs}

The \opendap software may also be used to write new programs. This may be
done either through one of the \opendap-supported API libraries, such as
netCDF or JGOFS, or by using the \opendap data access protocol directly.
There are advantages and disadvantages to each approach.
\indc{writing new programs} \indc{new programs!writing}
\indc{DODS!writing new programs}

The biggest advantage of writing new code using an \opendap-supported API
such as netCDF or JGOFS is that the programmer in question is probably
already familiar with the use of that API. Writing an \opendap program using
an adapted API is not significantly different than writing the same
program with the original API. While writing this new program, it will be
useful to remember that the data the program uses will often be remote,
implying that data retrieval may not be instantaneous, and that
implementation of local caching to store requested data might be a good
idea, but other than that, the process is the same as writing a program
using the regular API.
\indc{native API} \indc{API!native} \indc{protocol!data access}
\indc{Data Access Protocol}

It is also possible to use the \opendap data access protocol directly.
This is somewhat more involved than using one of the \opendap-compliant
API libraries, and C++ is the only language supported for this.
However, this approach can provide substantially more efficient
programs. For further information about this approach, refer to the
technical information about the DAP in \OPDapi .

% $Log: ch02a.tex,v $
% Revision 1.4  2003/09/04 19:42:06  tom
% DODS->OPeNDAP
%
% Revision 1.3  2001/05/04 14:39:52  tom
% added pdf to Makefile, added .dodsrc info to user guide
%
% Revision 1.2  2000/10/04 15:02:14  tom
% changed \figureplace definition, misc other cleaning
%
% Revision 1.1  1999/10/18 21:39:44  tom
% updated for new services, fixed server test section, add ifh intro.
%
%


%%% Local Variables: 
%%% mode: latex
%%% TeX-master: t
%%% End: 
