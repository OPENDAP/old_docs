% Preface to the OPeNDAP User Guide
%
% $Id$
%

\T\chapter*{Preface}
\T\addcontentsline{toc}{chapter}{Preface}

This document describes version \OPDversion\ of the \OPD\ (\opendap), a
data system intended to allow researchers transparent access to
oceanographic data---stored in any of several different file
formats---across the Internet. Using \opendap function libraries, many
existing data analysis programs can be easily modified to accommodate
access of distant datasets in a manner identical to the access of
local datasets. \opendap includes a protocol for the transmission of data
across the Internet, and supports selection of data using constraint
expressions, and translation of data from one format to another.

An overview of the system's use is presented, and specific tasks
illustrated, for data providers as well as for users.

\W\htmlmenu{4}
\W\chapter*{Preface}

%%%%%%%%%%%%%%%%%%%%%%%%%%%%%%%%%%%%%%%%%%%%%%%%%%%%%%%%%%%%%%%%%%%%
\section{Tasks Illustrated in this Guide}
\label{pref,tasks}

For a quick start to getting, installing, and using \opendap software, see the
list below of tasks described in this document.

\begin{itemize}

\item Getting the \opendap software. (\pagexref{install})

\item Installing the \opendap software. (\pagexref{install})

\item Using an \opendap client. (\pagexref{opd-client,using-example})

\item Re-linking a data analysis or display application to become a
  \opendap client. (\pagexref{opd-client,link-example})

  \tbd{Add information about the locator here.}
%\item Using the OPeNDAP data locator to find data. 
%(\pagexref{locator,locator})

\item Creating and installing an \opendap server.

  \begin{itemize}
    
  \item Installing the \opendap server and CGI filters.
    (\pagexref{opd-server,cgi-install})

  \item Starting and configuring the httpd server. Due to the variety
    of available servers, this task is beyond the scope of the manual.
    Please refer to the documentation for the particular server in question
    for more information.

  \item Implementing a new DODS-compliant API. (\OPDapi)

  \end{itemize}

\item Writing an \opendap CGI program. (\OPDapi )

\item Writing the CGI service programs. (\OPDapi )

\item A list of all supported APIs. (\pagexref{opd-client,supported-APIs})

\end{itemize}


%%%%%%%%%%%%%%%%%%%%%%%%%%%%%%%%%%%%%%%%%%%%%%%%%%%%%%%%%%%%%%%%%%%%
\section{Who is this Guide for?}
\label{pref,who}

The user documentation for \opendap covers two groups of users: those who
want to provide access to data vian \opendap and those who want to use
data. In many cases the people will be one and the same since most
providers will also be data users.

This documentation assumes that the readers are familiar with
computers, but are not necessarily programmers. 

This guide also contains technical information that will be of
assistance to programmers who plan to write new DODS-compliant APIs
for as yet unsupported data models. Data providers and data consumers
may find some general questions answered by this material, but it is
not necessary to know any of it in order to use the system.


%%%%%%%%%%%%%%%%%%%%%%%%%%%%%%%%%%%%%%%%%%%%%%%%%%%%%%%%%%%%%%%%%%%%
\section{Organization of this Document}

This book is organized into separate sections for data providers, data
consumers, and technical reference material for programmers.

\begin{description}

\item{\bf Part I} is for everybody who wants to use \opendap.

  \begin{description}
    
  \item{\bf\chapterref{intro,opd}} provides a high-level overview of
    the entire system.

  \end{description}
  
\item{\bf Part II} is for data consumers, that is, the people who want
  to look at data using the \opendap system.

  \begin{description}
    
  \item{\bf\chapterref{opd-client}} shows how to look at data using
    \opendap.  It includes a section about the theoretical and practical
    problems of data model translation.  It also explains how to build
    an \opendap client, which is the program used to look at \opendap data.
    
\tbd{Put stuff about using the CS here.}
%  \item{\bf\chapterref{locator}} introduces and explains the data
%    locator and the other facilities provided with OPeNDAP to hunt for
%    data.  It includes information about the server catalog and the
%    boolean services.

  \end{description}
  
\item{\bf Part III} is for data providers, or people who want to make
  their data available through \opendap servers.

  \begin{description}
    
  \item{\bf\chapterref{opd-server}} shows how to use \opendap to
    make your data available to others.  It explains how to set up a
    \opendap server to provide \opendap data to \opendap clients, and also
    contains information about modifying or writing an \opendap server.

\tbd{Put stuff about setting up a CS here.  Also participating in the
  grand CS.}
%  \item{\bf\chapterref{register}} explains how to advertise one's data.
%    It shows how to enter the location of a data set into the data locator
%    database, how to assemble the data catalog, and discusses the role of the
%    data provider.

  \end{description}
  
\item{\bf Part IV} contains technical information about how \opendap
  works. This information is provided to people who want to write new
  libraries to use \opendap through a currently unsupported API.

  \begin{description}
    
  \item{\bf\chapterref{data}} contains general information about
    Data and Data Models.  This is important information to have for
    people intending to use \opendap to provide data to others.  It covers
    the \opendap data attribute and data descriptor structures.  The
    chapter also contains a section outlining the problems associated
    with Data Model translation.

  \end{description}

\item{\bf Appendices}

  \begin{description}
    
  \item{\bf\appref{install}} contains the instructions for installing
    the \opendap libraries, and software that requires these libraries.

  \item{\bf Glossary} A small but useful collection of terms.

  \end{description}

\end{description}

%%%%%%%%%%%%%%%%%%%%%%%%%%%%%%%%%%%%%%%%%%%%%%%%%%%%%%%%%%%%%%%%%%%%

\listconventions

\tbd{Are there any other conventions that need illustration?}

% $Log: preface.tex,v $
% Revision 1.6  2003/09/04 19:42:06  tom
% DODS->OPeNDAP
%
% Revision 1.5  1999/08/30 14:19:28  tom
% updates prior to DODS release 3.0
%
% Revision 1.4  1999/02/04 17:42:13  tom
% modified to use dods-book.cls and Hyperlatex
%

%%% Local Variables: 
%%% mode: latex
%%% TeX-master: t
%%% End: 
