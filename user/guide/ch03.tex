
% $Id$

\chapter{Finding Data}
\labl{locator}

The DODS data location service is currently under development. When the
software becomes available for general use this section of the manual will be
updated and will describe how that service can be used.

%DODS provides three tools to use in finding data: the Locator, the server
%catalog Service, and the boolean Service.  The first can help narrow a
%search for data by identifying data servers which can provide data.  The
%second and third tools are used to make queries about specific data within
%a data set.

%\section{The Data Locator}
%\labl{locator:locator}

%\indc{Locator Service} \indc{data!locator} \indc{finding data}
%\indc{data!search} \indc{searching data} \indc{Service!Locator}
%DODS does not pretend to remove all the overhead of data searches.  A user
%will still have to keep track of the URLs of interesting data sets in the
%same way a user must now keep track of the names of files containing
%interesting data.  DODS provides a data locator Service in cooperation with
%the \ind{Global Change Master Directory} ({\tt http://gcmd.gsfc.nasa.gov/}).  
%\indc{GCMD|see{Global Change Master Directory}}
%This is
%simply a DODS server designed to respond with the URLs of datasets
%satisfying the query criteria.  A user can also use some less formal avenue
%to find URLs, such as asking a friend or consulting a bathroom wall.

%The DODS Locator is a dialog web page at the Global Change Master Directory
%database. DODS data URLs are archived at the GCMD site and the Locator
%query system, can search the database and return to the user a list of the
%URLs that contain data that satisfy the query request.  The user may then
%cut this list from the web browser and paste it into her application.

%The URL for the DODS data Locator at the GCMD is:

%{\tt http://gcmd.gsfc.nasa.gov/query1.html}

%\section{The Server Catalog Service}
%\labl{locator:catalog}

%\indc{Service!Server Catalog} \indc{Catalog Service} \indc{Service!Catalog}
%\indc{data!Catalog} \indc{ls}
%The Server Catalog Service (SCS) provides a listing of names and summary
%information of all the data sets accessible from a particular DODS server. 
%It is analogous to the UNIX {\tt ls} command which provides a simple listing of
%files in a directory.  The SCS supports a limited set of constraint options
%which permit a client to query DODS servers by dataset class.
%\tbd{The Server Catalog is said to eventually support constraint expressions.
%Right now it just returns the catalog of everything on that server.}

%The Server Catalog is returned in response to a server catalog URL. This
%URL is formed by appending {\tt catalog} after the cgi 
%directory name in the DODS
%URL. An example is shown in Figure~\refl{locator:fig:catalog}
%\tbd{When the catalog server supports limited CE's uncomment the latex source right here:}

%\begin{figure}[h]
%\begin{center}
%${}\overbrace{\tt http://dods.gso.uri.edu}^{URL}:
%\overbrace{\tt /cgi-bin/nc/catalog}^{cgi extension}$
%\caption{A Server Catalog Service URL}
%\end{center}
%\label{locator:fig:catalog}
%\end{figure}


%The Server Catalog for DODS Version 2.06 is a simple hypertext file, such
%as may be read with a Web client like Netscape or Mosaic.

%\section{The Boolean Service}
%\labl{locator:boolean}

%The \ind{Boolean Service} provides a simple means of determining whether or not a
%dataset is available for access from a DODS server or whether a given
%dataset has data for a specified set of constraints. \indc{Service!Boolean}

%The Boolean server is invoked with a DODS URL. Simply using the string bool
%in the URL, after the server name. For example, the netCDF boolean server
%at the DODS home site is 

%{\tt http://dods.gso.uri.edu/cgi-bin/nc/bool}

%To make a query of the site, a user would type:

%{\tt > dncdump  http://dods.gso.uri.edu/cgi-bin/nc/bool/data/fnoc1.nc?u<0}

%This URL will return true if there are any wind measurements in fnoc1.nc
%with a westward component.  Note that the return value of this URL is a
%boolean value.  It is meant to be used by a DODS client that can interpret
%boolean values.  This URL is suitable for use in a DODS constraint
%expression, for example. (See Section~\refl{dods-client:CE:url}. for information about using
%URLs in a constraint expression.) A regular web browser like Netscape will
%not be able to interpret the reply message produced by this URL.

%Note that when a constraint expression is specified to the Boolean Service
%there is an inherent ambiguity when the service returns false.  The service
%could return false because the data set does not exist or because the data
%set exists but no data within the range specified by the constraint
%expression exists.



