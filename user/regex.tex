% A guide to installing DODS servers.
%

\documentclass{dods-book}
\usepackage{longtable}

\rcsInfo $Id$
\newcommand{\DOCversion}{Version \rcsInfoRevision}

%%% Tex customizations and command definitions for DODS user
%%% guides, 2 October 1997 - tomfool
%%%
%%% $Id$

%%% $Log: layout.tex,v $
%%% Revision 1.5  1999/05/25 20:49:34  tom
%%% changed version numbers to 3.0
%%%
%%% Revision 1.4  1999/02/04 17:39:02  tom
%%% modified for use with dods-book.cls
%%%
%%% Revision 1.3  1998/03/13 20:43:26  tom
%%% writing of API manual.
%%%
%%% Revision 1.2  1998/02/12 15:47:25  tom
%%% updated for GUI doc
%%%
%%% Revision 1.1  1997/10/02 17:18:14  tom
%%% moved from user guide to boiler, made slightly more general,
%%% for use with other guides.
%%%

%%%%%%%%%%%%%%%%%%%%%%%%%%%%%%%%%%%%%%%%%%%%%%%%%%%%%%%%%%%%%%%%%%%%%
%%% This file is for TeX macros that are equally appropriate for the
%%% hard-copy (LaTeX) and html (hyperlatex) versions of the dods
%%% books.  If it's not appropriate for both, then it probably belongs
%%% in dods-book.cls or dods-book.hlx.

%%%%%%%%%%%%%%%%%%%%%%%%%%%%%%%%%%%%%%%%%%%%%%%%%%%%%%%%%%%%%%%%%%%%%
\NotSpecial{\do\_}% This removes the special meaning of `_', so for
                  % subscripts, there must be an `tex' environment
                  % around any diagram using subscripts, and an entire
                  % alternate html figure using <sub> tags.


%%%%%%%%%%%%%%%%%%%%%%%%%%%%%%%%%%%%%%%%%%%%%%%%%%%%%%%%%%%%%%%%%%%%%
%%% Different kinds of cross references.
%\newcommand{\chapterref}[1]{Chapter~\refl{#1} on page~\pagerefl{#1}}
\newcommand{\chapterref}[1]{\link{Chapter~\ref{#1}}{#1}}
\newcommand{\appref}[1]{\link{Appendix~\ref{#1}}%
  [~on page~\pageref{#1}]{#1}}
\newcommand{\sectionref}[1]{\link{Section~\ref{#1}}%
  [~on page~\pageref{#1}]{#1}} 
\newcommand{\pagexref}[1]{\link*{here}[page~\pageref{#1}]{#1}}
\newcommand{\tableref}[1]{\link{table~\ref{#1}}{#1}}
\newcommand{\Tableref}[1]{\link{Table~\ref{#1}}{#1}}
\newcommand{\figureref}[1]{\link{figure~\ref{#1}}{#1}}
\newcommand{\Figureref}[1]{\link{Figure~\ref{#1}}{#1}}

%%% A Prefatory list of the font conventions:
\newcommand{\listconventions} {
\section{Conventions}
\label{pref,conventions}

The \indn{typographic conventions} shown in
Table~\ref{typo-conventions} are followed in this guide and all the
other DODS documentation.

\begin{table}[htbp]
  \begin{center}
  \caption{Typographic Conventions}
  \label{typo-conventions}
  \begin{tabular}{|c|p{3in}|} \hline
    \lit{Literal text}  &  
         Typed by the computer, or in a code listing.\\ \hline
    \inp{User input}    &  
         Type this precisely as written.\\ \hline
    \var{Variables}     &   
         Used as a place holder for a user-specified or variable
         value. Choose an appropriate value and use that in place.\\
         \hline 
    \but{Button Text}\texonly{\rule{0pt}{2.5ex}}   
        &  Used to indicate text on a GUI button.\\ 
         \hline 
    \pdmenu{Menu Name}    &  This is the name of a GUI menu.\\ \hline 
  \end{tabular}
  \end{center}
\end{table}

When referring to a button in a menu, we will often use the notation:
\but{Menu,Button}. For example, \but{Options,Colors,Foreground} would
indicate the \but{Foreground} button in the \pdmenu{Colors} menu,
selected under the \pdmenu{Options} menu.
 }


%%% Local Variables: 
%%% mode: latex
%%% TeX-master: t
%%% End: 

%
% These are html links which are used often enough in writing about DODS to
% merit an input file.
% jhrg. 4/17/94
%
% File rationalized and updated while writing the DODS User
% Guide. Also includes other useful abbreviations.
% tomfool 3/15/96
%
% Moved to dods-def.tex so I can remove links to documents that no
% longer reflect reality.
% tomfool 2/13/98
%
% $Id$
%
% Make sure to include layout.tex *before* using this file.

%% NOTE NOTE NOTE NOTE NOTE NOTE NOTE NOTE NOTE NOTE NOTE NOTE NOTE NOTE 
%%
%% If this file causes problems when running latex, you may have to edit your
%% texmf.cnf file. Here's a meesage from Tom:
%% > Are these references that use relative addresses (like
%% > ../boiler/blah.tex)?  If they are, you should look for the texmf.cnf
%% > file.  (It's often at /usr/share/texmf/web2c/texmf.cnf, and look for
%% > the openout_any parameter.  Check there anyway; there were some recent
%% > (i.e. in the early '90's) security fixes to TeX.
%%
%%%%%%%%%%%%%%%%%%%%%%%%%%%%%%%%%%%%%%%%%%%%%%%%%%%%%%%%%%%%%%%%%%%%%%%%%%%

%%% These are some DODS-specific convenience commands.
\newcommand{\DODSroot}{\lit{\$DODS\_ROOT}}     
% $

\newcommand{\opendap}{OPeNDAP\xspace}

%%% The OPD books and reference material
\newcommand{\OPDDoc}{http://opendap.org/support/docs.html}
\newcommand{\DODSDoc}{http://opendap.org/support/docs.html}

% \newcommand{\OPDDoc}{http://www.unidata.ucar.edu/packages/dods}
% \newcommand{\DODSDoc}{http://www.unidata.ucar.edu/packages/dods}

\newcommand{\OPDhomeUrl}%
  {http://opendap.org}
\newcommand{\OPDexampleUrl}%
  {BROKEN--FIX ME!}%\OPDDoc/examples}
% \newcommand{\OPDftpUrl}%
%  {ftp://dods.gso.uri.edu/pub/dods}
\newcommand{\OPDftpUrl}%
  {ftp://ftp.unidata.ucar.edu/pub/opendap/}
\newcommand{\OPDuserUrl}%
  {\OPDhomeUrl/user/guide-html/}
\newcommand{\OPDmguiUrl}%
  {\l/user/mgui-html/}
\newcommand{\OPDapiUrl}%
  {\OPDhomeUrl/api/pguide-html/}
\newcommand{\OPDapirefUrl}%
  {\OPDhomeUrl/api/pref/html/}
\newcommand{\OPDffUrl}%
  {\OPDhomeUrl/user/servers/dff-html/}
\newcommand{\OPDquickUrl}%
  {\OPDhomeUrl/user/quick-html/}
\newcommand{\OPDinstallUrl}%
  {\OPDhomeUrl/server/install-html}%
\newcommand{\OPDregexUrl}%
  {\OPDhomeUrl/user/regex-html}%
\newcommand{\OPDjavaUrl}%
  {\OPDhomeUrl/home/swJava1.1/}
\newcommand{\OPDwclientUrl}%
  {\OPDhomeUrl/api/wc-html/}
\newcommand{\OPDwserverUrl}%
  {\OPDhomeUrl/api/ws-html/}
\newcommand{\OPDaggUrl}%
  {\OPDhomeUrl/server/agg-html/}

\newcommand{\OPDhome}{\xlink{OPeNDAP Home page}{\OPDhomeUrl}}
\newcommand{\OPDjava}{\xlink{OPeNDAP Java home page}{\OPDjavaUrl}}
\newcommand{\OPDexamples}{\xlink{OPeNDAP examples page}{\OPDexampleUrl}}
\newcommand{\OPDftp}{\xlink{OPeNDAP ftp site}{\OPDftpUrl}}
%% Book titles do *not* contain the article.
\newcommand{\OPDuser}[1][]{\xlink%
  {\cit{OPeNDAP User Guide}}{\OPDuserUrl{}#1}}
\newcommand{\OPDmgui}{\xlink%
  {\cit{OPeNDAP Matlab GUI}}{\OPDmguiUrl}}
\newcommand{\OPDapi}{\xlink%
  {\cit{OPeNDAP Toolkit Programmer's Guide}}{\OPDapiUrl}}
\newcommand{\OPDapiref}{\xlink%
  {\cit{OPeNDAP Toolkit Reference}}{\OPDapirefUrl}}
\newcommand{\OPDffbook}{\xlink%
  {\cit{OPeNDAP Freeform ND Server Manual}}{\OPDffUrl}}
\newcommand{\OPDquick}{\xlink%
  {\cit{OPeNDAP Quick Start Guide}}{\OPDquickUrl}}
\newcommand{\OPDinstall}{\xlink%
  {\cit{OPeNDAP Server Installation Guide}}{\OPDinstallUrl}}
\newcommand{\OPDregex}{\xlink%
  {\cit{Introduction to Regular Expressions}}{\OPDregexUrl}}
\newcommand{\OPDagg}{\xlink%
  {\cit{OPeNDAP Aggregation Server Guide}}{\OPDaggUrl}}
\newcommand{\OPDwclient}{\xlink%
  {\cit{Writing an OPeNDAP Client}}{\OPDwclientUrl}}
\newcommand{\OPDwserver}{\xlink%
  {\cit{Writing an OPeNDAP Server}}{\OPDwclientUrl}}

\newcommand{\OPDffs}{OPeNDAP Freeform ND Server}

%%% Other DODS links.
\newcommand{\homepage}% For hyperlatex
  {\OPDDoc/}
\newcommand{\OPDsupport}{\xlink{support@unidata.ucar.edu}{mailto:support@unidata.ucar.edu}}
\newcommand{\DODSsupport}{\xlink{support@unidata.ucar.edu}{mailto:support@unidata.ucar.edu}}
\newcommand{\DODS}{\xlink{Distributed Oceanographic Data System}{\OPDhomeUrl}}
\newcommand{\OPD}{\xlink{Open Source Project for a Data Access Protocol}{\OPDhomeUrl}}
\newcommand{\OPDtechList}{\xlink{OPeNDAP Mailing Lists}{\OPDhomeUrl/mailLists/}}

%%% DODS versions
%% This has been removed.  Documents should not have an automatic
%% version number, because then it appears as if they have been
%% updated when they haven't.  Put the relevant version number to
%% whatever software is being described into each document's preface. 

\newcommand{\ifh}{WWW Interface}

% external refs for DODS documents

\newcommand{\CGI}{\xlink{Common Gateway Interface}
  {http://hoohoo.ncsa.uiuc.edu/cgi/overview.html}}

\newcommand{\MIME}{\xlink{Multipurpose Internet Mail Extensions}
  {http://www.cis.ohio-state.edu/htbin/rfc/rfc1590.html}}

\newcommand{\netcdf}{\xlink{NetCDF}
  {http://www.unidata.ucar.edu/packages/netcdf/guide.txn_toc.html}}

\newcommand{\JGOFS}{\xlink{Joint Geophysical Ocean Flux Study}
  {http://www1.whoi.edu/jgofs.html}}

\newcommand{\jgofs}{\xlink{JGOFS}
  {http://www1.whoi.edu/jgofs.html}}

\newcommand{\hdf}{\xlink{HDF}
  {http://www.ncsa.uiuc.edu/SDG/Software/HDF/HDFIntro.html}}

\newcommand{\ffnd}{FreeForm ND}

\newcommand{\Cpp}{\texorhtml
  {{\rm {\small C}\raise.5ex\hbox{\footnotesize ++}}}
  {C\htmlsym{##43}\htmlsym{##43}}}

% Commands

% Use pdflink instead. jhrg 8/4/2006
% \newcommand{\pslink}[1]{\small
% \begin{quote}
%   A \xlink{PDF version}{#1} of this document is available.
% \end{quote}
% \normalsize
% }

% Use the copy of this in dods-paper.hlx/cls or cut and paste this on
% a document-by-document basis. This version conflicts with the
% version in the class. jhrg 8/4/2006
% \newcommand{\pdflink}[1]{\small
% \begin{quote}
%   A \xlink{PDF version}{#1} of this document is available.
% \end{quote}
% \normalsize
% }

%%%%%%%%%%%%%% DODS macros
%
% These are here so that older latex files will compile. Someday remove these
% and fix the files. 04/13/04 jhrg

\newcommand{\DODShomeUrl}%
  {\OPDhomeUrl}
\newcommand{\DODSexampleUrl}%
  {\OPDDoc/examples}
% \newcommand{\OPDftpUrl}%
%  {ftp://dods.gso.uri.edu/pub/dods}
\newcommand{\DODSftpUrl}%
  {ftp://ftp.unidata.ucar.edu/pub/opendap/}
\newcommand{\DODSuserUrl}%
  {\OPDhomeUrl/user/guide-html/}
\newcommand{\DODSmguiUrl}%
  {\OPDhomeUrl/user/mgui-html/}
\newcommand{\DODSapiUrl}%
  {\OPDhomeUrl/api/pguide-html/}
\newcommand{\DODSapirefUrl}%
  {\OPDhomeUrl/api/pref/html/}
\newcommand{\DODSffUrl}%
  {\OPDhomeUrl/user/servers/dff-html/}
\newcommand{\DODSquickUrl}%
  {\OPDhomeUrl/user/quick-html/}
\newcommand{\DODSinstallUrl}%
  {\OPDhomeUrl/server/install-html}%
\newcommand{\DODSregexUrl}%
  {\OPDhomeUrl/user/regex-html}%
\newcommand{\DODSjavaUrl}%
  {\OPDhomeUrl/home/swJava1.1/}
\newcommand{\DODSwclientUrl}%
  {\OPDhomeUrl/api/wc-html/}
\newcommand{\DODSwserverUrl}%
  {\OPDhomeUrl/api/ws-html/}
\newcommand{\DODSaggUrl}%
  {\OPDhomeUrl/server/agg-html/}

\newcommand{\DODShome}{\xlink{OPeNDAP Home page}{\DODShomeUrl}}
\newcommand{\DODSjava}{\xlink{OPeNDAP Java home page}{\DODSjavaUrl}}
\newcommand{\DODSexamples}{\xlink{OPeNDAP examples page}{\DODSexampleUrl}}
\newcommand{\DODSftp}{\xlink{OPeNDAP ftp site}{\DODSftpUrl}}
\newcommand{\DODSuser}[1][]{\xlink%
  {\cit{The OPeNDAP User Guide}}{\DODSuserUrl{}#1}}
\newcommand{\DODSmgui}{\xlink%
  {\cit{The OPeNDAP Matlab GUI}}{\DODSmguiUrl}}
\newcommand{\DODSapi}{\xlink%
  {\cit{The DODS Toolkit Programmer's Guide}}{\DODSapiUrl}}
\newcommand{\DODSapiref}{\xlink%
  {\cit{The DODS Toolkit Reference}}{\DODSapirefUrl}}
\newcommand{\DODSffbook}{\xlink%
  {\cit{The DODS Freeform ND Server Manual}}{\DODSffUrl}}
\newcommand{\DODSquick}{\xlink%
  {\cit{The DODS Quick Start Guide}}{\DODSquickUrl}}
\newcommand{\DODSinstall}{\xlink%
  {\cit{The DODS Server Installation Guide}}{\DODSinstallUrl}}
\newcommand{\DODSregex}{\xlink%
  {\cit{Introduction to Regular Expressions}}{\DODSregexUrl}}
\newcommand{\DODSagg}{\xlink%
  {\cit{OPeNDAP Aggregation Server Guide}}{\DODSaggUrl}}
\newcommand{\DODSwclient}{\xlink%
  {\cit{Writing an OPeNDAP Client}}{\DODSwclientUrl}}
\newcommand{\DODSwserver}{\xlink%
  {\cit{Writing an OPeNDAP Server}}{\DODSwclientUrl}}

\newcommand{\DODSffs}{DODS Freeform ND Server}

% $Log: dods-def.tex,v $
% Revision 1.24  2004/12/21 22:30:04  jimg
% Fixed pslink; Added pdflink.
%
% Revision 1.23  2004/12/14 05:19:17  tomfool
% restored fix to pslink
%
% Revision 1.22  2004/12/09 21:01:58  tomfool
% excised test.dods.org
%
% Revision 1.21  2004/12/09 18:50:21  tomfool
% de-dodsifying
%
% Revision 1.15  2004/04/24 21:37:22  jimg
% I added every directory in preparation for adding everyting. This is
% part of getting the opendap web pages going...
%
% Revision 1.14  2004/02/12 16:05:50  jimg
% Moved the log to the end of the file.
%
% Revision 1.13  2004/01/16 18:05:31  jimg
% Added a note from Tom about setting texmf.cnf to allow \include to process
% files with ../ in their pathnames. You can also change the include to input,
% but I think include may offer some advantages for bigger/complex things like
% the Guides.
%
% Revision 1.12  2003/12/28 21:48:22  tom
% added newer books
%
% Revision 1.11  2003/12/08 19:04:43  tom
% little adjustments for DODS->opendap
%
% Revision 1.10  2003/12/08 18:53:30  tom
% DODS->OPeNDAP
%
% Revision 1.9  2002/07/15 17:49:55  tom
% added \DODSDoc
%
% Revision 1.8  2001/05/04 15:07:45  tom
% fixed pslink to include pdf files
%
% Revision 1.7  2001/02/19 20:39:13  tom
% added links to the new regex intro.
%
% Revision 1.6  2000/03/23 18:27:52  tom
% added abbreviations
%
% Revision 1.5  1999/07/01 16:00:19  tom
% added a couple of web page references
%
% Revision 1.4  1999/05/25 20:49:34  tom
% changed version numbers to 3.0
%
% Revision 1.3  1999/02/04 17:27:08  tom
% adjusted for hyperlatex and dods-book.cls
%
%

%%% Local Variables: 
%%% mode: latex
%%% TeX-master: t
%%% TeX-master: t
%%% End: 


\htmldirectory{regex-html}
\htmlname{regex}
\htmltitle{Introduction to Regular Expressions}
\htmladdress{Tom Sgouros, \rcsInfoDate}
\htmlcss{/resources/dods-book.css}
%\W\renewcommand{\HlxIcons}[1]{http://top.gso.uri.edu/icons.n/}
\setcounter{htmldepth}{2}
\T\renewcommand{\b}[1]{\bgroup\tt\char '134 \egroup\lit{#1}}
%% I don't see where \xmlsym is used here and this is causing a problem with
%% hyperlatex. 04/14/04 jhrg
%% \W\renewcommand{\b}[1]{\lit{\xmlsym{##92}#1}}

%\makeindex

\begin{document}
%------------------------------------front matter
\title{Introduction to Regular Expressions\\\DOCversion}
\author{Tom Sgouros}
\date{\rcsInfoDate}
\pagenumbering{arabic}
\maketitle

%\copyrightmatter

\W\pslink{http://www.opendap.org/pdf/regex.pdf}

\tableofcontents
%\listoffigures
%\listoftables

\clearemptydoublepage

\T\chapter{Beginning Regular Expresions}

Regular Expressions are an advanced pattern-matching languages that
are a powerful feature of many modern computer programs.  Sadly, there
are almost as many varieties of regular expressions as there are
computer programs that use them.  This (very) brief introduction to
regular expressions is about the species found in the underbrush of
the Perl jungle.  Use the real Perl documentation for the definitive
guide, and a better introduction.  Use this guide if you're lazy, and
don't want to look away from your computer screen.

Some of the material here was adapted from Marc Meurrens's tutorial.
See \xlink{http://www.meurrens.org}{http://www.meurrens.org}.

\texorhtml{}{\htmlmenu{4}\chapter{Beginning Regular Expressions}}

A regular expression is simply an expression of characters used to test
whether some other set of characters matches some pattern described by
the expression.  Since patterns can be very complex, regular
expressions can also be very complex.  The vast majority of uses will
be for fairly simple applications, and the sections that follow
provide an introduction to those fairly straightforward uses.  Do note
that even with fairly simple constructs, regular expressions can be
constructed to match some very subtle patterns.

\section{The Simple Version}

In its simplest form, a regular expression is just a sequence of
ordinary characters, such as a word or phrase to search for.  For
example,

\begin{description}
\item[\lit{gauss}] would match any character string with the substring
  \lit{gauss}.  Thus, strings with \lit{gauss}, \lit{gaussian} or
  \lit{degauss} would all be matched, as would a string containing the
  phrases \lit{de-gauss the monitor} or \lit{gaussian elimination}.
\item[\lit{carbon}] Finds any string containing \lit{carbon}, such as
  \lit{carbonization} or \lit{hydrocarbons} or \lit{carbon-based life
    forms}.
\item[\lit{hydro}] Matches strings \lit{hydro}, \lit{hydrogen} or
  \lit{hydrodynamics}.
\item[\lit{top ten}] Spaces can be part of the regular expression.
  This would match \lit{top ten}, but also \lit{stop tension}.
\end{description}

There are several characters with special meanings.  These can't be
used in simple expressions like the ones above without being preceded
by a backslash (\b{}).  For example, what if you wanted to search for
a string containing a period?  Suppose we wished to search for
references to pi.  The following regular expression would not work:

\begin{vcode}{ib}
3.14  
\end{vcode}

This would indeed match \lit{3.14}, but it would also match
\lit{3514}, \lit{3f14}, or even \lit{3+14}.  In short, any string of
the form \lit{3x14}, where \lit{x} is any character, would be matched
by the regular expression above.

To get around this, you use the backslash character to ``quote'' the
period, to indicate that it is to be taken literally.  Thus, to search
for the string \lit{3.14}, we would use:

\begin{vcode}{ib}
3\.14
\end{vcode}

In general, whenever the backslash is placed before a special character,
the searcher treats the special character literally rather than invoking
its special meaning.

(Unfortunately, the backslash is used for other things besides
quoting.  Many ``normal'' characters take on special meanings when
preceded by a backslash.  The rule of thumb is that quoting a special
character turns it into a normal character, and quoting a normal
character \emph{may} turn it into a special character.  It's not very
satisfying as a rule of thumb, but it's the best available.)

These are the special characters:

\begin{vcode}
\ | ( ) [ ] { } ^ $ * + ? .
\end{vcode}
%$


\section{Difficult Characters}
\label{difficult}

Certain characters that are difficult to type have special
abbreviations\texorhtml{ as in \tableref{tab:special}}{:}

\begin{table}[h]
  \begin{center}
    \begin{tabular}[c]{|c|l|} \hline
      \b{a} & alarm (beep) \\ \hline
      \b{n} & newline \\ \hline
      \b{r} & carriage return \\ \hline
      \b{t} & tab \\ \hline
      \b{f} & formfeed \\ \hline
      \b{e} & escape \\ \hline
    \end{tabular}
    \caption{Special Character Abbreviations}
    \label{tab:special}
  \end{center}
\end{table}

You can use these abbreviations to match character strings that
contain these.  So, for example, the pattern \lit{a\b{n}b} would
match:

\begin{vcode}{ib}
a
b
\end{vcode}

Abbreviations like this, as well as any other character that matches
something more complicated than itself, are commonly called regular
expression \new{metacharacters}.  But don't let that bother you.  It's
just a name.

There are also a bunch of special abbreviations, or metacharacters,
for classes of characters (but also read \chapterref{classes}).
\texorhtml{These are listed in \tableref{tab:class-abbrevs}.}{}

\begin{table}[h]
  \begin{center}
    \begin{tabular}[c]{|c|l|} \hline
      \lit{.} & matches any character except newline \\ \hline
      \b{d} & a digit (anything except the characters
      0,1,2,3,4,5,6,7,8 or 9)\\ \hline
      \b{D} & not a digit \\ \hline
      \b{w} & a letter or number (or \lit{\_})\\ \hline
      \b{W} & not a letter or number (or \lit{\_})\\ \hline
      \b{s} & white space (space, tab, newline, formfeed, carriage
      return) \\ \hline
      \b{S} & not white space\\ \hline
    \end{tabular}
    \caption{Special Character Classes}
    \label{tab:class-abbrevs}
  \end{center}
\end{table}

Here are some examples.  The regular expression:

\begin{vcode}{ib}
a.z  
\end{vcode}

Will match any three-character sequence that starts with \lit{a} and
ends with \lit{z}, including \lit{a.z}, \lit{a_z}, \lit{abz},
\lit{a8z}, and so on.  The middle character usually can't be a newline
(\b{n}), although Perl contains ways to avoid this restriction.  Here
are more examples:

\begin{vcode}{ib}
2,\d-Dimethylbutane
\end{vcode}

would match \lit{2,2-Dimethylbutane}, \lit{2,3-Dimethylbutane} and so forth.
Similarly,

\begin{vcode}{ib}
1\.\d\d\d\d\d  
\end{vcode}

would match any six-digit floating-point number from \lit{1.00000} to
\lit{1.99999} inclusive.

The Capital version, \b{D}, is used to match anything except a digit,
so:

\begin{vcode}{ib}
a\Dz  
\end{vcode}

would match \lit{abz}, \lit{aTz} or \lit{a`z}, but would \emph{not}
match \lit{a2z}, \lit{a5z} or \lit{a9z}. 

There are a few other abbreviations we'll encounter.  In general, the
upper-case version matches the converse of the lower-case version.

Here are two more examples:

\begin{vcode}{ib}
a\wz
\end{vcode}

would match \lit{abz}, \lit{aTz}, \lit{a5z}, \lit{a\_z}, or any
three-character string starting with \lit{a}, ending with \lit{z},
and whose second character was either a letter (upper- or lower-case),
a number, or the underscore.  Similarly,

\begin{vcode}{ib}
a\Wz  
\end{vcode}

would not match \lit{abz}, \lit{aTz}, \lit{a5z}, or \lit{a\_z}.  It
would match \lit{a`z}, \lit{a:z}, \lit{a?z} or any three-character
string starting with \lit{a} and ending with \lit{z} and whose
second character was not a letter, number, or underscore.

\section{Multiple Matches}

If you want to match more than one of something, you can use one of
the regular expression \new{quantifiers} (also called \new{closures}).
These simply modify the preceding characters to say how many of these
should be matched.  \texorhtml{The available forms are listed in
  \tableref{tab:multiple}}{Here are the available forms:} 

\begin{table}[h]
  \begin{center}
    \begin{tabular}[c]{|c|l|} \hline
      \lit{*} & Zero or more\\ \hline
      \lit{+} & One or more\\ \hline
      \lit{?} & Zero or one\\ \hline
      \lit{\{2\}} & Two exactly\\ \hline
      \lit{\{2,\}} & Two or more\\ \hline
      \lit{\{2,5\}} & At least two but no more than five\\ \hline
    \end{tabular}
    \caption{Multiple Match Quantifiers}
    \label{tab:multiple}
  \end{center}
\end{table}

Examples:

\begin{description}
\item[\lit{m?ethane}] would match either \lit{ethane} or
  \lit{methane}.  

\item[\lit{comm?a}] would match either \lit{coma} or \lit{comma}.
  
\item[\lit{ab*c}] would match \lit{ac}, \lit{abc}, \lit{abbc},
  \lit{abbbc}, \lit{abbbbbbbbc}, and any string that starts with an
  \lit{a}, is followed by a sequence of \lit{b}'s, and ends with a
  \lit{c}.

\item[\lit{ab+c}] would \emph{not} match \lit{ac}, but it \emph{would}
  match \lit{abc}, \lit{abbc}, \lit{abbbc}, \lit{abbbbbbbbc} and so
  on.

\item[\lit{ab\{3\}c}] would match \lit{abbbc}, and nothing else.
  
\item[\lit{ab\{3,\}c}] would match \lit{abbbc}, \lit{abbbbc},
  \lit{abbbbbc}, and so on.

\item[\lit{ab\{3,5\}c}] would match \lit{abbbc}, \lit{abbbbc},
  \lit{abbbbbc}, and nothing else.
  
\item[\lit{cyclo.*ane}] would match \lit{cyclodecane},
  \lit{cyclohexane} and even \lit{cyclones drive me insane}, and any
  string that starts with \lit{cyclo}, is followed by an arbitrary
  string, and ends with \lit{ane}.  Note that the null string will be
  matched by the period-star pair; thus, \lit{cycloane} would be also
  match.
  
\item[\lit{cyclo...ane}] would match \lit{cyclodecane} and
  \lit{cyclohexane}, but would not match \lit{cyclones drive me
    insane.}  Only strings eleven characters long which start with
  \lit{cyclo} and end with \lit{ane} will be matched. (Note that
  \lit{cyclopentane} would not be matched, however, since cyclopentane
  has twelve characters, not eleven.)

\end{description}

Usually, these quantifiers match the largest number of multiples they
are allowed to.  You can change this so that a pattern matches the
smallest number possible. If you know enough about pattern matching to
want to change this behavior, you have graduated from this tutorial,
and need to check out the documentation listed in
\chapterref{elsewhere}.

\section{Grouping Strings and Alternate Patterns}

If you want to match multiple copies of groups of letters, you can
group them with parentheses, so that \lit{(apple)+} means \lit{apple},
and also \lit{appleapple}, and \lit{appleappleapple} and so on.

You can also provide matching alternatives by using the \lit{|}
character inside a group.  For example, \lit{(TEXT|text)} matches
\lit{TEXT} or \lit{text}, but not \lit{Text} or \lit{tExt}, and so
on.  To take care of all combinations of upper and lower case, you
could do this: \lit{(T|t)(E|e)(X|x)(T|t)}.  There are a variety of
other ways to do this, as you will see if you read on.


\section{Character Classes}
\label{classes}

In \tableref{tab:class-abbrevs}, we saw a list of characters that
stand for entire classes of characters.  If these classes aren't right
for you, you can create new classes with the \lit{[]} characters, and
the \lit{\^{}} character.  Any group of characters contained in square
brackets is interpreted as defining a class of characters.  So:

\begin{vcode}{ib}
a[bz]c
\end{vcode}

matches \lit{abc} and \lit{azc}, and nothing else.  You can put as
many characters as you like in the brackets, and indicate ranges with
the hyphen (\lit{-}).  (If you want to include a hyphen in the class,
list it first.)

\begin{description}
\item[\lit{a[bhi:'9]c}] matches \lit{abc}, \lit{ahc}, \lit{aic},
  \lit{a:c}, \lit{a'c}, and \lit{a9c}.

\item[\lit{a[4-7]c}] matches \lit{a4c}, \lit{a5c}, \lit{a6c}, and
  \lit{a7c}. 
  
\item[\lit{a[-0-9]c}] matches \lit{a-c}, \lit{a0c}, \lit{a1c}, and so
  on to \lit{a9c}.

\item[\lit{a[4-7]*c}] matches \lit{ac}, \lit{a4c}, \lit{a45c}, and any
  sequence that starts with \lit{a}, ends with \lit{c}, and has any
  combination of \lit{4}, \lit{5}, \lit{6}, or \lit{7} in between.
\end{description}

A character class can appear anywhere in a regular expression that a
regular character can appear.

You can also negate a class by including the \lit{\^{}} sign at the
start of the string.  The class \lit{[\^{}0-9]} includes everything
except digits, and is equivalent to \b{D}.

\begin{description}
\item[\lit{[\^{}abc]*}] Matches any string of characters that does not
  include \lit{a}, \lit{b}, or \lit{c}.

\item[\lit{[\^{}\b{(}\b{)}]}] Matches any string that doesn't contain
  parentheses.  Note that we had to quote the parentheses with the
  backslash character. 
\end{description}

\subsection{Special Classes}

Typically, the story doesn't end there.  There are several pre-defined
special character classes you can use.  Use them in class brackets
like this: \lit{[:alpha:]}.  You can negate these classes by putting a
\lit{\^{}} in front, like this: \lit{[:\^{}alpha:]}.

These classes may be useful to you if you want your patterns to be
independent of the locale in which they are used.  That is, the
definition of a ``printable character'' is different depending on
locales.  What is an unprintable control code in one locale may be an
extended character set accented character in another.

\texorhtml{\Tableref{tab:special-class} lists all the special classes
available.}{The available special classes are these:}

\begin{table}[h]
  \begin{center}
    \begin{tabular}[c]{|c|p{3in}|} \hline
      \lit{[:alpha:]} &  alphabetic characters \\ \hline
      \lit{[:alnum:]} &  alphanumeric characters \\ \hline
      \lit{[:ascii:]} &  7-bit ascii character \\ \hline
      \lit{[:cntrl:]} &  unprintable control character \\ \hline
      \lit{[:digit:]} &  a number (\lit{[0-9]} or \lit{\b{d}}) \\ \hline 
      \lit{[:graph:]} &  printable (a combination of \lit{alnum} and
                         \lit{punct})  \\ \hline
      \lit{[:lower:]} &  lower case letter \\ \hline
      \lit{[:print:]} &  printable (a combination of \lit{alnum} and
                         \lit{punct}, and the space character) \\ \hline
      \lit{[:punct:]} &  punctuation (not including the space
                         character) \\ \hline
      \lit{[:space:]} &  whitespace character (space, tab, carriage-return,
                         newline, vertical tab, and form-feed) \\  \hline   
      \lit{[:upper:]} &  upper case letter \\ \hline
      \lit{[:word:]} &   a component of a word (\lit{alpha} plus
                         \lit{\_}) \\ \hline
      \lit{[:xdigit:]} & a hex digit (\lit{[0-9a-fA-f]})  \\ \hline
    \end{tabular}
    \caption{Special Character Classes}
    \label{tab:special-class}
  \end{center}
\end{table}

You can use these special classes inside other character classes, like
this: \lit{[01[:alpha:]]}, which matches any alphabetic character, as
well as zero or one.


\section{Special Characters}

The basic matching characters have now been covered.  What's left is a
few special characters you can use to modify the matches you've come
up with.  For example, if you want to match a pattern when it appears
at the end of a string (or the end of a line), you can append \lit{\$}
to it.

\texorhtml{A list of all the special characters is shown in
\tableref{tab:modifiers}}{The special characters are shown here:}

\begin{table}[h]
  \begin{center}
    \begin{tabular}[c]{|c|p{3in}|} \hline
      \lit{\$} or \lit{\b{Z}} or \lit{\b{z}} & Match the end of the
                                               string\\ \hline 
      \lit{\^{}} or \lit{\b{A}} & Match the beginning of the string\\ \hline
      \lit{\b{b}} & Match only at a word boundary \\ \hline
      \lit{\b{B}} & Don't match at a word boundary \\ \hline
      \lit{?} & Change the behavior of the preceding multiple match to
    match the minimum number of characters possible.\\ \hline
    \end{tabular}
    \caption{Match Modifiers}
    \label{tab:modifiers}
  \end{center}
\end{table}

\begin{description}

\item[\lit{.*and\$}] matches \lit{hatband} and \lit{marching band},
  but not \lit{band-aid}.

\item[\lit{\^{}[Gg].*}] matches any string beginning with a g (upper or
  lower case).

\item[\lit{\^{}hello\$}] matches the string \lit{hello} and nothing else.

\item[\lit{.*\b{b}and.*}] matches \lit{He's androgynous} and
\lit{She's an andiron}, but not \lit{It's a hatband}.

\item[\lit{.*\b{B}and.*}] matches \lit{The food is bland} or \lit{This
land is your land}, but not \lit{a one and a two}.

\end{description}

The behavior of the \lit{?} modifier is a little subtle.  Usually, a
multiple match pattern matches as many copies of its pattern as
possible.  The \lit{?} changes this behavior to match the fewest.  For
further description and examples of its use, see the references in
\chapterref{elsewhere}. 

\chapter{DODS-Specific Examples}

If you are writing a configuration file for a DODS server, the
following examples may be of some assistance.

\begin{description}

\item[\lit{.*\b{.}(NC|nc|cdf|CDF)\$}]  matches any string that ends in
  \lit{.NC}, \lit{.nc}, \lit{.cdf}, or \lit{.CDF}.  So \lit{data.nc} and
  \lit{DATA.NC} would be matched, but not \lit{data.nc.gz} or
  \lit{data.Nc}.

\item[\lit{.*\b{.}(HDF|hdf)(.Z|.gz)*\$ hdf}]  matches any string of the
  form \lit{data.HDF} or \lit{data.hdf}, that may or may not have a
  \lit{.Z} or \lit{.gz} appended to it.

\item[\lit{\^{}dods-data\b{/}.*}]  matches any file name at all in the
  \lit{dods-data} directory.

\item[\lit{\^{}dods-data\b{/}.*\b{.}(NC|nc)(.Z|.gz)*\$}] matches any
  file in the \lit{dods-data} directory, of the form \lit{data.NC} or
  \lit{data.nc}, and that may or may not have \lit{.Z} or \lit{.gz}
  appended to it.

\end{description}


\chapter{Backreferences, Extensions, and What-Have-You}
\label{elsewhere}

The regular expression syntax contains methods for referring to
earlier matches in a string, embedding comments in an expression, and
extending the semantics to include whatever you please.  These topics
are not covered here, however.

If you've gotten this far, you understand enough to read the real
documentation.  If you want to know about regular expression arcana,
there are a multitude of places you can look.  My favorites are the
\emph{Programming Perl} book by Larry Wall (O'Reilly, 2000), or the
book \emph{Mastering Regular Expressions} by Jeffrey Friedl (O'Reilly,
1997).

Enjoy.

%\printindex

\end{document}





% $Log: regex.tex,v $
% Revision 1.7  2004/07/07 22:27:43  jimg
% Changed css to dods-book.css, switched to dods-book, updated pdf link and
% switched to rcsInfoDate.
%
% Revision 1.6  2004/04/24 21:37:23  jimg
% I added every directory in preparation for adding everyting. This is
% part of getting the opendap web pages going...
%
% Revision 1.5  2004/04/14 22:09:54  jimg
% hyperlatex fix
%
% Revision 1.4  2001/06/29 15:54:15  tom
% added PDF file generation
%
% Revision 1.3  2001/02/19 20:41:01  tom
% grammar, moved links to unidata.
%
% Revision 1.2  2000/12/18 20:51:28  tom
% second revision
%





