%
% A tutorial for writing a DODS client, using the Java or C++ class
% libraries. 
%
\documentclass{dods-paper}
% Uncomment this line to get a menu in a different  frame.  This makes
% this book good for presentation.
%\W\usepackage{frames

\rcsInfo $Id$
\newcommand{\DOCversion}{Document version \rcsInfoRevision}

%
% These are html links which are used often enough in writing about DODS to
% merit an input file.
% jhrg. 4/17/94
%
% File rationalized and updated while writing the DODS User
% Guide. Also includes other useful abbreviations.
% tomfool 3/15/96
%
% Moved to dods-def.tex so I can remove links to documents that no
% longer reflect reality.
% tomfool 2/13/98
%
% $Id$
%
% Make sure to include layout.tex *before* using this file.

%% NOTE NOTE NOTE NOTE NOTE NOTE NOTE NOTE NOTE NOTE NOTE NOTE NOTE NOTE 
%%
%% If this file causes problems when running latex, you may have to edit your
%% texmf.cnf file. Here's a meesage from Tom:
%% > Are these references that use relative addresses (like
%% > ../boiler/blah.tex)?  If they are, you should look for the texmf.cnf
%% > file.  (It's often at /usr/share/texmf/web2c/texmf.cnf, and look for
%% > the openout_any parameter.  Check there anyway; there were some recent
%% > (i.e. in the early '90's) security fixes to TeX.
%%
%%%%%%%%%%%%%%%%%%%%%%%%%%%%%%%%%%%%%%%%%%%%%%%%%%%%%%%%%%%%%%%%%%%%%%%%%%%

%%% These are some DODS-specific convenience commands.
\newcommand{\DODSroot}{\lit{\$DODS\_ROOT}}     
% $

\newcommand{\opendap}{OPeNDAP\xspace}

%%% The OPD books and reference material
\newcommand{\OPDDoc}{http://opendap.org/support/docs.html}
\newcommand{\DODSDoc}{http://opendap.org/support/docs.html}

% \newcommand{\OPDDoc}{http://www.unidata.ucar.edu/packages/dods}
% \newcommand{\DODSDoc}{http://www.unidata.ucar.edu/packages/dods}

\newcommand{\OPDhomeUrl}%
  {http://opendap.org}
\newcommand{\OPDexampleUrl}%
  {BROKEN--FIX ME!}%\OPDDoc/examples}
% \newcommand{\OPDftpUrl}%
%  {ftp://dods.gso.uri.edu/pub/dods}
\newcommand{\OPDftpUrl}%
  {ftp://ftp.unidata.ucar.edu/pub/opendap/}
\newcommand{\OPDuserUrl}%
  {\OPDhomeUrl/user/guide-html/}
\newcommand{\OPDmguiUrl}%
  {\l/user/mgui-html/}
\newcommand{\OPDapiUrl}%
  {\OPDhomeUrl/api/pguide-html/}
\newcommand{\OPDapirefUrl}%
  {\OPDhomeUrl/api/pref/html/}
\newcommand{\OPDffUrl}%
  {\OPDhomeUrl/user/servers/dff-html/}
\newcommand{\OPDquickUrl}%
  {\OPDhomeUrl/user/quick-html/}
\newcommand{\OPDinstallUrl}%
  {\OPDhomeUrl/server/install-html}%
\newcommand{\OPDregexUrl}%
  {\OPDhomeUrl/user/regex-html}%
\newcommand{\OPDjavaUrl}%
  {\OPDhomeUrl/home/swJava1.1/}
\newcommand{\OPDwclientUrl}%
  {\OPDhomeUrl/api/wc-html/}
\newcommand{\OPDwserverUrl}%
  {\OPDhomeUrl/api/ws-html/}
\newcommand{\OPDaggUrl}%
  {\OPDhomeUrl/server/agg-html/}

\newcommand{\OPDhome}{\xlink{OPeNDAP Home page}{\OPDhomeUrl}}
\newcommand{\OPDjava}{\xlink{OPeNDAP Java home page}{\OPDjavaUrl}}
\newcommand{\OPDexamples}{\xlink{OPeNDAP examples page}{\OPDexampleUrl}}
\newcommand{\OPDftp}{\xlink{OPeNDAP ftp site}{\OPDftpUrl}}
%% Book titles do *not* contain the article.
\newcommand{\OPDuser}[1][]{\xlink%
  {\cit{OPeNDAP User Guide}}{\OPDuserUrl{}#1}}
\newcommand{\OPDmgui}{\xlink%
  {\cit{OPeNDAP Matlab GUI}}{\OPDmguiUrl}}
\newcommand{\OPDapi}{\xlink%
  {\cit{OPeNDAP Toolkit Programmer's Guide}}{\OPDapiUrl}}
\newcommand{\OPDapiref}{\xlink%
  {\cit{OPeNDAP Toolkit Reference}}{\OPDapirefUrl}}
\newcommand{\OPDffbook}{\xlink%
  {\cit{OPeNDAP Freeform ND Server Manual}}{\OPDffUrl}}
\newcommand{\OPDquick}{\xlink%
  {\cit{OPeNDAP Quick Start Guide}}{\OPDquickUrl}}
\newcommand{\OPDinstall}{\xlink%
  {\cit{OPeNDAP Server Installation Guide}}{\OPDinstallUrl}}
\newcommand{\OPDregex}{\xlink%
  {\cit{Introduction to Regular Expressions}}{\OPDregexUrl}}
\newcommand{\OPDagg}{\xlink%
  {\cit{OPeNDAP Aggregation Server Guide}}{\OPDaggUrl}}
\newcommand{\OPDwclient}{\xlink%
  {\cit{Writing an OPeNDAP Client}}{\OPDwclientUrl}}
\newcommand{\OPDwserver}{\xlink%
  {\cit{Writing an OPeNDAP Server}}{\OPDwclientUrl}}

\newcommand{\OPDffs}{OPeNDAP Freeform ND Server}

%%% Other DODS links.
\newcommand{\homepage}% For hyperlatex
  {\OPDDoc/}
\newcommand{\OPDsupport}{\xlink{support@unidata.ucar.edu}{mailto:support@unidata.ucar.edu}}
\newcommand{\DODSsupport}{\xlink{support@unidata.ucar.edu}{mailto:support@unidata.ucar.edu}}
\newcommand{\DODS}{\xlink{Distributed Oceanographic Data System}{\OPDhomeUrl}}
\newcommand{\OPD}{\xlink{Open Source Project for a Data Access Protocol}{\OPDhomeUrl}}
\newcommand{\OPDtechList}{\xlink{OPeNDAP Mailing Lists}{\OPDhomeUrl/mailLists/}}

%%% DODS versions
%% This has been removed.  Documents should not have an automatic
%% version number, because then it appears as if they have been
%% updated when they haven't.  Put the relevant version number to
%% whatever software is being described into each document's preface. 

\newcommand{\ifh}{WWW Interface}

% external refs for DODS documents

\newcommand{\CGI}{\xlink{Common Gateway Interface}
  {http://hoohoo.ncsa.uiuc.edu/cgi/overview.html}}

\newcommand{\MIME}{\xlink{Multipurpose Internet Mail Extensions}
  {http://www.cis.ohio-state.edu/htbin/rfc/rfc1590.html}}

\newcommand{\netcdf}{\xlink{NetCDF}
  {http://www.unidata.ucar.edu/packages/netcdf/guide.txn_toc.html}}

\newcommand{\JGOFS}{\xlink{Joint Geophysical Ocean Flux Study}
  {http://www1.whoi.edu/jgofs.html}}

\newcommand{\jgofs}{\xlink{JGOFS}
  {http://www1.whoi.edu/jgofs.html}}

\newcommand{\hdf}{\xlink{HDF}
  {http://www.ncsa.uiuc.edu/SDG/Software/HDF/HDFIntro.html}}

\newcommand{\ffnd}{FreeForm ND}

\newcommand{\Cpp}{\texorhtml
  {{\rm {\small C}\raise.5ex\hbox{\footnotesize ++}}}
  {C\htmlsym{##43}\htmlsym{##43}}}

% Commands

% Use pdflink instead. jhrg 8/4/2006
% \newcommand{\pslink}[1]{\small
% \begin{quote}
%   A \xlink{PDF version}{#1} of this document is available.
% \end{quote}
% \normalsize
% }

% Use the copy of this in dods-paper.hlx/cls or cut and paste this on
% a document-by-document basis. This version conflicts with the
% version in the class. jhrg 8/4/2006
% \newcommand{\pdflink}[1]{\small
% \begin{quote}
%   A \xlink{PDF version}{#1} of this document is available.
% \end{quote}
% \normalsize
% }

%%%%%%%%%%%%%% DODS macros
%
% These are here so that older latex files will compile. Someday remove these
% and fix the files. 04/13/04 jhrg

\newcommand{\DODShomeUrl}%
  {\OPDhomeUrl}
\newcommand{\DODSexampleUrl}%
  {\OPDDoc/examples}
% \newcommand{\OPDftpUrl}%
%  {ftp://dods.gso.uri.edu/pub/dods}
\newcommand{\DODSftpUrl}%
  {ftp://ftp.unidata.ucar.edu/pub/opendap/}
\newcommand{\DODSuserUrl}%
  {\OPDhomeUrl/user/guide-html/}
\newcommand{\DODSmguiUrl}%
  {\OPDhomeUrl/user/mgui-html/}
\newcommand{\DODSapiUrl}%
  {\OPDhomeUrl/api/pguide-html/}
\newcommand{\DODSapirefUrl}%
  {\OPDhomeUrl/api/pref/html/}
\newcommand{\DODSffUrl}%
  {\OPDhomeUrl/user/servers/dff-html/}
\newcommand{\DODSquickUrl}%
  {\OPDhomeUrl/user/quick-html/}
\newcommand{\DODSinstallUrl}%
  {\OPDhomeUrl/server/install-html}%
\newcommand{\DODSregexUrl}%
  {\OPDhomeUrl/user/regex-html}%
\newcommand{\DODSjavaUrl}%
  {\OPDhomeUrl/home/swJava1.1/}
\newcommand{\DODSwclientUrl}%
  {\OPDhomeUrl/api/wc-html/}
\newcommand{\DODSwserverUrl}%
  {\OPDhomeUrl/api/ws-html/}
\newcommand{\DODSaggUrl}%
  {\OPDhomeUrl/server/agg-html/}

\newcommand{\DODShome}{\xlink{OPeNDAP Home page}{\DODShomeUrl}}
\newcommand{\DODSjava}{\xlink{OPeNDAP Java home page}{\DODSjavaUrl}}
\newcommand{\DODSexamples}{\xlink{OPeNDAP examples page}{\DODSexampleUrl}}
\newcommand{\DODSftp}{\xlink{OPeNDAP ftp site}{\DODSftpUrl}}
\newcommand{\DODSuser}[1][]{\xlink%
  {\cit{The OPeNDAP User Guide}}{\DODSuserUrl{}#1}}
\newcommand{\DODSmgui}{\xlink%
  {\cit{The OPeNDAP Matlab GUI}}{\DODSmguiUrl}}
\newcommand{\DODSapi}{\xlink%
  {\cit{The DODS Toolkit Programmer's Guide}}{\DODSapiUrl}}
\newcommand{\DODSapiref}{\xlink%
  {\cit{The DODS Toolkit Reference}}{\DODSapirefUrl}}
\newcommand{\DODSffbook}{\xlink%
  {\cit{The DODS Freeform ND Server Manual}}{\DODSffUrl}}
\newcommand{\DODSquick}{\xlink%
  {\cit{The DODS Quick Start Guide}}{\DODSquickUrl}}
\newcommand{\DODSinstall}{\xlink%
  {\cit{The DODS Server Installation Guide}}{\DODSinstallUrl}}
\newcommand{\DODSregex}{\xlink%
  {\cit{Introduction to Regular Expressions}}{\DODSregexUrl}}
\newcommand{\DODSagg}{\xlink%
  {\cit{OPeNDAP Aggregation Server Guide}}{\DODSaggUrl}}
\newcommand{\DODSwclient}{\xlink%
  {\cit{Writing an OPeNDAP Client}}{\DODSwclientUrl}}
\newcommand{\DODSwserver}{\xlink%
  {\cit{Writing an OPeNDAP Server}}{\DODSwclientUrl}}

\newcommand{\DODSffs}{DODS Freeform ND Server}

% $Log: dods-def.tex,v $
% Revision 1.24  2004/12/21 22:30:04  jimg
% Fixed pslink; Added pdflink.
%
% Revision 1.23  2004/12/14 05:19:17  tomfool
% restored fix to pslink
%
% Revision 1.22  2004/12/09 21:01:58  tomfool
% excised test.dods.org
%
% Revision 1.21  2004/12/09 18:50:21  tomfool
% de-dodsifying
%
% Revision 1.15  2004/04/24 21:37:22  jimg
% I added every directory in preparation for adding everyting. This is
% part of getting the opendap web pages going...
%
% Revision 1.14  2004/02/12 16:05:50  jimg
% Moved the log to the end of the file.
%
% Revision 1.13  2004/01/16 18:05:31  jimg
% Added a note from Tom about setting texmf.cnf to allow \include to process
% files with ../ in their pathnames. You can also change the include to input,
% but I think include may offer some advantages for bigger/complex things like
% the Guides.
%
% Revision 1.12  2003/12/28 21:48:22  tom
% added newer books
%
% Revision 1.11  2003/12/08 19:04:43  tom
% little adjustments for DODS->opendap
%
% Revision 1.10  2003/12/08 18:53:30  tom
% DODS->OPeNDAP
%
% Revision 1.9  2002/07/15 17:49:55  tom
% added \DODSDoc
%
% Revision 1.8  2001/05/04 15:07:45  tom
% fixed pslink to include pdf files
%
% Revision 1.7  2001/02/19 20:39:13  tom
% added links to the new regex intro.
%
% Revision 1.6  2000/03/23 18:27:52  tom
% added abbreviations
%
% Revision 1.5  1999/07/01 16:00:19  tom
% added a couple of web page references
%
% Revision 1.4  1999/05/25 20:49:34  tom
% changed version numbers to 3.0
%
% Revision 1.3  1999/02/04 17:27:08  tom
% adjusted for hyperlatex and dods-book.cls
%
%

%%% Local Variables: 
%%% mode: latex
%%% TeX-master: t
%%% TeX-master: t
%%% End: 


\title{Writing an \opendap\ Client}
\htmltitle{Writing an \opendap\ Client}

\author{Dan Holloway}
\T\date{Printed: \today \\ Revision: \rcsInfoRevision}
\W\date{Revision: \rcsInfoRevision}
\htmladdress{James Gallagher <jgallagher@gso.uri.edu>, \rcsInfoDate, 
  Revision: \rcsInfoRevision}
\htmldirectory{wc-html}

\begin{document}

\maketitle

\T\tableofcontents

\section{Preface}

This tutorial describes the steps required to enable your client
application to interact with the \opendap\ Data Access Protocol by using
the Java or \Cpp\ classes provided in the \opendap\ class libraries. It also
describes the trade offs between using these toolkits and a client-library
interface such as netCDF.

The \Cpp\ and Java toolkis are class libraries which provide for direct
interaction with a remote \opendap\ server. You'll need to write some code.

If your client application currently uses one of the netCDF, GDAL or OGR
APIs, then you only need to relink your application with the \opendap-enabled
version of the API library, allowing you to skip this tutorial entirely.
  
\section{Writing your own \opendap\ client}

\subsection{Choose a language}  

The \opendap\ project provides both \xlink{\Cpp}{\OPDapiUrl} and
\xlink{Java}{\OPDjavaUrl} implementations of the DAP. Each library includes
classes that implement the various objects which comprise the DAP software
for building clients. Each also includes some extra software which simplifies
building clients by managing virtual connections, handling data caching, et
cetera. To choose one of the toolkits, several factors should be weighed.
With which of the two programming languages are you most comfortable? What
type of computer will the client run on? For client development, both Java
and \Cpp\ are supported on win32 and Unix architectures. Java is more likely to
be supported on other architectures, such as Mac.

The DAP library is middle-ware. You can use it to build a completely new
client or to add network data access to an existing program (making it a
client). If you're interested in writing a client from scratch, simply choose
the toolkit/language you feel most comfortable. If you're going to take an
existing program and transform it into a client there are several
additional factors beyond programming language you should consider.

If the program you want to DAP-enable can read netCDF files, by far the
easiest way to achieve your goal is to use our DAP-enable netCDF \emph{client
library} (CL). This piece of software works like the standard netCDF CL but
has been modified to recognize DAP URLs when they are presented in place of
local file names. The \opendap\ netCDF CL has exactly the same functional
interface as the standard netCDF library available from Unidata. That makes
it a very powerful tool because it is possible to tap into a large number of
existing programs and build \opendap\ clients. Ferret, GRrADS and IDV are
complex programs which have been \opendap-enabled using the netCDF CL.

Using the netCDF CL is not without its caveats. First netCDF does
not do a good job of representing the entire \opendap\ data model (That's not
a dig at netCDF, it's a reflection of different design goals for two pieces
of software). It's very hard to access point data using the netCDF CL,
although we're working on this problem. Second, the \Cpp\ DAP library is used
to build the C version of the netCDF CL\footnote{Unidata bundles \opendap\
access with their standard release of the Java version of netCDF.} which
means that the application programs \emph{must} be linked use a \Cpp\
compiler. Finally, it can be tricky to build a shared object (aka DLL) using
the \Cpp\ DAP library.\footnote{As the \Cpp\ ABI becomes more widely supported,
this situation should change.} If you're going to do that, please contact us
through user support or the technical discussion list.

However, while the previous points are true, the principle distinction
between using the netCDF CL and one of the DAP class libraries is that with
the netcDF CL you often do not have to write any new software at all! If you
choose to use one of the DAP libraries, you're going to have to write some
code. 

If you plan to \opendap-enable an existing application using the Java
toolkit, the application typically must provide a Java Virtual Machine (VM).
You may be able to side-step this if you feel you can rely on the host OS to
provide a JVM and/or your target program already has a JVM embedded in it.
Still you must be sure there is a way to communicate data values between the
DAP library and the application.

To use the \Cpp\ toolkit you should insure that your client application can be
built with, or link with libraries constructed using \Cpp
compilers.\footnote{As of winter 2004 we're using gcc 2.95.x, 3.2, 3.3 and MS
Visual \Cpp.} Check the current list of supported architectures on the
web-site, or contact the DODS technical support (\OPDsupport), or write to
the \OPDtechList\ mailing list for information and/or help.

This tutorial will focus on using the \Cpp\ toolkit. In
Section~\ref{sec:javadap} the main programmatic differences between the two
class libraries is listed. Both toolkits are essentially the same, with only
minor differences between them.

\subsection{Client Architecture}  

In essence, an \opendap-enabled client uses overloaded URLs to form the 
requests to a remote data server.  Through the URL, the client connects to 
the remote server and issues one of several requests.   In response to
each request, the server will return a well-defined response that the client
can use to intern the structure and content of the remote data into local
data structures, as well as retrieve any attributes associated with the
remote data.

The \Cpp\ and Java toolkits share the same characteristics, though the
names of the objects and their methods may be slightly different. If
you understand how the clients are built, it will be easy to see how
your own client application can be \opendap-enabled with minimal
effort.  The \OPDapi\ provides a complete description of the \Cpp\
Toolkit and the \OPDjavaUrl\ provides a complete description of the
Java toolkit.

\section{The DAP Architecture}

The DAP can be thought of as a layered protocol composed of MIME, HTTP, basic
objects, and complex, presentation-style, responses.

\subsection{The DAP uses HTTP which in turn uses MIME}

Clients use HTTP when they make requests of DAP servers. HTTP is a fairly
straightforward protocol (\xlink{for general information on HTTP see
http://www.w3.org/Protocols/}{http://www.w3.org/Protocols/}, \xlink{and for
the HTTP/1.1 specification, see
http://www.w3.org/Protocols/rfc2616/rfc2616.html}{http://www.w3.org/Protocols/rfc2616/rfc2616.html}).
It uses pseudo-MIME documents to encapsulate both the request sent from
client to server and the response sent back. This is important for the DAP
because the DAP uses headers in both the \xlink{
request}{http://www.unidata.ucar.edu/packages/dods/design/dap-rfc-html/dap\_16.html}
and \xlink{
response}{http://www.unidata.ucar.edu/packages/dods/design/dap-rfc-html/dap\_22.html}
documents to transfer information. However, for a programmer who intends to
write a DAP server, exactly what gets written into those headers and how it
gets written is not important. Both the \Cpp\ and Java class libraries will
handle these tasks for you (look at the \xlink{DODSFilter
class}{http://www.unidata.ucar.edu/packages/dods/api/pref-html/DODSFilter.html}
to see how). It's important to know about, however, because if you decide not
to use the libraries, or the parts that automate generating the correct MIME
documents, then your server will have to generate the correct headers itself.

\subsection{The DAP defines three objects}

To transfer information from servers to clients, the DAP uses three objects.
Whenever a client asks a server for information, it does so by requesting one
of these three objects (note: this is not strictly true, but the whole truth
will be told in just a bit. For now, assume it's true). These are the Dataset
Descriptor Structure (DDS), Dataset Attribute Structure (DAS), and Data
object (DataDDS). These are described in considerable detail in other
documentation. The Programmer's Guide contains a description of the
\xlink{DDS and DAS objects (see
http://www.unidata.ucar.edu/packages/dods/api/pguide-html/)}{http://www.unidata.ucar.edu/packages/dods/api/pguide-html/pguide\_6.html}.
These objects contain the name and types of the variables in a dataset, along
with any attributes (name-value pairs) bound to the variables. The DataDDS
contains data values. We have implemented the SDKs so that the DataDDS is a
subclass of the DDS object that adds the capacity to store values with each
variable.

\begin{tabular}[c]{lll} \\
\xlink{COADS Climatology}{http://dodsdev.gso.uri.edu/dods-3.4/nph-dods/data/nc/coads\_climatology.nc.html} &
\xlink{DAS}{http://dodsdev.gso.uri.edu/dods-3.4/nph-dods/data/nc/coads\_climatology.nc.das} &
\xlink{DDS}{http://dodsdev.gso.uri.edu/dods-3.4/nph-dods/data/nc/coads\_climatology.nc.dds} \\
\xlink{NASA Scatterometer Data}{href="http://dodsdev.gso.uri.edu/dods-3.4/nph-dods/data/hdf/S2000415.HDF.ascii?Wind\_Speed\%5B0:1:457\%5D\%5B0:1:23\%5D\%5B0:1:3\%5D,Wind\_Dir\%5B0:1:457\%5D\%5B0:1:23\%5D\%5B0:1:3\%5D} &
\xlink{DAS}{http://dodsdev.gso.uri.edu/dods-3.4/nph-dods/data/hdf/S2000415.HDF.das} &
\xlink{DDS}{http://dodsdev.gso.uri.edu/dods-3.4/nph-dods/data/hdf/S2000415.HDF.dds} \\
Catalog of AVHRR Files &
\xlink{DAS}{http://dodsdev.gso.uri.edu/dods-3.4/nph-dods/data/ff/1998-6-avhrr.dat.das} &
\xlink{DDS}{http://dodsdev.gso.uri.edu/dods-3.4/nph-dods/data/ff/1998-6-avhrr.dat.dds} \\
\xlink{AHVRR Image}{http://dodsdev.gso.uri.edu/dods-3.4/nph-dods/data/dsp/east.coast.pvu.ascii?dsp\_band\_1\%5B0:1:511\%5D\%5B0:1:511\%5D} &
\xlink{DAS}{http://dodsdev.gso.uri.edu/dods-3.4/nph-dods/data/dsp/east.coast.pvu.das} &
\xlink{DDS}{http://dodsdev.gso.uri.edu/dods-3.4/nph-dods/data/dsp/east.coast.pvu.dds} \\
\end{tabular}   

The DAP models all datasets as collections of
\xlink{variables}{http://www.unidata.ucar.edu/packages/dods/api/pguide-html/pguide\_9.html}.
The \xlink{DDS and
DataDDS}{http://www.unidata.ucar.edu/packages/dods/api/pref-html/DDS.html}
objects are containers for those variables. How you represent your dataset
using the three objects and the DAP's data type hierarchy is covered in
\begin{iftex}
``Implementing the DDS object'' in \emph{Writing an \opendap\ Server}.
\end{iftex}
\begin{ifhtml}
\xlink{Implementing the DDS object}{../ws-html/server-implementing-dds.html}
in \emph{Writing an \opendap\ Server}.
\end{ifhtml}

\subsection{The DAP also defines services}

\note{Information of the DAP services is presented here for completeness and
  because using these can help speed and simplify development of you client.
  For example, you can use the HTML and ASCII services to look at a data
  source using only a web browser. Similarly, the INFO response can be used
  to look at the attributes and variables in a given data source.}

In the previous section we said that the DAP defined three objects and
all interaction with the server involved those three objects. In fact,
the DAP also defines other responses. They are:

\begin{description}
\item[ASCII] Data can be requested in CSV form.
\item[HTML] Each server can return an HTML form that facilitates building
 URLs.
\item[INFO] Each server can combine the DDS and DAS and present that as 
HTML.
\item[VERSION] Each server must be able to respond to a request for it's
  version and the version of the DAP it implements.
\item[HELP] Each server must be able to provide a rudimentary help response.
\end{description}

In each case the server's response to these requests is built using
one or more of the basic three objects.  Here are some links to
various datasets' ASCII, HTML and INFO responses:

\begin{tabular}[c]{llll} \\
\xlink{COADS Climatology}{http://dodsdev.gso.uri.edu/dods-3.4/nph-dods/data/nc/coads\_climatology.nc.html} &
\xlink{ASCII for the SST variable}{http://dodsdev.gso.uri.edu/dods-3.4/nph-dods/data/nc/coads\_climatology.nc.asc?SST} &
\xlink{HTML}{http://dodsdev.gso.uri.edu/dods-3.4/nph-dods/data/nc/coads\_climatology.nc.html} &
\xlink{INFO}{http://dodsdev.gso.uri.edu/dods-3.4/nph-dods/data/nc/coads\_climatology.nc.info} \\
\xlink{NASA Scatterometer Data}{ref="http://localhost/dods-3.4/nph-dods/data/hdf/S2000415.HDF.ascii?Wind\_Speed\%5B0:1:457\%5D\%5B0:1:23\%5D\%5B0:1:3\%5D,Wind\_Dir\%5B0:1:457\%5D\%5B0:1:23\%5D\%5B0:1:3\%5D} &
\xlink{ASCII for wind speed and direction}{http://dodsdev.gso.uri.edu/dods-3.4/nph-dods/data/hdf/S2000415.HDF.ascii?Wind\_Speed\%5B0:1:457\%5D\%5B0:1:23\%5D\%5B0:1:3\%5D,Wind\_Dir\%5B0:1:457\%5D\%5B0:1:23\%5D\%5B0:1:3\%5D} &
\xlink{HTML}{http://dodsdev.gso.uri.edu/dods-3.4/nph-dods/data/hdf/S2000415.HDF.html} &
\xlink{INFO}{http://dodsdev.gso.uri.edu/dods-3.4/nph-dods/data/hdf/S2000415.HDF.info} \\
Catalog of AVHRR Files &
\xlink{ASCII for values within a date range}{http://dodsdev.gso.uri.edu/dods-3.4/nph-dods/data/ff/1998-6-avhrr.dat.ascii?year,day\_num,DODS\_URL&day\_num\%3C170} &
\xlink{HTML}{http://dodsdev.gso.uri.edu/dods-3.4/nph-dods/data/ff/1998-6-avhrr.dat.html} &
\xlink{INFO}{http://dodsdev.gso.uri.edu/dods-3.4/nph-dods/data/ff/1998-6-avhrr.dat.info} \\
\xlink{AHVRR Image}{http://dodsdev.gso.uri.edu/dods-3.4/nph-dods/data/dsp/east.coast.pvu.ascii?dsp\_band\_1\%5B0:1:511\%5D\%5B0:1:511\%5D} &
\xlink{ASCII for the SST}{http://dodsdev.gso.uri.edu/dods-3.4/nph-dods/data/dsp/east.coast.pvu.ascii?dsp\_band\_1\%5B0:1:511\%5D\%5B0:1:511\%5D} &
\xlink{HTML}{http://dodsdev.gso.uri.edu/dods-3.4/nph-dods/data/dsp/east.coast.pvu.html} &
\xlink{INFO}{http://dodsdev.gso.uri.edu/dods-3.4/nph-dods/data/dsp/east.coast.pvu.info} \\
\end{tabular}
  
The VERSION and HELP responses can be see by appending \lit{help} or
\lit{version} to the end of the server's base URL. For example:

\begin{tabular}[c]{l} \\
\xlink{HELP: http://dodsdev.gso.uri.edu/dods-3.4/nph-dods/help}{http://dodsdev.gso.uri.edu/dods-3.4/nph-dods/help}\\
\xlink{VERSION: http://dodsdev.gso.uri.edu/dods-3.4/nph-dods/version}{http://dodsdev.gso.uri.edu/dods-3.4/nph-dods/version}\\
\end{tabular}

\subsection{Connecting to the server}
  
To manage the connection between the client application and the remote
server, the DAP uses two objects.  The \class{Connect} class manages
one connection to either a remote data server, or a local access.  The
\class{Connections} class is used to manage a set of instances to the
class \class{Connect}.  For each data source that the client
opens, there must be exactly one instance of the \class{Connect}
class.  The \OPDapi\ provides a description for the \Cpp\ toolkit's
usage.

\section{Getting ready to write your client}
  
An \opendap-enabled client application creates a connection to the
remote server using the \class{Connect} class, and then issues
requests to the remote server through the \class{Connect} class
methods. Please refer to the \xlink{\lit{Geturl.java}}{Geturl.java.html} and
\xlink{\lit{geturl.cc}}{geturl.cc.html} sources as examples of command-line
based DAP client applications written in Java and \Cpp\, respectively.

Most of the software is boilerplate. Following are sections of the
Geturl.java client application, later a description of the same
example in \Cpp\ will be provided.  Again, please refer to the complete
source listings referenced above.

\begin{vcode}{sib}
 DConnect url = null;
 try {
    url = new DConnect(nextURL, accept_deflate);
 }
\end{vcode}

This code snippet instantiates a new instance of the \class{DConnect}
class, passing the URL referencing the remote data server, and a
boolean flag indicating that the client can accept responses from the
server which are compressed.

\begin{vcode}{sib}
 if (get_data) {
    if ((cexpr==false) && (nextURL.indexOf('?') == -1)) {
        System.err.println("Must supply a constraint expression with -D.");
        continue;
    }
    for (int j=0; j<times; j++) {
        try {
           StatusUI ui = null;
           if (gui)
             ui = new StatusWindow(nextURL);
           DataDDS dds = url.getData(expr, ui);
           processData(url, dds, verbose, dump_data, accept_deflate);
        }
        catch (DODSException e) {
          System.err.println(e);
          System.exit(1);
        }
        catch (java.io.FileNotFoundException e) {
          System.err.println(e)
          System.exit(1);
        }
        catch (Exception e) {
          System.err.println(e);
          e.printStackTrace();
          System.exit(1);
        }
    }
 }
\end{vcode}  

This compound statement block initiaties a data request to the remote
server.  The \class{DConnect} method \lit{getData} forms the data
request to the remote server by appending the string, \var{expr},
containing the constraint-expression, onto the URL used in creating
the initial \class{DConnect} to the remote site.  The parameter,
\var{ui}, provides an optional \lit{StatusWindow} object to provide
the status of the current request to the client.

\begin{vcode}{sib} 
 catch (DODSException e) {
   System.err.println(e);
   System.exit(1);
 }
 catch (java.io.FileNotFoundException e) {
   System.err.println(e);
   System.exit(1);
 }
 catch (Exception e) {
   System.err.println(e);
   e.printStackTrace();
   System.exit(1);
 }
\end{vcode}
  
Completing the try block is a series of catch blocks that catch
exceptions thrown by the DAP library code, and Java input/output and
general exceptions.


The \Cpp\ toolkit provides similar functionality as the Java toolkit
though the parameters to the individual \class{Connect} methods may vary.
The \Cpp\ client application \xlink{geturl.cc}{geturl.cc.html}
uses the similar \Cpp\ toolkit classes as the Java toolkit to implement the
Geturl client application:

\begin{vcode}{sib}
  string name = argv[i];
  Connect url(name, trace, accept_deflate);
\end{vcode}

This code fragment declares an instance of the \class{Connect} class,
passing the URL referencing the remote data server, and a boolean
flag indicating that the client can accept responses from the server
that are compressed.

\begin{vcode}{sib}
else if (get_data) {
    if (expr.empty() && name.find('?') == string::npos)
	expr = "";

    for (int j = 0; j < times; ++j) {
	DataDDS dds;
	try {
	    DBG(cerr << "URL: " << url->URL(false) << endl);
	    DBG(cerr << "CE: " << expr << endl);
	    url->request_data(dds, expr);

	    if (verbose)
		fprintf( stderr, "Server version: %s\n",
			 url->get_version().c_str() ) ; 

	    print_data(dds, print_rows);
	}
	catch (Error &e) {
	    e.display_message();
	    continue;
	}
    }
}
\end{vcode}

This compound statement block initiaties a data request from the remote
server. The \class{Connect} method \lit{request\_data} forms the data request
to the remote server by appending the string, \var{expr}, containing the
constraint-expression, onto the URL used in creating the initial
\class{Connect} to the remote site.

\begin{vcode}{sib}
catch (Error &e) {
    e.display_message();
    continue
}
\end{vcode}
 
Completing the try block is a catch block that picks up exceptions thrown by
the DAP library code. The DAP \Cpp\ library throws several types of exceptions,
the most common of which are \class{Error} and \class{InternalErr}. All of
the exceptions are either instances of \class{Error} or are specializations
of it, so catching just \class{Error} will get everything.

\section{Subclassing the data types}

The DAP defines a data type hierarchy as the core of its data model. This
collection of data types includes scalar, vector and constructor types. Most
of the types are available in all modern programming languages with the
exceptions being \lit{Url}, \lit{Sequence} and \lit{Grid}. In the DAP
library, the class \lit{BaseType} is the root of the data type tree.

\subsection{A quick review of the data types supported by the DAP}

The DAP supports the common scalar data types such as Byte, 16- and 32-bit
signed and unsigned integers, and 32- and 64-bit floating point numbers. The
DAP also supports Strings and Urls as basic scalar types. The DAP includes
Arrays of unlimited size and dimensionality. The DAP also supports three
type-constructors: \lit{Structure}, \lit{Sequence} and \lit{Grid}. A
\lit{Structure} on the DAP mimics a struct in \lit{C}. A \lit{Sequence} is a
table-like data structure inherited from the JGOFS data system. It can be
used to hold information that might be stored in relational databases or
tables, either flat or hierarchical. The JGOFS, FreeForm and HDF servers all
use the \lit{Sequence} data type. Lastly, the \lit{Grid} data type is used to
bind an array to a group of \emph{map vectors}, single dimension arrays that
provide non-integral values for the indexes of the array. The most typical
use of a Grid is to provide latitude and longitude registration for some
georeferenced array data (e.g., a projected satellite image). The DAP does
not have a pointer data type, but in some cases the \lit{Url} data type can
be used as a pointer to variables between files. More information about the
\xlink{DAP's data type
hierarchy}{http://www.unidata.ucar.edu/packages/dods/api/pguide-html/pguide\_9.html}
is given in the Programmer's Guide.

\subsection{Creating the subclasses}

When you build a DAP client, you must create a collection of data type
subclasses.  That is each of the leaf classes in the preceding class
diagram must be subclassed by your client. This is pretty easy since a
good bit of the work is rote.

First we'll illustrate the parts that are mechanical. Here's an example from
the \Cpp\ Matlab client. The class is the Byte class. In the case of the matlab
client, this class doesn't do anything beyond the bare minimum, so it's a
good starting point:

\begin{vcode}{sib}
Byte *
NewByte(const string &n)
{
    return new ClientByte(n);
}

BaseType *
ClientByte::ptr_duplicate()
{
    return new ClientByte(*this);
}

bool
ClientByte::read(const string &)
{
  throw InternalErr(__FILE__, __LINE__, "Called unimplemented read method");
}
\end{vcode}

To create a child of any of the data type leaf classes, you must define three
methods and one function. Let's talk about the function first. The function
\lit{NewByte} is what Meyers\cite{meyers:ecpp} calls a \emph{virtual
constructor}. It's similar to a low-budget factory class (``low-budget''
because it's not a class). This function is used at various places in the DAP
library when it need to create instances of \lit{Byte} without knowing in
advance the dynamic type of the object that actually will be created. If all
this sounds a little weird, just remember that your \lit{Byte}, \lit{Int16},
..., \lit{Grid} classes --- whatever they may be called --- must all contain
an implementation of this function and each should all return a pointer to an
instance of the appropriate child class. These functions will be used by the
library to create instance of the classes you have defined when writing your
server. In this case of the example Matlab server, it's an instance of the
\lit{MATByte} class. If you look in the files for the Matlab server, you'll
see that the function \lit{NewGrid} returns a pointer to a new \lit{MATGrid},
and so on.

Second, a constructor must be implemented and should take the name of the
variable as its sole argument.

Third, your child classes should also define the \lit{ptr\_duplicate()}
method. This method returns a pointer to a new instance of an object in the
same class. Occasionally, in the DAP library, objects are declared with
pointers specified as \lit{BaseType *}. If the \lit{new} operator was used to
copy such an object, the copied object would be an instance of BaseType (the
static type of the object) not the type of the thing referenced (the dynamic
type)\footnote{This is the oft discussed phenomenon of `slicing,' see
Meyers\cite{meyers:ecpp}, Stroustrup\cite{stroustrup:cpp}, et c., for a
complete explanation.}. By using the \lit{ptr\_duplicate()} method the DAP
library is sure that when it copies an object, it's getting an instance of
the subclass defined by your server.

Unlike the case where you are subclassing the DAP variable classes to build a
server, there's no need to implement \lit{read()} when building a client. The
classes contain a default implementation od \lit{read()} that throws
InternalErr if it is ever called (which no simple client should ever
do.\footnote{If you're building a gateway, something that is both a client
  and a server, you'll need to implement \lit{read()}.}

%% The DAP Java Toolkit uses a similar mechanism to support subclassing
%% of the DAP base classes.  Following is an example from the Java Matlab
%% client application, from the source file \xlink{MatlabFactory.java}%
%% {MatlabFactory.java.html}.  In the Java DAP toolkit, for those classes
%% which are not required to be specialized for the underlying client
%% application, an instance of the base class may be returned through the
%% Factory interface.

%% \begin{vcode}{sib}
%% public class MatlabFactory implements BaseTypeFactory {
%%   //..................................
%%   /** 
%%    * Construct a new DByte.
%%    * @return the new DByte
%%    */
%%   public DByte newDByte() {
%%     return new DByte();
%%   }

%%   /**
%%    * Construct a new DByte with name n.
%%    * @param n the variable name
%%    * @return the new DByte
%%    */
%%   public DByte newDByte(String n) {
%%     return new DByte(n);
%%   }
%% \end{vcode}

%% For those classes specialized for the underlying client 
%% application, an instance to the specialized class is returned 
%% through the Factory interface.

%% \begin{vcode}{sib}
%%   //..................................
%%   /** 
%%    * Construct a new DArray.
%%    * @return the new DArray
%%    */
%%   public DArray newDArray() {
%%     return new MatlabArray();
%%   }

%%   /**
%%    * Construct a new MatlabArray with name n.
%%    * @param n the variable name
%%    * @return the new MatlabArray
%%    */
%%   public DArray newDArray(String n) {
%%     return new MatlabArray(n);
%%   }
%% \end{vcode}

\section{Accessing the DDS object}
\label{client-tut,implementing}

The Data Descriptor Structure (DDS) is a data structure used by the
DODS software to describe datasets and subsets of those datasets. The
DDS may be thought of as the declarations for the data structures that
will hold data requested by some DODS client. Part of the job of a
DODS server is to build a suitable DDS for a specific dataset and to
send it to the client. Depending on the data access API in use, this
may involve reading part of the dataset and inferring the DDS.  Other
APIs may require the server simply to read some ancillary data file
with the DDS in it.

For the client, the DDS object includes methods for reading the
persistent form of the object sent from a server. This includes
parsing the ASCII representation of the object and, possibly, reading
data received from a server into a data object.

Note that the class DDS is used to instantiate both DDS and DataDDS objects.
A DDS that is empty (contains no actual data) is used by servers to send
structural information to the client. The same DDS can be treated as a
DataDDS when data values are bound to the variables it defines.

For a complete description of the DDS layout and protocol, please
refer to \OPDuser\ and \OPDapi .

The DDS has an ASCII representation, which is what is transmitted from
a DODS server to a client. Here is the DDS representation of an entire
dataset containing a time series of worldwide grids of sea surface
temperatures:

\begin{vcode}{sib}
Dataset {
    Grid {
      ARRAY:
	 Int32 sst[time = 404][lat = 180][lon = 360];
      MAPS:
	 Float64 time[time = 404];
	 Float64 lat[lat = 180];
	 Float64 lon[lon = 360];
    } sst;
} weekly;
\end{vcode}
       
If the data request to this dataset includes a constraint expression,
the corresponding DDS might be different. For example, if the request
was only for northern hemisphere data at a specific time, the above
DDS might be modified to appear like this:

\begin{vcode}{sib}
Dataset {
    Grid {
      ARRAY:
	 Int32 sst[time = 1][lat = 90][lon = 360];
      MAPS:
	 Float64 time[time = 1];
	 Float64 lat[lat = 90];
	 Float64 lon[lon = 360];
    } sst;
} weekly;
\end{vcode}
       
The constraint has narrowed the area of interest; the range of latitude
values has been halved and there is only one time value in the returned
array. 

See \OPDuser\ , \OPDapiref\ for descriptions of the DODS data types.

Reading data from a DDS object is the heart of writing your own \opendap\
client. To integrate the information contained in the DDS, you must do two
things. First you must decide how the data type hierarchy that is part of the
DAP can be represented in your client application. Some client applications
cannot represent all possible DAP data types directly. Where possible the
client developer should strive to support as many data types as possible to
facilitate access to the wide variety of data accessible through \opendap\
servers. In practice, once you know how to map variables from the DAP into
your client application, writing code to build the DDS instance is easy.

\begin{vcode}{sib}
else if (get_dds) {
    for (int j = 0; j < times; ++j) {
	DDS dds;
	try {
	    url->request_dds(dds);
	}
	catch (Error &e) {
	    e.display_message();
	    delete url; url = 0;
	    continue;	// Goto the next URL or exit the loop.
	}

	if (verbose) {
	    fprintf( stderr, "Server version: %s\n",
			     url->get_version().c_str() ) ; 
	    fprintf( stderr, "DDS:\n" ) ;
	}

	dds.print(stdout);
    }
}
\end{vcode}

Above, the \class{Connect} method \lit{request\_dds} is called, passing a
reference to a DDS object. Following is an example from the \Cpp\ Matlab client
illustrates a simple traversal of the DDS object returned from the
\lit{connect::request\_data} method.
  
\begin{vcode}{sib}
static void
process_data(Connect &url, DDS &dds)
{
   if (verbose)
       cerr << "Server version: " << url.server_version() << endl;

   for (DDS::Vars_iter i = dds.var_begin(); i != dds.var_end(); i++) {
       BaseType *v = *i ;
       v->print_decl(cout, "", true);
       smart_newline(cout, v->type());
   }
}
\end{vcode}

In the \Cpp\ DAP classes STL iterators are used to iterate over the members
(i.e., variables) in the DDS object. The iterator \lit{i} references pointers
to the top-level \lit{BaseType} objects held by the DDS. See
Jouspurtis\cite{josuttis:cpp-stl} for information about the Standard Template
Library and STL iterators. The \OPDapi\ provides
a description for the each of the data types, and the methods available to
operate on them.

%% In the Java DAP toolkit, an \lit{Enumeration} type is used to iterate over 
%% the contents of the DDS class instance.
 
%% \begin{vcode}{sib}
%%  /*
%%   * Return an Enumeration of the variables in the dataset.
%%   */
%%  public Enumeration getVariables() {
%%     if(dds != null) 
%%         return dds.getVariables();
%%     else
%%         return null;
%%  }
%% \end{vcode}

%% The \lit{Enumeration} type can be used to iterate over
%% the contents of the DDS instance with a simple loop
%% construct:

%% \begin{vcode}{sib}
%%  private static void processData(DConnect url, DataDDS dds, boolean verbose,
%%                                   boolean dump_data, boolean compress) {

%%    Enumeration dodsVar = dds.getVariables();

%%    while (dodsVar.hasMoreElements()) {
%%       BaseType bt = (BaseType) dodsVar.nextElement();

%%       bt.printDecl(System.out);
%%       System.out.println();
%%    }
%% }
%% \end{vcode}
    
%% The first line declares an \lit{Enumeration} type variable, \lit{dodsVar} and
%% assigns the \lit{Enumeration} returned by the \class{DDS} \lit{ getVariables}
%% method. While the enumeration has values remaining, the example function
%% calls the \lit{printDecl} and \lit{printVal} methods of the underlying
%% \class{BaseType} to print out the declaration and values of each variable
%% referenced by the \lit{Enumeration} type.

\section{Accessing the DAS object}

The Data Attribute Structure (DAS) is a set of name-value pairs used to
describe the data in a particular dataset.\footnote{Often this is referred to
as the data set's \emph{meta data} or \emph{semantic meta data}.} The
name-value pairs are called the \emph{attributes}. The values may be of any
of the DODS simple data types (\lit{Byte}, \lit{Int16}, \lit{UInt16},
\lit{Int32}, \lit{UInt32}, \lit{Float32}, \lit{Float64}, \lit{String} and
\lit{URL}), and may be scalar or vector. (Note that all values are actually
stored as string data.)

A value may also consist of a set of other name-value pairs. This makes it
possible to nest collections of attributes, giving rise to a hierarchy of
attributes. The DAP uses this structure to provide information about
variables in a dataset.

In the following example of a DAS, several of the attribute collections have
names corresponding to the names of variables in a hypothetical dataset. The
attributes in that collection are said to belong to that variable. For
example, the \lit{lat} variable has an attribute units of
\lit{degrees\_north.}

\begin{vcode}{sib}
 Attributes {
    GLOBAL {
       String title "Reynolds Optimum Interpolation (OI) SST";
    }
    lat {
       String units "degrees_north";
       String long_name "Latitude";
       Float64 actual_range 89.5, -89.5;
    }
    lon {
       String units "degrees_east";
       String long_name "Longitude";
       Float64 actual_range 0.5, 359.5;
    }
    time {
       String units "days since 1-1-1 00:00:00";
       String long_name "Time";
       Float64 actual_range 726468., 729289.;
       String delta_t "0000-00-07 00:00:00";
    }
    sst {
       String long_name "Weekly Means of Sea Surface Temperature";
       Float64 actual_range -1.8, 35.09;
       String units "degC";
       Float64 add_offset 0.;
       Float64 scale_factor 0.0099999998;
       Int32 missing_value 32767;
   }
 }
\end{vcode}

Attributes may have arbitrary names, although in most datasets it is
important to choose these names so a reader will know what they
describe. In the above example, the GLOBAL attribute provides
information about the entire dataset.

Data attribute information is an important part of the the data
provided to a DODS client by a server, and the DAS is how this data is
packaged for sending (and how it is received).

An example of Attribute handling in a client application is provided in
the \lit{www-int} \Cpp\ source:

\begin{vcode}{sib}
void
LoaddodsProcessing::print_attr_table(AttrTable &at, ostream &os)
{
    for (AttrTable::Attr_iter i = at.attr_begin(); i != at.attr_end(); ++i) {
        int attr_num = at.get_attr_num(i);
        switch (at.get_attr_type(i)) {
          case Attr_container: {
              AttrTable *cont_atp = at.get_attr_table(i);
              os << "Structure" << endl << names.lookup(at.get_name(i), translate)
                 << " " << cont_atp->get_size() << endl;
              print_attr_table(*cont_atp, os);
              break;
          }

          case Attr_string:
          case Attr_url:
            if (attr_num == 1) {
                os << "String" << endl << names.lookup(at.get_name(i), translate) << endl
                   << at.get_attr(i) << endl;
            }
            else {
                os << "Array" << endl << "String " << names.lookup(at.get_name(i), translate)
                   << " 1" << endl << attr_num << endl;
                for (int j = 0; j < attr_num; ++j)
                    os << at.get_attr(i, j) << endl;

                os << endl;
            }
            break;

          // The remainder of this method's code has been elided. To see the 
          // complete method, look at the source file 
          // DODS/src/clients/ml-cmdln/LoaddodsProcessing.cc
        }
    }
}
\end{vcode}

As with the DDS, the DAS object is a container and STL iterators are used to
access its members (the attributes). There are several differences, however,
between the two containers. The DDS holds complete objects, each of which is
an instance of the class \class{BaseType}. A DAS, however, holds a collection
of attributes. Unless the attribute is itself a container for other attribute
type-name-value tuples, there is no contained object to access with methods
to run. Instead the DAS and AttrTable classes themselves provide methods that
are used to access the type, name and value of the attributes. These accessor
methods take as their arguments a STL iterator. 

For example, in the first case, the method \lit{AttrTable::get\_attr\_table()}
is used to get a pointer to an AttrTable (which is the `value' of a container
attribute). In the second case the \lit{AttrTable::get\_name()} and
\lit{AttrTable::get\_attr()} methods are used to get the name and value of
simple attributes. In each case the \lit{Attr\_iter} \lit{i} is passed to the
methods. 

%% In the Java DAP toolkit, similar to the DDS class an \lit{Enumeration}
%% type is used to iterate over the contents of the DAS and
%% \class{AttrTable} class instances.  The \OPDjavaUrl\ provides a
%% description of the Java DAP toolkits implementation of the DAS,
%% \class{AttrTable}, and \class{Attribute} classes.

%% \begin{vcode}{sib}
%%  public void writeAttributes(AttributeTable aTbl, String indent) {

%%     if(aTbl != null){

%%        Enumeration e = aTbl.getNames();

%%        while(e.hasMoreElements()){
%%           String aName = (String)e.nextElement();
%%           Attribute a = aTbl.getAttribute(aName);

%%           if (a.isContainer()) {
%%               pWrt.print(indent+aName+":\n");
%%               writeAttributes(a.getContainer(),indent+"  ");
%%           }
%%           else {
%%               pWrt.print(indent + aName + ": ");
%%               if(_Debug) { System.out.println("Getting attribute value enumeration for \""+aName +"\"...");}

%%               Enumeration es = a.getValues();
%%               if(_Debug) { System.out.println("Attribute Values enumeration: "+es);}
%%               int i = 0;
%%               while(es.hasMoreElements()){
%%                   String val = (String)es.nextElement();
%%                   if(_Debug) { System.out.println("Value " + i + ": "+val);}

%%                   pWrt.print(val);

%%                   if(es.hasMoreElements())
%%                      pWrt.print(", ");

%%                   i++;
%%               }
%%               pWrt.println("");
%%           }

%%        }
%%     }
%%  }
%% \end{vcode}

%% The preceding Java example provides the same functionality as the \Cpp\
%% example, but uses an \lit{Enumeration} type to iterate over the DAS 
%% container classes.

\section{Getting Data: Accessing the DataDDS object}

Up till now we have talked about access to a data source's meta data. The DDS
provides access to the syntactic meta data and the DAS provides semantic meta
data. Use the DataDDS to access data values held by the data source.

The \class{DataDDS} class is an extension of class \class{DDS} which 
contains the binary data values returned by the remote server.  It supports
the same methods as the class \class{DDS}, but the \lit{BaseType::buf2val()}
methods can be used to extract data held in the variable instances of
\class{BaseType}. Use the \class{DDS}, \class{BaseType} and their iterators
to access the variables. 

Use \class{Connect} to ask a remote server for a \class{DataDDS}. The
\class{Connect} provides \lit{Connect::request\_data()} to get the
\class{DataDDS}. The \lit{request\_data()} method accepts a constraint
expression which can be used to restrict the data returned by the remote
server. See \OPDuser\ for more information about constraint expression
syntax.\footnote{The details of the constraint expression syntax are covered
in The \xlink{DODS User Guide[constraint]} {\OPDuserUrl/constraint.html}.}

%% \begin{vcode}{sib}
%%  if (get_data) {
%%     if ((cexpr==false) && (nextURL.indexOf('?') == -1)) {
%%         System.err.println("Must supply a constraint expression with -D.");
%%         continue;
%%     }
%%     for (int j=0; j<times; j++) {
%%         try {
%%            StatusUI ui = null;
%%            if (gui)
%%              ui = new StatusWindow(nextURL);
%%            DataDDS dds = url.getData(expr, ui);
%%            processData(url, dds, verbose, dump_data, accept_deflate);
%%         }
%%         catch (DODSException e) {
%%           System.err.println(e);
%%           System.exit(1);
%%         }
%%         catch (java.io.FileNotFoundException e) {
%%           System.err.println(e)
%%           System.exit(1);
%%         }
%%         catch (Exception e) {
%%           System.err.println(e);
%%           e.printStackTrace();
%%           System.exit(1);
%%         }
%%     }
%%  }
%% \end{vcode}

%% The \class{Connect} method \lit getData() is passed an optional constraint-expression
%% requesting that a subsetting, or other server-side operation be performed on the
%% data before returning it to the client application.
%% The client application typically accesses the elements of the \lit{DataDDS} 
%% using the subclassed DAP data types which the client has specified.

%% In the Java DAP toolkit, an {\it Enumeration} type is used to iterate over 
%% the contents of the \class{DDS}, and \class{DataDDS} class instance. 

%% \begin{vcode}{sib}
%%  private static void processData(DConnect url, DataDDS dds, boolean verbose,
%%                                   boolean dump_data, boolean compress) {

%%    Enumeration dodsVar = dds.getVariables();

%%    while (dodsVar.hasMoreElements()) {
%%       BaseType bt = (BaseType) dodsVar.nextElement();

%%       bt.printDecl(System.out);
%%       System.out.println();
%%    }
%% }
%% \end{vcode}

%% To access the value contained in the element referenced
%% by the {\it Enumeration} type, the \class{BaseType} method
%% \lit{getValue()} returns the contents of the data types
%% data buffer.

%% \begin{vcode}{sib}
%%       DInt32 val = (DInt32) dodsVar.getValue();
%% \end{vcode}

Use iterators to traverse the \class{DataDDS} and access the individual DAP
data variable objects. The following example
prints the DAP data objects declaration, to access the binary data returned
by the server the DAP provides access methods to retrieve the data object's
buffer contents.

\begin{vcode}{sib}
static void
process_data(Connect &url, DDS *dds)
{
   if (verbose)
       cerr << "Server version: " << url.server_version() << endl;

   for (DDS::Vars_iter i = dds.var_begin(); i != dds.var_end(); i++) {
       BaseType *v = *i ;
       v->print_decl(cout, "", true);
       smart_newline(cout, v->type());
   }
}
\end{vcode}

The binary data returned by the server is stored in the \lit{\_buf} member of
each of the DAP's atomic data types. To retrieve the atomic data type's
buffer contents the \lit{BaseType::buf2val} method is used.

\begin{vcode}{sib}
n_bytes = dds->var(q)->buf2val((void **) &localVar);
\end{vcode}

%% The DAP \class{Sequence} data types can be visualized as a relational table 
%% structure consisting of rows and columns.  To access the elements
%% of the table the DAP provides accessor methods for each row and 
%% column element.  The client application uses these methods to 
%% reference the individual DAP data objects comprising the \class{Sequence}.

%% The following Java example is from the Java Matlab client
%% application and illustrates the use of the Java DAP classes
%% and methods to access elements from a \class{Sequence} data type.
 
%% \begin{vcode}{sib}
%% /**
%%  * This class takes a MatlabSequence object, and provides methods to return
%%  * the columns of the sequence as arrays of atomic types.  I wrote this before
%%  * I subclassed DSequence, so I may end up moving these functions into 
%%  * MatlabSequence and doing away with this class.
%%  *
%%  * Note: Java doesn't have any unsigned types, so the getU* functions return
%%  *       a signed variable
%%  * 
%%  */

%% class DodsSequenceProcessor extends Object {
%%     private MatlabSequence dodsSeq;

%%     public DodsSequenceProcessor(MatlabSequence seq) {
%%         dodsSeq = seq;
%%     }

%%     /** 
%%      * Get a column of DBytes from the sequence and return it as an 
%%      * array of bytes
%%      * @param name The name of the column
%%      * @return an array of bytes containing the data.
%%      */
%%     public byte[] getByte(String name) 
%%         throws NoSuchVariableException 
%%     {
%%         int numVars = dodsSeq.getRowCount();
%%         byte[] values = new byte[numVars];
%%         BaseType temp[] = null;

%%         try {
%%             temp = dodsSeq.getColumn(name);
%%         }
%%         catch(NoSuchVariableException e) {
%%             throw(e);
%%         }
        
%%         if(temp[0] instanceof DByte) {
%%             for(int i=0; i<numVars; i++) {
%%                 values[i] = ((DByte)temp[i]).getValue();
%%             }
%%         }
%%         return values;
%%     }

%%     ...
%% \end{vcode}

%% The code example illustrates accessing the values of a 
%% \class{Sequence} element, in this example of type \class{Byte}.
%% The first thing the client does is determine the number 
%% of rows in the \class{Sequence} which was returned by 
%% the server, and creates a vector of \lit{byte} of
%% that size.  The client creates an empty array of type
%% \class{BaseType} which is used to store the return values
%% from the \class{Sequence} method \lit{ getColumn(name)}.
%% Then for each \class{BaseType} element referenced in the
%% array the client accesses its buffer values with the
%% \class{BaseType} \lit{ getValue()} method.

The following \Cpp\ example is from the \Cpp\ Matlab client
application and illustrates the use of the \Cpp\ DAP classes
and methods to access elements from a \class{Sequence} data type.

\begin{vcode}{sib}
void 
ClientSequence::print_one_row(ostream &os, int row, string space,
                              bool print_row_num)
{

    const int elements = element_count();
    for (int j = 0; j < elements; ++j) {
        BaseType *bt_ptr = var_value(row, j);

        if (bt_ptr) {           // data
            bt_ptr->print_val(os, space, true);
        }
    }

}

void 
ClientSequence::print_val_by_rows(ostream &os, string space,
                                  bool print_decl_p,
                                  bool print_row_numners)
{
    const int rows = number_of_rows();
    for (int i = 0; i < rows; ++i) {
        print_one_row(os, i, space, false);
    }
}
\end{vcode}

The \Cpp\ client uses rows and columns to access the individual elements of the
\class{Sequence}. The \Cpp\ Matlab client uses two methods to accomplish the
extraction, the first, \lit{print\_val\_by\_row()} determines the number of
rows in the \class{Sequence} and calls the \lit{print\_one\_row()} for each
of the rows in the \class{Sequence}. The \Cpp\ DAP implementation of the
\class{Sequence} data type provides the \lit{ var\_value(row,col)} method to
access the individual elements of the \class{Sequence}. The
\lit{var\_value()} method returns a \class{BaseType} pointer to the row,
column element of the \class{Sequence}. To access the binary data value
stored in that element, the \class{BaseType} method \lit{buf2val()} can be
used. The preceding example simply prints the contents of the element, most
client applications would assign the contents to a local variable in the
workspace.

%% \begin{vcode}{sib}
%%     /** 
%%      * Return the data held in the MatlabArray as an array of 
%%      * atomic types.
%%      * @return The data.
%%      */
%%     public Object getData() {
%%         PrimitiveVector pv = getPrimitiveVector();
%%         if( (pv instanceof BaseTypePrimitiveVector) == false) 
%%             return pv.getInternalStorage();
        
%%         else {
%%             BaseTypePrimitiveVector basePV = (BaseTypePrimitiveVector)pv;
%%             BaseType varTemplate = (BaseType)basePV.getValue(0);
%%             if(varTemplate instanceof MatlabString) {
%%                 char[][] arrayData = new char[basePV.getLength()][];
%%                 for(int i=0;i<pv.getLength();i++) {
%%                     arrayData[i] = ((MatlabString)basePV.getValue(i)).getValue().toCharArray();
%%                 }
%%                 return arrayData;
%%             }
%%             else if(varTemplate instanceof MatlabURL) {
%%                 char[][] arrayData = new char[basePV.getLength()][];
%%                 for(int i=0;i<pv.getLength();i++) {
%%                     arrayData[i] = ((MatlabURL)basePV.getValue(i)).getValue().toCharArray();
%%                 }
%%                 return arrayData;
%%             }
%%             else return null;
%%         }
%%     }
%% \end{vcode}


\begin{vcode}{sib}
void
ClientArray::print_val(ostream &os, string, bool print_decl_p)
{
    if (print_decl_p) {
        os << type_name() << endl << var()->type_name() << " " 
           << get_matlab_name() << " " << dimensions(true)
           << endl;

        // Write the actual dimension sizes on a separate line.
        for (Pix p = first_dim(); p; next_dim(p))
            os << dimension_size(p, true) << " ";

        os << endl;
    }

    for (int i = 0; i < length(); ++i)
        var(i)->print_val(os, "", false);
}    
\end{vcode}


\section{Notes}

Here's a collection of information that might be important to specific
clients but is hard to fit into a general tutorial.

\begin{itemize}
  
\item Because there are no meta data requirements to serve data via the
  \opendap\ protocol, client applications may not find all the
  information they require to make use of the data.  The DAP currently
  supports ancillary DAS files at the remote server site.  In
  development is an Ancillary Information Service (AIS) which will
  permit these external meta data resources to be located with the
  remote data itself, at other remote server sites, or on the client.
  Any meta data augmented by the AIS will be clearly indicated in the
  attributes.

\item It is the client application's responsibility to provide the
  initial base URLs to the remote server site.  In development are
  data discovery services, including an ImportWizard which can query
  existing directory services such as the GCMD, to provide the base
  URLs to providers with installed \opendap\ data servers.

\item The DODS Project has developed two tools to help with serving
  datasets that contain many files. The first is to set up a `file
  server' a kind of catalog of URLs that is itself a DODS data set.
  The second is called the Aggregation Server (AS). The AS can
  automatically aggregate discrete datasets, accessed as either as
  files (in some cases) or URLs to produce a single data set. See the
  \OPDhome\ and/or contact tech support (\OPDsupport) for help with
  this.

\item You can get help from the \OPDhome\, the \OPDtechList\
  and the DODS user support desk (\OPDsupport).
  
\end{itemize} 

\appendix

\section{How the Java DAP library differs}
\label{sec:javadap}

Enumerations instead of iterators

Factory classes instead of virtual constructors

DAP core with separate client and server specializations via Java interfaces

The constraint evaluator is an object passed into the DDS (Java) rather than
a set of methods embedded in the DDS (\Cpp)

\bibliographystyle{plain}
\bibliography{../boiler/dods}
   
\end{document}

% $Log: writing_client.tex,v $
% Revision 1.10  2004/04/24 21:37:21  jimg
% I added every directory in preparation for adding everyting. This is
% part of getting the opendap web pages going...
%
% Revision 1.9  2004/02/12 17:15:23  jimg
% Left the Java Geturl cde in place. I think this is better than the version of
% geturl.cc that's been HTMLized. I'm not really sure how to tackle this
% document...
%
% Revision 1.8  2004/02/12 16:53:43  jimg
% Still more changes.
%
% Revision 1.7  2004/02/12 06:44:17  jimg
% Almost done updating. I've commented out the Java examples. I'm not sure what
% to do with them since they are based on a client we no longer support.
%
% Revision 1.6  2004/02/05 07:16:06  jimg
% Changes up to page 9.
%
% Revision 1.5  2004/02/04 15:14:16  jimg
% Changes up to page 5.
%
% Revision 1.4  2004/01/21 16:41:24  jimg
% Change the date/revision thing for the web page version. Now it does not say
% 'Printed: ...' since that doesn't quite make sense. There is a CVS date at
% the bottom of the page.
%
% Revision 1.3  2004/01/21 16:38:11  jimg
% Brought up-to-date WRT the new latex macros. May still need editing to match
% what's going on in version 3.4.
%
% Revision 1.2  2004/01/16 18:06:00  jimg
% Removed the newcommand opendap since it is now defined in dods-def.tex.
%
% Revision 1.1  2002/07/16 03:30:59  tom
% moved writing_client from archive/workshops
%

% Old log
% Revision 1.7  2002/05/13 19:35:15  tom
% fixed some `_'s.  Changed htmldirectory
%
% Revision 1.6  2002/05/13 19:20:55  dan
% Added information to the Accessing the DataDDS section.
%
% Revision 1.5  2002/05/13 17:50:53  tom
% grammar, formatting
%
% Revision 1.4  2002/05/13 15:41:46  tom
% fixed a couple of `_'s
%
% Revision 1.3  2002/05/13 14:31:50  dan
% Added Accessing DataDDS section, needs better
% examples for buf2val calls, with examples for
% accessing Arrays, Grids, and Sequences.
%
% Revision 1.2  2002/05/13 02:53:55  tom
% grammar checking, misc editing
%
% Revision 1.1  2002/05/12 23:29:12  dan
% *** empty log message ***
%

% Started life as writing_server.tex...
%
% Revision 1.4  2002/05/11 22:40:27  jimg
% Finished. Spell cheecked. Not read over, though.
%
% Revision 1.3  2002/05/10 23:26:25  jimg
% Added text for the DAS section.
%
% Revision 1.2  2002/05/10 21:00:17  jimg
% I ran a spell checker on this.
%
% Revision 1.1  2002/05/10 17:17:05  tom
% converted from html, checked in.
%

%%% Local Variables: 
%%% mode: latex
%%% TeX-master: t
%%% End: 
