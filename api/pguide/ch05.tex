% Chapter to the DODS Programmer's Guide
%
% $Id$
%

\chapter{Linking Your Program}
\label{pguide,linking}
\label{tk,linking}

\indc{linking!application}\indc{application!linking}
To link a user program to the DODS client-library version of a data
access API library, you need only substitute the name of the
reimplemented API library for the original, and add the DODS Data
Access Protocol library (\lit{libdap++}) to the link list.

Because the DAP library and the API have circular dependencies they
must each be included on the linker command line twice. An example of
this can be seen in the DODS-NetCDF Makefile; the value of \var{LIBS}
is passed to the linker \lit{ld}. Note that the API library is listed
first.
\indc{object libraries}\indc{libraries!object}

\begin{vcode}{ib}
LIBS = -lnc-dods -ldap++ -lnc-dods -ldap++ 
\end{vcode}

You should have users link their programs using \lit{gcc} or \lit{g++}
since the libraries are all built using those tools. In particular,
\lit{g++} includes libstdc++ (and libg++) by default when it builds an
executable program from object modules. If you use \lit{gcc} instead
of \lit{g++} when you link, be sure to include these libraries as well
after all the libraries listed above. If you don't use \lit{gcc}, but
instead use the linker directly (i.e, you call \lit{ld} yourself) you
are on your own - you can use \lit{gcc -V} to determine what flags and
additional libraries it uses that are specific to your system and then
experiment with those. We cannot tell you how to proceed since each
UNIX variant requires different flags and libraries.

% $Log: ch05.tex,v $
% Revision 1.5  2004/04/24 21:37:22  jimg
% I added every directory in preparation for adding everyting. This is
% part of getting the opendap web pages going...
%
% Revision 1.4  1999/07/22 18:54:56  tom
% fixed errors
%
% Revision 1.3  1999/02/04 17:46:05  tom
% Modified for dods-book.cls and Hyperlatex
%
% Revision 1.2  1998/12/07 15:51:33  tom
% updated for DODSv2.19
%
% Revision 1.1  1998/03/13 20:50:03  tom
% created API manual from James's Toolkit document

%%% Local Variables: 
%%% mode: latex
%%% TeX-master: "../pguide.tex"
%%% End: 
