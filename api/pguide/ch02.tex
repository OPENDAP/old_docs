% Chapter to the DODS Programmer's Guide
%
% $Id$
%
% $Log: ch02.tex,v $
% Revision 1.5  2000/10/04 15:00:23  tom
% changed \figureplace definition, cleaning up...
%
% Revision 1.4  1999/07/22 18:54:56  tom
% fixed errors
%
% Revision 1.3  1999/02/04 17:46:05  tom
% Modified for dods-book.cls and Hyperlatex
%
% Revision 1.2  1998/12/07 15:51:33  tom
% updated for DODSv2.19
%
% Revision 1.1  1998/03/13 20:50:02  tom
% created API manual from James's Toolkit document
%
%

\chapter{Managing Connections}
\label{pguide,connect}
\label{tk,manage-conns}

Because DODS uses a communication protocol (http) that does not
maintain state information about the link between two processes, these
connections are virtual, and all the information about them is
maintainted by the client.  The toolkit contains two classes to help
clients manage connections to one or more DODS servers.  One class,
\class{Connect} stores information used when a connection is
established. It also allows a program to provide local access to data
(without a DODS data server).

The second network class of the DODS Toolkit, \class{Connections},
manages instances of the \class{Connect} class. It provides a
mechanism for the client libraries to pass back to user programs the
type of object (such as \lit{int}, opaque pointer, \ldots) they expect and
then to use one of those objects to access the correct instance of
\class{Connect}.  \class{Connections} is a template class. That is, a
client library uses the \class{Connections} class to make an array of
some sub-class of \class{Connect}.

\section{Connect}

\indc{Connect class}\indc{class!Connect}\indc{connection!managing}
\indc{client-server connection}
The \class{Connect} class manages one connection to either a remote
data set, via a DODS data server, or a local access. For each data set
or file that the user program opens, there must be exactly one
instance of \class{Connect}. Because information needed for local
access is stored in an instance of \class{Connect}, this class may
also be used (sub-classed) for each client API that needs to maintain
additional information about the connection.\footnote{For example, the 
  subclasses \class{JGConnect} and \class{NCConnect} exist for JGOFS and
  NetCDF, respectively.}. In the API-specific child of \class{Connect}
additional members can be added to store state information that the
client-library needs to maintain the virtual connection (See
Section~\ref{tk,subclass-netio} for more information on making
virtual connections).

\figureplace{The Connect Class}{htb}
{fig,connect-diagram}{Connect.ps}{Connect.gif}{}

The \class{Connect} class is illustrated in
\figureref{fig,connect-diagram}.  These objects contain the following
information:

\begin{description}

\item[URL] The URL with which the client library opens the connection
  with the DODS server.
  
\item[\class{DDS}] The \class{DDS} of the opened dataset.  This must
  be retrieved from the dataset.
  
\item[\class{DAS}] The \class{DAS} of the opened dataset.  This must
  be retrieved from the dataset.
  
\item[\class{Gui}] The \class{Gui} object indicates a graphical widget
  on the client platform that displays information about the data
  transfer in progress.
  
\item[\class{Error}] The \class{Error} object contains information
  used to help the client process an error message from the server.

%
%\item[Constraint List] In the case of repeated accesses to the same
%  URL, the \class{Connect} class can be used to contain information
%  about the multiple connections.  Each object in this list contains a
%  constraint expression string (as it is used in the submitted URL)
%  and a constrained \class{DDS} that was its result.

\end{description}

The \class{Connect} class provides member functions to get the
dataset's \class{DAS}, \class{DDS} and data. The instance of
\class{Connect} (or a subclass of \class{Connect}) stores the URL as
the user provides it.

When the client library receives a URL via its ``open'' call (which
will not actually be called \lit{open} but performs the function of
opening a file or data set; e.g., NetCDF's \lit{ncopen}) it passes
that URL to the \class{Connect} member functions like
\lit{request\_das} and \lit{request\_dds}. These member functions
append an appropriate extension (\lit{.das} and \lit{.dds}, for
example) onto the URL and retrieve the resulting information from the
server, the \class{DAS} or \class{DDS} for the dataset.

%The \class{Connect} class also contains information about all of the
%constraints used to access data so far. This information includes all
%of the expressions as well as all of the resulting \class{DDS}s and
%values they returned.  This is a powerful form of caching that can be
%tricky to use. If the user program, via the API, has previously
%requested the exact same data (i.e., the constraint expressions match
%exactly for any two requests) then you can use the data already stored
%by the \class{Connect} object rather than send for the data a second
%time. The \class{Connect} class provides member functions to be used
%to search the previously submitted constraint expressions.

If you are re-implementing an API and must support function calls that
modify how data is accessed (e.g., by creating array slices or by
choosing one of a set of variables), then you will need to translate
those requests into a DODS constraint expression. You would then pass
these synthesized constraint expressions to the
\lit{Connect::request\_data} member function.  (See
\sectionref{tk,constraints} for more information about constraint
expressions.)

The specialized version of \class{Connect} is the place to put state
information needed by a recoded API or other client.  This can be
used, for example, to emulate an API that maintains a list of open
files. 

\section{Connections}

\indc{Connections class}
The class \class{Connections} is used to manage a set of instances of
the class \class{Connect} by providing a means to map an index or
opaque pointer to an instance of \class{Connect}.

When a new instance of \class{Connect} (or a descendent of
\class{Connect}; See Section~\ref{tk,subclass-netio}) is created, it
is added to the \class{Connections} object using the \lit{add\_connect}
member function. \lit{add\_connect} returns an integer that can be
used to access that instance of Connect at any time. Similarly, when
an instance of \class{Connect} is to be deleted, the object can be
referred to via the \class{Connections} object and this index.




%%% Local Variables: 
%%% mode: latex
%%% TeX-master: "../pguide.tex"
%%% End: 
