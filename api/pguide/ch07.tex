% Chapter 6 of the DODS Programmer's Guide
%  A Server functional specification
%
% $Id$
%
% $Log: ch07.tex,v $
% Revision 1.2  2000/10/04 15:00:23  tom
% changed \figureplace definition, cleaning up...
%
% Revision 1.1  2000/03/21 22:04:33  tom
% added server spec appendixes, fixed xref tags
%
%

\chapter{Overview of the DODS Client}
\label{pguide,client-arch}

A DODS client is any web client that makes a service request to a DODS
server.  Since several of the DODS services return ASCII and HTML
data, any web browser, such as Netscape Navigator can be considered a
DODS client, so long as it is in the process of making a suitable
request to a DODS server.  The clients of interest in this appendix,
however, are clients that use the DODS DAP (Data Access Protocol)
library to make their requests for data.

Of these clients, there are two varieties: clients that have been
written expressly for DODS, and clients that existed in some form
already, and that have been adapted to use with a DODS client
library.  

\figureplace[The Original Program]{The Original Program, untouched by
  DODS.  The application's code accesses data by calls to the netCDF
  library functions, linked with the program.  Data access is direct,
  with the application program accessing local disk files to read
  data.\htmlonly{\html{hr}}}{htb}{fig,orig-client}
  {orig-client.ps}{orig-client.gif}{}

\figureplace[The Modified Program]{The Modified Program, using the
  DODS netCDF client library.  The application's code now accesses
  data by calls to the DODS netCDF library functions.  These are
  written to be functionally identical to the original netCDF
  functions, but instead of using a local disk to retrieve data, this
  library invokes functions from the DODS DAP library, which makes
  HTTP GET requests to a DODS server.  The client code is unchanged.
  \htmlonly{\html{hr}}}{htb}{fig,client-arch2}
  {client-arch2.ps}{client-arch2.gif}{}

\figureplace[Another Way]{Another Way.  An application program can
  also call the DODS DAP directly, eliminating the need for a client
  library.  When starting from scratch, this is probably easiest,
  unless you are an old hand at one of the supported data access APIs.
  \htmlonly{\html{hr}}}{htb}{fig,client-arch1}
  {client-arch1.ps}{client-arch1.gif}{}


%%% Local Variables: 
%%% mode: latex
%%% TeX-master: t
%%% End: 
