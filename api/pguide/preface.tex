% Preface to the DODS Programmer's Guide.
%
% $Id$
%
% $Log: preface.tex,v $
% Revision 1.5  2004/02/18 06:38:48  jimg
% Various changes, mostly for the DODS --> OPD macros.
%
% Revision 1.4  1999/07/22 18:54:56  tom
% fixed errors
%
% Revision 1.3  1999/02/04 17:46:05  tom
% Modified for dods-book.cls and Hyperlatex
%
% Revision 1.2  1998/12/07 15:51:33  tom
% updated for DODSv2.19
%
% Revision 1.1  1998/03/13 20:50:04  tom
% created API manual from James's Toolkit document
%
%
%

\T\chapter*{Preface}
\T\addcontentsline{toc}{chapter}{Preface}

This document describes how to use the \opendap\ toolkit software to build
\opendap\ data servers, clients and client-libraries. Using the objects and
functions contained in the toolkit, you can create programs which serve data
over the internet as well as programs that can request data from any
\opendap\ server.

This document covers release \DODSversion\ and later of the DODS
software.

\begin{ifhtml}
  \htmlmenu{4}
  \chapter*{Preface}
\end{ifhtml}

%%%%%%%%%%%%%%%%%%%%%%%%%%%%%%%%%%%%%%%%%%%%%%%%%%%%%%%%%%%%%%%%%%%%
\section{Who is this Guide for?}
\label{pref,who}

This guide is for people who wish to use the \opendap\ software to write a
new \opendap\ data server, a new client, or a new client library. Typically,
this will only be those people who wish to serve data in a format that is not
currently supported by the DODS team, or who have an existing application
that uses an idiosyncratic or unusual API for data access. Most people will
be able to use one of the already written servers or client libraries. See
the \OPDuser\ for a list of these.

This documentation assumes that the readers are \Cpp\ programmers, are
familiar with networked applications, and the POSIX programming environment.
The DODS/\opendap project also provides a native Java class library (API)
that parallels the \Cpp\ spftware described here.\footnote{While this manual
describes the \Cpp\ toolkit in detail, all of the concepts and much of the
structure can be directly translated to the Java toolkit.}

Also available are two tutorials, \OPDwclient\ and \OPDwserver\ , which
descrie how to write a client or a server, respectively.

Because the type of information presented in a document like this depends to
a large extent on the needs of its readers we welcome your feedback and
comments. In particular, if you have any questions about individual sections,
email those questions and we'll send back an answer as well as including that
information in the next version of this document. Send queries to:
\DODSsupport.


%%%%%%%%%%%%%%%%%%%%%%%%%%%%%%%%%%%%%%%%%%%%%%%%%%%%%%%%%%%%%%%%%%%%
\section{Organization of this Document}

This Guide is divided into five chapters. 

\begin{description}
  
\item[\chapterref{tk,overview}] provides background information on the
  organization of the toolkit software.

\item[\chapterref{tk,manage-conns}] describes how to use the
Network I/O classes to manage virtual connections.

\item[\chapterref{tk,subclassing}] discusses how to sub-class the toolkit
\Cpp\ classes so that they are specialized for your specific use.

\item[\chapterref{tk,using}] describes in detail how to write certain
sections of both the data server and the client-library for a new API.

\item[\chapterref{tk,linking}] describes how to link user programs
with the new client-library implementation of an API. 

\texonly{\item[\chapterref{tk,classref}] contains complete
  descriptions of all the DODS classes.}

\end{description}

\htmlonly{Note that this is {\em not\/} a reference volume. See
  \OPDapiref\ for a concise listing of the member functions in each
  of the toolkit's classes.}

%%%%%%%%%%%%%%%%%%%%%%%%%%%%%%%%%%%%%%%%%%%%%%%%%%%%%%%%%%%%%%%%%%%%

\listconventions

%%% Local Variables: 
%%% mode: latex
%%% TeX-master: "../pguide.tex"
%%% End: 
